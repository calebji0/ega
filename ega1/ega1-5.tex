\section{Reduced preschemes; the separation condition}
\label{section:I.5}

\subsection{Reduced preschemes}
\label{subsection:I.5.1}

\begin{proposition}[5.1.1]
\label{I.5.1.1}
Let $(X,\sh{O}_X)$ be a prescheme, and $\sh{B}$ a quasi-coherent $\sh{O}_X$-algebra.
Then there exists a unique quasi-coherent $\sh{O}_X$-module $\sh{N}$ whose stalk $\sh{N}_x$ at any $x\in X$ is the nilradical of the ring $\sh{B}_x$.
When $X$ is affine, and, consequently, $\sh{B}=\widetilde{B}$, where $B$ is an algebra over $A(X)$, then we have $\sh{N}=\widetilde{\nilrad}$, where $\nilrad$ is the nilradical of $B$.
\end{proposition}

\begin{proof}
\oldpage[I]{128}
The statement is local, so it suffices to show the latter claim.
We know that $\widetilde{\nilrad}$ is a quasi-coherent $\sh{O}_X$-module \sref{I.1.4.1}, and that its stalk at a point $x\in X$ is the ideal $\nilrad_x$ of the ring of fractions $B_x$;
it remains to prove that the nilradical of $B_x$ is contained in $\nilrad_x$, the converse inclusion being evident.
Let $z/s$ be an element of the nilradical of $B_x$, with $z\in B$, and $s\not\in\mathfrak{j}_x$;
by hypothesis, there exists an integer $k$ such that $(z/s)^k=0$, which implies that there exists some $t\not\in\mathfrak{j}_x$ such that $tz^k=0$.
We conclude that $(tz)^k=0$, and, as a result, that $z/s=(tz)/(ts)\in\nilrad_x$.
\end{proof}

We say that the quasi-coherent $\sh{O}_X$-module $\sh{N}$ thus defined is the \emph{nilradical} of the $\sh{O}_X$-algebra $\sh{B}$; in particular, we denote by $\sh{N}_X$ the nilradical of $\sh{O}_X$.

\begin{corollary}[5.1.2]
\label{I.5.1.2}
Let $X$ be a prescheme;
the closed subprescheme of $X$ defined by the sheaf of ideals $\sh{N}_X$ is the only reduced subprescheme \sref[0]{0.4.1.4} of $X$ that has $X$ as its underlying space;
it is also the smallest subprescheme of $X$ that has $X$ as its underlying space.
\end{corollary}

\begin{proof}
Since the structure sheaf of the closed subprescheme of $Y$ defined by $\sh{N}_X$ is $\sh{O}_X/\sh{N}_X$, it is immediate that $Y$ is reduced and has $X$ as its underlying space, because $\sh{N}_x\neq\sh{O}_x$ for any $x\in X$.
To show the other claims, note that a subprescheme $Z$ of $X$ that has $X$ as its underlying space is defined by a sheaf of ideals $\sh{I}$ \sref{I.4.1.3} such that $\sh{I}_x\neq\sh{O}_x$ for any $x\in X$.
We can restrict to the case where $X$ is affine, say $X=\Spec(A)$ and $\sh{I}=\widetilde{\mathfrak{I}}$, where $\mathfrak{I}$ is an ideal of $A$;
then, for every $x\in X$, we have $\mathfrak{I}_x\subset\mathfrak{j}_x$, and so $\mathfrak{I}$ is contained in every prime ideal of $A$, and so also in their intersection $\nilrad$, the nilradical of $A$.
This proves that $Y$ is the small subprescheme of $X$ that has $X$ as its underlying space \sref{I.4.1.9};
furthermore, if $Z$ is distinct from $Y$, we necessarily have $\sh{I}_x\neq\sh{N}_x$ for at least one $x\in X$, and so \sref{I.5.1.1} $Z$ is not reduced.
\end{proof}

\begin{definition}[5.1.3]
\label{I.5.1.3}
We define the reduced prescheme associated to a prescheme $X$, denoted by $X_\red$, to be the unique reduced subprescheme of $X$ that has $X$ as its underlying space.
\end{definition}

Saying that a prescheme $X$ is reduced thus implies that $X=X_\red$.

\begin{proposition}[5.1.4]
\label{I.5.1.4}
For the prime spectrum of a ring $A$ to be a reduced (resp. integral) prescheme \sref{I.2.1.7}, it is necessary and sufficient for $A$ to be a reduced (resp. integral) ring.
\end{proposition}

\begin{proof}
Indeed, it follows immediately from \sref{I.5.1.1} that the condition $\sh{N}=(0)$ is necessary and sufficient for $X=\Spec(A)$ to be reduced;
the claim corresponding to integral rings is then a consequence of \sref{I.1.1.13}.
\end{proof}

Since every ring of fractions $\neq\{0\}$ of an integral ring is integral, it follows from \sref{I.5.1.4} that, for every \emph{locally integral} prescheme $X$, $\sh{O}_x$ is an \emph{integral} ring for every $x\in X$.
The converse is true whenever the underlying space of $X$ is \emph{locally Noetherian}:
indeed, $X$ is then reduced, and if $U$ is an affine open subset of $X$, which is a Noetherian space, then $U$ has only a finite number of irreducible components, and so its ring $A$ has only a finite number of minimal prime ideals \sref{I.1.1.14}.
If two of the irreducible components of $U$ had a common point $x$, then $\sh{O}_x$ would have at least two distinct minimal prime ideals, and would thus not be integral;
the components of $U$ are thus open subsets that are pairwise disjoint, and each of them is thus integral.

\begin{env}[5.1.5]
\label{I.5.1.5}
Let $f=(\psi,\theta):X\to Y$ be a morphism of preschemes;
\oldpage[I]{129}
the homomorphism $\theta_x^\sharp:\sh{O}_{\psi(x)}\to\sh{O}_x$ sends each nilpotent element of $\sh{O}_{\psi(x)}$ to a nilpotent element of $\sh{O}_x$;
by passing to the quotients, $\theta^\sharp$ induces a homomorphism
\[
  \omega:\psi^*(\sh{O}_Y/\sh{N}_Y)\to\sh{O}_X/\sh{N}_X.
\]
It is clear that, for every $x\in X$, $\omega_x:\sh{O}_{\psi(x)}/\sh{N}_{\psi(x)}\to\sh{O}_x/\sh{N}_x$ is a local homomorphism, and so $(\psi,\omega^\flat)$ is a morphism of preschemes $X_\red\to Y_\red$, which we denote by $f_\red$, and call the \emph{reduced} morphism associated to $f$.
It is immediate that, for morphisms $f:X\to Y$ and $g:Y\to Z$, we have $(g\circ f)_\red=g_\red\circ f_\red$, and so we have defined $X_\red$ as a \emph{functor}, \emph{covariant} in $X$.

The preceding definition shows that the diagram
\[
  \xymatrix{
    X_\red\ar[r]^{f_\red}\ar[d] &
    Y_\red\ar[d]\\
    X\ar[r]^f &
    Y
  }
\]
is commutative, where the vertical arrows are the injection morphisms;
in other words, $X_\red\to X$ is a \emph{functorial} morphism.
We note in particular that, if $X$ is reduced, then every morphism $f:X\to Y$ factors as $X\xrightarrow{f_\red}Y_\red\to Y$;
in other words, $f$ factors through the injection morphism $Y_\red\to Y$.
\end{env}

\begin{proposition}[5.1.6]
\label{I.5.1.6}
Let $f:X\to Y$ be a morphism;
if $f$ is surjective (resp. radicial, an immersion, a closed immersion, an open immersion, a local immersion, a local isomorphism), then so too is $f_\red$.
Conversely, if $f_\red$ is surjective (resp. radicial), then so too is $f$.
\end{proposition}

\begin{proof}
The proposition is trivial if $f$ is surjective;
if $f$ is radicial, then the proposition follows from the fact that, for every $x\in X$, the field $\kres(x)$ is the same for the preschemes $X$ and $X_\red$ \sref{I.3.5.8}.
Finally, if $f=(\psi,\theta)$ is an immersion, a closed immersion, or a local immersion (resp. an open immersion, or a local isomorphism), then the proposition follows from the fact that, if $\theta_x^\sharp$ is surjective (resp. bijective), then so too is the homomorphism obtained by passing to the quotients by the nilradicals $\sh{O}_{\psi(x)}$ and $\sh{O}_x$ (\sref{I.5.1.2} and \sref{I.4.2.2}) (cf. \sref{I.5.5.12}).
\end{proof}

\begin{proposition}[5.1.7]
\label{I.5.1.7}
If $X$ and $Y$ are $S$-preschemes, then the preschemes $X_\red\times_{S_\red}Y_\red$ and $X_\red\times_S Y_\red$ are identical, and canonically identified with a subprescheme of $X\times_S Y$ that has the same underlying subspace as the two aforementioned products.
\end{proposition}

\begin{proof}
The canonical identification of $X_\red\times_S Y_\red$ with a subprescheme of $X\times_S Y$ that has the same underlying space follows from \sref{I.4.3.1}.
Furthermore, if $\vphi$ and $\psi$ are the structure morphisms $X_\red\to S$ and $Y_\red\to S$ (respectively), then they factor through $S_\red$ \sref{I.5.1.5}, and since $S_\red\to S$ is a monomorphism, the first claim of the proposition follows from \sref{I.3.2.4}.
\end{proof}

\begin{corollary}[5.1.8]
\label{I.5.1.8}
The preschemes $(X\times_S Y)_\red$ and $(X_\red\times_{S_\red}Y_\red)_\red$ are canonically identified with one another.
\end{corollary}

\begin{proof}
This follows from \sref{I.5.1.2} and \sref{I.5.1.7}.
\end{proof}

We note that, even if $X$ and $Y$ are reduced preschemes, $X\times_S Y$ might not be reduced, because the tensor product of two reduced algebras can have nilpotent elements.

\begin{proposition}[5.1.9]
\label{I.5.1.9}
\oldpage[I]{130}
Let $X$ be a prescheme, and $\sh{I}$ a quasi-coherent sheaf of ideals of $\sh{O}_X$ such that $\sh{I}^n=0$ for some integer $n>0$.
Let $X_0$ be the closed subprescheme $(X,\sh{O}_X/\sh{I})$ of $X$;
for $X$ to be an affine scheme, it is necessary and sufficient for $X_0$ to be an affine scheme.
\end{proposition}

The condition is clearly necessary, so we will show that it is sufficient.
If we set $X_k=(X,\sh{O}_X/\sh{I}^{k+1})$, it is enough to prove by induction on $k$ that $X_k$ is affine, and so we are led to consider the base case, where $\sh{I}^2=0$.
We set
\begin{align*}
  A&=\Gamma(X,\sh{O}_X)\\
  A_0&=\Gamma(X_0,\sh{O}_{X_0})=\Gamma(X,\sh{O}_X/\sh{I}).
\end{align*}
The canonical homomorphism $\sh{O}_X\to\sh{O}_X/\sh{I}$ induces a homomorphism of rings $\vphi:A\to A_0$.
We will see below that $\vphi$ is \emph{surjective}, which implies that
\[
\label{I.5.1.9.1}
  0\to\Gamma(X,\sh{I})\to\Gamma(X,\sh{O}_X)\to\Gamma(X,\sh{O}_X/\sh{I})\to 0\tag{5.1.9.1}
\]
is an \emph{exact} sequence.
We now prove, assuming that this is true, the proposition.
Note that $\mathfrak{K}=\Gamma(X,\sh{I})$ is an ideal whose square is zero in $A$, and thus a module over $A_0=A/\mathfrak{K}$.
By hypothesis, we have $X_0=\Spec(A)$, and, since the underlying spaces of $X_0$ and $X$ are identical, $\mathfrak{K}=\Gamma(X_0,\sh{I})$;
Additionally, since $\sh{I}^2=0$, $\sh{I}$ is a quasi-coherent $(\sh{O}_X/\sh{I})$-module, so we have $\sh{I}\cong\widetilde{\mathfrak{K}}$ and $\mathfrak{K}_x=\sh{I}_x$ for all $x\in X_0$ \sref{I.1.4.1}.
With this in mind, let $X'=\Spec(A)$, and consider the morphism $f=(\psi,\theta):X\to X'$ of preschemes that corresponds to the identity map $A\to\Gamma(X,\sh{O}_X)$ \sref{I.2.2.4}.
For every affine open subset $V$ of $X$, the diagram
\[
  \xymatrix{
    A\ar[r]\ar[d] &
    \Gamma(V,\sh{O}_X|V)\ar[d]\\
    A_0=A/\mathfrak{K}\ar[r] &
    \Gamma(V,\sh{O}_{X_0}|V)
  }
\]
commutes, whence the diagram
\[
  \xymatrix{
    X' &
    X\ar[l]_f\\
    X'_0\ar[u]^{j'} &
    X_0\ar[l]_{f_0}\ar[u]_j
  }
\]
also commutes ($X'_0$ being the closed subprescheme of $X'$ defined by the quasi-coherent sheaf of ideals $\widetilde{R}$, and $j$ and $j'$ being the canonical injection morphisms).
But since $X_0$ is affine, $f_0$ is an isomorphism, and since the underlying continuous maps of $j$ and $j'$ are identity maps, we see straight away that $\psi:X\to X'$ is a homeomorphism.
Furthermore, the equation $\mathfrak{K}_x=\sh{I}_x$ shows that the restriction of $\theta^\sharp:\psi^*(\sh{O}_{X'})\to\sh{O}_X$ is an \emph{isomorphism} from $\psi^*(\widetilde{\mathfrak{K}})$ to $\sh{I}$;
additionally, by passing to the quotients, $\theta^\sharp$ gives an \emph{isomorphism} $\psi^*(\sh{O}_X/\widetilde{\mathfrak{K}})\to\sh{O}_X/\sh{I}$, because $f_0$ is an isomorphism;
we thus immediately conclude, by the 5 lemma (M,~I,~1.1), that $\theta^\sharp$ is itself an isomorphism, and thus that $f$ is an \emph{isomorphism}, and thus that $X$ is affine.
So everything reduces to proving the exactitude of \sref{I.5.1.9.1}, which will follow from showing that $\HH^1(X,\sh{I})=0$.
\oldpage[I]{131}
But $\HH^1(X,\sh{I})=\HH^1(X_0,\sh{I})$, and we have seen that $\sh{I}$ is a quasi coherent $\sh{O}_{X_0}$-module.
Our proof will thus follow from
\begin{lemma}[5.1.9.2]
\label{I.5.1.9.2}
If $Y$ is an affine scheme, and $\sh{F}$ a quasi-coherent $\sh{O}_Y$-module, then $\HH^1(Y,\sh{F})=0$.
\end{lemma}

\begin{proof}
This lemma will be proven in Chapter~III, §1, as a consequence of the more general theorem that $\HH^i(Y,\sh{F})=0$ for all $i>0$.
To give an independent proof, note that $\HH^1(Y,\sh{F})$ can be identified with the module $\Ext_{\sh{O}_Y}^1(Y;\sh{O}_Y,\sh{F})$ of extensions classes of the $\sh{O}_Y$-module $\sh{O}_Y$ by the $\sh{O}_Y$-module $\sh{F}$ (T,~4.2.3);
so everything reduces to proving that such an extension $\sh{G}$ is trivial.
But, for all $y\in Y$, there is a neighbourhood $V$ of $y$ in $Y$ such that $\sh{G}|V$ is isomorphic to $\sh{F}|Y\oplus\sh{O}_Y|V$ \sref[0]{0.5.4.9};
from this we conclude that $\sh{G}$ is a \emph{quasi-coherent} $\sh{O}_Y$-module.
If $A$ is the ring of $Y$, then we have $\sh{F}=\widetilde{M}$ and $\sh{G}=\widetilde{N}$, where $M$ and $N$ are $A$-modules, and, by hypothesis, $N$ is an extension of the $A$-module $A$ by the $A$-module $M$ \sref{I.1.3.11}.
Since this extension is necessarily trivial, the lemma is proven, and thus so too is \sref{I.5.1.9}.
\end{proof}

\begin{corollary}[5.1.10]
\label{I.5.1.10}
Let $X$ be a prescheme such that $\sh{N}_X$ is nilpotent.
For $X$ to be an affine scheme, it is necessary and sufficient for $X_\red$ to be an affine scheme.
\end{corollary}

\subsection{Existence of a subprescheme with a given underlying space}
\label{subsection:I.5.2}

\begin{proposition}[5.2.1]
\label{I.5.2.1}
For every locally closed subspace $Y$ of the underlying space of a prescheme $X$, there exists exactly one reduced subprescheme of $X$ that has $Y$ as its underlying space.
\end{proposition}

\begin{proof}
The uniqueness follows from \sref{I.5.1.2}, so it remains only to show the existence of the prescheme in question.

If $X$ is affine, given by some ring $A$, and $Y$ closed in $X$, then the proposition is immediate:
$\mathfrak{j}(Y)$ is the largest ideal $\mathfrak{a}\subset A$ such that $V(\mathfrak{a})=Y$, and it is equal to its radical \sref{I.1.1.4}[i], so $A/\mathfrak{j}(Y)$ is a reduced ring.

In the general case, for every affine open $U\subset X$ such that $U\cap Y$ is closed in $U$, consider the closed subprescheme $Y_U$ of $U$ defined by the sheaf of ideals associated to the ideal $\mathfrak{j}(U\cap Y)$ of $A(U)$, which is reduced.
We can show that, if $V$ is an affine open subset of $X$ contained in $U$, then $Y_V$ is \emph{induced} by $Y_U$ on $V\cap Y$;
but this induced prescheme is a closed subprescheme (of $V$) which is reduced and has $V\cap Y$ as its underlying space;
the uniqueness of $Y_V$ thus implies our claim.
\end{proof}

\begin{proposition}[5.2.2]
\label{I.5.2.2}
Let $X$ be reduced, $f:X \to Y$ a morphism, $Z$ a sub-prescheme closed over $Y$ such that $f(X) \subset Z$.
Then, $f$ factors as $X \xrightarrow{g} Z \xrightarrow{j} Y$, where $j$ is an injective morphism.
\end{proposition}

\begin{proof}
It results from the hypothesis that the sub-prescheme $f^{-1}(Z)$ of $X$ is closed for all the underlying $X$ \sref{I.4.4.1};
As $X$ is reduced, this closed sub-presheme coincides with $X$ \sref{I.5.1.2}, and therefore the proposition followed \sref{I.4.4.1}.
\end{proof}

\begin{corollary}[5.2.3]
\label{I.5.2.3}
Let $X$ be a reduced subprescheme of a prescheme $Y$; if $Z$ is the closed reduced subprescheme of $Y$ that has $\overline{X}$ as its underlying space, then $X$ is a subprescheme induced on an open subset of $Z$.
\end{corollary}

\begin{proof}
\oldpage[I]{132}
There is indeed an open subset $U$ of $Y$ such that $X=U\cap\overline{X}$;
since, by \sref{I.5.2.2}, $X$ is a reduced subprescheme of $Z$, the subprescheme $X$ is induced by $Z$ on the open subspace $X$ by uniqueness \sref{I.5.2.1}.
\end{proof}

\begin{corollary}[5.2.4]
\label{I.5.2.4}
Let $f:X\to Y$ be a morphism, and $X'$ (resp. $Y'$) a closed subprescheme of $X$ (resp. $Y$) defined by a quasi-coherent sheaf of ideals $\sh{J}$ (resp. $\sh{K}$) of $\sh{O}_X$ (resp. $\sh{O}_Y$).
Suppose that $X'$ is reduced, and that $f(X')\subset Y'$.
Then $f^*(\sh{K})\sh{O}_X\subset\sh{J}$.
\end{corollary}

\begin{proof}
Since, by \sref{I.5.2.2}, the restriction of $f$ to $X'$ factors as $X'\to Y'\to Y$, it suffices to apply \sref{I.4.4.6}.
\end{proof}

\subsection{Diagonal; graph of a morphism}
\label{subsection:I.5.3}

\begin{env}[5.3.1]
\label{I.5.3.1}
Let $X$ be an $S$-prescheme;
we define the \emph{diagonal morphism} of $X$ in $X\times_S X$, denoted by $\Delta_{X|S}$, or $\Delta_X$, or even $\Delta$ if no confusion may arise, to be the $S$-morphism $(1_X,1_X)_S$, or, in other words, the unique $S$-morphism $\Delta_X$ such that
\[
\label{I.5.3.1.1}
  p_1\circ\Delta_X=p_2\circ\Delta_X=1_X,
  \tag{5.3.1.1}
\]
where $p_1$ and $p_2$ are the projections of $X\times_S X$ (Definition~\sref{I.3.2.1}).
If $f:T\to X$ and $g:T\to Y$ are $S$-morphisms, we immediately have that
\[
\label{I.5.3.1.2}
  (f,g)_S=(f\times_S g)\circ\Delta_{T|S}.
  \tag{5.3.1.2}
\]

The reader will note that the preceding definition and the results stated in \sref{I.5.3.1} to \sref{I.5.3.8} are true in any category, \emph{provided that the products used within exist in the category}.
\end{env}

\begin{proposition}[5.3.2]
\label{I.5.3.2}
Let $X$ and $Y$ be $S$-preschemes;
if we make the canonical identification between $(X\times Y)\times(X\times Y)$ and $(X\times X)\times(Y\times Y)$, then the morphism $\Delta_{X\times Y}$ is identified with $\Delta_X\times\Delta_Y$.
\end{proposition}

\begin{proof}
Indeed, if $p_1:X\times X\to X$ and $q_1:Y\times Y\to Y$ are the projections onto the first component, then the projection onto the first component $(X\times Y)\times(X\times Y)\to X\times Y$ is identified with $p_1\times q_1$, and we have
\[
  (p_1\times q_1)\circ(\Delta_X\times\Delta_Y)=(p_1\circ\Delta_X)\times(q_1\circ\Delta_Y)=1_{X\times Y}
\]
and we can argue similarly for the projections onto the second component.
\end{proof}

\begin{env}
\label{I.5.3.3}
\end{env}

\begin{corollary}[5.3.4]
\label{I.5.3.4}
For every extension $S'\to S$ of the base prescheme, $\Delta_{X_{S'}}$ is canonically identified with $(\Delta_X)_{(S')}$.
\end{corollary}

\begin{proof}
It suffices to remark that $(X\times_S X)_{(S')}$ is canonically identified with $X_{(S')}\times_{S'}X_{(S')}$ \sref{I.3.3.10}.
\end{proof}

\begin{proposition}[5.3.5]
\label{I.5.3.5}
Let $X$ and $Y$ be $S$-preschemes, and $\vphi:S\to T$ a morphism of preschemes, which lets us consider every $S$-prescheme as a $T$-prescheme.
Let $f:X\to S$ and $g:Y\to S$ be the structure morphisms, $p$ and $q$ the projections of $X\times_S Y$, and $\pi=f\circ p=g\circ q$ the structure morphism $X\times_S Y\to S$.
Then the diagram
\[
\label{I.5.3.5.1}
  \xymatrix{
    X\times_S Y\ar[r]^{(p,q)_T}\ar[d]_\pi &
    X\times_T Y\ar[d]^{f\times_T g}\\
    S\ar[r]^{\Delta_{S|T}} &
    S\times_T S
  }
  \tag{5.3.5.1}
\]
\oldpage[I]{133}
commutes, and identifies $X\times_S Y$ with the product of the $(S\times_T S)$-preschemes $S$ and $X\times_T Y$, and the projections with $\pi$ and $(p,q)_T$.
\end{proposition}

\begin{proof}
By \sref{I.3.4.3}, we are led to proving the corresponding proposition \emph{in the category of sets}, replacing $X$, $Y$, and $S$ by $X(Z)_T$, $Y(Z)_T$, and $S(Z)_T$ (respectively), with $Z$ being an arbitrary $T$-prescheme.
But, in the category of sets, the proof is immediate and left to the reader.
\end{proof}

\begin{corollary}[5.3.6]
\label{I.5.3.6}
The morphism $(p,q)_T$ can be identified (letting $P=S\times_T S$) with $1_{X\times_T Y}\times_P\Delta_S$.
\end{corollary}

\begin{proof}
This follows from \sref{I.5.3.5} and \sref{I.3.3.4}.
\end{proof}

\begin{corollary}[5.3.7]
\label{I.5.3.7}
If $f:X\to Y$ is an $S$-morphism, then the diagram
\[
  \xymatrix{
    X\ar[r]^{(1_X,f)_S}\ar[d]_f &
    X\times_S Y\ar[d]^{f\times_S 1_Y}\\
    Y\ar[r]^{\Delta_Y} &
    Y\times_S Y
  }
\]
commutes, and identifies $X$ with the product of the $(Y\times_S Y)$-preschemes $Y$ and $X\times_S Y$.
\end{corollary}

\begin{proof}
It suffices to apply \sref{I.5.3.5}, replacing $S$ by $Y$, and $T$ by $S$, and noting that $X\times_Y Y=X$ \sref{I.3.3.3}.
\end{proof}

\begin{proposition}[5.3.8]
\label{I.5.3.8}
For $f:X\to Y$ to be a monomorphism of preschemes, it is necessary and sufficient for $\Delta_{X|Y}$ to be an isomorphism from $X$ to $X\times_Y X$.
\end{proposition}

\begin{proof}
Indeed, to say that $f$ is a monomorphism implies that, for every $Y$-prescheme $Z$, the corresponding map $f':X(Z)_Y\to Y(Z)_Y$ is an injection, and, since $Y(Z)_Y$ consists of a single element, this implies that $X(Z)_Y$ consists of a single element as well.
But this can also be expressed by saying that $X(Z)_Y\times X(Z)_Y$ is canonically isomorphic to $X(Z)_Y$; the former is exactly the set $(X\times_Y X)(Z)_Y$ \hyperref[I.3.4.3]{(3.4.3.1)}, which implies that $\Delta_{X|Y}$ is an isomorphism.
\end{proof}

\begin{proposition}[5.3.9]
\label{I.5.3.9}
The diagonal morphism $\Delta_X$ is an immersion from $X$ to $X\times_S X$.
\end{proposition}

\begin{proof}
Indeed, since the continuous maps $p_1$ and $\Delta_X$ from the underlying spaces are such that $p_1\circ\Delta_X$ is the identity, $\Delta_X$ is a homeomorphism from $X$ to $\Delta_X(X)$.
Similarly, the composite homomorphism $\sh{O}_x\to\sh{O}_{\Delta_X(x)}\to\sh{O}_x$ (composed of the homomorphisms corresponding to $p_1$ and $\Delta_X$) is the identity, which means that the homomorphism corresponding to $\Delta_X$ is surjective;
the proposition thus follows from \sref{I.4.2.2}.
\end{proof}

We say that the subprescheme of $X\times_S X$ associated to the immersion $\Delta_X$ \sref{I.4.2.1} is \emph{the diagonal} of $X\times_S X$.

\begin{corollary}[5.3.10]
\label{I.5.3.10}
Under the hypotheses of \sref{I.5.3.5}, $(p,q)_T$ is an immersion.
\end{corollary}

\begin{proof}
This follows from \sref{I.5.3.6} and \sref{I.4.3.1}.
\end{proof}

We say (under the hypotheses of \sref{I.5.3.5}) that $(p,q)_T$ is the \emph{canonical immersion} of $X\times_S Y$ into $X\times_T Y$.

\begin{corollary}[5.3.11]
\label{I.5.3.11}
Let $X$ and $Y$ be $S$-preschemes, and $f:X\to Y$ an $S$-morphism;
then the graph morphism $\Gamma_f=(1_X,f)_S$ of $f$ \sref{I.3.3.14} is an immersion of $X$ into $X\times_S Y$.
\end{corollary}

\begin{proof}
This is a particular case of Corollary \sref{I.5.3.10}, where we replace $S$ by $Y$, and $T$ by $S$ (cf. \sref{I.5.3.7}).
\end{proof}

\oldpage[I]{134}
The subprescheme of $X\times_S Y$ associated to the immersion $\Gamma_f$ \sref{I.4.2.1} is called \emph{the graph} of the morphism $f$;
the subpreschemes of $X\times_S Y$ that are graphs of morphisms $X\to Y$ are characterised by the property that the restriction to such a subprescheme $G$ of the projection $p_1:X\times_S Y\to X$ is an \emph{isomorphism} $g$ from $G$ to $X$:
$G$ is the the graph of the morphism $p_2\circ g^{-1}$, where $p_2$ is the projection $X\times_S Y\to Y$.

When we take, in particular, $X=S$, then the $S$-morphisms $S\to Y$ (which are exactly the \emph{$S$-sections} of $Y$ \sref{I.2.5.5}) are equal to their graph morphisms;
the subpreschemes of $Y$ that are the graphs of $S$-sections (in other words, those that are isomorphic to $S$ by the restriction of the structure morphism $Y\to S$) are then also called the \emph{images} of these sections, or, by an abuse of language, the \emph{$S$-sections} of $Y$.

\begin{corollary}[5.3.12]
\label{I.5.3.12}
With the hypotheses and notation of \sref{I.5.3.11}, for every morphism $g:S'\to S$, let $f'$ be the inverse image of $f$ under $g$ \sref{I.3.3.7};
then $\Gamma_{f'}$ is the inverse image of $\Gamma_f$ under $g$.
\end{corollary}

\begin{proof}
This is a particular case of \sref{I.3.3.10.1}.
\end{proof}

\begin{corollary}[5.3.13]
\label{I.5.3.13}
Let $f:X\to Y$ and $g:Y\to Z$ be morphisms;
if $g\circ f$ is an immersion (resp. a local immersion), then so too is $f$.
\end{corollary}

\begin{proof}
Indeed, $f$ factors as $X\xrightarrow{\Gamma_f}X\times_Z Y\xrightarrow{p_2}Y$.
Furthermore, $p_2$ can be identified with $(g\circ f)\times_Z 1_Y$ \sref{I.3.3.4};
if $g\circ f$ is an immersion (resp. a local immersion), then so too is $p_2$ (\sref{I.4.3.1} and \sref{I.4.5.5}), and since $\Gamma_f$ is an immersion \sref{I.5.3.11}, we are done, by \sref{I.4.2.4} (resp. \sref{I.4.5.5}).
\end{proof}

\begin{corollary}[5.3.14]
\label{I.5.3.14}
Let $j:X\to Y$ and $g:Z\to Z$ be $S$-morphisms.
If $j$ is an immersion (resp. a local immersion), then so too is $(j,g)_S$.
\end{corollary}

\begin{proof}
Indeed, if $p:Y\times_S Z\to Y$ is the projection onto the first component, then we have $j=p\circ(j,g)_S$, and it suffices to apply \sref{I.5.3.13}.
\end{proof}

\begin{proposition}[5.3.15]
\label{I.5.3.15}
If $f:X\to Y$ is an $S$-morphism, then the diagram
\[
\label{I.5.3.15.1}
  \xymatrix{
    X\ar[r]^{\Delta_X}\ar[d]_f &
    X\times_S X\ar[d]^{f\times_S f}\\
    Y\ar[r]^{\Delta_Y} &
    Y\times_S Y
  }\tag{5.3.15.1}
\]
commutes \emph{(in other words, $\Delta_X$ is a functorial morphism in the category of preschemes)}.
\end{proposition}

\begin{proof}
The proof is immediate and left to the reader.
\end{proof}

\begin{corollary}[5.3.16]
\label{I.5.3.16}
If $X$ is a subprescheme of $Y$, then the diagonal $\Delta_X(X)$ can be identified with a subprescheme of $\Delta_Y(Y)$, and the underlying space can be identified with
\[
  \Delta_Y(Y)\cap p_1^{-1}(X)=\Delta_Y\cap p_2^{-1}(X)
\]
($p_1$ and $p_2$ being the projections of $Y\times_S Y$).
\end{corollary}

\begin{proof}
Applying \sref{I.5.3.15} to the injection morphism $f:X\to Y$, we see that $f\times_S f$ is an immersion that identifies the underlying space of $X\times_S X$ with the subspace $p_1^{-1}(X)\cap p_2^{-1}(X)$ of $Y\times_S Y$ \sref{I.4.3.1};
further, if $z\in\Delta_Y(Y)\cap p_1^{-1}(X)$, then we have $z=\Delta_Y(y)$
\oldpage[I]{135}
and $y=p_1(z)\in X$, so $y=f(y)$, and $z=\Delta_Y(f(y))$ belongs to $\Delta_X(X)$ by the commutativity of \sref{I.5.3.15.1}.
\end{proof}

\begin{corollary}[5.3.17]
\label{I.5.3.17}
Let $f_1:Y\to X$ and $f_2:Y\to X$ be $S$-morphisms, and $y$ a point of $Y$ such that $f_1(y)=f_2(y)=x$, and such that the homomorphisms $\kres(x)\to\kres(y)$ corresponding to $f_1 $ and $f_2$ are identical.
Then, if $f=(f_1,f_2)_S$, the point $f(y)$ belongs to the diagonal $\Delta_{X|S}(X)$.
\end{corollary}

\begin{proof}
The two homomorphisms $\kres(x)\to\kres(y)$ corresponding to $f_i$ ($i=1,2$) define two $S$-morphisms $g_i:\Spec(\kres(y))\to\Spec(\kres(x))$ such that the diagrams
\[
  \xymatrix{
    \Spec(\kres(y))\ar[r]^{g_i}\ar[d] &
    \Spec(\kres(x))\ar[d]\\
    Y\ar[r]^{f_i} &
    X
  }
\]
commute.
The diagram
\[
  \xymatrix{
    \Spec(\kres(y))\ar[rr]^{(g_1,g_2)_S}\ar[d] & &
    \Spec(\kres(x))\times_S\Spec(\kres(x))\ar[d]\\
    Y\ar[rr]^{(f_1,f_2)_S} & &
    X\times_S X
  }
\]
thus also commutes.
But it follows from the equality $g_1=g_2$ that the image under $(g_1,g_2)_S$ of the unique point of $\Spec(\kres(y))$ belongs to the diagonal of $\Spec(\kres(x))\times_S\Spec(\kres(x))$;
the conclusion then follows from \sref{I.5.3.15}.
\end{proof}

\subsection{Separated morphisms and separated preschemes}
\label{subsection:I.5.4}

\begin{definition}[5.4.1]
\label{I.5.4.1}
We say that a morphism of preschemes $f:X\to Y$ is \emph{separated} if the diagonal morphism $X\to X\times_Y X$ is a \emph{closed} immersion; we then also say that $X$ is a \emph{separated prescheme over $Y$}, or a \emph{$Y$-scheme}.
We say that a prescheme $X$ is separated if it is separated over $\Spec(\bb{Z})$;
we then also say that $X$ is a \emph{scheme}\footnote{\emph{[Trans.] We repeat here the warning given at the very start of this translation: the early versions of the EGA use \emph{prescheme} to mean is now usually called a scheme, and \emph{scheme} for what is now usually called a separated scheme.}. Grothendieck himself later said that the more modern terminology was preferable, but we have decided to keep this translation `historically accurate' by using the older nomenclature.} \emph{(cf. \sref{I.5.5.7})}.
\end{definition}

By \sref{I.5.3.9}, for $X$ to be separated over $Y$, it is necessary and sufficient for $\Delta_X(X)$ to be a \emph{closed subspace} of the underlying space of $X\times_Y X$.

\begin{proposition}[5.4.2]
\label{I.5.4.2}
Let $S\to T$ be a separated morphism.
If $X$ and $Y$ are $S$-preschemes, then the canonical immersion $X\times_S Y\to X\times_T Y$ \sref{I.5.3.10} is closed.
\end{proposition}

\begin{proof}
Indeed, if we refer to the diagram in \sref{I.5.3.5.1}, we see that $(p,q)_T$ can be considered as being obtained from $\Delta_{S|T}$ by the extension $f\times_T g:X\times_T Y\to S\times_T S$ of the base prescheme $S\times_T S$;
the proposition then follows from \sref{I.4.3.2}.
\end{proof}

\begin{corollary}[5.4.3]
\label{I.5.4.3}
Let $Y$ be an $S$-scheme, and $f:X\to Y$ an $S$-morphism.
Then the graph morphism $\Gamma_f:X\to X\times_S Y$ \sref{I.5.3.11} is a closed immersion.
\end{corollary}

\begin{proof}
This is a particular case of \sref{I.5.4.2}, where we replace $S$ by $Y$, and $T$ by $S$.
\end{proof}

\begin{corollary}[5.4.4]
\label{I.5.4.4}
Let $f:X\to Y$ and $g:Y\to Z$ be morphisms, with $g$ separated.
If $g\circ f$ is a closed immersion, then so too is $f$.
\end{corollary}

\begin{proof}
The proof using \sref{I.5.4.3} is the same as that of \sref{I.5.3.13} using \sref{I.5.3.11}.
\end{proof}

\oldpage[I]{136}
\begin{corollary}[5.4.5]
\label{I.5.4.5}
Let $Z$ be an $S$-scheme, and $j:X\to Y$ and $g:X\to Z$ $S$-morphisms.
If $j$ is a closed immersion, then so too is $(j,g)_S:X\to Y\times_S Z$.
\end{corollary}

\begin{proof}
The proof using \sref{I.5.4.4} is the same as that of \sref{I.5.3.14} using \sref{I.5.3.13}.
\end{proof}

\begin{corollary}[5.4.6]
\label{I.5.4.6}
If $X$ is an $S$-scheme, then every $S$-section of $X$ \sref{I.2.5.5} is a closed immersion.
\end{corollary}

\begin{proof}
If $\vphi:X\to S$ is the structure morphism, and $\psi:S\to X$ an $S$-section of $X$, it suffices to apply \sref{I.5.4.5} to $\vphi\circ\psi=1_S$.
\end{proof}

\begin{corollary}[5.4.7]
\label{I.5.4.7}
Let $X$ be an integral prescheme with generic point $s$, and $X$ an $S$-scheme.
If two $S$-sections $f$ and $g$ are such that $f(s)=g(s)$, then $f=g$.
\end{corollary}

\begin{proof}
Indeed, if $x=f(s)=g(s)$, then the homomorphisms $\kres(x)\to\kres(s)$ corresponding to $f$ and $g$ are necessarily identical.
If $h=(f,g)_S$, we thus deduce \sref{I.5.3.17} that $h(s)$ belongs to the diagonal $Z=\Delta_X(X)$;
but since $S=\overline{\{s\}}$, and since $Z$ is closed by hypothesis, we have $h(S)\subset Z$.
It then follows from \sref{I.5.2.2} that $h$ factors as $S\to Z\to X\times_S X$, and we thus conclude that $f=g$, by definition of the diagonal.
\end{proof}

\begin{remark}[5.4.8]
\label{I.5.4.8}
If we suppose, conversely, that the conclusion of \sref{I.5.4.3} is true when $f=1_Y$, then we can conclude that $Y$ is separated over $S$;
similarly, if we suppose that the conclusion of \sref{I.5.4.5} applies to the two morphisms $Y\xrightarrow{\Delta_Y}Y\times_Z Y\xrightarrow{p_1}Y$, then we can conclude that $\Delta_Y$ is a closed immersion, and thus that $Y$ is separated over $Z$;
finally, if we assume that the conclusion of \sref{I.5.4.6} is true for the $Y$ section $\Delta_Y$ of the $Y$-prescheme $Y\times_S Y\to Y$, then this implies that $Y$ is separated over $S$.
\end{remark}

\subsection{Separation criteria}
\label{subsection:I.5.5}

\begin{proposition}[5.5.1]
\label{I.5.5.1}
\medskip\noindent
\begin{enumerate}
  \item[{\rm(i)}] Every monomorphism of preschemes (and, in particular, every immersion) is a separated morphism.
  \item[{\rm(ii)}] The composition of any two separated morphisms is separated.
  \item[{\rm(iii)}] If $f:X\to X'$ and $g:Y\to Y'$ are separated $S$-morphisms, then $f\times_S g$ is separated.
  \item[{\rm(iv)}] If $f:X\to Y$ is a separated $S$-morphism, then the $S'$-morphism $f_{(S')}$ is separated for every extension $S'\to S$ of the base prescheme.
  \item[{\rm(v)}] If the composition $g\circ f$ is separated, then $f$ is separated.
  \item[{\rm(vi)}] For a morphism $f$ to be separated, it is necessary and sufficient for $f_\red$ \sref{I.5.1.5} to be separated.
\end{enumerate}
\end{proposition}

\begin{proof}
Note that~(i) is an immediate consequence of \sref{I.5.3.8}.
If $f:X\to Y$ and $g:Y\to Z$ are morphisms, then the diagram
\[
\label{I.5.5.1.1}
  \xymatrix{
    X\ar[rr]^{\Delta_{X|Z}}\ar[dr]_{\Delta_{X|Y}} & &
    X\times_Z X\\
    & X\times_Y X\ar[ur]_j
  }
  \tag{5.5.1.1}
\]
commutes (where $j$ denotes the canonical immersion \sref{I.5.3.10}), as can be immediately verified.
If $f$ and $g$ are separated, then $\Delta_{X|Y}$ is a closed immersion by definition, and $j$ is a closed immersion by \sref{I.5.4.2}, whence $\Delta_{X|Z}$ is a closed immersion by \sref{I.4.2.4}, which
\oldpage[I]{137}
proves~(ii).
Given~(i) and~(ii), (iii) and~(iv) are equivalent \sref{I.3.5.1}, so it suffices to prove (iv).
But $X_{(S')}\times_{Y_{(S')}}X_{(S')}$ is canonically identified with $(X\times_Y X)\times_Y Y_{(S')}$ by \sref{I.3.3.11} and \hyperref[I.3.3.9]{(3.3.9.1)}, and we immediately see that the diagonal morphism $\Delta_{X_{(S')}}$ can then be identified with $\Delta_X\times_Y 1_{Y_{(S')}}$;
the proposition then follows from \sref{I.4.3.1}.

To prove (v), consider, as in \sref{I.5.3.13}, the factorisation $X\xrightarrow{\Gamma_f}X\times_Z Y\xrightarrow{p_2}Y$ of $f$, noting that $p_2=(g\circ f)\times_Z 1_Y$;
the hypothesis (that $g\circ f$ is separated) implies that $g_2$ is separated, by (iii) and (i), and, since $\Gamma_f$ is an immersion, $\Gamma_f$ is separated, by (i), whence $f$ is separated, by (ii).
Finally, to prove (vi), recall that the preschemes $X_\red\times_{Y_\red}X_\red$ and $X_\red\times_Y X_\red$ are canonically identified with one another \sref{I.5.1.7};
if we denote by $j$ the injection $X_\red\to X$, then the diagram
\[
  \xymatrix{
    X_\red\ar[r]^{\Delta_{X_\red}}\ar[d]_j &
    X_\red\times_Y X_\red\ar[d]^{j\times_Y j}\\
    X\ar[r]^{\Delta_X} &
    X\times_Y X
  }
\]
commutes \sref{I.5.3.15}, and the proposition follows from the fact that the vertical arrows are homeomorphisms of the underlying spaces \sref{I.4.3.1}.
\end{proof}

\begin{corollary}[5.5.2]
\label{I.5.5.2}
If $f:X\to Y$ is separated, then the restriction of $f$ to any subprescheme of $X$ is separated.
\end{corollary}

\begin{proof}
This follows from \sref{I.5.5.1}[i and ii].
\end{proof}

\begin{corollary}[5.5.3]
\label{I.5.5.3}
If $X$ and $Y$ are $S$-preschemes such that $Y$ is separated over $S$, then $X\times_S Y$ is separated over $X$.
\end{corollary}

\begin{proof}
This is a particular case of \sref{I.5.5.1}[iv].
\end{proof}

\begin{proposition}[5.5.4]
\label{I.5.5.4}
Let $X$ be a prescheme, and assume that its underlying space is a \emph{finite} union of closed subsets $X_k$ ($1\leq k\leq n$);
for each $k$, consider the reduced subprescheme of $X$ that has $X_k$ as its underlying space \sref{I.5.2.1}, and denote this again by $X_k$.
Let $f:X\to Y$ be a morphism, and, for each $k$, let $Y_k$ be a closed subset of $Y$ such that $f(X_k)\subset Y_k$;
we again denote by $Y_k$ the reduced subprescheme of $Y$ that has $Y_k$ as its underlying space, so that the restriction $X_k\to Y$ of $f$ to $X_k$ factors as $X_k\xrightarrow{f_k}Y_k\to Y$ \sref{I.5.2.2}.
For $f$ to be separated, it is necessary and sufficient for all the $f_k$ to be separated.
\end{proposition}

\begin{proof}
The necessity follows from \sref{I.5.5.1}[i, ii, and v].
Conversely, if the condition of the statement is satisfied, then each of the restrictions $X_k\to Y$ of $f$ is separated \hyperref[I.5.5.1]{(5.5.1, (i) and (ii))};
if $p_1$ and $p_2$ are the projections of $X\times_Y X$, then the subspace $\Delta_{X_k}(X_k)$ can be identified with the subspace $\Delta_X(X)\cap p_1^{-1}(X_k)$ of the underlying space of $X\times_Y X$ \sref{I.5.3.16};
these subspaces are closed in $X\times_Y X$, and thus so too is their union $\Delta_X(X)$.
\end{proof}

Suppose, in particular, that the $X_k$ are the \emph{irreducible components} of $X$;
then we can suppose that the $Y_k$ are the irreducible components of $Y$ \sref[0]{0.2.1.5};
Proposition~\sref{I.5.5.4} then, in this case, leads to the idea of separation in the case of \emph{integral} preschemes \sref{I.2.1.7}.

\oldpage[I]{138}
\begin{proposition}[5.5.5]
\label{I.5.5.5}
Let $(Y_\lambda)$ be an open cover of a prescheme $Y$;
for a morphism $f:X\to Y$ to be separated, it is necessary and sufficient for each of its restrictions $f^{-1}(Y_\lambda)\to Y_\lambda$ to be separated.
\end{proposition}

\begin{proof}
If we set $X_\lambda=f^{-1}(Y_\lambda)$, everything reduces, by taking \sref{I.4.2.4}[b] and the identification of the products $X_\lambda\times_Y X_\lambda$ and $X_\lambda\times_{Y_\lambda}X_\lambda$ into account, to proving that the $X_\lambda\times_Y X_\lambda$ form a cover of $X\times_Y X$.
But if we set $Y_{\lambda\mu}=Y_\lambda\cap Y_\mu$ and $X_{\lambda\mu}=X_\lambda\cap X_\mu=f^{-1}(Y_{\lambda\mu})$, then $X_\lambda\times_Y X_\mu$ can be identified with the product $X_{\lambda\mu}\times_{Y_{\lambda\mu}}X_{\lambda\mu}$ \sref{I.3.2.6.4}, and so also with $X_{\lambda\mu}\times_Y X_{\lambda\mu}$ \sref{I.3.2.5}, and finally with an open subset of $X_\lambda\times_Y X_\lambda$, which proves our claim \sref{I.3.2.7}.
\end{proof}

Proposition~\sref{I.5.5.4} allows us, by taking an affine open cover of $Y$, to restrict our study of separated morphisms to just those that take values in affine schemes.

\begin{proposition}[5.5.6]
\label{I.5.5.6}
Let $Y$ be an \emph{affine} scheme, $X$ a prescheme, and $(U_\alpha)$ a cover of $X$ by affine open subsets.
For a morphism $f:X\to Y$ to be separated, it is necessary and sufficient for $U_\alpha\cap U_\beta$ to be, for every pair of indices $(\alpha,\beta)$, an affine open subset, \emph{and} for the ring $\Gamma(U_\alpha\cap U_\beta,\sh{O}_X)$ to be generated by the union of the canonical images of the rings $\Gamma(U_\alpha,\sh{O}_X)$ and $\Gamma(U_\beta,\sh{O}_X)$.
\end{proposition}

\begin{proof}
The $U_\alpha\times_Y U_\beta$ form an open cover of $X\times_Y X$ \sref{I.3.2.7};
denoting the projections of $X\times_Y X$ by $p$ and $q$, we have
\begin{align*}
  \Delta_X^{-1}(U_\alpha\times_Y U_\beta)&=\Delta_X^{-1}(p^{-1}(U_\alpha)\cap q^{-1}(U_\beta))\\
                                         &=\Delta_X^{-1}(p^{-1}(U_\alpha))\cap\Delta_X^{-1}(q^{-1}(U_\beta))=U_\alpha\cap U_\beta;
\end{align*}
so everything reduces to proving that the restriction of $\Delta_X$ to $U_\alpha\cap U_\beta$ is a closed immersion into $U_\alpha\times_Y U_\beta$.
But this restriction is exactly $(j_\alpha,j_\beta)_Y$, where $j_\alpha$ (resp. $j_\beta$) denotes the injection morphism from $U_\alpha\cap U_\beta$ to $U_\alpha$ (resp. $U_\beta$), as follows from the definitions.
Since $U_\alpha\times_Y U_\beta$ is an affine scheme whose ring is canonically isomorphic to $\Gamma(U_\alpha,\sh{O}_X)\otimes_{\Gamma(Y,\sh{O}_Y)}\Gamma(U_\beta,\sh{O}_X)$ \sref{I.3.2.2}, we see that $U_\alpha\cap U_\beta$ must be an affine scheme, and that the map $h_\alpha\otimes h_\beta\mapsto h_\alpha h_\beta$ from the ring $A(U_\alpha\times_Y U_\beta)$ to $\Gamma(U_\alpha\cap U_\beta,\sh{O}_X)$ must be surjective \sref{I.4.2.3}, which finishes the proof.
\end{proof}

\begin{corollary}[5.5.7]
\label{I.5.5.7}
An affine scheme is separated \emph{(and is thus a \emph{scheme}, which justifies the terminology of \sref{I.5.4.1})}.
\end{corollary}

\begin{corollary}[5.5.8]
\label{I.5.5.8}
Let $Y$ be an affine scheme;
for $f:X\to Y$ to be a separated morphism, it is necessary and sufficient for $X$ to be separated \emph{(in other words, for $X$ to be a \emph{scheme})}.
\end{corollary}

\begin{proof}
Indeed, we see that the criteria of \sref{I.5.5.6} do not depend on $f$.
\end{proof}

\begin{corollary}[5.5.9]
\label{I.5.5.9}
For a morphism $f:X\to Y$ to be separated, it is necessary and sufficient for the induced prescheme $f^{-1}(U)$ to be separated, for every open subset of $U$ on which $Y$ induces a separated prescheme, and it is sufficient for it to be the case for every affine open subset $U\subset Y$.
\end{corollary}

\begin{proof}
The necessity of the condition follows from \sref{I.5.5.4} and \sref{I.5.5.1}[ii];
the sufficiency follows from \sref{I.5.5.4} and \sref{I.5.5.8}, taking into account the existence of affine open covers of $Y$.
\end{proof}

In particular, if $X$ and $Y$ are affine schemes, then \emph{every} morphism $X\to Y$ is separated.

\begin{proposition}[5.5.10]
\label{I.5.5.10}
Let $Y$ be a \emph{scheme}, and $f:X\to Y$ a morphism.
For every affine open subset $U$ of $X$, and every affine open subset $V$ of $Y$, $U\cap f^{-1}(V)$ is affine.
\end{proposition}

\begin{proof}
Let $p_1$ and $p_2$ be the projections of $X\times_\bb{Z}Y$;
the subspace $U\cap f^{-1}(V)$ is the image of $\Gamma_f(X)\cap p_1^{-1}(U)\cap p_2^{-1}(V)$ under $p_1$.
But $p_1^{-1}(U)\cap p_2^{-1}(V)$ can be identified with the underlying space of the
\oldpage[I]{139}
prescheme $U\times_\bb{Z}V$ \sref{I.3.2.7}, and is thus an affine scheme \sref{I.3.2.2};
since $\Gamma_f(X)$ is closed in $X\times_\bb{Z}Y$ \sref{I.5.4.3}, $\Gamma_f(X)\cap p_1^{-1}(U)\cap p_2^{-1}(V)$ is closed in $U\times_\bb{Z} V$, and so the prescheme induced by the subprescheme of $X\times_\bb{Z}Y$ associated to $\Gamma_f$ \sref{I.4.2.1} on the open subset $\Gamma_f(X)\cap p_1^{-1}(U)\cap p_2^{-1}(V)$ of its underlying space is a closed subprescheme of an affine scheme, and thus an affine scheme \sref{I.4.2.3}.
The proposition then follows from the fact that $\Gamma_f$ is an immersion.
\end{proof}

\begin{examples}[5.5.11]
\label{I.5.5.11}
The prescheme from Example~\sref{I.2.3.2} (``the projective line over a field $K$'') is \emph{separated}, because, for the cover $(X_1,X_2)$ of $X$ by affine open subsets, $X_1\cap X_2=U_{12}$ is affine, and $\Gamma(U_{12},\sh{O}_X)$, the ring of rational fractions of the form $f(s)/s^m$ with $f\in K[s]$, is generated by $K[s]$ and $1/s$, so the conditions of \sref{I.5.5.6} are satisfied.

With the same choice of $X_1$, $X_2$, $U_{12}$, and $U_{21}$ as in Example~\sref{I.2.3.2}, now take $u_{12}$ to be the isomorphism which sends $f(s)$ to $f(t)$;
we now obtain, by gluing, a \emph{non-separated integral} prescheme $X$, because the first condition of \sref{I.5.5.6} is satisfied, but not the second.
It is immediate here that $\Gamma(X,\sh{O}_X)\to\Gamma(X_1,\sh{O}_X)=K[s]$ is an isomorphism;
the inverse isomorphism defines a morphism $f:X\to\Spec(K[s])$ that is surjective, and for every $y\in\Spec(K[s])$ such that $\mathfrak{j}_y\neq(0)$, $f^{-1}(y)$ consists of a single point, but for $\mathfrak{j}_y=(0)$, $f^{-1}(y)$ consists of \emph{two} distinct points (we say that $X$ is the ``affine line over $K$ with the point $0$ doubled'').

We can also give examples where \emph{neither} of the two conditions of \sref{I.5.5.6} are satisfied.
First note that, in the prime spectrum $Y$ of the ring $A=K[s,t]$ of polynomials in two indeterminates over a field $K$, the open subset $U$ given by the union of $D(s)$ and $D(t)$ is \emph{not an affine open subset}.
Indeed, if $z$ is a section of $\sh{O}_Y$ over $U$, there exist two integers $m,n\geq0$ such that $s^mz$ and $t^nz$ are the restrictions of polynomials in $s$ and $t$ to $U$ \sref{I.1.4.1}, which is clearly possible only if the section $z$ extends to a section over the whole of $Y$, identified with a polynomial in $s$ and $t$.
If $U$ were an affine open subset, then the injection morphism $U\to Y$ would be an isomorphism \sref{I.1.7.3}, which is a contradiction, since $U\neq Y$.

With the above in mind, take two affine schemes $Y_1$ and $Y_2$, prime spectra of the rings $A_1=K[s_1,t_2]$ and $A_2=K[s_2,t_2]$ (respectively);
take $U_{12}=D(s_1)\cup D(t_1)$ and $U_{21}=D(s_2)\cup D(t_2)$, and take $u_{12}$ to be the restriction of an isomorphism $Y_2\to Y_1$ to $U_{21}$ corresponding to the isomorphism of rings that sends $f(s_1,t_1)$ to $f(s_2,t_2)$;
we then have an example where the conditions of \sref{I.5.5.6} are not satisfied (the integral prescheme thus obtained is called ``the affine plane over $K$ with the point $0$ doubled'').
\end{examples}

\begin{remark}[5.5.12]
\label{I.5.5.12}
Given some property \textbf{P} of morphisms of preschemes, consider the following propositions.
\begin{enumerate}
  \item[(i)] \emph{Every closed immersion has property~\textbf{P}.}
  \item[(ii)] \emph{The composition of any two morphisms that both have property~\textbf{P} also has property~\textbf{P}.}
  \item[(iii)] \emph{If $f:X\to X'$ and $g:Y\to Y'$ are $S$-morphisms that have property~\textbf{P}, then $f\times_S g$ has property~\textbf{P}.}
\oldpage[I]{140}
  \item[(iv)] \emph{If $f:X\to Y$ is an $S$-morphism that has property~\textbf{P}, then every $S'$-morphism $f_{(S')}$ obtained by an extension $S'\to S$ of the base prescheme also has property~\textbf{P}.}
  \item[(v)] \emph{If the composition $g\circ f$ of two morphisms $f:X\to Y$ and $g:Y\to Z$ has property~\textbf{P}, and $g$ is separated, then $f$ has property~\textbf{P}.}
  \item[(vi)] \emph{If a morphism $f:X\to Y$ has property~\textbf{P}, then so too does $f_\red$ \sref{I.5.1.5}.}
\end{enumerate}
If we suppose that (i) and (ii) are both true, then (iii) and (iv) are \emph{equivalent}, and (v) and (vi) are \emph{consequences} of (i), (ii), and (iii).

The first claim has already been shown \sref{I.3.5.1}.
Consider the factorisation \sref{I.5.3.13} $X\xrightarrow{\Gamma_f}X\times_Z Y\xrightarrow{p_2}Y$ of $f$;
the equation $p_2=(g\circ f)\times_Z 1_Y$ shows that, if $g\circ f$ has property~\textbf{P}, then so too does $p_2$, by (iii);
if $g$ is separated, then $\Gamma_f$ is a closed immersion \sref{I.5.4.3}, and so also has property~\textbf{P}, by (i);
finally, by (ii), $f$ has property~\textbf{P}.

Finally, consider the commutative diagram
\[
  \xymatrix{
    X_\red\ar[r]^{f_\red}\ar[d] &
    Y_\red\ar[d]\\
    X\ar[r]^f &
    Y,
  }
\]
where the vertical arrows are the closed immersions \sref{I.5.1.5}, and thus have property~\textbf{P}, by (i).
The hypothesis that $f$ has property~\textbf{P} implies, by (ii), that $X_\red\xrightarrow{f_\red}Y_\red\to Y$ has property~\textbf{P};
finally, since a closed immersion is separated \sref{I.5.5.1}[i], $f_\red$ has property~\textbf{P}, by (v).

Note that, if we consider the propositions
\begin{enumerate}
  \item[(i')] \emph{Every immersion has property~\textbf{P}};
  \item[(v')] \emph{If $g\circ f$ has property~\textbf{P}, then so too does $f$};
\end{enumerate}
then the above arguments show that (v') is a consequence of (i'), (ii), and (iii).
\end{remark}

\begin{env}[5.5.13]
\label{I.5.5.13}
Note that (v) and (vi) are again consequences of (i), (iii), and
\begin{enumerate}
  \item[(ii')] \emph{If $j:X\to Y$ is a closed immersion, and $g:Y\to Z$ is a morphism that has property~\textbf{P}, then $g\circ j$ has property~\textbf{P}}.
\end{enumerate}
Similarly, (v') is a consequence of (i'), (iii), and
\begin{enumerate}
  \item[(ii'')] \emph{If $j:X\to Y$ is an immersion, and $g:Y\to Z$ is a morphism that has property~\textbf{P}, then $g\circ j$ has property \textbf{P}}.
\end{enumerate}
This follows immediately from the arguments of \sref{I.5.5.12}.
\end{env}

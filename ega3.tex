\documentclass[oneside]{amsart}

\usepackage[all]{xy}
\usepackage[T1]{fontenc}
\usepackage{xstring}
\usepackage{xparse}
\usepackage{xr-hyper}
\usepackage{xcolor}
\definecolor{brightmaroon}{rgb}{0.76, 0.13, 0.28}
\usepackage[linktocpage=true,colorlinks=true,hyperindex,citecolor=blue,linkcolor=brightmaroon]{hyperref}
\usepackage[left=1.25in,right=1.25in,top=0.75in,bottom=0.75in]{geometry}
%\usepackage[charter,greekfamily=didot]{mathdesign}
%\usepackage{Baskervaldx}
\usepackage{amssymb}
\usepackage{mathrsfs}
\usepackage{mathpazo}
\linespread{1.05}

\usepackage[nobottomtitles]{titlesec}
\usepackage{marginnote}
\usepackage{enumerate}
\usepackage{longtable}
\usepackage{aurical}
\usepackage{microtype}

\externaldocument[what-]{what}
\externaldocument[intro-]{intro}
\externaldocument[ega0-]{ega0}
\externaldocument[ega1-]{ega1}
\externaldocument[ega2-]{ega2}
\externaldocument[ega3-]{ega3}
\externaldocument[ega4-]{ega4}

\newtheoremstyle{ega-env-style}%
  {}{}{\rmfamily}{}{\bfseries}{.}{ }{\thmnote{(#3)}}%

\newtheoremstyle{ega-thm-env-style}%
  {}{}{\itshape}{}{\bfseries}{. --- }{ }{\thmname{#1}\thmnote{ (#3)}}%

\newtheoremstyle{ega-defn-env-style}%
  {}{}{\rmfamily}{}{\bfseries}{. --- }{ }{\thmname{#1}\thmnote{ (#3)}}%

\theoremstyle{ega-env-style}
\newtheorem*{env}{---}

\theoremstyle{ega-thm-env-style}
\newtheorem*{theorem}{Theorem}
\newtheorem*{proposition}{Proposition}
\newtheorem*{lemma}{Lemma}
\newtheorem*{corollary}{Corollary}

\theoremstyle{ega-defn-env-style}
\newtheorem*{definition}{Definition}
\newtheorem*{example}{Example}
\newtheorem*{examples}{Examples}
\newtheorem*{remark}{Remark}
\newtheorem*{remarks}{Remarks}
\newtheorem*{notation}{Notation}

% indent subsections, see https://tex.stackexchange.com/questions/177290/.
% also make section titles bigger.
% also add § to \thesection, https://tex.stackexchange.com/questions/119667/ and https://tex.stackexchange.com/questions/308737/.
\makeatletter
\def\l@subsection{\@tocline{2}{0pt}{2.5pc}{2.2pc}{}}
\def\section{\@startsection{section}{1}%
  \z@{.7\linespacing\@plus\linespacing}{.5\linespacing}%
  {\normalfont\bfseries\Large\scshape\centering}}
\renewcommand{\@seccntformat}[1]{%
  \ifnum\pdfstrcmp{#1}{section}=0\textsection\fi%
  \csname the#1\endcsname.~}
\makeatother

%\allowdisplaybreaks[1]
%\binoppenalty=9999
%\relpenalty=9999

% for Chapter 0, Chapter I, etc.
% credit for ZeroRoman https://tex.stackexchange.com/questions/211414/
% added into scripts/make_book.py
%\newcommand{\ZeroRoman}[1]{\ifcase\value{#1}\relax 0\else\Roman{#1}\fi}
%\renewcommand{\thechapter}{\ZeroRoman{chapter}}

\def\mathcal{\mathscr}
\def\sh{\mathcal}                   % sheaf font
\def\bb{\mathbf}                    % bold font
\def\cat{\mathtt}                   % category font
\def\leq{\leqslant}                 % <=
\def\geq{\geqslant}                 % >=
\def\setmin{-}                      % set minus
\def\rad{\mathfrak{r}}              % radical
\def\nilrad{\mathfrak{N}}           % nilradical
\def\emp{\varnothing}               % empty set
\def\vphi{\phi}                     % for switching \phi and \varphi, change if needed
\def\HH{\mathrm{H}}                 % cohomology H
\def\CHH{\check{\HH}}               % Čech cohomology H
\def\RR{\mathrm{R}}                 % right derived R
\def\LL{\mathrm{L}}                 % left derived L
\def\dual#1{{#1}^\vee}              % dual
\def\kres{k}                        % residue field k
\def\C{\cat{C}}                     % category C
\def\op{^\cat{op}}                  % opposite category
\def\Set{\cat{Set}}                 % category of sets
\def\CHom{\cat{Hom}}                % functor category
\def\supertilde{{\,\widetilde{\,}\,}}   % use \supertilde instead of ^\sim
\def\GL{\bb{GL}}
\def\red{\mathrm{red}}
\def\rg{\operatorname{rg}}
\def\gr{\operatorname{gr}}
\def\Gr{\operatorname{Gr}}
\def\Hom{\operatorname{Hom}}
\def\Proj{\operatorname{Proj}}
\def\Tor{\operatorname{Tor}}
\def\Ext{\operatorname{Ext}}
\def\Supp{\operatorname{Supp}}
\def\Ker{\operatorname{Ker}}
\def\Im{\operatorname{Im}}
\def\Coker{\operatorname{Coker}}
\def\Spec{\operatorname{Spec}}
\def\Spf{\operatorname{Spf}}
\def\grad{\operatorname{grad}}
\def\dimc{\operatorname{dimc}}
\def\codim{\operatorname{codim}}
\def\id{\operatorname{id}}
\def\Der{\operatorname{Der}}
\def\Diff{\operatorname{Diff}}
\def\Hyp{\operatorname{Hyp}}

\renewcommand{\to}{\mathchoice{\longrightarrow}{\rightarrow}{\rightarrow}{\rightarrow}}
\newcommand{\from}{\mathchoice{\longleftarrow}{\leftarrow}{\leftarrow}{\leftarrow}}
\let\mapstoo\mapsto
\renewcommand{\mapsto}{\mathchoice{\longmapsto}{\mapstoo}{\mapstoo}{\mapstoo}}
\def\isoto{\simeq} % isomorphism

\renewcommand{\hat}[1]{\widehat{#1}}
\renewcommand{\tilde}[1]{\widetilde{#1}}


\def\shHom{\sh{H}\textup{\kern-2.2pt{\Fontauri\slshape om}}}   % sheaf Hom
\def\shProj{\sh{P}\textup{\kern-2.2pt{\Fontauri\slshape roj}}} % sheaf Proj
\def\shExt{\sh{E}\textup{\kern-2.2pt{\Fontauri\slshape xt}}}   % sheaf Ext
\def\shGr{\sh{G}\textup{\kern-2.2pt{\Fontauri\slshape r}}}     % sheaf Gr
\def\shDer{\sh{D}\,\textup{\kern-2.2pt{\Fontauri\slshape er}}} % sheaf Der
\def\shDiff{\sh{D}\,\textup{\kern-2.2pt{\Fontauri\slshape if{}f}}\,} % sheaf Diff
\def\shHomcont{\sh{H}\textup{\kern-2.2pt{\Fontauri\slshape om.\,cont}}}   % sheaf Hom.cont
\def\shAut{\sh{A}\textup{\kern-2.2pt{\Fontauri\slshape ut}}}   % sheaf Aut

% if unsure of a translation
%\newcommand{\unsure}[2][]{\hl{#2}\marginpar{#1}}
%\newcommand{\completelyunsure}{\unsure{[\ldots]}}
\def\unsure#1{#1 {\color{red}(?)}}
\def\completelyunsure{{\color{red}(???)}}

% use to mark where original page starts
\newcommand{\oldpage}[2][]{{\marginnote{\normalfont{\textbf{#1}~|~#2}}}\ignorespaces}
\def\sectionbreak{\begin{center}***\end{center}}

% for referencing environments.
% use as \sref{chapter-number.x.y.z}, with optional args
% for volume and indices, e.g. \sref[volume]{chapter-number.x.y.z}[i].
\NewDocumentCommand{\sref}{o m o}{%
  \IfNoValueTF{#1}%
    {\IfNoValueTF{#3}%
      {\hyperref[#2]{\normalfont{(\StrBehind{#2}{.})}}}%
      {\hyperref[#2]{\normalfont{(\StrBehind{#2}{.},~{#3})}}}}%
    {\IfNoValueTF{#3}%
      {\hyperref[#2]{\normalfont{(\textbf{#1},~\StrBehind{#2}{.})}}}%
      {\hyperref[#2]{\normalfont{(\textbf{#1},~\StrBehind{#2}{.},~{#3})}}}}%
}

% for marking changes following errata
% use as \erratum[volume]{correction} to say where the erratum is given
\newcommand{\erratum}[2][]{{#2}\marginpar{\footnotesize\textbf{Err}\textsubscript{#1}}}

% for referencing equations, use as \eref{eq:chapter-number.x.y.z}.
\newcommand{\eref}[1]{\hyperref[#1]{\normalfont{(\StrBehind{#1}{.})}}}



\begin{document}
\title{Cohomological study of coherent sheaves (EGA III)}
\maketitle

\phantomsection
\label{section:ega3}

build hack
\cite{I-1}

\tableofcontents

\section*{Summary}

\begin{longtable}{ll}
  \hyperref[section:III.1]{\textsection1}. & Cohomology of affine schemes.\\
  \hyperref[section:III.2]{\textsection2}. & Cohomological study of projective morphisms.\\
  \hyperref[section:III.3]{\textsection3}. & Finiteness theorem for proper morphisms.\\
  \hyperref[section:III.4]{\textsection4}. & The fundamental theorem of proper morphisms. Applications.\\
  \hyperref[section:III.5]{\textsection5}. & An existence theorem for coherent algebraic sheaves.\\
  \hyperref[section:III.6]{\textsection6}. & Local and global Tor functors; K\"unneth formula.\\
  \hyperref[section:III.7]{\textsection7}. & Base change for homological functors of sheaves of modules.\\

  \textsection8. & The duality theorem for projective bundles\\
  \textsection9. & Relative cohomology and local cohomology; local duality\\
  \textsection10. & Relations between projective cohomology and local cohomology. Formal completion technique along a divisor\\
  \textsection11. & Global and local Picard groups\footnote{EGA IV does not depend on \textsection\textsection8--11, and will probably be published before these chapters. \emph{[Trans.] These last four chapters were never published.}}
\end{longtable}
\bigskip

This chapter gives the fundamental theorems concerning the cohomology of coherent algebraic sheaves, with the exception of theorems explaining the theory of residues (duality theorems), which will be the subject of a later chapter.
Amongst all those included here, there are essentially six fundamental theorems, and each one is the subject of one of the first six chapters.
These results will prove to be essential tools in all that follows, even in questions which are not truly cohomological in their nature;
the reader will see the first such examples starting from \textsection4.
\textsection7 gives some more technical results, but ones which are constantly used in applications.
Finally, in \textsection\textsection8--11, we will develop certain results, related to the duality of coherent sheaves, that are particularly important for applications, and which can be explained even before the introduction of the full general theory of residues.

The content of \textsection\textsection1 and 2 is due to J.-P.~Serre, and the reader will observe that we have had only to follow (FAC).
\textsection8 and 9 are equally inspired by (FAC) (the changes necessitated by the different contexts, however, being less evident).
Finally, as we said in the Introduction, \textsection4 should be considered as the formalisation, in modern language, of the fundamental ``invariance theorem'' of Zariski's ``theory of holomorphic functions''.

We draw attention to the fact that the results of n\textsuperscript{o}3.4 (and the preliminary propositions of (\textbf{0},~13.4 to 13.7)) will not be used in what follows Chapter~III, and can thus be skipped in a first reading.
\bigskip

\section{Cohomology of affine schemes}
\label{section:III.1}

\subsection{Review of the exterior algebra complex}
\label{subsection:III.1.1}

\begin{env}[1.1.1]
\label{III.1.1.1}
Let $A$ be a ring, $\mathbf{f}=(f_i)_{1\leq i\leq r}$ a system of $r$ elements of $A$.
The \emph{exterior algebra complex $K_\bullet(\mathbf{f})$} corresponding to $\mathbf{f}$ is a chain complex (G, I, 2.2) defined in the following way: the graded $A$-module $K_\bullet(\mathbf{f})$ is equal to the \emph{exterior algebra $\wedge(A^r)$}, graded in the usual way, and the boundary map is the \emph{interior multiplication $i_\mathbf{f}$} by $\mathbf{f}$ considered as an element of the dual $\dual{(A^r)}$; we recall that $i_\mathbf{f}$ is an \emph{antiderivation} of degree $-1$ of $\wedge(A^r)$, and if $(\mathbf{e}_i)_{1\leq i\leq r}$ is the canonical basis of $A^r$, then we have $i_\mathbf{f}(\mathbf{e}_i)=f_i$; the verification of the condition $i_\mathbf{f}\circ i_\mathbf{f}=0$ is immediate.

An equivalent definition is the following: for each $i$, we consider a chain complex $K_\bullet(f_i)$ defined as follows: $K_0(f_i)=K_1(f_i)=A$, $K_n(f_i)=0$ for $n\neq 0,1$: the boundary map is defined by the condition that $d_1:A\to A$ is \emph{multiplication by $f_i$}.
We then take $K_\bullet(\mathbf{f})$ to be the \emph{tensor product $K_\bullet(f_1)\otimes K_\bullet(f_2)\otimes\cdots\otimes K_\bullet(f_r)$} (G, I, 2.7) with its total degree; the verification of the isomorphism from this complex to the complex defined above is immediate.
\end{env}

\begin{env}[1.1.2]
\label{III.1.1.2}
For every $A$-module $M$, we define the \emph{chain complex}
\[
\label{III.1.1.2.1}
  K_\bullet(\mathbf{f},M)=K_\bullet(\mathbf{f})\otimes_A M
  \tag{1.1.2.1}
\]
and the \emph{cochain complex} (G, I, 2.2)
\[
\label{III.1.1.2.2}
  K^\bullet(\mathbf{f},M)=\Hom_A(K_\bullet(\mathbf{f},M).
  \tag{1.1.2.2}
\]

If $g$ is a $k$-cochain of this latter complex, and if we set
\[
  g(i_1,\dots,i_k)=g(\mathbf{e}_{i_1}\wedge\cdots\wedge\mathbf{e}_{i_k}),
\]
then $g$ identifies with an \emph{alternating} map from $[1,r]^k$ to $M$, and it follows from the above definitions that we have
\[
\label{III.1.1.2.3}
  d^k g(i_1,i_2,\dots,i_{k+1})=\sum_{h=1}^{k+1}(-1)^{h-1}f_{i_h}g(i_1,\dots,\widehat{i_h},\dots,i_{k+1}).
  \tag{1.1.2.3}
\]
\end{env}

\begin{env}[1.1.3]
\label{III.1.1.3}
\oldpage[III]{83}
From the above complexes, we deduce as usual the \emph{homology and cohomology $A$-modules} (G, I, 2.2)
\[
  \HH_\bullet(\mathbf{f},M)=\HH_\bullet(K_\bullet(\mathbf{f},M)),
  \tag{1.1.3.1}
\]
\[
  \HH^\bullet(\mathbf{f},M)=\HH^\bullet(K^\bullet(\mathbf{f},M)).
  \tag{1.1.3.2}
\]

We define an \emph{$A$-isomorphism $K_\bullet(\mathbf{f},M)\isoto K^\bullet(\mathbf{f},M)$} by sending each chain $z=\sum(\mathbf{e}_{i_1}\wedge\cdots\wedge\mathbf{e}_{i_k})\otimes z_{i_1,\dots,i_k}$ to the cochain $g_z$ such that $g_z(j_1,\dots,j_{r-k})=\varepsilon z_{i_1,\dots,i_k}$, where $(j_h)_{1\leq h\leq r-k}$ is the strictly increasing sequence complementary to the strictly increasing sequence $(i_h)_{1\leq h\leq k}$ in $[1,r]$ and $\varepsilon=(-1)^\nu$, where $\nu$ is the number of inversions of the permutation $i_1,\dots,i_k,j_1,\dots,j_{r-k}$ of $[1,r]$.
We verify that $g_{dz}=d(g_z)$, which gives an isomorphism
\[
  \HH^i(\mathbf{f},M)\isoto\HH_{r-i}(\mathbf{f},M)\text{ for }0\leq i\leq r.
  \tag{1.1.3.3}
\]

In this chapter, we will especially consider the cohomology modules $\HH^\bullet(\mathbf{f},M)$.

For a given $\mathbf{f}$, it is immediate (G, I, 2.1) that $M\mapsto\HH^\bullet(\mathbf{f},M)$ is a \emph{cohomological functor} (T, II, 2.1) from the category of $A$-modules to the category of graded $A$-modules, zero in degrees $<0$ and $>r$.
In addition, we have
\[
  \HH^0(\mathbf{f},M)=\Hom_A(A/(\mathbf{f}),M),
  \tag{1.1.3.4}
\]
denoting by $(\mathbf{f})$ the ideal of $A$ generated by $f_1,\dots,f_r$; this follows immediately from (1.1.2.3), and it is clear that $\HH^0(\mathbf{f},M)$ identifies with the submodule of $M$ \emph{killed by $(\mathbf{f})$}.
Similarly, we have by (1.1.2.3) that
\[
  \HH^r(\mathbf{f},M)=M/\bigg(\sum_{i=1}^r f_i M\bigg)=(A/(\mathbf{f}))\otimes_A M.
  \tag{1.1.3.5}
\]

We will use the following known result, which we will recall a proof of to be complete:
\end{env}

\begin{proposition}[1.1.4]
\label{III.1.1.4}
Let $A$ be a ring, $\mathbf{f}=(f_i)_{1\leq i\leq r}$ a finite family of elements of $A$, and $M$ an $A$-module.
If, for $1\leq i\leq r$, the scaling $z\mapsto f_i\cdot z$ on $M_{i-1}=M/(f_1 M+\cdots+f_{i-1}M)$ is injective, then we have $\HH^i(\mathbf{f},M)=0$ for $i\neq r$.
\end{proposition}

It suffices to prove that $\HH_i(\mathbf{f},M)=0$ for all $i>0$ according to (1.1.3.3).
We argue by induction on $r$, the case $r=0$ being trivial.
Set $\mathbf{f}'=(f_i)_{1\leq i\leq r-1}$; this family satisfies the conditions in the statement, so if we set $L_\bullet=K_\bullet(\mathbf{f}',M)$, then we have $\HH_i(L_\bullet)=0$ for $i>0$ by hypothesis, and $\HH_0(L_\bullet)=M_{r-1}$ by virtue of (1.1.3.3) and (1.1.3.5).
To abbreviate, set $K_\bullet=K_\bullet(f_r)=K_0\oplus K_1$, with $K_0=K_1=A$, $d_1:K_1\to K_0$ multiplication by $f_r$; we have by definition \sref{III.1.1.1} that $K_\bullet(\mathbf{f},M)=K_\bullet\otimes_A L_\bullet$.
We have the following lemma:

\begin{lemma}[1.1.4.1]
\label{III.1.1.4.1}
Let $K_\bullet$ be a chain complex of free $A$-modules, zero except in dimensions $0$ and $1$.
For every chain complex $L_\bullet$ of $A$-modules, we have an exact sequence
\[
  0\to\HH_0(K_\bullet\otimes\HH_p(L_\bullet))\to\HH_p(K_\bullet\otimes L_\bullet)\to\HH_1(K_\bullet\otimes\HH_{p-1}(L_\bullet))\to 0
\]
for every index $p$.
\end{lemma}

\oldpage[III]{84}
This is a particular case of an exact sequence of low-order terms of the K\"unneth spectral sequence (M, XVII, 5.2 (a) and G, I, 5.5.2); it can be proved directly as follows.
Consider $K_0$ and $K_1$ as chain complexes (zero in dimensions $\neq 0$ and $\neq 1$ respectively); we then have an exact sequence of complexes
\[
  0\to K_0\otimes L_\bullet\to K_\bullet\otimes L_\bullet\to K_1\otimes L_\bullet\to 0,
\]
to which we can apply the exact sequence in homology
\[
  \cdots\to\HH_{p+1}(K_1\otimes L_\bullet)\xrightarrow{\partial}\HH_p(K_0\otimes L_\bullet)\to\HH_p(K_\bullet\otimes L_\bullet)\to\HH_p(K_1\otimes L_\bullet)\xrightarrow{\partial}\HH_{p-1}(K_0\otimes L_\bullet)\to\cdots.
\]
But it is evident that $\HH_p(K_0\otimes L_\bullet)=K_0\otimes\HH_p(L_\bullet)$ and $\HH_p(K_1\otimes L_\bullet)=K_1\otimes\HH_{p-1}(L_\bullet)$ for all $p$; in addition, we verify immediately that the operator $\partial:K_1\otimes\HH_p(L_\bullet)\to K_0\otimes\HH_p(L_\bullet)$ is none other than $d_1\otimes 1$; the lemma thus follows from the above exact sequence and the definition of $\HH_0(K_\bullet\otimes\HH_p(L_\bullet))$ and $\HH_1(K_\bullet\otimes\HH_{p-1}(L_\bullet))$.

The lemma having been established, the end of the proof of Proposition \sref{III.1.1.4} is immediate: the induction hypothesis of Lemma \sref{III.1.1.4.1} gives $\HH_p(K_\bullet\otimes L_\bullet)=0$ for $p\geq 2$; in addition if we show that $\HH_1(K_\bullet,\HH_0(L_\bullet))=0$, then we also deduce from Lemma \sref{III.1.1.4.1} that $\HH_1(K_\bullet\otimes L_\bullet)=0$; but by definition, $\HH_1(K_\bullet,\HH_0(L_\bullet))$ is none other than the kernel of the scaling $z\mapsto f_r\cdot z$ on $M_{r-1}$, and as by hypothesis this kernel is zero, this finishes the proof.

\begin{env}[1.1.5]
\label{III.1.1.5}
Let $\mathbf{g}=(g_i)_{1\leq i\leq r}$ be a second sequence of $r$ elements of $A$, and set $\mathbf{f}\mathbf{g}=(f_i g_i)_{1\leq i\leq r}$.
We can define a canonical homomorphism of complexes
\[
  \vphi_\mathbf{g}:K_\bullet(\mathbf{f}\mathbf{g})\to K_\bullet(\mathbf{f})
  \tag{1.1.5.1}
\]
as the canonical extension to the exterior algebra $\wedge(A^r)$ of the $A$-linear map $(x_1,\dots,x_r)\mapsto(g_1 x_1,\dots,g_r x_r)$ from $A^r$ to itself.
To see that we have a homomorphism of complexes, it suffices to note, in general, that if $u:E\to F$ is an $A$-linear map, and if $\mathbf{x}\in\dual{F}$ and $\mathbf{y}={}^t u(\mathbf{x})\in\dual{E}$, then we have the formula
\[
  (\wedge u)\circ i_\mathbf{y}=i_\mathbf{x}\circ(\wedge u);
  \tag{1.1.5.2}
\]
indeed, the two elements are antiderivations of $\wedge F$, and it suffices to check that they coincide on $F$, which follows immediately from the definitions.

When we identify $K_\bullet(\mathbf{f})$ with the tensor product of the $K_\bullet(f_i)$ \sref{III.1.1.1}, $\vphi_\mathbf{g}$ is the tensor product of the $\vphi_{g_i}$, where $\vphi_{g_i}$ is the identity in degree $0$ and multiplication by $g_i$ in degree $1$.
\end{env}

\begin{env}[1.1.6]
\label{III.1.1.6}
In particular, for every pair of integers $m$ and $n$ such that $0\leq n\leq m$, we have homomorphisms of complexes
\[
  \vphi_{\mathbf{f}^{m-n}}:K_\bullet(\mathbf{f}^m)\to K_\bullet(\mathbf{f}^n)
  \tag{1.1.6.1}
\]
and as a result, homomorphisms
\[
  \vphi_{\mathbf{f}^{m-n}}:K^\bullet(\mathbf{f}^n,M)\to K^\bullet(\mathbf{f}^m,M),
  \tag{1.1.6.2}
\]
\[
  \vphi_{\mathbf{f}^{m-n}}:\HH^\bullet(\mathbf{f}^n,M)\to\HH^\bullet(\mathbf{f}^m,M).
  \tag{1.1.6.3}
\]

\oldpage[III]{85}
The latter homomorphisms evidently satisfy the transitivity condition $\vphi_{\mathbf{f}^{m-p}}=\vphi_{\mathbf{f}^{m-n}}\circ\vphi_{\mathbf{f}^{n-p}}$ for $p\leq n\leq m$; they therefore define two \emph{inductive systems} of $A$-modules; we set
\[
  C^\bullet((\mathbf{f}),M)=\varinjlim_n K^\bullet(\mathbf{f}^n,M),
  \tag{1.1.6.4}
\]
\[
  \HH^\bullet((\mathbf{f}),M)=\HH^\bullet(C^\bullet((\mathbf{f}),M))=\varinjlim_n\HH^\bullet(\mathbf{f}^n,M),
  \tag{1.1.6.5}
\]
the last equality following from the fact that passing to the inductive limit commutes with the functor $\HH^\bullet$ (G, I, 2.1).
We will later see \sref{III.1.4.3} that $\HH^\bullet((\mathbf{f}),M)$ does not depend on the \emph{ideal $(\mathbf{f})$} of $A$ (and similarly on the $(\mathbf{f})$-pre-adic topology on $A$), which justifies the notations.

It is clear that $M\mapsto C^\bullet((\mathbf{f}),M)$ is an exact $A$-linear functor, and $M\mapsto\HH^\bullet((\mathbf{f}),M)$ is a cohomological functor.
\end{env}

\begin{env}[1.1.7]
\label{III.1.1.7}
Set $\mathbf{f}=(f_i)\in A^r$ and $\mathbf{g}=(g_i)\in A^r$; denote by $e_\mathbf{g}$ the left multiplication by the vector $\mathbf{g}\in A^r$ on the exterior algebra $\wedge(A^r)$; we know that we have the \emph{homotopy formula}
\[
  i_\mathbf{f}e_\mathbf{g}+e_\mathbf{g}i_\mathbf{f}=\langle\mathbf{g},\mathbf{f}\rangle 1
  \tag{1.1.7.1}
\]
in the $A$-module $A^r$ ($1$ denotes the identity automorphism of $A^r$); this relation also implies that \emph{in the complex $K_\bullet(\mathbf{f})$} we have
\[
  de_\mathbf{g}+e_\mathbf{g}d=\langle\mathbf{g},\mathbf{f}\rangle 1.
  \tag{1.1.7.2}
\]

If the ideal $(\mathbf{f})$ is equal to $A$, then there exists a $\mathbf{g}\in A^r$ such that $\langle\mathbf{g},\mathbf{f}\rangle=\sum_{i=1}^r g_i f_i=1$.
As a result (G, I, 2.4):
\end{env}

\begin{proposition}[1.1.8]
\label{III.1.1.8}
Suppose that the ideal $(\mathbf{f})$ generated by the $f_i$ is equal to $A$.
Then the complex $K_\bullet(\mathbf{f})$ is homotopically trivial, and so are the complexes $K_\bullet(\mathbf{f},M)$ and $K^\bullet(\mathbf{f},M)$ for every $A$-module $M$.
\end{proposition}

\begin{corollary}[1.1.9]
\label{III.1.1.9}
If $(\mathbf{f})=A$, then we have $\HH^\bullet(\mathbf{f},M)=0$ and $\HH^\bullet((\mathbf{f}),M)=0$ for every $A$-module $M$.
\end{corollary}

\begin{proof}
Indeed, we then have $(\mathbf{f}^n)=A$ for all $n$.
\end{proof}

\begin{remark}[1.1.10]
\label{III.1.1.10}
With the same notations as above, set $X=\Spec(A)$ and $Y$ the closed subprescheme of $X$ defined by the ideal $(\mathbf{f})$.
We will prove in \textsection9 that $\HH^\bullet((\mathbf{f}),M)$ is isomorphic to the cohomology $\HH_Y^\bullet(X,\widetilde{M})$ corresponding to the antifilter $\Phi$ of closed subsets of $Y$ (T, 3.2).
We will also show that Proposition \sref{III.1.2.3} applied to $X$ and to $\sh{F}=\widetilde{M}$ is a particular case of an exact sequence in cohomology
\[
  \cdots\to\HH_Y^p(X,\sh{F})\to\HH^p(X,\sh{F})\to\HH^p(X\setmin Y,\sh{F})\to\HH_Y^{p+1}(X,\sh{F})\to\cdots.
\]
\end{remark}

\subsection{\v Cech cohomology of an open cover}
\label{subsection:III.1.2}

\begin{notation}[1.2.1]
\label{III.1.2.1}
In this section, we denote:
\begin{enumerate}
  \item $X$ a prescheme;
  \item $\sh{F}$ a quasi-coherent $\sh{O}_X$-module;
\oldpage[III]{86}
  \item $A=\Gamma(X,\sh{O}_X)$, $M=\Gamma(X,\sh{F})$;
  \item $\mathbf{f}=(f_i)_{1\leq i\leq r}$ a finite system of elements of $A$;
  \item $U_i=X_{f_i}$, the open set \sref[0]{0.5.5.2} of the $x\in X$ such that $f_i(x)\neq 0$;
  \item $U=\bigcup_{i=1}^r U_i$;
  \item $\mathfrak{U}$ the cover $(U_i)_{1\leq i\leq r}$ of $U$.
\end{enumerate}
\end{notation}

\begin{env}[1.2.2]
\label{III.1.2.2}
Suppose that $X$ is either a prescheme whose underlying space is \emph{Noetherian} or a \emph{scheme} whose underlying space is \emph{quasi-compact}.
We then know \sref[I]{I.9.3.3} that we have $\Gamma(U_i,\sh{F})=M_{f_i}$.
We set
\[
  U_{i_0 i_1\cdots i_p}=\bigcap_{k=0}^p U_{i_k}=X_{f_{i_0}f_{i_1}\cdots f_{i_p}}
\]
\sref[0]{0.5.5.3}; so we also have
\[
  \Gamma(U_{i_0 i_1\cdots i_p},\sh{F})=M_{f_{i_0}f_{i_1}\cdots f_{i_p}}.
  \tag{1.2.2.1}
\]

We have \sref[0]{0.1.6.1} that $M_{f_{i_0}f_{i_1}\cdots f_{i_p}}$ identifies with the inductive limit $\varinjlim_n M_{i_0 i_1\cdots i_p}^{(n)}$, where the inductive system is formed by the $M_{i_0 i_1\cdots i_p}^{(n)}=M$, the homomorphisms $\vphi_{nm}:M_{i_0 i_1\cdots i_p}^{(m)}\to M_{i_0 i_1\cdots i_p}^{(n)}$ being multiplication by $(f_{i_0}f_{i_1}\cdots f_{i_p})^{n-m}$ for $m\leq n$.
We denote by $C_n^p(M)$ the set of \emph{alternating} maps from $[1,r]^{p+1}$ to $M$ (for all $n$); these $A$-modules also form an inductive system with respect to the $\vphi_{nm}$.
If $C^p(\mathfrak{U},\sh{F})$ is the group of \emph{alternating} \v Cech $p$-cochains relative to the cover $\mathfrak{U}$, with coefficients in $\sh{F}$ (G, II, 5.1), then it follows from the above that we can write
\[
  C^p(\mathfrak{U},\sh{F})=\varinjlim_n C_n^p(M).
  \tag{1.2.2.2}
\]

With the notations of \sref{III.1.1.2}, $C_n^p(M)$ identifies with $K^{p+1}(\mathbf{f}^n,M)$, and the map $\vphi_{nm}$ identifies with the map $\vphi_{\mathbf{f}^{n-m}}$ defined in \sref{III.1.1.6}.
We thus have, for every $p\geq 0$, a canonical functorial isomorphism
\[
  C^p(\mathfrak{U},\sh{F})\isoto C^{p+1}((\mathbf{f}),M).
  \tag{1.2.2.3}
\]

In addition, the formula (1.1.2.3) and the definition of the cohomology of a cover (G, II, 5.1) shows that the isomorphisms (1.2.2.3) are compatible with the coboundary maps.
\end{env}

\begin{proposition}[1.2.3]
\label{III.1.2.3}
If $X$ is a prescheme whose underlying space in Noetherian or a scheme whose underlying space is quasi-compact, then there exists a canonical functorial isomorphism in $\sh{F}$
\[
  \HH^p(\mathfrak{U},\sh{F})\isoto\HH^{p+1}((\mathbf{f}),M)\text{ for }p\geq 1.
  \tag{1.2.3.1}
\]

In addition, we have a functorial exact sequence in $\sh{F}$
\[
  0\to\HH^0((\mathbf{f}),M)\to M\to\HH^0(\mathfrak{U},\sh{F})\to\HH^1((\mathbf{f}),M)\to 0.
  \tag{1.2.3.2}
\]
\end{proposition}

\oldpage[III]{87}
\begin{proof}
The isomorphisms (1.2.3.1) are immediate consequences of what we saw in \sref{III.1.2.2}.
On the other hand, we have $C^0(\mathfrak{U},\sh{F})=C^1((\mathbf{f}),M)$; as a result, $\HH^0(\mathfrak{U},\sh{F})$ identifies with the subgroup of $1$-cocycles of $C^1((\mathbf{f}),M)$; as $M=C^0((\mathbf{f}),M)$, the exact sequence (1.2.3.2) is none other than the one given by the definition of the cohomology groups $\HH^0((\mathbf{f}),M)$ and $\HH^1((\mathbf{f}),M)$.
\end{proof}

\begin{corollary}[1.2.4]
\label{III.1.2.4}
Suppose that the $X_{f_i}$ are quasi-compact and that there exists $g_i\in\Gamma(U,\sh{F})$ such that $\sum_i g_i(f_i|U)=1|U$.
Then for every quasi-coherent $(\sh{O}_X|U)$-module $\sh{G}$, we have $\HH^p(\mathfrak{U},\sh{G})=0$ for $p>0$; if in addition $U=X$, then the canonical homomorphism (1.2.3.2) $M\to\HH^0(\mathfrak{U},\sh{F})$ is bijective.
\end{corollary}

\begin{proof}
As by hypothesis the $U_i=X_{f_i}$ are quasi-compact, so is $U$, and we can reduce to the case where $U=X$; the hypothesis then implies that $\HH^p((\mathbf{f}),M)=0$ for all $p\geq 0$ \sref{III.1.1.9}.
The corollary then follows immediately from (1.2.3.1) and (1.2.3.2).
\end{proof}

We note that since $\HH^0(\mathfrak{U},\sh{F})=\HH^0(U,\sh{F})$ (G, II, 5.2.2), we have again proved \sref[I]{I.1.3.7} as a special case.

\begin{remark}[1.2.5]
\label{III.1.2.5}
Suppose that $X$ is an \emph{affine scheme}; then the $U_i=X_{f_i}=D(f_i)$ are affine open sets, as well as the $U_{i_0 i_1\cdots i_p}$ (but $U$ is not necessarily affine).
In this case, the functors $\Gamma(X,\sh{F})$ and $\Gamma(U_{i_0 i_1\cdots i_p},\sh{F})$ are exact in $\sh{F}$ \sref[I]{I.1.3.11}.
If we have an exact sequence $0\to\sh{F}'\to\sh{F}\to\sh{F}''\to 0$ of quasi-coherent $\sh{O}_X$-modules, then the sequence of complexes
\[
  0\to C^\bullet(\mathfrak{U},\sh{F}')\to C^\bullet(\mathfrak{U},\sh{F})\to C^\bullet(\mathfrak{U},\sh{F}'')\to 0
\]
is exact, and thus gives an exact sequence in cohomology
\[
  \cdots\to\HH^p(\mathfrak{U},\sh{F})\to\HH^p(\mathfrak{U},\sh{F})\to\HH^p(\mathfrak{U},\sh{F}'')\xrightarrow{\partial}\HH^{p+1}(\mathfrak{U},\sh{F}')\to\cdots.
\]
On the other hand, if we set $M'=\Gamma(X,\sh{F}')$ and $M''=\Gamma(X,\sh{F}'')$, then the sequence $0\to M'\to M\to M''\to 0$ is exact; as $C^\bullet((\mathbf{f}),M)$ is an exact functor in $M$, we also have the exact sequence in cohomology
\[
  \cdots\to\HH^p((\mathbf{f}),M')\to\HH^p((\mathbf{f}),M)\to\HH^p((\mathbf{f}),M'')\xrightarrow{\partial}\HH^{p+1}((\mathbf{f}),M')\to\cdots.
\]

This being so, as the diagram
\[
  \xymatrix{
    0\ar[r] &
    C^\bullet(\mathfrak{U},\sh{F}')\ar[r]\ar[d] &
    C^\bullet(\mathfrak{U},\sh{F})\ar[r]\ar[d] &
    C^\bullet(\mathfrak{U},\sh{F}'')\ar[r]\ar[d] &
    0\\
    0\ar[r] &
    C^\bullet((\mathbf{f}),M')\ar[r] &
    C^\bullet((\mathbf{f}),M)\ar[r] &
    C^\bullet((\mathbf{f}),M'')\ar[r] &
    0
  }
\]
\oldpage[III]{88}
is commutative, we conclude that the diagrams
\[
  \xymatrix{
    \HH^p(\mathfrak{U},\sh{F}'')\ar[r]^\partial\ar[d] &
    \HH^{p+1}(\mathfrak{U},\sh{F}')\ar[d]\\
    \HH^{p+1}((\mathbf{f}),M'')\ar[r]^\partial &
    \HH^{p+2}((\mathbf{f}),M')
  }
  \tag{1.2.5.1}
\]
are commutative for all $p$~(G, I, 2.1.1).
\end{remark}

\subsection{Cohomology of an affine scheme}
\label{subsection:III.1.3}

\begin{theorem}[1.3.1]
\label{III.1.3.1}
Let $X$ be an affine scheme.
For every quasi-coherent $\sh{O}_X$-module $\sh{F}$, we have $\HH^p(X,\sh{F})=0$ for all $p>0$.
\end{theorem}

\begin{proof}
Let $\mathfrak{U}$ be a finite cover of $X$ by the affine open sets $X_{f_i}=D(f_i)$ ($1\leq i\leq r$); we then know that the ideal of $A=\Gamma(X,\sh{O}_X)$ generated by the $f_i$ is equal to $A$.
We thus conclude from Corollary~\sref{III.1.2.4} that we have $\HH^p(\mathfrak{U},\sh{F})=0$ for $p>0$.
As there are finite covers of $X$ by affine open sets which are arbitrarily fine \sref[I]{I.1.1.10}, the definition of \v Cech cohomology (G, II, 5.8) shows that we also have $\CHH^p(X,\sh{F})=0$ for $p>0$.
But this also applies to every prescheme $X_f$ for $f\in A$ \sref[I]{I.1.3.6}, hence $\CHH^p(X_f,\sh{F})=0$ for $p>0$.
As we have $X_f\cap X_g=X_{fg}$, we deduce that we also have $\HH^p(X,\sh{F})=0$ for all $p>0$, by virtue of (G, II, 5.9.2).
\end{proof}

\begin{corollary}[1.3.2]
\label{III.1.3.2}
Let $Y$ be a prescheme, $f:X\to Y$ an affine morphism \sref[II]{II.1.6.1}.
For every quasi-coherent $\sh{O}_X$-module $\sh{F}$, we have $\RR^q f_*(\sh{F})=0$ for $q>0$.
\end{corollary}

\begin{proof}
By definition $\RR^q f_*(\sh{F})$ is the $\sh{O}_Y$-module associated to the presheaf $U\mapsto\HH^q(f^{-1}(U),\sh{F})$, where $U$ varies over the open subsets of $Y$.
But the affine open sets form a basis for $Y$, and for such an open set $U$, $f^{-1}(U)$ is affine \sref[II]{II.1.3.2}, hence $\HH^q(f^{-1}(U),\sh{F})=0$ by Theorem~\sref{III.1.3.1}, which proves the corollary.
\end{proof}

\begin{corollary}[1.3.3]
\label{III.1.3.3}
Let $Y$ be a prescheme, $f:X\to Y$ an affine morphism.
For every quasi-coherent $\sh{O}_X$-module $\sh{F}$, the canonical homomorphism $\HH^p(Y,f_*(\sh{F}))\to\HH^p(X,\sh{F})$ (\textbf{0},~12.1.3.1) is bijective for all $p$.
\end{corollary}

\begin{proof}
It suffices (by \sref[0]{0.12.1.7}) to show that the edge homomorphisms $''E_2^{p0}=\HH^p(Y,f_*(\sh{F}))\to\HH^p(X,\sh{F})$ of the second spectral sequence of the composite functor $\Gamma f_*$ are bijective.
But the $E_2$-term of this spectral sequence is given by $''E_2^{pq}=\HH^p(Y,\RR^q f_*(\sh{F}))$ (G, II, 4.17.1), so it follows from Corollary~\sref{III.1.3.2} that $''E_2^{pq}=0$ for $q>0$, and the spectral sequence degenerates; hence our assertion \sref[0]{0.11.1.6}.
\end{proof}

\begin{corollary}[1.3.4]
\label{III.1.3.4}
Let $f:X\to Y$ be an affine morphism, $g:Y\to Z$ a morphism.
\oldpage[III]{89}
For every quasi-coherent $\sh{O}_X$-module $\sh{F}$, the canonical homomorphism $\RR^p g_*(f_*(\sh{F}))\to\RR^p(g\circ f)_*(\sh{F})$ (\textbf{0},~12.2.5.1) is bijective for all $p$.
\end{corollary}

\begin{proof}
It suffices to note that, according to Corollary~\sref{III.1.3.3}, for every affine open subset $W$ of $Z$, the canonical homomorphism $\HH^p(g^{-1}(W),f_*(\sh{F}))\to\HH^p(f^{-1}(g^{-1}(W)),\sh{F})$ is bijective; this proves that the homomorphism of presheaves defining the canonical homomorphism $\RR^p g_*(f_*(\sh{F}))\to\RR^p(g\circ f)_*(\sh{F})$ is bijective \sref[0]{0.12.2.5}.
\end{proof}

\subsection{Application to the cohomology of arbitrary preschemes}
\label{subsection:III.1.4}

\begin{proposition}[1.4.1]
\label{III.1.4.1}
Let $X$ be a scheme, $\mathfrak{U}=(U_\alpha)$ be a cover of $X$ by affine open sets.
For every quasi-coherent $\sh{O}_X$-module $\sh{F}$, the cohomology modules $\HH^\bullet(X,\sh{F})$ and $\HH^\bullet(\mathfrak{U},\sh{F})$ (over $\Gamma(X,\sh{O}_X)$) are canonically isomorphic.
\end{proposition}

\begin{proof}
As $X$ is a scheme, every finite intersection $V$ of open sets in the cover $\mathfrak{U}$ is affine \sref[I]{I.5.5.6}, so $\HH^q(V,\sh{F})=0$ for $g\geq 1$ by Theorem~\sref{III.1.3.1}.
The proposition then follows from a theorem of Leray (G, II, 5.4.1).
\end{proof}

\begin{remark}[1.4.2]
\label{III.1.4.2}
We note that the result of Proposition~\sref{III.1.4.1} is still true when the finite intersections of the sets $U_\alpha$ are affine, even when we do not necessarily assume that $X$ is a scheme.
\end{remark}

\begin{corollary}[1.4.3]
\label{III.1.4.3}
Let $X$ be a scheme with quasi-compact underlying space, $A=\Gamma(X,\sh{O}_X)$, and $\mathbf{f}=(f_i)_{1\leq i\leq r}$ a finite sequence of elements of $A$ such that the $X_{f_i}$ (notation of \sref{III.1.2.1}) are affine.
Then (with the notations of \sref{III.1.2.1}), for every quasi-coherent $\sh{O}_X$-module $\sh{F}$, we have a canonical isomorphism which is functorial in $\sh{F}$
\[
  \HH^q(U,\sh{F})\isoto\HH^{q+1}((\mathbf{f}),M)\text{ for }q\geq 1,
  \tag{1.4.3.1}
\]
and an exact sequence which is functorial in $\sh{F}$
\[
  0\to\HH^0((\mathbf{f}),M)\to M\to\HH^0(U,\sh{F})\to\HH^1((\mathbf{f}),M)\to 0.
  \tag{1.4.3.2}
\]
\end{corollary}

\begin{proof}
This follows immediately from Propositions~\sref{III.1.4.1} and \sref{III.1.2.3}.
\end{proof}

\begin{env}[1.4.4]
\label{III.1.4.4}
If $X$ is an \emph{affine scheme}, then it follows from Remark~\sref{III.1.2.5} and Proposition~\sref{III.1.4.1} that for all $q\geq 0$, the diagrams
\[
  \xymatrix{
    \HH^q(U,\sh{F}'')\ar[r]^\partial\ar[d] &
    \HH^{q+1}(U,\sh{F}')\ar[d]\\
    \HH^{q+1}((\mathbf{f}),M'')\ar[r]^\partial &
    \HH^{q+2}((\mathbf{f}),M')
  }
  \tag{1.4.4.1}
\]
corresponding to an exact sequence $0\to\sh{F}'\to\sh{F}\to\sh{F}''\to 0$ of quasi-coherent $\sh{O}_X$-modules (with the notations of Remark~\sref{III.1.2.5}) are commutative.
\end{env}

\begin{proposition}[1.4.5]
\label{III.1.4.5}
\oldpage[III]{90}
Let $X$ be a quasi-compact scheme, $\sh{L}$ an invertible $\sh{O}_X$-module, and consider the graded ring $A_\bullet=\Gamma_\bullet(\sh{L})$ \sref[0]{0.5.4.6}; then $\HH^\bullet(\sh{F},\sh{L})=\bigoplus_{n\in\bb{Z}}\HH^\bullet(X,\sh{F}\otimes\sh{L}^{\otimes n})$ is a graded $A_\bullet$-module, and for all $f\in A_n$, we have a canonical isomorphism
\[
  \HH^\bullet(X_f,\sh{F})\isoto(\HH^\bullet(\sh{F},\sh{L}))_{(f)}
  \tag{1.4.5.1}
\]
of $(A_\bullet)_{(f)}$-modules.
\end{proposition}

\begin{proof}
As $X$ is a quasi-compact scheme, we can calculate the cohomology of all the $\sh{O}_X$-modules $\sh{F}\otimes\sh{L}^{\otimes n}$ using the same finite cover $\mathfrak{U}=(U_i)$ consisting of the affine open sets such that the restriction $\sh{L}|U_i$ is isomorphic to $\sh{O}_X|U_i$ for each $i$ \sref{III.1.4.1}.
It is then immediate that the $U_i\cap X_f$ are affine open sets \sref[I]{I.1.3.6}, and we can thus calculate the cohomology $\HH^\bullet(X_f,\sh{F}\otimes\sh{L}^{\otimes n})$ using the cover $\mathfrak{U}|X_f=(U_i\cap X_f)$ \sref{III.1.4.1}.
It is immediate that for all $f\in A_n$, multiplication by $f$ defines a homomorphism $C^\bullet(\mathfrak{U},\sh{F}\otimes\sh{L}^m)\to C^\bullet(\mathfrak{U},\sh{F}\otimes\sh{L}^{\otimes(m+n)})$, hence a homomorphism $\HH^\bullet(\mathfrak{U},\sh{F}\otimes\sh{L}^{\otimes m})\to\HH^\bullet(\mathfrak{U},\sh{F}\otimes\sh{L}^{\otimes(m+n)})$, which establishes the first assertion.
On the other hand, for a given $f\in A_n$, it follows from \sref[I]{I.9.3.2} that we have an isomorphism of complexes of $(A_\bullet)_{(f)}$-modules
\[
  C^\bullet(\mathfrak{U}|X_f,\sh{F})\isoto\bigg(\!\!C^\bullet\bigg(\mathfrak{U},\bigoplus_{n\in\bb{Z}}\sh{F}\otimes\sh{L}^{\otimes n}\bigg)\!\!\bigg)_{(f)},
\]
taking into account \sref[I]{I.1.3.9}[ii].
Passing to the cohomology of these complexes, we induce the isomorphism (1.4.5.1), recalling that the functor $M\mapsto M_{(f)}$ is exact on the category of graded $A_\bullet$-modules.
\end{proof}

\begin{corollary}[1.4.6]
\label{III.1.4.6}
Suppose that the hypotheses of Proposition~\sref{III.1.4.5} are satisfied, and in addition suppose that $\sh{L}=\sh{O}_X$.
If we set $A=\Gamma(X,\sh{O}_X)$, then for all $f\in A$, we have a canonical isomorphism $\HH^\bullet(X_f,\sh{F})\isoto(\HH^\bullet(X,\sh{F}))_f$ of $A_f$-modules.
\end{corollary}

\begin{corollary}[1.4.7]
\label{III.1.4.7}
Let $X$ be a quasi-compact scheme, $f$ an element of $\Gamma(X,\sh{O}_X)$.
\begin{enumerate}
  \item[{\rm(i)}] Suppose that the open set $X_f$ is affine.
    Then for every quasi-coherent $\sh{O}_X$-module $\sh{F}$, every $i>0$, and every $\xi\in\HH^i(X,\sh{F})$, there exists an integer $n>0$ such that $f^n\xi=0$.
  \item[{\rm(ii)}] Conversely, suppose that $X_f$ is quasi-compact and that for every quasi-coherent sheaf of ideals $\sh{J}$ of $\sh{O}_X$ and every $\zeta\in\HH^1(X,\sh{J})$, there exists an $n>0$ such that $f^n\zeta=0$.
    Then $X_f$ is affine.
\end{enumerate}
\end{corollary}

\begin{proof}
\medskip\noindent
\begin{enumerate}
  \item[(i)] If $X_f$ is affine, then we have $\HH^i(X_f,\sh{F})=0$ for all $i>0$ \sref{III.1.3.1}, so the assertion follows directly from Corollary~\sref{III.1.4.6}.
  \item[(ii)] By virtue of Serre's criterion \sref[II]{II.5.2.1}, it suffices to prove that for every quasi-coherent sheaf of ideals $\sh{K}$ of $\sh{O}_X|X_f$, we have $\HH^1(X_f,\sh{K})=0$.
    As $X_f$ is a quasi-compact open set in a quasi-compact scheme $X$, there exists a quasi-coherent sheaf of ideals $\sh{J}$ of $\sh{O}_X$ such that $\sh{K}=\sh{J}|X_f$ \sref[I]{I.9.4.2}.
    According to Corollary~\sref{III.1.4.6}, we have $\HH^1(X_f,\sh{K})=(\HH^1(X,\sh{J}))_f$, and the hypothesis implies that the right hand side is zero, hence the assertion.
\end{enumerate}
\end{proof}

\begin{remark}[1.4.8]
\label{III.1.4.8}
We note that Corollary~\sref{III.1.4.7}[i] gives a simpler proof of the relation (\textbf{II},~4.5.13.2).
\end{remark}

\begin{lemma}[1.4.9]
\label{III.1.4.9}
\oldpage[III]{91}
Let $X$ be a quasi-compact scheme, $\mathfrak{U}=(U_i)_{1\leq i\leq n}$ a finite cover of $X$ by affine open sets, and $\sh{F}$ a quasi-coherent $\sh{O}_X$-module.
The complex of sheaves $\sh{C}^\bullet(\mathfrak{U},\sh{F})$ defined by the cover $\mathfrak{U}$ (G, II, 5.2) is then a quasi-coherent $\sh{O}_X$-module.
\end{lemma}

\begin{proof}
It follows from the definitions (G, II, 5.2) that $\sh{C}^p(\mathfrak{U},\sh{F})$ is the direct sum of the direct image sheaves of the $\sh{F}|U_{i_0\cdots i_p}$ under the canonical injection $U_{i_0\cdots i_p}\to X$.
The hypothesis that $X$ is a scheme implies that these injections are affine morphisms \sref[I]{I.5.5.6}, hence the $\sh{C}^p(\mathfrak{U},\sh{F})$ are quasi-coherent \sref[II]{II.1.2.6}.
\end{proof}

\begin{proposition}[1.4.10]
\label{III.1.4.10}
Let $u:X\to Y$ be a separated and quasi-compact morphism.
For every quasi-coherent $\sh{O}_X$-module $\sh{F}$, the $\RR^q u_*(\sh{F})$ are quasi-coherent $\sh{O}_Y$-modules.
\end{proposition}

\begin{proof}
The question is local on $Y$, so we can suppose that $Y$ is affine.
Then $X$ is a finite union of affine open sets $U_i$ ($1\leq i\leq n$); let $\mathfrak{U}$ be the cover $(U_i)$.
In addition, as $Y$ is a scheme, it follows from \sref[I]{I.5.5.10} that for every affine open $V\subset Y$, the canonical injection $u^{-1}(V)\to X$ is an affine morphism; we conclude (Proposition~\sref{III.1.4.1} and (G, II, 5.2)) that we have a canonical isomorphism
\[
  \HH^\bullet(u^{-1}(V),\sh{F})\isoto\HH^\bullet(\Gamma(V,\sh{K}^\bullet)),
  \tag{1.4.10.1}
\]
where we set $\sh{K}^\bullet=u_*(\sh{C}^\bullet(\mathfrak{U},\sh{F}))$.
According to Lemma~\sref{III.1.4.1} and \sref[I]{I.9.2.2}, $\sh{K}^\bullet$ is a quasi-coherent $\sh{O}_Y$-module; moreover, it constitutes a \emph{complex of sheaves} since so is $\sh{C}^\bullet(\mathfrak{U},\sh{F})$.
It then follows from the definition of the cohomology $\sh{H}^\bullet(\sh{K}^\bullet)$ (G, II, 4.1) that the latter consists of quasi-coherent $\sh{O}_Y$-modules \sref[I]{I.4.1.1}.
As (for $V$ affine in $Y$) the functor $\Gamma(V,\sh{G})$ is exact in $\sh{G}$ on the category of quasi-coherent $\sh{O}_Y$-modules, we have (G, II, 4.1)
\[
  \HH^\bullet(\Gamma(V,\sh{K}^\bullet))=\Gamma(V,\sh{H}^\bullet(\sh{K}^\bullet)).
  \tag{1.4.10.2}
\]

Finally, we note that it follows from the definition of the canonical homomorphism
\[
  \HH^\bullet(\mathfrak{U},\sh{F})\to\HH^\bullet(X,\sh{F}),
\]
given in (G, II, 5.2), that if $V'\subset V$ is a second affine open subset of $Y$, then the
diagram
\[
  \xymatrix{
    \HH^\bullet(u^{-1}(V),\sh{F})\ar[r]^\sim\ar[d] &
    \HH^\bullet(\Gamma(V,\sh{K}^\bullet))\ar[d]\\
    \HH^\bullet(u^{-1}(V'),\sh{F})\ar[r]^\sim &
    \HH^\bullet(\Gamma(V',\sh{K}^\bullet))
  }
\]
is commutative.
We thus conclude from the above that the isomorphisms (1.4.10.1) define an isomorphism of $\sh{O}_Y$-modules
\[
  \RR^\bullet u_*(\sh{F})\isoto\sh{H}^\bullet(\sh{K}^\bullet),
  \tag{1.4.10.3}
\]
\oldpage[III]{92}
and as a result, $\RR^\bullet u_*(\sh{F})$ is quasi-coherent.
\end{proof}

In addition, it follows from (1.4.10.3), (1.4.10.2), and (1.4.10.1) that:
\begin{corollary}[1.4.11]
\label{III.1.4.11}
Under the hypotheses of Proposition~\sref{III.1.4.10}, for every affine open set $V$ of $Y$, the canonical homomorphism
\[
  \HH^q(u^{-1}(V),\sh{F})\to\Gamma(V,\RR^1 u_*(\sh{F}))
  \tag{1.4.11.1}
\]
is an isomorphism for all $q\geq 0$.
\end{corollary}

\begin{corollary}[1.4.12]
\label{III.1.4.12}
Suppose that the hypotheses of Proposition~\sref{III.1.4.10} are satisfied, and in addition suppose that $Y$ is quasi-compact.
Then there exists an integer $r>0$ such that for every quasi-coherent $\sh{O}_X$-module $\sh{F}$ and every integer $q>r$, we have $\RR^q u_*(\sh{F})=0$.
If $Y$ is affine, then we can take for $r$ an integer such that there exists a cover of $X$ consisting of $r$ affine open sets.
\end{corollary}

\begin{proof}
As we can cover $Y$ by a finite number of affine open sets, we can reduce to proving the second assertion, by virtue of Corollary~\sref{III.1.4.11}.
If $\mathfrak{U}$ is a cover of $X$ by $r$ affine open sets, then we have $\HH^q(\mathfrak{U},\sh{F})=0$ for $q>r$, since the cochains of $C^q(\mathfrak{U},\sh{F})$ are alternating; the assertion thus follows from Proposition~\sref{III.1.4.1}.
\end{proof}

\begin{corollary}[1.4.13]
\label{III.1.4.13}
Suppose that the hypotheses of Proposition~\sref{III.1.4.10} are satisfied, and in addition suppose that $Y=\Spec(A)$ is affine.
Then for every quasi-coherent $\sh{O}_X$-module $\sh{F}$ and every $f\in A$, we have
\[
  \Gamma(Y_f,\RR^q u_*(\sh{F}))=(\Gamma(Y,\RR^q u_*(\sh{F}))_f
\]
up to canonical isomorphism.
\end{corollary}

\begin{proof}
This follows from the fact that $\RR^q u_*(\sh{F})$ is a quasi-coherent $\sh{O}_Y$-module \sref[I]{I.1.3.7}.
\end{proof}

\begin{proposition}[1.4.14]
\label{III.1.4.14}
Let $f:X\to Y$ be a separated and quasi-compact morphism, $g:Y\to Z$ an affine morphism.
For every quasi-coherent $\sh{O}_X$-module $\sh{F}$, the canonical homomorphism $\RR^p(g\circ f)_*(\sh{F})\to g_*(\RR^p f_*(\sh{F}))$ (\textbf{0},~12.2.5.2) is bijective for all $p$.
\end{proposition}

\begin{proof}
For every affine open subset $W$ of $Z$, $g^{-1}(W)$ is an affine open subset of $Y$.
The homomorphism of presheaves defining the canonical homomorphism
\[
  \RR^p(g\circ f)_*(\sh{F})\to g_*(\RR^p f_*(\sh{F}))
\]
\sref[0]{0.12.2.5} is thus bijective by Corollary~\sref{III.1.4.11}.
\end{proof}

\begin{proposition}[1.4.15]
\label{III.1.4.15}
Let $u:X\to Y$ be a separated morphism of finite type, $v:Y'\to Y$ a flat morphism of preschemes \sref[0]{0.6.7.1}; let $u'=u_{(Y')}$, such that we have the commutative diagram
\[
  \xymatrix{
    X\ar[d]_u &
    X'=X_{(Y')}\ar[l]_{v'}\ar[d]^{u'}\\
    Y &
    Y'\ar[l]_{v}.
  }
  \tag{1.4.15.1}
\]
Then for every quasi-coherent $\sh{O}_X$-module $\sh{F}$, $\RR^q u_*'(\sh{F}')$ is canonically isomorphic to $\RR^q u_*(\sh{F})\otimes_{\sh{O}_Y}\sh{O}_{Y'}=v^*(\RR^q u_*(\sh{F}))$ for all $q\geq 0$, where $\sh{F}'={v'}^*(\sh{F})=\sh{F}\otimes_{\sh{O}_Y}\sh{O}_{Y'}$.
\end{proposition}

\oldpage[III]{93}
\begin{proof}
The canonical homomorphism $\rho:\sh{F}\to v_*'({v'}^*(\sh{F}))$ (\textbf{0},~4.4.3.2) defines by functoriality a homomorphism
\[
  \RR^q u_*(\sh{F})\to\RR^q u_*(v_*'(\sh{F}')).
  \tag{1.4.15.2}
\]

On the other hand, we have, by setting $w=u\circ v'=v\circ u'$, the canonical homomorphisms (\textbf{0}~,12.2.5.1 and 12.2.5.2)
\[
  \RR^q u_*(v_*'(\sh{F}'))\to\RR^q w_*(\sh{F}')\to v_*(\RR^q u_*'(\sh{F}')).
  \tag{1.4.15.3}
\]

Composing (1.4.15.3) and (1.4.15.2), we have a homomorphism
\[
  \psi:\RR^q u_*(\sh{F})\to v_*(\RR^q u_*'(\sh{F}')),
\]
and finally we obtain a canonical homomorphism (whose definition does not make \emph{any assumptions} on $v$)
\[
  \psi^\sharp:v^*(\RR^q u_*(\sh{F}))\to\RR^q u_*'(\sh{F}'),
  \tag{1.4.15.4}
\]
and it is necessary to prove that it is an isomorphism when $v$ is \emph{flat}.
It is clear that the question is local on $Y$ and $Y'$, and we can therefore suppose that $Y=\Spec(A)$ and $Y'=\Spec(B)$; we will also use the following lemma:
\begin{lemma}[1.4.15.5]
\label{III.1.4.15.5}
Let $\vphi:A\to B$ be a ring homomorphism, $Y=\Spec(A)$, $X=\Spec(B)$, $f:X\to Y$ the morphism corresponding to $\vphi$, and $M$ a $B$-module.
For the $\sh{O}_X$-module $\widetilde{M}$ to be $f$-flat \sref[0]{0.6.7.1}, it is necessary and sufficient for $M$ to be a flat $A$-module.
In particular, for the morphism $f$ to be flat, it is necessary and sufficient for $B$ to be a flat $A$-module.
\end{lemma}

This follows from the definition \sref[0]{0.6.7.1} and from \sref[0]{0.6.3.3}, taking into account \sref[I]{I.1.3.4}.

This being so, it follows from (1.4.11.1) and the definitions of the homomorphisms (1.4.15.3) (cf.~\sref[0]{0.12.2.5}) that $\psi$ then corresponds to the composite morphism
\[
  \HH^q(X,\sh{F})\xrightarrow{\rho_q}\HH^q(X,v_*'({v'}^*(\sh{F})))\xrightarrow{\theta_q}\HH^q(X',{v'}^*(v_*'({v'}^*(\sh{F}))))\xrightarrow{\sigma_q}\HH^q(X',{v'}^*(\sh{F})),
\]
where $\rho_q$ and $\sigma_q$ are the homomorphisms in cohomology corresponding to the canonical morphisms $\rho$ and $\sigma:{v'}^*(v_*'(\sh{G}'))\to\sh{G}'$, and $\theta_q$ is the $\vphi$-morphism (\textbf{0},~12.1.3.1) relative to the $\sh{O}_X$-module $v_*'({v'}^*(\sh{F}))$.
But by the functoriality of $\theta_q$, we have the commutative diagram
\[
  \xymatrix{
    \HH^q(X,\sh{F})\ar[rr]^{\rho_q}\ar[dd]_{\theta_q} & &
    \HH^q(X,v_*'({v'}^*(\sh{F})))\ar[dd]^{\theta_q}\\\\
    \HH^q(X',{v'}^*(\sh{F}))\ar[rr]^{{v'}^*(\rho_q)} & &
    \HH^q(X',{v'}^*(v_*'({v'}^*(\sh{F})))),
  }
\]
\oldpage[III]{94}
and as by definition \sref[0]{0.4.4.3} ${v'}^*(\rho)$ is the inverse of $\sigma$, we see that the composite morphism considered above is finally none other than $\theta_q$; as a result, $\psi^\sharp$ is the associated $B$-homomorphism $\HH^q(X,\sh{F})\otimes_A B\to\HH^q(X',\sh{F}')$.
As $u$ is of finite type, $X$ is a finite union of affine open sets $U_i$ ($1\leq i\leq r$); let $\mathfrak{U}$ be the cover $(U_i)$.
As $v$ is an affine morphism, so is $v'$ \sref[II]{II.1.6.2}[iii], and as a result the $U_i'={v'}^{-1}(U_i)$ form an affine open cover $\mathfrak{U}'$ of $X'$.
We then know \sref[0]{0.12.1.4.2} that the diagram
\[
  \xymatrix{
    \HH^q(\mathfrak{U},\sh{F})\ar[r]^{\theta_q}\ar[d] &
    \HH^q(\mathfrak{U}',\sh{F}')\ar[d]\\
    \HH^q(X,\sh{F})\ar[r]^{\theta_q} &
    \HH^q(X',\sh{F}')
  }
\]
is commutative, and the vertical arrows are isomorphisms since $X$ and $X'$ are schemes \sref[I]{I.1.4.1}.
As a result, it suffices to prove that the canonical $\vphi$-morphism $\theta_q:\HH^q(\mathfrak{U},\sh{F})\to\HH^q(\mathfrak{U}',\sh{F}')$ is such that the associated $B$-homomorphism
\[
  \HH^q(\mathfrak{U},\sh{F})\otimes_A B\to\HH^q(\mathfrak{U}',\sh{F}')
\]
is an isomorphism.
For every sequence $\mathbf{s}=(i_k)_{0\leq k\leq p}$ of $p+1$ indices of $[1,r]$, set $U_\mathbf{s}=\bigcap_{k=0}^p U_{i_k}$, $U_\mathbf{s}'=\bigcap_{k=0}^p U_{i_k}'={v'}^{-1}(U_\mathbf{s})$, $M_\mathbf{s}=\Gamma(U_\mathbf{s},\sh{F})$, and $M_\mathbf{s}'=\Gamma(U_\mathbf{s}',\sh{F}')$.
The canonical map $M_\mathbf{s}\otimes_A B\to M_\mathbf{s}'$ is an isomorphism \sref[I]{I.1.6.5}, hence the canonical map $C^p(\mathfrak{U},\sh{F})\otimes_A B\to C^p(\mathfrak{U}',\sh{F}')$ is an isomorphism, by which $d\otimes 1$ identifies with the coboundary map $C^p(\mathfrak{U}',\sh{F}')\to C^{p+1}(\mathfrak{U}',\sh{F}')$.
As $B$ is a \emph{flat} $A$-module, it follows from the definition of the cohomology modules that the canonical map $\HH^q(\mathfrak{U},\sh{F})\otimes_A B\to\HH^q(\mathfrak{U}',\sh{F}')$ is an isomorphism \sref[0]{0.6.1.1}.
This result will later be generalized in \textsection6.
\end{proof}

\begin{corollary}[1.4.16]
\label{III.1.4.16}
Let $A$ be a ring, $X$ an $A$-scheme of finite type, and $B$ an $A$-algebra which is faithfully flat over $A$.
For $X$ to be affine, it is necessary and sufficient for $X\otimes_A B$ to be.
\end{corollary}

\begin{proof}
The condition is evidently necessary \sref[I]{I.3.2.2}; we show that it is sufficient.
As $X$ is separated over $A$ and the morphism $\Spec(B)\to\Spec(A)$ is flat, it follows from Proposition~\sref{III.1.4.1} that we have
\[
\label{III.1.4.16.1}
  \HH^i(X\otimes_A B,\sh{F}\otimes_A B)=\HH^i(X,\sh{F})\otimes_A B
  \tag{1.4.16.1}
\]
for every $i\geq 0$ and every quasi-coherent $\sh{O}_X$-module $\sh{F}$.
If $X\otimes_A B$ is affine, the left hand side of (1.4.16.1) is zero for $i=1$, hence so is $\HH^1(X,\sh{F})$ since $B$ is a faithfully flat $A$-module.
As $X$ is a quasi-compact scheme, we finish the proof by Serre's criterion \sref[II]{II.5.2.1}.
\end{proof}

\begin{proposition}[1.4.17]
\label{III.1.4.17}
Let $X$ be a prescheme, $0\to\sh{F}\xrightarrow{u}\sh{G}\xrightarrow{v}\sh{H}\to 0$ an exact sequence of $\sh{O}_X$-modules.
If $\sh{F}$ and $\sh{H}$ are quasi-coherent, then so is $\sh{G}$.
\end{proposition}

\oldpage[III]{95}
\begin{proof}
The question is local on $X$, so we can suppose that $X=\Spec(A)$ is affine, and it then suffices to prove that $\sh{G}$ satisfies the conditions (d1) and (d2) of \sref[I]{I.1.4.1} (with $V=X$).
The verification of (d2) is immediate, because if $t\in\Gamma(X,\sh{G})$ is zero when restricted to $D(f)$, then so is its image $v(t)\in\Gamma(X,\sh{H})$; therefore there exists an $m>0$ such that $f^m v(t)=v(f^m t)=0$ \sref[I]{I.1.4.1}, and as $\Gamma$ is left exact, $f^m t=u(s)$, where $s\in\Gamma(X,\sh{F})$; as $u$ is injective, the restriction of $s$ to $D(f)$ is zero, hence \sref[I]{I.1.4.1} there exists an integer $n>0$ such that $f^n s=0$; we finally deduce that $f^{m+n}t=u(f^n s)=0$.

We now check (d1); let $t'\in\Gamma(D(f),\sh{G})$; as $\sh{H}$ is quasi-coherent, there exists an integer $m$ such that $f^m v(t')=v(f^m t')$ extends to a section $z\in\Gamma(X,\sh{H})$ \sref[I]{I.1.4.1}.
But in virtue of Theorem~\sref{III.1.3.1} (or (\textbf{I},~5.1.9.2)) applied to the quasi-coherent $\sh{O}_X$-module $\sh{F}$, the sequence $\Gamma(X,\sh{G})\to\Gamma(X,\sh{H})\to 0$ is exact, so there exists $t\in\Gamma(X,\sh{G})$ such that $z=v(t)$; we thus see that $v(f^m t'-t'')=0$, denoting by $t''$ the restriction of $t$ to $D(f)$; thus we have $f^m t'-t''=u(s')$, where $s'\in\Gamma(D(f),\sh{F})$.
But as $\sh{F}$ is quasi-coherent, there exists an integer $n>0$ such that $f^n s'$ extends to a section $s\in\Gamma(X,\sh{F})$; as $f^{m+n}t'-f^n t''=u(f^n s')$, we see that $f^{m+n}t'$ is the restriction to $D(f)$ of a section $f^n t+u(f^n s)\in\Gamma(X,\sh{G})$, which finishes the proof.
\end{proof}


\input{ega3/ega3-2}
\section{Finiteness theorem for proper morphisms}
\label{section:III.3}

\subsection{The d\'evissage lemma}
\label{subsection:III.3.1}

\begin{definition}[3.1.1]
\label{III.3.1.1}
Let $\C$ be an abelian category.
We say that a subset $\C'$ of the set of objects of $\C$ is \emph{exact} if $0\in\C'$ and if, for every exact sequence $0\to A'\to A\to A''\to 0$ in $\C$ such that two of the objects $A$, $A'$, $A''$ are in $\C'$, then the third is also in $\C'$.
\end{definition}

\begin{theorem}[3.1.2]
\label{III.3.1.2}
Let $X$ be a Noetherian prescheme; we denote by $\C$ the abelian category of coherent $\sh{O}_X$-modules.
Let $\C'$ be an exact subset of $\C$, $X'$ a closed subset of the underlying space of $X$.
Suppose that for every closed irreducible subset $Y$ of $X'$, with generic point $y$, there exists an $\sh{O}_X$-module $\sh{G}\in\C'$ such that $\sh{G}_y$ is a $\kres(y)$-vector space of dimension~$1$.
Then every coherent $\sh{O}_X$-module with support contained in $X'$ is in $\C'$ (and in particular, if $X'=X$, then we have $\C'=\C$).
\end{theorem}

\begin{proof}
Consider the following property $\textbf{P}(Y)$ of a closed subset $Y$ of $X'$: every coherent $\sh{O}_X$-module with support contained in $Y$ is in $\C'$.
By virtue of the principle of Noetherian induction \sref[0]{0.2.2.2}, we see that we can reduce to showing that \emph{if $Y$ is a closed subset of $X'$ such that the property $\textbf{P}(Y')$ is true for every closed subset $Y'$ of $Y$, distinct from $Y$, then $\textbf{P}(Y)$ is true}.

Therefore, let $\sh{F}\in\C$ have support contained in $Y$, and we show that $\sh{F}\in\C'$.
Denote also by $Y$ the reduced closed subprescheme of $X$ having $Y$ for its underlying space \sref[I]{I.5.2.1}; it is defined by a coherent sheaf of ideals $\sh{J}$ of $\sh{O}_X$.
We know \sref[I]{I.9.3.4} that there exists an integer $n>0$ such that $\sh{J}^n\sh{F}=0$; for $1\leq k\leq n$, we thus have an exact sequence
\[
  0\to\sh{J}^{k-1}\sh{F}/\sh{J}^k\sh{F}\to\sh{F}/\sh{J}^k\sh{F}\to\sh{F}/\sh{J}^{k-1}\sh{F}\to 0
\]
 of coherent $\sh{O}_X$-modules (\sref[I]{I.5.3.6} and \sref[I]{I.5.3.3}); as $\C'$ is exact, we see, by induction on $k$, that it suffices to show that each of the $\sh{F}_k=\sh{J}^{k-1}\sh{F}/\sh{J}^k\sh{F}$ is in $\C'$.
We thus reduce to proving that $\sh{F}\in\C'$ under the additional hypothesis that $\sh{J}\sh{F}=0$; it is equivalent to say that $\sh{F}=j_*(j^*(\sh{F}))$, where $j$ is the canonical injection $Y\to X$.
Let us now consider two cases:
\begin{enumerate}
  \item[(a)] $Y$ is \emph{reducible}.
    Let $Y=Y'\cap Y''$, where $Y'$ and $Y''$ are closed subsets of $Y$, distinct from $Y$; denote also by $Y'$ and $Y''$ the reduced closed subpreschemes of $X$ having $Y$ and $Y''$ for their respective underlying spaces, which are defined respectively by sheaves of ideals $\sh{J}'$ and $\sh{J}''$ of $\sh{O}_X$.
    Set $\sh{F}'=\sh{F}\otimes_{\sh{O}_X}(\sh{O}_X/\sh{J}')$ and $\sh{F}''=\sh{F}\otimes_{\sh{O}_X}(\sh{O}_X/\sh{J}'')$.
    The canonical homomorphisms $\sh{F}\to\sh{F}'$ and $\sh{F}\to\sh{F}''$ thus define a homomorphism $u:\sh{F}\to\sh{F}'\oplus\sh{F}''$.
    We show that for every $z\not\in Y'\cap Y''$, the homomorphism $u_z:\sh{F}_z\to\sh{F}_z'\oplus\sh{F}_z''$ is \emph{bijective}.
    Indeed, we have $\sh{J}'\cap\sh{J}''=\sh{J}$, since the question is local and
\oldpage[III]{116}
    the above equality follows from (\sref[I]{I.5.2.1} and \sref[I]{I.1.1.5}); if $z\not\in Y''$, then we have $\sh{J}_z'=\sh{J}_z$, hence $\sh{F}_z'=\sh{F}_z$ and $\sh{F}_z''=0$, which establishes our assertion in this case; we reason similarly for $z\not\in Y'$.
    As a result, the kernel and cokernel of $u$, which are in $\C$ \sref[0]{0.5.3.4}, have their support in $Y'\cap Y''$, and thus is in $\C'$ by hypothesis; for the same reason, $\sh{F}'$ and $\sh{F}''$ are in $\C'$, hence also $\sh{F}'\oplus\sh{F}''$, as $\C'$ is exact.
    The conclusion then follows from the consideration of the two exact sequences
    \[
      0\to\Im u\to\sh{F}'\oplus\sh{F}''\to\Coker u\to 0,
    \]
    \[
      0\to\Ker u\to\sh{F}\to\Im u\to 0,
    \]
    and the hypothesis that $\C'$ is exact.
  \item[(b)] $Y$ is irreducible, and as a result, the subprescheme $Y$ of $X$ is \emph{integral}.
    If $y$ is its generic point, then we have $(\sh{O}_Y)_y=\kres(y)$, and as $j^*(\sh{F})$ is a coherent $\sh{O}_Y$-module, $\sh{F}_y=(j^*(\sh{F}))_y$ is a $\kres(y)$-vector space of finite dimension~$m$.
    By hypothesis, there is a coherent $\sh{O}_X$-module $\sh{G}\in\C'$ (necessarily of support $Y$) such that $\sh{G}_y$ is a $\kres(y)$-vector space of dimension~$1$.
    As a result, there is a $\kres(y)$-isomorphism $(\sh{G}_y)^m\isoto\sh{F}_y$, which is also an $\sh{O}_Y$-isomorphism, and as $\sh{G}^m$ and $\sh{F}$ are coherent, there exists an open neighbourhood $W$ of $y$ in $X$ and an isomorphism $\sh{G}^m|W\isoto\sh{F}|W$ \sref[0]{0.5.2.7}.
    Let $\sh{H}$ be the graph of this isomorphism, which is a coherent $(\sh{O}_X|W)$-submodule of $(\sh{G}^m\oplus\sh{F})|W$, canonically isomorphic to $\sh{G}^m|W$ and to $\sh{F}|W$; there thus exists a coherent $\sh{O}_X$-submodule $\sh{H}_0$ of $\sh{G}^m\oplus\sh{F}$, inducing $\sh{H}$ on $W$ and $0$ on $X\setmin Y$, since $\sh{G}^m$ and $\sh{F}$ have $Y$ for their support~\sref[I]{I.9.4.7}.
    The restrictions $v:\sh{H}_0\to\sh{G}^m$ and $w:\sh{H}_0\to\sh{F}$ of the canonical projections of $\sh{G}^m\oplus\sh{F}$ are then homomorphisms of coherent $\sh{O}_X$-modules, which, on $W$ and on $X\setmin Y$, reduce to isomorphisms; in other words, the kernels and cokernels of $v$ and $w$ have their support in the closed set $Y\setmin(Y\cap W)$, distinct from $Y$.
    They are in $\C'$; on the other hand, we have $\sh{G}^m\in\C'$ since $\sh{G}\in\C'$ and since $\C'$ is exact.
    We conclude successively, by the exactness of $\C'$, that $\sh{H}_0\in\C'$, then $\sh{F}\in\C'$.
Q.E.D.
\end{enumerate}
\end{proof}

\begin{corollary}[3.1.3]
\label{III.3.1.3}
Suppose that the exact subset $\C'$ of $\C$ has in addition the property that any coherent direct factor of a coherent $\sh{O}_X$-module $\sh{M}\in\C'$ is also in $\C'$.
In this case, the conclusion of Theorem~\sref{III.3.1.2} is still valid when the condition ``$\sh{G}_y$ is a $\kres(y)$-vector space of dimension~$1$'' is replaced by $\sh{G}_y\neq 0$ (this is equivalent to $\Supp(\sh{G})=Y$).
\end{corollary}

\begin{proof}
The reasoning of Theorem~\sref{III.3.1.2} must be modified only in the case~(b); now $\sh{G}_y$ is a $\kres(y)$-vector space of dimension $q>0$, and as a result, we have an $\sh{O}_Y$-isomorphism $(\sh{G}_y)^m\isoto(\sh{F}_y)^q$; the end of the reasoning in Theorem~\sref{III.3.1.2} then proves that $\sh{F}^q\in\C'$, and the additional hypothesis on $\C'$ implies that $\sh{F}\in\C'$.
\end{proof}

\subsection{The finiteness theorem: the case of usual schemes}
\label{subsection:III.3.2}

\begin{theorem}[3.2.1]
\label{III.3.2.1}
Let $Y$ be a locally Noetherian prescheme, $f:X\to Y$ a proper morphism.
For every coherent $\sh{O}_X$-module $\sh{F}$, the $\sh{O}_Y$-modules $\RR^q f_*(\sh{F})$ are coherent for $q\geq 0$.
\end{theorem}

\begin{proof}
Since the questions is local on $Y$, we can suppose $Y$ Noetherian, thus $X$ Noetherian~\sref[I]{I.6.3.7}.
The coherent $\sh{O}_X$-modules $\sh{F}$ for which the conclusion of Theorem~\sref{III.3.2.1} is true forms an \emph{exact} subset $\C'$ of the category $\C$ of coherent $\sh{O}_X$-modules.
\oldpage[III]{117}
Indeed, let $0\to\sh{F}'\to\sh{F}\to\sh{F}''\to 0$ is an exact sequence of coherent $\sh{O}_X$-modules; suppose for example that $\sh{F}'$ and $\sh{F}''$ belong to $\C'$; we have the long exact sequence in cohomology
\[
  \RR^{q-1}f_*(\sh{F}'')\xrightarrow{\partial}\RR^q f_*(\sh{F}')\to\RR^q f_*(\sh{F})\to\RR^q f_*(\sh{F}'')\xrightarrow{\partial}\RR^{q+1}f_*(\sh{F}'),
\]
in which by hypothesis the outer four terms are coherent; it is the same for the middle term $\RR^q f_*(\sh{F})$ by (\sref[0]{0.5.3.4}~and~\sref[0]{0.5.3.3}).
We show in the same way that when $\sh{F}$ and $\sh{F}'$ (resp.~$\sh{F}$ and $\sh{F}''$) are in $\C'$, then so is $\sh{F}''$ (resp.~$\sh{F}'$).
In addition, every coherent \emph{direct factor} $\sh{F}'$ of an $\sh{O}_X$-module $\sh{F}\in\C'$ belongs to $\C'$: indeed, $\RR^q f_*(\sh{F}')$ is then a direct factor of $\RR^q f_*(\sh{F})$~(G,~II,~4.4.4), therefore it is of finite type, and as it is quasi-coherent~\sref{III.1.4.10}, it is coherent, as $Y$ is Noetherian.
By virtue of Corollary~\sref{III.3.1.3}, we reduce to proving that when $X$ is \emph{irreducible} with generic point $x$, there exists \emph{one} coherent $\sh{O}_X$-module $\sh{F}$ belonging to $\C'$, such that $\sh{F}_x\neq 0$: indeed, if this point is established, then it can be applied to any irreducible closed subprescheme $Y$ of $X$, since if $j:Y\to X$ is the canonical injection, then $f\circ j$ is proper~\sref[II]{II.5.4.2}, and if $\sh{G}$ is a coherent $\sh{O}_Y$-module with support $Y$, then $j_*(\sh{G})$ is a coherent $\sh{O}_X$-module such that $\RR^q(f\circ j)_*(\sh{G})=\RR^q f_*(j_*(\sh{G}))$~(G,~II,~4.9.1), therefore we can apply Corollary~\sref{III.3.1.3}.

By virtue of Chow's lemma~\sref[II]{II.5.6.2}, there exists an irreducible prescheme $X'$ an a \emph{projective} and surjective morphism $g:X'\to X$ such that $f\circ g:X'\to Y$ is \emph{projective}.
There exists an ample $\sh{O}_X$-module $\sh{L}$ for $g$~\sref[II]{II.5.3.1}; we apply the fundamental theorem of projective morphisms~\sref{III.2.2.1} to $g:X'\to X$ and with $\sh{L}$: there thus exists an integer $n$ such that $\sh{F}=g_*(\sh{O}_{X'}(n))$ is a coherent $\sh{O}_X$-module and $\RR^q g_*(\sh{O}_{X'}(n))=0$ for all $q>0$; in addition, as $g^*(g_*(\sh{O}_{X'}(n)))\to\sh{O}_{X'}(n)$ is surjective for $n$ large enough~\sref{III.2.2.1}, we see that we can suppose, at the generic point $x$ of $X$, that we have $\sh{F}_x\neq 0$~\sref[II]{II.3.4.7}.
On the other hand, as $f\circ g$ is projective as $Y$ is Noetherian, the $\RR^q(f\circ g)_*(\sh{O}_{X'}(n))$ are \emph{coherent}~\sref{III.2.2.1}.
This being so, $\RR^\bullet(f\circ g)_*(\sh{O}_{X'}(n))$ is the abutment of a Leray spectral sequence, whose $E_2$-term is given by $E_2^{pq}=\RR^p f_*(\RR^q g_*(\sh{O}_{X'}(n)))$; the above shows that this spectral sequence degenerates, and we then know~\sref[0]{0.11.1.6} that $E_2^{p0}=\RR^p f_*(\sh{F})$ is isomorphic to $\RR^p(f\circ g)_*(\sh{O}_{X'}(n))$, which finishes the proof.
\end{proof}

\begin{corollary}[3.2.2]
\label{III.3.2.2}
Let $Y$ be a locally Noetherian prescheme.
For every proper morphism $f:X\to Y$, the direct image under $f$ of any coherent $\sh{O}_X$-module is a coherent $\sh{O}_Y$-module.
\end{corollary}

\begin{corollary}[3.2.3]
\label{III.3.2.3}
Let $A$ be a Noetherian ring, $X$ a proper scheme over $A$; for every coherent $\sh{O}_X$-module $\sh{F}$, the $\HH^p(X,\sh{F})$ are $A$-modules of finite type, and there exists an integer $r>0$ such that for every coherent $\sh{O}_X$-module $\sh{F}$ and all $p>r$, $\HH^p(X,\sh{F})=0$.
\end{corollary}

\begin{proof}
The second assertion has already been proved~\sref{III.1.4.12}; the first follows from the finiteness theorem~\sref{III.3.2.1}, taking into account Corollary~\sref{III.1.4.11}.
\end{proof}

In particular, if $X$ is a \emph{proper algebraic scheme} over a field $k$, then, for every coherent $\sh{O}_X$-module $\sh{F}$, the $\HH^p(X,\sh{F})$ are \emph{finite-dimensional} $k$-vector spaces.

\begin{corollary}[3.2.4]
\label{III.3.2.4}
Let $Y$ be a locally Noetherian prescheme, $f:X\to Y$ a morphism of finite type.
For every coherent $\sh{O}_X$-module $\sh{F}$ whose support in proper over $Y$~\sref[II]{II.5.4.10}, the $\sh{O}_Y$-modules $\RR^q f_*(\sh{F})$ are coherent.
\end{corollary}

\oldpage[III]{118}
\begin{proof}
Since the questions is local on $Y$, we can suppose $Y$ Noetherian, and it is the same for $X$~\sref[I]{I.6.3.7}.
By hypothesis, every closed subprescheme $Z$ of $X$ whose underlying space is $\Supp(\sh{F})$ is proper over $Y$, in other words, if $j:Z\to X$ is the canonical injection, then $f\circ j:Z\to Y$ is proper.
We can suppose that $Z$ is such that $\sh{F}=j_*(\sh{G})$, where $\sh{G}=j^*(\sh{F})$ is a coherent $\sh{O}_Z$-module~\sref[I]{I.9.3.5}; as we have $\RR^q f_*(\sh{F})=\RR^q(f\circ j)_*(\sh{G})$ by Corollary~\sref{III.1.3.4}, the conclusion follows immediately from Theorem~\sref{III.3.2.1}.
\end{proof}

\subsection{Generalization of the finiteness theorem (usual schemes)}
\label{subsection:III.3.3}

\begin{proposition}[3.3.1]
\label{III.3.3.1}
Let $Y$ be a Noetherian prescheme, $\sh{S}$ a quasi-coherent $\sh{O}_Y$-algebra of finite type, graded in positive degrees, $Y'=\Proj(\sh{S})$, and $g:Y'\to Y$ the structure morphism.
Let $f:X\to Y$ be a proper morphism, $\sh{S}'=f^*(\sh{S})$, $\sh{M}=\bigoplus_{k\in\bb{Z}}\sh{M}_k$ a quasi-coherent graded $\sh{S}'$-module of finite type.
Then the $\RR^p f_*(\sh{M})=\bigoplus_{k\in\bb{Z}}\RR^p f_*(\sh{M}_k)$ are graded $\sh{S}$-modules of finite type for all $p$.
Suppose in addition that the $\sh{S}$ are generated by $\sh{S}_1$; then, for every $p\in\bb{Z}$, there exists an integer $k_p$ such that for all $k\geq k_p$ and all $r>0$, we have
\[
\label{III.3.3.1.1}
  \RR^p f_*(\sh{M}_{k+r})=\sh{S}_r\RR^p f_*(\sh{M}_k).
  \tag{3.3.1.1}
\] 
\end{proposition}

\begin{proof}
The first assertion is identical to the statement of Theorem~\sref{III.2.4.1}[i], where we have simply replaced ``projective morphism'' by ``proper morphism''.
In the proof of Theorem~\sref{III.2.4.1}[i], the hypothesis on $f$ was only used to show (with the notation of this proof) that $\RR^p f_*'(\widetilde{\sh{M}})$ is a coherent $\sh{O}_{Y'}$-module.
With the hypothesis of Proposition~\sref{III.3.3.1}, $f'$ is proper~\sref[II]{II.5.4.2}[iii], so we can resume without change in the proof of Theorem~\sref{III.2.4.1}[i], thanks to the finiteness theorem~\sref{III.3.2.1}.

As for the second assertion, it suffices to remark that there is a finite affine open cover $(U_i)$ of $Y$ such that the restrictions to the $U_i$ of the two sides of~(3.3.1.1) are equal for all $k\geq k_{p,i}$~\sref[II]{II.2.1.6}[ii]; it suffices to take for $k_p$ the largest of the $k_{p,i}$.
\end{proof}

\begin{corollary}[3.3.2]
\label{III.3.3.2}
Let $A$ be a Noetherian ring, $\mathfrak{m}$ an ideal of $A$, $X$ a proper $A$-scheme, and $\sh{F}$ a coherent $\sh{O}_X$-module.
Then, for all $p\geq 0$, the direct sum $\bigoplus_{k\geq 0}\HH^p(X,\mathfrak{m}^k\sh{F})$ is a module of finite type over the ring $S=\bigoplus_{k\geq 0}\mathfrak{m}^k$; in particular, there exists an integer $k_p\geq 0$ such that for all $k\geq k_p$ and all $r>0$, we have
\[
\label{III.3.3.2.1}
  \HH^p(X,\mathfrak{m}^{k+r}\sh{F})=\mathfrak{m}^r\HH^p(X,\mathfrak{m}^k\sh{F}).
  \tag{3.3.2.1}
\]
\end{corollary}

\begin{proof}
It suffices to apply Proposition~\sref{III.3.3.1} with $Y=\Spec(A)$, $\sh{S}=\widetilde{S}$, $\sh{M}_k=\mathfrak{m}^k\sh{F}$, taking into account Corollary~\sref{III.1.4.11}.
\end{proof}

It should be remembered that the $S$-module structure on $\bigoplus_{k\geq 0}\HH^p(X,\mathfrak{m}^k\sh{F})$ is obtained by considering, for every $a\in\mathfrak{m}^r$, the map $\HH^p(X,\mathfrak{m}^k\sh{F})\to\HH^p(X,\mathfrak{m}^{k+r}\sh{F})$, which comes from the passage to cohomology of the multiplication map $\mathfrak{m}^r\sh{F}\to\mathfrak{m}^{k+r}\sh{F}$ defined by $a$~\sref{III.2.4.1}.

\subsection{Finiteness theorem: the case of formal schemes}
\label{subsection:III.3.4}

\oldpage[III]{119}
The results of this section (except the definition~\sref{III.3.4.1}) will not be used in the rest of this chapter.

\begin{env}[3.4.1]
\label{III.3.4.1}
Let $\mathfrak{X}$ and $\mathfrak{S}$ be two locally Noetherian formal preschemes~\sref[I]{I.10.4.2}, $f:\mathfrak{X}\to\mathfrak{S}$ a morphism of formal preschemes.
We say that $f$ is a \emph{proper} morphism if it satisfies the following conditions:
\begin{enumerate}
  \item[1st.] \emph{$f$ is a morphism of finite type~\sref[I]{I.10.13.3}}.
  \item[2nd.] \emph{If $\sh{K}$ is a sheaf of ideals of definition for $\mathfrak{S}$ and if we set $\sh{J}=f^*(\sh{K})\sh{O}_\mathfrak{X}$, $X_0=(\mathfrak{X},\sh{O}_\mathfrak{X}/\sh{J})$, $S_0=(\mathfrak{S},\sh{O}_\mathfrak{S}/\sh{K})$, then the morphism $f_0:X_0\to S_0$ induced by $f$~\sref[I]{I.10.5.6} is proper}.
\end{enumerate}
It is immediate that this definition does not depend on the sheaf of ideals of definition $\sh{K}$ for $\sh{S}$ considered; indeed, if $\sh{K}'$ is a second sheaf of ideals of definition such that $\sh{K}'\subset\sh{K}$, and if we set $\sh{J}'=f^*(\sh{K}')\sh{O}_\mathfrak{X}$, $X_0'=(\mathfrak{X},\sh{O}_\mathfrak{X}/\sh{J}')$, $S_0'=(\sh{S},\sh{O}_\mathfrak{S}/\sh{K}')$, then the morphism $f_0':X_0'\to S_0'$ induced by $f$ is such that the diagram
\[
  \xymatrix{
    X_0\ar[r]^{f_0}\ar[d]_i &
    S_0\ar[d]^j\\
    X_0'\ar[r]^{f_0'} &
    S_0'
  }
\]
is commutative, $i$ and $j$ being surjective immersions; it is equivalent to say that $f_0$ or $f_0'$ is proper, by virtue of~\sref[II]{II.5.4.5}.

We note that, for all $n\geq 0$, if we set $X_n=(\mathfrak{X},\sh{O}_\mathfrak{X}/\sh{J}^{n+1})$, $S_n=(\mathfrak{S},\sh{O}_\mathfrak{S}/\sh{K}^{n+1})$, then the morphism $f_n:X_n\to S_n$ induced by $f$~\sref[I]{I.10.5.6} is proper for all $n$ whenever it is for $n=0$~\sref[II]{II.5.4.6}.

If $g:Y\to Z$ is a proper morphism of locally Noetherian usual preschemes, $Z'$ a closed subset of $Z$, $Y'$ a closed subset of $Y$ such that $g(Y')\subset Z'$, then the extension $\widehat{g}:Y_{/Y'}\to Z_{/Z'}$ of $g$ to the completions~\sref[I]{I.10.9.1} is a proper morphism of formal preschemes, as it follows from the definition and from~\sref[II]{II.5.4.5}.

Let $\mathfrak{X}$ and $\mathfrak{S}$ be two locally Noetherian formal preschemes, $f:\mathfrak{X}\to\mathfrak{S}$ a morphism \emph{of finite type}~\sref[I]{I.10.13.3}; the notation being the same as above, we say that a subset $Z$ of the underlying space of $\mathfrak{X}$ is \emph{proper} over $\mathfrak{S}$ (or proper for $f$) if, considered as a subset of $X_0$, $Z$ is \emph{proper over $S_0$}~\sref[II]{II.5.4.10}.
All the properties of proper subsets of usual preschemes stated in~\sref[II]{II.5.4.10} are still true for the proper subsets of formal preschemes, as it follows immediately from the definitions.
\end{env}

\begin{theorem}[3.4.2]
\label{III.3.4.2}
Let $\mathfrak{X}$ and $\mathfrak{Y}$ be locally Noetherian formal preschemes, $f:\mathfrak{X}\to\mathfrak{Y}$ a proper morphism.
For every coherent $\sh{O}_\mathfrak{X}$-module $\sh{F}$, the $\sh{O}_\mathfrak{Y}$-modules $\RR^q f_*(\sh{F})$ are coherent for all $q\geq 0$.
\end{theorem}

Let $\sh{J}$ be a sheaf of ideals of definition for $\mathfrak{Y}$, $\sh{K}=f^*(\sh{J})\sh{O}_\mathfrak{X}$, and consider the $\sh{O}_\mathfrak{X}$-modules
\[
\label{III.3.4.2.1}
  \sh{F}_k=\sh{F}\otimes_{\sh{O}_\mathfrak{Y}}(\sh{O}_\mathfrak{Y}/\sh{J}^{k+1})=\sh{F}/\sh{K}^{k+1}\sh{F}\quad(k\geq 0)
  \tag{3.4.2.1}
\]
which evidently form a \emph{projective system} of topological $\sh{O}_\mathfrak{X}$-modules, such that
\oldpage[III]{120}
$\sh{F}=\varprojlim_k\sh{F}_k$~\sref[I]{I.10.11.3}.
On the other hand, it follows from Theorem~\sref{III.3.4.2} that each of the $\RR^q f_*(\sh{F})$, being coherent, is naturally equipped with a topological $\sh{O}_\mathfrak{Y}$-module structure~\sref[I]{I.10.11.6}, and so are the $\RR^q f_*(\sh{F}_k)$.
The canonical homomorphisms $\sh{F}\to\sh{F}_k=\sh{F}/\sh{K}^{k+1}\sh{F}$ canonically correspond to homomorphisms
\[
  \RR^q f_*(\sh{F})\to\RR^q f_*(\sh{F}_k),
\]
which are necessarily continuous for the topological $\sh{O}_\mathfrak{Y}$-module structures above~\sref[I]{I.10.11.6}, and form a projective system, giving the limit a canonical functorial homomorphism
\[
\label{III.3.4.2.2}
  \RR^q f_*(\sh{F})\to\varprojlim_k\RR^q f_*(\sh{F}_k),
  \tag{3.4.2.2}
\]
which will be a continuous homomorphism of topological $\sh{O}_\mathfrak{Y}$-modules.
We will prove along with Theorem~\sref{III.3.4.2} the
\begin{corollary}[3.4.3]
\label{III.3.4.3}
Each of the homomorphisms~(3.4.2.2) is a topological isomorphism.
In addition, if $\mathfrak{Y}$ is Noetherian, then the projective system $(\RR^q f_*(\sh{F}/\sh{K}^{k+1}\sh{F}))_{k\geq 0}$ satisfies the \emph{(ML)}-condition~\sref[0]{0.13.1.1}.
\end{corollary}
We will begin by establishing Theorem~\sref{III.3.4.2} and Corollary~\sref{III.3.4.3} when $Y$ is a Noetherian formal affine scheme~\sref[I]{I.10.4.1}:
\begin{corollary}[3.4.4]
\label{III.3.4.4}
Under the hypotheses of Theorem~\sref{III.3.4.2}, suppose in addition that $\mathfrak{Y}=\Spf(A)$, where $A$ is an adic Noetherian ring.
Let $\mathfrak{J}$ be an ideal of definition for $A$, and set $\sh{F}_k=\sh{F}/\mathfrak{J}^{k+1}\sh{F}$ for $k\geq 0$.
Then the $\HH^n(\mathfrak{X},\sh{F})$ are $A$-modules of finite type; the projective system $(\HH^n(\mathfrak{X},\sh{F}_k))_{k\geq 0}$ satisfies the \emph{(ML)}-condition for all $n$; if we set
\[
\label{III.3.4.4.1}
  N_{n,k}=\Ker\big(\HH^n(\mathfrak{X},\sh{F})\to\HH^n(\mathfrak{X},\sh{F}_k)\big)
  \tag{3.4.4.1}
\]
(also equal to $\Im(\HH^n(\mathfrak{X},\mathfrak{J}^{k+1}\sh{F})\to\HH^n(\mathfrak{X},\sh{F}))$ by the exact sequence in cohomology), then the $N_{n,k}$ define on $\HH^n(\mathfrak{X},\sh{F})$ a $\mathfrak{J}$-good filtration~\sref[0]{0.13.7.7}; finally, the canonical homomorphism
\[
\label{III.3.4.4.2}
  \HH^n(\mathfrak{X},\sh{F})\to\varprojlim_k\HH^n(\mathfrak{X},\sh{F}_k)
  \tag{3.4.4.2}
\]
is a topological isomorphism for all $n$ (the left hand side being equipped with the $\mathfrak{J}$-adic topology, the $\HH^n(\mathfrak{X},\sh{F}_k)$ with the discrete topology).
\end{corollary}

Set
\[
\label{III.3.4.4.3}
  S=\gr(A)=\bigoplus_{k\geq 0}\mathfrak{J}^k/\mathfrak{J}^{k+1},\ \sh{M}=\gr(\sh{F})=\bigoplus_{k\geq 0}\mathfrak{J}^k\sh{F}/\mathfrak{J}^{k+1}\sh{F}.
\]
We know that $\mathfrak{J}^\Delta$ is a sheaf of ideals of definition for $\mathfrak{Y}$~\sref[I]{I.10.3.1}; let $\sh{K}=f^*(\mathfrak{J}^\Delta)\sh{O}_\mathfrak{X}$, $X_0=(\mathfrak{X},\sh{O}_\mathfrak{X}/\sh{K})$, $Y_0=(\mathfrak{Y},\sh{O}_\mathfrak{Y}/\mathfrak{J}^\Delta)=\Spec(A_0)$, with $A_0=A/\mathfrak{J}$.
It is clear that the $\sh{M}_k=\mathfrak{J}^k\sh{F}/\mathfrak{J}^{k+1}\sh{F}$ are coherent $\sh{O}_{X_0}$-modules~\sref[I]{I.10.11.3}.
Consider on the other hand the quasi-coherent graded $\sh{O}_{X_0}$-algebra
\[
\label{III.3.4.4.4}
  \sh{S}=\sh{O}_{X_0}\otimes_{A_0}S=\gr(\sh{O}_\mathfrak{X})=\bigoplus_{k\geq 0}\sh{K}^k/\sh{K}^{k+1}.
  \tag{3.4.4.4}
\]

The hypothesis that $\sh{F}$ is a $\sh{O}_\mathfrak{X}$-module of finite type implies first that $\sh{M}$ is
\oldpage[III]{121}
a graded $\sh{S}$-module \emph{of finite type}.
Indeed, the question is local on $\mathfrak{X}$, and we can thus suppose that $\mathfrak{X}=\Spf(B)$, where $B$ is an adic Noetherian ring, and $\sh{F}=N^\Delta$, where $N$ is a $B$-module of finite type~\sref[I]{I.10.10.5}; we have in addition $X_0=\Spec(B_0)$, where $B_0=B/\mathfrak{J}B$, and the quasi-coherent $\sh{O}_{X_0}$-modules $\sh{S}$ and $\sh{M}$ are respectively equal to $\widetilde{S'}$ and $\widetilde{M'}$, where $S'=\bigoplus_{k\geq 0}((\mathfrak{J}^k/\mathfrak{J}^{k+1})\otimes_{A_0}B_0)$ and $M'=\bigoplus_{k\geq 0}((\mathfrak{J}^k/\mathfrak{J}^{k+1})\otimes_{A_0}N_0)$, with $N_0=N/\mathfrak{J}N$; we then evidently have $M'=S'\otimes_{B_0}N_0$, and as $N_0$ is a $B_0$-module of finite type, $M'$ is a $S'$-module of finite type, hence our assertion~\sref[I]{I.1.3.13}.

As the morphism $f_0:X_0\to Y_0$ is \emph{proper} by hypothesis, we can apply Corollary~\sref{III.3.3.2} to $\sh{S}$, $\sh{M}$, and the morphism $f_0$: taking into account Corollary~\sref{III.1.4.11}, we conclude that for \emph{all $n\geq 0$}, $\bigoplus_{k\geq 0}\HH^n(X_0,\sh{M}_k)$ is a graded $S$-module \emph{of finite type}.
This proves that the condition ($\text{F}_n$) of~\sref[0]{0.13.7.7} is satisfied for \emph{all $n\geq 0$}, when we consider the strictly projective system $(\sh{F}/\mathfrak{J}^k\sh{F})_{k\geq 0}$ of sheaves of abelian groups on $X_0$, each equipped with its natural ``filtered $A$-module'' structure.
We can thus apply~\sref[0]{0.13.7.7}, which proves that:
\begin{enumerate}
  \item[1st.] The projective system $(\HH^n(\mathfrak{X},\sh{F}_k))_{k\geq 0}$ satisfies the (ML)-condition.
  \item[2nd.] If $\HH^{\prime n}=\varprojlim_k\HH^n(\mathfrak{X},\sh{F}_k)$, then $\HH^{\prime n}$ is an $A$-module of finite type.
  \item[3rd.] The filtration defined on $\HH^{\prime n}$ by the kernels of the canonical homomorphisms $\HH^{\prime n}\to\HH^n(\mathfrak{X},\sh{F}_k)$ is $\mathfrak{J}$-good.
\end{enumerate}

Note that on the other hand, if we set $X_k=(\mathfrak{X},\sh{O}_\mathfrak{X}/\sh{K}^{k+1})$, then $\sh{F}_k$ is a coherent $\sh{O}_{X_k}$-module~\sref[I]{I.10.11.3}, and if $U$ is an affine open set in $X_0$, then $U$ is also an affine open set in each of the $X_k$~\sref[I]{I.5.1.9}, so $\HH^n(U,\sh{F}_k)=0$ for all $n>0$ and all $k$~\sref{III.1.3.1} and $\HH^0(U,\sh{F}_k)\to\HH^0(U,\sh{F}_h)$ is surjective for $h\leq k$~\sref[I]{I.1.3.9}.
We are thus in the conditions of~\sref[0]{0.13.3.2} and applying~\sref[0]{0.13.3.1} proves that $\HH^{\prime n}$ canonically identifies with $\HH^n(\mathfrak{X},\varprojlim_k\sh{F}_k)=\HH^n(\mathfrak{X},\sh{F})$; this finishes the proof of Corollary~\sref{III.3.4.4}.

\begin{env}[3.4.5]
\label{III.3.4.5}
We return to the proof of~\sref{III.3.4.2} and~\sref{III.3.4.3}.
We first prove the propositions for the case $\mathfrak{Y}=\Spf(A)$ envisaged in~\sref{III.3.4.4}; for this, for all $g\in A$, apply~\sref{III.3.4.4} to the Noetherian affine formal scheme induced on the open set $\mathfrak{Y}_g=\mathfrak{D}(g)$ of $\mathfrak{Y}$, which is equal to $\Spf(A_{\{g\}})$, and to the formal prescheme induced by $\mathfrak{X}$ on $f^{-1}(\mathfrak{Y}_g)$; note that $\mathfrak{Y}_g$ is also an affine open set in the prescheme $Y_k=(\mathfrak{Y},\sh{O}_\mathfrak{Y}/(\mathfrak{J}^\Delta)^{k+1})$, and as $\sh{F}_k$ is a coherent $\sh{O}_{X_k}$-module, we have
\[
  \HH^n(f^{-1}(\mathfrak{Y}_g),\sh{F}_k)=\Gamma(\mathfrak{Y}_g,\RR^n f_*(\sh{F}_k))
\]
for all $k\geq 0$ by virtue of Corollary~\sref{III.1.4.11}.
The canonical homomorphism
\[
  \HH^n(f^{-1}(\mathfrak{Y}_g),\sh{F})\to\varprojlim_k\Gamma(\mathfrak{Y}_g,\RR^n f_*(\sh{F}_k))
\]
is an isomorphism; but we have~\sref[0]{0.3.2.6}
\[
  \varprojlim_k\Gamma(\mathfrak{Y}_g,\RR^n f_*(\sh{F}_k))=\Gamma(\mathfrak{Y}_g,\varprojlim_k\RR^n f_*(\sh{F}_k)),
\]
\oldpage[III]{122}
and as the sheaf $\RR^n f_*(\sh{F})$ is the sheaf associated to the presheaf $\mathfrak{Y}_g\mapsto\HH^n(f^{-1}(\mathfrak{Y}_g),\sh{F})$ on the $\mathfrak{Y}_g$~\sref[0]{0.3.2.1}, we have shown that the homomorphism~(3.4.2.2) is \emph{bijective}.
Let us now prove that $\RR^n f_*(\sh{F})$ is a coherent $\sh{O}_\mathfrak{Y}$-module, and more precisely that we have
\[
\label{III.3.4.5.1}
  \RR^n f_*(\sh{F})=\big(\HH^n(\mathfrak{X},\sh{F})\big)^\Delta.
  \tag{3.4.5.1}
\]

With the above notation, we have, since $\sh{F}_k$ is a coherent $\sh{O}_{X_k}$-module~\sref{III.1.4.13},
\[
  \Gamma(\mathfrak{Y}_g,\RR^n f_*(\sh{F}_k))=(\Gamma(\mathfrak{Y},\RR^n f_*(\sh{F}_k)))_g=(\HH^n(\mathfrak{X},\sh{F}_k))_g.
\]

Now the $\HH^n(\mathfrak{X},\sh{F}_k)$ form a projective system satisfying (ML), and their projective limit $\HH^n(\mathfrak{X},\sh{F})$ is an $A$-module of finite type.
We conclude~\sref[0]{0.13.7.8} that we have
\[
  \varprojlim_k\big((\HH^n(\mathfrak{X},\sh{F}_k))_g\big)=\HH^n(\mathfrak{X},\sh{F})\otimes_A A_{\{g\}}=\Gamma(\mathfrak{Y}_g,(\HH^n(\mathfrak{X},\sh{F}))^\Delta),
\]
taking into account~\sref[I]{I.10.10.8} applied to $A$ and $A_{\{g\}}$; this proves~(3.4.5.1) since $\Gamma(\mathfrak{Y}_g,\RR^n f_*(\sh{F}))=\varprojlim_k\Gamma(\mathfrak{Y}_g,\RR^n f_*(\sh{F}_k))$.

As (3.4.2.2) is then an isomorphism of coherent $\sh{O}_\mathfrak{Y}$-modules, it is necessarily a \emph{topological} isomorphism~\sref[I]{I.10.11.6}.
Finally, it follows from the relations $\RR^n f_*(\sh{F}_k)=(\HH^n(\mathfrak{X},\sh{F}_k))^\Delta$ that the projective system $(\RR^n f_*(\sh{F}_k))_{k\geq 0}$ satisfies (ML)~\sref[I]{I.10.10.2}.

Once \sref{III.3.4.2} and \sref{III.3.4.3} are proved in the case where the formal prescheme $\mathfrak{Y}$ is affine Noetherian, it is immediate to pass to the general case for~\sref{III.3.4.2} and the first assertion of~\sref{III.3.4.3}, which are local on $\mathfrak{Y}$.
As for the second assertion of~\sref{III.3.4.3}, it suffices, $\mathfrak{Y}$ being Noetherian, to cover it by a finite number of Noetherian affine open sets $U_i$ and to note that the restrictions of the projective system $(\RR^q f_*(\sh{F}_k))$ to each of the $U_i$ satisfies (ML).
\end{env}

Along the way, we have in addition proved:
\begin{corollary}[3.4.6]
\label{III.3.4.6}
Under the hypotheses of Corollary~\sref{III.3.4.4}, the canonical homomorphism
\[
\label{III.3.4.6.1}
  \HH^q(\mathfrak{X},\sh{F})\to\Gamma(\mathfrak{Y},\RR^q f_*(\sh{F}))
  \tag{3.4.6.1}
\]
is bijective.
\end{corollary}


\input{ega3/ega3-4}
\input{ega3/ega3-5}
\input{ega3/ega3-6}
\input{ega3/ega3-7}

\bibliography{the}
\bibliographystyle{amsalpha}

\end{document}


\documentclass[oneside]{amsart}

\usepackage[all]{xy}
\usepackage[T1]{fontenc}
\usepackage{xstring}
\usepackage{xparse}
\usepackage{xr-hyper}
\usepackage{xcolor}
\definecolor{brightmaroon}{rgb}{0.76, 0.13, 0.28}
\usepackage[linktocpage=true,colorlinks=true,hyperindex,citecolor=blue,linkcolor=brightmaroon]{hyperref}
\usepackage[left=1.25in,right=1.25in,top=0.75in,bottom=0.75in]{geometry}
%\usepackage[charter,greekfamily=didot]{mathdesign}
%\usepackage{Baskervaldx}
\usepackage{amssymb}
\usepackage{mathrsfs}
\usepackage{mathpazo}
\linespread{1.05}

\usepackage[nobottomtitles]{titlesec}
\usepackage{marginnote}
\usepackage{enumerate}
\usepackage{longtable}
\usepackage{aurical}
\usepackage{microtype}

\externaldocument[what-]{what}
\externaldocument[intro-]{intro}
\externaldocument[ega0-]{ega0}
\externaldocument[ega1-]{ega1}
\externaldocument[ega2-]{ega2}
\externaldocument[ega3-]{ega3}
\externaldocument[ega4-]{ega4}

\newtheoremstyle{ega-env-style}%
  {}{}{\rmfamily}{}{\bfseries}{.}{ }{\thmnote{(#3)}}%

\newtheoremstyle{ega-thm-env-style}%
  {}{}{\itshape}{}{\bfseries}{. --- }{ }{\thmname{#1}\thmnote{ (#3)}}%

\newtheoremstyle{ega-defn-env-style}%
  {}{}{\rmfamily}{}{\bfseries}{. --- }{ }{\thmname{#1}\thmnote{ (#3)}}%

\theoremstyle{ega-env-style}
\newtheorem*{env}{---}

\theoremstyle{ega-thm-env-style}
\newtheorem*{theorem}{Theorem}
\newtheorem*{proposition}{Proposition}
\newtheorem*{lemma}{Lemma}
\newtheorem*{corollary}{Corollary}

\theoremstyle{ega-defn-env-style}
\newtheorem*{definition}{Definition}
\newtheorem*{example}{Example}
\newtheorem*{examples}{Examples}
\newtheorem*{remark}{Remark}
\newtheorem*{remarks}{Remarks}
\newtheorem*{notation}{Notation}

% indent subsections, see https://tex.stackexchange.com/questions/177290/.
% also make section titles bigger.
% also add § to \thesection, https://tex.stackexchange.com/questions/119667/ and https://tex.stackexchange.com/questions/308737/.
\makeatletter
\def\l@subsection{\@tocline{2}{0pt}{2.5pc}{2.2pc}{}}
\def\section{\@startsection{section}{1}%
  \z@{.7\linespacing\@plus\linespacing}{.5\linespacing}%
  {\normalfont\bfseries\Large\scshape\centering}}
\renewcommand{\@seccntformat}[1]{%
  \ifnum\pdfstrcmp{#1}{section}=0\textsection\fi%
  \csname the#1\endcsname.~}
\makeatother

%\allowdisplaybreaks[1]
%\binoppenalty=9999
%\relpenalty=9999

% for Chapter 0, Chapter I, etc.
% credit for ZeroRoman https://tex.stackexchange.com/questions/211414/
% added into scripts/make_book.py
%\newcommand{\ZeroRoman}[1]{\ifcase\value{#1}\relax 0\else\Roman{#1}\fi}
%\renewcommand{\thechapter}{\ZeroRoman{chapter}}

\def\mathcal{\mathscr}
\def\sh{\mathcal}                   % sheaf font
\def\bb{\mathbf}                    % bold font
\def\cat{\mathtt}                   % category font
\def\leq{\leqslant}                 % <=
\def\geq{\geqslant}                 % >=
\def\setmin{-}                      % set minus
\def\rad{\mathfrak{r}}              % radical
\def\nilrad{\mathfrak{N}}           % nilradical
\def\emp{\varnothing}               % empty set
\def\vphi{\phi}                     % for switching \phi and \varphi, change if needed
\def\HH{\mathrm{H}}                 % cohomology H
\def\CHH{\check{\HH}}               % Čech cohomology H
\def\RR{\mathrm{R}}                 % right derived R
\def\LL{\mathrm{L}}                 % left derived L
\def\dual#1{{#1}^\vee}              % dual
\def\kres{k}                        % residue field k
\def\C{\cat{C}}                     % category C
\def\op{^\cat{op}}                  % opposite category
\def\Set{\cat{Set}}                 % category of sets
\def\CHom{\cat{Hom}}                % functor category
\def\supertilde{{\,\widetilde{\,}\,}}   % use \supertilde instead of ^\sim
\def\GL{\bb{GL}}
\def\red{\mathrm{red}}
\def\rg{\operatorname{rg}}
\def\gr{\operatorname{gr}}
\def\Gr{\operatorname{Gr}}
\def\Hom{\operatorname{Hom}}
\def\Proj{\operatorname{Proj}}
\def\Tor{\operatorname{Tor}}
\def\Ext{\operatorname{Ext}}
\def\Supp{\operatorname{Supp}}
\def\Ker{\operatorname{Ker}}
\def\Im{\operatorname{Im}}
\def\Coker{\operatorname{Coker}}
\def\Spec{\operatorname{Spec}}
\def\Spf{\operatorname{Spf}}
\def\grad{\operatorname{grad}}
\def\dimc{\operatorname{dimc}}
\def\codim{\operatorname{codim}}
\def\id{\operatorname{id}}
\def\Der{\operatorname{Der}}
\def\Diff{\operatorname{Diff}}
\def\Hyp{\operatorname{Hyp}}

\renewcommand{\to}{\mathchoice{\longrightarrow}{\rightarrow}{\rightarrow}{\rightarrow}}
\newcommand{\from}{\mathchoice{\longleftarrow}{\leftarrow}{\leftarrow}{\leftarrow}}
\let\mapstoo\mapsto
\renewcommand{\mapsto}{\mathchoice{\longmapsto}{\mapstoo}{\mapstoo}{\mapstoo}}
\def\isoto{\simeq} % isomorphism

\renewcommand{\hat}[1]{\widehat{#1}}
\renewcommand{\tilde}[1]{\widetilde{#1}}


\def\shHom{\sh{H}\textup{\kern-2.2pt{\Fontauri\slshape om}}}   % sheaf Hom
\def\shProj{\sh{P}\textup{\kern-2.2pt{\Fontauri\slshape roj}}} % sheaf Proj
\def\shExt{\sh{E}\textup{\kern-2.2pt{\Fontauri\slshape xt}}}   % sheaf Ext
\def\shGr{\sh{G}\textup{\kern-2.2pt{\Fontauri\slshape r}}}     % sheaf Gr
\def\shDer{\sh{D}\,\textup{\kern-2.2pt{\Fontauri\slshape er}}} % sheaf Der
\def\shDiff{\sh{D}\,\textup{\kern-2.2pt{\Fontauri\slshape if{}f}}\,} % sheaf Diff
\def\shHomcont{\sh{H}\textup{\kern-2.2pt{\Fontauri\slshape om.\,cont}}}   % sheaf Hom.cont
\def\shAut{\sh{A}\textup{\kern-2.2pt{\Fontauri\slshape ut}}}   % sheaf Aut

% if unsure of a translation
%\newcommand{\unsure}[2][]{\hl{#2}\marginpar{#1}}
%\newcommand{\completelyunsure}{\unsure{[\ldots]}}
\def\unsure#1{#1 {\color{red}(?)}}
\def\completelyunsure{{\color{red}(???)}}

% use to mark where original page starts
\newcommand{\oldpage}[2][]{{\marginnote{\normalfont{\textbf{#1}~|~#2}}}\ignorespaces}
\def\sectionbreak{\begin{center}***\end{center}}

% for referencing environments.
% use as \sref{chapter-number.x.y.z}, with optional args
% for volume and indices, e.g. \sref[volume]{chapter-number.x.y.z}[i].
\NewDocumentCommand{\sref}{o m o}{%
  \IfNoValueTF{#1}%
    {\IfNoValueTF{#3}%
      {\hyperref[#2]{\normalfont{(\StrBehind{#2}{.})}}}%
      {\hyperref[#2]{\normalfont{(\StrBehind{#2}{.},~{#3})}}}}%
    {\IfNoValueTF{#3}%
      {\hyperref[#2]{\normalfont{(\textbf{#1},~\StrBehind{#2}{.})}}}%
      {\hyperref[#2]{\normalfont{(\textbf{#1},~\StrBehind{#2}{.},~{#3})}}}}%
}

% for marking changes following errata
% use as \erratum[volume]{correction} to say where the erratum is given
\newcommand{\erratum}[2][]{{#2}\marginpar{\footnotesize\textbf{Err}\textsubscript{#1}}}

% for referencing equations, use as \eref{eq:chapter-number.x.y.z}.
\newcommand{\eref}[1]{\hyperref[#1]{\normalfont{(\StrBehind{#1}{.})}}}



\begin{document}
\title{Elementary global study of some classes of morphisms (EGA II)}
\maketitle

\phantomsection
\label{section:ega2}

\tableofcontents

\section*{Summary}

\begin{longtable}{ll}
  \hyperref[section:II.1]{\textsection1}. & Affine morphisms.\\
  \hyperref[section:II.2]{\textsection2}. & Homogeneous prime spectra.\\
  \hyperref[section:II.3]{\textsection3}. & Homogeneous prime spectrum of a sheaf of graded algebras.\\
  \hyperref[section:II.4]{\textsection4}. & Projective bundles; ample sheaves.\\
  \hyperref[section:II.5]{\textsection5}. & Quasi-affine morphisms; quasi-projective morphisms; proper morphisms; projective morphisms.\\
  \hyperref[section:II.6]{\textsection6}. & Integral morphisms and finite morphisms.\\
  \hyperref[section:II.7]{\textsection7}. & Valuative criteria.\\
  \hyperref[section:II.8]{\textsection8}. & Blowup schemes; projective cones; projective closure.\\
\end{longtable}
\bigskip

\oldpage[II]{5}
The various classes of morphisms studied in this chapter are used extensively in cohomological methods; further study using these methods will be done in Chapter~III, where we make particular use of \textsection\textsection2, 4, and 5 of Chapter~II.
On a first reading, \textsection8 can be omitted: it supplements the formalism developed in \textsection\textsection1 and 3, reducing to easy applications of this formalism, and we will use it less consistently than the other results of this chapter.
\bigskip

\section{Affine morphisms}
\label{section:II.1}

\subsection{$S$-preschemes and $\mathcal{O}_S$-algebras}
\label{subsection:II.1.1}

\begin{env}[1.1.1]
\label{II.1.1.1}
Let $S$ be a prescheme, $X$ an $S$-prescheme, and $f:X\to S$ its structure morphism.
We know \sref[0]{0.4.2.4} that the direct image $f_*(\sh{O}_X)$ is an $\sh{O}_S$-algebra, which we
\oldpage[II]{6}
denote $\sh{A}(X)$ when there is little chance of confusion; if $U$ is an open subset of $S$, then we have
\[
  \sh{A}(f^{-1}(U))=\sh{A}(X)|U.
\]
Similarly, for every $\sh{O}_X$-module $\sh{F}$ (resp. every $\sh{O}_X$-algebra $\sh{B}$), we write $\sh{A}(\sh{F})$ (resp. $\sh{A}(\sh{B})$) for the direct image $f_*(\sh{F})$ (resp. $f_*(\sh{B})$) which is an $\sh{A}(X)$-module (resp. an $\sh{A}(X)$-algebra) and not only an $\sh{O}_S$-module (resp. an $\sh{O}_S$-algebra).
\end{env}

\begin{env}[1.1.2]
\label{II.1.1.2}
Let $Y$ be a second $S$-prescheme, $g:Y\to S$ its structure morphism, and $h:X\to Y$ an $S$-morphism; we then have the commutative diagram
\[
\label{II.1.1.2.1}
  \xymatrix{
    X\ar[rr]^h\ar[rd]_f & &
    Y\ar[ld]^g\\
    & S.
  }
  \tag{1.1.2.1}
\]

We have by definition $h=(\psi,\theta)$, where $\theta:\sh{O}_Y\to h_*(\sh{O}_X)=\psi_*(\sh{O}_X)$ is a homomorphism of sheaves of rings; we thus obtain \sref[0]{0.4.2.2} a homomorphism of $\sh{O}_S$-algebras $g_*(\theta):g_*(\sh{O}_Y)\to g_*(h_*(\sh{O}_X))=f_*(\sh{O}_X)$, in other words, a homomorphism of $\sh{O}_S$-algebras $\sh{A}(Y)\to\sh{A}(X)$, which we denote by $\sh{A}(h)$.
If $h':Y\to Z$ is a second $S$-morphism, then it is immediate that $\sh{A}(h'\circ h)=\sh{A}(h)\circ\sh{A}(h')$.
We have thus defined a \emph{contravariant functor $\sh{A}(X)$} from the category of $S$-preschemes to the category of $\sh{O}_S$-algebras.

Now let $\sh{F}$ be an $\sh{O}_X$-module, $\sh{G}$ an $\sh{O}_Y$-module, and $u:\sh{G}\to\sh{F}$ an $h$-morphism, that is \sref[0]{0.4.4.1} a homomorphism of $\sh{O}_Y$-modules $\sh{G}\to h_*(\sh{F})$.
Then $g_*(u):g_*(\sh{G})\to g_*(h_*(\sh{F}))=f_*(\sh{F})$ is a homomorphism $\sh{A}(\sh{G})\to\sh{A}(\sh{F})$ of $\sh{O}_S$-modules, which we denote by $\sh{A}(u)$; in addition, the pair $(\sh{A}(h),\sh{A}(u))$ form a \emph{di-homomorphism} from the $\sh{A}(Y)$-module $\sh{A}(\sh{G})$ to the $\sh{A}(X)$-module $\sh{A}(\sh{F})$.
\end{env}

\begin{env}[1.1.3]
\label{II.1.1.3}
If we fix the prescheme $S$, then we can consider the pairs $(X,\sh{F})$, where $X$ is an $S$-prescheme and $\sh{F}$ is an $\sh{O}_X$-module, as forming a \emph{category}, by defining a \emph{morphism} $(X,\sh{F})\to(Y,\sh{G})$ as a pair $(h,u)$, where $h:X\to Y$ is an $S$-morphism and $u:\sh{G}\to\sh{F}$ is an $h$-morphism.
We can then say that $(\sh{A}(X),\sh{A}(\sh{F}))$ is a \emph{contravariant functor} with values in the category whose objects are pairs consisting of an $\sh{O}_S$-algebra and a module over that algebra, and the morphisms are the di-homomorphisms.
\end{env}

\subsection{Affine preschemes over a prescheme}
\label{subsection:II.1.2}

\begin{definition}[1.2.1]
\label{II.1.2.1}
Let $X$ be an $S$-prescheme, and $f:X\to S$ its structure morphism.
We say that $X$ is \emph{affine over $S$} if there exists a cover $(S_\alpha)$ of $S$ by affine open sets such that for all $\alpha$, the prescheme induced by $X$ on the open set $f^{-1}(S_\alpha)$ is affine.
\end{definition}

\begin{example}[1.2.2]
\label{II.1.2.2}
Every closed subprescheme of $S$ is an affine $S$-prescheme over $S$ (\sref[I]{I.4.2.3} and \sref[I]{I.4.2.4}).
\end{example}

\begin{remark}[1.2.3]
\label{II.1.2.3}
An affine prescheme $X$ over $S$ is not necessarily an affine scheme, as the example $X=S$ shows \sref{II.1.2.2}.
On the other hand, if an affine scheme $X$ is an $S$-prescheme, then $X$ is not necessarily affine over
\oldpage[II]{7}
$S$ (see Example~\sref{II.1.3.3}).
However, remember that if $S$ is a \emph{scheme}, then every $S$-prescheme which is an affine scheme is affine over $S$ \sref[I]{I.5.5.10}.
\end{remark}

\begin{proposition}[1.2.4]
\label{II.1.2.4}
Every $S$-prescheme which is affine over $S$ is separated over $S$ (in other words, it is an $S$-scheme).
\end{proposition}

\begin{proof}
This follows immediately from \sref[I]{I.5.5.5} and \sref[I]{I.5.5.8}.
\end{proof}

\begin{proposition}[1.2.5]
\label{II.1.2.5}
Let $X$ be an $S$-scheme affine over $S$, and $f:X\to S$ its structure morphism.
For every open $U\subset S$, $f^{-1}(U)$ is affine over $U$.
\end{proposition}

\begin{proof}
By Definition~\sref{II.1.2.1}, we can reduce to the case where $S=\Spec(A)$ and $X=\Spec(B)$ are affine; then $f=({}^a\vphi,\widetilde{\vphi})$, where $\vphi:A\to B$ is a homomorphism.
As the $D(g)$ for $g\in A$ form a basis for $S$, we reduce to the case where $U=D(g)$; but we then know that $f^{-1}(U)=D(\vphi(g))$ (\textbf{I},~1.2.2.2), hence the proposition.
\end{proof}

\begin{proposition}[1.2.6]
\label{II.1.2.6}
Let $X$ be an $S$-scheme affine over $S$, and $f:X\to S$ its structure morphism.
For every quasi-coherent $\sh{O}_X$-module $\sh{F}$, $f_*(\sh{F})$ is a quasi-coherent $\sh{O}_S$-module.
\end{proposition}

\begin{proof}
Taking into account Proposition~\sref{II.1.2.4}, this follows from \sref[I]{I.9.2.2}[a].
\end{proof}

In particular, the $\sh{O}_S$-algebra $\sh{A}(X)=f_*(\sh{O}_X)$ is \emph{quasi-coherent}.

\begin{proposition}[1.2.7]
\label{II.1.2.7}
Let $X$ be an $S$-scheme affine over $S$.
For every $S$-prescheme $Y$, the map $h\mapsto\sh{A}(h)$ from the set $\Hom_S(Y,X)$ to the set $\Hom(\sh{A}(X),\sh{A}(Y))$ \sref{II.1.1.2} is bijective.
\end{proposition}

\begin{proof}
Let $f:X\to S$ and $g:Y\to S$ be the structure morphisms.
First, suppose that $S=\Spec(A)$ and $X=\Spec(B)$ are affine; we must prove that for every homomorphism $\omega:f_*(\sh{O}_X)\to g_*(\sh{O}_Y)$ of $\sh{O}_S$-algebras, there exists a unique $S$-morphism $h:Y\to X$ such that $\sh{A}(h)=\omega$.
By definition, for every open $U\subset S$, $\omega$ defines a homomorphism $\omega_U=\Gamma(U,\omega):\Gamma(f^{-1}(U),\sh{O}_X)\to\Gamma(g^{-1}(U),\sh{O}_Y)$ of $\Gamma(U,\sh{O}_S)$-algebras.
In particular, for $U=S$, this gives a homomorphism $\vphi:\Gamma(X,\sh{O}_X)\to\Gamma(Y,\sh{O}_Y)$ of $\Gamma(S,\sh{O}_S)$-algebras, to which corresponds a well-defined $S$-morphism $h:Y\to X$, since $X$ is affine \sref[I]{I.2.2.4}.
It remains to prove that $\sh{A}(h)=\omega$, or, in other words, that, for every open set $U$ of a basis for $S$, $\omega_U$ coincides with the homomorphism of algebras $\vphi_U$ corresponding to the $S$-morphism $g^{-1}(U)\to f^{-1}(U)$, a restriction of $h$.
We can reduce to the case where $U=D(\lambda)$, with $\lambda\in S$; then, if $f=({}^a\rho,\widetilde{\rho})$, where $\rho:A\to B$ is a ring homomorphism, we have $f^{-1}(U)=D(\mu)$, where $\mu=\rho(\lambda)$, and $\Gamma(f^{-1}(U),\sh{O}_X)$ is the ring of fractions $B_\mu$; the diagram
\[
  \xymatrix{
    B\ar[r]^\vphi\ar[d] &
    \Gamma(Y,\sh{O}_Y)\ar[d]\\
    B_\mu\ar[r]^{\vphi_U} &
    \Gamma(g^{-1}(U),\sh{O}_Y)
  }
\]
is commutative, and so is the analogous diagram where $\vphi_U$ is replaced by $\omega_U$; the equality $\vphi_U=\omega_U$ then follows from the universal property of rings of fractions \sref[0]{0.1.2.4}.

We now pass to the general case; let $(S_\alpha)$ be a cover of $S$ by affine open sets
\oldpage[II]{8}
such that the $f^{-1}(S_\alpha)$ are affine.
Then every homomorphism $\omega:\sh{A}(X)\to\sh{A}(Y)$ of $\sh{O}_S$-algebras gives by restriction a family of homomorphisms
\[
  \omega_\alpha:\sh{A}(f^{-1}(S_\alpha))\to\sh{A}(g^{-1}(S_\alpha))
\]
of $\sh{O}_{S_\alpha}$-algebras, hence a family of $S_\alpha$-morphisms $h_\alpha:g^{-1}(S_\alpha)\to f^{-1}(S_\alpha)$ by the above.
It remains to see that for every affine open set of a basis for $S_\alpha\cap S_\beta$, the restriction of $h_\alpha$ and $h_\beta$ to $g^{-1}(U)$ coincide, which is evident since by the above, these restrictions both correspond to the homomorphism $\sh{A}(X)|U\to\sh{A}(Y)|U$, a restriction of $\omega$.
\end{proof}

\begin{corollary}[1.2.8]
\label{II.1.2.8}
Let $X$ and $Y$ be two $S$-schemes which are affine over $S$.
For an $S$-morphism $h:Y\to X$ to be an isomorphism, it is necessary and sufficient for $\sh{A}(h):\sh{A}(X)\to\sh{A}(Y)$ to be an isomorphism.
\end{corollary}

\begin{proof}
This follows immediately from Proposition~\sref{II.1.2.7} and from the functorial nature of $\sh{A}(X)$.
\end{proof}

\subsection{Affine preschemes over $S$ associated to an $\mathcal{O}_S$-algebra}
\label{subsection:II.1.3}

\begin{proposition}[1.3.1]
\label{II.1.3.1}
Let $S$ be a prescheme.
For every quasi-coherent $\sh{O}_S$-algebra $\sh{B}$, there exists a prescheme $X$ affine over $S$, defined up to unique $S$-isomorphism, such that $\sh{A}(X)=\sh{B}$.
\end{proposition}

\begin{proof}
The uniqueness follows from Corollary~\sref{II.1.2.8}; we prove the existence of $X$.
For every affine open $U\subset S$, let $X_U$ be the prescheme $\Spec(\Gamma(U,\sh{B}))$; as $\Gamma(U,\sh{B})$ is a $\Gamma(U,\sh{O}_S)$-algebras, $X_U$ is an $S$-prescheme \sref[I]{I.1.6.1}.
In addition, as $\sh{B}$ is quasi-coherent, the $\sh{O}_S$-algebra $\sh{A}(X_U)$ canonically identifies with $\sh{B}|U$ (\sref[I]{I.1.3.7}, \sref[I]{I.1.3.13}, \sref[I]{I.1.6.3}).
Let $V$ be a second affine open subset of $S$, and let $X_{U,V}$ be the prescheme induced by $X_U$ on $f_U^{-1}(U\cap V)$, where $f_U$ denotes the structure morphism $X_U\to S$; $X_{U,V}$ and $X_{V,U}$ are affine over $U\cap V$ \sref{II.1.2.5}, and by definition $\sh{A}(X_{U,V})$ and $\sh{A}(X_{V,U})$ canonically identify with $\sh{B}|(U\cap V)$.
Hence there is \sref{II.1.2.8} a canonical $S$-isomorphism $\theta_{U,V}:X_{U,V}\to X_{U,V}$; in addition, if $W$ is a third affine open subset of $S$, and if $\theta_{U,V}'$, $\theta_{V,W}'$, and $\theta_{U,W}'$ are the restrictions of $\theta_{U,V}$, $\theta_{V,W}$, and $\theta_{U,W}$ to the inverse images of $U\cap V\cap W$ in $X_V$, $X_W$, and $X_W$ respectively under the structure morphisms, then we have $\theta_{U,V}'\circ\theta_{V,W}'=\theta_{U,W}'$.
As a result, there exists a prescheme $X$, a cover $(T_U)$ of $X$ by affine open sets, and for every $U$ an isomorphism $\vphi_U:X_U\to T_U$, such that $\vphi_U$ maps $f_U^{-1}(U\cap V)$ to $T_U\cap T_V$, and we have $\theta_{U,V}=\vphi_U^{-1}\circ\vphi_V$ \sref[I]{I.2.3.1}.
The morphism $g_U=f_U\circ\vphi_U^{-1}$ makes $T_U$ an $S$-prescheme, and the morphisms $g_U$ and $g_V$ coincide on $T_U\cap T_V$, hence $X$ is an $S$-prescheme.
In addition, it is clear by definition that $X$ is affine over $S$ and that $\sh{A}(T_U)=\sh{B}|U$, hence $\sh{A}(X)=\sh{B}$.
\end{proof}

We say that the $S$-scheme $X$ defined in this way is \emph{associated to the $\sh{O}_S$-algebra $\sh{B}$}, or is the \emph{spectrum of $\sh{B}$}, and we denote it by $\Spec(\sh{B})$.

\begin{corollary}[1.3.2]
\label{II.1.3.2}
Let $X$ be a prescheme affine over $S$, $f:X\to S$ the structure morphism.
For every affine open $U\subset S$, the induced prescheme on $f^{-1}(U)$ is the affine scheme with ring $\Gamma(U,\sh{A}(X))$.
\end{corollary}

\begin{proof}
\oldpage[II]{9}
As we can suppose that $X$ is associated to an $\sh{O}_S$-algebra by Propositions~\sref{II.1.2.6} and \sref{II.1.3.1}, the corollary follows from the construction of $X$ described in Proposition~\sref{II.1.3.1}.
\end{proof}

\begin{example}[1.3.3]
\label{II.1.3.3}
Let $S$ be the affine plane over a field $K$, where the point $0$ has been doubled \sref[I]{I.5.5.11}; with the notation of \sref[I]{I.5.5.11}, $S$ is the union of two affine open sets $Y_1$ and $Y_2$; if $f$ is the open immersion $Y_1\to S$, then $f^{-1}(Y_2)=Y_1\cap Y_2$ is not an affine open set in $Y_1$ (\emph{loc. cit.}), hence we have an example of an affine scheme which is not affine over $S$.
\end{example}

\begin{corollary}[1.3.4]
\label{II.1.3.4}
Let $S$ be an affine scheme; for an $S$-prescheme $X$ to be affine over $S$, it is necessary and sufficient for $X$ to be an affine scheme.
\end{corollary}

\begin{corollary}[1.3.5]
\label{II.1.3.5}
Let $X$ be a prescheme affine over a prescheme $S$, and let $Y$ be an $X$-prescheme.
For $Y$ to be affine over $S$, it is necessary and sufficient for $Y$ to be affine over $X$.
\end{corollary}

\begin{proof}
We immediately reduce to the case where $S$ is an affine scheme, and then we can reduce to the case where $X$ is an affine scheme \sref{II.1.3.4}; the two conditions of the statement then give that $Y$ is an affine scheme \sref{II.1.3.4}.
\end{proof}

\begin{env}[1.3.6]
\label{II.1.3.6}
Let $X$ be a prescheme affine over $S$.
To define a prescheme $Y$ affine \emph{over $X$}, it is equivalent, by Corollary~\sref{II.1.3.5}, to give a prescheme $Y$ affine \emph{over $S$}, and an $S$-morphism $g:Y\to X$; in other words (Proposition~\sref{II.1.3.1} and \sref{II.1.2.7}), it is equivalent to give a quasi-coherent $\sh{O}_S$-algebra $\sh{B}$ and a homomorphism $\sh{A}(X)\to\sh{B}$ of $\sh{O}_S$-algebras (which can be considered as defining on $\sh{B}$ an $\sh{A}(X)$-algebra structure).
If $f:X\to S$ is the structure morphism, then we have $\sh{B}=f_*(g_*(\sh{O}_Y))$.
\end{env}

\begin{corollary}[1.3.7]
\label{II.1.3.7}
Let $X$ be a prescheme affine over $S$; for $X$ to be of finite type over $S$, it is necessary and sufficient for the quasi-coherent $\sh{O}_S$-algebra $\sh{A}(X)$ to be of finite type \sref[I]{I.9.6.2}.
\end{corollary}

\begin{proof}
By definition \sref[I]{I.9.6.2}, we can reduce to the case where $S$ is affine; then $X$ is an affine scheme \sref{II.1.3.4}, and if $S=\Spec(A)$, $X=\Spec(B)$, then $\sh{A}(X)$ is the $\sh{O}_S$-algebra $\widetilde{B}$; as $\Gamma(U,\widetilde{B})=B$, the corollary follows from \sref[I]{I.9.6.2} and \sref[I]{I.6.3.3}.
\end{proof}

\begin{corollary}[1.3.8]
\label{II.1.3.8}
Let $X$ be a prescheme affine over $S$; for $X$ to be reduced, it is necessary and sufficient for the quasi-coherent $\sh{O}_X$-algebra $\sh{A}(X)$ to be reduced \sref[0]{0.4.1.4}.
\end{corollary}

\begin{proof}
The question is local on $S$; by Corollary~\sref{II.1.3.2}, the corollary follows from \sref[I]{I.5.1.1} and \sref[I]{I.5.1.4}.
\end{proof}

\subsection{Quasi-coherent sheaves over an affine prescheme over $S$}
\label{subsection:II.1.4}

\begin{proposition}[1.4.1]
\label{II.1.4.1}
Let $X$ be a prescheme affine over $S$, $Y$ an $S$-prescheme, and $\sh{F}$ (resp.~$\sh{G}$) a quasi-coherent $\sh{O}_X$-module (resp.~an $\sh{O}_Y$-module).
Then the map $(h,u)\mapsto(\sh{A}(h),\sh{A}(u))$ from the set of morphism $(Y,\sh{G})\to(X,\sh{F})$ to the set of di-homomorphisms $(\sh{A}(X),\sh{A}(\sh{F}))\to(\sh{A}(Y),\sh{A}(\sh{G}))$ (\sref{II.1.1.2} and \sref{II.1.1.3}) is bijective.
\end{proposition}

\begin{proof}
The proof follows exactly as that of Proposition~\sref{II.1.2.7} by using \sref[I]{I.2.2.5} and \sref[I]{I.2.2.4}, and the details are left to the reader.
\end{proof}

\begin{corollary}[1.4.2]
\label{II.1.4.2}
If, in addition to the hypotheses of Proposition~\sref{II.1.4.1}, we suppose that $Y$ is affine over $S$, then for $(h,u)$ to be an isomorphism, it is necessary and sufficient for $(\sh{A}(h),\sh{A}(u))$ to be a di-isomorphism.
\end{corollary}

\begin{proposition}[1.4.3]
\label{II.1.4.3}
\oldpage[II]{10}
For every pair $(\sh{B},\sh{M})$ consisting of a quasi-coherent $\sh{O}_S$-algebra $\sh{B}$ and a quasi-coherent $\sh{B}$-module $\sh{M}$ \emph{(considered as an $\sh{O}_S$-module or as a $\sh{B}$-module, which are equivalent~\sref[I]{I.9.6.1})}, there exists a pair $(X,\sh{F})$ consisting of a prescheme $X$ affine over $S$ and of a quasi-coherent $\sh{O}_X$-module $\sh{F}$, such that $\sh{A}(X)=\sh{B}$ and $\sh{A}(\sh{F})=\sh{M}$; in addition, this couple is determined up to unique isomorphism.
\end{proposition}

\begin{proof}
The uniqueness follows from Proposition~\sref{II.1.4.1} and Corollary~\sref{II.1.4.2}; the existence is proved as in Proposition~\sref{II.1.3.1}, and we leave the details to the reader.
\end{proof}

We denote by $\widetilde{\sh{M}}$ the $\sh{O}_X$-module $\sh{F}$, and we say that it is \emph{associated} to the quasi-coherent $\sh{B}$-module $\sh{M}$; for every affine open $U\subset S$, $\sh{M}|p^{-1}(U)$ (where $p$ is the structure morphism $X\to S$) is canonically isomorphic to $(\Gamma(U,\sh{M}))\supertilde$.

\begin{corollary}[1.4.4]
\label{II.1.4.4}
On the category of quasi-coherent $\sh{B}$-modules, $\widetilde{\sh{M}}$ is an additive covariant exact functor in $\sh{M}$, which commutes with inductive limits and direct sums.
\end{corollary}

\begin{proof}
We immediately reduce to the case where $S$ is affine, and the corollary then follows from \sref[I]{I.1.3.5}, \sref[I]{I.1.3.9}, and \sref[I]{I.1.3.11}.
\end{proof}

\begin{corollary}[1.4.5]
\label{II.1.4.5}
Under the hypotheses of Proposition~\sref{II.1.4.3}, for $\widetilde{\sh{M}}$ to be an $\sh{O}_X$-module of finite type, it is necessary and sufficient for $\sh{M}$ to be a $\sh{B}$-module of finite type.
\end{corollary}

\begin{proof}
We immediately reduce to the case where $S=\Spec(A)$ is an affine scheme.
Then $\sh{B}=\widetilde{B}$, where $B$ is an $A$-algebra of finite type \sref[I]{I.9.6.2}, and $\sh{M}=\widetilde{M}$, where $M$ is a $B$-module \sref[I]{I.1.3.13}; \emph{over the prescheme $X$}, $\sh{O}_X$ is associated to the ring $B$ and $\widetilde{\sh{M}}$ to the $B$-module $M$; for $\widetilde{\sh{M}}$ to be of finite type, it is therefore necessary and sufficient for $M$ to be of finite type \sref[I]{I.1.3.13}, hence our assertion.
\end{proof}

\begin{proposition}[1.4.6]
\label{II.1.4.6}
Let $Y$ be a prescheme affine over $S$, $X$ and $X'$ two preschemes affine over $Y$ \emph{(hence also over $S$ \sref{II.1.3.5})}.
Let $\sh{B}=\sh{A}(Y)$, $\sh{A}=\sh{A}(X)$, and $\sh{A}'=\sh{A}(X')$.
Then $X\times_Y X'$ is affine over $Y$ \emph{(thus also over $S$)}, and $\sh{A}(X\times_Y X')$ canonically identifies with $\sh{A}\otimes_\sh{B}\sh{A}'$.
\end{proposition}

\begin{proof}
By \sref[I]{I.9.6.1}, $\sh{A}\otimes_\sh{B}\sh{A}'$ is a quasi-coherent $\sh{B}$-algebra, thus also a quasi-coherent $\sh{O}_S$-algebra \sref[I]{I.9.6.1}; let $Z$ be the spectrum of $\sh{A}\otimes_\sh{B}\sh{A}'$ \sref{II.1.3.1}.
The canonical $\sh{B}$-homomorphisms $\sh{A}\to\sh{A}\otimes_\sh{B}\sh{A}'$ and $\sh{A}'\to\sh{A}\otimes_\sh{B}\sh{A}'$ correspond \sref{II.1.2.7} to $Y$-morphisms $Z\to X$ and $p':Z\to X'$.
To see that the triple $(Z,p,p')$ is a product $X\times_Y X'$, we can reduce to the case where $S$ is an affine scheme with ring $C$ \sref[I]{I.3.2.6.4}.
But then $Y$, $X$, and $X'$ are affine schemes \sref{II.1.3.4} whose rings $B$, $A$, and $A'$ are $C$-algebras such that $\sh{B}=\widetilde{B}$, $\sh{A}=\widetilde{A}$, and $\sh{A}'=\widetilde{A'}$.
We then know \sref[I]{I.1.3.13} that $\sh{A}\otimes_\sh{B}\sh{A}'$ canonically identifies with the $\sh{O}_S$-algebra $(A\otimes_B A')\supertilde$, hence the ring $A(Z)$ identifies with $A\otimes_B A'$ and the morphisms $p$ and $p'$ correspond to the canonical homomorphisms $A\to A\otimes_B A'$ and $A'\to A\otimes_B A'$.
The proposition then follows from \sref[I]{I.3.2.2}.
\end{proof}

\begin{corollary}[1.4.7]
\label{II.1.4.7}
Let $\sh{F}$ (resp.~$\sh{F}'$) be a quasi-coherent $\sh{O}_X$-module (resp.~$\sh{O}_{X'}$-module); then $\sh{A}(\sh{F}\otimes_Y\sh{F}')$ canonically identifies with $\sh{A}(\sh{F})\otimes_{\sh{A}(Y)}\sh{A}(\sh{F}')$.
\end{corollary}

\begin{proof}
We know that $\sh{F}\otimes_Y\sh{F}'$ is quasi-coherent over $X\times_Y X'$ \sref[I]{I.9.1.2}.
Let $g:Y\to S$, $f:X\to Y$, and $f':X'\to Y$ be the structure morphisms, such that the structure morphism
\oldpage[II]{11}
$h:Z\to S$ is equal to $g\circ f\circ p$ and to $g\circ f'\circ p'$.
We define a canonical homomorphism
\[
  \sh{A}(\sh{F})\otimes_{\sh{A}(Y)}\sh{A}(\sh{F}')\to\sh{A}(\sh{F}\otimes_Y\sh{F}')
\]
in the following way: for every open $U\subset S$, we have canonical homomorphisms $\Gamma(f^{-1}(g^{-1}(U)),\sh{F})\to\Gamma(h^{-1}(U),p^*(\sh{F}))$ and $\Gamma(f^{\prime-1}(g^{-1}(U)),\sh{F}')\to\Gamma(h^{-1}(U),p^{\prime*}(\sh{F}'))$ \sref[0]{0.4.4.3}, thus we obtain a canonical homomorphism
\[
  \Gamma(f^{-1}(g^{-1}(U)),\sh{F})\otimes_{\Gamma(g^{-1}(U),\sh{O}_Y)}\Gamma(f^{\prime-1}(g^{-1}(U)),\sh{F}')\to\Gamma(h^{-1}(U),p^*(\sh{F}))\otimes_{\Gamma(h^{-1}(U),\sh{O}_Z)}\Gamma(h^{-1}(U),p^{\prime*}(\sh{F}')).
\]

To see that we have defined an isomorphism of $\sh{A}(Z)$-modules, we can reduce to the case where $S$ (and as a result $X$, $X'$, $Y$, and $X\times_Y X'$) are affine scheme, and (with the notation of Proposition~\sref{II.1.4.6}), $\sh{F}=\widetilde{M}$, $\sh{F}'=\widetilde{M'}$, where $M$ (resp.~$M'$) is an $A$-module (resp.~an $A'$-module).
Then $\sh{F}\otimes_Y\sh{F}'$ identifies with the sheaf on $X\times_Y X'$ associated to the $(A\otimes_B A')$-module $M\otimes_B M'$ \sref[I]{I.9.1.3}, and the corollary follows from the canonical identification of the $\sh{O}_S$-modules $(M\otimes_B M')\supertilde$ and $\widetilde{M}\otimes_{\widetilde{B}}\widetilde{M'}$ (where $M$, $M'$, and $B$ are considered as $C$-modules) (\sref[I]{I.1.3.12} and \sref[I]{I.1.6.3}).
\end{proof}

If we apply Corollary~\sref{II.1.4.7} in particular to the case where $X=Y$ and $\sh{F}'=\sh{O}_{X'}$, then we see that the $\sh{A}'$-module $\sh{A}(f^{\prime*}(\sh{F}))$ identifies with $\sh{A}(\sh{F})\otimes_\sh{B}\sh{A}'$.

\begin{env}[1.4.8]
\label{II.1.4.8}
In particular, when $X=X'=Y$ ($X$ being affine over $S$), we see that if $\sh{F}$ and $\sh{G}$ are two quasi-coherent $\sh{O}_X$-modules, then we have
\[
\label{II.1.4.8.1}
  \sh{A}(\sh{F}\otimes_{\sh{O}_X}\sh{G})=\sh{A}(\sh{F})\otimes_{\sh{A}(X)}\sh{A}(\sh{G})
  \tag{1.4.8.1}
\]
up to canonical functorial isomorphism.
If in addition $\sh{F}$ admits a finite presentation, then it follows from \sref[I]{I.1.6.3} and \sref[I]{I.1.3.12} that
\[
\label{II.1.4.8.2}
  \sh{A}(\shHom_X(\sh{F},\sh{G}))=\shHom_{\sh{A}(X)}(\sh{A}(\sh{F}),\sh{A}(\sh{G}))
  \tag{1.4.8.2}
\]
up to canonical isomorphism.
\end{env}

\begin{remark}[1.4.9]
\label{II.1.4.9}
If $X$ and $X'$ are two preschemes affine over $S$, then the sum $X\sqcup X'$ is also affine over $S$, as the sum of two affine schemes is an affine scheme.
\end{remark}

\begin{proposition}[1.4.10]
\label{II.1.4.10}
Let $S$ be a prescheme, $\sh{B}$ a quasi-coherent $\sh{O}_S$-algebra, and $X=\Spec(\sh{B})$.
For a quasi-coherent sheaf of ideals $\sh{J}$ of $\sh{B}$, $\widetilde{\sh{J}}$ is quasi-coherent sheaf of ideals of $\sh{O}_X$, and the closed subprescheme $Y$ of $X$ defined by $\widetilde{\sh{J}}$ is canonically isomorphic to $\Spec(\sh{B}/\sh{J})$.
\end{proposition}

\begin{proof}
It follows immediately from \sref[I]{I.4.2.3} that $Y$ is affine over $S$; by Proposition~\sref{II.1.3.1}, we reduce to the case where $S$ is affine, and the proposition then follows immediately from \sref[I]{I.4.1.2}.
\end{proof}

We can also express the result of Proposition~\sref{II.1.4.10} by saying that if $h:\sh{B}\to\sh{B}'$ is a \emph{surjective} homomorphism of quasi-coherent $\sh{O}_S$-algebras, $\sh{A}(h):\Spec(\sh{B}')\to\Spec(\sh{B})$ is a \emph{closed immersion}.

\begin{proposition}[1.4.11]
\label{II.1.4.11}
\oldpage[II]{12}
Let $S$ be a prescheme, $\sh{B}$ a quasi-coherent $\sh{O}_S$-algebra, and $X=\Spec(\sh{B})$.
For every quasi-coherent sheaf of ideals $\sh{K}$ of $\sh{O}_S$, we have (denoting by $f$ the structure morphism $X\to S$) $f^*(\sh{K})\sh{O}_X=(\sh{K}\sh{B})\supertilde$ up to canonical isomorphism.
\end{proposition}

\begin{proof}
Since the questions is local on $S$, we can reduce to the case where $S=\Spec(A)$ is affine, and in this case the proposition is none other than \sref[I]{I.1.6.9}.
\end{proof}

\subsection{Change of base prescheme}
\label{subsection:II.1.5}

\begin{proposition}[1.5.1]
\label{II.1.5.1}
Let $X$ be a prescheme affine over $S$.
For every extension $g:S'\to S$ of the base prescheme, $X'=X_{(S')}=X\times_S S'$ is affine over $S'$.
\end{proposition}

\begin{proof}
If $f'$ is the projection $X'\to S'$, then it suffices to prove that $f^{\prime-1}(U')$ is an affine open set for every affine open subset $U'$ of $S'$ such that $g(U')$ is contained in an affine open subset $U$ of $S$ \sref{II.1.2.1}; we can thus reduce to the case where $S$ and $S'$ are affine, and it suffices to prove that $X'$ is then an affine scheme \sref{II.1.3.4}.
But then \sref{II.1.3.4} $X$ is an affine scheme, and if $A$, $A'$, and $B$ are the rings of $S$, $S'$, and $X$ respectively, then we know that $X'$ is the affine scheme with ring $A'\otimes_A B$ \sref[I]{I.3.2.2}.
\end{proof}

\begin{corollary}[1.5.2]
\label{II.1.5.2}
Under the hypotheses of Proposition~\sref{II.1.5.1}, let $f:X\to S$ be the structure morphism, $f':X'\to S'$ and $g':X'\to X$ the projections, such that the diagram
\[
  \xymatrix{
    X\ar[d]_f &
    X'\ar[l]_{g'}\ar[d]^{f'}\\
    S &
    S'\ar[l]_g
  }
\]
is commutative.
For every quasi-coherent $\sh{O}_X$-module $\sh{F}$, there exists a canonical isomorphism of $\sh{O}_{S'}$-modules
\[
\label{II.1.5.2.1}
  u:g^*(f_*(\sh{F}))\isoto f_*'(g^{\prime*}(\sh{F})).
  \tag{1.5.2.1}
\]
In particular, there exists a canonical isomorphism from $\sh{A}(X')$ to $g^*(\sh{A}(X))$.
\end{corollary}

\begin{proof}
To define $u$, it suffices to define a homomorphism
\[
  v:f_*(\sh{F})\to g_*(f_*'(g^{\prime*}(\sh{F})))=f_*(g_*'(g^{\prime*}(\sh{F})))
\]
and to set $u=v^\sharp$ \sref[0]{0.4.4.3}.
We take $v=f_*(\rho)$, where $\rho$ is the canonical homomorphism $\sh{F}\to g_*'(g^{\prime*}(\sh{F}))$ \sref[0]{0.4.4.3}.
To prove that $u$ is an isomorphism, we can reduce to the case where $S$ and $S'$, hence $X$ and $X'$, are affine; with the notation of Proposition~\sref{II.1.5.1}, we then have $\sh{F}=\widetilde{M}$, where $M$ is a $B$-module.
We then note immediately that $g^*(f_*(\sh{F}))$ and $f_*'(g^{\prime*}(\sh{F}))$ are both equal to the $\sh{O}_{S'}$-module associated to the $A'$-module $A'\otimes_A M$ (where $M$ is considered as an $A$-module), and that $u$ is the homomorphism associated to the identity (\sref[I]{I.1.6.3}, \sref[I]{I.1.6.5}, \sref[I]{I.1.6.7}).
\end{proof}

\begin{remark}[1.5.3]
\label{II.1.5.3}
We do not have that Corollary~\sref{II.1.5.2} remains true when $X$ is not assumed affine over $S$, even when $S'=\Spec(\kres(s))$ ($s\in S$) and $S'\to S$ is the canonical morphism \sref[I]{I.2.4.5}---in which case $X'$ is none other than the \emph{fibre $f^{-1}(s)$} \sref[I]{I.3.6.2}.
In other words, when $X$ is not affine over $S$, the operation
\oldpage[II]{13}
``direct image of quasi-coherent sheaves'' does not commute with the operation of ``passing to fibres''.
However, we will see in Chapter~III \sref[III]{3.4.2.4} a result in this sense, of an ``asymptotic'' nature, valid for \emph{coherent} sheaves on $X$ when $f$ is proper~(5.4) and $S$ is Noetherian.
\end{remark}

\begin{corollary}[1.5.4]
\label{II.1.5.4}
For every prescheme $X$ affine over $S$ and every $s\in S$, the fibre $f^{-1}(s)$ (where $f$ denoted the structure morphism $X\to S$) is an affine scheme.
\end{corollary}

\begin{proof}
It suffices to apply Proposition~\sref{II.1.5.1} with $S'=\Spec(\kres(s))$ and to use Corollary~\sref{II.1.3.4}.
\end{proof}

\begin{corollary}[1.5.5]
\label{II.1.5.5}
Let $X$ be an $S$-prescheme, $S'$ a prescheme affine over $S$; then $X'=X_{(S')}$ is a prescheme affine over $X$.
In addition, if $f:X\to S$ is the structure morphism, then there is a canonical isomorphism of $\sh{O}_X$-algebras $\sh{A}(X')\isoto f^*(\sh{A}(S'))$, and for every quasi-coherent $\sh{A}(S')$-module $\sh{M}$, a canonical di-isomorphism $f^*(\sh{M})\isoto\sh{A}(f^{\prime*}(\widetilde{\sh{M}}))$, denoting by $f'=f_{(S')}$ the structure morphism $X'\to S'$.
\end{corollary}

\begin{proof}
It suffices to swap the roles of $X$ and $S'$ in \sref{II.1.5.1} and \sref{II.1.5.2}.
\end{proof}

\begin{env}[1.5.6]
\label{II.1.5.6}
Now let $S$, $S'$ be two preschemes, $q:S'\to S$ a morphism, $\sh{B}$ (resp. $\sh{B}'$) a quasi-coherent $\sh{O}_S$-algebra (resp. $\sh{O}_{S'}$-algebra), $u:\sh{B}\to\sh{B}'$ a $q$-morphism (that is, a homomorphism $\sh{B}\to q_*(\sh{B}')$ of $\sh{O}_S$-algebras).
If $X=\Spec(\sh{B})$, $X'=\Spec(\sh{B}')$, then we canonically obtain a morphism
\[
  v=\Spec(u):X'\to X
\]
such that the diagram
\[
\label{II.1.5.6.1}
  \xymatrix{
    X'\ar[r]^-{v}\ar[d] & X\ar[d]\\
    S'\ar[r]^-{q} & S
  }
  \tag{1.5.6.1}
\]
is commutative (the vertical arrows being the structure morphisms).
Indeed, the data of $u$ is equivalent to that of a homomorphism of quasi-coherent $\sh{O}_{S'}$-algebras $u^\sharp:q^*(\sh{B})\to\sh{B}'$ \sref[0]{0.4.4.3}; this thus canonically defines an $S'$-morphism
\[
  w:\Spec(\sh{B}')\to\Spec(q^*(\sh{B}))
\]
such that $\sh{A}(w)=u^\sharp$ \sref{II.1.2.7}.
On the other hand, it follows from \sref{II.1.5.2} that $\Spec(q^*(\sh{B}))$ canonically identifies with $X\times_S S'$; the morphism $v$ is the composition $X'\xrightarrow{w}X\times_S S'\xrightarrow{p_1}X$ of $w$ with the first projection, and the commutativity of \sref{II.1.5.6.1} follows from the definitions.
Let $U$ (resp. $U'$) be an affine open of $S$ (resp. $S'$) such that $q(U')\subset U$, $A=\Gamma(U,\sh{O}_S)$, $A'=\Gamma(U',\sh{O}_{S'})$ their rings, $B=\Gamma(U,\sh{B})$, $B'=\Gamma(U',\sh{B}')$; the restriction of $u$ to a $(q|U')$-morphism: $\sh{B}|U\to\sh{B}'|U'$ corresponds to a di-homomorphism of algebras $B\to B'$; if $V$, $V'$ are the inverse images of $U$, $U'$ in $X$, $X'$ respectively, under the structure morphisms, then the morphism $V'\to V$, the restriction of $v$, corresponds \sref[I]{I.1.7.3} to the above di-homomorphism.
\end{env}

\begin{env}[1.5.7]
\label{II.1.5.7}
Under the same hypotheses as in \sref{II.1.5.6}, let $\sh{M}$ be a quasi-coherent $\sh{B}$-module; there is then a canonical isomorphism of $\sh{O}_{X'}$-modules
\[
\label{II.1.5.7.1}
  v^*(\widetilde{\sh{M}})\isoto(q^*(\sh{M})\otimes_{q^*(\sh{B})}\sh{B}')\supertilde.
  \tag{1.5.7.1}
\]

\oldpage[II]{14}
Indeed, the canonical isomorphism \sref{II.1.5.2.1} gives a canonical isomorphism from $p_1^*(\widetilde{\sh{M}})$ to the sheaf on $\Spec(q^*(\sh{B}))$ associated to the $q^*(\sh{B})$-module $q^*(\sh{M})$, and it then suffices to apply \sref{II.1.4.7}.
\end{env}

\subsection{Affine morphisms}
\label{subsection:II.1.6}

\begin{env}[1.6.1]
\label{II.1.6.1}
We say that a morphism $f:X\to Y$ of preschemes is \emph{affine} if it defines $X$ as a prescheme affine over $Y$.
The properties of preschemes affine over another translates as follows in this language:
\end{env}

\begin{proposition}[1.6.2]
\label{II.1.6.2}
\medskip\noindent
\begin{enumerate}
  \item[{\rm(i)}] A closed immersion is affine.
  \item[{\rm(ii)}] The composition of two affine morphisms is affine.
  \item[{\rm(iii)}] If $f:X\to Y$ is an affine $S$-morphism, then $f_{(S')}:X_{(S')}\to Y_{(Y')}$ is affine for every base change $S'\to S$.
  \item[{\rm(iv)}] If $f:X\to Y$ and $f':X'\to Y'$ are two affine $S$-morphisms, then
    \[
      f\times_S f':X\times_S X'\to Y\times_S Y'
    \]
    is affine.
  \item[{\rm(v)}] If $f:X\to Y$ and $g:Y\to Z$ are two morphisms such that $g\circ f$ is affine and $g$ is separated, then $f$ is affine.
  \item[{\rm(vi)}] If $f$ is affine, then so if $f_\red$.
\end{enumerate}
\end{proposition}

\begin{proof}
By \sref[I]{I.5.5.12}, it suffices to prove (i), (ii), and (iii).
But (i) is none other than Example~\sref{II.1.2.2}, and (ii) is none other than Corollary~\sref{II.1.3.5}; finally, (iii) follows from Proposition~\sref{II.1.5.1}, since $X_{(S')}$ identifies with the product $X\times_Y Y_{(S')}$ \sref[I]{I.3.3.11}.
\end{proof}

\begin{corollary}[1.6.3]
\label{II.1.6.3}
If $X$ is an affine scheme and $Y$ is a scheme, then every morphism $f:X\to Y$ is affine.
\end{corollary}

\begin{proposition}[1.6.4]
\label{II.1.6.4}
Let $Y$ be a locally Noetherian prescheme, $f:X\to Y$ a morphism of finite type.
For $f$ to be affine, it is necessary and sufficient for $f_\red$ to be.
\end{proposition}

\begin{proof}
By \sref{II.1.6.2}[vi], we see only need to prove the sufficiency of the condition.
It suffices to prove that if $Y$ is affine and Noetherian, then $X$ is affine; but $Y_\red$ is then affine, so the same is true for $X_\red$ by hypothesis.
Now $X$ is Noetherian, so the conclusion follows from \sref[I]{I.6.1.7}.
\end{proof}

\subsection{Vector bundle associated to a sheaf of modules}
\label{subsection:II.1.7}

\begin{env}[1.7.1]
\label{II.1.7.1}
Let $A$ be a ring, $E$ an $A$-module.
Recall that we call the \emph{symmetric algebra} on $E$ and denote by $\bb{S}(E)$ (or $\bb{S}_A(E)$) the quotient algebra of the tensor algebra $\bb{T}(E)$ by the two-sided ideal generated by the elements $x\otimes y-y\otimes x$, where $x$ and $y$ vary over $E$.
The algebra $\bb{S}(E)$ is characterized by the following universal property: if $\sigma$ is the canonical map $E\to\bb{S}(E)$ (obtained by composing $E\to\bb{T}(E)$ with the canonical map $\bb{T}(E)\to\bb{S}(E)$), then every $A$-linear map $E\to B$, where $B$ is a \emph{commutative} $A$-algebra, factors uniquely as $E\xrightarrow{\sigma}\bb{S}(E)\xrightarrow{g}B$, where $g$ is an $A$-homomorphism \emph{of algebras}.
We immediately deduce from this characterization that for two $A$-modules $E$ and $F$, we have
\[
  \bb{S}(E\oplus F)=\bb{S}(E)\otimes\bb{S}(F)
\]
\oldpage[II]{15}
up to canonical isomorphism; in addition, $\bb{S}(E)$ is a covariant functor in $E$, from the category of $A$-modules to that of commutative $A$-algebras; finally, the above characterization also shows that if $E=\varinjlim E_\lambda$, then we have $\bb{S}(E)=\varinjlim\bb{S}(E_\lambda)$ up to canonical isomorphism.
By abuse of language, a product $\sigma(x_1)\sigma(x_2)\cdots\sigma(x_n)$, where $x_i\in E$, is often denoted by $x_1 x_2\cdots x_n$ if no confusion follows.
The algebra $\bb{S}(E)$ is \emph{graded}, $\bb{S}_n(E)$ being the set of linear combinations of $n$ elements of $E$ ($n\geq 0$); the algebra $\bb{S}(A)$ is canonically isomorphic to the polynomial algebra $A[T]$ is an indeterminate, and the algebra $\bb{S}(A^n)$ with the polynomial algebra in $n$ indeterminates $A[T_1,\dots,T_n]$.
\end{env}

\begin{env}[1.7.2]
\label{II.1.7.2}
Let $\vphi$ be a ring homomorphism $A\to B$.
If $F$ is a $B$-module, then the canonical map $F\to\bb{S}(F)$ gives a canonical map $F_{[\vphi]}\to\bb{S}(F)_{[\vphi]}$, which thus factors as $F_{[\vphi]}\to\bb{S}(F_{[\vphi]})\to\bb{S}(F)_{[\vphi]}$; the canonical homomorphism $\bb{S}(F_{[\vphi]})\to\bb{S}(F)_{[\vphi]}$ is surjective, but not necessarily bijective.
If $E$ is an $A$-module, then every di-homomorphism $E\to F$ (that is to say, every $A$-homomorphism $E\to F_{[\vphi]}$) thus canonically gives an $A$-homomorphism of algebras $\bb{S}(E)\to\bb{S}(F_{[\vphi]})\to\bb{S}(F)_{[\vphi]}$, that is to say a di-homomorphism of algebras $\bb{S}(E)\to\bb{S}(F)$.

With the same notations, for every $A$-module $E$, $\bb{S}(E\otimes_A B)$ canonically identifies with the algebra $\bb{S}(E)\otimes_A B$; this follows immediately from the universal property of $\bb{S}(E)$ \sref{II.1.7.1}.
\end{env}

\begin{env}[1.7.3]
\label{II.1.7.3}
Let $R$ be a multiplicative subset of the ring $A$; apply \sref{II.1.7.2} to the ring $B=R^{-1}A$, and remembering that $R^{-1}E=E\otimes_A R^{-1}A$, we see that we have $\bb{S}(R^{-1}E)=R^{-1}\bb{S}(E)$ up to canonical isomorphism.
In addition, if $R'\supset R$ is a second multiplicative subset of $A$, then the diagram
\[
  \xymatrix{
    {R^{-1}E}\ar[r]\ar[d] & {{R'}^{-1}E}\ar[d]\\
    {\bb{S}(R^{-1}E)}\ar[r] & {\bb{S}({R'}^{-1}E)}
  }
\]
is commutative.
\end{env}

\begin{env}[1.7.4]
\label{II.1.7.4}
Now let $(S,\sh{A})$ be a ringed space, and let $\sh{E}$ be a $\sh{A}$-module over $S$.
If to any open $U\subset S$ we associate the $\Gamma(U,\sh{A})$-module $\bb{S}(\Gamma(U,\sh{E}))$, then we define (see the functorial nature of $\bb{S}(E)$ \sref{II.1.7.2}) a presheaf of algebras; we say that the associated sheaf, which we denote by $\bb{S}(\sh{E})$ or $\bb{S}_\sh{A}(\sh{E})$ is the \emph{symmetric $\sh{A}$-algebra} on the $\sh{A}$-module $\sh{E}$.
It follows immediately from \sref{II.1.7.1} that $\bb{S}(\sh{E})$ is a solution to a universal problem: every homomorphism of $\sh{A}$-modules $\sh{E}\to\sh{B}$, where $\sh{B}$ is an $\sh{A}$-algebra, factors uniquely as $\sh{E}\to\bb{S}(\sh{E})\to\sh{B}$, the second arrow being a homomorphism of $\sh{A}$-algebras.
There is thus a bijective correspondence between homomorphisms $\sh{E}\to\sh{B}$ of $\sh{A}$-modules and homomorphisms $\bb{S}(\sh{E})\to\sh{B}$ of $\sh{A}$-algebras.
In particular, every homomorphism $u:\sh{E}\to\sh{F}$ of $\sh{A}$-modules defines a homomorphism $\bb{S}(u):\bb{S}(\sh{E})\to\bb{S}(\sh{F})$ of $\sh{A}$-algebras, and $\bb{S}(\sh{E})$ is thus a covariant functor in $\sh{E}$.

\oldpage[II]{16}
By \sref{II.1.7.2} and the commutativity of $\bb{S}$ with inductive limits, we have $(\bb{S}(\sh{E}))_x=\bb{S}(\sh{E}_x)$ for every point $x\in S$.
If $\sh{E}$, $\sh{F}$ are two $\sh{A}$-modules, then $\bb{S}(\sh{E}\oplus\sh{F})$ canonically identifies with $\bb{S}(\sh{E})\otimes_\sh{A}\bb{S}(\sh{F})$, as we see for the corresponding presheaves.

We also note that $\bb{S}(\sh{E})$ is a graded $\sh{A}$-algebra, the infinite direct sum of the $\bb{S}_n(\sh{E})$, where the $\sh{A}$-module $\bb{S}_n(\sh{E})$ is the sheaf associated to the presheaf $U\mapsto\bb{S}_n(\Gamma(U,\sh{E}))$.
If we take in particular $\sh{E}=\sh{A}$, then we see that $\bb{S}_\sh{A}(\sh{A})$ identifies with $\sh{A}[T]=\sh{A}\otimes_\bb{Z}\bb{Z}[T]$ ($T$ indeterminate, $\bb{Z}$ being considered as a simple sheaf).
\end{env}

\begin{env}[1.7.5]
\label{II.1.7.5}
Let $(T,\sh{B})$ be a second ringed space, $f$ a morphism $(S,\sh{A})\to(T,\sh{B})$.
If $\sh{F}$ is a $\sh{B}$-module, then $\bb{S}(f^*(\sh{F}))$ canonically identifies with $f^*(\bb{S}(\sh{F}))$; indeed, if $f=(\psi,\theta)$, then by definition \sref[0]{0.4.3.1},
\[
  \bb{S}(f^*(\sh{F}))=\bb{S}(\psi^*(\sh{F})\otimes_{\psi^*(\sh{B})}\sh{A})=\bb{S}(\psi^*(\sh{F}))\otimes_{\psi^*(\sh{B})}\sh{A}
\]
\sref{II.1.7.2}; for every open $U$ of $S$ and every section $h$ of $\bb{S}(\psi^*(\sh{F}))$ over $U$, $h$ coincides, in a neighbourhood $V$ of every point $s\in U$, with an element of $\bb{S}(\Gamma(V,\psi^*(\sh{F})))$; if we refer to the definition of $\psi^*(\sh{F})$ \sref[0]{0.3.7.1} and take into account that every element of $\bb{S}(E)$ for a module $E$ is a linear combination of a finite number of products of elements of $E$, then we see that there is a neighbourhood $W$ of $\psi(s)$ in $T$, a section $h'$ of $\bb{S}(\sh{F})$ over $W$, and a neighbourhood $V'\subset V\cap\psi^{-1}(W)$ of $s$ such that $h$ coincides with $t\mapsto h'(\psi(t))$ over $V'$; hence out assertion.
\end{env}

\begin{proposition}[1.7.6]
\label{II.1.7.6}
Let $A$ be a ring, $S=\Spec(A)$ its prime spectrum, $\sh{E}=\widetilde{M}$ the $\sh{O}_S$-module associated to an $A$-module $M$; then the $\sh{O}_S$-algebra $\bb{S}(\sh{E})$ is associated to the $A$-algebra $\bb{S}(M)$.
\end{proposition}

\begin{proof}
For every $f\in A$, $\bb{S}(M_f)=(\bb{S}(M))_f$ \sref{II.1.7.3}, and the proposition thus follows from Definition \sref[I]{I.1.3.4}.
\end{proof}

\begin{corollary}[1.7.7]
\label{II.1.7.7}
If $S$ is a prescheme, $\sh{E}$ a quasi-coherent $\sh{O}_S$-module, then the $\sh{O}_S$-algebra $\bb{S}(\sh{E})$ is quasi-coherent.
If in addition $\sh{E}$ is of finite type, then each of the $\sh{O}_S$-modules $\bb{S}_n(\sh{E})$ is of finite type.
\end{corollary}

\begin{proof}
The first assertion is an immediate consequence of \sref{II.1.7.6} and of \sref[I]{I.1.4.1}; the second follows from the fact that if $E$ is an $A$-module of finite type, then $\bb{S}_n(E)$ is an $A$-module of finite type; we then apply \sref[I]{I.1.3.13}.
\end{proof}

\begin{definition}[1.7.8]
\label{II.1.7.8}
Let $\sh{E}$ be a quasi-coherent $\sh{O}_S$-module.
We call the \emph{vector bundle over $S$ defined by $\sh{E}$} and denote by $\bb{V}(\sh{E})$ the spectrum \sref{II.1.3.1} of the quasi-coherent $\sh{O}_S$-algebra $\bb{S}(\sh{E})$.
\end{definition}

By \sref{II.1.2.7}, for every $S$-prescheme $X$, there is a canonical bijective correspondence between the $S$-morphisms $X\to\bb{V}(\sh{E})$ and the homomorphisms of \emph{$\sh{O}_S$-algebras $\bb{S}(\sh{E})\to\sh{A}(X)$}, thus also between these $S$-morphisms and the homomorphisms of \emph{$\sh{O}_S$-modules $\sh{E}\to\sh{A}(X)=f_*(\sh{O}_X)$} (where $f$ is the structure morphism $X\to S$).
In particular:
\begin{env}[1.7.9]
\label{II.1.7.9}
Take for $X$ a subprescheme induced by $S$ on an \emph{open $U\subset S$}.
Then the $S$-morphisms $U\to\bb{V}(\sh{E})$ are none other than the $U$-sections \sref[I]{I.2.5.5} of the $U$-prescheme induced by $\bb{V}(\sh{E})$ on the open $p^{-1}(U)$ (where $p$ is the structure morphism $\bb{V}(\sh{E})\to S$).
From what we have just seen, these $U$-sections bijectively correspond to homomorphisms of $\sh{O}_S$-modules $\sh{E}\to j_*(\sh{O}_S|U)$ (where $j$ is the canonical injection $U\to S$), or
\oldpage[II]{17}
equivalently \sref[0]{0.4.4.3} with the $(\sh{O}_S|U)$-homomorphisms $j^*(\sh{E})=\sh{E}|U\to\sh{O}_S|U$.
In addition, it is immediate that the restriction to an open $U'\subset U$ of an $S$-morphism $U\to\bb{V}(\sh{E})$ corresponds to the restriction to $U'$ of the corresponding homomorphism $\sh{E}|U\to\sh{O}_S|U$.
We conclude that \emph{the sheaf of germs of $S$-sections} of $\bb{V}(\sh{E})$ is canonically identified with the \emph{dual $\dual{\sh{E}}$} of $\sh{E}$.

In particular, if we set $X=U=S$, then the \emph{zero} homomorphism $\sh{E}\to\sh{O}_S$ corresponds to a canonical $S$-section of $\bb{V}(\sh{E})$, called the \emph{zero $S$-section} (cf.~\sref{II.8.3.3}).
\end{env}

\begin{env}[1.7.10]
\label{II.1.7.10}
Now take $X$ to be the spectrum $\{\xi\}$ of a field $K$; the structure morphism $f:X\to S$ then corresponds to a monomorphism $\kres(s)\to K$, where $s=f(\xi)$ \sref[I]{I.2.4.6}; the $S$-morphisms $\{\xi\}\to\bb{V}(\sh{E})$ are none other than the \emph{geometric points of $\bb{V}(\sh{E})$ with values in the extension $K$ of $\kres(s)$} \sref[I]{I.3.4.5}, points which are localized at the points of $p^{-1}(s)$.
The set of these points, which we can call \emph{the rational geometric fibre over $K$} of $\bb{V}(\sh{E})$ \emph{over the point $s$}, is identified by \sref{II.1.7.8} with the set of homomorphisms of $\sh{O}_S$-modules $\sh{E}\to f_*(\sh{O}_X)$, or, equivalently \sref[0]{0.4.4.3} with the set of homomorphisms of $\sh{O}_X$-modules $f^*(\sh{E})\to\sh{O}_X=K$.
But we have by definition \sref[0]{0.4.3.1} $f^*(\sh{E})=\sh{E}_s\otimes_{\sh{O}_s}K=\sh{E}^s\otimes_{\kres(s)}K$, setting $\sh{E}^s=\sh{E}_s/\mathfrak{m}_s\sh{E}_s$; the geometric fibre of $\bb{V}(\sh{E})$ rational over $K$ over $s$ thus identifies with the \emph{dual} of the \emph{$K$-vector space $\sh{E}^s\otimes_{\kres(s)}K$}; if $\sh{E}^s$ or $K$ is of finite dimension over $\kres(s)$, then this dual also identifies with $\dual{(\sh{E}^s)}\otimes_{\kres(s)}K$, denoting by $\dual{(\sh{E}^s)}$ the dual of the $\kres(s)$-vector space $\sh{E}^s$.
\end{env}

\begin{proposition}[1.7.11]
\label{II.1.7.11}
\medskip\noindent
\begin{enumerate}
  \item[{\rm(i)}] $\bb{V}(\sh{E})$ is a contravariant functor in $\sh{E}$ from the category of quasi-coherent $\sh{O}_S$-modules to the category of affine $S$-schemes.
  \item[{\rm(ii)}] If $\sh{E}$ is an $\sh{O}_S$-module of finite type, then $\bb{V}(\sh{E})$ is of finite type over $S$.
  \item[{\rm(iii)}] If $\sh{E}$ and $\sh{F}$ are two quasi-coherent $\sh{O}_S$-modules, then $\bb{V}(\sh{E}\oplus\sh{F})$ canonically identifies with $\bb{V}(\sh{E})\times_S\bb{V}(\sh{F})$.
  \item[{\rm(iv)}] Let $g:S'\to S$ be a morphism; for every quasi-coherent $\sh{O}_S$-module $\sh{E}$, $\bb{V}(g^*(\sh{E}))$ canonically identifies with $\bb{V}(\sh{E})_{(S')}=\bb{V}(\sh{E})\times_S S'$.
  \item[{\rm(v)}] A surjective homomorphism $\sh{E}\to\sh{F}$ of quasi-coherent $\sh{O}_S$-modules corresponds to a closed immersion $\bb{V}(\sh{F})\to\bb{V}(\sh{E})$.
\end{enumerate}
\end{proposition}

\begin{proof}
(i) is an immediate consequence of \sref{II.1.2.7}, taking into account that every homomorphism of $\sh{O}_S$-modules $\sh{E}\to\sh{F}$ canonically defines a homomorphism of $\sh{O}_S$-algebras $\bb{S}(\sh{E})\to\bb{S}(\sh{F})$.
(ii) follows immediately from the definition \sref[I]{I.6.3.1} and the fact that if $E$ is an $A$-module of finite type, then $\bb{S}(E)$ is an $A$-algebra of finite type.
To prove (iii), it suffices to start with the canonical isomorphism $\bb{S}(\sh{E}\oplus\sh{F})\isoto\bb{S}(\sh{E})\otimes_{\sh{O}_S}\bb{S}(\sh{F})$ \sref{II.1.7.4} and to apply \sref{II.1.4.6}.
Similarly, to prove (iv), it suffices to start with the canonical isomorphism $\bb{S}(g^*(\sh{E}))\isoto g^*(\bb{S}(\sh{E}))$ \sref{II.1.7.5} and to apply \sref{II.1.5.2}.
Finally, to establish (v), it suffices to remark that if the homomorphism $\sh{E}\to\sh{F}$ is surjective, then so is the corresponding homomorphism $\bb{S}(\sh{E})\to\bb{S}(\sh{F})$ of $\sh{O}_S$-algebras, and the conclusion follows from \sref{II.1.4.10}.
\end{proof}

\begin{env}[1.7.12]
\label{II.1.7.12}
Take in particular $\sh{E}=\sh{O}_S$; the prescheme $\bb{V}(\sh{O}_S)$ is the affine $S$-scheme, spectrum of the $\sh{O}_S$-algebra $\bb{S}(\sh{O}_S)$ which identifies with the $\sh{O}_S$-algebra $\sh{O}_S[T]=\sh{O}_S\otimes_\bb{Z}\bb{Z}[T]$
\oldpage[II]{18}
($T$ indeterminate); this is evident when $S=\Spec(\bb{Z})$, by virtue of \sref{II.1.7.6}, and we pass from there to the general case by considering the structure morphism $S\to\Spec(\bb{Z})$ and using \sref{II.1.7.11}[iv].
Because of this result, we set $\bb{V}(\sh{O}_S)=S[T]$, and we thus have the formula
\[
\label{II.1.7.12.1}
  S[T]=S\otimes_\bb{Z}\bb{Z}[T].
  \tag{1.7.12.1}
\]

The identification of the sheaf of germs of $S$-sections of $S[T]$ with $\sh{O}_S$, already seen in \sref[I]{I.3.3.15}, here in a more general context, as a special case of \sref{II.1.7.9}.
\end{env}

\begin{env}[1.7.13]
\label{II.1.7.13}
For every $S$-prescheme $X$, we have seen \sref{II.1.7.8} that $\Hom_S(X,S[T])$ canonically identifies with $\Hom_{\sh{O}_S}(\sh{O}_S,\sh{A}(X))$, which is canonically isomorphic to $\Gamma(S,\sh{A}(X))$, and as a result is equipped with the structure of a ring; in addition, to every $S$-morphism $h:X\to Y$ there corresponds a morphism $\Gamma(\sh{A}(h)):\Gamma(S,\sh{A}(Y))\to\Gamma(S,\sh{A}(X))$ for the ring structures \sref{II.1.1.2}.
When we equip $\Hom_S(X,S[T])$ with a ring structure as defined, then we can see that $\Hom(X,S[T])$ can be considered as a \emph{contravariant functor} in $X$, from the category of $S$-preschemes to that of rings.
On the other hand, $\Hom_S(X,\bb{V}(\sh{E}))$ is likewise identified with $\Hom_{\sh{O}_S}(\sh{E},\sh{A}(X))$ (where $\sh{A}(X)$ is considered as an \emph{$\sh{O}_S$-module});
as a result, we can canonically equip it with a \emph{module} structure over the ring $\Hom_S(X,S[T])$, and we see as above that the pair
\[
  (\Hom_S(X,S[T]),\Hom(X,\bb{V}(\sh{E})))
\]
is a contravariant functor in $X$, with values in the category whose elements are the pairs $(A,M)$ consisting of a ring $A$ and an $A$-module $M$, the morphisms being di-homomorphisms.

We will interpret these facts by saying that $S[T]$ is an \emph{$S$-scheme of rings} and that $\bb{V}(\sh{E})$ is an \emph{$S$-scheme of modules} on the $S$-scheme of rings $S[T]$ (cf.~Chapter~0,~\textsection8).
\end{env}

\begin{env}[1.7.14]
\label{II.1.7.14}
We will see that the structure of an $S$-scheme of modules defined on the $S$-scheme $\bb{V}(\sh{E})$ allows us to reconstruct the $\sh{O}_S$-module $\sh{E}$ up to unique isomorphism: for this, we will show that $\sh{E}$ is canonically isomorphic to an $\sh{O}_S$-submodule of $\bb{S}(\sh{E})=\sh{A}(\bb{V}(\sh{E}))$, defined by means of this structure.
Indeed \sref{II.1.7.4} the set $\Hom_{\sh{O}_S}(\bb{S}(\sh{E}),\sh{A}(X))$ of homomorphisms of \emph{$\sh{O}_S$-algebras} is canonically identified with $\Hom_{\sh{O}_S}(\sh{E},\sh{A}(X))$, the set of homomorphisms of \emph{$\sh{O}_S$-modules}: if $h$ and $h'$ are two elements of this latter set, $s_i$ ($1\leq i\leq n$) sections of $\sh{E}$ over an open $U\subset S$, $t$ a section of $\sh{A}(X)$ over $U$, then we have by definition
\[
  (h+h')(s_1 s_2\cdots s_n)=\prod_{i=1}^n(h(s_i)+h'(s_i))
\]
and
\[
  (t\cdot h)(s_1 s_2\cdots s_n)=t^n\prod_{i=1}^n h(s_i).
\]

This being so, if $z$ is a section of $\bb{S}(\sh{E})$ over $U$, then $h\mapsto h(z)$ is a map from $\Hom_S(X,\bb{V}(\sh{E}))=\Hom_{\sh{O}_S}(\bb{S}(\sh{E}),\sh{A}(X))$ to $\Gamma(U,\sh{A}(X))$.
We will
\oldpage[II]{19}
show that $\sh{E}$ is identified with a submodule of $\bb{S}(\sh{E})$ such that, \emph{for every open $U\subset S$, every section $z$ of this $\sh{O}_S$-submodule of $U$, and every $S$-prescheme $X$, the map $h\mapsto h(z)$ from $\Hom_{\sh{O}_S}(\bb{S}(\sh{E})|U,\sh{A}(X)|U)$ to $\Gamma(U,\sh{A}(X))$ is a homomorphism of $\Gamma(U,\sh{A}(X))$-modules}.

It is immediate that $\sh{E}$ has this property; to show the converse, we can reduce to proving that when $S=\Spec(A)$, $\sh{E}=\widetilde{M}$, a section $z$ of $\bb{S}(\sh{E})$ over $S$ that (for $U=S$) has the property stated above is necessarily a section of $\sh{E}$; we then have $z=\sum_{n=0}^\infty z_n$, where $z_n\in\bb{S}_n(M)$, and it is a question of proving that $z_n=0$ for $n\neq 1$.
Set $B=\bb{S}(M)$ and take for $X$ the prescheme $\Spec(B[T])$, where $T$ is an indeterminate.
The set $\Hom_{\sh{O}_S}(\bb{S}(\sh{E}),\sh{A}(X))$ identifies with the set of ring homomorphisms $h:B\to B[T]$ \sref[I]{I.1.3.13}, and from what we saw above, we have $(T\cdot h)(z)=\sum_{n=0}^\infty T^n h(z_n)$: the hypothesis on $z$ implies that we have $\sum_{n=0}^\infty T^n h(z_n)=T\cdot\sum_{n=0}^\infty h(z_n)$ for every homomorphism $h$.
In in particular we take for $h$ the canonical injection, then $\sum_{n=0}^\infty T^n z_n=T\cdot\sum_{n=0}^\infty z_n$, which implies the conclusion $z_n=0$ for $n\neq 1$.
\end{env}

\begin{proposition}[1.7.15]
\label{II.1.7.15}
Let $Y$ be a prescheme whose underlying space is Noetherian, or a quasi-compact scheme.
Every affine $Y$-scheme $X$ of finite type over $Y$ is $Y$-isomorphic to a closed $Y$-subscheme of a $Y$-scheme of the form $\bb{V}(\sh{E})$, where $\sh{E}$ is a quasi-coherent $\sh{O}_Y$-module of finite type.
\end{proposition}

\begin{proof}
The quasi-coherent $\sh{O}_Y$-algebra $\sh{A}(X)$ is of finite type \sref{II.1.3.7}.
The hypotheses imply that $\sh{A}(X)$ is generated by a quasi-coherent $\sh{O}_Y$-submodule of finite type $\sh{E}$ \sref[I]{I.9.6.5}; by definition, this implies that the canonical homomorphism $\bb{S}(\sh{E})\to\sh{A}(X)$ canonically extending the injection $\sh{E}\to\sh{A}(X)$ is \emph{surjective}; the conclusion then follows from \sref{II.1.4.10}.
\end{proof}


\section{Homogeneous prime spectra}
\label{section:II.2}

\subsection{Generalities on graded rings and modules}
\label{subsection:II.2.1}

\begin{notation}[2.1.1]
\label{II.2.1.1}
Given a \emph{positively graded} ring $S$, we denote by $S_n$ the subset of $S$ consisting of homogeneous elements of degree $n$ ($n\geq 0$), by $S_+$ the (direct) sum of the $S_n$ for $n>0$;
we have $1\in S_0$, $S_0$ is a subring of $S$, $S_+$ is a graded ideal of $S$, and $S$ is the direct sum of $S_0$ and $S_+$.
If $M$ is a \emph{graded} module over $S$ (with positive or negative degrees), we similarly denote by $M_n$ the $S_0$-module consisting of homogeneous elements of $M$ of degree $n$ (with $n\in\bb{Z}$).

For every integer $d>0$, we denote by $S^{(d)}$ the direct sum of the $S_{nd}$;
by considering the elements of $S_{nd}$ as homogeneous of degree $n$, the $S_{nd}$ define on $S^{(d)}$ a graded ring structure.

For every integer $k$ such that $0\leq k\leq d-1$, we denote by $M^{(d,k)}$ the direct sum
\oldpage[II]{20}
of the $M_{nd+k}$ ($n\in\bb{Z}$);
this is a graded $S^{(d)}$-module when we consider the elements of $M_{nd+k}$ as homogeneous of degree $n$.
We write $M^{(d)}$ in place of $M^{(d,0)}$.

With the above notation, for every integer $n$ (positive or negative), we denote by $M(n)$ the graded $S$-module defined by $(M(n))_k=M_{n+k}$ for every $k\in\bb{Z}$.
In particular, $S(n)$ will be a graded $S$-module such that $(S(n))_k=S_{n+k}$, by agreeing to set $S_n=0$ for $n<0$.
We say that a graded $S$-module $M$ is \emph{free} if it is isomorphic, considered as a \emph{graded} module, to a direct sum of modules of the form $S(n)$;
as $S(n)$ is a monogeneous $S$-module, generated by the element $1$ of $S$ considered as an element of degree $-n$, it is equivalent to say that $M$ admits a \emph{basis} over $S$ consisting of \emph{homogeneous} elements.

We say that a graded $S$-module $M$ \emph{admits a finite presentation} if there exists an exact sequence $P\to Q\to M\to 0$, where $P$ and $Q$ are finite direct sums of modules of the form $S(n)$ and the homomorphisms are of degree $0$ (cf.~\sref{II.2.1.2}).
\end{notation}

\begin{env}[2.1.2]
\label{II.2.1.2}
Let $M$ and $N$ be two graded $S$-modules;
we define on $M\otimes_S N$ a \emph{graded} $S$-module structure in the following way.
On the tensor product $M\otimes_\bb{Z}N$, we can define a graded $\bb{Z}$-module structure (where $\bb{Z}$ is graded by $\bb{Z}_0=\bb{Z}$, $\bb{Z}_n=0$ for $n\neq 0$) by setting $(M\otimes_\bb{Z}N)_q=\bigoplus_{m+n=q}M_m\otimes_\bb{Z}N_n$ (as $M$ and $N$ are respectively direct sums of the $M_m$ and the $N_n$, we know that we can canonically identify $M\otimes_\bb{Z}N$ with the direct sum of all the $M_m\otimes_\bb{Z}N_n$).
This being so, we have $M\otimes_S N=(M\otimes_\bb{Z}N)/P$, where $P$ is the $\bb{Z}$-submodule of $M\otimes_\bb{Z}N$ generated by the elements $(xs)\otimes y-x\otimes(sy)$ for $x\in M$, $y\in N$, $s\in S$;
it is clear that $P$ is a \emph{graded} $\bb{Z}$-submodule of $M\otimes_\bb{Z}N$, and we see immediately that we obtain a graded $S$-module structure on $M\otimes_S N$ by passing to the quotient.

For two graded $S$-modules $M$ and $N$, recall that a homomorphism $u:M\to N$ of $S$-modules is said to be \emph{of degree $k$} if $u(M_j)\subset N_{j+k}$ for all $j\in\bb{Z}$.
If $H_n$ denotes the set of all the homomorphisms of degree $n$ from $M$ to $N$, then we denote by $\Hom_S(M,N)$ the (direct) \emph{sum} of the $H_n$ ($n\in\bb{Z}$) in the $S$-module $H$ of all the homomorphisms (of $S$-modules) from $M$ to $N$;
in general, $\Hom_S(M,N)$ is not equal to the later.
However, we have $H=\Hom_S(M,N)$ when $M$ is \emph{of finite type};
indeed, we can then suppose that $M$ is generated by a finite number of homogeneous elements $x_i$ ($1\leq i\leq n$), and every homomorphism $u\in H$ can be written in a unique way as $\sum_{k\in\bb{Z}}u_k$, where for each $k$, $u_k(x_i)$ is equal to the homogeneous component of degree $k+\deg(x_i)$ of $u(x_i)$ ($1\leq i\leq n$), which implies that $u_k=0$ except for a finite number of indices;
we have by definition that $u_k\in H_k$, hence the conclusion.

We say that the elements of degree $0$ of $\Hom_S(M,N)$ are the \emph{homomorphisms of graded $S$-modules}.
It is clear that $S_m H_n\subset H_{m+n}$, so the $H_n$ define on $\Hom_S(M,N)$ a graded $S$-module structure.

It follows immediately from these definitions that we have
\[
\label{II.2.1.2.1}
  M(m)\otimes_S N(n)=(M\otimes_S N)(m+n),
\tag{2.1.2.1}
\]
\[
\label{II.2.1.2.2}
  \Hom_S(M(m),N(n))=(\Hom_S(M,N))(n-m),
\tag{2.1.2.2}
\]
for two graded $S$-modules $M$ and $N$.

\oldpage[II]{21}
Let $S$ and $S'$ be two graded rings;
a homomorphism of \emph{graded rings $\varphi:S\to S'$} is a homomorphism of rings such that $\varphi(S_n)\subset S_n'$ for all $n\in\bb{Z}$ (in other words, $\varphi$ must be a homomorphism \emph{of degree $0$} of graded $\bb{Z}$-modules).
The data of such a homomorphism defines on $S'$ a \emph{graded} $S'$-module structure;
equipped with this structure and its graded ring structure, we say that $S'$ is a \emph{graded $S'$-algebra}.

If $M$ is also a graded $S$-module, then the tensor product $M\otimes_S S'$ of \emph{graded} $S$-modules is equipped in a natural way with a \emph{graded} $S'$-module structure, the grading being defined as above.
\end{env}

\begin{lemma}[2.1.3]
\label{II.2.1.3}
Let $S$ be a ring graded in positive degrees.
For a subset $E$ of $S_+$ consisting of homogeneous elements to generate $S_+$ as an $S$-module, it is necessary and sufficient for $E$ to generate $S$ an an $S_0$-algebra.
\end{lemma}

\begin{proof}
The condition is evidently sufficient; we show that it is necessary.
Let $E_n$ (resp. $E^n$) be the set of elements of $E$ equal to $n$ (resp. $\leq n$);
it suffices to show, by induction on $n>0$, that $S_n$ is the $S_0$-module generated by the elements of degree $n$ which are products of elements of $E^n$.
This is evident for $n=1$ by virtue of the hypothesis;
the latter also shows that $S_n=\sum_{p=0}^{n-1}S_p E_{n-p}$, and the induction argument is then immediate.
\end{proof}

\begin{corollary}[2.1.4]
\label{II.2.1.4}
For $S_+$ to be an ideal of finite type, it is necessary and sufficient for $S$ to be an $S_0$-algebra of finite type.
\end{corollary}

\begin{proof}
We can always assume that a finite system of generators of the $S_0$-algebra $S$ (resp. of the $S$-ideal $S_+$) consists of homogeneous elements, by replacing each of the generators considered by its homogeneous components.
\end{proof}

\begin{corollary}[2.1.5]
\label{II.2.1.5}
For $S$ to be Noetherian, it is necessary and sufficient for $S_0$ to be Noetherian and for $S$ to be an $S_0$-algebra of finite type.
\end{corollary}

\begin{proof}
The condition is evidently sufficient;
it is necessary, since $S_0$ is isomorphic to $S/S_+$ and $S_+$ must be an ideal of finite type \sref{II.2.1.4}.
\end{proof}

\begin{lemma}[2.1.6]
\label{II.2.1.6}
Let $S$ be a ring graded in positive degrees, which is an $S_0$-algebra of finite type.
Let $M$ be a graded $S$-module of finite type.
Then:
\begin{enumerate}
  \item[{\rm(i)}] The $M_n$ are $S_0$-modules of finite type, and there exists an integer $n_0$ such that $M_n=0$ for $n\leq n_0$.
  \item[{\rm(ii)}] There exists an integer $n_1$ and an integer $h>0$ such that, for every integer $n\geq n_1$, we have $M_{n+h}=S_h M_n$.
  \item[{\rm(iii)}] For every pair of integers $(d,k)$ such that $d>0$, $0\leq k\leq d-1$, $M^{(d,k)}$ is an $S^{(d)}$-module of finite type.
  \item[{\rm(iv)}] For every integer $d>0$, $S^{(d)}$ is an $S_0$-algebra of finite type.
  \item[{\rm(v)}] There exists an integer $h>0$ such that $S_{mh}=(S_h)^m$ for all $m>0$.
  \item[{\rm(vi)}] For every integer $n>0$, there exists an integer $m_0$ such that $S_m\subset S_+^n$ for all $m\geq m_0$.
\end{enumerate}
\end{lemma}

\begin{proof}
We can assume that $S$ is generated (as an $S_0$-algebra) by homogeneous elements $f_i$, of degrees $h_i$ ($1\leq i\leq r$), and $M$ is generated (as an $S$-module) by homogeneous elements $x_j$ of degrees $k_j$ ($1\leq j\leq s$).
It is clear that $M_n$ is formed by linear combinations,
\oldpage[II]{22}
with coefficients in $S_0$, of elements $f_1^{\alpha_1}\cdots f_r^{\alpha_r}x_j$ such that the $\alpha_i$ are integers $\geq 0$ satisfying $k_j+\sum_i\alpha_i h_i=n$;
for each $j$, there are only finitely many systems $(\alpha_i)$ satisfying this equation, since the $h_i$ are $>0$, hence the first assertion of (i);
the second is evident.
On the other hand, let $h$ be the l.c.m. of the $h_i$ and set $g_i=f_i^{h/h_i}$ ($1\leq i\leq r$) such that all the $g_i$ are of degree $h$;
let $z_\mu$ be the elements of $M$ of the form $f_1^{\alpha_1}\cdots f_r^{\alpha_r}x_j$ with $0\leq\alpha_i<h/h_i$ for $1\leq i\leq r$;
there are finitely many of these elements, so let $n_1$ be the largest of their degrees.
It is clear that for $n\geq n_1$, every element of $M_{n+h}$ is a linear combination of the $z_\mu$ whose coefficients are monomials of degree $>0$ with respect to the $g_i$, so we have $M_{n+h}=S_h M_n$, which establishes (ii).
In a similar way, we see (for all $d>0$) that an element of $M^{(d,k)}$ is a linear combinations, with coefficients in $S_0$, of elements of the form $g^d f_1^{\alpha_1}\cdots f_r^{\alpha_r}x_j$ with $0\leq\alpha_i<d$, $g$ being a homogeneous element of $S$;
hence (iii);
(iv) then follows from (iii) and from Lemma~\sref{II.2.1.3}, by taking $M=S_+$, since $(S_+)^{(d)}=(S^{(d)})_+$.
The assertion of (v) is deduced from (ii) by taking $M=S$.
Finally, for a given $n$, there are finitely many systems $(\alpha_i)$ such that $\alpha_i\geq 0$ and $\sum_i\alpha_i<n$, so if $m_0$ is the largest value of the sum $\sum_i\alpha_i h_i$ of these systems, then we have $S_m\subset S_+^n$ for $m>m_0$, which proves (vi).
\end{proof}

\begin{corollary}[2.1.7]
\label{II.2.1.7}
If $S$ is Noetherian, then so is $S^{(d)}$ for every integer $d>0$.
\end{corollary}

\begin{proof}
This follows from \sref{II.2.1.5} and \sref{II.2.1.6}[iv].
\end{proof}

\begin{env}[2.1.8]
\label{II.2.1.8}
Let $\mathfrak{p}$ be a \emph{graded} prime ideal of the graded ring $S$;
$\mathfrak{p}$ is thus a direct sum of the subgroups $\mathfrak{p}_n=\mathfrak{p}\cap S_n$.
Suppose that \emph{$\mathfrak{p}$ does not contain $S_+$}.
Then if $f\in S_+$ is not in $\mathfrak{p}$, the relation $f^n x\in\mathfrak{p}$ is equivalent to $x\in\mathfrak{p}$;
in particular, if $f\in S_d$ ($d>0$), for all $x\in S_{m-nd}$, then the relation $f^n x\in\mathfrak{p}_m$ is equivalent to $x\in\mathfrak{p}_{m-nd}$.
\end{env}

\begin{proposition}[2.1.9]
\label{II.2.1.9}
Let $n_0$ be an integer $>0$;
for all $n\geq n_0$, let $\mathfrak{p}_n$ be a subgroup of $S_n$.
For there to exist a graded prime ideal $\mathfrak{p}$ of $S$ not containing $S_+$ and such that $\mathfrak{p}\cap S_n=\mathfrak{p}_n$ for all $n\geq n_0$, it is necessary and sufficient for the following coniditions to be satisfied:
\begin{enumerate}
  \item[{\rm(1st)}] $S_m\mathfrak{p}_n\subset\mathfrak{p}_{m+n}$ for all $m\geq 0$ and all $n\geq n_0$.
  \item[{\rm(2nd)}] For $m\geq n_0$, $n\geq n_0$, $f\in S_m$, $g\in S_n$, the relation $fg\in\mathfrak{p}_{m+n}$ implies $f\in\mathfrak{p}_m$ or $g\in\mathfrak{p}_n$.
  \item[{\rm(3rd)}] $\mathfrak{p}_n\neq S_n$ for at least one $n\geq n_0$.
\end{enumerate}
In addition, the graded prime ideal $\mathfrak{p}$ is then unique.
\end{proposition}

\begin{proof}
It is evident that the conditions (1st) and (2nd) are necessary.
In addition, if $\mathfrak{p}\not\supset S_+$, then there exists at least one $k>0$ such that $\mathfrak{p}\cap S_k\neq S_k$;
if $f\in S_k$ is not in $\mathfrak{p}$, the relation $\mathfrak{p}\cap S_n=S_n$ implies $\mathfrak{p}\cap S_{n-mk}=S_{n-mk}$ according to \sref{2.2.1.8};
therefore, if $\mathfrak{p}\cap S_n=S_n$ for a certain value of $n$, we would have $\mathfrak{p}\supset S_+$ contrary to the hypothesis, which proves that (3rd) is necessary.
Conversely, suppose that the conditions (1st), (2nd), and (3rd) are satisfied.
Note that if for an integer $d\geq n_0$, $f\in S_d$ is not in $\mathfrak{p}_d$, then, if $\mathfrak{p}$ exists, $\mathfrak{p}_m$, for $m<n_0$, is necessarily equal to the set of the $x\in S_m$ such that $f^r x\in\mathfrak{p}_{m+rd}$, except for a finite number of values of $r$.
This already proves that if $\mathfrak{p}$ exists, then it is unique.
It remains to show that if we define the $\mathfrak{p}_m$ for $m<n_0$ by the previous condition, then $\mathfrak{p}=\sum_{n=0}^\infty\mathfrak{p}_n$ is a prime ideal.
First, note that by virtue of (2nd), for $m\geq n_0$, $\mathfrak{p}_m$ is also defined as the set of the $x\in S_m$ such that $f^r x\in\mathfrak{p}_{m+rd}$ except for a finite number of values of $r$.
This
\oldpage[II]{23}
being so, if $g\in S_m$, $x\in\mathfrak{p}_n$, then we have $f^r gx\in\mathfrak{p}_{m+n+rd}$ except for a finite number of values of $r$, so $gx\in\mathfrak{p}_{m+n}$, which proves that $\mathfrak{p}$ is an ideal of $S$.
To establish that this ideal is prime, in other words that the ring $S/\mathfrak{p}$, graded by the subgroups $S_n/\mathfrak{p}_n$, is an integral domain, it suffices (by considering the components of higher degree of two elements of $S/\mathfrak{p}$) to prove that if $x\in S_m$ and $y\in S_n$ are such that $x\not\in\mathfrak{p}_m$ and $y\not\in\mathfrak{p}_n$, then $xy\not\in\mathfrak{p}_{m+n}$.
If not, for $r$ large enough, we would have $f^{2r}xy\in\mathfrak{p}_{m+n+2rd}$;
but we have $f^r y\not\in\mathfrak{p}_{n+rd}$ for all $r>0$;
it then follows from (2nd) that, except for a finite number of values of $r$, we have $f^r x\in\mathfrak{p}_{m+rd}$, and we conclude that $x\in\mathfrak{p}_m$ contrary to the hypothesis.
\end{proof}

\begin{env}[2.1.10]
\label{II.2.1.10}
We say that a subset $\mathfrak{J}$ of $S_+$ is an \emph{ideal of $S_+$} if it is an ideal of $S$, and $\mathfrak{J}$ is a \emph{graded prime ideal of $S_+$} if it is the intersection of $S_+$ and a graded prime ideal of $S$ \emph{not containing $S_+$} (this prime ideal is also unique according to Proposition~\sref{II.2.1.9}).
If $\mathfrak{J}$ is an ideal of $S_+$, the \emph{radical of $\mathfrak{J}$ in $S_+$} is the set of elements of $S_+$ which have a power in $\mathfrak{J}$, in other words the set $\rad_+(\mathfrak{J})=\rad(\mathfrak{J})\cap S_+$;
in particular, the radical of $0$ in $S_+$ is then called the \emph{nilradical} of $S_+$ and denoted by $\nilrad_+$: this is the set of nilpotent elements of $S_+$.
If $\mathfrak{J}$ is an \emph{graded} ideal of $S_+$, then its radical $\rad_+(\mathfrak{J})$ is a \emph{graded} ideal: by passing to the quotient ring $S/\mathfrak{J}$, we can reduce to the case $\mathfrak{J}=0$, and it remains to see that if $x=x_h+x_{h+1}+\cdots+x_k$ is nilpotent, then so are the $x_i\in S_i$ ($1\leq h\leq i\leq k$);
we can assume $x_k\neq 0$ and the component of highest degree of $x^n$ is then $x_k^n$, hence $x_k$ is nilpotent, and we then argue by induction on $k$.
We say that the graded ring $S$ is \emph{essentially reduced} if $\nilrad_+=0$, in other words, if $S_+$ does not contain nilpotent elements $\neq 0$.
\end{env}

\begin{env}[2.1.11]
\label{II.2.1.11}
We note that if, in the graded ring $S$, an element $x$ is a zero-divisor, then so is its component of highest degree.
We say that a ring $S$ is \emph{essentially integral} if the ring $S_+$ (\emph{without the unit element}) does not contain a zero-divisor and is $\neq 0$;
it suffices that a homogeneous element $\neq 0$ in $S_+$ is not a zero-divisor in this ring.
It is clear that if $\mathfrak{p}$ is a graded prime ideal of $S_+$, then $S/\mathfrak{p}$ is essentially integral.

Let $S$ be an essentially integral graded ring, and let $x_0\in S_0$:
if there then exists \emph{a} homogeneous element $f\neq 0$ of $S_+$ such that $x_0 f=0$, then we have $x_0 S_+=0$, since we have $(x_0 g)f=(x_0 f)g=0$ for all $g\in S_+$, and the hypothesis thus implies $x_0 g=0$.
For $S$ to be integral, it is necessary and sufficient for $S_0$ to be integral and the annihilator of $S_+$ in $S_0$ to be $0$.
\end{env}


\subsection{Rings of fractions of a graded ring}
\label{subsection:II.2.2}

\begin{env}[2.2.1]
\label{II.2.2.1}
Let $S$ be a graded ring, in positive degrees, $f$ a \emph{homogeneous} element of $S$, of degree $d>0$;
then the ring of fractions $S'=S_f$ is graded, taking for $S_n'$ the set of the $x/f^k$, where $x\in S_{n+kd}$ with $k\geq 0$ (we observe here that $n$ can take arbitrary negative values);
we denote the subring $S_0'=(S_f)_0$ of $S'$ consisting of elements \emph{of degree $0$} by the notation $S_{(f)}$.

If $f\in S_d$, then the monomials $(f/1)^h$ in $S_f$ ($h$ a positive or negative integer) form a \emph{free system} over the ring $S_{(f)}$, and the set of their linear combinations is none other than
\oldpage[II]{24}
the ring $(S^{(d)})_f$, which is thus \emph{isomorphic to $S_{(f)}[T,T^{-1}]=S_{(f)}\otimes_\bb{Z}\bb{Z}[T,T^{-1}]$} (where $T$ is an indeterminate).
Indeed, if we have a relation $\sum_{h=-a}^b z_h(f/1)^h=0$ with $z_h=x_h/f^m$, where the $x_h$ are in $S_{md}$, then this relation is equivalent by definition to the existence of a $k>-a$ such that $\sum_{h=-a}^b f^{h+k}x_h=0$, and as the degrees of the terms of this sum are distinct, we have $f^{h+k}x_h=0$ for all $h$, hence $z_h=0$ for all $h$.

If $M$ is a graded $S$-module, then $M'=M_f$ is a graded $S_f$-module, $M_n'$ being the set of the $z/f^k$ with $z\in M_{n+kd}$ ($k\geq 0$);
we denote by $M_{(f)}$ the set of the homomogenous elements of degree $0$ of $M'$;
it is immediate that $M_{(f)}$ is an $S_{(f)}$-module and that we have $(M^{(d)})_f=M_{(f)}\otimes_{S_{(f)}}(S^{(d)})_f$.
\end{env}

\begin{lemma}[2.2.2]
\label{II.2.2.2}
Let $d$ and $e$ be integers $>0$, $f\in S_d$, $g\in S_e$.
There exists a canonical ring isomorphism
\[
  S_{(fg)}\isoto(S_{(f)})_{g^d/f^e};
\]
if we canonically identify these two rings, then there exists a canonical module isomorphism
\[
  M_{(fg)}\isoto(M_{(f)})_{g^d/f^e}.
\]
\end{lemma}

\begin{proof}
Indeed, $fg$ divides $f^e g^d$, and this latter element divides $(fg)^{de}$, so the graded rings $S_{fg}$ and $S_{f^e g^d}$ are canonically identified;
on the other hand, $S_{f^e g^d}$ also identifies with $(S_{f^e})_{g^d/1}$ \sref[0]{0.1.4.6}, and as $f^e/1$ is invertible in $S_{f^e}$, $S_{f^e g^d}$ also identifies with $(S_{f^e})_{g^d/f^e}$.
The element $g^d/f^e$ is of degree $0$ in $S_{f^e}$;
we immediately conclude that the subring of $(S_{f^e})_{g^d/f^e}$ consisting of elements of degree $0$ is $(S_{(f^e)})_{g^d/f^e}$, and as we evidently have $S_{(f^e)}=S_{(f)}$, this proves the first part of the proposition;
the second is established in a similar way.
\end{proof}

\begin{env}[2.2.3]
\label{II.2.2.3}
Under the hypotheses of \sref{II.2.2.2}, it is clear that the canonical homomorphism $S_f\to S_{fg}$ \sref[0]{0.1.4.1}, which sends $x/f^k$ to $g^k x/(fg)^k$, is of degree $0$, thus gives by restriction a \emph{canonical homomorphism $S_{(f)}\to S_{(fg)}$}, such that the diagram
\[
  \xymatrix{
    & S_{(f)}\ar[dl]\ar[dr]\\
    S_{(fg)}\ar[rr]^-{\sim} & &
    (S_{(f)})_{g^d/f^e}
  }
\]
is commutative.
We similarly define a canonical homomorphism $M_{(f)}\to M_{(fg)}$.
\end{env}

\begin{lemma}[2.2.4]
\label{II.2.2.4}
If $f$ and $g$ are two homogeneous elements of $S_+$, then the ring $S_{(fg)}$ is generated by the union of the canonical images of $S_{(f)}$ and $S_{(g)}$.
\end{lemma}

\begin{proof}
By virtue of Lemma~\sref{II.2.2.2}, it suffices to see that $1/(g^d/f^e)=f^{d+e}/(fg)^d$ belongs to the canonical image of $S_{(g)}$ in $S_{(fg)}$, which is evident by definition.
\end{proof}

\begin{proposition}[2.2.5]
\label{II.2.2.5}
Let $d$ be an integer $>0$ and let $f\in S_d$.
Then there exists a canonical ring isomorphisms $S_{(f)}\isoto S^{(d)}/(f-1)S^{(d)}$;
if we identify these two rings by this isomorphism, then there exists a canonical module isomorphism $M_{(f)}\isoto M^{(d)}/(f-1)M^{(d)}$.
\end{proposition}

\begin{proof}
The first of these isomorphisms is defined by sending $x/f^n$, where $x\in S_{nd}$, to the element $\overline{x}$, the class of $x\text{ mod. }(f-1)S^{(d)}$;
this map is well-defined, because we have the congruence $f^h x\equiv x\,(\text{mod.}\,(f-1)S^{(d)})$ for all $x\in S^{(d)}$, so if $f^h x=0$ for an $h>0$,
\oldpage[II]{25}
then we have $\overline{x}=0$.
On the other hand, if $x\in S_{nd}$ is such that $x=(f-1)y$ with $y=y_{hd}+y_{(h+1)d}+\cdots+y_{kd}$ with $y_{jd}\in S_{jd}$ and $y_{hd}\neq 0$, then we necessarily have $h=n$ and $x=-y_{hd}$, as well as the relations $y_{(j+1)d}=fy_{jd}$ for $h\leq j\leq k-1$, $fy_{kd}=0$, which ultimately gives $f^{k-n}x=0$;
we send every class $\overline{x}\text{ mod. }(f-1)S^{(d)}$ of an element $x\in S_{nd}$ to the element $x/f^n$ of $S_{(f)}$, since the preceding remark shows that this map is well-defined.
It is immediate that these two maps thus defined are ring homomorphisms, each the reciprocal of the other.
We proceed exactly the same way for $M$.
\end{proof}

\begin{corollary}[2.2.6]
\label{II.2.2.6}
If $S$ is Noetherian, then so is $S_{(f)}$ for $f$ homogeneous of degree $>0$.
\end{corollary}

\begin{proof}
This follows immediately from Corollary~\sref{II.2.1.7} and Proposition~\sref{II.2.2.5}.
\end{proof}

\begin{env}[2.2.7]
\label{II.2.2.7}
Let $T$ be a multiplicative subset of $S_+$ consisting of \emph{homogeneous} elements;
$T_0=T\cup\{1\}$ is then a multiplicative subset of $S$;
as the elements of $T_0$ are homogeneous, the ring $T_0^{-1}S$ is still graded in the evident way;
we denote by $S_{(T)}$ the subring of $T_0^{-1}S$ consisting of elements of order $0$, that is to say, the elements of the form $x/h$, where $h\in T$ and $x$ is homogeneous of degree equal to that of $h$.
We know \sref[0]{0.1.4.5} that $T_0^{-1}S$ is canonically identified with the inductive limit of the rings $S_f$, where $f$ varies over $T$ (with respect to the canonical homomorphisms $S_f\to S_{fg}$);
as this identification respects the degrees, it identifies $S_{(T)}$ with the \emph{inductive limit} of the $S_{(f)}$ for $f\in T$.
For every graded $S$-module $M$, we similarly define the module $M_{(T)}$ (over the ring $S_{(T)}$) consisting of elements of degree $0$ of $T_0^{-1}M$, and we see that this module is the inductive limit of the $M_{(f)}$ for $f\in T$.

If $\mathfrak{p}$ is a graded prime ideal of $S_+$, then we denote by $S_{(\mathfrak{p})}$ and $M_{(\mathfrak{p})}$ the ring $S_{(T)}$ and the module $M_{(T)}$ respectively, where $T$ is the set of \emph{homogeneous} elements of $S_+$ which do not belong to $\mathfrak{p}$.\end{env}


\subsection{Homogeneous prime spectrum of a graded ring}
\label{subsection:II.2.3}

\begin{env}[2.3.1]
\label{II.2.3.1}
Given a graded ring $S$, in positive degrees, we call the \emph{homogeneous prime spectrum} of $S$ and denote it by $\Proj(S)$ the set of graded prime ideals of $S_+$ \sref{II.2.1.10}, or equivalently the set of graded prime ideals of $S$ \emph{not containing $S_+$};
we will define a \emph{scheme} structure having $\Proj(S)$ as the underlying set.
\end{env}

\begin{env}[2.3.2]
\label{II.2.3.2}
For every subset $E$ of $S$, let $V_+(E)$ be the set of graded prime ideals of $S$ containing $S$ and not containing $S_+$;
this is thus the subset $V(E)\cap\Proj(S)$ of $\Spec(S)$.
From \sref[I]{I.1.1.2} we deduce:
\[
\label{II.2.3.2.1}
  V_+(0)=\Proj(S),\ V_+(S)=V_+(S_+)=\emp,
\tag{2.3.2.1}
\]
\[
\label{II.2.3.2.2}
  V_+\big(\textstyle\bigcup_\lambda E_\lambda\big)=\textstyle\bigcap_\lambda V_+(E_\lambda),
\tag{2.3.2.2}
\]
\[
\label{II.2.3.2.3}
  V_+(EE')=V_+(E)\cup V_+(E').
\tag{2.3.2.3}
\]

We do not change $V_+(E)$ by replacing $E$ with the graded ideal generated by $E$;
in addition, if $\mathfrak{J}$ is a graded ideal of $S$, then we have
\[
  V_+(\mathfrak{J})=V_+\big(\textstyle\bigcup_{q\geq n}(\mathfrak{J}\cap S_q)\big)
\tag{2.3.2.4}
\]
\oldpage[II]{26}
for all $n>0$: indeed, if $\mathfrak{p}\in\Proj(S)$ contains the homogeneous elements of $\mathfrak{J}$ of degree $\geq n$, then as by hypothesis there exists a homogeneous element $f\in S_d$ not contained in $\mathfrak{p}$, for every $m\geq 0$ and every $x\in S_m\cap\mathfrak{J}$, we have $f^r x\in\mathfrak{J}\cap S_{m+rd}$ for all but finitely many values of $r$, so $f^r x\in\mathfrak{p}\cap S_{m+rd}$, which implies that $x\in\mathfrak{p}\cap S_m$ \sref{II.2.1.9}.

Finally, we have, for every graded ideal $\mathfrak{J}$ of $S$,
\[
  V_+(\mathfrak{J})=V_+(\rad_+(\mathfrak{J})).
\tag{2.3.2.5}
\]
\end{env}

\begin{env}[2.3.3]
\label{II.2.3.3}
By definition, the $V_+(E)$ are the closed subsets of $X=\Proj(S)$ for the topology induced by the spectral topology of $\Spec(S)$, which we also call the \emph{spectral topology} on $X$.
For all $f\in S$, we set
\[
\label{II.2.3.3.1}
  D_+(f) = D(f)\cap\Proj(S) = \Proj(S)\setmin V_+(f)
\tag{2.3.3.1}
\]
and so, for any two elements $f$ and $g$ of $S$ \sref[I]{I.1.1.9.1},
\[
\label{II.2.3.3.2}
  D_+(fg) = D_+(f)\cap D_+(g).
\tag{2.3.3.2}
\]
\end{env}

\begin{proposition}[2.3.4]
\label{II.2.3.4}
The $D_+(f)$, as $f$ runs over the set of homogeneous elements of $S_+$, form a base for the topology of $X=\Proj(S)$.
\end{proposition}

\begin{proof}
It follows from \sref{II.2.3.2.2} and \sref{II.2.3.2.4} that every closed subset of $X$ is the intersection of sets of the form $V_+(f)$, where $f$ is homogeneous of degree $>0$.
\end{proof}

\begin{env}[2.3.5]
\label{II.2.3.5}
Let $f$ be a \emph{homogeneous} element of $S_+$, of degree $d>0$;
for every graded prime ideal $\mathfrak{p}$ of $S$ that does not contain $f$, we know that the set of the $x/f^n$, where $x\in\mathfrak{p}$ and $n\geq0$, is a prime ideal of the ring of fractions $S_f$ \sref[0]{0.1.2.6};
its intersection with $S_{(f)}$ is thus a prime ideal of $S_{(f)}$, which we denote by $\psi_f(\mathfrak{p})$:
it is the set of the $x/f^n$ for $n\geq0$ and $x\in\mathfrak{p}\cap S_{nd}$.
We have thus defined a map
\[
  \psi_f: D_+(f)\to\Spec(S_{(f)});
\]
furthermore, if $g\in S_e$ is another homogeneous element of $S_+$, then we have a commutative diagram
\[
\label{II.2.3.5.1}
  \xymatrix{
    D_+(f) \ar[r]^{\psi_f}
    & \Spec(S_{(f)})
  \\D_+(fg) \ar[u] \ar[r]_{\psi_{fg}}
    & \Spec(S_{(fg)}) \ar[u]
  }
\tag{2.3.5.1}
\]
where the vertical arrow on the left is the inclusion, and the vertical arrow on the right is the map ${}^a\!\omega_{fg,f}$ induced by the canonical homomorphism $\omega=\omega_{fg,f}: S_{(f)}\to S_{(fg)}$ \sref[I]{I.1.2.1}.
Indeed, if $x/f^n\in\omega^{-1}(\psi_{fg}(\mathfrak{p}))$, with $fg\not\in\mathfrak{p}$, then, by definition, $g^nx/(fg)^n\in\psi_{fg}(\mathfrak{p})$, so $g^nx\in\mathfrak{p}$, and so $x\in\mathfrak{p}$;
the converse is evident.
\end{env}

\begin{proposition}[2.3.6]
\label{II.2.3.6}
The map $\psi_f$ is a homeomorphism from $D_+(f)$ to $\Spec(S_{(f)})$.
\end{proposition}

\begin{proof}
Firstly, $\psi_f$ is continuous;
this is since, if $h\in S_{nd}$ is such that $h/f^n\in\psi_f(\mathfrak{p})$, then, by definition, $h\in\mathfrak{p}$, and conversely, and so $\psi_f^{-1}(D(h/f^n))=D_+(hf)$, and our claim then follows from \sref{II.2.3.3.2}.
Furthermore, the $D_+(hf)$, where $h$ runs over the sets $S_{nd}$, form a topology of $D_+(f)$, by \sref{II.2.3.4} and \sref{II.2.3.3.2};
the
\oldpage[II]{27}
above thus proves, taking into account the ($T_0$) axiom, which holds in $D_+(f)$ and in $\Spec(S_{(f)})$, that $\psi_f$ is injective and that the inverse map $\psi_f(D_+(f))\to D_+(f)$ is continuous.
Finally, to see that $\psi_f$ is surjective, we note that, if $\mathfrak{q}_0$ is a prime ideal of $S_{(f)}$, and if, for all $n>0$, we denote by $\mathfrak{p}_n$ the set of $x\in S_n$ such that $x^d/f^n\in\mathfrak{q}_0$, then the $\mathfrak{p}_n$ satisfy the conditions of \sref{II.2.1.9}:
if $x,y\in S_n$ are such that $x^d/f^n,y^d/f^n\in\mathfrak{q}_0$, then $(x+y)^{2d}/f^{2n}\in\mathfrak{q}_0$, whence $(x+y)^d/f^n\in\mathfrak{q}_0$, since $\mathfrak{q}_0$ is prime;
this proves that the $\mathfrak{p}_n$ are subgroups of the $S_n$, and the verification of the other conditions of \sref{II.2.1.9} is immediate, taking into account the fact that $\mathfrak{q}_0$ is prime.
If $\mathfrak{p}$ is the graded prime ideal of $S$ thus defined, then indeed $\psi_f(\mathfrak{p})=\mathfrak{q}_0$, since, if $x\in S_{nd}$, then having $x/f^n\in\mathfrak{q}_0$ and $x^d/f^{nd}\in\mathfrak{q}_0$ is equivalent to $\mathfrak{q}_0$ being prime.
\end{proof}

\begin{corollary}[2.3.7]
\label{II.2.3.7}
To have $D_+(f)=\emp$, it is necessary and sufficient for $f$ to be nilpotent.
\end{corollary}

\begin{proof}
To have $\Spec(S_{(f)})=\emp$, it is necessary and sufficient to have $S_{(f)}=0$, or indeed to have $1=0$ in $S_f$, which means, by definition, that $f$ is nilpotent.
\end{proof}

\begin{corollary}[2.3.8]
\label{II.2.3.8}
Let $E$ be a subset of $S_+$.
Then the following conditions are equivalent:
\begin{enumerate}
  \item[{\rm(a)}] $V_+(E) = X = \Proj(S)$.
  \item[{\rm(b)}] Every element of $E$ is nilpotent.
  \item[{\rm(c)}] The homogeneous components of every element of $E$ are nilpotent.
\end{enumerate}
\end{corollary}

\begin{proof}
It is clear that (c) implies (b), and that (b) implies (a).
If $\mathfrak{J}$ is the graded ideal of $S$ generated by $E$, then condition~(a) is equivalent to requiring that $V_+(\mathfrak{J})=X$;
\emph{a fortiori}, (a) implies that every homogeneous element $f\in\mathfrak{J}$ is such that $V_+(f)=X$, and so $f$ is nilpotent by \sref{II.2.3.7}.
\end{proof}

\begin{corollary}[2.3.9]
\label{II.2.3.9}
If $\mathfrak{J}$ is a graded ideal of $S_+$, then $\mathfrak{r}_+(\mathfrak{J})$ is the intersection of the graded prime ideals of $S_+$ that contain $\mathfrak{J}$.
\end{corollary}

\begin{proof}
By considering the graded ring $S/\mathfrak{J}$, we can reduce to the case where $\mathfrak{J}=0$.
We need to prove that, if $f\in S_+$ is not nilpotent, then there exists a graded prime ideal of $S$ that does not contain $f$;
but at least one of the homogeneous components of $f$ is not nilpotent, and we can thus suppose $f$ to be homogeneous;
the claim then follows from \sref{II.2.3.7}.
\end{proof}

\begin{env}[2.3.10]
\label{II.2.3.10}
For every subset $Y$ of $X=\Proj(S)$, let $\mathfrak{j}_+(Y)$ be the set of $f\in S_+$ such that $Y\subset V_+(f)$;
this is equivalent to saying that $\mathfrak{j}_+(Y)=\mathfrak{j}(Y)\cap S_+$;
then $\mathfrak{j}_+(Y)$ is an ideal of $S_+$ that is equal to its radical in $S_+$.
\end{env}

\begin{proposition}[2.3.11]
\label{II.2.3.11}
\begin{enumerate}
  \item[{\rm(i)}] For every subset $E$ of $S_+$, $\mathfrak{j}_+(V_+(E))$ is the radical in $S_+$ of the graded ideal of $S_+$ generated by $E$.
  \item[{\rm(ii)}] For every subset $Y$ of $X$, $V_+(\mathfrak{j}_+(Y))=\overline{Y}$, where $\overline{Y}$ is the closure of $Y$ in $X$.
\end{enumerate}
\end{proposition}

\begin{proof}
\begin{enumerate}
  \item[{\rm(i)}] If $\mathfrak{J}$ is the graded ideal of $S_+$ generated by $E$, then $V_+(E)=V_+(\mathfrak{J})$, and the claim then follows from \sref{II.2.3.9}.
  \item[{\rm(ii)}] Since $V_+(\mathfrak{J})=\bigcap_{f\in\mathfrak{J}}V_+(f)$, having $Y\subset V_+(\mathfrak{J})$ implies that $Y\subset V_+(f)$ for every $f\in\mathfrak{J}$, and thus $\mathfrak{j}_+(Y)\supset\mathfrak{J}$, whence $V_+(\mathfrak{j}_+(Y))\subset V_+(\mathfrak{J})$, which proves (ii) by the definition of the closed subsets.
\end{enumerate}
\end{proof}

\begin{corollary}[2.3.12]
\label{II.2.3.12}
The closed subsets $Y$ of $X=\Proj(S)$ are in bijective correspondence with the graded ideals of $S_+$ that are equal to their radical in $S_+$, via the inclusion-reversing maps $Y\mapsto\mathfrak{j}_+(Y)$ and $\mathfrak{J}\mapsto V_+(\mathfrak{J})$;
the union $Y_1\cup Y_2$ of two closed subsets of $X$ corresponds
\oldpage[II]{28}
to $\mathfrak{j}_+(Y_1)\cap\mathfrak{j}_+(Y_2)$, and the intersection of an arbitrary family $(Y_\lambda)$ of closed subsets corresponds to the radical in $S_+$ of the sum of the $\mathfrak{j}_+(Y_\lambda)$.
\end{corollary}

\begin{corollary}[2.3.13]
\label{II.2.3.13}
Let $\mathfrak{J}$ be a graded ideal of $S_+$;
to have $V_+(\mathfrak{J})=\emp$, it is necessary and sufficient for every element $f$ of $S_+$ to have a power $f^n$ in $\mathfrak{J}$.
\end{corollary}

This above corollary can also be expressed in one of the following equivalent forms:

\begin{corollary}[2.3.14]
Let $(f_\alpha)$ be a family of homogeneous elements of $S_+$.
For the $D_+(f_\alpha)$ to form a cover of $X=\Proj(S)$, it is necessary and sufficient for every element of $S_+$ to have a power in the ideal generated by the $f_\alpha$.
\label{II.2.3.14}
\end{corollary}

\begin{corollary}[2.3.15]
\label{II.2.3.15}
Let $(f_\alpha)$ be a family of homogeneous elements of $S_+$, and $f$ an element of $S_+$.
Then the following are equivalent:
\begin{enumerate}
  \item[{\rm(a)}] $D_+(f)\subset\bigcup_\alpha D_+(f_\alpha)$;
  \item[{\rm(b)}] $V_+(f)\supset\bigcap_\alpha V_+(f_\alpha)$;
  \item[{\rm(c)}] $f$ has a power in the ideal generated by the $f_\alpha$.
\end{enumerate}
\end{corollary}

\begin{corollary}[2.3.16]
\label{II.2.3.16}
For $X=\Proj(S)$ to be empty, it is necessary and sufficient for every element of $S_+$ to be nilpotent.
\end{corollary}

\begin{corollary}[2.3.17]
\label{II.2.3.17}
In the bijective correspondence described in \sref{II.2.3.12}, the \emph{irreducible} closed subsets of $X$ correspond to the graded \emph{prime} ideals of $S_+$.
\end{corollary}

\begin{proof}
If $Y=Y_1\cup Y_2$, where $Y_1$ and $Y_2$ are distinct closed subsets of $Y$, then
\[
  \mathfrak{j}_+(Y) = \mathfrak{j}_+(Y_1)\cap\mathfrak{j}_+(Y_2)
\]
with the ideals $\mathfrak{j}_+(Y_1)$ and $\mathfrak{j}_+(Y_2)$ being distinct from $\mathfrak{j}_+(Y)$, and so $\mathfrak{j}_+(Y)$ is not prime.
Conversely, if $\mathfrak{J}$ is a graded ideal of $S_+$ that is not prime, then there exists elements $f,g\in S_+$ such that $f\not\in\mathfrak{J}$ and $g\not\in\mathfrak{J}$, but $fg\in\mathfrak{J}$;
then $V_+(f)\supset V_+(\mathfrak{J})$ and $V_+(g)\supset V_+(\mathfrak{J})$, but $V_+(\mathfrak{J})\subset V_+(f)\cup V_+(g)$, by \sref{II.2.3.2.3};
we thus conclude that $V_+(\mathfrak{J})$ is the union of the closed subsets $V_+(f)\cap V_+(\mathfrak{J})$ and $V_+(g)\cap V_+(\mathfrak{J})$, which are distinct from $V_+(\mathfrak{J})$.
\end{proof}


\subsection{The scheme structure on $\mathrm{Proj}(S)$}
\label{subsection:II.2.4}

\begin{env}[2.4.1]
\label{II.2.4.1}
Let $f$ and $g$ be homogeneous elements of $S_+$;
consider the affine schemes $Y_f=\Spec(S_{(f)})$, $Y_g=\Spec(S_{(g)})$, and $Y_{fg}=\Spec(S_{(fg)})$.
By \sref{II.2.2.2}, the morphism $w_{fg,f} = ({}^a\!\omega_{fg,f},\widetilde{\omega}_{fg,f})$ from $Y_{fg}$ to $Y_f$, corresponding to the canonical homomorphism $\omega_{fg,f}: S_{(f)}\to S_{(fg)}$, is an \emph{open immersion} \sref[I]{I.1.3.6}.
Using the inverse homeomorphism of $\psi_f: D_+(f)\to Y_f$ \sref{II.2.3.6}, we can transport the affine scheme structure of $Y_f$ to $D_+(f)$;
by the commutativity of diagram~\sref{II.2.3.5.1}, the affine scheme $D_+(fg)$ can thus be identified with the induced scheme on the open subset $D_+(fg)$ of the underlying space of the affine scheme $D_+(f)$.
It is then clear (taking \sref{II.2.3.4} into account) that $X=\Proj(S)$ is endowed with a unique \emph{prescheme} structure, whose restriction to each $D_+(f)$ is the affine scheme that we have just defined.
Furthermore:
\end{env}

\begin{proposition}[2.4.2]
\label{II.2.4.2}
The prescheme $\Proj(S)$ is a scheme.
\end{proposition}

\begin{proof}
It suffices \sref[I]{I.5.5.6} to show, for any homogeneous $f$ and $g$ in $S_+$, that $D_+(f)\cap D_+(g)=D_+(fg)$ is affine, and that its ring is generated by the canonical images of the rings of $D_+(f)$ and $D_+(g)$;
the first point is evident by definition, and the second follows from \sref{II.2.2.4}.
\end{proof}

\oldpage[II]{29}
Whenever we speak of the homogeneous prime spectrum $\Proj(S)$ as a \emph{scheme}, it will always mean with respect to the structure that we have just defined.

\begin{example}[2.4.3]
\label{II.2.4.3}
Let $S=K[T_1,T_2]$, where $K$ is a field, $T_1$ and $T_2$ are indeterminates, and $S$ is graded by total degree.
It follows from \sref{II.2.3.14} that $\Proj(S)$ is the union of $D_+(T_1)$ and $D_+(T_2)$;
we immediately see that these affine schemes are isomorphic to $K[T]$, and that $\Proj(S)$ is obtained by the gluing of these two affine schemes as described in \sref[I]{I.2.3.2} (cf. \sref{II.7.4.14}).
\end{example}

\begin{proposition}[2.4.4]
\label{II.2.4.4}
Let $S$ be a positively graded ring, and let $X$ be the scheme $\Proj(S)$.
\begin{enumerate}
  \item[{\rm(i)}] If $\mathfrak{N}_+$ is the nilradical of $S_+$ \sref{II.2.1.10}, then the scheme $X_\red$ is canonically isomorphic to $\Proj(S/\mathfrak{N}_+)$;
    in particular, if $S$ is essentially reduced, then $\Proj(S)$ is reduced.
  \item[{\rm(ii)}] If $S$ is essentially reduced, then, for $X$ to be integral, it is necessary and sufficient for $S$ to be essentially integral.
\end{enumerate}
\end{proposition}

\begin{proof}
\begin{enumerate}
  \item[{\rm(i)}] Let $\overline{S}$ be the graded ring $S/\mathfrak{N}_+$, and denote by $x\mapsto\overline{x}$ the canonical homomorphism $S\to\overline{S}$ of degree~$0$.
    For all $f\in S_d$ ($d>0$), the canonical homomorphism $S_f\to\overline{S}$ \sref[0]{0.1.5.1} is surjective and of degree~$0$, and thus gives, by restriction, a surjective homomorphism $S_{(f)}\to\overline{S}_{(\overline{f})}$;
    if we suppose that $f\not\in\mathfrak{N}_+$, then we immediately see that $\overline{S}_{(\overline{f})}$  is reduced, and that the kernel of the above homomorphism is the nilradical of $S_{(f)}$, or, in other words, that $\overline{S}_{(\overline{f})}=(S_{(f)})_\red$.
    So to this homomorphism corresponds a closed immersion $D_+(\overline{f})\to D_+(f)$ that identifies $D_+(\overline{f})$ with $(D_+(f))_\red$ \sref[I]{I.5.1.2}, and which is, in particular, a homeomorphism of the underlying spaces of these two affine schemes.
    Furthermore, if $g\not\in\mathfrak{N}_+$ is another homogeneous element of $S_+$, then the diagram
    \[
      \xymatrix{
        S_{(f)} \ar[r] \ar[d]
        & \overline{S}_{(\overline{f})} \ar[d]
      \\S_{(fg)} \ar[r]
        & \overline{S}_{(\overline{fg})}
      }
    \]
    is commutative;
    since, further, the $D_+(f)$, for $f$ homogeneous, of degree $>0$, and $f\not\in\mathfrak{N}_+$, form a cover of $X=\Proj(S)$ \sref{II.2.3.7}, we see that the morphisms $D_+(\overline{f})\to D_+(f)$ are the restrictions of a closed immersion $\Proj(\overline{S})\to\Proj(S)$ which is a homeomorphism of the underlying spaces;
    whence the conclusion \sref[I]{I.5.1.2}.
  \item[{\rm(ii)}] Suppose that $S$ is essentially integral, or, in other words, that $(0)$ is a graded prime ideal of $S_+$ that is distinct from $S_+$;
    then $X$ is reduced, by (i), and irreducible, by \sref{II.2.3.17}.
    Conversely, suppose that $S$ is essentially reduced and that $X$ is integral;
    then, for homogeneous $f\neq0$ in $S_+$, we have that $D_+(f)\neq\emp$ \sref{II.2.3.7};
    the hypothesis that $X$ is irreducible implies that $D_+(f)\cap D_+(g)\neq\emp$ for homogeneous $f,g\neq0$ in $S_+$;
    thus $fg\neq0$, by \sref{II.2.3.3.2}, and we thus conclude that $S_+$ has no zero divisors, whence the first claim.
\end{enumerate}
\end{proof}

\begin{env}[2.4.5]
\label{II.2.4.5}
Given a commutative ring $A$, recall that we say that a graded ring $S$ is a \emph{graded $A$-algebra} if it is endowed with the structure of an $A$-algebra such that each of its subgroups $S_n$ is an $A$-module;
for this, it suffices for $S_0$ to be
\oldpage[II]{30}
an $A$-algebra, or, in other words, we define the structure of a graded $A$-algebra on $S$ by defining the structure of an $A$-algebra on $S_0$ and setting $\alpha\cdot x=(\alpha\cdot1)x$ for $\alpha\in A$ and $x\in S_n$.
\end{env}

\begin{proposition}[2.4.6]
\label{II.2.4.6}
Suppose that $S$ is a graded $A$-algebra.
Then, on $X=\Proj(S)$, the structure sheaf $\sh{O}_X$ is an $A$-algebra (where $A$ is considered as a simple sheaf on $X$);
in other words, $X$ is a scheme over $\Spec(A)$.
\end{proposition}

\begin{proof}
It suffices to note that, for every homogeneous $f$ in $S_+$, $S_{(f)}$ is an algebra over $A$, and that the diagram
\[
  \xymatrix{
    S_{(f)} \ar[rr] && S_{(fg)}
  \\ & A \ar[ul] \ar[ur] &
  }
\]
is commutative, for homogeneous $f,g$ in $S_+$.
\end{proof}

\begin{proposition}[2.4.7]
\label{II.2.4.7}
Let $S$ be a positively graded ring.
\begin{enumerate}
  \item[{\rm(i)}] For every integer $d>0$, there exists a canonical isomorphism from the scheme $\Proj(S)$ to the scheme $\Proj(S^{(d)})$.
  \item[{\rm(ii)}] Let $S'$ be the graded ring such that $S_0=\bb{Z}$ and $S'_n=S_n$ (considered as a $\bb{Z}$-module) for $n>0$.
    Then there exists a canonical isomorphism from the scheme $\Proj(S)$ to the scheme $\Proj(S')$.
\end{enumerate}
\end{proposition}

\begin{proof}
\begin{enumerate}
  \item[{\rm(i)}] We first show that the map $\mathfrak{p}\mapsto\mathfrak{p}\cap S^{(d)}$ is a bijection from the set $\Proj(S)$ to the set $\Proj(S^{(d)})$.
    Indeed, suppose that we have a graded prime ideal $\mathfrak{p}'\in\Proj(S^{(d)})$, and let $\mathfrak{p}_{nd}=\mathfrak{p}'\cap S_{nd}$ ($n\geq0$).
    For all $n>0$ that are not multiples of $d$, define $\mathfrak{p}_n$ as the set of $x\in S_n$ such that $x^d\in\mathfrak{p}_{nd}$;
    if $x,y\in\mathfrak{p}_n$, then $(x+y)^{2d}\in\mathfrak{p}_{2nd}$, and so $(x+y)^d\in\mathfrak{p}_{nd}$, since $\mathfrak{p}'$ is prime;
    it is immediate that the $\mathfrak{p}_n$ thus defined, for $n\geq0$, satisfy the conditions of \sref{II.2.1.9}, and so there exists a unique prime ideal $\mathfrak{p}\in\Proj(S)$ such that $\mathfrak{p}\cap S^{(d)}=\mathfrak{p}'$.
    Since, for every homogeneous $f$ in $S_+$, we have that $V_+(f)=V_+(f^d)$ \sref{II.2.3.2.3}, we see that the above bijection is a \emph{homeomorphism} of topological spaces.
    Finally, with the same notation, $S_{(f)}$ and $S_{(f^d)}$ are canonically identified \sref{II.2.2.2}, and so $\Proj(S)$ and $\Proj(S^{(d)})$ are canonically identified as \emph{schemes}.
  \item[{\rm(ii)}] If, to each $\mathfrak{p}\in\Proj(S)$, we associate the unique prime ideal $\mathfrak{p}'\in\Proj(S')$ such that $\mathfrak{p}'\cap S_n=\mathfrak{p}\cap S_n$ for every $n>0$ \sref{II.2.1.9}, then it is clear that we have defined a canonical homeomorphism $\Proj(S)\xrightarrow{\sim}\Proj(S')$ of the underlying spaces, since $V_+(f)$ is the same set for $S$ and $S'$ when $f$ is a homogeneous element of $S_+$.
  Since, further, $S_{(f)}=S'_{(f)}$, $\Proj(S)$ and $\Proj(S')$ can be identified as \emph{schemes}.
\end{enumerate}
\end{proof}

\begin{corollary}[2.4.8]
\label{II.2.4.8}
If $S$ is a graded $A$-algebra, and $S'_A$ the graded $A$-algebra such that $(S'_A)_0=A$ and $(S'_A)_n=S_n$ for $n>0$, then there exists a canonical isomorphism from $\Proj(S)$ to $\Proj(S'_A)$.
\end{corollary}

\begin{proof}
In fact, these two schemes are both canonically isomorphic to $\Proj(S')$, using the notation of \sref{II.2.4.7}[(ii)].
\end{proof}


\subsection{The sheaf associated to a graded module}
\label{subsection:II.2.5}

\begin{env}[2.5.1]
\label{II.2.5.1}
Let $M$ be a \emph{graded} $S$-module.
The, for every homogeneous $f$ in $S_+$, $M_{(f)}$ is an $S_{(f)}$-module, and thus has a corresponding quasi-coherent associated sheaf $(M_{(f)})\supertilde$ on the affine scheme $\Spec(S_{(f)})$, identified with $D_+(f)$ \sref[I]{I.1.3.4}.
\end{env}

\oldpage[II]{31}
\begin{proposition}[2.5.2]
\label{II.2.5.2}
There exists on $X=\Proj(S)$ exactly one quasi-coherent $\sh{O}_X$-module $\widetilde{M}$ such that, for $f$ homogeneous in $S_+$, we have $\Gamma(D_+(f),\widetilde{M})=M_{(f)}$, with the restriction homomorphism $\Gamma(D_+(f),\widetilde{M})\to\Gamma(D_+(fg),\widetilde{M})$, for $f$ and $g$ homogeneous in $S_+$, being the canonical homomorphism $M_{(f)}\to M_{(fg)}$ \sref{II.2.2.3}.
\end{proposition}

\begin{proof}
Suppose that $f\in S_d$ and $g\in S_e$.
Since $D_+(fg)$ can be identified with the prime spectrum of $(S_{(f)})_{g^d/f^e}$ by \sref{II.2.2.2}, the restriction to $D_+(fg)$ of the sheaf $(M_{(f)})\supertilde$ on $D_+(f)$ is canonically identified with the sheaf associated to the module $(M_{(f)})_{g^d/f^e}$ \sref[I]{I.1.3.6}, and thus also with $(M_{(fg)})\supertilde$ \sref{II.2.2.2};
we thus conclude that there exists a canonical isomorphism
\[
  \theta_{g,f}: (M_{(f)})\supertilde|D_+(fg) \xrightarrow{\sim} (M_{(g)})\supertilde|D_+(fg)
\]
such that, if $h$ is a third homogeneous element of $S_+$, then $\theta_{f,h}=\theta_{f,g}\circ\theta_{g,h}$ in $D_+(fgh)$.
Consequently \sref[0]{0.3.3.1} there exists a quasi-coherent $\sh{O}_X$-module $\sh{F}$ on $X$, and, for every homogeneous $f$ in $S_+$, an isomorphism $\eta_f$ from $\sh{F}|D_+(f)$ to $(M_{f})\supertilde$ such that $\theta_{g,f}=\eta_g\circ\eta_f^{-1}$.
If we then consider the sheaf $\sh{G}$ associated to the presheaf (on the base of the topology of $X$ given by the $D_+(f)$) defined by $D_+(f)\mapsto M_{(f)}$, with the canonical homomorphisms $M_{(f)}\to M_{(fg)}$ as restriction homomorphisms, then the above proves that $\sh{F}$ and $\sh{G}$ are isomorphic (taking \sref[I]{I.1.3.7} into account);
the sheaf $\sh{G}$ is denoted by $\widetilde{M}$, and indeed satisfies the conditions of the statement.
We have, in particular, $\widetilde{S}=\sh{O}_X$.
\end{proof}

\begin{definition}[2.5.3]
\label{II.2.5.3}
We say that the quasi-coherent $\sh{O}_X$-module $\widetilde{M}$ defined in \sref{II.2.5.2} is \emph{associated} to the graded $S$-module $M$.
\end{definition}

Recall that the graded $S$-modules form a category when we restrict from arbitrary homomorphisms of graded modules to homomorphisms \emph{of degree~$0$}.
With this convention:

\begin{proposition}[2.5.4]
\label{II.2.5.4}
The functor $M\mapsto\widetilde{M}$ is an exact additive covariant functor from the category of graded $S$-modules to the category of quasi-coherent $\sh{O}_X$-modules, and it commutes with inductive limits and direct sums.
\end{proposition}

\begin{proof}
Indeed, since these properties are local, it suffices to show that they are satisfied for the sheaves of the form $\widetilde{M}|D_+(f)=(M_{(f)})\supertilde$;
but the functors $M\mapsto M_f$, $N\mapsto N_0$ (to the category of graded $S_f$-modules), and $P\mapsto\widetilde{P}$ (to the category of $S_{(f)}$-modules) all have the three properties of exactness and of commutativity with inductive limits and direct sums (\sref[I]{I.1.3.5} and \sref[I]{I.1.3.9});
whence the proposition.
\end{proof}

We denote by $\widetilde{u}$ the homomorphism $\widetilde{M}\to\widetilde{N}$ corresponding to a homomorphism $u: M\to N$ of degree~$0$.
We immediately deduce from \sref{I.2.5.4} that the results of \sref[I]{I.1.3.9} and \sref[I]{I.1.3.10} still hold for graded $S$-modules and homomorphisms of degree~$0$ (with the sense given here to $\widetilde{M}$), with the proofs being purely formal.

\begin{proposition}[2.5.5]
\label{II.2.5.5}
For all $\mathfrak{p}\in X=\Proj(S)$, we have $\widetilde{M}_\mathfrak{p}=M_{(\mathfrak{p})}$.
\end{proposition}

\begin{proof}
By definition, $\widetilde{M}_\mathfrak{p}=\varinjlim\Gamma(D_+(f),\widetilde{M})$, where $f$ runs over the set of homogeneous elements of $S_+$ such that $f\not\in\mathfrak{p}$;
since $\Gamma(D_+(f),\widetilde{M})=M_{(f)}$, the proposition follows from the definition of $M_{(\mathfrak{p})}$ \sref{II.2.2.7}
\end{proof}

\oldpage[II]{32}
In particular, the \emph{local ring $(\sh{O}_X)_\mathfrak{p}$} is exactly the ring $S_{(\mathfrak{p})}$, the set of fractions $x/f$ with $f$ homogeneous in $S_+$ and not belonging to $\mathfrak{p}$, and with $x$ homogeneous of the same degree as $f$.

Even more particularly, if $S$ is \emph{essential integral}, so that $\Proj(S)=X$ is \emph{integral} \sref{II.2.4.4}, and if $\xi=(0)$ is the generic point of $X$, then the \emph{field of rational functions $R(X)=\sh{O}_\xi$} is field consisting of elements $fg^{-1}$ with $f$ and $g$ homogeneous of the same degree in $S_+$, and with $g\neq 0$.

\begin{proposition}[2.5.6]
\label{II.2.5.6}
If, for all $z\in M$ and all homogeneous $f$ in $S_+$, there exists a power of $f$ that annihilates $z$, then $\widetilde{M}=0$.
This sufficient condition is also necessary if the $S_0$-algebra $S$ is generated by the set $S_1$ of homogeneous elements of degree~$1$.
\end{proposition}

\begin{proof}
The condition $\widetilde{M}=0$ is equivalent to $M_{(f)}=0$ for all homogeneous $f$ in $S_+$.
On the other hand, if $f\in S_d$, to say that $M_{(f)}=0$ implies that, for all homogeneous $z\in M$ whose degree is some multiple of $d$, there exists a power $f^n$ such that $f^nz=0$.
If $M_{(f)}=0$ for all $f\in S_1$, then the condition of the statement is thus satisfied for all $f\in S_1$;
the condition is \emph{a fortiori} satisfied for all homogeneous $f$ in $S_+$ if $S_1$ generates $S$, since every homogeneous element of $S_+$ is then a linear combination of products of elements of $S_1$.
\end{proof}

\begin{proposition}[2.5.7]
\label{II.2.5.7}
Let $d>0$ be an integer, and let $f\in S_d$.
Then, for all $n\in\bb{Z}$, the $(\sh{O}_X|D_+(f))$-module $(S(nd))\supertilde|D_+(f)$ is canonically isomorphic to $\sh{O}_X|D_+(f)$.
\end{proposition}

\begin{proof}
Indeed, multiplication by the invertible element $(f/1)^n$ of $S_f$ gives a bijection from $S_{(f)}=(S_f)_0$ to $(S_f)_{nd}=(S_f(nd))_0=(S(nd)_f)_0=S(nd)_{(f)}$;
in other words, the $S_{(f)}$-modules $S_{(f)}$ and $S(nd)_{(f)}$ are canonically isomorphic, whence the proposition.
\end{proof}

\begin{corollary}[2.5.8]
\label{II.2.5.8}
On the open subset $U=\bigcup_{f\in S_d}D_+(f)$, the restriction of the $\sh{O}_X$-module $(S(nd))\supertilde$ is an invertible $(\sh{O}_X|U)$-module \sref[0]{0.5.4.1}.
\end{corollary}

\begin{corollary}[2.5.9]
\label{II.2.5.9}
If the ideal $S_+$ of $S$ is generated by the set $S_1$ of homogeneous elements of degree~$1$, then the $\sh{O}_X$-module $(S(n))\supertilde$ is invertible for all $n\in\bb{Z}$.
\end{corollary}

\begin{proof}
It suffices to remark that $X=\bigcup_{f\in S_1}D_+(f)$, by the hypothesis \sref{II.2.3.14} and to apply \sref{II.2.5.8} with $U=X$.
\end{proof}

\begin{env}[2.5.10]
\label{II.2.5.10}
We set, for the rest of this section,
\[
\label{II.2.5.10.1}
  \sh{O}_X(n) = (S(n))\supertilde
\tag{2.5.10.1}
\]
for all $n\in\bb{Z}$, and, for every open subset $U$ of $X$, and every $(\sh{O}_X|U)$-module $\sh{F}$,
\[
\label{II.2.5.10.2}
  \sh{F}(n) = \sh{F}\otimes_{\sh{O}_X|U}(\sh{O}_X(n)|U)
\tag{2.5.10.2}
\]
for all $n\in\bb{Z}$.
If the ideal $S_+$ is generated by $S_1$, then the functor $\sh{F}(n)$ is \emph{exact} in $\sh{F}$ for all $n\in\bb{Z}$, since $\sh{O}_X(n)$ is then an \emph{invertible} $\sh{O}_X$-module.
\end{env}

\begin{env}[2.5.11]
\label{II.2.5.11}
Let $M$ and $N$ be graded $S$-modules.
For all $f\in S_d$ ($d>0$), we define a canonical functorial homomorphism of $S_{(f)}$-modules by
\[
\label{II.2.5.11.1}
  \lambda_f: M_{(f)}\otimes_{S_{(f)}}N_{(f)} \to (M\otimes_S N)_{(f)}
\tag{2.5.11.1}
\]
\oldpage[II]{33}
by composing the homomorphism $M_{(f)}\otimes_{S_{(f)}}N_{(f)}\to M_f\otimes_{S_f}N_f$ (coming from the injections $M_{(f)}\to M_f$, $N_{(f)}\to N_f$, and $S_{(f)}\to S_f$) with the canonical isomorphism $M_f\otimes_{S_f}N_f\xrightarrow{\sim}(M\otimes_S N)_f$ \sref[0]{0.1.3.4}, and by noting that, by the definition of the tensor product of two graded modules, this latter isomorphism preserves degrees;
for $x\in M_{md}$ and $y\in N_{nd}$ ($m,n\geq0$), we thus have
\[
  \lambda_f((x/f^m)\otimes(y/f^n)) = (x\otimes y)/f^{m+n}.
\]

It immediately follows from this definition that, if $g\in S_e$ ($e>0$), then the diagram
\[
  \xymatrix{
    M_{(f)}\otimes_{S_{(f)}}N_{(f)} \ar[r]^{\lambda_f} \ar[d]
    & (M\otimes_S N)_{(f)} \ar[d]
  \\M_{(fg)}\otimes_{S_{(fg)}}N_{(fg)} \ar[r]_{\lambda_{fg}}
    & (M\otimes_S N)_{(fg)}
  }
\]
(where the vertical arrow on the right is the canonical homomorphism, and the one on the left comes from the canonical homomorphisms) commutes.
Thus $\lambda$ induces a canonical functorial homomorphism of $\sh{O}_X$-modules
\[
\label{II.2.5.11.2}
  \lambda: \widetilde{M}\otimes_{\sh{O}_X}\widetilde{N} \to (M\otimes_S N)\supertilde.
\tag{2.5.11.2}
\]

Consider, in particular, graded ideals $\mathfrak{J}$ and $\mathfrak{K}$ of $S$;
since $\widetilde{\mathfrak{J}}$ and $\widetilde{\mathfrak{K}}$ are sheaves of ideals of $\sh{O}_X$, we have a canonical homomorphism $\widetilde{\mathfrak{J}}\otimes_{\sh{O}_X}\widetilde{\mathfrak{K}}\to\sh{O}_X$;
the diagram
\[
\label{II.2.5.11.3}
  \xymatrix{
    \widetilde{\mathfrak{J}}\otimes_{\sh{O}_X}\widetilde{\mathfrak{K}} \ar[rr]^\lambda \ar[dr]
    && (\widetilde{\mathfrak{J}}\otimes_S\widetilde{\mathfrak{K}})\supertilde \ar[dl]
  \\&\sh{O}_X&
  }
\tag{2.5.11.3}
\]
then commutes.
Indeed, we can reduce to verifying this on each open subset $D_+(f)$ (for $f$ homogeneous in $S_+$), and this follows immediately from the definition \sref{2.5.11.1} of $\lambda_f$ and from \sref[I]{I.1.3.13}.

Finally, note that, if $M$, $N$, and $P$ are graded $S$-modules, then the diagram
\[
\label{II.2.5.11.4}
  \xymatrix{
    \widetilde{M}\otimes_{\sh{O}_X}\widetilde{N}\otimes_{\sh{O}_X}\widetilde{P} \ar[r]^{\lambda\otimes1} \ar[d]_{1\otimes\lambda}
    & (M\otimes_S N)\supertilde\otimes_{\sh{O}_X}\widetilde{P} \ar[d]^\lambda
  \\\widetilde{M}\otimes_{\sh{O}_X}(N\otimes_S P)\supertilde \ar[r]_\lambda
    & (M\otimes_S N\otimes_S P)\supertilde
  }
\tag{2.5.11.4}
\]
\oldpage[II]{34}
commutes.
It again suffices to verify this on each open subset $D_+(f)$, and this follows immediately from the definitions and from \sref[I]{I.1.3.13}.
\end{env}

\begin{env}[2.5.12]
\label{II.2.5.12}
Under the hypotheses of \sref{II.2.5.11}, we define a functorial canonical homomorphism of $S_{(f)}$-modules
\[
\label{II.2.5.12.1}
  \mu_f: (\Hom_S(M,N))_{(f)} \to \Hom_{S_{(f)}}(M_{(f)},N_{(f)})
\tag{2.5.12.1}
\]
by sending $u/f^n$, where $u$ is a homomorphism of degree~$nd$, to the homomorphism $M_{(f)}\to N_{(f)}$ that sends $x/f^m$ ($x\in M_{md}$) to $u(x)/f^{m+n}$.
For $g\in S_e$ ($e>0$), we again have a commutative diagram:
\[
  \xymatrix{
    (\Hom_S(M,N))_{(f)} \ar[r]^{\mu_f} \ar[d]
    & \Hom_{S_{(f)}}(M_{(f)},N_{(f)}) \ar[d]
  \\(\Hom_S(M,N))_{(fg)} \ar[r]_{\mu_{fg}}
    & \Hom_{S_{(fg)}}(M_{(fg)},N_{(fg)})
  }
\]
(where the vertical arrow on the left is the canonical homomorphism, and the one on the right comes from the canonical homomorphisms).
We thus again conclude (taking \sref[I]{I.1.3.8} into account) that the $\mu_f$ define a functorial canonical homomorphism of $\sh{O}_X$-modules
\[
\label{II.2.5.12.2}
  \mu: (\Hom_S(M,N))\supertilde \to \shHom_{\sh{O}_X}(\widetilde{M},\widetilde{N})
\tag{2.5.12.2}
\]
\end{env}

\begin{proposition}[2.5.13]
\label{II.2.5.13}
Suppose that the ideal $S_+$ is generated by $S_1$.
Then the homomorphism $\lambda$ \sref{II.2.5.11.2} is an isomorphism;
so too is the homomorphism $\mu$ \sref{II.2.5.12.2} if the graded $S$-module $M$ admits a finite presentation \sref{II.2.1.1}
\end{proposition}

\begin{proof}
Since $X$ is the union of the $D_+(f)$ for $f\in S_1$ \sref{II.2.3.14}, we are led to proving that $\lambda_f$ and $\mu_f$ are isomorphisms, under the given hypotheses, whenever $f$ is homogeneous and \emph{of degree~$1$}.
But we can then define a $\bb{Z}$-bilinear map $M_m\times N_n\to M_{(f)}\otimes_{S_{(f)}}N_{(f)}$ by sending $(x,y)$ to the element $(x/f^m)\otimes(y/f^n)$ (if $m<0$, we write $x/f^m$ to mean $f^{-m}x/1$);
these maps define a $\bb{Z}$-linear map $M\otimes_{\bb{Z}}N\to M_{(f)}\otimes_{S_{(f)}}N_{(f)}$, and, if $s\in S_q$, this map sends $(sx)\otimes y$ to $(s/f^q)((x/f^m)\otimes(y/f^n))$ (for $x\in M_m$ and $y\in N_n$).
We thus obtain a di-homomorphism of modules $\gamma_f: M\otimes_S N\to M_{(f)}\otimes_{S_{(f)}}N_{(f)}$, with respect to the canonical homomorphism $S\to S_{(f)}$ (sending $s\in S_q$ to $s/f^q$).
Suppose furthermore that, for an element $\sum_i(x_i\otimes y_i)$ of $M\otimes_S N$ (with $x_i$ and $y_i$ homogeneous of degree $m_i$ and $n_i$, respectively), we have that $f^r\sum_i(x_i\otimes y_i)=0$, or, in other words, that $\sum_i(f^rx_i\otimes y_i)=0$.
We thus deduce, by \sref[0]{0.1.3.4}, that $\sum_i(f^rx_i/f^{m_i+r})\otimes(y_i/f^{n_i})=0$, i.e. $\gamma_f(\sum_i(x_i\otimes y_i))=0$.
Then $\gamma_f$ factors as $M\otimes_S N\to(M\otimes_S N)_f\xrightarrow{\gamma'_f}M_{(f)}\otimes_{S_{(f)}}N_{(f)}$;
if $\lambda'_f$ is the restriction of $\gamma'$
\oldpage[II]{35}
to $(M\otimes_S N)_{(f)}$, then we can immediately show that $\lambda_f$ and $\lambda'_f$ are inverse $S_{(f)}$-homomorphisms, whence the first part of the proposition.

To prove the second part, suppose that $M$ is the cokernel of a homomorphism $P\to Q$ of graded $S$-modules, with $P$ and $Q$ being direct sums of a finite number of modules of the form $S(n)$;
using the left-exactness of $\Hom_S(L,N)$ in $L$, and the exactness of $M_{(f)}$ in $M$, we can immediately reduce to proving that $\mu_f$ is an isomorphism whenever $M=S(n)$.
But, for any homogeneous $z$ in $N$, let $u_z$ be the homomorphism from $S(n)$ to $N$ such that $u_z(1)=z$;
we immediately see that $\eta: z\to u_z$ is an isomorphism of degree~$0$ from $N(-n)$ to $\Hom_S(S(n),N)$.
There is a corresponding isomorphism
\[
  \eta_f: (N(-n))_{(f)} \to (\Hom_S(S(n),N))_{(f)}.
\]

Now let $\eta'_f$ be the isomorphism $N_{(f)}\to\Hom_{S_{(f)}}(S(n)_{(f)},N_{(f)})$ that, to any $z'\in N_{(f)}$, associates the homomorphism $v_{z'}$ that is such that $v_{z'}(s/f^k)=sz'/f^{n+k}$ (for $s\in S_{n+k}=(S(n))_k$).
We easily note that the composed map
\[
  (N(-n))_{(f)}
  \xrightarrow{\eta_f} (\Hom_S(S(n),N))_{(f)}
  \xrightarrow{\mu_f} \Hom_{S_{(f)}}(S(n)_{(f)},N_{(f)})
  \xrightarrow{{\eta'_f}^{-1}} N_{(f)}
\]
is the isomorphism $z/f^h\mapsto z/f^{h-n}$ from $(N(-n))_{(f)}$ to $N_{(f)}$, and thus $\mu_f$ is an isomorphism.
\end{proof}

If the ideal $S_+$ is generated by $S_1$, then we deduce from \sref{II.2.5.13} that, for every graded ideal $\mathfrak{J}$ of $S$, and for every graded $S$-module $M$, we have
\[
\label{II.2.5.13.1}
  \widetilde{\mathfrak{J}}\cdot\widetilde{M} = (\mathfrak{J}\cdot M)\supertilde
\tag{2.5.13.1}
\]
up to canonical isomorphism;
this follows from the commutativity of the diagram
\[
  \xymatrix{
    \widetilde{\mathfrak{J}}\otimes_{\sh{O}_X}\widetilde{M} \ar[rr]^\lambda \ar[dr]
    && (\mathfrak{J}\otimes_S M)\supertilde \ar[dl]
  \\&\widetilde{M}
  }
\]
which we can verify as we did for \sref{II.2.5.11.3}.

\begin{corollary}[2.5.14]
\label{II.2.5.14}
Suppose that $S$ is generated by $S_1$.
For any $m,n\in\bb{Z}$, we then have:
\[
\label{II.2.5.14.1}
  \sh{O}_X(m)\otimes_{\sh{O}_X}\sh{O}_X(n) = \sh{O}_X(m+n)
\tag{2.5.14.1}
\]
\[
\label{II.2.5.14.2}
  \sh{O}_X(n) = (\sh{O}_X(1))^{\otimes n}
\tag{2.5.14.2}
\]
up to canonical isomorphism.
\end{corollary}

\begin{proof}
The first equation follows from \sref{II.2.5.13} and from the existence of the canonical isomorphism $S(m)\otimes_S S(n)\xrightarrow{\sim}S(m+n)$ of degree~$0$ that sends the element $1\otimes1$ (where the first $1$ is in $(S(m))_{-m}$ and the second is in $(S(n))_{-n}$) to the element $1\in(S(m+n))_{-(m+n)}$.
It then suffices to prove the second equation for $n=-1$, and, by \sref{II.2.5.13}, this reduces to seeing that $\Hom_S(S(1),S)$ is canonically isomorphic to $S(-1)$, which can be immediately proven by going back to the definitions \sref{II.2.1.2} and by remembering that $S(1)$ is a monogeneous $S$-module.
\end{proof}

\oldpage[II]{36}
\begin{corollary}[2.5.15]
\label{II.2.5.15}
Suppose that $S$ is generated by $S_1$.
Then, for every graded $S$-module $M$, and for every $n\in\bb{Z}$, we have
\[
\label{II.2.5.15.1}
  (M(n))\supertilde = \widetilde{M}(n)
\tag{2.5.15.1}
\]
up to canonical isomorphism.
\end{corollary}

\begin{proof}
This follows from definitions \sref{II.2.5.10.2} and \sref{II.2.5.10.1}, from Proposition~\sref{II.2.5.13}, and from the existence of a canonical isomorphism $M(n)\xrightarrow{\sim}M\otimes_S S(n)$ of degree~$0$ that, to every $z\in(M(n))_h=M_{n+h}$, associates $z\otimes1\in M_{n+h}\otimes(S(n))_{-n}\subset(M\otimes_S S(n))_h$.
\end{proof}

\begin{env}[2.5.16]
\label{II.2.5.16}
We denote by $S'$ the graded ring such that $S'_0=\bb{Z}$, and $S'_n=S_n$ for $n>0$.
Then, if $f\in S_d$ ($d>0$), we have that $(S(n))_{(f)}=(S'(n))_{(f)}$ for all $n\in\bb{Z}$, since an element of $(S'(n))_{(f)}$ is of the form $x/f^k$, with $x\in S'_{n+kd}$ ($k>0$), and we can always take $k$ to be such that $n+kd\neq0$.
Since $X=\Proj(S)$ and $X'=\Proj(S')$ are canonically identified \sref{II.2.4.7}[(ii)], we see that, for all $n\in\bb{Z}$, $\sh{O}_X(n)$ and $\sh{O}_{X'}(n)$ are the images of one another under the above identification.

Note also that, for all $d>0$ and all $n\in\bb{Z}$, we have
\[
  (S^{(d)}(n))_h = S_{(n+h)d} = (S(nd))_{hd}
\]
for $f\in S_d$, and thus $(S^{(d)}(n))_{(f)}=(S(nd))_{(f)}$.
We know that the schemes $X=\Proj(S)$ and $X^{(d)}=\Proj(S^{(d)})$ are canonically identified \sref{II.2.4.7}[(ii)];
the above shows that, if the $S_0$-algebra $S^{(d)}$ is generated by $S_d$, then $\sh{O}_X(nd)$ and $\sh{O}_{X^{(d)}}(n)$ are the images of one another under this identification, for all $n\in\bb{Z}$.
\end{env}

\begin{proposition}[2.5.17]
\label{II.2.5.17}
Let $d>0$ be an integer, and let $U=\bigcup_{f\in S_d}D_+(f)$.
Then the restriction to $U$ of the canonical homomorphism $\sh{O}_X(nd)\otimes_{\sh{O}_X}\sh{O}_X(-nd)\to\sh{O}_X$ is an isomorphism for every integer $n$.
\end{proposition}

\begin{proof}
By \sref{II.2.5.16}, we can restrict to the case where $d=1$, and the conclusion then follows from the proof of \sref{II.2.5.13}.
\end{proof}


\subsection{The graded $S$-module associated to a sheaf on $\operatorname{Proj}(S)$}
\label{subsection:II.2.6}

\emph{We suppose all throughout this section that the ideal $S_+$ of $S$ is generated by the set $S_1$ of homogeneous elements of degree~$1$.}

\begin{env}[2.6.1]
\label{II.2.6.1}
The $\sh{O}_X$-module $\sh{O}_X(1)$ is \emph{invertible} \sref{II.2.5.9};
we thus define, for every $\sh{O}_X$-module $\sh{F}$ \sref[0]{0.5.4.6},
\[
\label{II.2.6.1.1}
  \Gamma_\bullet(\sh{F})
  = \Gamma_\bullet(\sh{O}_X(1),\sh{F})
  = \bigoplus_{n\in\bb{Z}} \Gamma(X,\sh{F}(n))
\tag{2.6.1.1}
\]
taking \sref{II.2.5.14.2} into account.
Recall \sref[0]{0.5.4.6} that $\Gamma_\bullet(\sh{O}_X)$ is endowed with the structure of a \emph{graded ring}, and $\Gamma_\bullet(\sh{F})$ with the structure of a \emph{graded $\Gamma_\bullet(\sh{O}_X)$-module}.

Since $\sh{O}_X(n)$ is locally free, $\Gamma_\bullet(\sh{F})$ is a \emph{left exact} additive covariant functor in $\sh{F}$;
in particular, if $\sh{J}$ is a sheaf of ideals of $\sh{O}_X$, then $\Gamma_\bullet(\sh{J})$ is canonically identified with a \emph{graded idea} of $\Gamma_\bullet(\sh{O}_X)$.
\end{env}

\begin{env}[2.6.2]
\label{II.2.6.2}
Let $M$ be a graded $S$-module;
for every $f\in S_d$ ($d>0$), $x\mapsto x/1$ is a homomorphism of abelian groups $M_0\to M_{(f)}$, and, since $M_{(f)}$ is canonically identified
\oldpage[II]{37}
with $\Gamma(D_+(f),\widetilde{M})$, we thus obtain a homomorphism of abelian groups $\alpha_0^f: M_0\to\Gamma(D_+(f),\widetilde{M})$.
It is clear that, for every $g\in S_e$ ($e>0$), the diagram
\[
  \xymatrix{
    & \Gamma(D_+(f),\widetilde{M}) \ar[dd]
  \\M_0 \ar[ur]^{\alpha_0^f} \ar[dr]_{\alpha_0^{fg}}
  \\& \Gamma(D_+(fg),\widetilde{M})
  }
\]
commutes;
this implies that, for all $x\in M_0$, the sections $\alpha_0^f(x)$ and $\alpha_0^g(x)$ of $M$ agree on $D_+(f)\cap D_+(g)$, and thus there exists a unique section $\alpha_0(x)\in\Gamma(X,\widetilde{M})$ whose restriction to each $D_+(f)$ is $\alpha_0^f(x)$.
We have thus defined (without using the hypothesis that $S$ be generated by $S_1$) a homomorphism of abelian groups
\[
\label{II.2.6.2.1}
  \alpha_0: M_0\to\Gamma(X,\widetilde{M}).
\tag{2.6.2.1}
\]

Applying this result to the graded $S$-module $M(n)$ (for each $n\in\bb{Z}$), we obtain, for each $n\in\bb{Z}$, a homomorphism of abelian groups
\[
\label{II.2.6.2.2}
  \alpha_n: M_n=(M(n))_0\to\Gamma(X,\widetilde{M}(n))
\tag{2.6.2.2}
\]
(taking \sref{II.2.5.15});
whence we obtain a functorial homomorphism (of degree~$0$) of graded abelian groups
\[
\label{II.2.6.2.3}
  \alpha: M\to\Gamma_\bullet(\widetilde{M})
\tag{2.6.2.3}
\]
(also denoted by $\alpha_M$) which, on each $M_n$, agrees with $\alpha_n$.

If we take, in particular, $M=S$, then we immediately see (taking into account the definition \sref[0]{0.5.4.6} of multiplication in $\Gamma_\bullet(\sh{O}_X)$) that $\alpha: S\to\Gamma_\bullet(\sh{O}_X)$ is a homomorphism of graded rings, and that, for every graded $S$-module $M$, \sref{II.2.6.2.3} is a di-homomorphism of graded modules.
\end{env}

\begin{proposition}[2.6.3]
\label{II.2.6.3}
For every $f\in S_d$ ($d>0$), $D_+(f)$ is identical to the set of $\mathfrak{p}\in X$ on which the section $\alpha_d(f)$ of $\sh{O}_X(d)$ does not vanish \sref[0]{0.5.5.2}.
\end{proposition}

\begin{proof}
Since $X=\bigcup_{g\in S_1}D_+(g)$ by hypothesis, it suffices to show that, for all $g\in S_1$, the set of $\mathfrak{p}\in D_+(g)$ on which $\alpha_d(f)$ does not vanish is identical to $D_+(fg)$.
But the restriction of $\alpha_d(f)$ to $D_+(g)$ is, by definition, the section corresponding to the element $f/1$ of $(S(d))_{(g)}$;
under the canonical isomorphism $(S(d))_{(g)}\xrightarrow{\sim}S_{(g)}$ \sref{II.2.5.7}, this section of $\sh{O}_X(d)$ over $D_+(g)$ is identified with the section of $\sh{O}_X$ over $D_+(g)$ that corresponds to the element $f/g^d$ of $S_{(g)}$;
to say that this section vanishes at $\mathfrak{p}\in D_+(g)$ implies that $f/g^d\in\mathfrak{q}$, where $\mathfrak{q}$ is the prime ideal of $S_{(g)}$ corresponding to $\mathfrak{p}$ \sref{II.2.3.6};
by definition, this implies that $f\in\mathfrak{p}$, whence the proposition.
\end{proof}

\begin{env}[2.4.6]
\label{II.2.4.6}
Now let $\sh{F}$ be an $\sh{O}_X$-modules, and set $M=\Gamma_\bullet(\sh{F})$;
by the existence of the homomorphism of graded rings $\alpha: S\to\Gamma_\bullet(\sh{O}_X)$, we can consider $M$ as a graded $S$-module.
For every $f\in S_d$ ($d>0$), it follows from \sref{II.2.6.3} that the restriction to $D_+(f)$ of the section $\alpha_d(f)$ of $\sh{O}_X(d)$ is invertible;
thus so too is the restriction to $D_+(f)$ of the section $\alpha_d(f^n)$ of $\sh{O}_X(nd)$, for all $n>0$.
So let $z\in M_{nd}=\Gamma(X,\sh{F}(nd)$ ($n>0$);
if there exists an integer $k\geq0$ such that the restriction to $D_+(f)$ of $f^kz$, i.e. the
\oldpage[II]{38}
section $(z|D_+(f))(\alpha_d(f^k)|D_+(f))$ of $\sh{F}((n+k)d)$, is zero, then, by the above remark, we also have that $z|D_+(f)=0$.
This shows that we have defined an $S_{(f)}$-homomorphism $\beta_f: M_{(f)}\to\Gamma(D_+(f),\sh{F})$ by sending the element $z/f^n$ to the section $(z|D_+(f))(\alpha_d(f^n)|D_+(f))^{-1}$ of $\sh{F}$ over $D_+(f)$.
We can further immediately show that the diagram
\[
\label{II.2.6.4.1}
  \xymatrix{
    M_{(f)} \ar[r]^{\beta_f} \ar[d]
    & \Gamma(D_+(f),\sh{F}) \ar[d]
  \\M_{(fg)} \ar[r]_{\beta_{fg}}
    & \Gamma(D_+(fg),\sh{F})
  }
\tag{2.6.4.1}
\]
commutes for $g\in S_e$ ($e>0$).
If we recall that $M_{(f)}$ is canonically identified with $\Gamma(D_+(f),\widetilde{M})$, and that the $D_+(f)$ form a base for the topology of $X$ \sref{II.2.3.4}, then we see that the $\beta_f$ come from a unique canonical homomorphism of $\sh{O}_X$-modules
\[
\label{II.2.6.4.2}
  \beta:(\Gamma_\bullet(\sh{F}))\supertilde\to\sh{F}
\tag{2.6.4.2}
\]
(also denoted by $\beta_{\sh{F}}$) which is evidently functorial.
\end{env}

\begin{proposition}[2.6.5]
\label{II.2.6.5}
Let $M$ be a graded $S$-module, and $\sh{F}$ an $\sh{O}_X$-module;
then the composite homomorphisms
\[
\label{II.2.6.5.1}
  \widetilde{M}
  \xrightarrow{\widetilde{\alpha}} (\Gamma_\bullet(\widetilde{M}))\supertilde
  \xrightarrow{\beta} \widetilde{M}
\tag{2.6.5.1}
\]
\[
\label{II.2.6.5.2}
  \Gamma_\bullet(\sh{F})
  \xrightarrow{\alpha} \Gamma_\bullet((\Gamma_\bullet(\sh{F}))\supertilde)
  \xrightarrow{\Gamma_\bullet(\beta)} \Gamma_\bullet(\sh{F})
\tag{2.6.5.2}
\]
are the identity isomorphisms.
\end{proposition}

\begin{proof}
The proof for \sref{II.2.6.5.1} is local:
in an open subset $D_+(f)$, it follows immediately from the definitions, along with the fact that $\beta$, applied to quasi-coherent sheaves, is determined by its action on the sections over $D_+(f)$ \sref[I]{I.1.3.8}.
The proof for \sref{II.2.6.5.2} is done for each degree separately:
if we set $M=\Gamma_\bullet(\sh{F})$, then $M_n=\Gamma(X,\sh{F}(n))$, and $(\Gamma_\bullet(\widetilde{M}))_n=\Gamma(X,\widetilde{M}(n))=\Gamma(X,(M(n))\supertilde)$.
But if $f\in S_1$ and $z\in M_n$, then $\alpha_n^f(z)$ is the element $z/1$ of $(M(n))_{(f)}$, equal to $(f/1)^n(z/f^n)$;
it corresponds, via $\beta_f$, to the section
\[
  \Big(\big(\alpha_1(f)\big)^n|D_+(f)\Big) \Big(\big(z|D_+(f)\big)\big((\alpha_1(f))^n|D_+(f)\big)^{-1}\Big)
\]
over $D_+(f)$, i.e. the restriction of $z$ to $D_+(f)$, which finishes the proof for \sref{II.2.6.5.2}.
\end{proof}


\subsection{Finiteness conditions}
\label{subsection:II.2.7}

\begin{proposition}[2.7.1]
\label{II.2.7.1}
\begin{enumerate}
  \item[{\rm(i)}] If $S$ is a graded Noetherian ring, then $X=\Proj(S)$ is a Noetherian scheme.
  \item[{\rm(ii)}] If $S$ is a graded $A$-algebra of finite type, then $X=\Proj(S)$ is a scheme of finite type over $Y=\Spec(A)$.
\end{enumerate}
\end{proposition}

\oldpage[II]{39}
\begin{proof}
\begin{enumerate}
  \item[{\rm(i)}] If $S$ is Noetherian, then the ideal $S_+$ admits a finite system of homogeneous generators $f_i$ ($1\leq i\leq p$), thus \sref{II.2.3.14} the underlying space $X$ is the union of the $D_+(f_i)=\Spec(S_{(f_i)})$, and everything then reduces to showing that each of the $S_{(f_i)}$ is Noetherian, which follows from \sref{II.2.2.6}.
  \item[{\rm(ii)}] The hypothesis implies that $S_0$ is an $A$-algebra of finite type, and that $S$ is an $S_0$-algebra of finite type, and so $S_+$ is an ideal of finite type \sref{II.2.1.4}.
    We are thus reduced, as in (i), to showing that $S_{(f)}$ is an $A$-algebra of finite type for all $f\in S_d$.
    By \sref{II.2.2.5}, it suffices to show that $S^{(d)}$ is an $A$-algebra of finite type, which follows from \sref{II.2.1.6}.
\end{enumerate}
\end{proof}

\begin{env}[2.7.2]
\label{II.2.7.2}
In what follows, we consider the following finiteness conditions for a graded $S$-module $M$:
\begin{enumerate}
  \item[{\rm(TF)}] There exists an integer $n$ such that the submodule $\bigoplus_{k\geq n}M_k$ is an $S$-module of finite type.
  \item[{\rm(TN)}] There exists an integer $n$ such that $M_k=0$ for $k\geq n$.
\end{enumerate}

If $M$ satisfies (TN), then $M_{(f)}=0$ for all homogeneous $f$ in $S_+$, and thus $\widetilde{M}=0$.

Let $M$ and $N$ be graded $S$-modules;
we say that a homomorphism $u: M\to N$ of degree~$0$ is (TN)-\emph{injective} (resp. (TN)-\emph{surjective}, (TN)-\emph{bijective}) if there exists an integer $n$ such that $u_k:M_k\to N_k$ is injective (resp. surjective, bijective) for $k\geq n$.
Saying that $u$ is (TN)-injective (resp. (TN)-surjective) thus reduces to saying that $\Ker u$ (resp. $\Coker u$) satisfies (TN).
By \sref{II.2.5.4}, if $u$ is (TN)-injective (resp. (TN)-surjective, (TN)-bijective), then $\widetilde{u}$ is injective (resp. surjective, bijective);
if $u$ is (TN)-bijective, then we also say that $u$ is a (TN)-\emph{isomorphism}.
\end{env}

\begin{proposition}[2.7.3]
\label{II.2.7.3}
Let $S$ be a graded ring such that the ideal $S_+$ is of finite type, and let $M$ be a graded $S$-module.
\begin{enumerate}
  \item[{\rm(i)}] If $M$ satisfies condition (TF) then the $\sh{O}_X$-module $\widetilde{M}$ is of finite type.
  \item[{\rm(ii)}] Suppose that $M$ satisfies (TF);
    for $\widetilde{M}=0$, it is necessary and sufficient for $M$ to satisfy (TN).
\end{enumerate}
\end{proposition}

\begin{proof}
We have just seen that condition (TN) implies that $\widetilde{M}=0$.
If $M$ satisfies (TF), then the graded submodule $M'=\bigoplus_{k\geq n}M_k$, which is, by hypothesis, of finite type, is such that $M/M'$ satisfies (TN);
thus $(M/M')\supertilde$, and the exactness of the functor $\widetilde{M}$ \sref{II.2.5.4} implies that $\widetilde{M}=\widetilde{M'}$;
to prove that $\widetilde{M}$ is of finite type, we can thus reduce to the case where $M$ is \emph{of finite type} .
Since the question is local, it suffices to prove that $M_{(f)}$ is an $S_{(f)}$-module of finite type \sref[I]{I.1.3.9};
but $M^{(d)}$ is an $S^{(d)}$-module of finite type \sref{II.2.1.6}[iii], and our claim then follows from \sref{II.2.2.5}.

Now suppose that $M$ satisfies (TF) and that $\widetilde{M}=0$;
then, with the same notation as above, we have that $\widetilde{M'}=0$, and condition (TN) for $M'$ is equivalent to condition (TN) for $M$, so to prove that $\widetilde{M}=0$ implies that $M$ satisfies (TN), we can again restrict to the case where $M$ is generated by a finite number of homogeneous elements $x_i$ ($1\leq i\leq p$);
let $(f_j)_{1\leq j\leq q}$ be a system of homogeneous generators of the ideal $S_+$.
By hypothesis, $M_{(f_j)}=0$ for all $j$, and so there exists an integer $n$ such that $f_j^nx_i=0$ for any $i$ and $j$.
Let $n_j$ be the degree of $f_j$, and let $m$ be the largest value of $\sum_j r_jn_j$ for the system of finitely many integers $(r_j)$ such that $\sum_j r_j\leq nq$;
it is then clear
\oldpage[II]{40}
that, if $k>m$, then $S_kx_i=0$ for all $i$;
if $h$ is the largest of the degrees of the $x_i$, then we conclude that $M_k=0$ for $k>h+m$, which finishes the proof.
\end{proof}

\begin{corollary}[2.7.4]
\label{II.2.7.4}
Let $S$ be a graded ring such that the ideal $S_+$ is of finite type;
for $X=\Proj(S)=\emp$, it is necessary and sufficient for there to exist $n$ such that $S_k=0$ for $k\geq n$.
\end{corollary}

\begin{proof}
The condition $X=\emp$ is equivalent to $\sh{O}_X=\widetilde{S}=0$, and $S$ is a monogeneous $S$-module.
\end{proof}

\begin{theorem}[2.7.5]
\label{II.2.7.5}
Suppose that the ideal $S_+$ is generated by a finite number of homogeneous elements of degree~$1$;
let $X=\Proj(S)$.
Then, for every quasi-coherent $\sh{O}_X$-module $\sh{F}$, the canonical homomorphism $\beta:(\Gamma_\bullet(\sh{F}))\supertilde\to\sh{F}$ \sref{II.2.6.4} is an isomorphism.
\end{theorem}

\begin{proof}
If $S_+$ is generated by a finite number of elements $f_i\in S_1$, then $X$ is the union of the subspaces $\Spec(S_{(f_i)})$ \sref{II.2.3.6}, which are quasi-compact, and so $X$ is quasi-compact;
furthermore, $X$ is a scheme \sref{II.2.4.2};
by \sref[I]{I.9.3.2}, \sref{II.2.5.14.2}, and \sref{II.2.6.3}, we have, for all $f\in S_d$ ($d>0$), a canonical isomorphism $(\Gamma_\bullet(\sh{F}))_{(\alpha_d(f))}\xrightarrow{\sim}\Gamma(D_+(f),\sh{F})$;
also, by definition, $(\Gamma_\bullet(\sh{F}))_{(\alpha_d(f))}$ (where $\Gamma_\bullet(\sh{F})$ is thought of as a $\Gamma_\bullet(\sh{O}_X)$-module) is exactly $(\Gamma_\bullet(\sh{F}))_{(f)}$ (where $\Gamma_\bullet(\sh{F})$ is thought of as an $S$-module);
if we refer to the definition \sref[I]{I.9.3.1} of the above canonical isomorphism, then we see that it agrees with the homomorphism $\beta_f$, whence the theorem.
\end{proof}

\begin{remark}[2.7.6]
\label{II.2.7.6}
If we suppose that the graded ring $S$ is \emph{Noetherian}, then the condition of \sref{II.2.7.5} is satisfied \emph{ipso facto} as soon as we suppose that the ideal $S_+$ is generated by the set $S_1$ of homogeneous elements of degree~$1$.
\end{remark}

\begin{corollary}[2.7.7]
\label{II.2.7.7}
Under the hypotheses of \sref{II.2.7.5}, every quasi-coherent $\sh{O}_X$-module $\sh{F}$ is isomorphic to an $\sh{O}_X$-module of the form $\widetilde{M}$, where $M$ is a graded $S$-module.
\end{corollary}

\begin{corollary}[2.7.8]
\label{II.2.7.8}
Under the hypotheses of \sref{II.2.7.5}, every quasi-coherent $\sh{O}_X$-module $\sh{F}$ of finite type is isomorphic to an $\sh{O}_X$-module of the form $\widetilde{N}$, where $N$ is a graded $S$-module of finite type.
\end{corollary}

\begin{proof}
We can suppose that $\sh{F}=\widetilde{M}$, where $M$ is a graded $S$-module \sref{II.2.7.7}.
Let $(f_\lambda)_{\lambda\in L}$ be a system of homogeneous generators of $M$;
for every finite subset $H$ of $L$, let $M_H$ be the graded submodule of $M$ generated by the $f_\lambda$ such that $\lambda\in H$;
it is clear that $M$ is the inductive limit of its submodules $M_H$, and so $\sh{F}$ is the inductive limit of its sub-$\sh{O}_X$-modules $\widetilde{M_H}$ \sref{II.2.5.4}.
But, since $\sh{F}$ is of finite type, and since the underlying space of $X$ is quasi-compact, it follows from \sref[0]{0.5.2.3} that $\sh{F}=\widetilde{M_H}$ for some finite subset $H$ of $L$.
\end{proof}

\begin{corollary}[2.7.9]
\label{II.2.7.9}
Under the hypotheses of \sref{II.2.7.5}, let $\sh{F}$ be a quasi-coherent $\sh{O}_X$-module of finite type.
Then there exists an integer $n_0$ such that, for all $n\geq n_0$, $\sh{F}(n)$ is isomorphic to a quotient of an $\sh{O}_X$-module of the form $\sh{O}_X^k$ (for some $k>0$ depending on $n$), and is thus generated by a finite number of its sections over $X$ \sref[0]{0.5.1.1}.
\end{corollary}

\begin{proof}
By \sref{II.2.7.8}, we can suppose that $\sh{F}=\widetilde{M}$, where $M$ is a quotient of a finite direct sum of $S$-modules of the form $S(m_i)$;
by \sref{II.2.5.4}, we can thus restrict to the case where $M=S(m)$, and so $\sh{F}(n)=(S(m+n))\supertilde=\sh{O}_X(m+n)$ \sref{II.2.5.15}.
It thus suffices to prove

\oldpage{41}
  \begin{lemma}[2.7.9.1]
  \label{II.2.7.9.1}
  Under the hypotheses of \sref{II.2.7.5}, for all $n\geq0$, there exists an integer $k$ (depending on $n$) and a surjective homomorphism $\sh{O}_X^k\to\sh{O}_X(n)$.
  \end{lemma}

It suffices \sref{II.2.7.2} to show that, for suitable $k$, there is a (TN)-surjective homomorphism $u$ \emph{of degree~$0$} from the graded product $S$-module $S^k$ to $S(n)$.
But $(S(n))_0=S_n$, and, by hypothesis, $S_h=S_1^h$ for all $h>0$, and so $SS_n=S_n+S_{n+1}+\ldots$.
Since $S_n$ is an $S_0$-module of finite type (\sref{II.2.1.5} and \sref{II.2.1.6}[i]), consider a system of generators $(a_i)_{1\leq i\leq k}$ of this module;
consider the homomorphism $u$ that sends $a_i$ to the $i$-th element of the canonical basis of $S^k$ ($1\leq i\leq k$);
since $\Coker u$ can then be identified with $(S(n))_{-n}+\ldots+(S(n))_{-1}$, $u$ is indeed the desired homomorphism.
\end{proof}

\begin{corollary}[2.7.10]
\label{II.2.7.10}
Under the hypotheses of \sref{II.2.7.5}, let $\sh{F}$ be a quasi-coherent $\sh{O}_X$-module of finite type.
Then there exists an integer $n_0$ such that, for all $n\geq n_0$, $\sh{F}$ is isomorphic to a quotient of an $\sh{O}_X$-module of the form $(\sh{O}_X(-n))^k$ (for some $k$ depending on $n$).
\end{corollary}

\begin{proposition}[2.7.11]
\label{II.2.7.11}
Suppose that the hypotheses of \sref{II.2.7.5} are satisfied, and let $M$ be a graded $S$-module.
Then:
\begin{enumerate}
  \item[{\rm(i)}] The canonical homomorphism $\widetilde{\alpha}:\widetilde{M}\to(\Gamma_\bullet(\widetilde{M}))\supertilde$ is an isomorphism.
  \item[{\rm(ii)}] Let $\sh{G}$ be a quasi-coherent sub-$\sh{O}_X$-module of $\widetilde{M}$, and let $N$ be the graded sub-$S$-module of $M$ given by the  inverse image of $\Gamma_\bullet(\sh{G})$ under $\alpha$.
    Then $\widetilde{N}=\sh{G}$ (where $\widetilde{N}$ is identified, by \sref{II.2.5.4}, with a sub-$\sh{O}_X$-module of $\widetilde{M}$).
\end{enumerate}
\end{proposition}

\begin{proof}
Since $\beta:(\Gamma_\bullet(\widetilde{M}))\supertilde\to\widetilde{M}$ is an isomorphism (by \sref{II.2.7.5}), $\widetilde{\alpha}$ is the inverse isomorphism (by \sref{II.2.6.5.1}), whence (i).
Let $P$ be the graded submodule $\alpha(M)$ of $\Gamma_\bullet(\widetilde{M})$;
since $\widetilde{M}$ is an exact functor \sref{II.2.5.4}, the image of $\widetilde{M}$ under $\widetilde{\alpha}$ is equal to $\widetilde{P}$, and so, by (i), $\widetilde{P}=(\Gamma_\bullet(\widetilde{M}))\supertilde$.
Set $Q=\Gamma_\bullet(\sh{G})\cap P$;
by the above, and by \sref{II.2.5.4}, we have that $\widetilde{Q}=(\Gamma_\bullet(\sh{G}))\supertilde$, and so the restriction of $\beta$ to $\widetilde{Q}$ is an \emph{isomorphism} from this $\sh{O}_X$-module to $G$ by \sref{II.2.7.5}.
But, by the definition of $N$, and by \sref{II.2.5.4}, the restriction of the isomorphism $\widetilde{\alpha}$ to $\widetilde{N}$ is an isomorphism from $\widetilde{N}$ to $\widetilde{Q}$, whence the conclusion, by \sref{II.2.6.5.1}.
\end{proof}


\subsection{Functorial behaviour}
\label{subsection:II.2.8}

\begin{env}[2.8.1]
\label{II.2.8.1}
Let $S$ and $S'$ be positively graded rings, and $\varphi:S'\to S$ a homomorphism of graded rings.
We denote by $G(\varphi)$ the open subset of $X=\Proj(S)$ given by the complement of $V_+(\varphi(S'_+))$, or, equivalently, the union of the $D_+(\varphi(f'))$ where $f'$ runs over the set of homogeneous elements of $S'_+$.
The restriction to $G(\varphi)$ of the continuous map ${}^a\varphi$ from $\Spec(S)$ to $\Spec(S')$ \sref[I]{I.1.2.1} is thus a continuous map from $G(\varphi)$ to $\Proj(S')$, which we again denote, with an abuse of language, by ${}^a\varphi$.
If $f'\in S'_+$ is homogeneous, then
\[
\label{II.2.8.1.1}
  {}^a\varphi^{-1}(D_+(f')) = D_+(\varphi(f'))
\tag{2.8.1.1}
\]
taking into account the fact that ${}^a\varphi$ sends $G(\varphi)$ to $\Proj(S')$, as well as \sref[I]{I.1.2.2.2}.
The homomorphism $\varphi$ also canonically defines (with the same notation) a homomorphism of graded rings $S'_{f'}\to S_f$, whence, by restriction to the degree~$0$ elements,
\oldpage[II]{42}
a homomorphism $S'_{(f')}\to S_{(f)}$, which we denote by $\varphi_{(f)}$;
there is a corresponding \sref[I]{I.1.6.1} morphism of affine schemes $({}^a\varphi_{(f)},\widetilde{\varphi}_{(f)}):\Spec(S_{(f)})\to\Spec(S'_{(f')})$.
If we canonically identify $\Spec(S_{(f)})$ with the scheme induced by $\Proj(S)$ on $D_+(f)$ \sref{II.2.3.6}, then we have defined a morphism $\Phi_f:D_+(f)\to D_+(f')$, and ${}^a\varphi_{(f)}$ is identified with the restriction of ${}^a\varphi$ to $D_+(f)$.
It is also immediate that, if $g'$ is another homogeneous element of $S'_+$, and $g=\varphi(g')$, then the diagram
\[
  \xymatrix{
    D_+(f) \ar[r]^{\Phi_f}
    & D_+(f')
  \\D_+(fg) \ar[u] \ar[r]_{\Phi_{fg}}
    & D_+(f'g') \ar[u]
  }
\]
commutes, by the fact that the diagram
\[
  \xymatrix{
    S'_{(f')} \ar[r]^{\varphi_{(f)}} \ar[d]_{\omega_{f'g',f'}}
    & S_{(f)} \ar[d]^{\omega_{fg,f}}
  \\S'_{(f'g')} \ar[r]_{\varphi_{(fg)}}
    & S_{(fg)}
  }
\]
commutes.
Taking the definition of $G(\varphi)$, along with \sref{II.2.3.3.2}, we thus see that:
\end{env}

\begin{proposition}[2.8.2]
\label{II.2.8.2}
Given a homomorphism of graded rings $\varphi: S'\to S$, there exists exactly one morphism $({}^a\varphi,\widetilde{\varphi})$ from the induced prescheme $G(\varphi)$ to $\Proj(S')$ (said to be \emph{associated to $\varphi$}, and denoted by $\Proj(\varphi)$) such that, for every homogeneous element $f'\in S'_+$, the restriction of this morphism to $D_+(\varphi(f'))$ agrees with the morphism associated to the homomorphism $S'_{(f')}\to S_{(\varphi(f'))}$ corresponding to $\varphi$.
\end{proposition}

\begin{proof}
With the above notation, if $f'\in S'_d$, then the diagram
\[
\label{II.2.8.2.1}
  \xymatrix{
    S'_{(f')} \ar[r]^{\varphi_{(f)}} \ar[d]_{\sim}
    & S_{(f)} \ar[d]^{\sim}
  \\{S'}^{(d)}/(f'-1){S'}^{(d)} \ar[r]
    & S^{(d)}/(f-1)S^{(d)}
  }
\tag{2.8.2.1}
\]
commutes (the vertical arrows being the isomorphisms \sref{II.2.2.5}).
\end{proof}

\begin{corollary}[2.8.3]
\label{II.2.8.3}
\begin{enumerate}
  \item[{\rm(i)}] The morphism $\Proj(\varphi)$ is affine.
  \item[{\rm(ii)}] If $\Ker(\varphi)$ is nilpotent (and, in particular, if $\varphi$ is injective), then the morphism $\Proj(\varphi)$ is dominant.
\end{enumerate}
\end{corollary}

\begin{proof}
Claim~(i) is an immediate consequence of \sref{II.2.8.2} and \sref{II.2.8.1.1}.
Claim~(ii) follows since, if $\Ker(\varphi)$ is nilpotent, then, for every homogeneous $f'$ in $S'_+$, we immediately see that $\Ker(\varphi_f)$ (with $f=\varphi(f')$) is nilpotent, and thus so too is $\Ker(\varphi_{(f)})$;
we then apply \sref{II.2.8.2} and \sref[I]{I.1.2.7}
\end{proof}

\oldpage[II]{43}
We note that there are, in general, morphisms from $\Proj(S)$ to $\Proj(S')$ that are not affine, and that thus do not come from homomorphisms of graded rings $S'\to S$;
an example is the structure morphism $\Proj(S)\to\Spec(A)$ when $A$ is a field ($\Spec(A)$ thus being identified with $\Proj(A[T])$, cf.~\sref{II.3.1.7});
indeed, this follows from \sref[I]{I.2.3.2}.

\begin{env}[2.8.4]
\label{II.2.8.4}
Let $S''$ be another positively graded ring, and $\varphi':S''\to S'$ a homomorphism of graded rings, and set $\varphi''=\varphi\circ\varphi'$.
By \sref{II.2.8.1.1} and the formula ${}^a\varphi''=({}^a\varphi')\circ({}^a\varphi)$, we immediately see that $G(\varphi'')\subset G(\varphi)$, and that, if $\Phi$, $\Phi'$, and $\Phi''$ are the morphisms associated to $\varphi$, $\varphi'$, and $\varphi''$ (respectively), then $\Phi''=\Phi'\circ(\Phi|G(\varphi''))$.
\end{env}

\begin{env}[2.8.5]
\label{II.2.8.5}
Suppose that $S$ (resp. $S'$) is a graded $A$-algebra (resp. a graded $A'$-algebra), and let $\psi:A'\to A$ be a ring homomorphism such that the diagram
\[
  \xymatrix{
    A' \ar[r]^{\psi} \ar[d]
    & A \ar[d]
  \\S' \ar[r]_{\varphi}
    & S
  }
\]
commutes.
We can then consider $G(\varphi)$ (resp. $\Proj(S')$) as a scheme over $\Spec(A)$ ($resp. \Spec(A')$);
if $\Phi$ (resp. $\Psi$) is the morphism associated to $\varphi$ (resp. $\psi$), then the diagram
\[
  \xymatrix{
    G(\varphi) \ar[r]^{\Phi} \ar[d]
    & \Proj(S') \ar[d]
  \\\Spec(A) \ar[r]_{\Psi}
    & \Spec(A')
  }
\]
commutes: it suffices to prove this for the restriction of $\Phi$ to $D_+(f)$, where $f=\varphi(f')$, with $f'$ homogeneous in $S'_+$;
this then follows from the fact that the diagram
\[
  \xymatrix{
    A' \ar[r]^{\psi} \ar[d]
    & A \ar[d]
  \\S'_{(f')} \ar[r]_{\varphi_{(f)}}
    & S_{(f)}
  }
\]
commutes.
\end{env}

\begin{env}[2.8.6]
\label{II.2.8.6}
Now let $M$ be a graded $S$-module, and consider the $S'$-module $M_{[\varphi]}$, which is evidently graded.
Let $f'$ be homogeneous in $S'_+$, and let $f=\varphi(f')$;
we know \sref[0]{0.1.5.2} that there is a canonical isomorphism $(M_{[\varphi]})_{f'}\xrightarrow{\sim}(M_f)_{[\varphi_f]}$, and it is immediate that this isomorphism preserves degree, whence a canonical isomorphism $(M_{[\varphi]})_{(f')}\xrightarrow{\sim}(M_{(f)})_{[\varphi_{(f)}]}$.
To this isomorphism, there canonically corresponds an isomorphism of sheaves $(M_{[\varphi]})\supertilde|D_+(f')\xrightarrow{\sim}(\Phi_f)_*(\widetilde(M)|D_+(f))$ (\sref{II.2.5.2} and \sref[I]{I.1.6.3}).
Furthemore,
\oldpage{43}
if $g'$ is another homogeneous element of $S'_+$, and $g=\varphi(g')$, then the diagram
\[
  \xymatrix{
    (M_{[\varphi]})_{(f')} \ar[r]^{\sim} \ar[d]
    & (M_{(f)})_{[\varphi_{(f)}]} \ar[d]
  \\(M_{[\varphi]})_{(f'g')} \ar[r]^{\sim}
    & (M_{(fg)})_{[\varphi_{(fg)}]}
  }
\]
commutes, whence we immediately conclude that the isomorphism
\[
  (M_{[\varphi]})\supertilde|D_+(f'g') \xrightarrow{\sim} (\Phi_{fg})_*(\widetilde{M}|D_+(fg))
\]
is the restriction to $D_+(f'g')$ of the isomorphism $(M_{[\varphi]})\supertilde|D_+(f')\xrightarrow{\sim}(\Phi_f)_*(\widetilde{M}|D_+(f))$.
Since $\Phi_f$ is the restriction to $D_+(f)$ of the morphism $\Phi$, we see that, taking \sref{II.2.8.1.1} into account, and setting $X'=\Proj(S)'$:
\end{env}

\begin{proposition}[2.8.7]
\label{II.2.8.7}
There exists a canonical functorial isomorphism from the $\sh{O}_{X'}$-module $(M_{[\varphi]})^\supertilde$ to the $\sh{O}_{X'}$-module $\Phi_*(\widetilde{M}|G(\varphi))$.
\end{proposition}

We thus immediately deduce a canonical functorial map from the set of $\varphi$-morphisms $M'\to M$ from a graded $S'$-module to the graded $S$-module $M$, to the set of $\Phi$-morphisms $\widetilde{M'}\to\widetilde{M}|G(\varphi)$.
With the notation of \sref{II.2.8.4}, if $M''$ is a graded $S''$-module, then, to the composition of a $\varphi$-morphism $M'\to M$ and a $\varphi'$-morphism $M''\to M'$, canonically corresponds the composition of $\widetilde{M'}G(\varphi')\to\widetilde{M}|G(\varphi'')$ and $\widetilde{M''}\to\widetilde{M'}|G(\varphi')$.

\begin{proposition}[2.8.8]
\label{II.2.8.8}
Under the hypotheses of \sref{II.2.8.1}, let $M'$ be a graded $S'$-module.
Then there exists a canonical functorial homomorphism $\nu$ from the $(\sh{O}_X|G(\varphi))$-module $\Phi^*(\widetilde{M'})$ to the $(\sh{O}_X|G(\varphi))$-module $(M'\otimes_{S'}S)\supertilde|G(\varphi)$.
If the ideal $S'_+$ is generated by $S'_1$, then $\nu$ is an isomorphism.
\end{proposition}

\begin{proof}
Indeed, for $f'\in S'_d$ ($d>0$), we define a canonical functorial homomorphism of $S_{(f)}$-modules (where $f=\varphi(f')$)
\[
\label{II.2.8.8.1}
  \nu_f: M'_{(f')}\otimes_{S'_{(f')}}S_{(f)} \to (M'\otimes_{S'}S)_{(f)}
\tag{2.8.8.1}
\]
by composing the homomorphism $M'_{(f')}\otimes_{S'_{(f')}}S_{(f)}\to M'_{f'}\otimes_{S'_{f'}}S_f$ and the canonical isomorphism $M'_{f'}\otimes_{S'_{f'}}S_f\xrightarrow{\sim}(M'\otimes_{S'}S)_f$ \sref[0]{0.1.5.4}, and noting that the latter preserves degrees.
We can immediately verify the compatibility of $\nu_f$ with the restriction operators from $D_+(f)$ to $D_+(fg)$ (for any $g'\in S'_+$ and $g=\varphi(g')$), whence the definition of the homomorphism
\[
  \nu: \Phi^*(\widetilde{M'}) \to (M'\otimes_{S'}S)\supertilde|G(\varphi)
\]
taking \sref[I]{I.1.6.5} into account.
To prove the second claim, it suffices to show that $\nu_f$ is an isomorphism for all $f'\in S_1$, since $G(\varphi)$ is then a union of the $D_+(\varphi(f'))$.
We first define a $\bb{Z}$-bilinear $M'_m\times S_n\to M'_{(f')}\otimes_{S'_{(f')}}S_{(f)}$ by sending $(x',s)$ to the element $(x'/{f'}^m)\otimes(s/f^n)$ (with the convention that $x'/{f'}^m$ is ${f'}^{-m}x'/1$ when $m<0$).
\oldpage[II]{45}
We claim that, in the proof of \sref{II.2.5.13}, this map gives rise to a di-homomorphism of modules
\[
  \eta_f: M'\otimes_{S'}S \to M'_{(f')}\otimes_{S_{(f')}}S_{(f)}.
\]
Furthermore, if, for $r>0$, we have $f^r\sum_i(x'_i\otimes s_i)=0$, then this can also be written as $\sum_i({f'}^rx'_i\otimes s_i)=0$, whence, by \sref[0]{0.1.5.4}, $\sum_i({f'}^rx_i/{f'}^{m_i+r})\otimes(s_i/f^{n_i})=0$, i.e. $\eta_f(\sum_i x_i\otimes y_i)0=$, which proves that $\eta_f$ factors as $M'\otimes_{S'}S\to(M'\otimes_{S'}S)_f\xrightarrow{\eta'_f}M'_{(f')}\otimes_{S'_{(f')}}S_{(f)}$;
we finally can prove that $\eta'_f$ and $\nu_f$ are inverse isomorphisms to one another.

In particular, it follows from \sref{II.2.1.2.1} that we have a canonical homomorphism
\[
\label{II.2.8.8.2}
  \Phi^*(\sh{O}_{X'}(n)) \xrightarrow{\sim} \sh{O}_X(n)|G(\varphi)
\tag{2.8.8.2}
\]
for all $n\in\bb{Z}$.
\end{proof}

\begin{env}[2.8.9]
\label{II.2.8.9}
Let $A$ and $A'$ be rings, and $\psi:A'\to A$ a ring homomorphism, defining a morphism $\Psi:\Spec(A)\to\Spec(A')$.
Let $S'$ be a positively graded $A'$-algebra, and set $S=S'\otimes_{A'}A$, which is evidently an $A$-algebra graded by the $S'_n\otimes_{A'}A$;
the map $\varphi:s'\to s'\otimes1$ is then a graded ring homomorphism that makes the diagram \sref{II.2.8.5.1} commute.
Since $S_+$ is here the $A$-module generated by $\varphi(S'_+)$, we have $G(\varphi)=\Proj(S)=X$;
whence, setting $X'=\Proj(S')$, we have the commutative diagram
\[
\label{II.2.8.9.1}
  \xymatrix{
    X \ar[r]^{\Phi} \ar[d]_p
    & X' \ar[d]
  \\Y \ar[r]_{\Psi}
    & Y'
  }
\tag{2.8.9.1}
\]

Now let $M'$ be a graded $S'$-module, and set $M=M'\otimes_{A'}A=M'\otimes_{S'}S$.
Under these conditions:
\end{env}

\begin{proposition}[2.8.10]
\label{II.2.8.10}
The diagram \sref{II.2.8.9.1} identifies the scheme $X$ with the product $X'\times_{Y'}Y$;
furthermore, the canonical homomorphism $\nu:\Phi^*(\widetilde{M'})\to\widetilde{M}$ \sref{II.2.8.8} is an isomorphism.
\end{proposition}

\begin{proof}
The first claim will be proven if we show that, for every homogenous $f'$ in $S'_+$, setting $f=\varphi(f')$, the restrictions of $\Phi$ and $p$ to $D_+(f)$ identify this scheme with the product $D_+(f')\times_{Y'}Y$ \sref[I]{I.3.2.6.2};
in other words, it suffices \sref[I]{I.3.2.2} to prove that $S_{(f)}$ is canonically identified with $S_f\xrightarrow{\sim}S'_{f'}\otimes_{A'}A$, which is immediate by the existence of the canonical isomorphism $S_f\xrightarrow{\sim}S'_{f'}\otimes_{A'}A$ that preserves degrees \sref[0]{0.1.5.4}.
The second claim then follows from the fact that $M'_{(f')}\otimes_{S'_{(f')}}S_{(f)}$ can be identified, by the above, with $M'_{(f')}\otimes_{A'}A$, and this can be identified with $M_{(f)}$, since $M_f$ is canonically identified with $M'_{f'}\otimes_{A'}A$ by an isomorphism that preserves degrees.
\end{proof}

\begin{corollary}[2.8.11]
\label{II.2.8.11}
For every integer $n\in\bb{Z}$, $\widetilde{M}(n)$ can be identified with $\Phi^*(\widetilde{M'}(n))=\widetilde{M'}(n)\otimes_{Y'}\sh{O}_Y$;
in particular, $\sh{O}_X(n)=\Phi^*(\sh{O}_{X'}(n))=\sh{O}_{X'}(n)\otimes_{Y'}\sh{O}_Y$.
\end{corollary}

\begin{proof}
This follows from \sref{II.2.8.10} and \sref{II.2.5.15}.
\end{proof}

\oldpage[II]{46}

\begin{env}[2.8.12]
\label{II.2.8.12}
Under the hypotheses of \sref{II.2.8.9}, for $f'\in S'_d$ ($d>0$) and $f=\varphi(f')$, the diagram
\[
  \xymatrix{
    M'_{(f')} \ar[r]^-{\sim} \ar[d]
    & {M'}^{(d)}/(f'-1){M'}^{(d)} \ar[d]
  \\M_{(f)} \ar[r]^-{\sim}
    & M^{(d)}/(f-1)M^{(d)}
  }
\]
(cf. \sref{II.2.2.5}) commutes.
\end{env}

\begin{env}[2.8.13]
\label{II.2.8.13}
Keep the notation and hypotheses of \sref{II.2.8.9}, and let $\sh{F}'$ be an $\sh{O}_{X'}$-module;
if we set $\sh{F}=\Phi^*(\sh{F}')$, then, for all $n\in\bb{Z}$, we have $\sh{F}(n)=\Phi^*(\sh{F}'(n))$, by \sref{II.2.8.11} and \sref[0]{0.4.3.3}.
Then \sref[0]{0.3.7.1} we have a canonical homomorphism
\[
  \Gamma(\rho): \Gamma(X',\sh{F}'(n)) \to \Gamma(X,\sh{F}(n))
\]
which gives a canonical di-homomorphism of graded modules
\[
  \Gamma_\bullet(\sh{F}') \to \Gamma_\bullet(\sh{F}).
\]

Suppose that the ideal $S_+$ is generated by $S_1$, and that $\sh{F}'=\widetilde{M'}$, thus $\sh{F}=\widetilde{M}$ with $M=M'\otimes_{A'}A$.
If $f'$ is homogeneous in $S'_+$, and $f=\varphi(f')$, then we have seen that $M_{(f)}=M'_{(f')}\otimes_{A'}A$, and the diagram
\[
  \xymatrix{
    M'_0 \ar[r] \ar[d]
    & M'_{(f')} \ar[d]
    & =\Gamma(D_+(f'),\widetilde{M'})
  \\M_0 \ar[r]
    & M_{(f)}
    & =\Gamma(D_+(f),\widetilde{M})
  }
\]
thus commutes;
we immediately conclude from this remark, and from the definition of the homomorphism $\alpha$ \sref{II.2.6.2}, that the diagram
\[
\label{II.2.8.13.1}
  \xymatrix{
    M' \ar[r]^-{\alpha_{M'}} \ar[d]
    & \Gamma_\bullet(\widetilde{M'}) \ar[d]
  \\M \ar[r]_-{\alpha_M}
    & \Gamma_\bullet(\widetilde{M})
  }
\tag{2.8.13.1}
\]
commutes.
Similarly, the diagram
\[
\label{II.2.8.13.2}
  \xymatrix{
    (\Gamma_\bullet(\sh{F}'))\supertilde \ar[r]^-{\beta_{\sh{F}'}} \ar[d]
    & \sh{F}' \ar[d]
  \\(\Gamma_\bullet(\sh{F}))\supertilde \ar[r]_-{\beta_{\sh{F}}}
    & \sh{F}
  }
\tag{2.8.13.2}
\]
commutes (the vertical arrow on the right being the canonical $\Phi$-morphism $\sh{F}'\to\Phi^*(\sh{F}')=\sh{F}$).
\end{env}

\oldpage[II]{47}

\begin{env}[2.8.14]
\label{II.2.8.14}
Still keeping the notation and hypotheses of \sref{II.2.8.9}, let $N'$ be another graded $S'$-module, and let $N=N'\otimes_{A'}A$.
It is immediate that the canonical di-homomorphisms $M'\to M$ and $N'\to N$ give a di-homomorphism $M'\otimes_{S'}N'\to M\otimes_S N$ (with respect to the canonical ring homomorphism $S'\to S$), and thus also an $S$-homomorphism $(M'\otimes_{S'}N')\otimes_{A'}A\to M\otimes_S N$ of degree~$0$, to which corresponds (taking \sref{II.2.8.10} into account) a homomorphism of $\sh{O}_X$-modules
\[
\label{II.2.8.14.1}
  \Phi^*((M'\otimes_{S'}N')\supertilde) \to (M\otimes_S N)\supertilde.
\tag{2.8.14.1}
\]

Furthermore, we can immediately verify that the diagram
\[
\label{II.2.8.14.2}
  \xymatrix{
    \Phi^*(\widetilde{M'}\otimes_{\sh{O}_{X'}}\widetilde{N'}) \ar[r]^\sim \ar[d]_{\Phi^*(\lambda)}
    & \widetilde{M}\otimes_{\sh{O}_X}\widetilde{N} \ar[d]^\lambda
    & =\Phi^*(\widetilde{M'})\otimes_{\sh{O}_X}\Phi^*(\widetilde{N'})
  \\\Phi^*((M'\otimes_{S'}N')\supertilde) \ar[r]
    & (M\otimes_S N)\supertilde
  }
\tag{2.8.14.2}
\]
commutes, with the first line being the canonical isomorphism \sref[0]{0.4.3.3}.
If the ideal $S'_+$ is generated by $S'_1$, then it is clear that $S_+$ is generated by $S_1$, and the two vertical arrows of \sref{II.2.8.14.2} are then isomorphisms \sref{II.2.5.13};
it is thus also the case for \sref{II.2.8.14.1}.

We similarly have a canonical di-homomorphism $\Hom_{S'}(M',N')\to\Hom_S(M,N)$ by sending a homomorphism $u'$ of degree~$k$ to the homomorphism $u'\otimes1$, which is also of degree~$k$;
from this, we again deduce a $S$-homomorphism of degree~$0$
\[
  (\Hom_{S'}(M',N'))\otimes_{A'}A \to \Hom_S(M,N)
\]
whence a homomorphism of $\sh{O}_X$-modules
\[
\label{II.2.8.14.3}
  \Phi^*((\Hom_{S'}(M',N'))\supertilde) \to (\Hom_S(M,N))^\supertilde.
\tag{2.8.14.3}
\]

Furthermore, the diagram
\[
  \xymatrix{
    \Phi^*((\Hom_{S'}(M',N'))\supertilde) \ar[r] \ar[d]_{\Phi^*(\mu)}
    & (\Hom_S(M,N))^\supertilde \ar[d]^\mu
  \\\Phi^*(\shHom_{\sh{O}_{X'}}(\widetilde{M'},\widetilde{N'})) \ar[r]
    & \shHom_{\sh{O}_X}(\widetilde{M},\widetilde{N})
  }
\]
commutes (the second horizontal line being the canonical homomorphism \sref[0]{0.4.4.6}).
\end{env}

\oldpage[II]{48}

\begin{env}[2.8.15]
\label{II.2.8.15}
With the notation and hypotheses of \sref{II.2.8.1}, it follows from \sref{II.2.4.7} that we do not change the morphism $\Phi$, up to isomorphism, when we replace $S_0$ and $S'_0$ by $\bb{Z}$, and $\varphi_0$ by the identity map, and thus when we replace $S$ and $S'$ by $S^{(d)}$ and ${S'}^{(d)}$ (respectively) ($d>0$), and $\varphi$ by its restriction $\varphi^{(d)}$ to $S^{(d)}$.
\end{env}


\subsection{Closed subschemes of a scheme $\operatorname{Proj}(S)$}
\label{subsection:II.2.9}

\begin{env}[2.9.1]
\label{II.2.9.1}
If $\varphi:S\to S'$ is a homomorphism of graded rings, then we say that $\varphi$ is (TN)-\emph{surjective} (resp. (TN)-\emph{injective}, (TN)-\emph{bijective}) if there exists an integer $n$ such that, for $k\geq n$, $\varphi_k:S_k\to S'_k$ is \emph{surjective} (resp. \emph{injective}, \emph{bijective}).
Instead of saying that $\varphi$ is (TN)-bijective, we sometimes say that it is a (TN)-\emph{isomorphism}.
\end{env}

\begin{proposition}[2.9.2]
\label{II.2.9.2}
Let $S$ be a positively graded ring, and let $X=\Proj(S)$.
\begin{enumerate}
  \item[\rm{(i)}] If $\varphi:S\to S'$ is a (TN)-surjective homomorphism of graded rings, then the corresponding morphism $\Phi$ \sref{II.2.8.1} is defined on the whole of $\Proj(S')$, and is a closed immersion of $\Proj(S')$ into $X$.
    If $\mathfrak{J}$ is the kernel of $\varphi$, then the closed subscheme of $X$ associated to $\Phi$ is defined by the quasi-coherent sheaf of ideals $\widetilde{\mathfrak{J}}$ of $\sh{O}_X$.
  \item[\rm{(ii)}] Suppose further that the ideal $S_+$ is generated by a finite number of homogeneous elements of degree~$1$.
    Let $X'$ be a closed subscheme of $X$ defined by a quasi-coherent sheaf of ideals $\sh{J}$ of $\sh{O}_X$.
    Let $\mathfrak{J}$ be the graded ideal of $S$ given by the inverse image of $\Gamma_\bullet(\sh{J})$ under the canonical homomorphism $\alpha:S\to\Gamma_\bullet(\sh{O}_X)$ \sref{II.2.6.2}, and set $S'=S/\mathfrak{J}$.
    Then $X'$ is the subscheme associated to the closed immersion $\Proj(S')\to X$ corresponding to the canonical homomorphism of graded rings $S\to S'$.
\end{enumerate}
\end{proposition}

\begin{proof}
\begin{enumerate}
  \item[\rm{(i)}] We can suppose that $\varphi$ is surjective \sref{II.2.9.1}.
    Since, by hypothesis, $\varphi(S_+)$ generates $S'_+$, we have $G(\varphi)=\Proj(S')$.
    Now, the second claim can be checked locally on $X$;
    so let $f$ be a homogeneous element of $S_+$, and set $f'=\varphi(f)$.
    Since $\varphi$ is a surjective homomorphism of graded rings, we immediately see that $\varphi_{(f')}:S_{(f)}\to S'_{(f')}$ is surjective, and that its kernel is $\mathfrak{J}_{(f)}$, which proves (i) \sref[I]{I.4.2.3}.
  \item[\rm{(ii)}] By (i), we are led to proving that the homomorphism $\widetilde{j}:\widetilde{\mathfrak{J}}\to\sh{O}_X$ induced by the canonical injection $j:\mathfrak{J}\to S$ is an isomorphism from $\widetilde{\mathfrak{J}}$ to $\sh{J}$, which follows from \sref{II.2.7.11}.
\end{enumerate}
\end{proof}

We note that $\mathfrak{J}$ is the \emph{largest} of the graded ideals $\mathfrak{J}'$ of $S$ such that $\widetilde{j}(\widetilde{\mathfrak{J'}})=\sh{J}$, since we can immediately show, using the definitions \sref{II.2.6.2}, that this equation implies that $\alpha(\mathfrak{J}')\subset\Gamma_\bullet(\sh{J})$.

\begin{corollary}[2.9.3]
\label{II.2.9.3}
Suppose that the hypotheses of \sref{II.2.9.2}[(i)] are satisfied, and further that the ideal $S_+$ is generated by $S_1$;
then $\Phi^*((S(n))\supertilde)$ is canonically isomorphic to $(S'(n))\supertilde$ for all $n\in\bb{Z}$, and so $\Phi^*(\sh{F}(n))$ is canonically isomorphic to $\Phi^*(\sh{F})(n)$ for every $\sh{O}_X$-module $\sh{F}$.
\end{corollary}

\begin{proof}
This is a particular case of \sref{II.2.8.8}, taking \sref{II.2.5.10.2} into account.
\end{proof}

\begin{corollary}[2.9.4]
\label{II.2.9.4}
Suppose that the hypotheses of \sref{II.2.9.2}[(ii)] are satisfied.
For the closed sub-prescheme $X'$ of $X$ to be integral, it is necessary and sufficient for the graded ideal $\mathfrak{J}$ to be prime in $S$.
\end{corollary}

\oldpage[II]{49}

\begin{proof}
Since $X'$ is isomorphic to $\Proj(S/\mathfrak{J})$, the condition is sufficient by \sref{II.2.4.4}.
To see that it is necessary, consider the exact sequence $0\to\sh{J}\to\sh{O}_X\to\sh{O}_X/\sh{J}$, which gives the exact sequence
\[
  0 \to \Gamma_\bullet(\sh{J}) \to \Gamma_\bullet(\sh{O}_X) \to \Gamma_\bullet(\sh{O}_X/\sh{J}).
\]

It suffices to prove that, if $f\in S_m$ and $g\in S_n$ are such that the image in $\Gamma_\bullet(\sh{O}_X/\sh{J})$ of $\alpha_{n+m}(fg)$ is zero, then the image of either $\alpha_m(f)$ or $\alpha_n(g)$ is zero.
But, by definition, these images are sections of invertible $(\sh{O}_X/\sh{J})$-modules $\sh{L}=(\sh{O}_X/\sh{J})(m)$ and $\sh{L}'=(\sh{O}_X/\sh{J})(n)$ over the integral scheme $X'$;
the hypothesis implies that the product of these two sections is zero in $\sh{L}\otimes\sh{L}'$ (\sref{II.2.9.3} and \sref{II.2.5.14.1}), and so one of them is zero by \sref[I]{I.7.4.4}.
\end{proof}

\begin{corollary}[2.9.5]
\label{II.2.9.5}
Let $A$ be a ring, $M$ an $A$-module, $S$ a graded $A$-algebra generated by the set $S_1$ of homogeneous elements of degree~$1$, $u:M\to S_1$ a surjective homomorphism of $A$-modules, and $\overline{u}:\bb{S}(M)\to S$ the homomorphism (of $A$-algebras) from the symmetric algebra $\bb{S}(M)$ of $M$ to $S$ that extends $u$.
Then the morphism corresponding to $\overline{u}$ is a closed immersion of $\Proj(S)$ into $\Proj(\bb{S}(M))$.
\end{corollary}

\begin{proof}
Indeed, $\overline{u}$ is surjective by hypothesis, and so it suffices to apply \sref{II.2.9.2}
\end{proof}

\section{Homogeneous spectrum of a sheaf of graded algebras}
\label{section:II.3}


\subsection{Homogeneous spectrum of a quasi-coherent graded $\mathcal{O}_Y$-algebra}
\label{subsection:II.3.1}

\begin{env}[3.1.1]
\label{II.3.1.1}
Let $Y$ be a prescheme, $\sh{S}$ a graded $\sh{O}_Y$-algebra, and $\sh{M}$ a graded $\sh{S}$-module.
If $\sh{S}$ is \emph{quasi-coherent}, then each of its homogenous components $\sh{S}_n$ is a \emph{quasi-coherent} $\sh{O}_Y$-module, since they are the images of $\sh{S}$ under a homomorphism from $\sh{S}$ to itself (\sref[I]{I.1.3.8} and \sref[I]{I.1.3.9});
similarly, if $\sh{M}$ is quasi-coherent as an $\sh{O}_Y$-module, then its homogenous components $\sh{M}_n$ are also quasi-coherent, and the converse is also true.
For an integer $d>0$, we denote by $\sh{S}^{(d)}$ the direct sum of the $\sh{O}_Y$-modules $\sh{S}_{nd}$ (for $n\in\bb{Z}$), which is quasi-coherent if $\sh{S}$ is \sref[I]{I.1.3.9};
for every integer $k$ such that $0\leq k\leq d-1$, we denote by $\sh{M}^{(d,k)}$ (or $\sh{M}^{(d)}$, for $k=0$) the direct sum of the $\sh{M}_{nd+k}$ (for $n\in\bb{Z}$), which is a graded $\sh{S}^{(d)}$-module, and quasi-coherent if both $\sh{S}$ and $\sh{M}$ are quasi-coherent \sref[I]{I.9.6.1}.
We denote by $\sh{M}(n)$ the graded $\sh{S}$-module such that $(\sh{M}(n))_k=\sh{M}_{n+k}$ for all $k\in\bb{Z}$;
if $\sh{S}$ and $\sh{M}$ are quasi-coherent, then $\sh{M}(n)$ is a quasi-coherent graded $\sh{S}$-module \sref[I]{I.9.6.1}.

We say that $\sh{M}$ is a graded $\sh{S}$-module \emph{of finite type} (resp. admitting a \emph{finite presentation})  if, for all $y\in Y$, there exists an open neighbourhood $U$ of $y$, along with integers $n_i$ (resp. integers $m_i$ and $n_j$) such that there is a surjective degree~$0$ homomorphism $\bigoplus_{i=1}^r(\sh{S}(n_i)|U)\to\sh{M}|U$ (resp. such that $\sh{M}|U$ is isomorphic to the cokernel of a degree~$0$ homomorphism $\bigoplus_{i=1}^r(\sh{S}(m_i)|U)\to\bigoplus_{j 1}^s(\sh{S}(n_J)|U)$).

Let $U$ be an affine open of $Y$, with ring $A=\Gamma(U,\sh{O}_Y)$;
by hypothesis, the graded $(\sh{O}_Y|U)$-algebra $\sh{S}|U$ is isomorphic to $\widetilde{S}$, where $S=\Gamma(U,\sh{S})$ is a graded $A$-algebra \sref[I]{I.1.4.3};
\oldpage[II]{50}
set $X_U=\Proj(\Gamma(U,\sh{S}))$.
Let $U'\subset U$ be another affine open of $Y$, with ring $A'$, and let $j:U'\to U$ be the canonical injection, which corresponds to the restriction homomorphism $A\to A'$;
we have that $\sh{S}|U'=j^*(\sh{S}|U)$, and so $S'=\Gamma(U',\sh{S})$ is canonically identified with $X_U\times_U U'$, and thus also with $f_U^{-1}(U')$, where we denote by $f_U$ the structure morphism $X_U\to U$ \sref[I]{I.4.4.1}.
We denote by $\sigma_{U',U}$ the canonical isomorphism $f_U^{-1}(U')\simto X_{U'}$ thus defined, and by $\rho_{U',U}$ the open immersion $X_{U'}\to X_U$ obtained by composing $\sigma_{U',U}^{-1}$ with the canonical injection $f_U^{-1}(U')\to X_U$.
It is immediate that, if $U''\subset U'$ is another affine open of $Y$, then $\rho_{U'',U}=\rho_{U'',U'}\circ\rho_{U',U}$.
\end{env}

\begin{proposition}[3.1.2]
\label{II.3.1.2}
Let $Y$ be a prescheme.
For every quasi-coherent positively graded $\sh{O}_Y$-algebra, there exists exactly one (up to $Y$-isomorphism) prescheme $X$ over $Y$ satisfying the following property:
if $f:X\to Y$ is the structure morphism, then, for every affine open $U$ of $Y$, there exists an \emph{isomorphism} $\eta_U$ from the induced prescheme $f^{-1}(U)$ to $X_U=\Proj(\Gamma(U,\sh{S}))$ such that, if $V$ is another affine open of $Y$ that is contained in $U$, then the diagram
\[
\label{II.3.1.2.1}
  \xymatrix{
    f^{-1}(V) \ar[r]^{\eta_V} \ar[d]
    & X_V \ar[d]^{\rho_{V,U}}
  \\f^{-1}(U) \ar[r]_{\eta_V}
    & X_U
  }
\tag{3.1.2.1}
\]
commutes.
\end{proposition}

\begin{proof}
Given affine opens $U$ and $V$ of $Y$, let $X_{U,V}$ be the prescheme induced on $f_U^{-1}(U\cap V)$ by $X_U$;
we are going to define a $Y$-isomorphism $\theta_{U,V}:X_{V,U}\simto X_{U,V}$.
For this, we consider an affine open $W\subset U\cap V$:
by composing the isomorphisms
\[
  f_U^{-1}(W)
  \xrightarrow{\sigma_{W,U}} X_W
  \xrightarrow{\sigma_{W,V}^{-1}} f_V^{-1}(W),
\]
we obtain an isomorphism $\tau_W$, and we immediately see that, if $W'\subset W$ is an affine open, then $\tau_{W'}$ is the restriction of $\tau_W$ to $f_U^{-1}(W')$;
the $\tau_W$ are thus indeed the restrictions of a $Y$-isomorphism $\theta_{V,U}$.
Further, if $U$, $V$, and $W$ are affine open subsets of $Y$, and $\theta'_{U,V}$, $\theta'_{V,W}$, and $\theta'_{U,W}$ the restrictions of $\theta_{U,V}$, $\theta_{V,W}$, and $\theta_{U,W}$ (respectively) to the inverse images of $U\cap V\cap W$ in $X_V$, $X_W$, and $X_W$ (respectively), then it follows from the above definitions that we have $\theta'_{U,V}\circ\theta'_{V,W}=\theta'_{U,W}$.
The existence of some $X$ satisfying the properties in the statement thus follows from \sref[I]{I.2.3.1};
its uniqueness up to $Y$-isomorphism is trivial, taking \sref{I.3.1.2.1} into account.
\end{proof}

\begin{env}[3.1.3]
\label{II.3.1.3}
We say that the prescheme $X$ defined in \sref{II.3.1.2} is the \emph{homogeneous spectrum} of the quasi-coherent graded $\sh{O}_Y$-algebra $\sh{S}$, and we denote it by $\Proj(\sh{S})$.
It is immediate that $\Proj(\sh{S})$ is \emph{separated over $Y$} (\sref{II.2.4.2} and \sref[I]{I.5.5.5});
if $\sh{S}$ is an $\sh{O}_Y$-algebra \emph{of finite type} \sref[I]{I.9.6.2}, then $\Proj(\sh{S})$ is \emph{of finite type} over $Y$ (\sref{II.2.7.1}[(ii)] and \sref[I]{I.6.3.1}).

If $f$ is the structure morphism $X\to Y$, then it is immediate that, for every prescheme induced by $Y$ on an open subset $U$ of $Y$, $f^{-1}(U)$ can be identified with the homogeneous spectrum $\Proj(\sh{S}|U)$.
\end{env}

\begin{proposition}[3.1.4]
\label{II.3.1.4}
Let $f\in\Gamma(Y,\sh{S}_d)$ for $d>0$.
Then there exists an open subset $X_f$ of the underlying space of $X=\Proj(\sh{S})$ that satisfies the following property:
for every affine open $U$ of $Y$, we have $X_f\cap\vphi^{-1}(U)=D_+(f|U)$ in $\vphi^{-1}(U)$ identified with $X_U=\Proj(\Gamma(U,\sh{S}))$, where $\vphi$ denotes the structure morphism $X\to Y$.
\oldpage[II]{51}
Furthermore, the prescheme induced on $X_f$ is affine over $Y$, and is canonically isomorphic to $\Spec(\sh{S}^{(d)}/(f-1)\sh{S}^{[d]})$ \sref{II.1.3.1}.
\end{proposition}

\begin{proof}
We have $f|U\in\Gamma(U,\sh{S}_d)=(\Gamma(U,\sh{S}))_d$.
If $U$ and $U'$ are affine opens of $Y$ such that $U'\subset U$, then $f|U'$ is the image of $f|U$ under the restriction homomorphism
\[
  \Gamma(U,\sh{S}) \to \Gamma(U',\sh{S})
\]
and so $D_+(f|U')$ is equal (with the notation of \sref{II.3.1.1}) to the prescheme induced on the inverse image $\rho_{U',U}^{-1}(D_+(f|U))$ in $X_{U'}$ \sref{II.2.8.1};
whence the first claim.
Furthermore, the prescheme induced on $D_+(f|U)$ by $X_U$ is canonically identified with $\Spec((\Gamma(U,\sh{S}))_{f|U})$, with these identifications being compatible with the restriction homomorphisms \sref{II.2.8.1};
the second claim then follows from \sref{II.2.2.5} and from the commutativity of the diagram \sref{II.2.8.2.1}.
\end{proof}

We also say that $X_f$ (as an open subset of the underlying space $X$) is the set of $x\in X$ where $f$ \emph{does not vanish}.

\begin{corollary}[3.1.5]
\label{II.3.1.5}
If $f\in\Gamma(Y,\sh{S}_d)$ and $g\in\Gamma(Y,\sh{S}_e)$, then
\[
\label{II.3.1.5.1}
  X_{fg} = X_f\cap X_g.
\tag{3.1.5.1}
\]
\end{corollary}

\begin{proof}
It suffices to consider the intersection of the two sets with a set $\vphi^{-1}(U)$, where $U$ is an affine open in $Y$, and to then apply formula \sref{II.2.3.3.2}.
\end{proof}

\begin{corollary}[3.1.6]
\label{II.3.1.6}
Let $(f_\alpha)$ be a family of sections of $\sh{S}$ over $Y$ such that $f_\alpha\in\Gamma(Y,\sh{S}_{d_\alpha})$;
if the sheaf of ideals of $\sh{S}$ generated by this family \sref[0]{0.5.1.1} contains all the $\sh{S}_n$ starting from a certain rank, then the underlying space $X$ is the union of the $X_{f_\alpha}$.
\end{corollary}

\begin{proof}
For every affine open $U$ of $Y$, $\vphi^{-1}(U)$ is the union of the $X_{f_\alpha}\cap\vphi^{-1}(U)$ \sref{II.2.3.14}.
\end{proof}

\begin{corollary}[3.1.7]
\label{II.3.1.7}
Let $\sh{A}$ be a quasi-coherent $\sh{O}_Y$-algebra;
set
\[
  \sh{S} = \sh{A}[T] = \sh{A}\otimes_{\bb{Z}}\bb{Z}[T]
\]
where $T$ is an indeterminate (and $\bb{Z}$ and $\bb{Z}[T]$ are considered as simple sheaves over $Y$).
Then $X=\Proj(\sh{S})$ is canonically identified with $\Spec(\sh{A})$.
In particular, $\Proj(\sh{O}_Y[T])$ is identified with $Y$.
\end{corollary}

\begin{proof}
By applying \sref{II.3.1.6} to the unique section $f\in\Gamma(Y,\sh{S})$ that is equal to $T$ at each point of $Y$< we see that $X_f=X$.
Further, here we have $d=1$, and $\sh{S}^{(1)}/(f-1)\sh{S}^{(1)}=\sh{S}/(f-1)\sh{S}$ is canonically isomorphic to $\sh{A}$, whence the corollary \sref{II.1.2.2}.
\end{proof}

Let $g\in\Gamma(Y,\sh{O}_Y)$;
if we take $\sh{S}=\sh{O}_Y[T]$, then $g\in\Gamma(Y,\sh{S}_0)$;
let
\[
  h = gT\in\Gamma(Y,\sh{S}_1).
\]
If $X=\Proj(\sh{S})$, then the canonical identification defined in \sref{II.3.1.7} identifies $X_h$ with the open subset $Y_g$ of $Y$ (in the sense of \sref[0]{0.5.5.2}):
indeed, we can restrict to the case where $Y=\Spec(A)$ is affine, and everything then reduces (taking \sref{II.2.2.5} into account) to the fact that the ring of fractions $A_g$ is canonically identified with $A[T]/(gT-1)A[T]$ \sref[0]{0.1.2.3}.

\begin{proposition}[3.1.8]
\label{II.3.1.8}
Let $\sh{S}$ be a quasi-coherent positively-graded $\sh{O}_Y$-algebra.
Then
\begin{enumerate}
  \item[(i)] For all $d>0$, there exists a canonical $Y$-isomorphism from $\Proj(\sh{S})$ to $\Proj(\sh{S}^{(d)})$.
\oldpage[II]{52}
  \item[(ii)] Let $\sh{S}'$ be the graded $\sh{O}_Y$-algebra given by the direct sum of $\sh{O}_Y$ with the $\sh{S}_n$ (for $n\geq0$);
    then $\Proj(\sh{S}')$ and $\Proj(\sh{S})$ are canonically $Y$-isomorphic.
  \item[(iii)] Let $\sh{L}$ be an invertible $\sh{O}_Y$-module \sref[0]{0.5.4.1}, and let $\sh{S}_{(\sh{L})}$ be the graded $\sh{O}_Y$-algebra given by the direct sum of the $\sh{S}_d\otimes\sh{L}^{\otimes d}$ (for $d\geq0$);
    then $\Proj(\sh{S})$ and $\Proj(\sh{S}_{(\sh{L})})$ are canonically $Y$-isomorphic.
\end{enumerate}
\end{proposition}

\begin{proof}
In each of the three cases, it suffices to define the isomorphism locally on $Y$, since the verification of compatibility with the restriction operations from one open subset to a smaller one is trivial.
We can thus suppose that $Y$ is affine, and then (i) follows from \sref{II.2.4.7}[(i)], and (ii) follows from \sref{II.2.4.8}.
As for (iii), if we further suppose that $\sh{L}$ is isomorphic to $\sh{O}_Y$ (which we are allowed to do, since the question is local on $Y$), then the isomorphism between $\Proj(\sh{S})$ and $\Proj(\sh{S}_{(\sh{L})})$ is evident;
to define a \emph{canonical} isomorphism, let $Y=\Spec(A)$ and $\sh{S}=\widetilde{S}$, where $S$ is a graded $A$-algebra, and let $c$ be a generator of the free $A$-module $L$ such that $\sh{L}=\widetilde{L}$;
then, for all $n>0$, $x_n\mapsto x_n\otimes c^{\otimes n}$ is an $A$-isomorphism from $S_n$ to $S_n\otimes L^{\otimes n}$, and these $A$-isomorphisms define an $A$-isomorphism of graded algebras $\vphi_c:S\to S_{(L)}=\bigoplus_{n\geq0}S_n\otimes L^{\otimes n}$.
So let $f\in S_+$ be homogeneous of degree~$d$;
for all $x\in S_{nd}$, we have that $(x\otimes c^{nd})/(f\otimes c^d)^n=(x\otimes(\varepsilon c)^{nd})/(f\otimes(\varepsilon c)^d)^n$ for every invertible element $\varepsilon\in A$, which shows that the isomorphism $S_{(f)}\to(S_{(L)})_{(f\otimes c^d)}$ induced from $\vphi_c$ is \emph{independent} of the generator $c$ of $L$ considered, and thus finishes the proof.
\end{proof}

\begin{env}[3.1.9]
\label{II.3.1.9}
Recall (\sref[0]{0.4.1.3} and \sref[I]{I.1.3.14}) that, for the quasi-coherent graded $\sh{O}_Y$-algebra $\sh{S}$ to be \emph{generated by the $\sh{O}_Y$-module $\sh{S}_1$}, it is necessary and sufficient for there to exist an affine open cover $(U_\alpha)$ of $Y$ such that the graded algebra $\Gamma(U_\alpha,\sh{S})$ over $\Gamma(U_\alpha,\sh{S}_0)$ is generated by the set $\Gamma(U_\alpha,\sh{S}_1)$ of its homogeneous elements of degree~$1$.
For every open $V$ of $Y$, $\sh{S}|V$ is then generated by the $(\sh{O}_Y|V)$-module $\sh{S}_1|V$.
\end{env}

\begin{proposition}[3.1.10]
\label{II.3.1.10}
Suppose that there exists a finite affine open cover $(U_i)$ of $Y$ such that each graded algebra $\Gamma(U_i,\sh{S})$ is of finite type over $\Gamma(U_i\sh{O}_Y)$.
Then there exists $d>0$ such that $\sh{S}^{(d)}$ is generated by $\sh{S}_d$, with $\sh{S}_d$ an $\sh{O}_Y$-module of finite type.
\end{proposition}

\begin{proof}
Indeed, it follows from \sref{II.2.1.6}[(v)] that, for each $i$, there exists an integer $m_i$ such that $\Gamma(U_i,\sh{S}_{nm_i})=(\Gamma(U_i,\sh{S}_{m_i}))^n$ for all $n>0$;
it suffices to take $d$ to be a common multiple of all the $m_i$, taking \sref{II.2.1.6}[(i)] into account.
\end{proof}

\begin{corollary}[3.1.11]
\label{II.3.1.11}
Under the hypotheses of \sref{II.3.1.10}, $\Proj(\sh{S})$ is $Y$-isomorphic to a homogeneous spectrum $\Proj(\sh{S}')$, where $\sh{S}'$ is a graded $\sh{O}_Y$-algebra generated by $\sh{S}'_1$, with $\sh{S}'_1$ an $\sh{O}_Y$-module of finite type.
\end{corollary}

\begin{proof}
It suffices to take $\sh{S}'=\sh{S}^{(d)}$, where $d$ satisfies the property of \sref{II.3.1.10}, and to then apply \sref{II.3.1.8}[(i)]
\end{proof}

\begin{env}[3.1.12]
\label{II.3.12}
If $\sh{S}$ is a quasi-coherent positively-graded $\sh{O}_Y$-algebra, we know \sref[I]{I.5.1.1} that its \emph{nilradical} $\sh{N}$ is a quasi-coherent $\sh{O}_Y$-module;
we say that $\sh{N}_+=\sh{N}\cap\sh{S}_+$ is the \emph{nilradical of $\sh{S}_+$};
this is a quasi-coherent graded $\sh{S}_0$-module, since we can immediately reduce to the case where $Y$ is affine, and the proposition then follows from \sref{II.2.1.10}.
For all $y\in Y$, $(\sh{N}_+)_y$ is then the nilradical of $(\sh{S}_+)_y=(\sh{S}_y)_+$ \sref[I]{I.5.1.1}.
We say that the graded $\sh{O}_Y$-algebra $\sh{S}$ is \emph{essentially reduced} if $\sh{N}_+=0$, which is equivalent
\oldpage[II]{53}
to saying that $\sh{S}_y$ is an essentially reduced graded $\sh{O}_y$-algebra for all $y\in Y$.
For every graded $\sh{O}_Y$-algebra $\sh{S}$, $\sh{S}/\sh{N}_+$ is essentially reduced.

We say that $\sh{S}$ is \emph{integral} if, for all $y\in Y$, $\sh{S}_y$ is an integral ring and, furthermore, $(\sh{S}_y)_+=(\sh{S}_+)_y\neq0$.
\end{env}

\begin{proposition}[3.1.13]
\label{II.3.1.13}
Let $\sh{S}$ be a positively-graded $\sh{O}_Y$-algebra.
If $X=\Proj(\sh{S})$, then the $Y$-scheme $X_\red$ is canonically isomorphic to $\Proj(\sh{S}/\sh{N}_+)$;
in particular, if $\sh{S}$ is essentially reduced, then $X$ is reduced.
\end{proposition}

\begin{proof}
The fact that $X'=\Proj(\sh{S}/\sh{N}_+)$ is reduced follows immediately from \sref{II.2.4.4}[(i)], since the property is local;
further, for every affine open $U\subset Y$, ${\vphi'}^{-1}(U)$ is equal to $(\vphi^{-1}(U))_\red$ (where we denote by $\vphi$ and $\vphi'$ the structure morphisms $X\to Y$ and $X'\to Y$, respectively);
we immediately see that the canonical $U$-morphisms ${\vphi'}^{-1}(U)\to\vphi^{-1}(U)$ are compatible with the restriction operations, and thus define a closed immersion $X'\to X$ that is a homeomorphism of the underlying spaces;
whence the conclusion \sref[I]{I.5.1.2}.
\end{proof}

\begin{proposition}[3.1.14]
\label{II.3.1.14}
Let $Y$ be an integral prescheme, and $\sh{S}$ a quasi-coherent graded $\sh{O}_Y$-algebra such that $\sh{S}_0=\sh{O}_Y$.
\begin{enumerate}
  \item[(i)] If $\sh{S}$ is integral \sref{II.3.1.12}, then $X=\Proj(\sh{S})$ is integral, and the structure morphism $\vphi: X\to Y$ is dominant.
  \item[(ii)] Suppose furthermore that $\sh{S}$ is essentially reduced.
    Then, conversely, if $X$ is integral and $\vphi$ is dominant, then $\sh{S}$ is integral.
\end{enumerate}
\end{proposition}

\begin{proof}
\medskip\noindent
\begin{enumerate}
  \item[(i)] If $(U_\alpha)$ is a base of $Y$ consisting of non-empty affine opens, then it suffices to prove the proposition in the case where $Y$ is replaced by one of the $U_\alpha$, and $\sh{S}$ by $\sh{S}|U_\alpha$:
    indeed, one one hand it will follow that the underlying space $\vphi^{-1}(U_\alpha)$ are irreducible opens (and thus non-empty) of $X$ such that $\vphi^{-1}(U_\alpha)\cap\vphi^{-1}(U_\beta)\neq\emp$ for any pair of indices (since $U_\alpha\cap U_\beta$ contains some $U_\gamma$), and so $X$ is irreducible \sref[0]{0.2.1.4};
    on the other hand, $X$ will be reduced, since this is a local property, and so $X$ will indeed be integral, with $\vphi(X)$ dense in $Y$.

    So suppose that $Y=\Spec(A)$, where $A$ is integral \sref[I]{I.5.1.4}, and that $\sh{S}=\widetilde{S}$, where $S$ is a graded $A$-algebra;
    the hypothesis is that, for every $y\in Y$, $\widetilde{S}_y=S_y$ is an integral graded ring such that $(S_y)_+\neq0$.
    It suffices to prove that $S$ is an \emph{integral} ring, since then we will have that $S_+\neq0$, and we can then apply \sref{II.2.4.4}[(ii)].
    But let $f,g\neq0$ be elements of $S$, and suppose that $fg=0$;
    for all $y\in Y$ we then have that $(f/1)(g/1)=0$ in $S_y$, and so $f/1=0$ or $g/1=0$ by hypothesis.
    Suppose, for example, that $f/1=0$ in $S_y$;
    this implies that there exists $a\in A$ such that $a\not\in\mathfrak{j}_y$ and $af=0$;
    we then have, for \emph{all} $z\in Y$, that $(a/1)(f/1)=0$ in the \emph{integral} ring $S_z$, and since $a/1\neq0$ (since $A$ is integral), $f/1=0$, which implies that $f=0$.
  \item[(ii)] Since the question is local on $Y$, we can again suppose that $Y=\Spec(A)$, with $A$ integral, and that $\sh{S}=\widetilde{S}$.
    By hypothesis, for all $y\in Y$, $(S_y)_+$ does not contain any non-zero nilpotent elements, and the same is true of $(S_0)_y=A_y$ by hypothesis;
    so $S_y$ is a reduced ring for all $y\in Y$, and we thus conclude first of all that $S$ itself is reduced \sref[I]{I.5.1.1}.
    The hypothesis that $X$ is integral implies that $S$ is essentially integral \sref{II.2.4.4}[(ii)], and everything then reduces to showing that the annihilator $\mathfrak{J}$ of $S_+$ in $A=S_0$ is just $0$ \sref{II.2.1.11}.
    If this were not the case, we would have that $(S_h)_+=0$ for some $h\neq0$ in $\mathfrak{J}$, and thus \sref{II.3.1.1} that $\vphi^{-1}(D(h))=\emp$, and $\vphi(X)$ would not be dense in $Y$, contradicting the hypothesis (since $D(h)\neq\emp$, since $h$ is not nilpotent).
\oldpage[II]{54}
\end{enumerate}
\end{proof}


\subsection{Sheaf on $\operatorname{Proj}(\mathcal{S})$ associated to a graded $\mathcal{S}$-module}
\label{subsection:II.3.2}

\begin{env}[3.2.1]
\label{II.3.2.1}
Let $Y$ be a prescheme, $\sh{S}$ a quasi-coherent positively-graded $\sh{O}_Y$-algebra, and $\sh{M}$ a quasi-coherent graded $\sh{S}$-module (on $(Y,\sh{O}_Y)$, or, equivalently \sref[I]{I.9.6.1}, on the ringed space $(Y,\sh{S})$).
With the notation of \sref{II.3.1.1}, we denote by $\widetilde{\sh{M}}_U$ the quasi-coherent $\sh{O}_{X_U}$-module $(\Gamma(U,\sh{M}))\supertilde$;
for $U'\subset U$, $\Gamma(U',\sh{M})$ is canonically identified with $\Gamma(U,\sh{M})\otimes_A A'$ \sref[I]{I.1.6.4};
thus we have $\widetilde{\sh{M}}_{U'}=\rho_{U',U}^*(\widetilde{\sh{M}}_U)$ \sref{II.2.8.11}.
\end{env}

\begin{proposition}[3.2.2]
\label{II.3.2.2}
There exists on $\Proj(\sh{S})=X$ exactly one quasi-coherent $\sh{O}_X$-module $\sh{M}$ such that, for every affine open $U$ of $Y$, we have $\eta_U^*((\Gamma(U,\sh{M}))\supertilde)=\widetilde{M}|f^{-1}(U)$ (denoting by $\eta_U$ the isomorphism $f^{-1}(U)\simto\Proj(\Gamma(U,\sh{S}))$), where $f$ is the structure morphism $X\to Y$.
\end{proposition}

\begin{proof}
Since $\rho_{U',U}$ is identified with the injection morphism $f^{-1}(U')\to f^{-1}(U)$ \sref{II.3.1.2.1}, the proposition follows immediately from the relation $\widetilde{\sh{M}}_{U'}=\rho_{U',U}^*(\widetilde{\sh{M}}_U)$ and from the gluing principle for sheaves \sref[0]{0.3.3.1}.
\end{proof}

We say that $\widetilde{\sh{M}}$ is the $\sh{O}_X$-module \emph{associated to} the quasi-coherent graded $\sh{S}$-module $\sh{M}$.

\begin{proposition}[3.2.3]
\label{II.3.2.3}
Let $\sh{M}$ be a quasi-coherent graded $\sh{S}$-module, and let $f\in\Gamma(Y,\sh{S}_d)$ (for $d>0$).
If $\xi_f$ is the canonical isomorphism from $X_f$ to the $Y$-prescheme $Z_f=\Spec(\sh{S}^{(d)}/(f-1)\sh{S}^{(d)})$ \sref{II.3.1.4}, then $(\xi_f)_*(\widetilde{\sh{M}}|X_f)$ is the $\sh{O}_{Z_f}$-module $(\sh{M}^{(l)}/(f-1)\sh{M}^{(d)})$ \sref{II.1.4.3}.
\end{proposition}

\begin{proof}
Since the question is local on $Y$, we can immediately reduce to \sref{II.2.2.5}, taking into account the commutativity of the diagram in \sref{II.2.8.12.1}.
\end{proof}

\begin{proposition}[3.2.4]
\label{II.3.2.4}
The $\sh{O}_X$-module $\widetilde{\sh{M}}$ is an exact additive covariant functor in $\sh{M}$, from the category of quasi-coherent graded $\sh{S}$-modules to the category of quasi-coherent $\sh{O}_X$-modules, that commutes with inductive limits and direct sums.
\end{proposition}

\begin{proof}
Since the question is local on $Y$, we can reduce to \sref[I]{I.1.3.11}, \sref[I]{I.1.3.9}, and \sref{II.2.5.4}.
\end{proof}

In particular, if $\sh{N}$ is a quasi-coherent graded sub-$\sh{S}$-module of $\sh{M}$, then $\widetilde{\sh{N}}$ is canonically identified with with a quasi-coherent sub-$\sh{O}_X$-module of $\widetilde{\sh{M}}$;
in particular, for every quasi-coherent graded sheaf $\sh{J}$ of ideals of $\sh{S}$, $\widetilde{\sh{J}}$ is a quasi-coherent sheaf of ideals of $\sh{O}_X$.

If $\sh{M}$ is a quasi-coherent graded $\sh{S}$-module, and $\sh{I}$ a quasi-coherent sheaf of ideals of $\sh{O}_Y$, then $\sh{I}\sh{M}$ is a quasi-coherent graded sub-$\sh{S}$-module of $\sh{M}$, and we have
\[
\label{II.3.2.4.1}
  (\sh{I}\sh{M})\supertilde = \sh{I}\cdot\widetilde{\sh{M}}
\tag{3.2.4.1}
\]
(where the right-hand side is defined as in \sref[0]{0.4.3.5}).
It suffices to verify this formula in the case where $Y=\Spec(A)$ is affine, $\sh{S}=\widetilde{S}$, with $S$ a graded $A$-algebra, $\sh{M}=\widetilde{M}$, with $M$ a graded $S$-module, and $\sh{I}=\mathfrak{I}$, with $\mathfrak{I}$ an ideal of $A$.
\oldpage[II]{55}
For every homogeneous element $f$ of $S_+$, the restriction to $D_+(f)=\Spec(S_{(f)})$ of the left-hand side of \sref{II.3.2.4.1} can be associated with $(\mathfrak{I}M)_{(f)}=\mathfrak{I}\cdot M_{(f)}$, and the same is true of the restriction of the right-hand side, given \sref[I]{I.1.3.13} and \sref[i]{I.1.6.9}.

\begin{proposition}[3.2.5]
\label{II.3.2.5}
Let $f\in\Gamma(Y,\sh{S}_d)$ (for $d>0$).
On the open subset $X_f$, the $(\sh{O}_X|X_f)$-module $(\sh{S}(nd))\supertilde X_f$ is canonically isomorphic to $\sh{O}_X|X_f$ for all $n\in\bb{Z}$.
In particular, if the $\sh{O}_Y$-algebra $\sh{S}$ is generated by $\sh{S}_1$ \sref{II.3.1.9}, then the $\sh{O}_X$-modules $(\sh{S}(n))\supertilde$ are invertible for all $n\in\bb{Z}$.
\end{proposition}

\begin{proof}
Indeed, for every affine open $U$ of $Y$, we defined in \sref{II.2.5.7} a canonical isomorphism from $(\sh{S}(nd))\supertilde|(X_f\cap\vphi^{-1}(U))$ to $\sh{O}_X|(X_f\cap\vphi^{-1}(U))$, taking \sref{II.3.1.4} into account (where $\vphi$ is the structure morphism $X\to Y$);
it is immediate that these isomorphisms are compatible with the restriction from $U$ to an affine open $U'\subset U$, whence the first claim.
To prove the second, it suffices to note that, if $\sh{S}$ is generated by $\sh{S}_1$, then there is a cover $(U_\alpha)$ of $Y$ by affine opens such that $\Gamma(U_\alpha,\sh{S})$ is generated by $\Gamma(U_\alpha,\sh{S})_1=\Gamma(U_\alpha,\sh{S}_1)$;
we can then apply the result of \sref{II.2.5.9}, since the property of being invertible is local.
\end{proof}

We again set, for all $n\in\bb{Z}$,
\[
\label{II.3.2.5.1}
  \sh{O}_X(n) = (\sh{S}(n))\supertilde
\tag{3.2.5.1}
\]
and, for all $\sh{O}_X$-modules $\sh{F}$,
\[
\label{II.3.2.5.2}
  \sh{F}(n) = \sh{F}\otimes_{\sh{O}_X}\sh{O}_X(n).
\tag{3.2.5.2}
\]

It follows immediately from these definitions that, for every open subset $U$ of $Y$, we have
\[
  ((\sh{S}|U)(n))\supertilde = \sh{O}_X(n)|f^{-1}(U)
\]
where $f$ is the structure morphism $X\to Y$.

\begin{proposition}[3.2.6]
\label{II.3.2.6}
Let $\sh{M}$ and $\sh{N}$ be quasi-coherent graded $\sh{S}$-modules.
Then there exists a canonical functorial (in $\sh{M}$ and $\sh{N}$) homomorphism
\[
\label{II.3.2.6.1}
  \lambda : \widetilde{\sh{M}}\otimes_{\sh{O}_X}\widetilde{\sh{N}} \to (\sh{M}\otimes_{\sh{S}}\sh{N})\supertilde
\tag{3.2.6.1}
\]
and a canonical functorial (in $\sh{M}$ and $\sh{N}$) homomorphism
\[
\label{II.3.2.6.2}
  \mu : (\shHom_{\sh{S}}(\sh{M},\sh{N}))\supertilde \to \shHom_{\sh{O}_X}(\widetilde{\sh{M}},\widetilde{\sh{N}}).
\tag{3.2.6.2}
\]

Furthermore, if $\sh{S}$ is generated by $\sh{S}_1$ \sref{II.3.1.9}, then $\lambda$ is an isomorphism;
if, further, $\sh{M}$ admits a finite presentation \sref{II.3.1.1}, then $\mu$ is an isomorphism.
\end{proposition}

\begin{proof}
The isomorphisms $\lambda$ and $\mu$ were defined in \sref{II.2.5.11.2} and \sref{II.2.5.12.2} in the case where $Y$ is affine;
since these definitions are local, they transfer immediately to the general case considered here, taking \sref{II.2.8.14} into account.
\end{proof}

\begin{corollary}[3.2.7]
\label{II.3.2.7}
If $\sh{S}$ is generated by $\sh{S}_1$, then, for any $m,n\in\bb{Z}$,
\[
\label{II.3.2.7.1}
  \sh{O}_X(m)\otimes_{\sh{O}_X}\sh{O}_X(n) = \sh{O}_X(m+n)
\tag{3.2.7.1}
\]
\[
\label{II.3.2.7.2}
  \sh{O}_X(n) = (\sh{O}_X(1))^{\otimes n}
\tag{3.2.7.2}
\]
up to canonical isomorphism.
\end{corollary}

\oldpage[II]{56}

\begin{corollary}[3.2.8]
\label{II.3.2.8}
If $\sh{S}$ is generated by $\sh{S}_1$, then, for any graded $\sh{S}$-module $\sh{M}$ and any $n\in\bb{Z}$,
\[
\label{II.3.2.8.1}
  (\sh{M}(n))\supertilde = \widetilde{\sh{M}}(n)
\tag{3.2.8.1}
\]
up to canonical isomorphism.
\end{corollary}

\begin{proof}
This follows from the corresponding properties in the case where $Y$ is affine (\sref{II.2.5.14} and \sref{II.2.5.15}), along with \sref{II.2.8.11}.
\end{proof}

\begin{remarks}[3.2.9]
\label{II.3.2.9}
\medskip\noindent
  \begin{enumerate}
    \item If $\sh{S}=\sh{A}[T]$, with $\sh{A}$ a quasi-coherent $\sh{O}_Y$-algebra \sref{II.3.1.7}, then we immediately see that all the invertible $\sh{O}$-modules $\sh{O}(n)$ are canonically isomorphic to $\sh{O}_X$.

      Furthermore, let $\sh{N}$ be a quasi-coherent $\sh{A}$-module, and set $\sh{M}=\sh{N}\otimes_{\sh{A}}\sh{A}[T]$.
      It then follows from \sref{II.3.2.3} and \sref{II.3.1.7} that, under the canonical identification of $X=\Proj(\sh{A}[T])$ with $X'=\Spec(\sh{A})$, the $\sh{O}_X$-module $\widetilde{\sh{M}}$ is identified with the $\sh{O}_{X'}$-module $\widetilde{\sh{N}}$ associated to $\sh{N}$ (in the sense of \sref{II.1.4.3}).
    \item Let $\sh{S}$ be an arbitrary graded $\sh{O}_Y$-algebra, and $\sh{S}'$ the graded $\sh{O}_Y$-algebra such that $\sh{S}'_0=\sh{O}_Y$ and $\sh{S}'_n=\sh{S}_n$ for all $n>0$;
      then the canonical isomorphism from $X=\Proj(\sh{S})$ to $X'=\Proj(\sh{S}')$ \sref{II.3.1.8}[(ii)] identifies $\sh{O}_X(n)$ with $\sh{O}_{X'}(n)$ for all $n\in\bb{Z}$.
      This follows from the same proposition for the affine case \sref{II.2.5.16} and from the fact that the identifications, for the affine opens of $Y$, commute with the restriction operations.
      Similarly, let $X^{(d)}=\Proj(\sh{S}^{(d)})$;
      then the canonical isomorphism from $X$ to $X^{(d)}$ \sref{II.3.1.8}[(i)] identifies $\sh{O}_X(nd)$ with $\sh{O}_{X^{(d)}}(n)$ for all $n\in\bb{Z}$.
  \end{enumerate}
\end{remarks}

\begin{proposition}[3.2.10]
\label{II.3.2.10}
Let $\sh{L}$ be an invertible $\sh{O}_Y$-module, and $g$ the canonical isomorphism from $X_{(\sh{L})}=\Proj(\sh{S}_{(\sh{L})})$ to $X=\Proj(\sh{S})$ \sref{II.3.1.8}[(iii)].
Then, for any $n\in\bb{Z}$, $g_*(\sh{O}_{X_{(\sh{L})}}(n))$ is canonically isomorphic to $\sh{O}_X(n)\otimes_Y\sh{L}^{\otimes n}$.
\end{proposition}

\begin{proof}
Suppose first of all that $Y$ is affine, of ring $A$, and that $\sh{L}=\widetilde{L}$, where $L$ is a free monogenous $A$-module.
With the notation from the proof of \sref{II.3.1.8}[(iii)], we define, for $f\in S_d$, an isomorphism from $(S(n))_{(f)}\otimes_A L^{\otimes n}$ to $(S_{(L)}(n))_{(f\otimes c^d)}$ by sending $(x/f^k)\otimes c^n$, where $x\in S_{kd+n}$, to the element $(x\otimes c^{n+kd})/(f\otimes c^d)^k$;
it is immediate that this isomorphism is independent of the chosen generator $c$ of $L$;
further, the isomorphisms thus defined for each $f\in S_+$ are compatible with the restriction operators $D_+(f)\to D_+(fg)$.
Finally, in the general case, we easily see, from the definitions \sref{II.3.1.1}, that the isomorphisms thus defined for each affine open $U$ of $Y$ are compatible with passing from $U$ to an affine open $U'\subset U$.
\end{proof}


\subsection{Graded $\mathcal{S}$-module associated to a sheaf on $\operatorname{Proj}(\mathcal{S})$}
\label{subsection:II.3.3}

\emph{Throughout this entire section we suppose that the graded $\sh{O}_Y$-algebra $\sh{S}$ is generated by $\sh{S}_1$ \sref{II.3.1.9}.}
Recall that, by \sref{II.3.1.8}[(i)], this restrictive assumption is not essential, thanks to finiteness conditions \sref{II.3.1.10}.

\begin{env}[3.3.1]
\label{II.3.3.1}
Let $p$ be the structure morphism $X=\Proj(\sh{S})\to Y$.
For every $\sh{O}_X$-module $\sh{F}$, set
\[
\label{II.3.3.1.1}
  \bbGamma_*(\sh{F}) = \bigoplus_{n\in\bb{Z}}p_*(\sh{F}(n))
\tag{3.3.1}
\]
\oldpage[II]{57}
and, in particular,
\[
\label{II.3.3.1.2}
  \bbGamma_*(\sh{O}_X) = \bigoplus_{n\in\bb{Z}}p_*(\sh{O}_X(n)).
\tag{3.3.1.2}
\]

We know \sref[0]{0.4.2.2} that there exists a canonical homomorphism
\[
  p_*(\sh{F})\otimes_{\sh{O}_Y}p_*(\sh{G}) \to p_*(\sh{F}\otimes_{\sh{O}_X}\sh{G})
\]
for $\sh{O}_X$-modules $\sh{F}$ and $\sh{G}$;
we thus deduce from \sref{II.3.2.7.1} that $\bbGamma_*(\sh{O}_X)$ is endowed with the structure of a \emph{graded $\sh{O}_Y$-algebra}, and \sref{II.3.2.5.2} defines the structure of a \emph{graded $\bbGamma_*(\sh{O}_X)$-module} on $\bbGamma_*(\sh{F})$.

By \sref{II.3.2.5}, and by left exactness of the functor $p_*$ \sref[0]{0.4.2.1}, $\bbGamma_*(\sh{F})$ is an additive and \emph{left exact} covariant functor in $\sh{F}$ from the category of $\sh{O}_X$-modules to the category of graded $\sh{O}_Y$-modules (where the morphisms are the homomorphisms of degree~$0$).
In particular, if $\sh{J}$ is a sheaf of ideals of $\sh{O}_X$, then $\bbGamma_*(\sh{J})$ can be identified with a \emph{graded sheaf of ideals} in $\bbGamma_*(\sh{O}_X)$.
\end{env}

\begin{env}[3.3.2]
\label{II.3.3.2}
Let $\sh{M}$ be a quasi-coherent graded $\sh{S}$-module.
For every affine open $U$ of $Y$, we defined in \sref{II.2.6.2} a homomorphism of abelian groups
\[
  \alpha_{0,U} : \Gamma(U,\sh{M}_0) \to \Gamma(p^{-1}(U),\widetilde{\sh{M}}).
\]
It is immediate that these homomorphisms commute with the restriction operations \sref{II.2.8.13.1} and thus define (without using the hypothesis that $\sh{S}$ is generated by $\sh{S}_1$) a homomorphism of sheaves of abelian groups
\[
\label{II.3.3.2.1}
  \alpha_0 : \sh{M}_0 \to p_*(\widetilde{\sh{M}}).
\tag{3.3.2.1}
\]

Applying this result to each of the $\sh{M}_n=(\sh{M}(n))_0$, and taking \sref{II.3.2.8.1} into account, we can define a homomorphism of sheaves of abelian groups
\[
\label{II.3.3.2.2}
  \alpha_n : \sh{M}_n \to p_*(\widetilde{\sh{M}}(n))
\tag{3.3.2.2}
\]
for all $n\in\bb{Z}$, whence a functorial homomorphism (of degree~$0$) of graded sheaves of abelian groups
\[
\label{II.3.3.2.3}
  \alpha : \sh{M} \to \bbGamma_*(\widetilde{\sh{M}})
\tag{3.3.2.3}
\]
(also denoted $\alpha_{\sh{M}}$).

By taking $\sh{M}=\sh{S}$ in particular, we see that $\alpha:\sh{S}\to\bbGamma_*(\sh{O}_X)$ is a homomorphism of graded $\sh{O}_Y$-algebra, and that \sref{II.3.3.2.3} is a di-homomorphism of graded modules, with respect to this homomorphism of graded algebras.

We again note that to each of the $\alpha_n$ there corresponds \sref[0]{0.4.4.3} a canonical homomorphism of $\sh{O}_X$-modules
\[
\label{II.3.3.2.4}
  \alpha_n^\sharp : p^*(\sh{M}_n) \to \widetilde{\sh{M}}(n).
\tag{3.3.2.4}
\]

We can easily verify that this homomorphism is exactly the one which corresponds functorially \sref{II.3.2.4} to the canonical homomorphism (of degree~$0$) of \emph{graded} $\sh{O}_Y$-modules
\[
\label{II.3.3.2.5}
  \sh{M}_n\otimes_{\sh{O}_Y}\sh{S} \to \sh{M}(n)
\tag{3.3.2.5}
\]
\oldpage[II]{57}
where the grading of the right-hand side comes naturally from that of $\sh{S}$.
We can restrict to the case where $Y=\Spec(A)$ is affine, $\sh{M}=\widetilde{M}$, and $\sh{S}=\widetilde{S}$, with the graded $A$-algebra $S$ being generated by $S_1$, so that, as $f$ runs over $S_1$, the $D_+(f)$ form a cover of $X$.
By the definitions \sref{II.2.6.2}, we see then see, taking \sref[I]{I.1.6.7} into account, that the restriction to $D_+(f)$ of the homomorphism \sref{II.3.3.2.4} corresponds \sref[I]{I.1.3.8} to the homomorphism of $S_{(f)}$-modules $M_n\otimes_A S_{(f)}\to(S(n))_{(f)}$ that sends $x\otimes1$ (where $x\in M_n$) to $x/1$;
this proves the claim.
\end{env}

\begin{proposition}[3.3.3]
\label{II.3.3.3}
For every section $f\in\Gamma(Y,\sh{S}_d)$ (where $d>0$), $X_f$ is identical to the set of points of $X$ where $\alpha_d(f)$ (thought of as a section of $\sh{O}_X(d)$) does not vanish \sref[0]{0.5.5.2}.
\end{proposition}

\begin{proof}
(Note that $\alpha_d(f)$ is a section of $p_*(\sh{O}_X(d))$ over $Y$, but by definition such a section is also a section of $\sh{O}(d)$ over $X$ \sref[0]{0.4.2.1}).
The definition of $X_f$ \sref{II.3.1.4} lets us reduce to the case where $Y$ is affine, which has already been dealt with in \sref{II.2.6.3}.
\end{proof}

\begin{env}[3.3.4]
\label{II.3.4.4}
From now on, we suppose, in addition to the hypothesis at the start of this section, that, for every quasi-coherent $\sh{O}_X$-module $\sh{F}$, the $p_*(\sh{F}(n))$ are \emph{quasi-coherent} on $Y$, so that $\bbGamma_*(\sh{F})=\bigoplus_{n\in\bb{Z}}p_*(\sh{F}(n))$ is also a quasi-coherent $\sh{O}_Y$-module (\sref[I]{I.1.4.1} and \sref[I]{I.1.3.9});
this will always be the case if $X$ is \emph{of finite type} over $Y$ \sref[I]{I.9.2.2}.
We thus conclude that $(\bbGamma_*(\sh{F}))\supertilde$ is defined, and is a quasi-coherent $\sh{O}_X$-module.
For every affine open $U$ of $Y$< we have (\sref[I]{I.1.3.9} and \sref[I]{I.2.5.4})
\begin{align*}
  \Big( \Gamma(U,\bigoplus_{n\in\bb{Z}}p_*(\sh{F}(n))) \Big)\supertilde
  &= \bigoplus_{n\in\bb{Z}}\Big( \Gamma(U,p_*(\sh{F}(n))) \Big)\supertilde
\\&= \bigoplus_{n\in\bb{Z}}\Big( \Gamma(p^{-1}(U),\sh{F}(n)) \Big)\supertilde
\\&= \Big( \bigoplus_{n\in\bb{Z}}\Gamma(p^{-1}(U),\sh{F}(n)) \Big)\supertilde
\\&= (\bbGamma_*(\sh{F}|p^{-1}(U)))\supertilde
\end{align*}
and so \sref{II.2.6.4} we have a canonical homomorphism
\[
  \beta_U : \Big( \Gamma(U,\bigoplus_{n\in\bb{Z}})p_*(\sh{F}(n)) \Big)\supertilde \to \sh{F}|p^{-1}(U).
\]

Furthermore, the commutativity of \sref{II.2.8.13.2} shows that these homomorphism commute with the restriction operations on $Y$;
we thus obtain a canonical functorial homomorphism
\[
\label{II.3.3.4.1}
  \beta : (\bbGamma_*(\sh{F}))\supertilde \to \sh{F}
\tag{3.3.4.1}
\]
(also denoted $\beta_{\sh{F}}$) for quasi-coherent $\sh{O}_X$-modules.
\end{env}

\begin{proposition}[3.3.5]
\label{II.3.3.5}
Let $\sh{M}$ be a quasi-coherent graded $\sh{S}$-module, and $\sh{F}$ a quasi-coherent $\sh{O}_X$-module;
then the composite homomorphisms
\[
\label{II.3.3.5.1}
  \widetilde{\sh{M}} \xrightarrow{\widetilde{\alpha}} (\bbGamma_*(\widetilde{\sh{M}}))\supertilde \xrightarrow{\beta} \widetilde{\sh{M}}
\tag{3.3.5.1}
\]
\[
\label{II.3.3.5.2}
  \bbGamma_*(\sh{F}) \xrightarrow{\alpha} \bbGamma_*((\bbGamma_*(\sh{F}))\supertilde) \xrightarrow{\bbGamma_*(\beta)} \bbGamma_*(\sh{F})
\tag{3.3.5.2}
\]
are the identity isomorphisms.
\end{proposition}

\begin{proof}
The question is local on $Y$, so we can reduce to \sref{II.2.6.5}.
\end{proof}


\subsection{Finiteness conditions}
\label{subsection:II.3.4}

\oldpage[II]{59}
\begin{proposition}[3.4.1]
\label{II.3.4.1}
Let $Y$ be a prescheme, and $\sh{S}$ a quasi-coherent $\sh{O}_Y$-algebra generated by $\sh{S}_1$ \sref{II.3.1.9};
suppose further that $\sh{S}_1$ is of finite type.
Then $X=\Proj(\sh{S})$ is of finite type over $Y$.
\end{proposition}

\begin{proof}
Since the question is local on $Y$, we can suppose that $Y$ is affine of ring $A$;
then $\sh{S}=\widetilde{S}$, where $S=\Gamma(Y,\sh{S})$, and by hypothesis $S$ is an $A$-algebra generated by $S_1=\Gamma(Y,\sh{S}_1)$, where we can further suppose that $S_1$ is an $A$-module of finite type (\sref[I]{I.1.3.9} and \sref[I]{I.1.3.12}).
Then $S$ is a graded $A$-algebra of finite type, and we can reduce to \sref{II.2.7.1}[(ii)].
\end{proof}

\begin{env}[3.4.2]
\label{II.3.4.2}
Let $\sh{S}$ be a quasi-coherent graded $\sh{O}_Y$-algebra;
for a quasi-coherent graded $\sh{S}$-module $\sh{M}$, consider the following finiteness conditions:
\begin{enumerate}
  \item[(\textbf{TF})] There exists an integer $n$ such that the $\sh{S}$-module $\bigoplus_{k\geq n}\sh{M}_k$ is of finite type.
  \item[(\textbf{TN})] There exists an integer $n$ such that $\sh{M}_k=0$ for $k\geq n$.
\end{enumerate}

If $\sh{M}$ satisfies (\textbf{TN}), then $\widetilde{\sh{M}}=0$, since this is a local property on $Y$ \sref{II.2.7.2}.

Let $\sh{M}$ and $\sh{N}$ be quasi-coherent graded $\sh{S}$-modules;
we say that a homomorphism $u:\sh{M}\to\sh{N}$ of degree~$0$ is \emph{(\textbf{TN})-injective} (resp. \emph{(\textbf{TN})-surjective}, \emph{(\textbf{TN})-bijective}) if there exists an integer $n$ such that $u_k:\sh{M}_k\to\sh{N}_k$ is injective (resp. surjective, bijective) for $k\geq n$;
then $\widetilde{u}:\widetilde{\sh{M}}\to\widetilde{\sh{N}}$ is injective (resp. surjective, bijective) by \sref{II.2.7.2}, since this is a local property on $Y$, and taking \sref[I]{I.1.3.9} into account;
if $u$ is (\textbf{TN})-bijective, then we also say that $u$ is a \emph{(\textbf{TN})-isomorphism}.
\end{env}

\begin{proposition}[3.4.3]
\label{II.3.4.3}
Let $Y$ be a prescheme, and $\sh{S}$ a quasi-coherent graded $\sh{O}_Y$-algebra generated by $\sh{S}_1$, with $S_1$ assumed to be of finite type.
Let $\sh{M}$ be a quasi-coherent graded $\sh{S}$-module.
\begin{enumerate}
  \item[(i)] If $\sh{M}$ satisfies (\textbf{TF}), then $\widetilde{\sh{M}}$ is of finite type.
  \item[(ii)] Suppose that $\sh{M}$ satisfies (\textbf{TF});
    for $\widetilde{\sh{M}}=0$, it is necessary and sufficient for $\sh{M}$ to satisfy (\textbf{TN}).
\end{enumerate}
\end{proposition}

\begin{proof}
Since the questions are local on $Y$, we can reduce to the case where $Y$ is affine of ring $A$, $\sh{S}=\widetilde{S}$, where $S$ is a graded $A$-algebra such that the ideal $S_+$ is of finite type, and $\sh{M}=\widetilde{M}$, where $M$ is a graded $S$-module;
the proposition then follows from \sref{II.2.7.3}.
\end{proof}

\begin{theorem}[3.4.4]
\label{II.3.4.4}
Let $Y$ be a prescheme, and $\sh{S}$ a quasi-coherent graded $\sh{O}_Y$-algebra generated by $\sh{S}_1$, where $\sh{S}_1$ is assumed to be of finite type;
let $X=\Proj(\sh{S})$.
For every quasi-coherent $\sh{O}_X$-module $\sh{F}$, the canonical homomorphism $\beta$ \sref{II.3.3.4} is an isomorphism.
\end{theorem}

\begin{proof}
Note first of all that $\beta$ is defined, by \sref{II.3.4.1}.
To see that $\beta$ is an isomorphism, we can reduce to the case where $Y$ is affine of ring $A$, $\sh{S}=\widetilde{S}$, where $S$ is a graded $A$-algebra generated by $S_1$, and $S_1$ is an $A$-module of finite type.
It then suffices to apply \sref{II.2.7.5}.
\end{proof}

\begin{corollary}[3.4.5]
\label{II.3.4.5}
Under the hypotheses of \sref{II.3.4.4}, every quasi-coherent $\sh{O}_X$-module $\sh{F}$ is isomorphic to an $\sh{O}_X$-module of the form $\widetilde{\sh{M}}$, where $\sh{M}$ is a quasi-coherent graded $\sh{S}$-module.
If, further, $\sh{F}$ is of finite type, and if we assume that $Y$ is a quasi-compact scheme, or that the underlying space of $Y$ is Noetherian, then we can take $\sh{M}$ to be of finite type.
\end{corollary}

\oldpage[II]{60}
\begin{proof}
The first claim follows immediately from \sref{II.3.4.4} by taking $\sh{M}=\bbGamma_*(\sh{F})$.
To prove the second, it suffices to show that $\sh{M}$ is the inductive limit of its \emph{graded} sub-$\sh{S}$-modules of finite type $\sh{N}_\lambda$: indeed, it will follow from this that $\widetilde{\sh{M}}$ is the inductive limit of the $\widetilde{\sh{N}}_\lambda$ \sref{II.3.2.4}, and so $\sh{F}$ is the inductive limit of the $\beta(\widetilde{\sh{N}}_\lambda)$;
since $X$ is quasi-compact (\sref{II.3.4.1} and \sref[I]{I.6.3.1}) and $\sh{F}$ is of finite type, $\sh{F}$ will necessarily be equal to one of the $\beta(\widetilde{\sh{N}}_\lambda)$ \sref[0]{0.5.2.3}.

To define the $\sh{N}_\lambda$ having $\sh{M}$ as their inductive limit, it suffices to consider, for each $n\in\bb{Z}$, the quasi-coherent $\sh{O}_Y$-module $\sh{M}_n$, which is the inductive limit of its sub-$\sh{O}_Y$-modules $\sh{M}_n^{(\mu_n)}$ of finite type, by the hypotheses on $Y$ \sref[I]{I.9.4.9};
it is immediate that $\sh{P}_{\mu_n}=\sh{S}\cdot\sh{M}_n^{(\mu_n)}$ is a graded $\sh{S}$-module of finite type, and we immediately see that taking $\sh{N}_\lambda$ to be the finite sums of the $\sh{S}$-modules of the form $\sh{P}_{\mu_n}$ gives the desired objects.
\end{proof}

\begin{corollary}[3.4.6]
\label{II.3.4.6}
Suppose that the hypotheses of \sref{II.3.4.4} are satisfied, and further that the underlying space of $Y$ is quasi-compact;
let $\sh{F}$ be a quasi-coherent $\sh{O}_X$-module of finite type.
Then there exists $n_0$ such that, for all $n\geq n_0$, the canonical homomorphism $\sigma:p^*(p_*(\sh{F}(n)))\to\sh{F}(n)$ \sref[0]{0.4.4.3} is surjective.
\end{corollary}

\begin{proof}
For all $y\in Y$, let $U$ be an affine open neighbourhood of $y$ in $Y$.
There exists an integer $n_0(U)$ such that, for all $n\geq n_0(U)$, $\sh{F}(n)|p^{-1}(U)$ is generated by a finite number of its sections over $p^{-1}(U)$ \sref{II.2.7.9};
but these sections are the canonical images of sections of $p^*(p_*(\sh{F}(n)))$ over $p^{-1}(U)$ (\sref[0]{0.3.7.1} and \sref[0]{0.4.4.3}), so $\sh{F}(n)|p^{-1}(U)$ is equal to the canonical image of $p^*(p_*(\sh{F}(n)))|p^{-1}(U)$.
Finally, since $Y$ is quasi-compact, there exists a finite cover of $Y$ by affine opens $U_i$, and taking $n_0$ to be the largest of the $n_0(U_i)$ finishes the proof.
\end{proof}

\begin{remarks}[3.4.7]
\label{II.3.4.7}
If $p=(\psi,\theta):X\to Y$ is a morphism of ringed spaces, and $\sh{F}$ an $\sh{O}_X$-module, the fact that the canonical homomorphism $\sigma:p^*(p_*(\sh{F}))\to\sh{F}$ is surjective can be explained in the following way \sref[0]{0.4.4.1}: for all $x\in X$, and every section $s$ of $\sh{F}$ over an open neighbourhood $V$ of $x$, there exists an open neighbourhood $U$ of $p(x)$ in $Y$, a finite number of sections $t_i$ (for $1\leq i\leq m$) of $\sh{F}$ over $p^{-1}(U)$, a neighbourhood $W\subset V\cap p^{-1}(U)$ of $x$, and sections $a_i$ (for $1\leq i\leq m$) of $\sh{O}_X$ over $W$ such that
\[
  s|W = \sum_{i=1}^m a_i\cdot(t_i|W).
\]
If $Y$ is an \emph{affine scheme} and $p_*(\sh{F})$ is \emph{quasi-coherent}, this condition is equivalent to $\sh{F}$ being \emph{generated by its sections over $X$} \sref[0]{0.5.5.1}: indeed, if $Y=\Spec(A)$, we can suppose that $U=D(f)$ with $f\in A$;
then there exists an integer $n>0$ and sections $s_i$ of $\sh{F}$ over $X$ such that $t_i$ is the restriction to $p^{-1}(U)$ of $s_ig^n$, where $g=\theta(f)$ (by applying \sref[I]{I.1.4.1} to $p_*(\sh{F})$);
since $g$ is invertible over $p^{-1}(U)$, we thus have
\[
  s|W = \sum_i b_i\cdot(s_i|W)
\]
where $b_i=a_i(g|W)^{-n}$, whence the claim.
If $Y$ is affine, then the corollary \sref{II.3.4.6} recovers \sref{II.2.7.9}.

\oldpage[II]{61}
We thus conclude that, when $Y$ is an arbitrary prescheme, the following three conditions, for a quasi-coherent $\sh{O}_X$-module $\sh{F}$ such that $p_*(\sh{F})$ is \emph{quasi-coherent}, are equivalent:
\begin{enumerate}
  \item[a)] \emph{The canonical homomorphism $\sigma:p^*(p_*(\sh{F}))\to\sh{F}$ is surjective.}
  \item[b)] \emph{There exists a quasi-coherent $\sh{O}_Y$-module $\sh{G}$ and a surjective homomorphism $p^*(\sh{G})\to\sh{F}$.}
  \item[c)] \emph{For every affine open $U$ of $Y$, $\sh{F}|p^{-1}(U)$ is generated by its sections over $p^{-1}(U)$.}
\end{enumerate}

Indeed, we have just proven the equivalence between \emph{a)} and \emph{c)}.
It is also clear that \emph{a)} implies \emph{b)}, since $p_*(\sh{F})$ is quasi-coherent by hypothesis.
Conversely, every homomorphism $u:p^*(\sh{G})\to\sh{F}$ factors as $p^*(\sh{G})\to p^*(p_*(\sh{F}))\xrightarrow{\sigma}\sh{F}$ \sref[0]{0.3.5.4.4}, so if $u$ is surjective then so too is $\sigma$, which proves that \emph{b)} implies \emph{a)}.
\end{remarks}

\begin{corollary}[3.4.8]
\label{II.3.4.8}
Suppose that the hypotheses of \sref{II.3.4.4} are satisfied, and suppose further that $Y$ is a quasi-compact scheme, or that the underlying space of $Y$ is Noetherian.
Let $\sh{F}$ be a quasi-coherent $\sh{O}_X$-module of finite type;
then there exists an integer $n_0$ such that, for all $n\geq n_0$, $\sh{F}$ is isomorphic to a quotient of an $\sh{O}_X$-module of the form $(p^*(\sh{G}))(-n)$, where $\sh{G}$ is a quasi-coherent $\sh{O}_Y$-module of finite type (that depends on $n$).
\end{corollary}

\begin{proof}
Since the structure morphism $X\to Y$ is separated and of finite type, $p_*(\sh{F}(n))$ is quasi-coherent \sref[I]{I.9.2.2}[b)], and thus the inductive limit of its quasi-coherent sub-$\sh{O}_Y$-modules of finite type, by the hypotheses on $Y$ \sref[I]{I.9.4.9}.
We thus deduce, by \sref{II.3.4.6}, \sref[0]{0.4.3.2}, and \sref[0]{0.5.2.3}, that $\sh{F}(n)$ is the canonical image of an $\sh{O}_X$-module of the form $p^*(\sh{G})$, where $\sh{G}$ is a quasi-coherent sub-$\sh{O}_Y$-module of $p_*(\sh{F}(n))$ of finite type;
the corollary then follows from \sref{II.3.2.5.2} and \sref{II.3.2.7.1}.
\end{proof}


\subsection{Functorial behaviour}
\label{subsection:II.3.5}

\begin{env}[3.5.1]
\label{II.3.5.1}
Let $Y$ be a prescheme, and $\sh{S}$ and $\sh{S}'$ quasi-coherent positively-graded $\sh{O}_Y$-algebras;
let $X=\Proj(\sh{S})$ and $X'=\Proj(\sh{S}')$, and let $p$ and $p'$ be the structure morphisms from $X$ and $X'$ to $Y$.
Let $\vphi:\sh{S}'\to\sh{S}$ be an $\sh{O}_Y$-homomorphism of graded algebras.
For every affine open $U$ of $Y$, let $S_U=\Gamma(U,\sh{S})$ and $S'_U=\Gamma(U,\sh{S}')$;
the homomorphism $\vphi$ defines a homomorphism $\vphi_U:S'_U\to S_U$ of graded $A_U$-algebras, where $A_u=\Gamma(U,\sh{O}_Y)$.
There is a corresponding open subset $G(\vphi_U)$ in $p^{-1}(U)$ and morphism $\Phi_U:G(\vphi_U)\to p'^{-1}(U)$ \sref{II.2.8.1}.
Furthermore, if $V\subset U$ is an affine open, then the diagram
\[
\label{II.3.5.1.1}
  \xymatrix{
    S'_U \ar[r]^{\vphi_U} \ar[d]
    & S_U \ar[d]
  \\S'_V \ar[r]_{\vphi_V}
    & S_V
  }
\tag{3.5.1.1}
\]
commutes, and we immediately see, by definition \sref{II.2.8.1}, that we have
\[
  G(\vphi_V) = G(\vphi_U)\cap p^{-1}(V)
\]
and that $\Phi_V$ is the restriction  to $G(\vphi_V)$ of $\Phi_U$.
We have thus defined an open subset $G(\vphi)$ of $X$ such that $G(\vphi)\cap p^{-1}(U)=G(\vphi_U)$ for every affine open $U\subset Y$, and an \emph{affine} $Y$-morphism $\Phi:G(\vphi)\to X'$, which we say is \emph{associated} to $\vphi$, and which we denote by $\Proj(\vphi)$.
\oldpage[II]{62}
If, for all $y\in Y$, there is an affine neighbourhood $U$ of $y$ such that the $\Gamma(U,\sh{O}_Y)$-module $\Gamma(U,\sh{S}_+)$ is generated by $\vphi(\Gamma(U,\sh{S}'_+))$, then $G(\vphi_U)=p^{-1}(U)$, and so $G(\vphi)=X$.
\end{env}

\begin{proposition}[3.5.2]
\label{II.3.5.2}
\medskip\noindent
\begin{enumerate}
  \item[(i)] If $\sh{M}$ is a quasi-coherent graded $\sh{S}$-module, then there exists a canonical functorial isomorphism from the $\sh{O}_{X'}$-module $(\sh{M}_{[\vphi]})\supertilde$ to the $\sh{O}_{X'}$-module $\Phi_*(\widetilde{\sh{M}}|G(\vphi))$.
  \item[(ii)] If $\sh{M}'$ is a quasi-coherent graded $\sh{S}'$-module, then there exists a canonical functorial isomorphism $\nu$ from the $(\sh{O}_X|G(\vphi))$-module $\Phi^*(\widetilde{\sh{M}'})$ to the $(\sh{O}_X|G(\vphi))$-module $(\sh{M}'\otimes_{\sh{S}'}\sh{S})\supertilde|G(\vphi)$.
    If $\sh{S}'$ is generated by $\sh{S}'_1$, then $\nu$ is an isomorphism.
\end{enumerate}
\end{proposition}

\begin{proof}
The homomorphisms in question are indeed already defined if $Y$ is affine (\sref{II.2.8.7} and \sref{II.2.8.8}), and in the general case it suffices to check that they are compatible with the restriction of an affine open of $Y$ to a smaller open, which follows immediately from the commutativity of \sref{II.3.5.1.1}.
\end{proof}

In particular, for all $n\in\bb{Z}$, we have a canonical homomorphism
\[
\label{II.3.5.2.1}
  \Phi^*(\sh{O}_{X'}(n)) \to \sh{O}_X(n)|G(\vphi).
\tag{3.5.2.1}
\]

\begin{proposition}[3.5.3]
\label{II.3.5.3}
Let $Y$ and $Y'$ be preschemes, $\psi:Y'\to Y$ a morphism, and $\sh{S}$ a quasi-coherent graded $\sh{O}_Y$-algebra;
set $\sh{S}'=\psi^*(\sh{S})$.
Then the $Y'$-scheme $X'=\Proj(\sh{S}')$ is canonically identified with $\Proj(\sh{S})\times_Y Y'$.
Furthermore, if $\sh{M}$ is a quasi-coherent graded $\sh{S}$-module, then the $\sh{O}_{X'}$-module $(\vphi^*(\sh{M}))\supertilde$ can be identified with $\widetilde{\sh{M}}\otimes_Y\sh{O}_{Y'}$.
\end{proposition}

\begin{proof}
Note first of all that $\psi^*(\sh{S})$ and $\psi^*(\sh{M})$ are quasi-coherent $\sh{O}_{Y'}$-modules, as are their homogenous components \sref[0]{0.5.1.4}.
Let $U$ be an affine open of $Y$, $U'\subset\psi^{-1}(U)$ an affine open of $Y'$, and $A$ and $A'$ the rings of $U$ and $U'$, respectively;
then $\sh{S}|U=\widetilde{S}$, where $S$ is a graded $A$-algebra, and $\sh{S}'|U'$ can be identified with $(S\otimes_A A')\supertilde$ \sref[I]{I.1.6.5};
the first claim then follows from \sref{II.2.8.10} and \sref[I]{I.3.2.6.2}, since we immediately see that the projection $\Proj(\sh{S}'|U')\to\Proj(\sh{S}|U)$ defined by the above identification is compatible with the restriction operations on $U$ and $U'$, and thus indeed defines a morphism $\Proj(\sh{S}')\to\Proj(\sh{S})$.
Now let
\begin{align*}
  q &: \Proj(\sh{S}) \to Y
\\q'&: \Proj(\sh{S}') \to Y'
\end{align*}
be the structure morphisms;
$q'^{-1}(U')$ can then be identified with $q^{-1}(U)\times_U U'$, and the two sheaves $(\psi^*(\sh{M}))\supertilde|q'^{-1}(U')$ and $(\widetilde{\sh{M}}\otimes_Y\sh{O}_{Y'})|q'^{-1}(U')$ are then both canonically identified with $(M\otimes_A A')\supertilde$, where we set $M=\Gamma(U,\sh{M})$, by \sref{II.2.8.10} and \sref[I]{I.1.6.5};
whence the second claim, since we can again immediately see the compatibility of the above identifications with the restriction operations.
\end{proof}

\begin{corollary}[3.5.4]
\label{II.3.5.4}
With the notation of \sref{II.3.5.3}, $\sh{O}_{X'}(n)$ is canonically identified with $\sh{O}_X(n)\otimes_Y\sh{O}_{Y'}$ for all $n\in\bb{Z}$ (where $X=\Proj(\sh{S})$).
\end{corollary}

\begin{proof}
Indeed, with the notation of \sref{II.3.5.3}, it is clear that $\psi^*(\sh{S}(n))=\sh{S}'(n)$ for all $n\in\bb{Z}$.
\end{proof}

\begin{env}[3.5.5]
\label{II.3.5.5}
Keeping the above notation, denote by $\Psi$ the canonical projection $X'\to X$, and set $\sh{M}'=\psi^*(\sh{M})$;
we further suppose that $\sh{S}$ is generated by $\sh{S}_1$, and that $X$ is of finite type over $Y$;
\oldpage[II]{63}
it then follows that $\sh{S}'$ is generated by $\sh{S}'_1$ (as can be seen by reducing to the case where $Y$ and $Y'$ are affine), and that $X'$ is of finite type over $Y'$ \sref[I]{I.6.3.4}.
Let $\sh{F}$ be an $\sh{O}_X$-module, and set $\sh{F}'=\Psi^*(\sh{F})$;
it then follows from \sref{II.3.5.4} and \sref[0]{0.4.3.3} that $\sh{F}'(n)=\Psi^*(\sh{F}(n))$ for all $n\in\bb{Z}$.
We further define a canonical $\Psi$-homomorphism $\theta_n:q_*(\sh{F}(n))\to q'_*(\sh{F}'(n))$ in the following way: given the commutativity of the diagram
\[
  \xymatrix{
    X \ar[d]_{q}
    & X' \ar[l]_{\Psi} \ar[d]^{q'}
  \\Y
    & Y' \ar[l]^{\psi}
  }
\]
it is enough to define a homomorphism $q_*(\sh{F}(n))\to\psi_*(q'_*(\Psi^*(\sh{F}(n))))=q_*(\Psi_*(\Psi^*(\sh{F}(n))))$, and it suffices to take the homomorphism $\theta_n=q_*(\rho_n)$, where $\rho_n$ is the canonical homomorphism $\sh{F}(n)\to\Psi_*(\Psi^*(\sh{F}(n)))$ \sref[0]{0.4.4.3}.
It is immediate that, for every affine open $U$ of $Y$ and every affine open $U'$ of $Y'$ such that $U'\subset\psi^{-1}(U)$, the homomorphism $\theta_n$ gives, on sections, the canonical homomorphism \sref[0]{0.3.7.2} $\Gamma(q^{-1}(U),\sh{F}(n))\to\Gamma(q'^{-1}(U'),\sh{F}'(n))$.
The commutativity of \sref{II.2.8.13.2} then shows that, if $\sh{F}$ is quasi-coherent, the diagram
\[
  \xymatrix{
    \sh{F} \ar[r]^{\rho}
    & \sh{F}'
  \\(\bbGamma_*(\sh{F}))\supertilde \ar[u]^{\beta_{\sh{F}}} \ar[r]_{\widetilde{\theta}}
    & (\bbGamma_*(\sh{F}'))\supertilde \ar[u]_{\beta_{\sh{F}'}}
  }
\]
commutes (the top horizontal arrow being the canonical $\Psi$-morphism $\sh{F}\to\Psi^*(\sh{F})$).

Similarly, the commutativity of \sref{II.2.8.13.1} shows that the diagram
\[
  \xymatrix{
    \bbGamma_*(\widetilde{\sh{M}}) \ar[r]^{\theta}
    & \bbGamma_*(\widetilde{\sh{M}'})
  \\\sh{M} \ar[r]_{\rho} \ar[u]^{\alpha_{\sh{M}}}
    & \sh{M}' \ar[u]_{\alpha_{\sh{M}'}}
  }
\]
commutes (the bottom horizontal arrow being the canonical $\psi$-morphism $\sh{M}\to\psi^*(\sh{M})$).
\end{env}

\begin{env}[3.5.6]
\label{II.3.5.6}
Now consider preschemes $Y$ and $Y'$, a morphism $g:Y'\to Y$, a quasi-coherent graded $\sh{O}_Y$-algebra (resp. quasi-coherent graded $\sh{O}_{Y'}$-algebra) $\sh{S}$ (resp. $\sh{S}'$), and a $g$-morphism of graded algebras $u:\sh{S}\to\sh{S}'$, i.e. a $\sh{O}_Y$-homomorphism of graded algebras $\sh{S}\to g_*(\sh{S}')$;
we already know that this is equivalent to giving a $\sh{O}_{Y'}$-homomorphism of graded algebras $u^\sharp:g^*(\sh{S})\to\sh{S}'$.
We thus canonically obtain from $u^\sharp$ a $Y'$-morphism $W=\Proj(u^\sharp):G(u^\sharp)\to\Proj(g^*(\sh{S}))$, where $G(u^\sharp)$ is an open of $X'=\Proj(\sh{S}')$ \sref{II.3.5.1}.
We also know that $X''=\Proj(g^*(\sh{S}))$ is canonically identified with $X\times_Y Y'$, by taking $X=\Proj(\sh{S})$ \sref{II.3.5.3};
\oldpage[II]{64}
composing the first projection $p:X\times_Y Y'\to X$ with $\Proj(u^\sharp)$, we thus obtain a morphism $v:G(u^\sharp)\to X$, which we denote by $\Proj(u)$, and which is such that the diagram
\[
  \xymatrix{
    G(u^\sharp) \ar[r]^{v} \ar[d]
    & X \ar[d]
  \\Y' \ar[r]_{g}
    & Y
  }
\]
commutes.

Furthermore, for every quasi-coherent graded $\sh{O}_Y$-module $\sh{M}$, we have a canonical $v$-morphism
\[
\label{II.3.5.6.1}
  \nu : \widetilde{\sh{M}} \to (g^*(\sh{M})\otimes_{g^*(\sh{S})}\sh{S}')\supertilde|G(u^\sharp).
\tag{3.5.6.1}
\]

Indeed, $\nu^\sharp$ is given by composing the homomorphisms
\[
  v^*(\widetilde{\sh{M}})
  = w^*(p^*(\widetilde{\sh{M}}))
  \to w^*((g^*(\sh{M}))\supertilde)
  \to (g^*(\sh{M})\otimes_{g^*(\sh{S})}\sh{S}')\supertilde|G(u^\sharp)
\]
where the first arrow comes from the isomorphism \sref{II.3.5.3} and the second is the homomorphism \sref{II.3.5.2}[(i)];
if $\sh{S}$ is generated by $\sh{S}_1$, then it follows from \sref{II.3.5.2} that $\nu^\sharp$ is an \emph{isomorphism}.

As a particular case of \sref{II.3.5.6.1}, we have, for all $n\in\bb{Z}$, a canonical $v$-morphism
\[
\label{II.3.5.6.2}
  \nu : \sh{O}_X(n) \to \sh{O}_{X'}(n)|G(u^\sharp).
\tag{3.5.6.2}
\]
\end{env}


\subsection{Closed subpreschemes of a prescheme $\operatorname{Proj}(\mathcal{S})$}
\label{subsection:II.3.6}

\begin{env}[3.6.1]
\label{II.3.6.1}
Let $Y$ be a prescheme, and $\vphi:\sh{S}\to\sh{S}'$ a degree~$0$ homomorphism of quasi-coherent graded $\sh{O}_Y$-algebras.
We say that $\vphi$ is \emph{(\textbf{TN})-surjective} (resp. \emph{(\textbf{TN})-injective}, \emph{(\textbf{TN})-bijective}) if there exists $n$ such that, for all $k\geq n$, $\vphi_k:\sh{S}_k\to\sh{S}'_k$ is surjective (resp. injective, bijective).
If this is the case, then we can reduce the study of the corresponding morphism $\Phi:\Proj(\sh{S}')\to\Proj(\sh{S})$ to the case where $\vphi$ is \emph{surjective} (resp. \emph{injective}, \emph{bijective}).
We prove this as in \sref{II.2.9.1} (which is the particular case where $Y$ is affine) by using \sref{II.3.1.8}.
Instead of saying that $\vphi$ is (\textbf{TN})-bijective, we also say that it is a \emph{(\textbf{TN})-isomorphism}.
\end{env}

\begin{proposition}[3.6.2]
\label{II.3.6.2}
Let $Y$ be a prescheme, and $\sh{S}$ a quasi-coherent graded $\sh{O}_Y$-algebra;
set $X=\Proj(\sh{S})$.
\begin{enumerate}
  \item[(i)] If $\vphi:\sh{S}\to\sh{S}'$ is a (\textbf{TN})-surjective homomorphism of graded $\sh{O}_Y$-algebras, then the corresponding morphism $\Phi=\Proj(\vphi)$ \sref{II.3.5.1} is defined on all of $\Proj(\sh{S}')$ and is a closed immersion of $\Proj(\sh{S}')$ into $X$.
    If $\sh{J}$ is the kernel of $\vphi$, then the closed subprescheme of $X$ associated to $\Phi$ is defined by the quasi-coherent sheaf of ideals $\widetilde{\sh{J}}$ in $\sh{O}_X$.
  \item[(ii)] Suppose further that $\sh{S}_0=\sh{O}_Y$, that $\sh{S}$ is generated by $\sh{S}_1$, and that $\sh{S}_1$ is of finite type.
    Let $X'$ be a closed subprescheme of $X=\Proj(\sh{S})$, defined by a quasi-coherent sheaf of ideals $\sh{I}$ in $\sh{O}_X$.
\oldpage[II]{65}
    Let $\sh{J}$ be the quasi-coherent graded sheaf of ideals of $\sh{S}$, given by the inverse image of $\bbGamma_*(\sh{I})$ by the canonical homomorphism $\alpha:\sh{S}\to\bbGamma_*(\sh{O}_X)$ \sref{II.3.3.2}, and set $\sh{S}'=\sh{S}|\sh{J}$.
    Then $X'$ is the subprescheme associated \sref[I]{I.4.2.1} to the closed immersion $\Proj(\sh{S}')\to X$ corresponding to the canonical homomorphism $\sh{S}\to\sh{S}'$ of graded $\sh{O}_Y$-algebras.
\end{enumerate}
\end{proposition}

\begin{proof}
\medskip\noindent
\begin{enumerate}
  \item[(i)] We can assume that $\vphi$ is surjective \sref{II.3.6.1}.
    Then, for every affine open $U$ of $Y$, $\Gamma(U,\sh{S})\to\Gamma(U,\sh{S}')$ is surjective \sref[I]{I.1.3.9}, so \sref{II.3.5.1} $G(\vphi)=X$.
    We can then immediately reduce to proving the proposition in the case where $Y$ is affine, and this follows from \sref{II.2.9.2}[(i)].
  \item[(ii)] We can reduce to proving that the homomorphism $\widetilde{\sh{J}}\to\sh{O}_X$ induced by the canonical injection $\sh{J}\to\sh{S}$ is an isomorphism from $\widetilde{\sh{J}}$ to $\sh{I}$;
    since the question is local on $Y$, we can take $Y$ to be affine of ring $A$, which implies that $\sh{S}=\widetilde{S}$, where $S$ is a graded $A$-algebra generated by $S_1$, with $S_1$ of finite type over $A$.
    It then suffices to apply \sref{II.2.9.2}[(ii)].
\end{enumerate}
\end{proof}

\begin{corollary}[3.6.3]
\label{II.3.6.3}
Under the conditions of \sref{II.3.6.2}[(i)], suppose further that $\sh{S}$ is generated by $\sh{S}_1$.
Then $\Phi^*(\sh{O}_X(n))$ is canonically identified with $\sh{O}_{X'}(n)$ for all $n\in\bb{Z}$.
\end{corollary}

\begin{proof}
We have defined such a canonical isomorphism when $Y$ is affine \sref{II.2.9.3};
in the general case, it suffices to show that the isomorphisms thus defined for each affine open $U$ of $Y$ are compatible with the passage from $U$ to an affine open $U'\subset U$, which is immediate.
\end{proof}

\begin{corollary}[3.6.4]
\label{II.3.6.4}
Let $Y$ be a prescheme, $\sh{S}$ a quasi-coherent graded $\sh{O}_Y$-algebra generated by $\sh{S}_1$, $\sh{M}$ a quasi-coherent $\sh{O}_Y$-module, $u$ a surjective $\sh{O}_Y$-homomorphism $\sh{M}\to\sh{S}_1$, and $\overline{u}:\bb{S}_{\sh{O}_Y}(\sh{M})\to\sh{S}$ the homomorphism of graded $\sh{O}_Y$-algebras that extends $u$ \sref{II.1.7.4}.
Then the morphism corresponding to $\overline{u}$ is a closed immersion of $\Proj(\sh{S})$ into $\Proj(\bb{S}_{\sh{O}_Y}(\sh{M}))$.
\end{corollary}

\begin{proof}
Indeed, $\overline{u}$ is surjective by hypothesis, and we apply \sref{II.3.6.1}[(i)].
\end{proof}


\subsection{Morphisms from a prescheme to a homogeneous spectrum}
\label{subsection:II.3.7}


% \subsection{Criteria for immersion into a homogeneous spectrum}
% \label{subsection:II.3.8}

\section{Projective bundles; ample sheaves}
\label{section:II.4}


\subsection{Definition of projective bundles}
\label{subsection:II.4.1}

\begin{definition}[4.1.1]
\label{II.4.1.1}
Let $Y$ be a prescheme, $\sh{E}$ a quasi-coherent $\sh{O}_Y$-module, and $\bb{S}_{\sh{O}_Y}(\sh{E})$ the symmetric $\sh{O}_Y$-algebra of $\sh{E}$ \sref{II.1.7.4}, which is quasi-coherent \sref{II.1.7.7}.
We define the \emph{projective bundle on $Y$ defined by $\sh{E}$}, denoted $\bb{P}(\sh{E})$, to be the $Y$-scheme $P=\Proj(\bb{S}_{\sh{O}_Y}(\sh{E}))$.
The $\sh{O}_P$-module $\sh{O}_P(1)$ is called the \emph{fundamental sheaf on $P$}.
\end{definition}

When $Y$ is affine of ring $A$, then we have $\sh{E}=\widetilde{E}$ for some $A$-module $E$, and we then write $\bb{P}(E)$ instead of $\bb{P}(\widetilde{E})$.

When we take $\sh{E}=\sh{O}_Y^n$, we write $\bb{P}_Y^{n-1}$ instead of $\bb{P}(\sh{E})$;
if, further, $Y$ is affine of ring $A$, then we also write $\bb{P}_A^{n-1}$ instead of $\bb{P}_Y^{n-1}$.
Since $\bb{S}_{\sh{O}_Y}(\sh{O}_Y)$ is canonically identified with $\sh{O}_Y[T]$ \sref{II.1.7.4}, $\bb{P}_Y^0$ is canonically identified with $Y$ \sref{II.3.1.7};
Example~\sref{II.2.4.3} is then exactly $\bb{P}_K^1$.

\begin{env}[4.1.2]
\label{II.4.1.2}
Let $\sh{E}$ and $\sh{F}$ be quasi-coherent $\sh{O}_Y$-modules;
let $u:\sh{E}\to\sh{F}$ be an $\sh{O}_Y$-homomorphism;
there is a canonically corresponding homomorphism $\bb{S}(u):\bb{S}_{\sh{O}_Y}(\sh{E})\to\bb{S}_{\sh{O}_Y}(\sh{F})$ of graded $\sh{O}_Y$-algebras \sref{II.1.7.4}.
If $u$ is \emph{surjective}, then so too is $\bb{S}(u)$, and thus \sref{II.3.6.2}[(i)] $\Proj(\bb{S}(u))$ is a \emph{closed immersion} $\bb{P}(\sh{F})\to\bb{P}(\sh{E})$, which we denote by $\bb{P}(u)$.
We can thus say that $\bb{P}(\sh{E})$ is a \emph{contravariant functor} in $\sh{E}$, with the condition that we only consider \emph{surjective} morphisms of quasi-coherent $\sh{O}_Y$-modules.

Still supposing that $u$ is surjective, and letting $P=\bb{P}(\sh{E})$, $Q=\bb{P}(\sh{F})$, and $j=\bb{P}(u)$, we have, up to isomorphism, that
\[
\label{II.4.1.2.1}
  j^*(\sh{O}_P(n)) = \sh{O}_Q(n)
  \qquad\mbox{for all $n\in\bb{Z}$}
  \tag{4.1.2.1}
\]
by \sref{II.3.6.3}.
\end{env}

\begin{env}[4.1.3]
\label{II.4.1.3}
Now let $\psi:Y'\to Y$ be a morphism, and let $\sh{E}'=\psi^*(\sh{E})$;
then $\bb{S}_{\sh{O}_{Y'}}(\sh{E}') = \psi^*(\bb{S}_{\sh{O}_Y}(\sh{E}))$ \sref{II.1.7.5};
thus \sref{II.3.5.3}
\[
\label{II.4.1.3.1}
  \bb{P}(\psi^*(\sh{E})) = \bb{P}(\sh{E})\times_Y Y'
  \tag{4.1.3.1}
\]
up to canonical isomorphism;
furthermore, we clearly have that
\[
  \psi^*((\bb{S}_{\sh{O}_Y}(\sh{E}))(n)) = (\bb{S}_{\sh{O}_{Y'}}(\sh{E}'))(n)
\]
for all $n\in\bb{Z}$, whence, letting $P=\bb{P}(\sh{E})$ and $P'=\bb{P}(\sh{E}')$, we have \sref{II.3.5.4}, up to isomorphism, that
\[
\label{II.4.1.3.2}
  \sh{O}_{P'}(n) = \sh{O}_p(n)\otimes_Y\sh{O}_{Y'}
  \qquad\mbox{for all $n\in\bb{Z}$.}
  \tag{4.1.3.2}
\]
\end{env}

\oldpage[II]{72}
\begin{proposition}[4.1.4]
\label{II.4.1.4}
Let $\sh{L}$ be an invertible $\sh{O}_Y$-module.
For every quasi-coherent $\sh{O}_Y$-module $\sh{E}$, there exists a canonical $Y$-isomorphism $i_\sh{L}:\bb{P}(\sh{E})\xrightarrow{\sim}\bb{P}(\sh{E}\otimes\sh{L})$;
furthermore, if we let $P=\bb{P}(\sh{E})$ and $Q=\bb{P}(\sh{E}\otimes\sh{L})$, then $i_\sh{L}^*(\sh{O}_Q(n))$ is canonically isomorphic to $\sh{O}_P(n)\otimes_Y\sh{L}^{\otimes n}$ for all $n\in\bb{Z}$.
\end{proposition}

\begin{proof}
Note first of all that, if $A$ is a ring, $E$ an $A$-module, and $L$ a \emph{free monogenous} $A$-module, then we can canonically define a homomorphism of $A$-modules
\[
  \bb{S}_n(E\otimes L) \to \bb{S}_n(E)\otimes L^{\otimes n}
\]
by sending $(x_1\otimes y_1)\ldots(x_n\otimes y_n)$ to the element
\[
  (x_1x_2\ldots x_n)\otimes(y_1\otimes y_2\otimes\ldots\otimes y_n)
  \qquad\mbox{($x_i\in E$, $y_i\in L$, for $i\leq i\leq n$);}
\]
we can immediately see (by restricting to the case where $L=A$) that this homomorphism is in fact an isomorphism.
We thus obtain a canonical isomorphism of graded $A$-algebras $\bb{S}_A(E\otimes L)\xrightarrow{\sim}\bigoplus_{n\geq0}\bb{S}_n(E)\otimes L^{\otimes n}$.
By returning to the conditions of \sref{II.4.1.4}, the above remarks allow us to define a canonical isomorphism of graded $\sh{O}_Y$-algebras
\[
\label{II.4.1.4.1}
  \bb{S}_{\sh{O}_Y}(\sh{E}\otimes_{\sh{O}_Y}\sh{L}) \xrightarrow{\sim} \bigoplus_{n\geq0}\bb{S}_n(\sh{E})\otimes_{\sh{O}_Y}\sh{L}^{\otimes n}
  \tag{4.1.4.1}
\]
by defining this isomorphism as an isomorphism of presheaves, and taking into account \sref{II.1.7.4}, \sref[I]{I.1.3.9}, and \sref[I]{I.1.3.12}.
The proposition then follows from \sref{II.3.1.8}[(iii)] and \sref{II.3.2.10}.
\end{proof}

\begin{env}[4.1.5]
\label{II.4.1.5}
With the hypotheses of \sref{II.4.1.1}, let $P=\bb{P}(\sh{E})$, and denote by $p$ the structure morphism $P\to Y$.
Since, by definition, $\sh{E}=(\bb{S}_{\sh{O}_Y}(\sh{E}))_1$, we have a canonical homomorphism $\alpha_1:\sh{E}\to p_*(\sh{O}_P(1))$ \sref{II.3.3.2.2}, and thus \sref[0]{0.4.4.3} also a canonical homomorphism
\[
\label{II.4.1.5.1}
  \alpha_1^\sharp: p^*(\sh{E}) \to \sh{O}_P(1).
  \tag{4.1.5.1}
\]
\end{env}

\begin{proposition}[4.1.6]
\label{II.4.1.6}
The canonical homomorphism \sref{II.4.1.5.1} is surjective.
\end{proposition}

\begin{proof}
We have seen, in \sref{II.3.3.2}, that $\alpha_1^\sharp$ corresponds functorially to the canonical homomorphism $\sh{E}\otimes_{\sh{O}_Y}\bb{S}_{\sh{O}_Y}(\sh{E}) \to (\bb{S}_{\sh{O}_Y}(\sh{E}))(1)$;
since, by definition, $\sh{E}$ generates $\bb{S}_{\sh{O}_Y}(\sh{E})$, this homomorphism is surjective, whence the conclusion, by \sref{II.3.2.4}
\end{proof}


\subsection{Morphisms from a prescheme to a projective bundle}
\label{subsection:II.4.2}

\begin{env}[4.2.1]
\label{II.4.2.1}
Keeping the notation of \sref{II.4.1.5}, let $X$ be a $Y$-prescheme, $q:X\to Y$ the structure morphism, and let $r:X\to P$ be a $Y$-\emph{morphism} such that the following diagram commutes:
\[
  \xymatrix{
    P \ar[d]_p & X \ar[l]_r \ar[dl]^q
  \\Y
  }
\]
\oldpage[II]{73}

Since the functor $r^*$ is right exact \sref[0]{0.4.3.1}, we obtain, from the surjective homomorphism in \sref{II.4.1.5.1}, a surjective homomorphism
\[
  r^*(\alpha_1^\sharp): r^*(p^*(\sh{E})) \to r^*(\sh{O}_P(1)).
\]

But $r^*(p^*(\sh{E}))=q^*(\sh{E})$, and $r^*(\sh{O}_P(1))$ is locally isomorphic to $r^*(\sh{O}_P)=\sh{O}_X$, or, in other words, the latter is an \emph{invertible} sheaf $\sh{L}_r$ on $\sh{O}_X$, and so we have defined, given $r$, a canonical surjective $\sh{O}_X$-homomorphism
\[
\label{II.4.2.1.1}
  \vphi_r:q^*(\sh{E}) \to \sh{L}_r.
  \tag{4.2.1.1}
\]

When $Y=\Spec(A)$ is affine, and $\mathscr{E}=\widetilde{E}$, we can further clarify this homomorphism in the following way:
given $f\in E$, it follows from \sref{II.2.6.3} that
\[
\label{II.4.2.1.2}
  r^{-1}(D_+(f)) = X_{\vphi_r^\flat(f)}.
  \tag{4.2.1.2}
\]

Now let $V$ be an affine open subset of $X$ that is contained inside $r^{-1}(D_+(f))$, and let $B$ be its ring, which is an $A$-algebra;
let $S=\bb{S}_A(E)$;
the restriction of $r$ to $V$ corresponds to an $A$-homomorphism $\omega:\bb{S}_f\to B$, and we have that $q^*(\sh{E})|V = (E\otimes_A B)\supertilde$ and $\sh{L}_r|V = \widetilde{L_r}$, whence $L_r = (S(1))_{(f)}\otimes_{S_{(f)}}B_{[\omega]}$ \sref[I]{I.1.6.5}.
The restriction of $\vphi_r$ to $q^*(\sh{E})|V$ corresponds to the $B$-homomorphism $u:E\otimes_A B\to L_r$, which sends $x\otimes1$ to $(x/1)\otimes f = (f/1)\otimes\omega(x/f)$.
The canonical extension of $\vphi_r$ to a homomorphism of $\sh{O}_X$-algebras
\[
  \psi_r: q^*(\bb{S}(\sh{E})) = \bb{S}(q^*(\sh{E})) \to \bb{S}(\sh{L}_r) = \bigoplus_{n\geq0}\sh{L}_r^{\otimes n}
\]
is thus such that the restriction of $\psi_r$ to $q^*(\bb{S}_n(\sh{E}))|V$ corresponds to the homomorphism $\bb{S}_n(\sh{E}\otimes_A B) = \bb{S}_n(E)\otimes_A B \to L_r^{\otimes n}$ that sends $s\otimes1$ to $(f/1)^{\otimes n}\otimes\omega(s/f^n)$.
\end{env}

\begin{env}[4.2.2]
\label{II.4.2.2}
Conversely, suppose that we are given a morphism $q:X\to Y$, an invertible $\sh{O}_X$-module $\sh{L}$, and a quasi-coherent $\sh{O}_Y$-module $\sh{E}$;
to each homomorphism $\vphi:q^*(\sh{E})\to\sh{L}$ there canonically corresponding homomorphism of quasi-coherent $\sh{O}_X$-algebras
\[
  \psi: \bb{S}(q^*(\sh{E})) = q^*(\bb{S}(\sh{E})) \to \bigoplus_{n\geq0}\sh{L}^{\otimes n}
\]
and thus \sref{II.3.7.1} a $Y$-morphism $r_{\sh{L},\psi}:G(\psi)\to\Proj(\bb{S}(\sh{E}))=\bb{P}(\sh{E})$, which we denote by $r_{\sh{L},\vphi}$.
If $\vphi$ is \emph{surjective}, then so too is $\psi$, and thus \sref{II.3.7.4} $r_{\sh{L},\vphi}$ is \emph{everywhere defined}.
Furthermore, with the notation of \sref{II.4.2.1} and \sref{II.4.2.2}:
\end{env}

\begin{proposition}[4.2.3]
\label{II.4.2.3}
Given a morphism $q:X\to Y$ and a quasi-coherent $\sh{O}_Y$-module $\sh{E}$, maps $r\to(\sh{L}_r,\vphi_r)$ and $(\sh{L},\vphi)\to r_{\sh{L},\vphi}$ give a bijective correspondence between the set of $Y$-morphisms $r:X\to\bb{P}(\sh{E})$ and the set of equivalence classes of pairs $(\sh{L},\vphi)$ of an invertible $\sh{O}_X$-module $\sh{L}$ and a surjective homomorphism $\vphi:q^*(\sh{E})\to\sh{L}$, where such pairs $(\sh{L},\vphi)$ and $(\sh{L}',\vphi')$ are defined to be equivalent if there exists an $\sh{O}_X$-isomorphism $\tau:\sh{L}\xrightarrow{\sim}\sh{L}'$ such that $\vphi'=\tau\circ\vphi$.
\end{proposition}

\begin{proof}
Start first with a $Y$-morphism $r:X\to\bb{P}(\sh{E})$, and construct $\sh{L}_r$ and $\vphi_r$ \sref{II.4.2.1}, and let $r'=r_{\sh{L}_r,\vphi_r}$;
it follows immediately from \sref{II.4.2.1} and \sref{II.3.7.2} that the morphisms $r$ and $r'$ are identical (by taking the generator of $\sh{L}_r$ to be the element $(f/1)\otimes1$ to define the homomorphisms $v_n$ of \sref{II.3.7.2}).
Conversely, take a pair $(\sh{L},\vphi)$ and construct
\oldpage[II]{74}
$r''=r_{\sh{L},\vphi}$, and then $\sh{L}_{r''}$ and $\vphi_{r''}$;
we will show that there exists a canonical isomorphism $\tau:\sh{L}_{r''}\xrightarrow{\sim}\sh{L}$ such that $\vphi=\tau\circ\vphi_{r''}$;
to define it, we can restrict to the case where $Y=\Spec(A)$ and $X=\Spec(B)$ are affine, and (with the notation of \sref{II.4.2.1} and \sref{II.3.7.2}) associate to each element $(x/1)\otimes1$ of $L_{r''}$ (where $x\in E$) the element $v_1(x)c$ of $L$.
We immediately see that $\tau$ does not depend on the chosen generator $c$ of $L$;
since $v_1$ is surjective by hypothesis, to prove that $\tau$ is an isomorphism it suffices to to show that, if $x/1=0$ in $(S(1))_{(f)}$, then $v_1(x)/1=0$ in $B_g$;
but the first equality implies that $f^nx=0$ in $\bb{S}_{n+1}(E)$ for some $n$, and this implies that $v_{n+1}(f^nx) = g^nv_1(x) = 0$ in $B$, whence the conclusion.
Finally, it is immediate that, for any two equivalent pairs $(\sh{L},\vphi)$ and $(\sh{L}',\vphi')$, we have $r_{\sh{L},\vphi}=r_{\sh{L}',\vphi'}$.
\end{proof}

In particular, for $X=Y$:
\begin{theorem}[4.2.4]
\label{II.4.2.4}
The set of $Y$-sections of $\bb{P}(\sh{E})$ is in canonical bijective correspondence with the set of quasi-coherent sub-$\sh{O}_Y$-modules $\sh{F}$ of $\sh{E}$ such that $\sh{E}/\sh{F}$ is invertible.
\end{theorem}

We note that this property corresponds to the classical definition of ``projective space'' as the set of hyperplanes of a vector space (the classical case corresponding to $Y=\Spec(K)$, where $K$ is a field, and $\sh{E}=\widetilde{E}$, where $E$ is a finite-dimensional $K$-vector space; the $\sh{F}$ having the property described in \sref{II.4.2.4} then correspond to the hyperplanes of $E$, and we already know that the $Y$-sections of $\bb{P}(\sh{E})$ are then the \emph{$K$-rational points of $\bb{P}(\sh{E})$} \sref[I]{I.3.4.5}).

\begin{remark}[4.2.5]
\label{II.4.2.5}
Since there is a canonical bijective correspondence between $Y$-morphisms from $X$ to $P$ and their graph morphisms, $X$-sections of $P\times_Y X$ \sref[I]{I.3.3.14}, we see that, conversely, \sref{II.4.2.3} can be deduced from \sref{II.4.2.4}.
Denote by $\Hyp_Y(X,\sh{E})$ the set of quasi-coherent sub-$\sh{O}_X$-modules $\sh{F}$ of $\sh{E}\otimes_Y\sh{O}_X=q^*(\sh{E})$ such that $q^*(\sh{E})/\sh{F}$ is an invertible $\sh{O}_X$-module.
If $g:X'\to X$ is a $Y$-morphism, then it follows from the fact that $g^*$ is right exact that $g^*q^*(\sh{E})/\sh{F})=g^*q^*(\sh{E}))/g^*(\sh{F})$, and so the latter sheaf is invertible, and thus $\Hyp_Y(X,\sh{E})$ is a \emph{contravariant functor} into the category of $Y$-preschemes.
We can thus interpret the theorem \sref{II.4.2.4} as defining a \emph{canonical isomorphism} of functors $\Hom_Y(X,\bb{P}(\sh{E}))$ and $\Hyp_Y(X,\sh{E})$, where both functors are contravariant in the variable $X$ and map into the category of $Y$-preschemes.
This also gives a characterisation of the projective bundle $P=\bb{P}(\sh{E})$ by the following \emph{universal property}, which is much closer to the geometric intuition than the constructions from §§2--3:
for every morphism $q:X\to Y$ and every invertible $\sh{O}_X$-module $\sh{L}$ that is a quotient of $\sh{E}\otimes_{\sh{O}_Y}\sh{O}_X$, there exists a unique $Y$-morphism $r:X\to P$ such that $\sh{L}=r^*(\sh{O}_P(1))$.

We will see later that we can similarly define, amongst other things, ``Grassmannian'' schemes.
\end{remark}

\begin{corollary}[4.2.6]
\label{II.4.2.6}
Suppose that every invertible $\sh{O}_Y$-module is trivial \sref[I]{I.2.4.8}.
Let $V$ be the group $\Hom_{\sh{O}_Y}(\sh{E},\sh{O}_Y)$, considered as a module over the ring $A=\Gamma(Y,\sh{O}_Y)$, and let $V^*$ be the subset of $V$ consisting of surjective homomorphisms.
Then the set of $Y$-sections of $\bb{P}(\sh{E})$ is canonically identified with $V^*/A^*$, where $A^*$ is the group of units of $A$.
\end{corollary}

\oldpage[II]{75}
In particular:
\begin{enumerate}
  \item The corollary \sref{II.4.2.6} applies whenever $Y$ is a \emph{local scheme} \sref[I]{I.2.4.8}.
    Let $Y$ be an arbitrary prescheme, $y$ a point of $Y$, and $Y'=\Spec(\kres(y))$;
    then the fibre $p^{-1}(y)$ of $\bb{P}(\sh{E})$ can, by \sref{II.4.1.3.1}, be identified with $\bb{P}(\sh{E}^y)$, where $\sh{E}^y = \sh{E}_y\otimes_{\sh{O}_y}\kres(y) = \sh{E}_y/\mathfrak{m}_y\sh{E}_y$ is considered as a vector space over $\kres(y)$.
    More generally, if $K$ is an extension of $\kres(y)$, then $p^{-1}(y)\otimes_{\kres(y)}K$ can be identified with $\bb{P}(\sh{E}^y\otimes_{\kres(y)}K)$.
    The corollary \sref{II.4.2.6} then shows that the set of \emph{geometric points of $\bb{P}(\sh{E})$ with values in the extension $K$ of $\kres(y)$} \sref[I]{I.3.4.5}, which we can also call the \emph{rational geometric fibre over $K$ of $\bb{P}(\sh{E})$ over the point $y$}, can be identified with the \emph{projective space} associated to the \emph{dual} of the $K$-vector space $\sh{E}^y\otimes_{\kres(y)}K$.
  \item Suppose that $Y$ is affine of ring $A$, and, further, that every invertible $\sh{O}_Y$-module is trivial;
    further, take $\sh{E}=\sh{O}_Y^n$;
    then, in \sref{II.4.2.6}, $V$ can be identified with $A^n$ \sref[I]{I.1.3.8}, and $V^*$ with the sets of systems $(f_i)_{1\leq i\leq n}$ of elements of $A$ that generate the ideal $A$;
    any two such systems define the same $Y$-section of $\bb{P}_Y^{n-1}=\bb{P}_A^{n-1}$, or, in other words, \emph{the same point of $\bb{P}_A^{n-1}$ with values in $A$}, if and only if one of them can be obtained from the other by multiplication by an invertible element of $A$.
\end{enumerate}

These properties justify the terminology ``projective bundle'' for $\bb{P}(\sh{E})$.
We note that the definitions that we will similarly obtain for ``projective space'' is in fact \emph{dual} to the classical definition;
this is imposed upon us by the necessity of being able to define $\bb{P}(\sh{E})$ for \emph{arbitrary} quasi-coherent $\sh{O}_Y$-modules $\sh{E}$, and not just locally free ones.

\begin{remark}[4.2.7]
\label{II.4.2.7}
We will see, in Chapter~V, that, if $Y$ is connected and locally Noetherian, and if $\sh{E}$ is locally free, then, letting $P=\bb{P}(\sh{E})$, every invertible $\sh{O}_P$-module is isomorphic to an $\sh{O}_P$-module of the form $\sh{L}'\otimes_{\sh{O}_Y}\sh{O}_P(m)$, with $\sh{L}'$ some invertible $\sh{O}_Y$-module, well defined up to isomorphism, and $m$ some well defined integer.
In other words, $\HH^1(P,\sh{O}_P^*)$ is isomorphic to $\bb{Z}\times\HH^1(Y,\sh{O}_Y^*)$ \sref[0]{0.5.4.7}.
We will also see (\sref[III]{III.2.1.14}, taking \sref[0]{0.5.4.10} into account) that $p_*(\sh{L}^{\otimes m})=0$ if $m<0$, and $p_*(\sh{L}^{\otimes m})$ is isomorphic to $\sh{L}'\otimes_{\sh{O}_Y}(\bb{S}_{\sh{O}_Y}(\sh{E}))_m$ if $m\geq0$.
If $\sh{F}$ is a quasi-coherent $\sh{O}_Y$-module, then every $Y$-morphism $\bb{P}(\sh{E})\to\bb{P}(\sh{F})$ is determined by the data of an invertible $\sh{O}_Y$-module, an integer $m\geq0$, and an $\sh{O}_Y$-homomorphism $\psi:\sh{F}\to\sh{L}'\otimes_{\sh{O}_Y}(\bb{S}_{\sh{O}_Y}(\sh{E}))_m$ such that the corresponding homomorphism $\psi^\sharp$ of $\sh{O}_{\bb{P}(\sh{F})}$-modules is surjective.
We will also see that, if the $Y$-morphism in question is an isomorphism, then $m=1$ and $\sh{F}$ is isomorphic to $\sh{E}\otimes_{\sh{O}_Y}\sh{L}'$ (the converse of \sref{II.4.1.4}).
This will allow us to determine the sheaf of germs of automorphisms of $\bb{P}(\sh{E})$ as the quotient of the sheaf of groups $\shAut(\sh{E})$ (which is locally isomorphic to $\GL(n,\sh{O}_Y)$ is $\sh{E}$ is of rank $n$) by $\sh{O}_Y^*$.
\end{remark}

\begin{env}[4.2.8]
\label{II.4.2.8}
Keeping the notation of \sref{II.4.2.1}, let $u:X'\to X$ be a morphism;
if the $Y$-morphism $r:X\to P$ corresponds to the homomorphism $\vphi:q^*(\sh{E})\to\sh{L}$, then the $Y$-morphism $r\circ u$ corresponds to $u^*(\vphi):u^*(q^*(\sh{E}))\to u^*(\sh{L})$, as follows immediately from the definitions.
\end{env}

\begin{env}[4.2.9]
\label{II.4.2.9}
Let $\sh{E}$ and $\sh{F}$ be quasi-coherent $\sh{O}_Y$-modules, $v:\sh{E}\to\sh{F}$ a surjective homomorphism, and $j=\bb{P}(v)$ the corresponding closed immersion $\bb{P}(\sh{F})\to\bb{P}(\sh{E})$ \sref{II.4.1.2}.
If the $Y$-morphism $r:X\to\bb{P}(\sh{F})$ corresponds to the homomorphism $\vphi:q^*(\sh{F})\to\sh{L}$, then the
\oldpage[II]{76}
$Y$-morphism $j\circ r$ corresponds to $q^*(\sh{E})\xrightarrow{q^*(v)}q^*(\sh{F})\xrightarrow{\vphi}\sh{L}$;
this again follows from the definition given in \sref{II.4.2.1}.
\end{env}

\begin{env}[4.2.10]
\label{II.4.2.10}
Let $\psi:Y'\to Y$ be a morphism, and let $\sh{E}'=\psi^*(\sh{E})$.
If the $Y$-morphism $r:X\to P$ corresponds to the homomorphism $\vphi:q^*(\sh{E})\to\sh{L}$, then the $Y'$-morphism
\[
  r_{(Y')}: X_{(Y')} \to P' = \bb{P}(\sh{E}')
\]
corresponds to $\vphi_{(Y')}:q_{(Y')}^*(\sh{E}') = q^*(\sh{E})\otimes_Y\sh{O}_{Y'} \to \sh{L}\otimes_Y\sh{O}_{Y'}$.
Indeed, by \sref{II.4.1.3.1}, we have the commutative diagram
\[
  \xymatrix{
    Y' \ar[d]
    & P'=P_{(Y')} \ar[l]_{p_{(Y')}} \ar[d]^{u}
    & X_{(Y')} \ar[l]_{r_{(Y')}} \ar[d]^{v}
  \\Y
    & P \ar[l]_{p}
    & X \ar[l]_{r}
  }
\]

From \sref{II.4.1.3.1}, we have
\[
  (r_{(Y')})^*(\sh{O}_{P'}(1)) = (r_{(Y')})^*(u^*(\sh{O}_P(1))) = v^*(r^*(\sh{O}_P(1))) = v^*(\sh{L}) = \sh{L}\otimes_Y\sh{O}_{Y'};
\]
we also know that $u^*(\alpha_1^\sharp)$ is exactly the canonical homomorphism $\alpha_1^\sharp:(p_{(Y')})^*(\sh{E}')\to\sh{O}_{P'}(1)$;
we can see this by explicitly calculating the canonical homomorphisms $\alpha_1^\sharp$ to $P$ and $P'$ as in \sref{II.4.1.6}.
Whence our claim.
\end{env}


\subsection{The Segre morphism}
\label{subsection:II.4.3}

\begin{env}[4.3.1]
\label{II.4.3.1}
Let $Y$ be a prescheme, and $\sh{E}$ and $\sh{F}$ quasi-coherent $\sh{O}_Y$-modules;
let $P_1=\bb{P}(\sh{E})$ and $P_2=\bb{P}(\sh{F})$, and denote the structure morphisms by $p_1:P_1\to Y$ and $p_2:P_2\to Y$.
Let $Q=P_1\times_Y P_2$, and let $q_1:Q\to P_1$ and $q_2:Q\to P_2$ be the canonical projections;
then the $\sh{O}_Q$-module $\sh{L}=\sh{O}_{P_1}(1)\otimes_Y \sh{O}_{P_2}(1) = q_1^*(\sh{O}_{P_1}(1))\otimes_{\sh{O}_Q}q_2^*(\sh{O}_{P_2}(1))$ is invertible, since it is the tensor product of of two invertible $\sh{O}_Q$-modules \sref[0]{0.5.4.4}.
Also, if $r=p_1\circ q_1=p_2\circ q_2$ is the structure morphism $Q\to Y$, then $r^*(\sh{E}\otimes_{\sh{O}_Y}\sh{F}) = q_1^*(p_1^*(\sh{E}))\otimes_{\sh{O}_Q}q_2^*(p_2^*(\sh{F}))$ \sref[0]{0.4.3.3};
the canonical surjective homomorphisms \sref{II.4.1.5.1} $p_1^*(\sh{E})\to\sh{O}_{P_1}(1)$ and $p_2^*(\sh{F})\to\sh{O}_{P_2}(1)$ thus give, by taking the tensor product, a canonical homomorphism
\[
\label{II.4.3.1.1}
  s: r^*(\sh{E}\otimes_{\sh{O}_Y}\sh{F}) \to \sh{L}
  \tag{4.3.1.1}
\]
which is evidently surjective;
from this we obtain \sref{II.4.2.2} a canonical morphism, called the \emph{Segre morphism}:
\[
\label{II.4.3.1.2}
  \varsigma: \bb{P}(\sh{E})\times_Y\bb{P}(\sh{F}) \to \bb{P}(\sh{E}\otimes_{\sh{O}_Y}\sh{F}).
  \tag{4.3.1.2}
\]

We can study the morphism $\varsigma$ more explicitly in the case where $Y=\Spec(A)$ is affine, and $\sh{E}=\widetilde{E}$ and $\sh{F}=\widetilde{F}$, where $E$ and $F$ are $A$-modules, whence $\sh{E}\otimes_{\sh{O}_Y}\sh{F}=(E\otimes_A F)^\sim$ \sref[I]{I.1.3.12};
let $R=\bb{S}_A(E)$, $S=\bb{S}_A(F)$, and $T=\bb{S}_A(E\otimes_A F)$;
let $f\in E$ and $g\in F$, and consider the affine open
\[
  D_+(f) \times_Y D_+(g) = \Spec(B)
\]
\oldpage[II]{77}
of $Q$, where $B=R_{(f)}\otimes_A S_{(g)}$;
the restriction of $\sh{L}$ to this affine open is $\widetilde{L}$, where
\[
  L = (R(1))_{(f)} \otimes_A (S(1))_{(g)}
\]
and the element $c=(f/1)\otimes(g/1)$ is a generator of $L$ considered as a free $B$-module \sref{II.2.5.7}.
The homomorphism \sref{II.4.3.1.1} corresponds to the homomorphism
\[
  (x\otimes y)\otimes b \mapsto b((x/1)\otimes(y/1))
\]
from $(E\otimes_A F)\otimes_A B$ to $L$.
With the notation of \sref{II.3.7.2}, we thus have that $v_1(x\otimes y)=(x/f)\otimes(y/g)$;
the restriction of $\varsigma$ to $D_+(f)\times_Y D_+(g)$ is a morphism from this affine scheme to $D_+(f\otimes g)$, corresponding to the ring homomorphism $\omega:T_{(f\otimes g)}\to R_{(f)}\otimes_A S_{(g)}$ defined by
\[
\label{II.4.3.1.3}
  \omega((x\otimes y)/(f\otimes g)) = (x/f)\otimes(y/g)
  \tag{4.3.1.3}
\]
for $x\in E$ and $y\in F$.
\end{env}

\begin{env}[4.3.2]
\label{II.4.3.2}
It follows from \sref{II.4.2.3} that we have a canonical isomorphism
\[
\label{II.4.3.2.1}
  \tau: \varsigma^*(\sh{O}_P(1)) \xrightarrow{\sim} \sh{O}_{P_1}(1)\otimes_Y\sh{O}_{P_2}(1)
  \tag{4.3.2.1}
\]
where we let $P=\bb{P}(\sh{E}\otimes_{\sh{O}_Y}\sh{F})$.
We will show that, for $x\in\Gamma(Y,\sh{E})$ and $y\in\Gamma(Y,\sh{F})$, we have
\[
\label{II.4.3.2.2}
  \tau(\alpha_1(x\otimes y)) = \alpha_1(x)\otimes\alpha_1(y).
  \tag{4.3.2.2}
\]

Indeed, we can restrict to the case where $Y$ is affine, and we then have, with the notation of \sref{II.4.3.1} and \sref{II.2.6.2}, that $\alpha_1^{f\otimes g}(x\otimes y)=(x\otimes y)/1$ in $(T(1))_{(f\otimes g)}$, that $\alpha_1^f(x)=x/1$ in $(R(1))_{(f)}$, and that $\alpha_1^g(y)=y/1$ in $(S(1))_{(g)}$.
The definition of $\tau$ given in \sref{II.4.2.3} and the calculation of $v_1$ done in \sref{II.4.3.1} then immediately prove the claim \sref{II.4.3.2.2}.
From this we obtain the equation
\[
\label{II.4.3.2.3}
  \varsigma^{-1}(P_{x\otimes y}) = (P_1)_x\times_Y(P_2)_y
  \tag{4.3.2.3}
\]
with the notation of \sref{II.3.1.4}.
Indeed, taking \sref{II.3.3.3} into account, the equation \sref{II.4.3.2.2} (by restricting to the affine case, with the help of \sref[I]{I.3.2.7} and \sref[I]{I.3.2.3}) leaves us only to prove the following lemma:
\begin{lemma}[4.3.2.4]
\label{II.4.3.2.4}
Let $B$ and $B'$ be $A$-algebras, and let $Y=\Spec(A)$, $Z=\Spec(B)$, and $Z'=\Spec(B')$;
then $D(t\otimes t')=D(t)\times_Y D(t')$ for any $t\in B$, $t'\in B'$.
\end{lemma}
\begin{proof}
Indeed, if $p:Z\times_Y Z'\to Z$ and $p':Z\times_Y Z'\to Z'$ are the canonical projections, then it follows from \sref[I]{I.1.2.2.2} that $p^{-1}(D(t))=D(t\otimes1)$ and $p'^{-1}(D(t'))=D(1\otimes t')$;
the conclusion follows from \sref[I]{I.3.2.7} and \sref[I]{I.1.1.9.1}, since $(t\otimes1)(1\otimes t')=t\otimes t'$.
\end{proof}
\end{env}

\begin{proposition}[4.3.3]
\label{II.4.3.3}
The Segre morphism is a closed immersion.
\end{proposition}

\begin{proof}
Since the question is local on $Y$, we can restrict to the case where $Y$ is affine.
With the notation of \sref{II.4.3.1} and \sref{II.4.3.1}, the $D_+(f\otimes g)$ then form a basis for the topology of $P$, since the $f\otimes g$ generate $T$ when $f$ runs over $E$ and $g$ runs over $F$.
By \sref{II.4.3.2.3}, we also know that $\varsigma^{-1}(D_+(f\otimes g))=D_+(f)\times_Y D_+(g)$.
It thus suffices \sref[I]{I.4.2.4} to prove that the restriction of $\varsigma$ to $D_+(f)\times_Y D_+(g)$ is a closed immersion into $D_+(f\otimes g)$.
But, with the same notation, the equation \sref{II.4.3.1.3} shows that $\omega$ is \emph{surjective}, which completes the proof.
\end{proof}

\begin{env}[4.3.4]
\label{II.4.3.4}
The Segre morphism is \emph{functorial} in $\sh{E}$ and $\sh{F}$, if we consider only
\oldpage[II]{78}
\emph{surjective} homomorphisms of quasi-coherent $\sh{O}_Y$-modules.
Indeed, we must then show that, if $\sh{E}\to\sh{E}'$ is a surjective $\sh{O}_Y$-homomorphism, then the diagram
\[
  \xymatrix{
    \bb{P}(\sh{E}')\times\bb{P}(\sh{F}) \ar[r]^{j\times1} \ar[d]_{\varsigma}
    & \bb{P}(\sh{E})\times\bb{P}(\sh{F}) \ar[d]^{\varsigma}
  \\\bb{P}(\sh{E}'\otimes\sh{F}) \ar[r]
    &\bb{P}(\sh{E}\otimes\sh{F})
  }
\]
commutes, where $j$ denotes the canonical closed immersion $\bb{P}(\sh{E}')\to\bb{P}(\sh{E})$.
Let $P'_1=\bb{P}(\sh{E}')$ and keep the notation from \sref{II.4.3.1};
then $j\times1$ is a closed immersion \sref[I]{I.4.3.1} and, up to isomorphism,
\[
  (j\times1)^*(\sh{O}_{P_1}(1)\otimes\sh{O}_{P_2}(1))
  = j^*(\sh{O}_{P_1}(1))\otimes\sh{O}_{P_2}(1)
  = \sh{O}_{P'_1}(1)\otimes\sh{O}_{P_2}(1)
\]
by \sref{II.4.1.2.1} and \sref[I]{I.9.1.5};
our claim then follows from \sref{II.4.2.8} and \sref{II.4.2.9}.
\end{env}

\begin{env}[4.3.5]
\label{II.4.3.5}
With the notation of \sref{II.4.3.1}, let $\psi:Y'\to Y$ be a morphism, and let $\sh{E}'=\psi^*(\sh{E})$ and $\sh{F}'=\psi^*(\sh{F})$;
then the Segre morphism $\bb{P}(\sh{E}')\times\bb{P}(\sh{F}')\to\bb{P}(\sh{E}'\otimes\sh{F}')$ can be identified with $\varsigma_{(Y')}$.
Indeed, keeping the notation of \sref{II.4.3.1}, let $P'_1=\bb{P}(\sh{E}')$ and $P_2=\bb{P}(\sh{F}')$;
we know \sref{II.4.1.3.1} that $P'_i$ can be identified with $(P_i)_{(Y')}$ ($i=1,2$), and so the structure morphism $P'_1\times P'_2\to Y'$ can be identified with $r_{(Y')}$.
Also $\sh{E}'\otimes\sh{F}'$ can be identified with $\psi^*(\sh{E}\otimes\sh{F})$, and so $\bb{P}(\sh{E}'\otimes\sh{F}')$ can be identified with $(\bb{P}(\sh{E}\otimes\sh{F}))_{(Y')}$ \sref{II.4.1.3.1}.
Finally, $\sh{O}_{P'_1}(1)\otimes_Y\sh{O}_{P'_2}(1)=\sh{L}'$ can be identified with $\sh{L}\otimes_Y\sh{O}_{Y'}$, by \sref{II.4.1.3.1} and \sref[I]{I.9.1.11}.
The canonical homomorphism $(r_{(Y)})^*(\sh{E}'\otimes\sh{F}')\to\sh{L}'$ can then be identified with $s_{(Y')}$, and our claim follows from \sref{II.4.2.10}.
\end{env}

\begin{remark}[4.3.6]
\label{II.4.3.6}
The prescheme given by the \emph{sum} of $\bb{P}(\sh{E})$ and $\bb{P}(\sh{F})$ is even canonically isomorphic to a \emph{closed subprescheme of $\bb{P}(\sh{E}\oplus\sh{F})$}.
Indeed, the surjective homomorphisms $\sh{E}\oplus\sh{F}\to\sh{E}$ and $\sh{E}\oplus\sh{F}\to\sh{F}$ correspond to closed immersions $\bb{P}(\sh{E})\to\bb{P}(\sh{E}\oplus\sh{F})$ and $\bb{P}(\sh{F})\to\bb{P}(\sh{E}\oplus\sh{F})$;
everything then reduces to showing that the underlying spaces of the closed subpreschemes of $\bb{P}(\sh{E}\oplus\sh{F})$ obtained in this way have empty intersection.
Since the question is local on $Y$, we can adopt the notation of \sref{II.4.3.1};
but $\bb{S}_n(E)$ and $\bb{S}_n(F)$ can be identified with submodules of $\bb{S}_n(E\oplus F)$ with intersection consisting only of $0$;
if $\mathfrak{p}$ is a graded prime ideal of $\bb{S}(E)$ such that $\mathfrak{p}\cap\bb{S}_n(E)\neq\bb{S}_n(E)$ for any $n\geq0$, then there exists a corresponding graded prime ideal of $\bb{S}(E\oplus F)$ whose intersection with $\bb{S}_n(E)$ is $\mathfrak{p}\cap\bb{S}_n(E)$, but who also \emph{contains} $\bb{S}_+(F)$, as we immediately see;
thus no point in $\Proj(\bb{S}(E))$ can have the same image in $\Proj(\bb{S}(E\oplus F))$ as any point in $\Proj(\bb{S}(F))$.
\end{remark}


\subsection{Immersions into projective bundles; very ample sheaves}
\label{subsection:II.4.4}

\begin{proposition}[4.4.1]
\label{II.4.4.1}
Let $Y$ be a quasi-compact scheme, or a prescheme whose underlying space is Noetherian, $q:X\to Y$ a morphism \emph{of finite type}, and $\sh{L}$ an invertible $\sh{O}_X$-module.
\begin{enumerate}
  \item[\rm{(i)}] Let $\sh{S}$ be a positively-graded quasi-coherent $\sh{O}_Y$-algebra, and $\psi:q^*(\sh{S})\to\bigoplus_{n\geq0}\sh{L}^{\otimes n}$ a homomorphism of graded algebras.
    For $r_{\sh{L},\psi}$ to be everywhere defined and an immersion, it is necessary and
\oldpage[II]{79}
    sufficient for there to exist an integer $n\geq0$ and a quasi-coherent sub-$\sh{O}_Y$-module \emph{of finite type} $\sh{E}$ of $\sh{S}_n$ such that the homomorphism $\psi'=\psi_n\circ q^*(j):q^*(\sh{E})\to\sh{L}^{\otimes n}=\sh{L}'$ (where $j$ is the injection $\sh{E}\to\sh{S}_n$) is surjective and such that the morphism $r_{\sh{L}',\vphi'}:X\to\bb{P}(\sh{E})$ is an immersion.
  \item[\rm{(ii)}] Let $\sh{F}$ be a quasi-coherent $\sh{O}_Y$-module, and $\vphi:q^*(\sh{F})\to\sh{L}$ a surjective homomorphism.
    For the morphism $r_{\sh{L},\vphi}$ to be an immersion $X\to\bb{P}(\sh{F})$, it is necessary and sufficient for there to exist a quasi-coherent sub-$\sh{O}_Y$-module \emph{of finite type} $\sh{E}$ of $\sh{F}$ such that the homomorphism $\vphi'=\vphi\circ q(j):q^*(\sh{E})\to\sh{L}$ (where $j$ is the canonical injection $\sh{E}\to\sh{F}$) is surjective and such that the morphism $r_{\sh{L},\vphi'}:X\to\bb{P}(\sh{E})$ is an immersion.
\end{enumerate}
\end{proposition}

\begin{proof}
\medskip\noindent
\begin{enumerate}
  \item[\rm{(i)}] The fact that $r_{\sh{L},\vphi}$ is everywhere defined and is an immersion is equivalent, by \sref{II.3.8.5}, to the existence of some $n\geq0$ and $\sh{E}$ such that, if $\sh{S}'$ is the subalgebra of $\sh{S}$ generated by $\sh{E}$, the homomorphism $q^*(\sh{E})\to\sh{L}^{\otimes n}$ is surjective and the morphism $r_{\sh{L},\psi'}:X\to\Proj(\sh{S}')$ is everywhere defined and is an immersion.
    We already have a canonical surjective homomorphism $\bb{S}(\sh{E})\to\sh{S}'$ to which there exists a corresponding closed immersion $\Proj(\sh{S}')\to\bb{P}(\sh{E})$ \sref{II.3.6.2};
    whence the conclusion.
  \item[\rm{(ii)}] Since $\sh{F}$ is the inductive limit of its quasi-coherent submodules of finite type $\sh{E}_\lambda$ \sref[I]{I.9.4.9}, $\bb{S}(\sh{F})$ is the inductive limit of the $\bb{S}(\sh{E}_\lambda)$;
    the conclusion then follows from \sref{II.3.8.4}, by observing that we can take all the $n_i$ in the proof of \sref{II.3.8.4} to be equal to $1$:
    indeed, supposing that $Y$ is affine, if $r=r_{\sh{L},\vphi}$ is an immersion, then $r(X)$ is a quasi-compact subspace of $\bb{P}(\sh{F})$ that we can cover by finitely many open subsets of $\bb{P}(\sh{F})$ of the form $D_+(f)$, with $f\in F$, such that $D_+(f)\cap r(X)$ is closed.
\end{enumerate}
\end{proof}

\begin{definition}[4.4.2]
\label{II.4.4.2}
Let $Y$ be a prescheme, and $q:X\to Y$ a morphism.
We say that an invertible $\sh{O}_X$-module $\sh{L}$ is \emph{very ample for $q$}, or \emph{relative to $q$} (or \emph{very ample for} (or \emph{relative to}) \emph{$Y$}, or simply \emph{very ample}, if $q$ is clear from the context) if there exists a quasi-coherent $\sh{O}_Y$-module $\sh{E}$ and a $Y$-immersion $i$ from $X$ to $P=\bb{P}(\sh{E})$ such that $\sh{L}$ is isomorphic to $i^*(\sh{O}_P(1))$.
\end{definition}

It is equivalent \sref{II.4.2.3} to say that there exists a quasi-coherent $\sh{O}_Y$-module $\sh{E}$ and a \emph{surjective} homomorphism $\vphi:q^*(\sh{E})\to\sh{L}$ such that $r_{\sh{L},\vphi}:X\to\bb{P}(\sh{E})$ is an \emph{immersion}.

We note that the existence of a very ample (for $Y$) $\sh{O}_X$-module implies that $q$ is \emph{separated} (\sref{II.3.1.3} and \sref[I]{I.5.5.1}[(i) and (ii)]).

\begin{corollary}[4.4.3]
\label{II.4.4.3}
Suppose that there exists a graded quasi-coherent $\sh{O}_Y$-algebra $\sh{S}$, generated by $\sh{S}_1$, and a $Y$-immersion $i:X\to P=\Proj(\sh{S})$ such that $\sh{L}$ is isomorphic to $i^*(\sh{O}_P(1))$;
then $\sh{L}$ is very ample relative to $q$.
\end{corollary}

\begin{proof}
If $\sh{F}=\sh{S}_1$, then the canonical homomorphism $\bb{S}(\sh{F})\to\sh{S}$ is surjective, and so, by compositing with the corresponding closed immersion $\Proj(\sh{S})\to\bb{P}(\sh{F})$ \sref{II.3.6.2} and the immersion $i$, we obtain an immersion $j:X\to\bb{P}(\sh{F})=P'$ such that $\sh{L}$ is isomorphic to $j^*(\sh{O}_{P'}(1))$ \sref{II.3.6.3}.
\end{proof}

\begin{proposition}[4.4.4]
\label{II.4.4.4}
Let $q:X\to Y$ be a quasi-compact morphism, and $\sh{L}$ an invertible $\sh{O}_X$-module.
Then the following properties are equivalent:
\begin{enumerate}
  \item[\rm{(a)}] $\sh{L}$ is very ample relative to $q$.
  \item[\rm{(b)}] $q_*(\sh{L})$ is quasi-coherent, the canonical homomorphism $\sigma:q^*(q_*(\sh{L}))\to\sh{L}$ is surjective, and the morphism $r_{\sh{L},\sigma}:X\to\bb{P}(q_*(\sh{L}))$ is an immersion.
\end{enumerate}
\end{proposition}

\begin{proof}
Since $q$ is quasi-compact, we know that $q_*(\sh{L})$ is quasi-coherent if $q$ is separated \sref[I]{I.9.2.2}.

\oldpage[II]{80}
We know \sref{II.3.4.7} that the existence of a surjective homomorphism $\vphi:q^*(\sh{E})\to\sh{L}$ (with $\sh{E}$ a quasi-coherent $\sh{O}_Y$-module) implies that $\sigma$ is surjective;
furthermore, given the factorisation $q^*(\sh{E})\to q^*(q_*(\sh{L}))\xrightarrow{\sigma}\sh{L}$ of $\vphi$, there is a canonically corresponding factorisation
\[
  q^*(\bb{S}(\sh{E})) \to q^*(\bb{S}(q_*(\sh{L}))) \to \bigoplus_{n\geq0}\sh{L}^{\otimes n}
\]
and so \sref{II.3.8.3} the hypothesis that $r_{\sh{L},\vphi}$ is an immersion implies that so too is $j=r_{\sh{L},\sigma}$;
furthermore \sref{II.4.2.4}, $\sh{L}$ is isomorphic to $j^*(\sh{O}_{P'}(1))$, where $P'=\bb{P}(q_*(\sh{L}))$.
We thus see that (a) and (b) are equivalent.
\end{proof}

\begin{corollary}[4.4.5]
\label{II.4.4.5}
Suppose that $q$ is quasi-compact.
For $\sh{L}$ to be very ample relative to $Y$, it is necessary and sufficient for there to exist an open cover $(U_\alpha)$ of $Y$ such that $\sh{L}|q^{-1}(U_\alpha)$ is very ample relative to $U_\alpha$ for every $\alpha$.
\end{corollary}

\begin{proof}
Indeed, condition (b) of \sref{II.4.4.4} is local on $Y$.
\end{proof}

\begin{proposition}[4.4.6]
\label{II.4.4.6}
Let $Y$ be a quasi-compact scheme, or a prescheme whose underlying space is Noetherian, $q:X\to Y$ a morphism \emph{of finite type}, and $\sh{L}$ an invertible $\sh{O}_X$-module.
Then conditions (a) and (b) of \sref{II.4.4.4} are equivalent to the following:
\begin{enumerate}
  \item[\rm{(a')}] There exists a quasi-coherent $\sh{O}_Y$-module $\sh{E}$ \emph{of finite type} and a surjective homomorphism $\vphi:q^*(\sh{E})\to\sh{L}$ such that $r_{\sh{L},\vphi}$ is an immersion.
  \item[\rm{(b')}] There exists a coherent sub-$\sh{O}_Y$-module $\sh{E}$ of $q_*(\sh{L})$ \emph{of finite type} that has the properties stated in condition~(a').
\end{enumerate}
\end{proposition}

\begin{proof}
It is clear that (a') or (b') imply (a);
also (a) implies (a'), by \sref{II.4.4.1}, and similarly (b) implies (b').
\end{proof}

\begin{corollary}[4.4.7]
\label{II.4.4.7}
Suppose that $Y$ is a quasi-compact scheme, or a Noetherian prescheme.
If $\sh{L}$ is very ample for $q$, then there exists a graded quasi-coherent $\sh{O}_Y$-algebra $\sh{S}$ such that $\sh{S}_1$ is of finite type and generates $\sh{S}$, and also a \emph{dominant open} $Y$-immersion $i:X\to P=\Proj(\sh{S})$ such that $\sh{L}$ is isomorphic to $i^*(\sh{O}_P(1))$.
\end{corollary}

\begin{proof}
Indeed, condition~(b) of \sref{II.4.4.6} is satisfied;
the structure morphism $p:\bb{P}(\sh{E})=P'\to Y$ is then separated and of finite type \sref{II.3.1.3}, and so $P'$ is a quasi-compact scheme (resp. a Noetherian prescheme) if $Y$ is a quasi-compact scheme (resp. a Noetherian prescheme).
Let $Z$ be the closure \sref[I]{I.9.5.11} of the subprescheme $X'$ of $P'$ associated to the immersion $j=r_{\sh{L},\vphi}$ from $X$ into $P'$;
it is clear that $j$ factors as a dominant open immersion $i:X\to Z$ followed by the canonical injection $Z\to P'$.
But $Z$ can be identified with a prescheme $\Proj(\sh{S})$, where $\sh{S}$ is a graded $\sh{O}_Y$-algebra equal to the quotient of $\bb{S}(\sh{E})$ by a graded quasi-coherent sheaf of ideals \sref{II.3.6.2}, and it is clear that $\sh{S}_1$ is of finite type and generates $\sh{S}$;
furthermore, $\sh{O}_Z(1)$ is the inverse image of $\sh{O}_{P'}(1)$ by the canonical injection \sref{II.3.6.3}, and so $\sh{L}=i^*(\sh{O}_Z(1))$.
\end{proof}

\begin{proposition}[4.4.8]
\label{II.4.4.8}
Let $q:X\to Y$ be a morphism, $\sh{L}$ a very ample (relative to $q$) $\sh{O}_X$-module, and $\sh{L}'$ an invertible $\sh{O}_X$-module, such that there exists a quasi-coherent $\sh{O}_Y$-module $\sh{E}'$ and a surjective homomorphism $q^*(\sh{E}')\to\sh{L}'$.
Then $\sh{L}\otimes_{\sh{O}_X}\sh{L}'$ is very ample relative to $q$.
\end{proposition}

\begin{proof}
The hypothesis implies the existence of a $Y$-morphism $r':X\to P'=\bb{P}(\sh{E}')$ such that $\sh{L}'=r'^*(\sh{O}_{P'}(1))$ \sref{II.4.2.1}.
There is, by hypothesis, a quasi-coherent $\sh{O}_Y$-module $\sh{E}$ and a
\oldpage[II]{81}
$Y$-immersion $r:X\to P=\bb{P}(\sh{E})$ such that $\sh{L}=r^*(\sh{O}_P(1))$.
Let $Q=\bb{P}(\sh{E}\otimes\sh{E}')$, and consider the Segre morphism $\varsigma:P\times_Y P'\to Q$ \sref{II.4.3.1}.
Since $r$ is an immersion, so too is $(r,r')_Y:X\to P\times_Y P'$ \sref[I]{I.5.3.14};
but since $\varsigma$ is an immersion \sref{II.4.3.3}, so too is $r'':X\xrightarrow{(r,r')}P\times_Y P'\xrightarrow{\varsigma}Q$.
But also \sref{II.4.3.2.1} $\varsigma(\sh{O}_Q(1))$ is isomorphic to $\sh{O}_P(1)\otimes_Y\sh{O}_{P'}(1)$, and so \sref[I]{I.9.1.4} $r''^*(\sh{O}_Q(1))$ is isomorphic to $\sh{L}\otimes\sh{L}'$, which proves the proposition.
\end{proof}

\begin{corollary}[4.4.9]
\label{II.4.4.9}
Let $q:X\to Y$ be a morphism.
\begin{enumerate}
  \item Let $\sh{L}$ be an invertible $\sh{O}_X$-module, and $\sh{K}$ an invertible $\sh{O}_Y$-module.
    For $\sh{L}$ to be very ample relative to $q$, it is necessary and sufficient for $\sh{L}\otimes q^*(\sh{K})$ to be so.
  \item If $\sh{L}$ and $\sh{L}'$ are very ample (relative to $q$) $\sh{O}_X$-modules, then so too is $\sh{L}\otimes\sh{L}'$;
    in particular, $\sh{L}^{\otimes n}$ is very ample relative to $q$ for all $n>0$.
\end{enumerate}
\end{corollary}

\begin{proof}
Claim~(ii) is an immediate consequence of \sref{II.4.4.8}, as well as the necessity of condition~(i);
conversely, if $\sh{L}\otimes q^*(\sh{K})$ is very ample, then so too is $(\sh{L}\otimes q^*(\sh{K}))\otimes q^*(\sh{K}^{-1})$, by the above, and the latter $\sh{O}_X$-module is isomorphic to $\sh{L}$ (\sref[0]{0.5.4.3} and \sref[0]{0.5.4.5}).
\end{proof}

\begin{proposition}[4.4.10]
\label{II.4.4.10}
\medskip\noindent
\begin{enumerate}
  \item[\rm{(i)}] For every prescheme $Y$, every invertible $\sh{O}_Y$-module $\sh{L}$ is very ample relative to the identity morphism $1_Y$.
  \item[\rm{(i \emph{bis})}] Let $f:X\to Y$ be a morphism, and $j:X'\to X$ an immersion.
    If $\sh{L}$ is a very ample (relative to $f$) $\sh{O}_X$-module, then $j^*(\sh{L})$ is very ample relative to $f\circ j$.
  \item[\rm{(ii)}] Let $Z$ be a quasi-compact prescheme, $f:X\to Y$ a morphism of finite type, $g:Y\to Z$ a quasi-compact morphism, $\sh{L}$ a very ample (relative to $f$) $\sh{O}_X$-module, and $\sh{K}$ a very ample (relative to $g$) $\sh{O}_Y$-module.
    Then there exists some integer $n_0>0$ such that $\sh{L}\otimes f^*(\sh{K}^{\otimes n})$ is very ample relative to $g\circ f$ for all $n\geq n_0$.
  \item[\rm{(iii)}] Let $f:X\to Y$ and $g:Y'\to Y$ be morphisms, and let $X'=X_{(Y')}$.
    If $\sh{L}$ is a very ample (relative to $f$) $\sh{O}_X$-module, then $\sh{L}'=\sh{L}\otimes_Y\sh{O}_{Y'}$ is a very ample (relative to $f_{(Y')}$) $\sh{O}_X$-module.
  \item[\rm{(iv)}] Let $f_i:X_i\to Y_i$ ($i=1,2$) be $S$-morphism.
    If $\sh{L}_i$ is a very ample (relative to $f_i$) $\sh{O}_{X_i}$-module ($i=1,2$), then $\sh{L}_1\otimes_S\sh{L}_2$ is very ample relative to $f_1\times_S f_2$.
  \item[\rm{(v)}] Let $f:X\to Y$ and $g:Y\to Z$ be morphisms.
    If an $\sh{O}_X$-module $\sh{L}$ is very ample relative to $g\circ f$, then it is also very ample relative to $f$.
  \item[\rm{(vi)}] Let $f:X\to Y$ be a morphism, and $j$ the canonical injection $X_\red\to X$.
    If an $\sh{O}_X$-module $\sh{L}$ is very ample relative to $f$, then $j^*(\sh{L})$ is very ample relative to $f_\red$.
\end{enumerate}
\end{proposition}

\begin{proof}
Property~(i~\emph{bis}) follows immediately from the definition \sref{II.4.4.2}, and it is immediate that (vi) follows formally from (i~\emph{bis}) and (v), by an argument copied from the proof of \sref[I]{I.5.5.12}, which we leave to the reader.
To prove (v), we consider, as in \sref[I]{I.5.5.12}, the factorisation $X\xrightarrow{\Gamma_f}X\times_Z Y\xrightarrow{p_2}Y$, where $p_2=(g\circ f)\times1_Y$.
It follows from the hypothesis and from (i) and (iv) that $\sh{L}\otimes_{\sh{O}_Z}\sh{O}_Y$ is very ample for $p_2$;
but also $\sh{L}=\Gamma_f^*(\sh{L}\otimes_{\sh{O}_Z}\sh{O}_Y)$ \sref[I]{I.9.1.4}, and $\Gamma_f$ is an immersion \sref[I]{I.5.3.11};
we can thus apply (i~\emph{bis}).

\oldpage[II]{82}
To prove (i), we apply the definition \sref{II.4.4.2} with $\sh{E}=\sh{L}$, and note that then $\bb{P}(\sh{E})$ can be identified with $Y$ \sref{II.4.1.4}.

Now we prove (iii).
There exists a quasi-coherent $\sh{O}_Y$-module $\sh{E}$ and a $Y$-immersion $i:X\to\bb{P}(\sh{E})=P$ such that $\sh{L}=i^*(\sh{O}_P(1))$;
if we let $\sh{E}'=g^*(\sh{E})$, then $\sh{E}'$ is a quasi-coherent $\sh{O}_{Y'}$-module, and we have that $P'=\bb{P}(\sh{E}')=P_{(Y')}$ \sref{II.4.1.3.1}, that $i_{(Y')}$ is an immersion from $X_{(Y')}$ into $P'$ \sref[I]{I.4.3.2}, and that $\sh{L}'$ is isomorphic to $(i_{(Y')})^*(\sh{O}_{P'}(1))$ \sref{II.4.2.10}.

To prove (iv), note that there is, by hypothesis, a $Y_i$-immersion $r_i:X_i\to P_i=\bb{P}(\sh{E}_i)$, where $\sh{E}_i$ is a quasi-coherent $\sh{O}_{Y_i}$-module, and $\sh{L}_i=r_i^*(\sh{O}_{P_i}(1))$ ($i=1,2$);
$r_1\times_S r_2$ is an $S$-immersion of $X_1\times_S X_2$ into $P_1\times_S P_2$ \sref[I]{I.4.3.1}, and the inverse image of $\sh{O}_{P_1}(1)\otimes_S\sh{O}_{P_2}(1)$ under this immersion is $\sh{L}_1\otimes_S\sh{L}_2$.
Now let $T=Y_1\times_S Y_2$, and let $p_1$ and $p_2$ be the projections from $T$ to $Y_1$ and $Y_2$, respectively.
If we let $P'_i=\bb{P}(p_i^*(\sh{E}_i))$ ($i=1,2$), then $P'_i=P_i\times_{Y_i}T$, by \sref{II.4.1.3.1}, and so
\[
  P'_1\times_T P'_2
  = (P_1\times_{Y_1}T)\times_T(P_2\times_{Y_2}T)
  = P_1\times_{Y_1}(T\times_{Y_2}P_2)
  = P_1\times_{Y_1}(Y_1\times_S P_2)
  = P_1\times_S P_2
\]
up to canonical isomorphism.
Similarly, $\sh{O}_{P'_i}(1)=\sh{O}_{P_i}(1)\otimes_{Y_i}\sh{O}_T$ \sref{II.4.1.3.2}, and an analogous calculation (based in particular on \sref[I]{I.9.1.9.1} and \sref[I]{I.9.1.2}) shows that, in the above identification, $\sh{O}_{P'_1}(1)\otimes_T\sh{O}_{P'_2}(1)$ can be identified with $\sh{O}_{P_1}\otimes_S\sh{O}_{P_2}(1)$.
We can thus consider $r_1\times_S r_2$ as a $T$-immersion from $X_1\times_S X_2$ into $P'_1\times_T P'_2$, with the inverse image of $\sh{O}_{P'_1}(1)\otimes_T\sh{O}_{P'_2}(1)$ under this immersion being $\sh{L}_1\otimes_S\sh{L}_2$.
We then finish the argument as in \sref{II.4.4.8} by using the Segre morphism.

It remains only to prove (ii).
We can first of all restrict to the case where $Z$ is an affine scheme, since, in general, there exists a finite cover $(U_i)$ of $Z$ by affine opens;
if the proposition were proven for $\sh{K}|g^{-1}(U_i)$, $\sh{L}|f^{-1}(g^{-1}(U))$, and an integer $n_i$, then it would suffice to take $n_0$ to be the largest of the $n_i$ to prove the proposition for $\sh{K}$ and $\sh{L}$ \sref{II.4.4.5}.
The hypothesis implies that $f$ and $g$ are separated morphisms, and so $X$ and $Y$ are quasi-compact \emph{schemes}.

There is an immersion $r:X\to P=\bb{P}(\sh{E})$, where $\sh{E}$ is a quasi-coherent $\sh{O}_Y$-module \emph{of finite type}, and $\sh{L}=r^*(\sh{O}_P(1))$, by \sref{II.4.4.6}.
We will see that there exists a very ample (relative to the composed morphism $P\xrightarrow{h}Y\xrightarrow{g}Z$) $\sh{O}_P$-module $\sh{M}$ such that $\sh{O}_P(1)$ is isomorphic to $\sh{M}\otimes_Y\sh{K}^{\otimes(-m)}$ for some integer $m$.
For $n\geq m+1$, $\sh{O}_P(1)\otimes_Y\sh{K}^{\otimes n}$ will then be very ample for $Z$, by hypothesis and by (iv) applied to the morphisms $h:P\to Y$ and $1_Y$;
since $r$ is an immersion and $\sh{L}\otimes f^*(\sh{K}^{\otimes n}) = r^*(\sh{O}_P(1)\otimes_Y\sh{K}^{\otimes n})$, the conclusion will then follow from (i~\emph{bis}).
To prove our claim concerning $\sh{O}_P(1)$, we will use the following lemma:

  \begin{lemma}[4.4.10.1]
  \label{II.4.4.10.1}
  Let $Z$ be a quasi-compact scheme, or a prescheme whose underlying space is Noetherian, and let $g:Y\to Z$ be a quasi-compact morphism, $\sh{K}$ a very ample (with respect to $g$) invertible $\sh{O}_Y$-module, and $\sh{E}$ a quasi-coherent $\sh{O}_Y$-module of finite type.
  Then there exists an integer $m_0$ such that, for all $m\geq m_0$, $\sh{E}$ is isomorphic to a quotient of an $\sh{O}_Y$-module of the form $g^*(\sh{F})\otimes\sh{K}^{\otimes(-m)}$, where $\sh{F}$ is a quasi-coherent $\sh{O}_Z$-module of finite type (depending on $m$).
  \end{lemma}

This lemma will be proven in \sref{II.4.5.10.1};
the reader can verify that \sref{II.4.4.10} is not used anywhere in \sref{subsection:II.4.5}.

\oldpage[II]{83}
Assuming this lemma, there exists a closed immersion $j_1$ from $P$ to
\[
  P_1 = \bb{P}(g^*(\sh{F})\otimes\sh{K}^{\otimes(-m)})
\]
such that $\sh{O}_P(1)$ is isomorphic to $j_1^*(\sh{O}_{P_1}(1))$ \sref{II.4.1.2}.
Now, there exists an isomorphism from $P_1$ to $P_2=\bb{P}(g^*(\sh{F}))$, sending $\sh{O}_{P_2}(1)\otimes_Y\sh{K}^{\otimes(-m)}$ to $\sh{O}_{P_1}(1)$ \sref{II.4.1.4};
we thus have a closed immersion $j_2:P\to P_2$ such that $\sh{O}_P(1)$ is isomorphic to $j_2^*(\sh{O}_{P_2}(1))\otimes_Y\sh{K}^{\otimes(-m)}$.
Finally, $P_2$ can be identified with $P_3\times_Z Y$, where $P_3=\bb{P}(\sh{F})$, and $\sh{O}_{P_2}(1)$ with $\sh{O}_{P_3}(1)\otimes_Z\sh{O}_Y$ \sref{II.4.1.3}.
By definition, $\sh{O}_{P_3}(1)$ is very ample for $Z$;
since so too is $\sh{K}$, we conclude, from (iv), that $\sh{O}_{P_2}(1)\otimes_Y\sh{K}$ is very ample for $Z$;
so too is $\sh{M}=j_2^*(\sh{O}_{P_2}(1)\otimes_Y\sh{K})$ by (i~\emph{bis}), and $\sh{O}_P(1)$ is isomorphic to $\sh{M}\otimes_Y\sh{K}^{\otimes(-m-1)}$, which finishes the proof.
\end{proof}

\begin{proposition}[4.4.11]
\label{II.4.4.11}
Let $f:X\to Y$ and $f':X'\to Y$ be morphisms, $X''$ the sum prescheme $X\sqcup X'$, and $f''$ the morphism $X''\to Y$ that agrees with $f$ (resp. $f'$) on $X$ (resp. $X'$).
Let $\sh{L}$ (resp. $\sh{L}'$) be an invertible $\sh{O}_X$-module (resp. invertible $\sh{O}_{X'}$-module), and let $\sh{L}''$ be the invertible $\sh{O}_{X''}$-module that agrees with $\sh{L}$ (resp. $\sh{L}'$) on $X$ (resp. $X'$).
For $\sh{L}''$ to be very ample relative to $f''$, it is necessary and sufficient for $\sh{L}$ to be very ample relative to $f$ and for $\sh{L}'$ to be very ample relative to $f'$.
\end{proposition}

\begin{proof}
We can immediately restrict to the case where $Y$ is affine.
If $\sh{L}''$ is very ample then so too are $\sh{L}$ and $\sh{L}'$, by \sref{II.4.4.10}[(i~\emph{bis})].
Conversely, if $\sh{L}$ and $\sh{L}'$ are very ample, then it follows immediately from the definition \sref{II.4.4.2} and from \sref{II.4.3.6} that $\sh{L}''$ is very ample.
\end{proof}


\subsection{Ample sheaves}
\label{subsection:II.4.5}

\begin{env}[4.5.1]
\label{II.4.5.1}
Given a prescheme $X$ and an invertible $\sh{O}_X$-module $\sh{L}$, we define, for every $\sh{O}_X$-module $\sh{F}$ (when there will be no confusion possible over $\sh{L}$) $\sh{F}(n)=\sh{F}\otimes_{\sh{O}_X}\sh{L}^{\otimes n}$ ($n\in\bb{Z}$);
we also define $S=\bigoplus_{n\geq0}\Gamma(X,\sh{L}^{\otimes n})$ (a graded subring of the ring $\Gamma_\bullet(\sh{L})$ defined in \sref[0]{0.5.4.6}).
If we consider $X$ as a $\bb{Z}$-prescheme, and we denote by $p$ the structure morphism $X\to\Spec(\bb{Z})$, then there is a bijective correspondence between homomorphisms $p^*(\widetilde{S})\to\bigoplus_{n\geq0}\sh{L}^{\otimes n}$ of graded $\sh{O}_X$-algebras and endomorphisms of the graded ring $S$ \sref[I]{I.2.2.5};
the homomorphism $\varepsilon:p^*(\widetilde{S})\to\bigoplus_{n\geq0}\sh{L}^{\otimes n}$ that then corresponds to the \emph{identity} automorphism of $S$ is said to be \emph{canonical}.
There is a corresponding \sref{II.3.7.1} morphism $G(\varepsilon)\to\Proj(S)$ that is also said to be \emph{canonical}.
\end{env}

\begin{theorem}[4.5.2]
\label{II.4.5.2}
Let $X$ be a quasi-compact scheme or a prescheme whose underlying space is Noetherian, $\sh{L}$ an invertible $\sh{O}_X$-module, and $S$ the graded ring $\bigoplus_{n\geq0}\Gamma(X,\sh{L}^{\otimes n})$.
Then the following conditions are equivalent:
\begin{enumerate}
  \item[\rm{(a)}] When $f$ runs over the set of homogeneous elements of $S_+$, the $X_f$ form a base of the topology of $X$.
  \item[\rm{(a')}] When $f$ runs over the set of homogeneous elements of $S_+$, the $X_f$ that are affine form a cover of $X$.
  \item[\rm{(b)}] The canonical morphism $G(\varepsilon)\to\Proj(S)$ \sref{II.4.5.1} is everywhere defined and is a dominant open immersion.
\oldpage[II]{84}
  \item[\rm{(b')}] The canonical morphism $G(\varepsilon)\to\Proj(S)$ is everywhere defined and is a homeomorphism from the underlying space of $X$ to a subspace of $\Proj(S)$.
  \item[\rm{(c)}] For every quasi-coherent $\sh{O}_X$-module $\sh{F}$, if we denote by $\sh{F}_n$ the sub-$\sh{O}_X$-module of $\sh{F}(n)$ generated by the sections of $\sh{F}(n)$ over $X$, then $\sh{F}$ is the sum of the sub-$\sh{O}_X$-modules $\sh{F}_n(-n)$ over the integers $n>0$.
  \item[\rm{(c')}] Property~\rm{(c)} holds for every quasi-coherent sheaf of ideals of $\sh{O}_X$.
\end{enumerate}

Furthermore, if $(f_\alpha)$ is a family of homogeneous elements of $S_+$ such that the $X_{f_\alpha}$ are affine, then the restriction to $\bigcup_\alpha X_{f_\alpha}$ of the canonical morphism $X\to\Proj(S)$ is an isomorphism from $\bigcup_\alpha X_{f_\alpha}$ to $\bigcup_\alpha(\Proj(S))_{f_\alpha}$.
\end{theorem}

\begin{proof}
  It is clear that (b) implies (b'), and (b') implies (a) by \sref{II.3.7.3.1} (taking into account the fact that $\varepsilon^\flat$ is the identity).
  Condition~(a) implies (a'), since every $x\in X$ has an affine neighbourhood $U$ such that $\sh{L}|U$ is isomorphic to $\sh{O}_X|U$;
  if $f\in\Gamma(X,\sh{L}^{\otimes n})$ is such that $x\in X_f\subset U$, then $X_f$ is also the set of $x'\in U$ such that $(f|U)(x')\neq0$, and it is thus an affine open subset \sref[I]{I.1.3.6}.
  To prove that (a') implies (b), it suffices to prove the last claim of the theorem, and to further prove that, if $X=\bigcup_\alpha X_{f_\alpha}$, then condition~(iv) of \sref{II.3.8.2} is satisfied.
  This latter point follows immediately from \sref[I]{I.9.3.1}[(i)].
  As for the last claim of \sref{II.4.5.2}, since $X_{f_\alpha}$ is the inverse image of $(\Proj(S))_{f_\alpha}$ under $G(\varepsilon)\to\Proj(S)$, it suffices to apply \sref[I]{I.9.3.2}.
  Thus (a), (a'), (b), and (b') are all equivalent.

  To show that (a') implies (c), note that, if $X_f$ is affine (with $f\in S_k$), then $\sh{F}|X_f$ is generated by its sections over $X_f$ \sref[I]{I.1.3.9};
  on the other hand \sref[I]{I.9.3.1}[(ii)], such a section $s$ is of the form $(t|X_f)\otimes(f|X_f)^{-m}$, where $t\in\Gamma(X,\sh{F}(km))$;
  by definition, $t$ is also a section of $\sh{F}_{km}$, so $s$ is indeed a section of $\sh{F}_{km}(-km)$ over $X_f$, which proves (c).
  It is clear that (c) implies (c'), so it remains only to show that (c') implies (a).
  But let $U$ be an open neighbourhood of $x\in X$, and let $\sh{J}$ be a quasi-coherent sheaf of ideals of $\sh{O}_X$ defining a closed subprescheme of $X$ that has $X\setmin U$ as its underlying space \sref[I]{I.5.2.1}.
  Hypothesis~(c') implies that there exists an integer $n>0$ and a section $f$ of $\sh{J}(n)$ over $X$ such that $f(x)\neq0$.
  But we clearly have $f\in S_n$, and $x\in X_f\subset U$, which proves (a).
\end{proof}

When $X$ is a prescheme whose underlying space is Noetherian, the equivalent conditions of \sref{II.4.5.2} imply that $X$ is a \emph{scheme}, since it is isomorphic to a subprescheme of the scheme $S=\Proj(A)$, by \sref{II.4.5.2}[(b)].

\begin{definition}[4.5.3]
\label{II.4.5.3}
We say that an invertible $\sh{O}_X$-module $\sh{L}$ is \emph{ample} if $X$ is a quasi-compact scheme and if the equivalent conditions of \sref{II.4.5.2} are satisfied.
\end{definition}

It evidently follows from criterion~(a) of \sref{II.4.5.2} that, if $\sh{L}$ is an ample $\sh{O}_X$-module, then, for every open subset $U$ of $X$, $\sh{L}|U$ is an ample $(\sh{O}_X|U)$-module.

It follows from the proof of \sref{II.4.5.2} that the \emph{affine} $X_f$ form a base of the topology of $X$.
Furthermore:

\begin{corollary}[4.5.4]
\label{II.4.5.4}
Let $\sh{L}$ be an ample $\sh{O}_X$-module.
For every finite subspace $Z$ of $X$ and every neighbourhood $U$ of $Z$, there exists an integer $n$ and some $f\in\Gamma(X,\sh{L}^{\otimes n})$ such that $X_f$ is an affine neighbourhood of $Z$ contained in $U$.
\end{corollary}

\begin{proof}
\oldpage[II]{85}
By \sref{II.4.5.2}[(b)], it suffices to prove that, for every finite subset $Z'$ of $\Proj(S)$ and every open neighbourhood $U$ of $Z'$, there exists a homogeneous element $f\in S_+$ such that $Z\subset(\Proj(S))_f\subset U$ \sref{II.2.4.1}.
But, by definition, the closed set $Y$, complement of $U$ in $\Proj(S)$, is of the form $V_+(\mathfrak{I})$, where $\mathfrak{I}$ is a graded ideal of $S$ that does not contain $S_+$ \sref{II.2.3.2};
also, the points of $Z'$ are, by definition, graded ideals $\mathfrak{p}_i$ of $S_+$ that do not contain $\sh{J}$ \sref{II.2.3.1}.
There thus exists an element $f\in\mathfrak{I}$ that does not belong to any of the $\mathfrak{p}_i$ (Bourbaki, \emph{Alg. comm.}, chap.~II, \S1, no.~1, prop.~2), and, since the $\mathfrak{p}_i$ are graded, the argument made \emph{loc. cit.} shows that we can even take $f$ to be homogeneous;
this element then satisfies the claim.
\end{proof}

\begin{proposition}[4.5.5]
\label{II.4.5.5}
Suppose that $X$ is a quasi-compact scheme or a prescheme whose underlying space is Noetherian.
Then conditions~(a) to (c') of \sref{II.4.5.2} are equivalent to the following:
\begin{enumerate}
  \item[\rm{(d)}] For every quasi-coherent $\sh{O}_X$-module $\sh{F}$ of finite type, there exists an integer $n_0$ such that, for all $n\geq n_0$, $\sh{F}(n)$ is generated by its sections over $X$.
  \item[\rm{(d')}] For every quasi-coherent $\sh{O}_X$-module $\sh{F}$ of finite type, there exist integers $n>0$ and $k>0$ such that $\sh{F}$ is isomorphic to a quotient of the $\sh{O}_X$-module $\sh{L}^{\otimes(-n)}\otimes\sh{O}_X^k$.
  \item[\rm{(d'')}] Property~(d') holds for every quasi-coherent sheaf of ideals of $\sh{O}_X$ of finite type.
\end{enumerate}
\end{proposition}

\begin{proof}
Since $X$ is quasi-compact, if a quasi-coherent $\sh{O}_X$-module $\sh{F}$ of finite type is such that $\sh{F}(n)$ (which is of finite type) is generated by its sections over $X$, then $\sh{F}(n)$ is generated by a \emph{finite} number of these sections \sref[0]{0.5.2.3}, and so (d) implies (d'), and it is clear that (d') implies (d'').
Since every quasi-coherent $\sh{O}_X$-module $\sh{G}$ is the inductive limits of its sub-$\sh{O}_X$-modules of finite type \sref[I]{I.9.4.9}, to satisfy condition~(c') of \sref{II.4.5.2}, it suffices to do so for a quasi-coherent sheaf of ideals of $\sh{O}_X$ that is of finite type, and (d'') thus implies (c').
It remains only to show that, if $\sh{L}$ is ample, then property~(d) is satisfied.
Consider a finite cover of $X$ by $X_{f_i}$ ($f_i\in S_{n_i}$), that we can assume to be affine;
by replacing the $f_i$ with suitable powers (which does not alter the $X_{f_i}$), we can assume that all the $n_i$ are equal to one single integer $m$.
The sheaf $\sh{F}|X_{f_i}$, being of finite type, by hypothesis, is generated by a finite number of its sections $h_{ij}$ over $X_{f_i}$ \sref[I]{I.1.3.13};
so there exists an integer $k_0$ such that the section $h_{ij}\otimes f_i^{\otimes k_0}$ extends to a section of $\sh{F}(k_0m)$ over $X$ for every pair $(i,j)$ \sref[I]{I.9.3.1}.
\emph{A fortiori}, the $h_{ij}\otimes f_i^{\otimes k_0}$ extend to sections of $\sh{F}(km)$ over $X$ for every $k\geq k_0$, and, for these values of $k$, $\sh{F}(km)$ is thus generated by its sections over $X$.
For every $p$ such that $0<p<m$, $\sh{F}(p)$ is also of finite type, and so there exists an integer $k_p$ such that $\sh{F}(p)(km)=\sh{F}(p+km)$ is generated by its sections over $X$ for all $k\geq k_p$.
Taking $n_0$ to be the largest of the $k_pm$, we thus conclude that $\sh{F}(n)$ is generated by its sections over $X$ for all $n\geq n_0$, since such an $n$ is of the form $n=km+p$, with $k\geq k_p$ and $0\leq p<m$.
\end{proof}

\begin{proposition}[4.5.6]
\label{II.4.5.6}
  Let $X$ be a quasi-compact scheme, and $\sh{L}$ an invertible $\sh{O}_X$-module.
  \begin{enumerate}
    \item[\rm{(i)}] Let $n>0$ be an integer. For $\sh{L}$ to be ample, it is necessary and sufficient for $\sh{L}^{\otimes n}$ to be ample.
    \item[\rm{(ii)}] Let $\sh{L}'$ be an invertible $\sh{O}_X$-module such that, for all $x\in X$, there exists an integer $n>0$
\oldpage[II]{86}
      and a section $s'$ of $\sh{L}'^{\otimes n}$ over $X$ such that $s'(x)\neq0$.
      Then, if $\sh{L}$ is ample, so too is $\sh{L}\otimes\sh{L}'$.
  \end{enumerate}
\end{proposition}

\begin{proof}
  Property~(i) is an evident consequence of criterion~(a) of \sref{II.4.5.2}, since $X_{f^{\otimes n}}=X_f$.
  On the other hand, if $\sh{L}$ is ample, then, for every $x\in X$ and every neighbourhood $U$ of $x$, there exists some $m>0$ and $f\in\Gamma(X,\sh{L}^{\otimes m})$ such that $x\in X_f\subset U$ \sref{II.4.5.2}[(a)];
  if $f'\in\Gamma(X,\sh{L}'^{\otimes n})$ is such that $f'(x)\neq0$, then $s(x)\neq0$ for $s=f^{\otimes n}\otimes f'^{\otimes m}\in\Gamma(X,(\sh{L}\otimes\sh{L}')^{\otimes mn})$, and so $x\in X_s\subset X_f\subset U$, which proves that $\sh{L}\otimes\sh{L}'$ is ample \sref{II.4.5.2}[(a)].
\end{proof}

\begin{corollary}[4.5.7]
\label{II.4.5.7}
The tensor product of two ample $\sh{O}_X$-modules is ample.
\end{corollary}

\begin{corollary}[4.5.8]
\label{II.4.5.8}
Let $\sh{L}$ be an ample $\sh{O}_X$-module, and $\sh{L}'$ an invertible $\sh{O}_X$-module;
then there exists an integer $n_0>0$ such that $\sh{L}^{\otimes n}\otimes\sh{L}'$ is ample and generated by its sections over $X$ for $n\geq n_0$.
\end{corollary}

\begin{proof}
It follows from \sref{II.4.5.5} that there exists an integer $m_0$ such that $\sh{L}^{\otimes m}\otimes\sh{L}'$ is generated by its sections over $X$ for all $m\geq m_0$;
by \sref{II.4.5.6}, we can then take $n_0=m_0+1$.
\end{proof}

\begin{remark}[4.5.9]
\label{II.4.5.9}
Let $P=\HH^1(X,\sh{O}_X^\times)$ be the group of classes of invertible $\sh{O}_X$-modules \sref[0]{0.5.4.7}, and let $P^+$ be the subset of $P$ consisting of classes of ample sheaves.
Suppose that $P^+$ is \emph{non-empty}.
Then it follows from \sref{II.4.5.7} and \sref{II.4.5.8} that
\[
  P^+ + P^+ \subset P^+
  \quad\text{and}\quad
  P^+ - P^+ = P
\]
or, in other words, $P^+\cup\{0\}$ is the set of \emph{positive} elements in $P$ for a \emph{preorder} structure on $P$ that is compatible with its group structure, and is even \emph{archimedean}, by \sref{II.4.5.8}.
This is why we sometimes say ``positive sheaf'' instead of ample sheaf, and ``negative sheaf'' for the inverse of an ample sheaf (but we will not use this terminology).
\end{remark}

\begin{proposition}[4.5.10]
\label{II.4.5.10}
Let $Y$ be an affine scheme, $q:X\to Y$ a quasi-compact separated morphism, and $\sh{L}$ an invertible $\sh{O}_X$-module.
\begin{enumerate}
  \item[\rm{(i)}] If $\sh{L}$ is very ample for $q$, then $\sh{L}$ is ample.
  \item[\rm{(ii)}] Suppose further that the morphism $q$ is \emph{of finite type}.
    Then, for $\sh{L}$ to be ample, it is necessary and sufficient for it to posses one of the following properties:
    \begin{enumerate}
      \item[\rm{(e)}] There exists $n_0>0$ such that, for every integer $n\geq n_0$, $\sh{L}^{\otimes n}$ is very ample for $q$.
      \item[\rm{(e')}] There exists $n>0$ such that $\sh{L}^{\otimes n}$ is very ample for $q$.
    \end{enumerate}
\end{enumerate}
\end{proposition}

\begin{proof}
The first claim follows from the definition \sref{II.4.4.2} of a very ample $\sh{O}_X$-module: if $A$ is the ring of $Y$, then there exists an $A$-module $E$ and a surjective homomorphism
\[
  \psi: q^*((\bb{S}(E))^{\supertilde}) \to \bigoplus_{n\geq0}\sh{L}^{\otimes n}
\]
such that $i=r_{\sh{L},\psi}$ is an everywhere-defined immersion $X\to P=\bb{P}(\widetilde{E})$ and such that $\sh{L}=i^*(\sh{O}_P(1))$;
since the $D_+(f)$ for $f$ homogeneous in $(\bb{S}(E))_+$ form a base for the topology of $P$, and since $i^{-1}(D_+(f))=X_{\psi^\flat(f)}$, by \sref{II.3.7.3.1}, we see that condition~(a) of \sref{II.4.5.2} is satisfied, and so $\sh{L}$ is ample.

Now to prove that, if $q$ is of finite type and $\sh{L}$ is ample, then condition~(e) is satisfied.
Firstly, it follows from criterion~(b) of \sref{II.4.5.2} and from \sref{II.4.4.1}[(i)] that there exists
\oldpage[II]{87}
an integer $k_0$ such that $\sh{L}^{\otimes k_0}$ is very ample relative to $q$.
Also, by \sref{II.4.5.5}, there exists an integer $m_0$ such that, for all $m\geq m_0$, $\sh{L}^{\otimes m}$ is generated by its sections over $X$.
Let $n_0=k_0+m_0$;
if $n\geq n_0$, then we can write $n=k_0+m$ with $m\geq m_0$, whence $\sh{L}^{\otimes n}=\sh{L}^{\otimes k_0}\otimes\sh{L}^{\otimes m}$.
Since $\sh{L}^{\otimes m}$ is generated by its sections over $X$, it follows from \sref{II.4.4.8} and \sref{II.3.4.7} that $\sh{L}^{\otimes n}$ is very ample relative to $q$.
Finally, it is clear that (e) implies (e'), and (e') implies that $\sh{L}$ is ample by (i) and by \sref{II.4.5.6}[(i)]

  \begin{env}[4.5.10.1]
  \label{II.4.5.10.1}
  \emph{[Proof of Lemma~\sref{II.4.4.10.1}].}  
  Let $\sh{E}(n)=\sh{E}\otimes\sh{K}^{\otimes n}$;
  since $g$ is separated \sref{II.4.4.2}, the proof of \sref{II.3.4.8} applies, and shows that the canonical homomorphism $g^*(g_*(\sh{E}(n)))\to\sh{E}(n)$ is surjective for $n$ large enough.
  Furthermore, since $Z$ is quasi-compact, the proof of \sref{II.3.4.6} means that if suffices to prove the claim in the case where $Z$ is affine.
  But $\sh{K}$ is then ample, by \sref{II.4.5.10}[(i)], and the conclusion follows from \sref{II.4.5.5}[(d')].
  \end{env}
\end{proof}

\begin{corollary}[4.5.11]
\label{II.4.5.11}
Let $Y$ be an affine scheme, $q:X\to Y$ a separated morphism of finite type, $\sh{L}$ an ample $\sh{O}_X$-module, and $\sh{L}'$ an invertible $\sh{O}_X$-module.
Then there exists an integer $n_0$ such that, for all $n\geq n_0$, $\sh{L}^{\otimes n}\otimes\sh{L}'$ is very ample relative to $q$.
\end{corollary}

\begin{proof}
There exists $m_0$ such that, for $m\geq m_0$, $\sh{L}^{\otimes m}\otimes\sh{L}'$ is generated by its sections over $X$ \sref{II.4.5.8};
there also exists $k_0$ such that $\sh{L}^{\otimes k}$ is very ample relative to $q$ for $k\geq k_0$.
Then $\sh{L}^{\otimes(k+m_0)}\otimes\sh{L}'$ is very ample if $k\geq k_0$ (\sref{II.4.4.8} and \sref{II.3.4.7}).
\end{proof}

\begin{remark}[4.5.12]
\label{II.4.5.12}
We do not know if the hypothesis that an $\sh{O}_X$-module $\sh{L}$ is such that $\sh{L}^{\otimes n}$ is very ample (relative to $q$) implies the same conclusion for $\sh{L}^{\otimes(n+1)}$.
\end{remark}

\begin{proposition}[4.5.13]
\label{II.4.5.13}
Let $X$ be a quasi-compact prescheme, $Z$ a closed prescheme of $X$ defined by a \emph{nilpotent} quasi-coherent sheaf $\sh{J}$ of ideals of $\sh{O}_X$, and $j$ the canonical injection $Z\to X$.
For an invertible $\sh{O}_X$-module $\sh{L}$ to be ample, it is necessary and sufficient for $\sh{L}'=j^*(\sh{L})$ to be an ample $\sh{O}_Z$-module.
\end{proposition}

\begin{proof}
The condition is \emph{necessary}.
Indeed, for every section $f$ of $\sh{L}^{\otimes n}$ over $X$, let $f'$ be its canonical image $f\otimes1$, which is a section of $\sh{L}'^{\otimes n}=\sh{L}^{\otimes n}\otimes_{\sh{O}_X}(\sh{O}_X/\sh{J})$ over the space $Z$ (which is identical to $X$);
it is clear that $X_f=Z_{f'}$, and so the criterion~(a) of \sref{II.4.5.2} shows that $\sh{L}'$ is ample.

To see that the condition is \emph{sufficient}, note first of all that we can restrict to the case where $\sh{J}^2=0$, by considering the (finite) sequence of preschemes $X_k=(X,\sh{O}_X/\sh{J}^{k+1})$ with each prescheme being a closed subprescheme of the next, defined by a square-zero sheaf of ideals.
But $X$ is a scheme, since $X_\red$ is a scheme by hypothesis (\sref{II.4.5.3} and \sref[I]{I.5.5.1}).
Criterion~(a) of \sref{II.4.5.2} shows that it suffices to prove
  \begin{lemma}[4.5.13.1]
  \label{II.4.5.13.1}
  Under the hypotheses of \sref{II.4.5.13}, suppose further that $\sh{J}$ is square-zero;
  with $\sh{L}$ being an invertible $\sh{O}_X$-module, let $g$ be a section of $\sh{L}'^{\otimes n}$ over $Z$ such that $Z_g$ is affine.
  Then there exists an integer $m>0$ such that $g^{\otimes m}$ is the canonical image of a section $f$ of $\sh{L}^{\otimes nm}$ over $X$.
  \end{lemma}

  \begin{proof}
  We have the exact sequence of $\sh{O}_X$-modules
  \[
    0 \to \sh{J}(n) \to \sh{O}_X(n)=\sh{L}^{\otimes n} \to \sh{O}_Z(n)=\sh{L}'^{\otimes n} \to 0
  \]
\oldpage[II]{88}
  since $\sh{F}(n)$ is an exact functor in $\sh{F}$;
  from this, we have the exact sequence of cohomology
  \[
    0 \to \Gamma(X,\sh{J}(n)) \to \Gamma(X,\sh{L}^{\otimes n}) \to \Gamma(X,\sh{L}'^{\otimes n}) \xrightarrow{\partial} \HH^1(X,\sh{J}(n))
  \]
  that sends, in particular, $g$ to an element $\partial g\in\HH^1(X,\sh{J}(n))$.
  \end{proof}

  Note that, since $\sh{J}^2=0$, $\sh{J}$ can be considered as a quasi-coherent $\sh{O}_Z$-module, and, for all $k$, $\sh{L}'^{\otimes k}\otimes_{\sh{O}_Z}\sh{J}(n)=\sh{J}(n+k)$;
  for every section $s\in\Gamma(X,\sh{L}'^{\otimes k})$, tensor multiplication with $s$ is thus a homomorphism $\sh{J}(n)\to\sh{J}(n+k)$ of $\sh{O}_Z$-modules, which then gives a homomorphism $\HH^i(X,\sh{J}(n))\xrightarrow{s}\HH^i(X,\sh{J}(n+k))$ of cohomology groups.

  With this, we will see that
  \[
  \label{II.4.5.13.2}
    g^{\otimes m}\otimes\partial g = 0
  \tag{4.5.13.2}
  \]
  for $m>0$ large enough.
  In fact, $Z_g$ is an \emph{affine open} subset of $Z$, and so $\HH^1(Z_g,\sh{J}(n))=0$ when $\sh{J}(n)$ is considered as an \emph{$\sh{O}_Z$-module} \sref[I]{I.5.1.9.2}.
  In particular, if we set $g'=g|Z_g$, and if we consider its image under the map $\partial:\Gamma(Z_g,\sh{L}'^{\otimes n})\to\HH^1(Z_g,\sh{J}(n))$, then $\partial g'=0$.
  To better explain this equation, we note that, in dimension~$1$, the cohomology of a sheaf of abelian groups is the same as its Čech cohomology (G, II, 5.9);
  to calculate $\partial g$, we must thus consider a fine-enough open cover $(U_\alpha)$ of $X$, that we can suppose to be \emph{finite} and consisting of affine opens, and take, for each $\alpha$, a section $g_\alpha\in\Gamma(U_\alpha,\sh{L}^{\otimes n})$ whose canonical image in $\Gamma(U_\alpha,\sh{L}'^{\otimes n})$ is $g|U_\alpha$, and to consider the cocycle class $(g_{\alpha|\beta}-g_{\beta|\alpha})$, with $g_{\alpha|\beta}$ being the restriction of $g_\alpha$ to $U_\alpha\cap U_\beta$ (with this cocycle taking values in $\sh{J}(n)$).
  We can further suppose that $\partial g'$ is calculated in the same way, by means of a cover given by the $U_\alpha\cap Z_g$, and restrictions $g_\alpha|(U_\alpha\cap Z_g)$ (by replacing, if necessary, $(U_\alpha)$ by a finer cover);
  the equation $\partial g'=0$ then implies that there exists, for each $\alpha$, a section $h_\alpha\in\Gamma(U_\alpha\cap Z_g,\sh{J}(n))$ such that $(g_{\alpha|\beta}-g_{\beta|\alpha})|(U_\alpha\cap U_\beta\cap Z_g) = h_{\alpha|\beta}-h_{\beta|\alpha}$, where $h_{\alpha|\beta}$ denotes the restriction of $h_\alpha$ to $U_\alpha\cap U_\beta\cap Z_g$ (G, II, 5.11).
  Then there exists an integer $m>0$ such that $g^{\otimes m}\otimes h_\alpha$ is the restriction to $U_\alpha\cap Z_g$ of a section $t_\alpha\in\Gamma(U_\alpha,\sh{J}(n+nm))$ for all $\alpha$ \sref[I]{I.9.3.1};
  thus $g^{\otimes m}\otimes(g_{\alpha|\beta}-g_{\beta|\alpha})=t_{\alpha|\beta}-t_{\beta|\alpha}$ for every pair of indices, which proves \sref{II.4.5.13.2}.

  Now note that, if $s\in\Gamma(X,\sh{O}_Z(p))$ and $t\in\Gamma(X,\sh{O}_Z(q))$, then, in the group $\HH^1(X,\sh{J}(p+q))$,
  \[
  \label{II.4.5.13.3}
    \partial(s\otimes t) = (\partial s)\otimes t + s\otimes(\partial t).
  \tag{4.5.13.3}
  \]

  Indeed, we can again calculate the two members by considering an open cover $(U_\alpha)$ of $X$, and, for each $\alpha$, a section $s_\alpha\in\Gamma(U_\alpha,\sh{O}_X(p))$ (resp. $t_\alpha\in\Gamma(U_\alpha,\sh{O}_X(q))$) whose canonical image in $\Gamma(U_\alpha,\sh{O}_Z(p)$ (resp. $\Gamma(U_\alpha,\sh{O}_Z(q))$) is $s|U_\alpha$ (resp. $t|U_\alpha$);
  equation~\sref{II.4.5.13.3} then follows from the equations
  \[
    (s_{\alpha|\beta}\otimes t_{\alpha|\beta}) - (s_{\beta|\alpha}\otimes t_{\beta|\alpha})
    = (s_{\alpha|\beta} - s_{\beta|\alpha})\otimes t_{\alpha|\beta} + s_{\beta|\alpha}\otimes(t_{\alpha|\beta} - t_{\beta|\alpha})
  \]
  with the same notation as above.
  By induction on $k$, we thus have
  \[
  \label{II.4.5.13.4}
    \partial(g^{\otimes k}) = (kg^{\otimes (k-1)})\otimes(\partial g)
  \tag{4.5.13.4}
  \]
\oldpage[II]{89}
  and we thus conclude from \sref{II.4.5.13.2} and \sref{II.4.5.13.4} that $\partial(g^{\otimes(m+1)}=0$;
  thus $g^{\otimes(m+1)}$ is the canonical image of a section $f$ of $\sh{L}^{\otimes n(m+1)}$ over $X$, which proves \sref{II.4.5.13}.
\end{proof}

\begin{corollary}[4.5.14]
\label{II.4.5.14}
Let $X$ be a Noetherian scheme, and $j$ the canonical injection $X_\red\to X$.
For an invertible $\sh{O}_X$-module $\sh{L}$ to be ample, it is necessary and sufficient for $j^*(\sh{L})$ to be an ample $\sh{O}_{X_\red}$-module.
\end{corollary}

\begin{proof}
This follows from \sref[I]{I.6.1.6}
\end{proof}


\subsection{Relatively ample sheaves}
\label{subsection:II.4.6}

\begin{definition}[4.6.1]
\label{II.4.6.1}
Let $f: X\to Y$ be a quasi-compact morphism, and $\sh{L}$ an invertible $\sh{O}_X$-module.
We say that $\sh{L}$ is \emph{ample relative to $f$}, or \emph{relative to $Y$}, or \emph{$f$-ample}, or \emph{$Y$-ample} (or even simply \emph{ample} if no confusion may arise with the notion defined in \sref{II.4.5.3}) if there exists an affine open cover $(U_\alpha)$ of $Y$ such that, if we set $X_\alpha=f^{-1}(U_\alpha)$, then $\sh{L}|X_\alpha$ is an ample $\sh{O}_{X_\alpha}$-module for all $\alpha$.
\end{definition}

The existence of an $f$-ample $\sh{O}_X$-module implies that $f$ is necessarily \emph{separated} (\sref{II.4.5.3} and \sref[I]{5.5.5}).

\begin{proposition}[4.6.2]
\label{II.4.6.2}
Let $f: X\to Y$ be a quasi-compact morphism, and $\sh{L}$ an invertible $\sh{O}_X$-module.
If $\sh{L}$ is very ample relative to $f$, then it is ample relative to $f$.
\end{proposition}

\begin{proof}
This follows from the local (on $Y$) character of the notion of a very ample sheaf \sref{II.4.4.5}, from the definition \sref{II.4.6.1}, and from criterion~\sref{II.4.5.10}[(i)].
\end{proof}

\begin{proposition}[4.6.3]
\label{II.4.6.3}
Let $f: X\to Y$ be a quasi-compact morphism, and $\sh{L}$ an invertible $\sh{O}_X$-module, and let $\sh{S}$ be the graded $\sh{O}_Y$-algebra $\bigoplus_{n\geq0}f_*(\sh{L}^{\otimes n})$.
Then the following conditions are equivalent:
\begin{enumerate}
  \item[\rm{(a)}] $\sh{L}$ is $f$-ample.
  \item[\rm{(b)}] $\sh{S}$ is quasi-coherent, and the canonical homomorphism $\sigma:f^*(\sh{S})\to\bigoplus_{n\geq0}\sh{L}^{\otimes n}$ \sref[0]{0.4.4.3} is such that the $Y$-morphism $r_{\sh{L},\sigma}:G(\sigma)\to\Proj(\sh{S})=P$ is everywhere defined and is an dominant open immersion.
  \item[\rm{(b')}] The morphism $f$ is separated, and the $Y$-morphism $r_{\sh{L},\sigma}$ is everywhere defined and is a homeomorphism from the underlying space of $X$ to a subspace of $\Proj(\sh{S})$.
\end{enumerate}

Furthermore, if any of the above are satisfied, then, for all $n\in\bb{Z}$, the canonical homomorphism
\[
\label{II.4.6.3.1}
  r_{\sh{L},\sigma}^*(\sh{O}_P(n)) \to \sh{L}^{\otimes n}
\tag{4.6.3.1}
\]
defined in \sref{II.3.7.9.1} is an isomorphism.

Finally, for every quasi-coherent $\sh{O}_X$-module $\sh{F}$, if we set $\sh{M}=\bigoplus_{n\geq0}f_*(\sh{F}\otimes\sh{L}^{\otimes n})$, then the canonical homomorphism
\[
\label{II.4.6.3.2}
  r_{\sh{L},\sigma}^*(\widetilde{\sh{M}}) \to \sh{F}
\tag{4.6.3.2}
\]
defined in \sref{II.3.7.9.2} is an isomorphism.
\end{proposition}

\begin{proof}
We note that (a) implies that $f$ is separated, and thus that $\sh{S}$ is quasi-coherent \sref[I]{I.9.2.2}[(a)].
Since $r_{\sh{L},\sigma}$ being an everywhere defined immersion is of a local (on $Y$) character, to prove that (a) implies (b) we can suppose that $Y$ is affine and $\sh{L}$ ample;
\oldpage[II]{90}
the claim then follows from \sref{II.4.5.2}[(b)].
It is clear that (b) implies (b');
finally, to prove that (b') implies (a), it suffices to consider an affine open cover $(U_\alpha)$ of $Y$ and to apply the criterion~\sref{II.4.5.2}[(b')] to each sheaf $\sh{L}|f^{-1}(U_\alpha)$.

For the final two claims, we use the fact that $\sigma^\flat$ is here the identity, and the clarification of the homomorphisms \sref{II.3.7.9.1} and \sref{II.3.7.9.2};
from this, it immediately follows that \sref{II.4.6.3.1} is an isomorphism.
As for \sref{II.4.6.3.2}, we can restrict to the case where $Y$ is affine, and thus $\sh{L}$ ample;
it is clear that the homomorphism \sref{II.4.6.3.2} is injective, and criterion~\sref{II.4.5.2}[(c)] shows that it is surjective, whence the conclusion.
\end{proof}

\begin{corollary}[4.6.4]
\label{II.4.6.4}
Let $(U_\alpha)$ be an open cover of $Y$.
For $\sh{L}$ to be ample relative to $Y$, it is necessary and sufficient for $\sh{L}|f^{-1}(U_\alpha)$ to be ample relative to $U_\alpha$ for all $\alpha$.
\end{corollary}

\begin{proof}
Condition~(b) is in fact local on $Y$.
\end{proof}

\begin{corollary}[4.6.5]
\label{II.4.6.5}
Let $\sh{K}$ be an invertible $\sh{O}_Y$-module.
For $\sh{L}$ to be $Y$-ample, it is necessary and sufficient for $\sh{L}\otimes f^*(\sh{K})$ to be $Y$-ample.
\end{corollary}

\begin{proof}
This is an evident consequence of \sref{II.4.6.4}, by taking the $U_\alpha$ to be such that $\sh{K}|U_\alpha$ is isomorphic to $\sh{O}_Y|U_\alpha$ for all $\alpha$.
\end{proof}

\begin{corollary}[4.6.6]
\label{II.4.6.6}
Suppose that $Y$ is affine;
for $\sh{L}$ to be $Y$-ample, it is necessary and sufficient for $\sh{L}$ to be ample.
\end{corollary}

\begin{proof}
This is an immediate consequence of the definition \sref{II.4.6.1} and the criteria \sref{II.4.6.3}[(b)] and \sref{II.4.5.2}[(b)], since here $\Proj(\sh{S})=\Proj(\Gamma(Y,\sh{S}))$ by definition.
\end{proof}

\begin{corollary}[4.6.7]
\label{II.4.6.7}
Let $f: X\to Y$ be a quasi-compact morphism.
Suppose that there exists a quasi-coherent $\sh{O}_Y$-module $\sh{E}$, and a $Y$-morphism $g:X\to P=\bb{P}(\sh{E})$ that is a homeomorphism from the underlying space of $X$ to a subspace of $\bb{P}$;
then $\sh{L}=g^*(\sh{O}_P(1))$ is $Y$-ample.
\end{corollary}

\begin{proof}
We can assume $Y$ to be affine;
the corollary then follows from the criterion~\sref{II.4.5.2}[(a)], from equation~\sref{II.3.7.3.1}, and from \sref{II.4.2.3}.
\end{proof}

\begin{proposition}[4.6.8]
\label{II.4.6.8}
Let $X$ be a quasi-compact scheme or a prescheme whose underlying space is Noetherian, and let $f:X\to Y$ be a quasi-compact separated morphism.
For an invertible $\sh{O}_X$-module $\sh{L}$ to be $f$-ample, it is necessary and sufficient for one of the following equivalent conditions to be satisfied:
\begin{enumerate}
  \item[\rm{(c)}] For every quasi-coherent $\sh{O}_X$-module $\sh{F}$ of finite type, there exists an integer $n_0>0$ such that, for all $n\geq n_0$, the canonical homomorphism $\sigma:f^*(f_*(\sh{F}\otimes\sh{L}^{\otimes n}))\to\sh{F}\otimes\sh{L}^{\otimes n}$ is surjective.
  \item[\rm{(c')}] For every quasi-coherent sheaf $\sh{J}$ of ideals of $\sh{O}_X$ of finite type, there exists an integer $n>0$ such that the canonical homomorphism $\sigma:f^*(f_*(\sh{J}\otimes\sh{L}^{\otimes n}))\to\sh{J}\otimes\sh{L}^{\otimes n}$ is surjective.
\end{enumerate}
\end{proposition}

\begin{proof}
Since $X$ is quasi-compact, so too is $f(X)$, and so there exists a finite cover $(U_i)$ of $f(X)$ consisting of affine open subsets $U_i$ of $Y$.
To prove condition~(c) when $\sh{L}$ is $f$-ample, we can replace $Y$ by the $U_i$, and $X$ by the $f^{-1}(U_i)$, since, if we obtain, for each $i$, an integer $n_i$ such that (c) holds true (for $U_i$, $f^{-1}(U_i)$, and $\sh{L}|f^{-1}(U_i)$) for all $n\geq n_i$, then it suffices to take $n_0$ to be the largest of the $n_i$ in order to obtain (c) for $Y$, $X$, and $\sh{L}$.
But if $Y$ is affine, condition~(c) follows from \sref{II.4.5.5}[(d)], taking \sref{II.4.6.6} into account.
It is trivial that (c) implies (c').
Finally, to prove that (c') implies that $\sh{L}$ is $f$-ample, we can again restrict to the case where $Y$ is affine:
in fact, every quasi-coherent sheaf $\sh{J}_i$ of ideals of $\sh{O}_X|f^{-1}(U_i)$ of finite type is the restriction of a coherent sheaf of ideals of $\sh{O}_X$ of finite type \sref[I]{I.9.4.7}, and hypothesis~(c') implies that
\oldpage[II]{91}
$\sh{J}_i\otimes(\sh{L}^{\otimes n}|f^{-1}(U_i))$ is generated by its sections (taking \sref[I]{I.9.2.2} and \sref{II.3.4.7} into account);
it thus suffices to apply criterion~\sref{II.4.5.5}[(d'')].
\end{proof}

\begin{proposition}[4.6.9]
\label{II.4.6.9}
Let $f:X\to Y$ be a quasi-compact morphism, and $\sh{L}$ an invertible $\sh{O}_X$-module.
\begin{enumerate}
  \item[\rm{(i)}] Let $n>0$ be an integer.
    For $\sh{L}$ to be $f$-ample, it is necessary and sufficient for $\sh{L}^{\otimes n}$ to be $f$-ample.
  \item[\rm{(ii)}] Let $\sh{L}'$ be an invertible $\sh{O}_X$-module, and suppose that there exists an integer $n>0$ such that the canonical homomorphism $\sigma:f^*(f_*(\sh{L}'^{\otimes n}))\to\sh{L}'^{\otimes n}$ is surjective.
    Then, if $\sh{L}$ if $f$-ample, so too is $\sh{L}\otimes\sh{L}'$.
\end{enumerate}
\end{proposition}

\begin{proof}
We can in fact immediately restrict to the case where $Y$ is affine, and the proposition is then an immediate consequence of \sref{II.4.5.6}.
\end{proof}

\begin{corollary}[4.6.10]
\label{II.4.6.10}
The tensor product of two $f$-ample $\sh{O}_X$-modules is $f$-ample.
\end{corollary}

\begin{proposition}[4.6.11]
\label{II.4.6.11}
Let $Y$ be a quasi-compact prescheme, $f:X\to Y$ a morphism of finite type, and $\sh{L}$ an invertible $\sh{O}_X$-module.
For $\sh{L}$ to be $f$-ample, it is necessary and sufficient for it to posses one of the following equivalent properties:
\begin{enumerate}
  \item[\rm{(d)}] There exists some $n_0>0$ such that, for every integer $n\geq n_0$, $\sh{L}^{\otimes n}$ is very ample relative to $f$.
  \item[\rm{(d')}] There exists some $n>0$ such that $\sh{L}^{\otimes n}$ is very ample relative to $f$.
\end{enumerate}
\end{proposition}

\begin{proof}
If $\sh{L}$ is ample relative to $f$, then there exists a \emph{finite} cover $(U_i)$ of $Y$ by affine open subsets such that the $\sh{L}|f^{-1}(U_i)$ are ample.
We thus conclude \sref{II.4.5.10} that there exists an integer $n_0$ such that $\sh{L}^{\otimes n}|f^{-1}(U_i)$ is very ample relative to $f^{-1}(U_i)\to U_i$ for all $n\geq n_0$ and every $i$, and so $\sh{L}^{\otimes n}$ is very ample relative to $f$ \sref{II.4.5.5}.
Conversely, (d') already implies that $\sh{L}^{\otimes n}$ is $f$-ample \sref{II.4.6.2}, and thus so too is $\sh{L}$ \sref{II.4.6.9}[(i)].
\end{proof}

\begin{corollary}[4.6.12]
\label{II.4.6.12}
Let $Y$ be a quasi-compact prescheme, $f:X\to Y$ a morphism of finite type, and $\sh{L}$ and $\sh{L}'$ invertible $\sh{O}_X$-modules.
If $\sh{L}$ is $f$ ample, then there exists some $n_0$ such that $\sh{L}^{\otimes n}\otimes\sh{L}'$ is very ample relative to $f$ for all $n\geq n_0$.
\end{corollary}

\begin{proof}
We argue as in \sref{II.4.6.11}, by using a finite affine open cover of $Y$ and \sref{II.4.5.11}.
\end{proof}

\begin{proposition}[4.6.13]
\label{II.4.6.13}
\begin{enumerate}
  \item[\rm{(i)}] For every prescheme $Y$, every invertible $\sh{O}_Y$-module $\sh{L}$ is ample relative to the identity morphism $1_Y$.
  \item[\rm{(i \emph{bis})}] Let $f:X\to Y$ be a quasi-compact morphism, and $j:X'\to X$ a quasi-compact morphism that is a homeomorphism from the underlying space of $X'$ to a subspace of $X$.
    If $\sh{L}$ is an $\sh{O}_X$-module that is ample relative to $f$, then $j^*(\sh{L})$ is ample relative to $f\circ j$.
  \item[\rm{(ii)}] Let $Z$ be a quasi-compact prescheme, $f:X\to Y$ and $g:Y\to Z$ quasi-compact morphisms, $\sh{L}$ an $\sh{O}_X$-module that is ample relative to $f$, and $\sh{K}$ an $\sh{O}_Y$-module that is ample relative to $g$.
    Then there exists an integer $n_0>0$ such that $\sh{L}\otimes f^*(\sh{K}^{\otimes n})$ is ample relative to $g\circ f$ for all $n\geq n_0$.
  \item[\rm{(iii)}] Let $f:X\to Y$ be a quasi-compact morphism, and $g:Y'\to Y$ a morphism, and let $X'=X_{(Y')}$.
    If $\sh{L}$ is an $\sh{O}_X$-module that is ample relative to $f$, then $\sh{L}'=\sh{L}\otimes_Y\sh{O}_{Y'}$ is an $\sh{O}_{X'}$-module that is ample relative to $f_{(Y')}$.
  \item[\rm{(iv)}] Let $f_i:X_i\to Y_i$ ($i=1,2$) be quasi-compact $S$-morphisms.
    If $\sh{L}_i$ is an $\sh{O}_{X_i}$-modules that is ample relative to $f_i$ ($i=1,2$), then $\sh{L}_1\otimes_S\sh{L}_2$ is ample relative to $f_1\times_S f_2$.
  \item[\rm{(v)}] Let $f:X\to Y$ and $g:Y\to Z$ be morphisms such that $g\circ f$ is quasi-compact.
    If an
\oldpage[II]{92}
    $\sh{O}_X$-module $\sh{L}$ is ample relative to $g\circ f$, and if $g$ is separated or the underlying space of $X$ is locally Noetherian, then $\sh{L}$ is ample relative to $f$.
  \item[\rm{(vi)}] Let $f:X\to Y$ be a quasi-compact morphism, and $j$ the canonical injection $X_\red\to X$.
    If $\sh{L}$ is an $\sh{O}_X$-module that is ample relative to $f$, then $j^*(\sh{L})$ is ample relative to $f_\red$.
\end{enumerate}
\end{proposition}

\begin{proof}
Note first of all that (v) and (vi) follow from (i), (i~\emph{bis}), and (iv) by the same argument as in \sref{II.4.4.10}, by using \sref{II.4.6.4} instead of \sref{II.4.4.5};
we leave the details of the argument to the reader.
Claim~(i) is trivially a consequence of \sref{II.4.4.10}[(i)] and \sref{II.4.6.2}.
To prove (i~\emph{bis}), (iii), and (iv), we will use the following lemma:
  \begin{lemma}[4.6.13.1]
  \label{II.4.6.13.1}
  \begin{enumerate}
    \item[\rm{(i)}] Let $u:Z\to S$ be a morphism, $\sh{L}$ an invertible $\sh{O}_S$-module, and $s$ a section of $\sh{L}$ over $S$, and let $s'$ be the canonically corresponding section $u^*(\sh{L})=\sh{L}'$ over $Z$.
      Then $Z_{s'}=u^{-1}(S_s)$.
    \item[\rm{(ii)}] Let $Z$ and $Z'$ be $S$-preschemes, $p$ and $p'$ the projections of $T=Z\times_S Z'$, $\sh{L}$ (resp. $\sh{L}'$) an invertible $\sh{O}_Z$-module (resp. invertible $\sh{O}_{Z'}$-module), and let $t$ (resp. $t'$) be a section of $\sh{L}$ (resp. $\sh{L}'$) over $Z$ (resp. $Z'$), and $s$ (resp. $s'$) the corresponding section of $p^*(\sh{L})$ (resp. $p'^*(\sh{L}')$) over $Z\times_S Z'$.
      Then $T_{s\otimes s'}=Z_t\times_S Z'_{t'}$.
  \end{enumerate}
  \end{lemma}

  \begin{proof}
  It follows from the definitions that we can restrict to the case where all the preschemes in question are affine.
  Furthermore, in (i), we can suppose that $\sh{L}=\sh{O}_B$;
  claim~(i) then follows immediately from \sref[I]{I.1.2.2.2}.
  Similarly, in (ii), we can restrict to the case where $\sh{L}=\sh{O}_Z$ and $\sh{L}'=\sh{O}_{Z'}$, and then the claim reduces to Lemma~\sref{II.4.3.2.4}.
  \end{proof}

We now prove (i~\emph{bis}).
We can assume that $Y$ is affine \sref{II.4.6.4}, and thus that $\sh{L}$ is ample \sref{II.4.6.6};
if $s$ runs over the set of the union of the $\Gamma(X,\sh{L}^{\otimes n})$ ($n>0$), then the $X_s$ form a base for the topology of $X$ \sref{II.4.5.2}[(a)], and so, by hypothesis, the $j^{-1}(X_s)$ form a base of the topology of $X'$;
we thus conclude, by Lemma~\sref{II.4.6.13.1}[(i)] and \sref{II.4.5.2}[(a)], that $j^*(\sh{L})$ is ample.

Next we prove (iii).
We can again suppose that $Y$ and $Y'$ are affine \sref{II.4.6.4}, whence it follows that the projection $h:X'\to X$ is affine \sref{II.1.5.5}.
Since $\sh{L}$ is ample \sref{II.4.6.6}, if $s$ runs over the set of sections of the $\sh{L}^{\otimes n}$ ($n>0$) over $X$ such that $X_s$ is affine, then the $X_s$ cover $X$ \sref{II.4.5.2}[(a')], and so the $h^{-1}(X_s)$ are affine \sref{II.1.2.5} and cover $X'$;
it thus follows again from Lemma~\sref{II.4.6.13.1}[(i)] and from \sref{II.4.5.2}[(a')] that $\sh{L}'$ is ample, since the morphism $f_{(Y')}$ is quasi-compact \sref[I]{I.6.6.4}[(iii)].

To prove (iv), note first of all that $f_1\times_S f_2$ is quasi-compact \sref[I]{I.6.6.4}[(iv)].
We can further suppose that $S$, $Y_1$, and $Y_2$ are affine (\sref{II.4.6.4} and \sref[I]{I.3.2.7}), and thus that $\sh{L}_i$ is ample ($i=1,2$) \sref{II.4.6.6}.
The open subsets $(X_1)_{s_1}\times_S(X_2)_{s_2}$ form a cover of $X_1\times_S X_2$, where $s_i$ runs over the sections of $\sh{L}_i^{\otimes n_i}$ such that $(X_i)_{s_i}$ is affine \sref{II.4.5.2}[(a')].
Then, replacing $s_1$ and $s_2$ with suitable powers, which does not change the $(X_i)_{s_i}$, we can assume that $n_1=n_2$.
We thus deduce, from \sref{II.4.6.13.1}[(ii)] and \sref{II.4.5.2}[(a')], that $\sh{L}_1\otimes_S\sh{L}_2$ is ample, whence the claim, since $Y_1\times_S Y_2$ is affine \sref{II.4.6.6}.

It remains only to prove (ii).
By the same argument as in \sref{II.4.4.10}, but here using \sref{II.4.6.4}, we can restrict to the case where $Z$ is affine.
Since $\sh{K}$ is then ample, and $Y$ quasi-compact, there exists a finite number of sections $s_i\in\Gamma(Y,\sh{K}^{\otimes k_i})$ such that the $Y_{s_i}$
\oldpage[II]{93}
are \emph{affine} and cover $Y$ \sref{II.4.5.2}[(a')];
replacing the $s_i$ with suitable powers, we can further suppose that all the $k_i$ are equal to one single integer $k$.
Let $s'_i$ be the sections of $f^*(\sh{K}^{\otimes k})$ over $X$ that canonically correspond to the $s_i$, so that the $X_{s'_i}=f^{-1}(Y_{s_i})$ \sref{II.4.6.1.13}[(i)] cover $X$.
Since $\sh{L}|X_{s'_i}$ is ample (\sref{II.4.6.4} and \sref{II.4.6.6}), there exists, for each $i$, a finite number of sections $t_{ij}\in\Gamma(X,\sh{L}^{\otimes n_{ij}})$ such that the $X_{ij}$ are affine, contained in the $X_{s'_i}$, and cover $X_{s'_i}$ \sref{II.4.5.2}[(a')];
we can also suppose that all the $n_{ij}$ are equal to one single integer $n$.
With this in mind, $X$ is separated and quasi-compact, and so there exists an integer $m>0$, and, for every $(i,j)$, a section
\[
  u_{ij}\in\Gamma(X,\sh{L}^{\otimes n}\otimes_X f^*(\sh{K}^{\otimes mk}))
\]
such that $t_{ij}\otimes s'^{\otimes m}_i$ is the restriction to $X_{s'_i}$ of $u_{ij}$ \sref[I]{I.9.3.1};
furthermore, $X_{u_{ij}}=X_{t_{ij}}$, and so the $X_{u_{ij}}$ are affine and cover $X$.
We can also suppose that $m$ is of the form $nr$;
if we set $n_0=rk$, then we see \sref{II.4.5.2}[(a')] that $\sh{L}\otimes_{\sh{O}_X}f^*(\sh{K}^{\otimes n_0})$ is ample.
Furthermore, there exists $h_0>0$ such that $\sh{K}^{\otimes h}$ is generated by its sections over $Y$ for all $h\geq h_0$ \sref{II.4.5.5};
\emph{a fortiori}, $f^*(\sh{K}^{\otimes h})$ is generated by its sections over $X$ for all $h\geq h_0$, by definition of the inverse images (\sref[0]{0.3.7.1} and \sref{II.4.4.1}).
We thus conclude that $\sh{L}\otimes f^*(\sh{K}^{\otimes n_0+h})$ is ample for all $h\geq h_0$ \sref{II.4.5.6}, which finishes the proof.
\end{proof}

\begin{remark}[4.6.14]
\label{II.4.6.14}
Under the conditions of (ii), we refrain from believing that $\sh{L}\otimes f^*(\sh{K})$ is ample for $g\circ f$;
in fact, since $\sh{L}\otimes f^*(\sh{K}^{-1})$ is also ample for $f$ \sref{II.4.6.5}, we would conclude that $\sh{L}$ is ample for $g\circ f$;
taking, in particular, $g$ to be the identity morphism, \emph{every} invertible $\sh{O}_X$-module would be ample for $f$, which is not the case in general (see \sref{II.5.1.6}, \sref{II.5.3.4}[(i)], and \sref{II.5.3.1}).
\end{remark}

\begin{proposition}[4.6.15]
\label{II.4.6.15}
Let $f:X\to Y$ be a quasi-compact morphism, $\sh{J}$ a quasi-coherent locally-nilpotent sheaf of ideals of $\sh{O}_X$, $Z$ the closed subprescheme of $X$ defined by $\sh{J}$, and $j:Z\to X$ the canonical injection.
For an invertible $\sh{O}_X$-module $\sh{L}$ to be ample for $f$, it is necessary and sufficient for $j^*(\sh{L})$ to be ample for $f\circ j$.
\end{proposition}

\begin{proof}
Since the question is local on $Y$ \sref{II.4.6.4}, we can suppose $Y$ to be affine;
since $X$ is then quasi-compact, we can suppose $\sh{J}$ to be nilpotent.
Taking \sref{II.4.6.6} into account, the proposition is then exactly the same as \sref{II.4.5.13}.
\end{proof}

\begin{corollary}[4.6.16]
\label{II.4.6.16}
Let $X$ be a locally Noetherian prescheme, and $f:X\to Y$ a quasi-compact morphism.
For an invertible $\sh{O}_X$-module $\sh{L}$ to be ample for $f$, it is necessary and sufficient for its inverse image $\sh{L}'$ under the canonical injection $X_\red\to X$ to be ample for $f_\red$.
\end{corollary}

\begin{proof}
We have already seen that the condition is necessary \sref{II.4.6.13}[(vi)];
conversely, if it is satisfied, then we can restrict, to prove that $\sh{L}$ is ample for $f$, to the case where $Y$ is affine \sref{II.4.6.4};
then $Y_\red$ is also affine, and so $\sh{L}'$ is ample \sref{II.4.6.6}, and so too is $\sh{L}$ by \sref{II.4.5.13}, since $X$ is then Noetherian and $X_\red$ a closed subprescheme of $X$ defined by a quasi-coherent nilpotent sheaf of ideals \sref[I]{I.6.1.6}.
\end{proof}

\begin{proposition}[4.6.17]
\label{II.4.6.17}
With the notation and hypotheses of \sref{II.4.4.11}, for $\sh{L}''$ to be ample relative to $f''$, it is necessary and sufficient for $\sh{L}$ to be ample relative to $f$ and $\sh{L}'$ ample relative to $f'$.
\end{proposition}

\oldpage[II]{94}
\begin{proof}
The necessity of the condition follows from \sref{II.4.6.13}[(i~\emph{bis})],.
To see that the condition is sufficient, we can restrict to the case where $Y$ is affine, and then the fact that $\sh{L}''$ is ample follows from criterion~\sref{II.4.5.2}[(a)] applied to $\sh{L}$, $\sh{L}'$, and $\sh{L}''$, noting that a section of $\sh{L}$ over $X$ can be extended (by $0$) to a section of $\sh{L}''$ over $X''$.
\end{proof}

\begin{proposition}[4.6.18]
\label{II.4.6.18}
Let $Y$ be a quasi-compact prescheme, $\sh{S}$ a quasi-coherent graded $\sh{O}_Y$-algebra of finite type, and $X=\Proj(\sh{S})$, and let $f:X\to Y$ the structure morphism.
Then $f$ is of finite type, and there exists an integer $d>0$ such that $\sh{O}_X(d)$ is invertible and $f$-ample.
\end{proposition}

\begin{proof}
By \sref{II.3.1.10}, there exists an integer $d>0$ such that $\sh{S}^{(d)}$ is generated by $\sh{S}_d$.
We know that, under the canonical isomorphism between $X$ and $X^{(d)}=\Proj(\sh{S}^{(d)})$, $\sh{O}_X(d)$ is identified with $\sh{O}_{X^{(d)}}(1)$ \sref{II.3.2.9}[(ii)].
We thus see that we can restrict to the case where $\sh{S}$ is generated by $\sh{S}_1$;
the proposition then follows from \sref{II.4.4.3} and \sref{II.4.6.2} (taking into account the fact that $f$ is a morphism of finite type \sref{II.3.4.1}).
\end{proof}

\section{Quasi-affine morphisms; quasi-projective morphisms; proper morphisms; projective morphisms}
\label{section:II.5}

\subsection{Quasi-affine morphisms}
\label{subsection:II.5.1}

\begin{definition}[5.1.1]
\label{II.5.1.1}
We define a quasi-affine scheme to be a scheme isomorphic to some subscheme induced on some quasi-compact open subset of an affine scheme.
We say that a morphism $f:X\to Y$ is quasi-affine, or that $X$ (considered as a $Y$-prescheme via $f$) is a quasi-affine $Y$-scheme, if there exists a cover $(U_\alpha)$ of $Y$ by affine open subsets such that the $f^{-1}(U_\alpha)$ are quasi-affine schemes.
\end{definition}

It is clear that a quasi-affine morphism is \emph{separated} (\sref[I]{I.5.5.5} and \sref[I]{I.5.5.8}) and \emph{quasi-compact} \sref[I]{I.6.6.1};
every affine morphisms is evidently quasi-affine.

Recall that, for any prescheme $X$, setting $A=\Gamma(X,\sh{O}_X)$, the identity homomorphism $A\to A=\Gamma(X,\sh{O}_X)$ defines a morphism $X\to\Spec(A)$, said to be \emph{canonical} \sref[I]{I.2.2.4};
this is nothing but the canonical morphism defined in \sref{II.4.5.1} for the specific case where $\sh{L}=\sh{O}_X$, if we remember that $\Proj(A[T])$ is canonically identified with $\Spec(A)$ \sref{II.3.1.7}.

\begin{proposition}[5.1.2]
\label{II.5.1.2}
Let $X$ be a quasi-compact scheme or a prescheme whose underlying space is Noetherian, and $A$ the ring $\Gamma(X,\sh{O}_X)$.
The following conditions are equivalent.
\begin{enumerate}
  \item[{\rm(a)}] $X$ is a quasi-affine scheme.
  \item[{\rm(b)}] The canonical morphism $u:X\to\Spec(A)$ is an open immersion.
  \item[{\rm(b')}] The canonical morphism $u:X\to\Spec(A)$ is a homeomorphism from $X$ to some subspace of the underlying space of $\Spec(A)$.
  \item[{\rm(c)}] The $\sh{O}_X$-module $\sh{O}_X$ is very ample relative to $u$ \sref{II.4.4.2}.
  \item[{\rm(c')}] The $\sh{O}_X$-module $\sh{O}_X$ is ample \sref{II.4.5.1}.
  \item[{\rm(d)}] When $f$ ranges over $A$, the $X_f$ form a basis for the topology of $X$.
  \item[{\rm(d')}] When $f$ varies over $A$, the $X_f$ that are affine form a cover of $X$.
\oldpage[II]{95}
  \item[{\rm(e)}] Every quasi-coherent $\sh{O}_X$-module is generated by its sections over $X$.
  \item[{\rm(e')}] Every quasi-coherent sheaf of ideals of $\sh{O}_X$ of finite type is generated by its sections over $X$.
\end{enumerate}
\end{proposition}

\begin{proof}
It is clear that (b) implies (a), and (a) implies (c) by \sref{II.4.4.4}[b] applied to the identity morphism (taking into account the remark preceding this proposition);
Furthermore, (c) implies (c$'$) \sref{II.4.5.10}[i], and (c$'$), (b), and (b$'$) are all equivalent by \sref{II.4.5.2}[b] and \sref{II.4.5.2}[b$'$].
Finally, (c$'$) is the same as each of (d), (d$'$), (e), and (e$'$) by \sref{II.4.5.2}[a], \sref{II.4.5.2}[a$'$], \sref{II.4.5.2}[c], and \sref{II.4.5.5}[d$''$].
\end{proof}

We further observe that, with the previous notation, the $X_f$ that are affine form a \emph{basis} for the topology of $X$, and that the canonical morphism $u$ is \emph{dominant} \sref{II.4.5.2}.

\begin{corollary}[5.1.3]
\label{1.5.1.3}
Let $X$ be a quasi-compact prescheme.
If there exists a morphism $v:X\to Y$ from $X$ to some affine scheme $Y$ (which would be a homeomorphism from $X$ to some open subspace of $Y$), then $X$ is quasi-affine.
\end{corollary}

\begin{proof}
There exists a family $(g_\alpha)$ of sections of $\sh{O}_Y$ over $Y$ such that the $D(g_\alpha)$ form a basis for the topology of $v(X)$;
if $v=(\psi,\theta)$ and we set $f_\alpha=\theta(g_\alpha)$, then we have $X_{f_\alpha}=\psi^{-1}(D(g_\alpha))$ \sref[I]{I.2.2.4.1}, so the $X_{f_\alpha}$ form a basis for the topology of $X$, and the criterion \sref{II.5.1.2}[d] is satisfied.
\end{proof}

\begin{corollary}[5.1.4]
\label{II.5.1.4}
If $X$ is a quasi-affine scheme, then \emph{every} invertible $\sh{O}_X$-module is very ample (relative to the canonical morphism), and \emph{a fortiori} ample.
\end{corollary}

\begin{proof}
Such a module $\sh{L}$ is generated by its sections over $X$ \sref{II.5.1.2}[e], so $\sh{L}\otimes\sh{O}_X=\sh{L}$ is very ample \sref{II.4.4.8}.
\end{proof}

\begin{corollary}[5.1.5]
\label{II.5.1.5}
Let $X$ be a quasi-compact prescheme.
If there exists an invertible $\sh{O}_X$-module $\sh{L}$ such that $\sh{L}$ and $\sh{L}^{-1}$ are ample, then $X$ is a quasi-affine scheme.
\end{corollary}

\begin{proof}
Indeed, $\sh{O}_X=\sh{L}\otimes\sh{L}^{-1}$ is then ample \sref{II.4.5.7}.
\end{proof}

\begin{proposition}[5.1.6]
\label{II.5.1.6}
Let $f:X\to Y$ be a quasi-compact morphism.
Then the following conditions are equivalent.
\begin{enumerate}
  \item[{\rm(a)}] The morphism $f$ is quasi-affine.
  \item[{\rm(b)}] The $\sh{O}_Y$-algebra $f_*(\sh{O}_X)=\sh{A}(X)$ is quasi-coherent, and the canonical morphism $X\to\Spec(\sh{A}(X))$ corresponding to the identity morphism $\sh{A}(X)\to\sh{A}(X)$ \sref{II.1.2.7} is an open immersion.
  \item[{\rm(b')}] The $\sh{O}_Y$-algebra $\sh{A}(X)$ is quasi-coherent, and the canonical morphism $X\to\Spec(\sh{A}(X))$ is a homeomorphism from $X$ to some subspace of $\Spec(\sh{A}(X))$.
  \item[{\rm(c)}] The $\sh{O}_X$-module $\sh{O}_X$ is very ample for $f$.
  \item[{\rm(c')}] The $\sh{O}_X$-module $\sh{O}_X$ is ample for $f$.
  \item[{\rm(d)}] The morphism $f$ is separated, and, for every quasi-coherent $\sh{O}_X$-module $\sh{F}$, the canonical homomorphism $\sigma:f^*(f_*(\sh{F}))\to\sh{F}$ \sref[0]{0.4.4.3} is surjective.
\end{enumerate}

Furthermore, whenever $f$ is quasi-affine, every invertible $\sh{O}_X$-module $\sh{L}$ is very ample relative to $f$.
\end{proposition}

\begin{proof}
The equivalence between (a) and (c$'$) follows from the local (on $Y$) character of the $f$-ampleness \sref{II.4.6.4}, Definition~\sref{II.5.1.1}, and the criterion \sref{II.5.1.2}[c$'$].
The other properties are local on $Y$
\oldpage[II]{96}
and thus follow immediately from \sref{II.5.1.2} and \sref{II.5.1.4}, taking into account the fact that $f_*(\sh{F})$ is quasi-coherent whenever $f$ is separated \sref[I]{I.9.2.2}[a].
\end{proof}

\begin{corollary}[5.1.7]
\label{II.5.1.7}
Let $f:X\to Y$ be a quasi-affine morphism.
For every open subset $U$ of $Y$, the restriction $f^{-1}(U)\to U$ of $f$ is quasi-affine.
\end{corollary}

\begin{corollary}[5.1.8]
\label{II.5.1.8}
Let $Y$ be an affine scheme, and $f:X\to Y$ a quasi-compact morphism.
For $f$ to be quasi-affine, it is necessary and sufficient for $X$ to be a quasi-affine scheme.
\end{corollary}

\begin{proof}
This is an immediate consequence of \sref{II.5.1.6} and \sref{II.4.6.6}.
\end{proof}

\begin{corollary}[5.1.9]
\label{II.5.1.9}
Let $Y$ be a quasi-compact scheme or a prescheme whose underlying space is Noetherian, and $f:X\to Y$ a morphism of \emph{finite type}.
If $f$ is quasi-affine, then there exists a quasi-coherent $\sh{O}_Y$-subalgebra $\sh{B}$ of $\sh{A}(X)=f_*(\sh{O}_X)$ of \emph{finite type} \sref[I]{I.9.6.2} such that the morphism $X\to\Spec(\sh{B})$ corresponding to the canonical injection $\sh{B}\to\sh{A}(X)$ is an immersion.
Further, every quasi-coherent $\sh{O}_Y$-subalgebra $\sh{B}'$ of finite type over $\sh{A}(X)$ containing $\sh{B}$ has the same property.
\end{corollary}

\begin{proof}
Indeed, $\sh{A}(X)$ is the inductive limit of its quasi-coherent $\sh{O}_Y$-subalgebras of finite type \sref[I]{I.9.6.5};
the result is then a particular case of \sref{II.3.8.4}, taking into account the identification of $\Spec(\sh{A}(X))$ with $\Proj(\sh{A}(X)[T])$ \sref{II.3.1.7}.
\end{proof}

\begin{proposition}[5.1.10]
\label{II.5.1.10}
\medskip\noindent
\begin{enumerate}
  \item[{\rm(i)}] A quasi-compact morphism $X\to Y$ that is a homeomorphism from the underlying space of $X$ to some subspace of the underlying space of $Y$ (so, in particular, any closed immersion) is quasi-affine.
  \item[{\rm(ii)}] The composition of any two quasi-affine morphisms is quasi-affine.
  \item[{\rm(iii)}] If $f:X\to Y$ is a quasi-affine $S$-morphism, then $f_{(S')}:X_{(S')}\to Y_{(S')}$ is a quasi-affine morphism for any extension $S'\to S$ of the base prescheme.
  \item[{\rm(iv)}] If $f:X\to Y$ and $g:X'\to Y'$ are quasi-affine $S$-morphisms, then $f\times_S g$ is quasi-affine.
  \item[{\rm(v)}] If $f:X\to Y$ and $g:Y\to Z$ are morphisms such that $g\circ f$ is quasi-affine, and if $g$ is separated or the underlying space of $X$ is locally Noetherian, then $f$ is quasi-affine.
  \item[{\rm(vi)}] If $f$ is a quasi-affine morphism, then so is $f_\red$.
\end{enumerate}
\end{proposition}

\begin{proof}
Taking into account the criterion \sref{II.5.1.6}[c$'$], all of (i), (iii), (iv), (v), and (vi) follow immediately from \sref{II.4.6.13}[i \emph{bis}], \sref{II.4.6.13}[iii], \sref{II.4.6.13}[iv], \sref{II.4.6.13}[v], and \sref{II.4.6.13}[vi] (respectively).
To prove (ii), we can restrict to the case where $Z$ is affine, and then the claim follows directly from applying \sref{II.4.6.13}[ii] to $\sh{L}=\sh{O}_X$ and $\sh{K}=\sh{O}_Y$.
\end{proof}

\begin{remark}[5.1.11]
\label{II.5.1.11}
Let $f:X\to Y$ and $g:Y\to Z$ be morphisms such that $X\times_Z Y$ is locally Noetherian.
Then the graph immersion $\Gamma_f:X\to X\times_Z Y$ is quasi-affine, since it is quasi-compact \sref[I]{I.6.3.5}, and since \sref[I]{I.5.5.12} shows that, in (v), the conclusion still holds true if we remove the hypothesis that $g$ is separated.
\end{remark}

\begin{proposition}[5.1.12]
\label{II.5.1.12}
Let $f:X\to Y$ be a quasi-compact morphism, and $g:X'\to X$ a quasi-affine morphism.
If $\sh{L}$ is an ample (for $f$) $\sh{O}_X$-module, then $g^*(\sh{L})$ is an ample (for $f\circ g$) $\sh{O}_{X'}$-module.
\end{proposition}

\begin{proof}
Since $\sh{O}_{X'}$ is very ample for $g$, and the question is local on $Y$ \sref{II.4.6.4}, it follows from \sref{II.4.6.13}[ii] that there exists (for $Y$ affine) an integer $n$ such that
\[
  g^*(\sh{L}^{\otimes n})=(g^*(\sh{L}))^{\otimes n}
\]
is ample for $f\circ g$, and so $g^*(\sh{L})$ is ample for $f\circ g$ \sref{II.4.6.9}
\end{proof}

\subsection{Serre's criterion}
\label{subsection:II.5.2}

\begin{theorem}[5.2.1]
\label{II.5.2.1}
\emph{(Serre's criterion).}
Let $X$ be a quasi-compact scheme or a prescheme whose underlying space is Noetherian.
The following conditions are equivalent.
\begin{enumerate}
  \item[{\rm(a)}] $X$ is an affine scheme.
  \item[{\rm(b)}] There exists a family of elements $f_\alpha\in A=\Gamma(X,\sh{O}_X)$ such that the $X_{f_\alpha}$ are affine, and such that the ideal generated by the $f_\alpha$ in $A$ is equal to $A$ itself.
  \item[{\rm(c)}] The functor $\Gamma(X,\sh{F})$ is exact in $\sh{F}$ on the category of quasi-coherent $\sh{O}_X$-modules, or, in other words, if
    \[
      0\to\sh{F}'\to\sh{F}\to\sh{F}''\to 0
      \tag{*}
    \]
    is an exact sequence of quasi-coherent $\sh{O}_X$-modules, then the sequence
    \[
     0\to\Gamma(X,\sh{F}')\to\Gamma(X,\sh{F})\to\Gamma(X,\sh{F}'')\to 0
    \]
    is also exact.
  \item[{\rm(c')}] Condition~{\rm(c)} holds for every exact sequence {\rm($*$)} of quasi-coherent $\sh{O}_X$-modules such that $\sh{F}$ is isomorphic to an $\sh{O}_X$-submodule of $\sh{O}_X^n$ for some finite $n$.
  \item[{\rm(d)}] $\HH^1(X,\sh{F})=0$ for every quasi-coherent $\sh{O}_X$-module $\sh{F}$.
  \item[{\rm(d$'$)}] $\HH^1(X,\sh{J})=0$ for every quasi-coherent sheaf of ideals $\sh{J}$ of $\sh{O}_X$.
\end{enumerate}
\end{theorem}

\begin{proof}
It is evident that (a) implies (b); furthermore, (b) implies that the $X_{f_\alpha}$ cover $X$, because, by hypothesis, the section $1$ is a linear combination of the $f_\alpha$, and the $D(f_\alpha)$ thus cover $\Spec(A)$.
The final claim of \sref{II.4.5.2} thus implies that $X\to\Spec(A)$ is an isomorphism.

We know tha (a) implies (c) \sref[I]{I.1.3.11}, and (c) trivially implies (c$'$).
We now prove that (c$'$) implies (b).
First of all, (c$'$) implies that, for every \emph{closed} point $x\in X$ and every open neighbourhood $U$ of $x$, there exists some $f\in A$ such that $x\in X_f\subset X\setmin U$.
Let $\sh{J}$ (resp. $\sh{J}'$) be the quasi-coherent sheaf of ideals of $\sh{O}_X$ defining the reduced closed subprescheme of $X$ that has $X\setmin U$ (resp. $(X\setmin U)\cup\{x\}$) as its underlying space \sref[I]{I.5.2.1};
it is clear that we have $\sh{J}'\subset\sh{J}$, and that $\sh{J}''=\sh{J}/\sh{J}'$ is a quasi-coherent $\sh{O}_X$ module that has support equal to $\{x\}$, and such that $\sh{J}_x''=\kres(x)$.
Hypothesis~(c$'$) applied to the exact sequence $0\to\sh{J}'\to\sh{J}\to\sh{J}''\to0$ shows that $\Gamma(X,\sh{J})\to\Gamma(X,\sh{J}'')$ is surjective.
The section of $\sh{J}''$ whose germ at $x$ is $1_x$ is thus the image of some section $f\in\Gamma(X,\sh{J})\subset\Gamma(X,\sh{O}_X)$, and we have, by definition, that $f(x)=1_x$ and $f(y)=0$ in $X\setmin U$, which establishes our claim.
Now, if $U$ is affine, then so is $X_f$ \sref[I]{I.1.3.6}, so the union of the $X_f$ that are affine ($f\in A$) is an open set $Z$ that contains \emph{all the closed points} of $X$;
since $X$ is a quasi-compact Kolmogoroff space, we necessarily have $Z=X$ \sref[0]{0.2.1.3}.
Because $X$ is quasi-compact, there are a \emph{finite} number of elements $f_i\in A$ ($1\leq i\leq n$) such that the $X_{f_i}$ are affine and cover $X$.
So consider the homomorphism $\sh{O}_X^n\to\sh{O}_X$ defined by the sections $f_i$ \sref[0]{0.5.1.1};
since, for all $x\in X$, at least one of the $(f_i)_x$ is invertible, this homomorphism is \emph{surjective}, and we thus have an exact sequence $0\to\sh{R}\to\sh{O}_X^n\to\sh{O}_X\to0$, where $\sh{R}$ is a quasi-coherent $\sh{O}_X$-submodule of $\sh{O}_X$.
It then follows
\oldpage[II]{98}
from (c$'$) that the corresponding homomorphism $\Gamma(X,\sh{O}_X^n)\to\Gamma(X,\sh{O}_X)$ is surjective, which proves (b).

Finally, (a) implies (d) \sref[I]{I.5.1.9.2}, and (d) trivially implies (d$'$).
It remains to show that (d$'$) implies (c$'$).
But if $\sh{F}'$ is a quasi-coherent $\sh{O}_X$-submodule of $\sh{O}_X^n$, then the filtration $0\subset\sh{O}_X\subset\sh{O}_X^2\subset\ldots\subset\sh{O}_X^n$ defines a filtration of $\sh{F}'$ given by the $\sh{F}'_k=\sh{F}\cap\sh{O}_X^k$ ($0\leq k\leq n$), which are quasi-coherent $\sh{O}_X$-modules \sref[I]{I.4.1.1}, and $\sh{F}'_{k+1}/\sh{F}'_k$ is isomorphic to a quasi-coherent $\sh{O}_X$-submodule of $\sh{O}_X^{k+1}/\sh{O}_X^k=\sh{O}_X$, which is to say, a quasi-coherent sheaf of ideals of $\sh{O}_X$.
Hypothesis~(d$'$) thus implies that $\HH^1(X,\sh{F}'_{k+1}/\sh{F}'_k)=0$;
the exact cohomology sequence $\HH^1(X,\sh{F}'_k)\to\HH^1(X,\sh{F}'_{k+1})\to\HH^1(X,\sh{F}'_{k+1}/\sh{F}'_k)=0$ then lets us prove by induction on $k$ that $H^1(X,\sh{F}'_k)=0$ for all $k$.
\end{proof}

\begin{remark}[5.2.1.1]
\label{II.5.2.1.1}
When $X$ is a \emph{Noetherian} prescheme, we can replace ``quasi-coherent'' by ``coherent'' in the statements of (c$'$) and (d$'$).
Indeed, in the proof of the fact that (c$'$) implies (b), $\sh{J}$ and $\sh{J}'$ are then \emph{coherent} sheaves of ideals, and, furthermore, every quasi-coherent submodule of a coherent module is coherent \sref[I]{I.6.1.1};
whence the conclusion.
\end{remark}

\begin{corollary}[5.2.2]
\label{II.5.2.2}
Let $f:X\to Y$ be a separated quasi-compact morphism.
The following conditions are equivalent.
\begin{enumerate}
  \item[{\rm(a)}] The morphism $f$ is an affine morphism.
  \item[{\rm(b)}] The functor $f_*$ is exact on the category of quasi-coherent $\sh{O}_X$-modules.
  \item[{\rm(c)}] For every quasi-coherent $\sh{O}_X$-module $\sh{F}$, we have $\RR^1 f_*(\sh{F})=0$.
  \item[{\rm(c')}] for every quasi-coherent sheaf of ideals $\sh{J}$ of $\sh{O}_X$, we have $\RR^1 f_*(\sh{J})=0$.
\end{enumerate}
\end{corollary}

\begin{proof}
All these conditions are local on $Y$, by definition of the functor $\RR^1 f_*$ (T,~3.7.3), and so we can assume that $Y$ is affine.
If $f$ is affine, then $X$ is affine, and property~(b) is nothing more than \sref[I]{I.1.6.4}.
Conversely, we now show that (b) implies (a):
for every quasi-coherent $\sh{O}_X$-module $\sh{F}$, we have that $f_*(\sh{F})$ is a quasi-coherent $\sh{O}_Y$-module \sref[I]{I.9.2.2}[a].
By hypothesis, the functor $f_*(\sh{F})$ is exact in $\sh{F}$, and the functor $\Gamma(Y,\sh{G})$ is exact in $\sh{G}$ (in the category of quasi-coherent $\sh{O}_Y$-modules) because $Y$ is affine \sref[I]{I.1.3.11};
so $\Gamma(Y,f_*(\sh{F}))=\Gamma(X,\sh{F})$ is exact in $\sh{F}$, which proves our claim, by \sref{II.5.2.1}[c].

If $f$ is affine, then $f^{-1}(U)$ is affine for every affine open subset $U$ of $Y$ \sref{II.1.3.2}, and so $\HH^1(f^{-1}(U),\sh{F})=0$ \sref{II.5.2.1}[d], which, by definition, implies that $\RR^1 f_*(\sh{F})=0$.
Finally, suppose that condition~(c$'$) is satisfied;
the exact sequence of terms of low degree in the Leray spectral sequence (G,~II,~4.17.1 and I,~4.5.1) give, in particular, the exact sequence
\[
  0\to\HH^1(Y,f_*(\sh{J}))\to\HH^1(X,\sh{J})\to\HH^0(Y,\RR^1 f_*(\sh{J})).
\]
Since $Y$ is affine, and $f_*(\sh{J})$ quasi-coherent \sref[I]{I.9.2.2}[a], we have that $\HH^1(Y,f_*(\sh{J}))=0$ \sref{II.5.2.1};
hypothesis~(c$'$) thus implies that $\HH^1(X,\sh{J})=0$, and we conclude, by \sref{II.5.2.1}, that $X$ is an affine scheme.
\end{proof}

\begin{corollary}[5.2.3]
\label{II.5.2.3}
If $f:X\to Y$ is an affine morphism, then, for every quasi-coherent $\sh{O}_X$-module $\sh{F}$, the canonical homomorphism $\HH^1(Y,f_*(\sh{F}))\to\HH^1(X,\sh{F})$ is bijective.
\end{corollary}

\oldpage[II]{99}
\begin{proof}
We have the exact sequence
\[
  0\to\HH^1(Y,f_*(\sh{F}))\to\HH^1(X,\sh{F})\to\HH^0(Y,\RR^1 f_*(\sh{F}))
\]
of terms of low degree in the Leray spectral sequence, and the conclusion follows from \sref{II.5.2.2}.
\end{proof}

\begin{remark}[5.2.4]
\label{II.5.2.4}
In Chapter~III,~\textsection1, we prove that, if $X$ is affine, then we have $\HH^i(X,\sh{F})=0$ for all $i>0$ and all quasi-coherent $\sh{O}_X$-modules $\sh{F}$.
\end{remark}

\subsection{Quasi-projective morphisms}
\label{subsection:II.5.3}

\begin{definition}[5.3.1]
\label{II.5.3.1}
We say that a morphism $f:X\to Y$ is \emph{quasi-projective}, or that $X$ (considered as a $Y$-prescheme via $f$) is \emph{quasi-projective over $Y$}, or that $X$ is a \emph{quasi-projective $Y$-scheme}, if $f$ is of finite type and there exists an invertible $f$-ample $\sh{O}_X$-module.
\end{definition}

We note that this notion \emph{is not local on $Y$}:
the counterexamples of Nagata~\cite{II-26} and Hironaka show that, even if $X$ and $Y$ are non-singular algebraic schemes over an algebraically closed field, every point of $Y$ can have an affine neighbourhood $U$ such that $f^{-1}(U)$ is quasi-projective over $U$, without $f$ being quasi-projective.

We note that a quasi-projective morphism is necessarily \emph{separated} \sref{II.4.6.1}.
When $Y$ is quasi-compact, it is equivalent to say either that $f$ is quasi-projective, or that $f$ is of finite type and there exists a \emph{very ample} (relative to $f$) $\sh{O}_X$-module (\sref{II.4.6.2} and \sref{II.4.6.11}).
Further:

\begin{proposition}[5.3.2]
\label{II.5.3.2}
Let $Y$ be a quasi-compact scheme or a prescheme whose underlying space is Noetherian, and let $X$ be a $Y$-prescheme.
The following conditions are equivalent.
\begin{enumerate}
  \item[{\rm(a)}] $X$ is a quasi-projective $Y$-scheme.
  \item[{\rm(b)}] $X$ is of finite type over $Y$, and there exists some quasi-coherent $\sh{O}_Y$-module $\sh{E}$ of finite type such that $X$ is $Y$-isomorphic to a subprescheme of $\bb{P}(\sh{E})$.
  \item[{\rm(c)}] $X$ is of finite type over $Y$, and there exists some quasi-coherent graded $\sh{O}_Y$-algebra $\sh{S}$ such that $\sh{S}_1$ is of finite type and generates $\sh{S}$, and such that $X$ is $Y$-isomorphic to a induced subprescheme on some everywhere-dense open subset of $\Proj(\sh{S})$.
\end{enumerate}
\end{proposition}

\begin{proof}
This follows immediately from the previous remark and from \sref{II.4.4.3}, \sref{II.4.4.6}, and \sref{II.4.4.7}.
\end{proof}

We note that, whenever $Y$ is a \emph{Noetherian} prescheme, we can, in conditions~(b) and (c) of \sref{II.5.3.2}, remove the hypothesis that $X$ is of finite type over $Y$, since this automatically satisfied \sref[I]{I.6.3.5}.

\begin{corollary}[5.3.3]
\label{II.5.3.3}
Let $Y$ be a quasi-compact scheme such that there exists an ample $\sh{O}_Y$-module $\sh{L}$ \sref{II.4.5.3}.
For a $Y$-scheme $X$ to be quasi-projective, it is necessary and sufficient for it to be of finite type over $Y$ and also isomorphic to a $Y$-subscheme of a projective bundle of the form $\bb{P}_Y^r$.
\end{corollary}

\begin{proof}
If $\sh{E}$ is a quasi-coherent $\sh{O}_Y$-module of finite type, then $\sh{E}$ is isomorphic to a quotient of an $\sh{O}_Y$-module $\sh{L}^{\otimes(-n)}\otimes_{\sh{O}_Y}\sh{O}_Y^k$ \sref{II.4.5.5}, and so $\bb{P}(\sh{E})$ is isomorphic to a closed subscheme of $\bb{P}_Y^{k-1}$ (\sref{II.4.1.2} and \sref{II.4.1.4}).
\end{proof}

\begin{proposition}[5.3.4]
\label{II.5.3.4}
\medskip\noindent
\begin{enumerate}
  \item[{\rm(i)}] A quasi-affine morphism of finite type (and, in particular, a quasi-compact immersion, or an affine morphism of finite type) is quasi-projective.
  \item[{\rm(ii)}] If $f:X\to Y$ and $g:Y\to Z$ are quasi-projective, and if $Z$ is quasi-compact, then $g\circ f$ is quasi-projective.
\oldpage[II]{100}
  \item[{\rm(iii)}] If $f:X\to Y$ is a quasi-projective $S$-morphism, then $f_{(S')}:X_{(S')}\to Y_{(S')}$ is quasi-projective for every extension $S'\to S$ of the base prescheme.
  \item[{\rm(iv)}] If $f:X\to Y$ and $g:X'\to Y'$ are quasi-projective $S$-morphisms, then $f\times_S g$ is quasi-projective.
  \item[{\rm(v)}] If $f:X\to Y$ and $g:Y\to Z$ are morphisms such that $g\circ f$ is quasi-projective, and if $g$ is separated or $X$ locally Noetherian, then $f$ is quasi-projective.
  \item[{\rm(vi)}] If $f$ is a quasi-projective morphism, then so is $f_\red$.
\end{enumerate}
\end{proposition}

\begin{proof}
(i) follows from \sref{II.5.1.6} and \sref{II.5.1.10}[i].
The other claims are immediate consequences of Definition~\sref{II.5.3.1}, of the properties of morphisms of finite type \sref[I]{I.6.3.4}, and of \sref{II.4.6.13}.
\end{proof}

\begin{remark}[5.3.5]
\label{II.5.3.5}
We note that we can have $f_\red$ being quasi-projective without $f$ being quasi-projective, even if we assume that $Y$ is the spectrum of an algebra of finite rank over $\bb{C}$ and that $f$ is proper.
\end{remark}

\begin{corollary}[5.3.6]
\label{II.5.3.6}
If $X$ and $X'$ are quasi-projective $Y$-schemes, then $X\sqcup X'$ is a quasi-projective $Y$-scheme.
\end{corollary}

\begin{proof}
This follows from \sref{II.4.6.18}.
\end{proof}

\subsection{Proper morphisms and universally closed morphisms}
\label{subsection:II.5.4}

\begin{definition}[5.4.1]
\label{II.5.4.1}
We say that a morphism of preschemes $f:X\to Y$ is \emph{proper} if it satisfies the following two conditions:
\begin{enumerate}
  \item[(a)] $f$ is separated and of finite type; and
  \item[(b)] for every prescheme $Y'$ and every morphism $Y'\to Y$, the projection $f_{(Y')}:X\times_Y Y'\to Y'$ is a closed morphism \sref[I]{I.2.2.6}.
\end{enumerate}

When this is the case, we also say that $X$ (considered as a $Y$-prescheme with structure morphism $f$) is proper over $Y$.
\end{definition}

It is immediate that conditions~(a) and (b) are \emph{local} on $Y$.
To show that the image of a closed subset $Z$ of $X\times_Y Y'$ under the projection $q:X\times_Y Y'\to Y'$ is closed in $Y$, it suffices to see that $q(Z)\cap U'$ is closed in $U'$ for every affine open subset $U'$ of $Y'$;
since $q(Z)\cap U'=q(Z\cap q^{-1}(U'))$, and since $q^{-1}(U')$ can be identified with $X\times_Y U'$ \sref[I]{I.4.4.1}, we see that to satisfy condition~(b) of Definition~\sref{II.5.4.1}, we can restrict to the case where $Y$ is an \emph{affine} scheme.
We further see \sref{II.5.3.6} that, if $Y$ is locally Noetherian, then we can even restrict to proving (b) in the case where $Y'$ is of finite type over $Y$.

It is clear that every proper morphism is \emph{closed}.

\begin{proposition}[5.4.2]
\label{II.5.4.2}
\medskip\noindent
\begin{enumerate}
  \item[{\rm(i)}] A closed immersion is a proper morphism.
  \item[{\rm(ii)}] The composition of two proper morphisms is proper.
  \item[{\rm(iii)}] If $X$ and $Y$ are $S$-preschemes, and $f:X\to Y$ a proper $S$-morphism, then $f_{(S')}:X_{(S')}\to Y_{(S')}$ is proper for every extension $S'\to S$ of the base prescheme.
  \item[{\rm(iv)}] If $f:X\to Y$ and $g:X'\to Y'$ are proper $S$-morphisms, then $f\times_S g:X\times_S Y\to X'\times_S Y'$ is a proper $S$-morphism.
\end{enumerate}
\end{proposition}

\oldpage[II]{101}
\begin{proof}
It suffices to prove (i), (ii), and (iii) \sref[I]{I.3.5.1}.
In each of the three cases, verifying condition~(a) of \sref{II.5.4.1} follows from previous results (\sref[I]{I.5.5.1} and \sref{II.6.4.3}); it remains to verify condition~(b).
It is immediate in case (i), because if $X\to Y$ is a closed immersion, then so is $X\times_Y Y'\to Y\times_Y Y'=Y'$ (\sref[I]{I.4.3.2} and \sref{II.3.3.3}).
To prove (ii), consider two proper morphisms $X\to Y$ and $Y\to Z$, and a morphism $Z'\to Z$.
We can write $X\times_Z Z'=X\times_Y(Y\times_Z Z')$ \sref[I]{I.3.3.9.1}, and so the projection $X\times_Z Z'\to Z'$ factors as $X\times_Y(Y\times_Z Z')\to Y\times_Z Z'\to Z'$.
Taking the initial remark into account, (ii) follows from the fact that the composition of two closed morphisms is closed.
Finally, for every morphism $S'\to S$, we can identify $X_{(S')}$ with $X\times_Y Y_{(S')}$ \sref[I]{I.3.3.11}; for every morphism $Z\to Y_{(S')}$, we can write
\[
  X_{(S')}\times_{Y_{(S')}}Z=(X\times_Y Y_{(S')})\times_{Y_{(S')}}Z=X\times_Y Z;
\]
since by hypothesis $X\times_Y Z\to Z$ is closed, this proves (iii).
\end{proof}

\begin{corollary}[5.4.3]
\label{II.5.4.3}
Let $f:X\to Y$ and $g:Y\to Z$ be morphisms such that $g\circ f$ is proper.
\begin{enumerate}
  \item[{\rm(i)}] If $g$ is separated, then $f$ is proper.
  \item[{\rm(ii)}] If $g$ is separated and of finite type, and if $f$ is surjective, then $g$ is proper.
\end{enumerate}
\end{corollary}

\begin{proof}
(i) follows from \sref{II.5.4.2} by the general procedure \sref[I]{I.5.5.12}.
To prove (ii), we need only verify that condition~(b) of Definition~\sref{II.5.4.1} is satisfied.
For every morphism $Z'\to Z$, the diagram
\[
  \xymatrix{
    X\times_Z Z'\ar[r]^{f\times1_{Z'}}\ar[dr]_p &
    Y\times_Z Z'\ar[d]^{p'}\\
    & Z'
  }
\]
(where $p$ and $p'$ are the projections) commutes \sref[I]{I.3.2.1};
furthermore, $f\times1_{Z'}$ is surjective because $f$ is surjective \sref[I]{I.3.5.2}, and $p$ is a closed morphism by hypothesis.
Every closed subset $F$ of $Y\times_Z Z'$ is thus the image under $f\times1_{Z'}$ of some closed subset $E$ of $X\times_Z Z'$, so $p'(F)=p(E)$ is closed in $Z'$ by hypothesis, whence the corollary.
\end{proof}

\begin{corollary}[5.4.4]
\label{II.5.4.4}
If $X$ is a proper prescheme over $Y$, and $\sh{S}$ a quasi-coherent $\sh{O}_Y$-algebra, then every $Y$-morphism $f:X\to\Proj(\sh{S})$ is proper (and \emph{a fortiori} closed).
\end{corollary}

\begin{proof}
The structure morphism $p:\Proj(\sh{S})\to Y$ is separated, and $p\circ f$ is proper by hypothesis.
\end{proof}

\begin{corollary}[5.4.5]
\label{II.5.4.5}
Let $f:X\to Y$ be a separated morphism of finite type.
Let $(X_i)_{1\leq i\leq n}$ (resp. $(Y_i)_{1\leq i\leq n}$) be a finite family of closed subpreschemes of $X$ (resp. $Y$), and $j_i$ (resp. $h_i$) the canonical injection $X_i\to X$ (resp. $Y_i\to Y$).
Suppose that the underlying space of $X$ is the union of the $X_i$, and that, for all $i$, there is a morphism $f_i:X_i\to Y_i$, such that the diagram
\[
  \xymatrix{
    X_i\ar[r]^{f_i}\ar[d]_{j_i} &
    Y_i\ar[d]^{h_i}\\
    X\ar[r]^f &
    Y
  }
\]
commutes.
Then, for $f$ to be proper, it is necessary and sufficient for all of the $f_i$ to be proper.
\end{corollary}

\oldpage[II]{102}
\begin{proof}
If $f$ is proper, then so is $f\circ j_i$, because $j_i$ is a closed immersion \sref{II.5.4.2};
since $h_i$ is a closed immersion, and thus a separated morphism, $f_i$ is proper, by \sref{II.5.4.3}.
Conversely, suppose that all of the $f_i$ are proper, and consider the prescheme $Z$ given by the \emph{sum} of the $X_i$; let $u$ be the morphism $Z\to X$ which reduces to $j_i$ on each $X_i$.
The restriction of $f\circ u$ to each $X_i$ is equal to $f\circ j_i=h_i\circ f_i$, and is thus proper, because both the $h_i$ and the $f_i$ are \sref{II.5.4.2};
it then follows immediately from Definition~\sref{II.5.4.1} that $u$ is proper.
But since by hypothesis $u$ is surjective, we conclude that $f$ is proper by \sref{II.5.4.3}.
\end{proof}

\begin{corollary}[5.4.6]
\label{II.5.4.6}
Let $f:X\to Y$ be a separated morphism of finite type; for $f$ to be proper, it is necessary and sufficient for $f_\red:X_\red\to Y_\red$ to be proper.
\end{corollary}

\begin{proof}
This is a particular case of \sref{II.5.4.5}, with $n=1$, $X_1=X_\red$, and $Y_1=Y_\red$ \sref[I]{I.5.1.5}.
\end{proof}

\begin{env}[5.4.7]
\label{II.5.4.7}
If $X$ and $Y$ are Noetherian preschemes, and $f:X\to Y$ a separated morphism of finite type, then we can, to show that $f$ is proper, restrict to the the case of \emph{dominant} morphisms and \emph{integral} preschemes.
Indeed, let $X_i$ ($1\leq i\leq n$) be the (finitely many) irreducible components of $X$, and consider, for each $i$, the unique reduced closed subprescheme of $X$ that has $X_i$ as its underlying space, which we again denote by $X_i$ \sref[I]{I.5.2.1}.
Let $Y_i$ be the unique reduced closed subprescheme of $Y$ that has $\overline{f(X_i)}$ as its underlying space.
If $g_i$ (resp. $h_i$) is the injection morphism $X_i\to X$ (resp. $Y_i\to Y$), then we conclude that $f\circ g_i=h_i\circ f_i$, where $f_i$ is a dominant morphism $X_i\to Y_i$ \sref[I]{I.5.2.2};
we are then under the right conditions to apply \sref{II.5.4.5}, and for $f$ to be proper, it is necessary and sufficient for all the $f_i$ to be proper.
\end{env}

\begin{corollary}[5.4.8]
\label{II.5.4.8}
Let $X$ and $Y$ be separated $S$-preschemes of finite type over $S$, and $f:X\to Y$ an $S$-morphism.
For $f$ to be proper, it is necessary and sufficient that, for every $S$-prescheme $S'$, the morphism $f\times_S 1_{S'}:X\times_S S'\to Y\times_S S'$ be closed.
\end{corollary}

\begin{proof}
First note that, if $g:X\to S$ and $h:Y\to S$ are the structure morphisms, then we have, by definition, $g=h\circ f$, and so $f$ is separated and of finite type (\sref[I]{I.5.5.1} and \sref{II.6.3.4}).
If $f$ is proper, then so is $f\times_S 1_{S'}$ \sref{II.5.4.2}; \emph{a fortiori}, $f\times_S 1_{S'}$ is closed.
Conversely, suppose that the conditions of the statement are satisfied, and let $Y'$ be a $Y$-prescheme;
$Y'$ can also be considered as an $S$-prescheme, and since $Y\to S$ is separated, $X\times_Y Y'$ can be identified with a closed subprescheme of $X\times_S Y'$ \sref[I]{I.5.4.2}.
In the commutative diagram
\[
  \xymatrix{
    X\times_Y Y'\ar[r]^{f\times1_{Y'}}\ar[d] &
    Y\times_Y Y'=Y'\ar[d]\\
    X\times_S Y'\ar[r]^{f\times1_{S'}} &
    Y\times_S Y',
  }
\]
the vertical arrows are closed immersions; it thus immediately follows that if $f\times1_{S'}$ is a closed morphism, then so is $f\times1_{Y'}$
\end{proof}

\begin{remark}[5.4.9]
\label{II.5.4.9}
We say that a morphism $f:X\to Y$ is \emph{universally closed} if it satisfies condition~(b) of Definition~\sref{II.5.4.1}.
The reader will observe that,
\oldpage[II]{103}
in \sref{II.5.4.2} to \sref{II.5.4.8}, we can replace every occurrence of ``proper'' with ``universally closed'' without changing the validity of the results (and in the hypotheses of \sref{II.5.4.3}, \sref{II.5.4.5}, \sref{II.5.4.6}, and \sref{II.5.4.8}, we can omit the finiteness conditions).
\end{remark}

\begin{env}[5.4.10]
\label{II.5.4.10}
Let $f:X\to Y$ be a morphism of finite type.
We say that a closed subset $Z$ of $X$ is \emph{proper on $Y$} (or \emph{$Y$-proper}, or \emph{proper for $f$}) if the restriction of $f$ to a closed subprescheme of $X$, with underlying space $Z$ \sref[1]{1.5.2.1}, is \emph{proper}.
Since this restriction is then separated, it follows from \sref{II.5.4.6} and \sref[I]{I.5.5.1}[vi] that the preceding property \emph{does not depend} on the closed subprescheme of $X$ that has $Z$ as its underlying space.
If $g:X'\to X$ is a \emph{proper} morphism, then $g^{-1}(Z)$ is a \emph{proper} subset of $X'$:
if $T$ is a subprescheme of $X$ that has $Z$ as its underlying space, it suffices to note that the restriction of $g$ to the closed subprescheme $g^{-1}(T)$ of $X'$ is a proper morphism $g^{-1}(T)\to T$, by \sref{II.5.4.2}[iii], and to then apply \sref{II.5.4.2}[ii].
Further, if $X''$ is a $Y$-\emph{scheme} of finite type, and $u:X\to X''$ a $Y$-morphism, then $u(Z)$ is a \emph{proper} subset of $X''$;
indeed, let us take $T$ to be the reduced closed subprescheme of $X$ having $Z$ as its underlying space;
then the restriction of $f$ to $T$ is proper, and thus so is the restriction of $u$ to $T$ \sref{II.5.4.3}[i], thus $u(Z)$ is closed in $X''$;
let $T''$ be a closed subprescheme of $X''$ having $u(Z)$ as its underlying space \sref[I]{I.5.2.1}, such that $u|T$ factors as $T\xrightarrow{v}T''\xrightarrow{j}X''$, where $j$ is the canonical injection \sref[I]{I.5.2.2}, and $v$ is thus proper and surjective \sref{II.5.4.5};
if $g$ is the restriction to $T''$ of the structure morphism $X''\to Y$, then $g$ is separated and of finite type, and we have that $f|T=g\circ v$;
it thus follows from \sref{II.5.4.3}[ii] that $g$ is proper, whence our assertion.
\end{env}

It follows, in particular, from these remarks that, if $Z$ is a $Y$-proper subset of $X$, then
\begin{enumerate}
  \item for every closed subprescheme $X'$ of $X$, $Z\cap X'$ is a $Y$-proper subset of $X'$; and
  \item if $X$ is a subprescheme of a $Y$-scheme of finite type $X''$, then $Z$ is also a $Y$-proper subset of $X''$ (and so, in particular, is \emph{closed in $X''$}).
\end{enumerate}

\subsection{Projective morphisms}
\label{subsection:II.5.5}

\begin{proposition}[5.5.1]
\label{II.5.5.1}
Let $X$ be a $Y$-prescheme.
The following conditions are equivalent.
\begin{enumerate}
  \item[{\rm(a)}] $X$ is $Y$-isomorphic to a \emph{closed} subprescheme of a projective bundle $\bb{P}(\sh{E})$, where $\sh{E}$ is a quasi-coherent $\sh{O}_Y$-module of finite type.
  \item[{\rm(b)}] There exists a quasi-coherent graded $\sh{O}_Y$-algebra $\sh{S}$ such that $\sh{S}_1$ is of finite type and generates $\sh{S}$, and such that $X$ is $Y$-isomorphic to $\Proj(\sh{S})$.
\end{enumerate}
\end{proposition}

\begin{proof}
Condition~(a) implies (b), by \sref{II.3.6.2}[ii]: if $\sh{J}$ is a quasi-coherent graded sheaf of ideals of $\bb{S}(\sh{E})$, then the quasi-coherent graded $\sh{O}_Y$-algebra $\sh{S}=\bb{S}(\sh{E})/\sh{J}$ is generated by $\sh{S}_1$, and $\sh{S}_1$, the canonical image of $\sh{E}$, is an $\sh{O}_Y$-module of finite type.
Condition~(b) implies (a) by \sref{II.3.6.2} applied to the case where $\sh{M}\to\sh{S}_1$ is the identity map.
\end{proof}

\oldpage[II]{104}
\begin{definition}[5.5.2]
\label{II.5.5.2}
We say that a $Y$-prescheme $X$ is \emph{projective} on $Y$, or is a projective $Y$-scheme, if it satisfies either of the (equivalent) conditions~(a) and (b) of \sref{II.5.5.1}.
We say that a morphism $f:X\to Y$ is projective if it makes $X$ a projective $Y$-scheme.
\end{definition}

It is clear that if $f:X\to Y$ is projective, then there exists a \emph{very ample} (relative to $f$) $\sh{O}_X$-module \sref{II.4.4.2}.
\begin{theorem}[5.5.3]
\label{II.5.5.3}
\medskip\noindent
\begin{enumerate}
  \item[{\rm(i)}] Every projective morphism is quasi-projective and proper.
  \item[{\rm(ii)}] Conversely, let $Y$ be a quasi-compact scheme or a prescheme whose underlying space is Noetherian; then every morphism $f:X\to Y$ that is quasi-projective and proper is projective.
\end{enumerate}
\end{theorem}

\begin{proof}
\medskip\noindent
\begin{enumerate}
  \item[(i)] It is clear that if $f:X\to Y$ is projective, then it is of finite type and quasi-projective (thus, in particular, separated); furthermore, it follows immediately from \sref{II.5.5.1}[b] and \sref{II.3.5.3} that if $f$ is projective, then so is $f\times_Y 1_{Y'}:X\times_Y Y'\to Y'$ for every morphism $Y'\to Y$.
     To show that $f$ is universally closed, it is thus enough to show that a projective morphism $f$ is \emph{closed}.
     Since the question is local on $Y$, we can suppose that $Y=\Spec(A)$, thus \sref{II.5.5.1} $X=\Proj(S)$, where $S$ is a graded $A$-algebra generated by a finite number of elements of $S_1$.
     For all $y\in Y$, the fibre $f^{-1}(y)$ can be identified with $\Proj(S)\times_Y\Spec(\kres(y))$ \sref[I]{I.3.6.1}, and so also with $\Proj(S\otimes_A\kres(y))$ \sref{II.2.8.10};
     so $f^{-1}(y)$ is empty if and only if $S\otimes_A\kres(y)$ satisfies condition~(TN) \sref{II.2.7.4}, or, in other words, if $S_n\otimes_A\kres(y)=0$ for sufficiently large $n$.
     But since $(S_n)_y$ is an $\sh{O}_y$-module of finite type, the preceding condition implies that $(S_n)_y=0$ for sufficiently large $N$, by Nakayama's lemma.
     If $\mathfrak{a}_n$ is the annihilator in $A$ of the $A$-module $S_n$, then the preceding condition also implies that $\mathfrak{a}_n\subset\mathfrak{j}_n$ for sufficiently large $n$ \sref[0]{0.1.7.4}.
     But since $S_nS_1=S_{n+1}$, by hypothesis, we have that $\mathfrak{a}_n\subset\mathfrak{a}_{n+1}$, and if $\mathfrak{a}$ is the union of the $\mathfrak{a}_n$, then we see that $f(X)=V(\mathfrak{a})$, which proves that $f(X)$ is closed in $Y$.
     If now $X'$ is an arbitrary closed subset of $X$, then there exists a closed subprescheme of $X$ that has $X'$ as its underlying space \sref[I]{I.5.2.1}, and it is clear \sref{II.5.5.1}[a] that the morphism $X'\to X\xrightarrow{f}Y$ is projective, and so $f(X')$ is closed in $Y$.
   \item[(ii)] The hypothesis on $Y$ and the fact that $f$ is quasi-projective implies the existence of a quasi-coherent $\sh{O}_Y$-module $\sh{E}$ of finite type, as well as a $Y$-immersion $j:X\to\bb{P}(\sh{E})$ \sref{II.5.3.2}.
      But since $f$ is proper, $j$ is \emph{closed}, by \sref{II.5.4.4}, and so $f$ is projective.
\end{enumerate}
\end{proof}

\begin{remark}[5.5.4]
\label{II.5.5.4}
\medskip\noindent
\begin{enumerate}
  \item[(i)] Let $f:X\to Y$ be a morphism such that $f$ is proper, such that there exists a \emph{very ample} (relative to $f$) $\sh{O}_X$-module $\sh{L}$, and such that the quasi-coherent $\sh{O}_Y$-module $\sh{E}=f_*(\sh{L})$ is \emph{of finite type}.
    Then $f$ is a \emph{projective} morphism: indeed \sref{II.4.4.4}, there is then a $Y$-immersion $r:X\to\bb{P}(\sh{E})$, and, since $f$ is proper, $r$ is a \emph{closed} immersion \sref{II.5.4.4}.
    We will see in Chapter~III, \textsection3, that when $Y$ is \emph{locally Noetherian}, the third condition above ($\sh{E}$ being of finite type) is a consequence of the first two, and so the first two conditions \emph{characterise}, in this case, the projective morphisms, and if $Y$ is quasi-compact, then we can replace the second condition (the existence of a very ample (relative to $f$) $\sh{O}_X$-module $\sh{L}$) by the hypothesis that there exists an \emph{ample} (relative to $f$) $\sh{O}_X$-module \sref{II.4.6.11}.
  \item[(ii)] Let $Y$ be a quasi-compact scheme such that there exists an ample $\sh{O}_Y$-module.
    For a $Y$-scheme $X$ to be \emph{projective}, it is necessary and sufficient for it to be $Y$-isomorphic to a \emph{closed} $Y$-subscheme of a projective bundle of the form $\bb{P}_Y^r$.
    The condition is clearly sufficient.
\oldpage[II]{105}
    Conversely, if $X$ is projective over $Y$, then it is quasi-projective, and so there exists a $Y$-immersion $j$ of $X$ into some $\bb{P}_Y^r$ \sref{II.5.3.3} that is \emph{closed}, by \sref{II.5.4.4} and \sref{II.5.5.3}.
  \item[(iii)] The argument of \sref{II.5.5.3} shows that, for every prescheme $Y$ and every integer $r\geq0$, the structure morphism $\bb{P}_Y^r\to Y$ is \emph{surjective}, because if we set $\sh{S}=\bb{S}_{\sh{O}_Y}(\sh{O}_Y^{r+1})$, then we evidently have $\sh{S}_y=\bb{S}_{\kres(y)}(\kres(y)^{r+1})$ \sref{II.1.7.3}, and so $(\sh{S}_n)_y\neq0$ for any $y\in Y$ or any $n\geq0$.
  \item[(iv)] It follows from the examples of Nagata~\cite{II-26} that there exist proper morphisms that are not quasi-projective.
\end{enumerate}
\end{remark}

\begin{proposition}[5.5.5]
\label{II.5.5.5}
\medskip\noindent
\begin{enumerate}
  \item[{\rm(i)}] A closed immersion is a projective morphism.
  \item[{\rm(ii)}] If $f:X\to Y$ and $g:Y\to Z$ are projective morphisms, and if $Z$ is a quasi-compact scheme or a prescheme whose underlying space is Noetherian, then $g\circ f$ is projective.
  \item[{\rm(iii)}] If $f:X\to Y$ is a projective $S$-morphism, then $f_{(S')}:X_{(S')}\to Y_{(S')}$ is projective for every extension $S'\to S$ of the base prescheme.
  \item[{\rm(iv)}] If $f:X\to Y$ and $g:X'\to Y'$ are projective $S$-morphisms, then so is $f\times_S g$.
  \item[{\rm(v)}] If $g\circ f$ is a projective morphism, and if $g$ is separated, then $f$ is projective.
  \item[{\rm(vi)}] If $f$ is projective, then so is $f_\red$.
\end{enumerate}
\end{proposition}

\begin{proof}
(i) follows immediately from \sref{II.3.1.7}.
We have to show (iii) and (iv) separately, because of the restriction introduced on $Z$ in (ii) (cf.~\sref[I]{I.3.5.1}).
To show (iii), we restrict to the case where $S=Y$ \sref[I]{I.3.3.11}, and the claim then immediately follows from \sref{II.5.5.1}[b] and \sref{II.3.5.3}.
To show (iv), we are immediately led to the case where $X=\bb{P}(\sh{E})$ and $X=\bb{P}(\sh{E}')$, where $\sh{E}$ (resp. $\sh{E}'$) is a quasi-coherent $\sh{O}_Y$-module (resp. quasi-coherent $\sh{O}_{Y'}$-module) of finite type.
Let $p$ and $p'$ be the canonical projections of $T=Y\times_S Y'$ to $Y$ and $Y'$ (respectively); by \sref{II.4.1.3.1}, we have $\bb{P}(p^*(\sh{E})) = \bb{P}(\sh{E})\times_Y T$ and $\bb{P}(p^{\prime *}(\sh{E}'))=\bb{P}(\sh{E}')\times_{Y'}T$; whence
\begin{align*}
  \bb{P}(p^*(\sh{E}))\times_T\bb{P}(p^{\prime *}(\sh{E}'))&=(\bb{P}(\sh{E})\times_Y T)\times_T(T\times_{Y'}\bb{P}(\sh{E}'))\\
                                              &=\bb{P}(\sh{E})\times_Y(T\times_{Y'}\bb{P}(\sh{E}'))=\bb{P}(\sh{E})\times_S\bb{P}(\sh{E}')
\end{align*}
by replacing $T$ with $Y\times_S Y'$, and using \sref[I]{I.3.3.9.1}.
But $p^*(\sh{E})$ and $p^{\prime *}(\sh{E}')$ are of finite type over $T$ \sref[0]{0.5.2.4}, and thus so is $p^*(\sh{E})\otimes_{\sh{O}_T}p^{\prime *}(\sh{E}')$;
since $\bb{P}(p^*(\sh{E}))\times_T\bb{P}(p^{\prime *}(\sh{E}'))$ can be identified with a closed subprescheme of $p^*(\sh{E})\otimes_{\sh{O}_T}p^{\prime *}(\sh{E}')$ \sref{II.4.3.3}, this proves (iv).
To show (v) and (vi), we can apply \sref[I]{I.5.5.13}, because every closed subprescheme of a projective $Y$-scheme is a projective $Y$-scheme, by \sref{II.5.5.1}[a].

It remains to prove (ii); by the hypothesis on $Z$, this follows from \sref{II.5.5.3}, \sref{II.5.3.4}[ii], and \sref{II.5.4.2}[ii].
\end{proof}

\begin{proposition}[5.5.6]
\label{II.5.5.6}
If $X$ and $X'$ are projective $Y$-schemes, then $X\sqcup X'$ is a projective $Y$-scheme.
\end{proposition}

\begin{proof}
This is an evident consequence of \sref{II.5.5.2} and \sref{II.4.3.6}.
\end{proof}

\begin{proposition}[5.5.7]
\label{II.5.5.7}
Let $X$ be a projective $Y$-scheme, and $\sh{L}$ a $Y$-ample $\sh{O}_X$-module; then, for every section $f$ of $\sh{L}$ over $X$, $X_f$ is affine over $Y$.
\end{proposition}

\oldpage[II]{106}
\begin{proof}
Since the question is local on $Y$, we can assume that $Y=\Spec(A)$; furthermore, $X_{f^{\otimes n}}=X_f$, so by replacing $\sh{L}$ with some suitable $\sh{L}^{\otimes n}$, we can assume that $\sh{L}$ is very ample relative to the structure morphism $q:X\to Y$ \sref{II.4.6.11}.
The canonical homomorphism $\sigma:q^*(q_*(\sh{L}))\to\sh{L}$ is thus surjective, and the corresponding morphism
\[
  r=r_{\sh{L},\sigma}:X\to P=\bb{P}(q_*(\sh{L}))
\]
is an immersion such that $\sh{L}=r^*(\sh{O}_P(1))$ \sref{II.4.4.4}; furthermore, since $X$ is proper over $Y$, the immersion $r$ is closed \sref{II.5.4.4}.
But by definition, $f\in\Gamma(Y,q_*(\sh{L}))$, and $\sigma^\flat$ is the identity of $q_*(\sh{L})$; it then follows from Equation~\sref{II.3.7.3.1} that we have $X_f=r^{-1}(D_+(f))$;
so $X_f$ is a closed subprescheme of the affine scheme $D_+(f)$, and is thus also an affine scheme.
\end{proof}

In the particular case where $Y=X$, we obtain (taking \sref{II.4.6.13}[i] into account) the following corollary, whose direct proof is immediate anyway:
\begin{corollary}[5.5.8]
\label{II.5.5.8}
Let $X$ be a prescheme, and $\sh{L}$ an invertible $\sh{O}_X$-module.
For every section $f$ of $\sh{L}$ over $X$, $X_f$ is affine over $X$ (and thus also an affine scheme whenever $X$ is an affine scheme).
\end{corollary}

\subsection{Chow's lemma}
\label{subsection:II.5.6}

\begin{theorem}[5.6.1]
\label{II.5.6.1}
\emph{(Chow's lemma)}.
Let $S$ be a prescheme, and $X$ an $S$-scheme of finite type.
Suppose that the following conditions are satisfied:
\begin{enumerate}
  \item[{\rm(a)}] $S$ is Noetherian;
  \item[{\rm(b)}] $S$ is a quasi-compact scheme, and $X$ has a finite number of irreducible components.
\end{enumerate}
Under these hypotheses,
\begin{enumerate}
  \item[{\rm(i)}] there exists a \emph{quasi-projective} $S$-scheme $X'$, and an $S$-morphism $f:X'\to X$ that is both\emph{projective} and \emph{surjective};
  \item[{\rm(ii)}] we can take $X'$ and $f$ to be such that there exists an open subset $U\subset X$ for which $U'=f^{-1}(U)$ is dense in $X'$, and for which the restriction of $f$ to $U'$ is an isomorphism $U'\isoto U$; and
  \item[{\rm(iii)}] if $X$ is reduced (resp. irreducible, integral), then we can assume that $X'$ is reduced (resp. irreducible, integral).
\end{enumerate}
\end{theorem}

\begin{proof}
The proof proceeds in multiple steps.
\begin{enumerate}
  \item[(A)] We can first restrict to the case where $X$ is \emph{irreducible}.
    Indeed, in hypothesis~(a), $X$ is Noetherian, and so, in the two hypotheses, the irreducible components $X_i$ of $X$ are finite in number.
    If the theorem is shown to be true for each of the reduced closed preschemes of $X$ having the $X_i$ as their underlying spaces, and if $X'_i$ and $f_i:X'_i\to X_i$ are the prescheme and the morphism corresponding to $X_i$ (respectively), then the prescheme $X'$ given by the \emph{sum} of the $X'_i$, and the morphism $f:X'\to X$ whose restriction to each $X'_i$ is $j_i\circ f_i$ (where $j_i$ is the canonical injection $X_i\to X$) satisfy the conclusion of the theorem.
    It is immediate that $X'$ is reduced if all of the $X'_i$ are; furthermore, we can satisfy (ii) by taking $U$ to be the union of the sets $U_i\cap\complement\left(\bigcup_{j\neq i}X_j\right)$.
    Finally, since the $X'_i$ are quasi-projective over $S$, so is $X'$
\oldpage[II]{107}
    \sref{II.5.3.6}; similarly, the morphisms $X'_i\to X$ are projective by \sref{II.5.5.5}[i] and \sref{II.5.5.5}[ii], and so $f$ is projective \sref{II.5.5.6}, and is clearly surjective, by definition.
  \item[(B)] Now suppose that $X$ is \emph{irreducible}.
    Since the structure morphism $r:X\to S$ is of finite type, there exists a finite cover $(S_i)$ of $S$ by affine open subsets, and for each $i$ there is a finite cover $(T_{ij})$ of $r^{-1}(S_i)$ by affine open subsets, and the morphisms $T_{ij}\to S_i$ are of finite type, and so quasi-projective \sref{II.5.3.4}[i];
    since in both hypotheses~(a) and (b) the immersion $S_i\to S$ is quasi-compact, it is also quasi-projective \sref{II.5.3.4}[i], and so the restriction of $r$ to $T_{ij}$ is a quasi-projective morphism \sref{II.5.3.4}[ii].
    Denote the $T_{ij}$ by $U_k$ ($1\leq k\leq n$).
    There exists, for each index $k$, an open immersion $\vphi_k:U_k\to P_k$, where $P_k$ is projective over $S$ (\sref{II.5.3.2} and \sref{II.5.5.2}).
    Let $U=\bigcap_k U_k$; since $X$ is irreducible, and the $U_k$ nonempty, $U$ is nonempty, and thus dense in $X$; the restrictions of the $\vphi_k$ to $U$ define a morphism
    \[
      \vphi:U\to P=P_1\times_S P_2\times_S\cdots\times_S P_n
    \]
    such that the diagrams
    \[
    \label{II.5.6.1.1}
      \xymatrix{
        U\ar[r]^\vphi\ar[d]_{j_k} &
        P\ar[d]^{p_k}\\
        U_k\ar[r]^{\vphi_k} &
        P_k
      }
      \tag{5.6.1.1}
    \]
    commute, where $j_k$ is the canonical injection $U\to U_k$, and $p_k$ the canonical projection $P\to P_k$.
    If $j$ is the canonical injection $U\to X$, then the morphism $\psi=(j,\vphi)_S:U\to X\times_S P$ is an \emph{immersion} \sref[I]{I.5.3.14}.
    In hypothesis~(a), $X\times_S P$ is locally Noetherian (\sref{II.3.4.1}, \sref[I]{I.6.3.7}, and \sref[I]{I.6.3.8});
    in hypothesis~(b), $X\times_S P$ is a quasi-compact scheme (\sref[I]{I.5.5.1} and \sref[I]{I.6.6.4});
    in both cases, the \emph{closure} $X'$ in $X\times_S P$ of the subprescheme $Z$ associated to $\psi$ (and so with underlying space $\psi(U)$) exists, and $\psi$ factors as
    \[
    \label{II.5.6.1.2}
      \psi:U\xrightarrow{\psi'}X'\xrightarrow{h}X\times_S P
      \tag{5.6.1.2}
    \]
    where $\psi'$ is an \emph{open immersion} and $h$ a \emph{closed immersion} \sref[I]{I.9.5.10}.
    Let $q_1:X\times_S P\to X$ and $q_2:X\times_S P\to P$ be the canonical projections; we set
    \[
    \label{II.5.6.1.3}
      f:X'\xrightarrow{h}X\times_S P\xrightarrow{q_1}X,
      \tag{5.6.1.3}
    \]
    \[
    \label{II.5.6.1.4}
      g:X'\xrightarrow{h}X\times_S P\xrightarrow{q_2}P.
      \tag{5.6.1.4}
    \]
    We will see that $X'$ and $f$ satisfy the conclusion of the theorem.
  \item[(C)] First we show that $f$ is \emph{projective} and \emph{surjective}, and that the restriction of $f$ to $U'=f^{-1}(U)$ is an \emph{isomorphism} from $U'$ to $U$.
    Since the $P_k$ are projective over $S$, so is $P$ \sref{II.5.5.5}[iv], and so $X\times_S P$ is projective over $X$ \sref{II.5.5.5}[iii], and thus so is $X'$, which is a closed subprescheme of $X\times_S P$.
    Furthermore, we have $f\circ\psi'=q_1\circ(h\circ\psi')=q_1\circ\psi=j$, so $f(X')$ contains the open everywhere-dense subset $U$ of $X$; but $f$ is a \emph{closed} morphism \sref{II.5.5.3}, so $f(X')=X$.
    Now note that $q_1^{-1}(U)=U\times_S P$ is induced on an open subset of $X\times_S P$, and, by definition, the prescheme $U'=h^{-1}(U\times_S P)$ is induced by $X'$ on the open subset $U'$; it is thus the closure \emph{relative to}
\oldpage[II]{108}
    $U\times_S P$ of the prescheme $Z$ \sref[I]{I.9.5.8}.
    But the immersion $\psi$ factors as $U\xrightarrow{\Gamma_\vphi}U\times_S P\xrightarrow{j\times1}X\times_S P$, and since $P$ is separated over $S$, the graph morphism $\Gamma_\vphi$ is a closed immersion \sref[I]{I.5.4.3}, and so $Z$ is a \emph{closed} subprescheme of $U\times_S P$, whence $U'=Z$.
    Since $\psi$ is an immersion, the restriction of $f$ to $U'$ is an isomorphism onto $U$, and the inverse of $\psi'$; finally, by the definition of $X'$, $U'$ is dense in $X'$.
  \item[(D)] We now show that $g$ is an \emph{immersion}, which will imply that $X'$ is \emph{quasi-projective} over $S$, because $P$ is projective over $S$.
    Set
    \begin{align*}
      V_k  &=\vphi_k(U_k) \quad(\text{open subset of }P_k)\\
      W_k  &=p_k^{-1}(V_k)\quad(\text{open subset of }P)\\
      U_k' &=f^{-1}(U_k)  \quad(\text{open subset of }X')\\
      U_k''&=g^{-1}(W_k)  \quad(\text{open subset of }X').
    \end{align*}
    It is clear that the $U'_k$ form an open cover of $X'$; we will first see that the $U_k''$ also form an open cover of $X'$, by showing that $U_k'\subset U_k''$.
    For this, it will suffice to show that the diagram
    \[
    \label{II.5.6.1.5}
      \xymatrix{
        U_k'\ar[r]^{g|U_k'}\ar[d]_{f|U_k'} &
        P\ar[d]^{p_k}\\
        U_k\ar[r]^{\vphi_k} &
        P_k
      }
      \tag{5.6.1.5}
    \]
    commutes.
    But the prescheme $U_k'=h^{-1}(U_k\times_S P)$ is induced by $X'$ on the open subset $U_k'$, and is thus the closure of $Z=U'\subset U_k'$ relative to $U_k'$ \sref[I]{I.9.5.8}.
    To show the commutativity of \sref{II.5.6.1.5}, it thus suffices (since $P_k$ is an $S$-scheme) to show that composing the diagram with the canonical injection $U'\to U_k'$ (or, equivalently, thanks to the isomorphism from $U'$ to $U$, with $\psi$) gives us a commutative diagram \sref[I]{I.9.5.6}.
    But, by definition, the diagram thus obtains is exactly \sref{II.5.6.1.1}, whence our claim.

    The $W_k$ thus form an open cover of $g(X')$; to show that $g$ is an immersion, it suffices to show that each of the restrictions $g|U_k''$ is an immersion into $W_k$ \sref[I]{I.4.2.4}.
    For this, consider the morphism $u_k:W_k\xrightarrow{p_k}V_k\xrightarrow{\vphi_k^{-1}}U_k\to X$; since $X$ is separated over $S$, the graph morphism $\Gamma_{u_k}:W_k\to X\times_S W_k$ is a closed immersion \sref[I]{I.5.4.3}, and so the graph $T_k=\Gamma_{u_k}(W_k)$ is a closed subprescheme of $X\times_S W$;
    if we show that $U'\to X\times_S W_k$ factors through this subprescheme, then the map from the subprescheme induced by $X'$ on the open subset $X_k''$ of $X'$ to $X\times_S W_k$ will also factor through this graph, by \sref[I]{I.9.5.8}.
    Since the restriction of $q_2$ to $T_k$ is an isomorphism onto $W_k$, the restriction of $g$ to $X''_k$ will be an immersion into $W_k$, and our claim will be proven.
    Let $v_k$ be the canonical injection $U'\to X\times_S W_k$; we have to show that there exists a morphism $w_k:U'\to W_k$ such that $v_k=\Gamma_{u_k}\circ w_k$.
    By the definition of the product, it suffices to prove that $q_1\circ v_k=u_k\circ q_2\circ v_k$ \sref[I]{I.3.2.1}, or, by composing on the right
\oldpage[II]{109}
    with the isomorphism $\psi':U\to U'$, that $q_1\circ\psi=u_k\circ q_2\circ\psi$.
    But since $q_1\circ\psi=j$ and $q_2\circ\psi=\vphi$, our claim follows from the commutativity of \sref{II.5.6.1.1}, taking into account the definition of $u_k$.
  \item[(E)] It is clear that since $U$, and thus $U'$, is irreducible, so is the $X'$ from the preceding construction, and the morphism $f$ is thus \emph{birational} \sref[I]{I.2.2.9}.
    If in addition $X$ is reduced, then so is $U'$, and hence $X'$ is also reduced \sref[I]{I.9.5.9}.
    This finishes the proof.
\end{enumerate}
\end{proof}

\begin{corollary}[5.6.2]
\label{II.5.6.2}
Suppose that one of the hypotheses, \emph{(a)} and \emph{(b)}, of \sref{II.5.6.1} is satisfied.
For $X$ to be proper over $S$, it is necessary and sufficient for there to exist a projective scheme $X'$ over $S$, and a surjective $S-$morphism $f:X'\to X$ (which is thus projective, by \sref{II.5.5.5}[v]).
Whenever this is the case, we can further choose $f$ to be such that there exists a dense open subset $U$ of $X$ for which the restriction of $f$ to $f^{-1}(U)$ is an isomorphism $f^{-1}(U)\isoto U$, and for which $f^{-1}(U)$ is dense in $X'$.
If in addition $X$ is irreducible (resp. reduced), then we can assume that $X'$ is also irreducible (resp. reduced); when $X$ and $X'$ are irreducible, $f$ is a birational morphism.
\end{corollary}

\begin{proof}
The condition is sufficient, by \sref{II.5.5.3} and \sref{II.5.4.3}[ii].
It is necessary because, with the notation of \sref{II.5.6.1}, if $X$ is proper over $S$, then $X'$ is proper over $S$, because it is projective over $X$, and thus proper over $X$ \sref{II.5.5.3}, and our claim follows from \sref{II.5.4.2}[ii]; furthermore, since $X'$ is quasi-projective over $S$, it is projective over $S$, by \sref{II.5.5.3}.
\end{proof}

\begin{corollary}[5.6.3]
\label{II.5.6.3}
Let $S$ be a locally Noetherian prescheme, and $X$ an $S$-scheme of finite type over $S$, with structure morphism $f_0:X\to S$.
For $X$ to be proper over $S$, it is necessary and sufficient that, for every morphism \emph{of finite type} $S'\to S$, $(f_0)_{(S')}:X_{(S')}\to S'$ be a closed morphism.
It even suffices for this condition to be verified only for every $S$-prescheme of the form $S'=S\otimes_\bb{Z}\bb{Z}[T_1,\ldots,T_n]$ (where the $T_i$ are indeterminates).
\end{corollary}

\begin{proof}
The condition being clearly necessary, we now show that it is sufficient.
Since the question is local on $S$ and $S'$ \sref{II.5.4.1}, we can suppose that $S$ and $S'$ are affine and Noetherian.
By Chow's lemma, there exists a projective $S$-scheme $P$, an immersion $j:X'\to P$, and a surjective projective morphism $f:X'\to X$, such that the diagram
\[
  \xymatrix{
    X\ar[d]_{f_0} &
    X'\ar[l]_f\ar[d]^j\\
    S &
    P\ar[l]_r
  }
\]
commutes.
Since $P$ is of finite type over $S$, the first hypothesis implies that the projection $q_2:X\times_S P\to P$ is a \emph{closed} morphism.
But the immersion $j$ is the composition of $q_2$ and the morphism $f\times1$ from $X'\times_S P$ to $X\times_S P$;
but $f$, being projective, is proper \sref{II.5.5.3}, and so $f\times1$ is closed.
We thus conclude that $j$ is a closed immersion, and thus proper \sref{II.5.4.2}[i].
Furthermore, the structure morphism $r:P\to S$ is projective, and thus proper \sref{II.5.5.3}, so $f_0\circ f=r\circ j$ is proper \sref{II.5.4.2}[ii];
finally, since $f$ is surjective, $f_0$ is proper, by \sref{II.5.4.3}.

To prove the proposition using only the second, weaker hypothesis (where $S'$ is of the form $S\otimes_\bb{Z}\bb{Z}[T_1,\ldots,T_n]$), it suffices to show that it implies the first.
But, if $S'$ is affine and of finite type over $S=\Spec(A)$,
\oldpage[II]{110}
then we have $S'=\Spec(A[c_1,\ldots,c_n])$ \sref[I]{I.6.3.3}, and $S'$ is thus isomorphic to a closed subprescheme of $S''=\Spec(A[T_1,\ldots,T_n])$ (where the $T_i$ are indeterminates).
In the commutative diagram
\[
  \xymatrix{
    X\times_S S'\ar[r]^{1_X\times j}\ar[d]_{(f_0)_{(S')}} &
    X\times_S S''\ar[d]^{(f_0)_{(S'')}}\\
    S'\ar[r]^j &
    S''
  }
\]
both $j$ and $1_X\times j$ are closed immersions \sref[I]{I.4.3.1}, and $(f_0)_{(S')}$ is closed by hypothesis; thus $(f_0)_{(S'')}$ is also closed.
\end{proof}


\section{Integral morphisms and finite morphisms}
\label{section:II.6}


\subsection{Preschemes integral over another prescheme}
\label{subsection:II.6.1}

\begin{definition}[6.1.1]
\label{II.6.1.1}
Let $X$ be an $S$-prescheme, with structure morphism $f:X\to S$.
We say that $X$ is \emph{integral over $S$}, or that $f$ is an \emph{integral morphism}, if there exists a cover $(S_\alpha)$ of $S$ by affine opens such that, for all $\alpha$, the induced prescheme $f^{-1}(S_\alpha)$ is an affine scheme whose ring $B_\alpha$ is an integral algebra over the ring $A_\alpha$ of $S_\alpha$.
We say that $X$ is \emph{finite over $S$}, or that $f$ is a \emph{finite morphism} if $X$ is integral and of finite type over $S$.
\end{definition}

If $S$ is affine of ring $A$, then we also say ``integral (resp. finite) over $A$'' instead of ``integral (resp. finite) over $S$''.

\begin{env}[6.1.2]
\label{II.6.1.2}
It is clear that, if $X$ is integral over $S$, then it is \emph{affine} over $S$.
For an affine prescheme $X$ over $S$ to be integral (resp. finite) over $S$ it is necessary and sufficient that the associated quasi-coherent $\sh{O}_S$-algebra $\sh{A}(X)$ be such that there exist a cover $(S_\alpha)$ of $S$ by affine opens having the property that, for all $\alpha$, $\Gamma(S_\alpha,\sh{A}(X))$ is an integral (resp. integral and of finite type) algebra over $\Gamma(S_\alpha,\sh{O}_S)$.
A quasi-coherent $\sh{O}_S$-algebra with this property is said to be \emph{integral} (resp. \emph{finite}) over $\sh{O}_S$.
Giving an integral (resp. finite) prescheme over $S$ is thus \sref{II.1.3.1} the same as giving a quasi-coherent $\sh{O}_S$-algebra that is integral (resp. finite) over $\sh{O}_S$.
Note that a quasi-coherent $\sh{O}_S$-algebra $\sh{B}$ is finite if and only if it is an \emph{$\sh{O}_S$-module of finite type} \sref[I]{I.1.3.9};
it is equivalent to say that $\sh{B}$ is an \emph{integral} $\sh{O}_S$-algebra \emph{of finite type}, since an algebra that is integral and of finite type over a ring $A$ is an $A$-module of finite type.
\end{env}

\begin{proposition}[6.1.3]
\label{II.6.1.3}
Let $S$ be a locally Noetherian prescheme.
For an affine prescheme $X$ over $S$ to be finite over $S$, it is necessary and sufficient that the $\sh{O}_S$-algebra $\sh{A}(X)$ be coherent.
\end{proposition}

\begin{proof}
Taking the preceding remark into account, this reduces to noting that, if $S$ is locally Noetherian, then the quasi-coherent $\sh{O}_S$-modules of finite type are exactly the coherent $\sh{O}_S$-modules \sref[I]{I.1.5.1}.
\end{proof}

\begin{proposition}[6.1.4]
\label{II.6.1.4}
Let $X$ be an integral (resp. finite) prescheme over $S$, with structure morphism $f:X\to S$.
Then, for every affine open $U\subset S$ of ring $A$, $f^{-1}(U)$ is an affine scheme whose ring $B$ is an integral (resp. finite) algebra over $A$.
\end{proposition}

\oldpage[II]{111}
\begin{proof}
We first prove the following lemma:

  \begin{lemma}[6.1.4.1]
  \label{II.6.1.4.1}
  Let $A$ be a ring, $M$ an $A$-module, and $(g_i)_{1\leq i\leq m}$ a finite system of elements of $A$ such that the $D(g_i)$ (for $1\leq i\leq m$) cover $\Spec(A)$.
  If, for all $i$, $M_{g_i}$ is an $A_{g_i}$-module of finite type, then $M$ is an $A$-module of finite type.
  \end{lemma}

  \begin{proof}
  We can assume that $M_{g_i}$ admits a finite system of generators $(m_{ij}/g_i^n)$ with $m_{ij}\in M$, with $n$ the same for all indices $i$.
  We will show that the $m_{ij}$ for a system of generators of $M$.
  By hypothesis, for each $i$, there exist $a_{ij}\in A$ and some integer $p$ (independent of $i$) such that, in $M_{g_i}$, $m/1=(\sum_i a_{ij}m_{ij})/g_i^p$;
  this implies that there exists an integer $r\geq p$ such that, for all $i$, we have $g_i^rm\in M'$.
  But, since the $D(g_i^r)=D(g_i)$ cover $\Spec(A)$, the ideal of $A$ generated by the $g_i^r$ is equal to $A$, or, in other words, there exist elements $a_i\in A$ such that $\sum_i a_ig_i^r$;
  then $m=(\sum_i a_i g_i^r)m\in M'$, whence the lemma.
  \end{proof}

Now we already know \sref{II.1.3.2} that $f^{-1}(U)$ is affine.
If $\vphi$ is the homomorphism $A\to B$ corresponding to $f$, then there exists a finite cover of $U$ by opens $D(g_i)$ (where $g_i\in A$) such that, if $h_i=\vphi(g_i)$, $B_{h_i}$ is an integral (resp. integral and finite) algebra over $A_{g_i}$.
Indeed, there exists a cover of $U$ by affine opens $V_\alpha\subset U$ such that, if $A_\alpha=A(V_\alpha)$, $B_\alpha=A(f^{-1}(V_\alpha))$ is an integral (resp. finite) algebra over $A_\alpha$.
Every $x\in U$ belongs to some $V_\alpha$, so there exists $g\in A$ such that $x\in D(g)\subset V_\alpha$;
if $g_\alpha$ is the image of $g$ in $A_\alpha$, then $A(D(g))=A_g=(A_\alpha)_{g_\alpha}$;
let $h=\vphi(g)$, and let $h_\alpha$ be the image of $g_\alpha$ in $B_\alpha$;
we have
\[
  A(D(h)) = B_h = (B_\alpha)_{h_\alpha}
\]
and, since $B_\alpha$ is integral (resp. finite) over $A_\alpha$, $(B_\alpha)_{h_\alpha}$ is integral (resp. finite) over $(A_\alpha)_{g_\alpha}$.
It now suffices to use the fact that $U$ is quasi-compact to obtain the desired cover.

If we suppose first of all that the $B_{h_i}$ are integral and finite over the $A_{g_i}$, then since $B_{h_i}$ can also be written as $B_{g_i}$ as an $A_{g_i}$-module, Lemma~\sref{II.6.1.4.1} shows that, in this case, $B$ is an $A$-module of finite type.

Now suppose only that each $B_{h_i}$ is integral over $A_{g_i}$;
let $b\in B$, and let $C$ be the sub-$A$-algebra of $B$ generated by $b$.
For all $i$, $C_{h_i}$ is the algebra over $A_{g_i}$ generated by $b/1$ in $B_{h_i}$;
it follows from the hypothesis that each $C_{h_i}$ is an $A_{g_i}$-module of finite type, and so \sref{II.6.1.4.1} $C$ is an $A$-module of finite type, which proves that $B$ is integral over $A$.
\end{proof}

\begin{proposition}[6.1.5]
\label{II.6.1.5}
\medskip\noindent
\begin{enumerate}
  \item[(i)] A closed immersion is finite (and \emph{a fortiori} integral).
  \item[(ii)] The composition of two finite (resp. integral) morphisms is finite (resp. integral).
  \item[(iii)] If $f:X\to Y$ is a finite (resp. integral) $S$-morphism, then $f_{(S')}:X_{(S')}\to Y_{(S')}$ is finite (resp. integral) for any base extension $S'\to S$.
  \item[(iv)] If $f:X\to Y$ and $g:X'\to Y'$ are finite (resp. integral) $S$-morphisms, then $f\times_S g:X\times_S Y\to X'\times_S Y'$ is finite (resp. integral).
  \item[(v)] If $f:X\to Y$ and $g:Y\to Z$ are morphisms such that $g\circ f$ is finite (resp. integral), if $g$ is separated, then $f$ is finite (resp. integral).
  \item[(vi)] If $f:X\to Y$ is a finite (resp. integral) morphism, then $f_\red$ is finite (resp. integral).
\end{enumerate}
\end{proposition}

\oldpage[II]{112}
\begin{proof}
By \sref[I]{I.5.5.12}, it suffices to prove (i), (ii), and (iii).
To prove that a closed immersion $X\to S$ is finite, we can restrict to the case where $S=\Spec(A)$, and everything then follows from noting that a quotient ring $A/\mathfrak{J}$ is a monogeneous $A$-module.
To prove that the composition of two finite (resp. integral) morphism $X\to Y$, $Y\to Z$ is finite (resp. integral), we can again assume that $Z$ (and thus $X$ and $Y$ \sref{II.1.3.4}) is affine, and then the claim is equivalent to saying that, if $B$ is a finite (resp. integral) $A$-algebra and $C$ a finite (resp. integral) $B$-algebra, then $C$ is a finite (resp. integral) $A$-algebra, which is immediate.
Finally, to prove (iii), we can restrict to the case where $S=Y$, since $X_{(S')}$ can be identified with $X\times_Y Y_{(S')}$ \sref[I]{I.3.3.11};
we can further suppose that $S=\Spec(A)$ and $S'=\Spec(A')$;
then $X$ is affine of ring $B$ \sref{II.1.3.4}, and $X_{(S')}$ affine of ring $A'\otimes_A B$, and it suffices to note that, if $B$ is a finite (resp. integral) $A$-algebra, then $A'\otimes_A B$ is a finite (resp. integral) $A'$-algebra.
\end{proof}

We also note that, if $X$ and $Y$ are $S$-preschemes that are finite (resp. integral) over $S$, then their \emph{sum} $X\sqcup Y$ is a finite (resp. integral) prescheme over $S$, since this reduces to showing that, if $B$ and $C$ are finite (resp. integral) $A$-algebras over $A$, then so too is $B\times C$.

\begin{corollary}[6.1.6]
\label{II.6.1.6}
If $X$ is an integral (resp. finite) prescheme over $S$, then, for every open $U\subset S$, $f^{-1}(U)$ is integral (resp. finite) over $U$.
\end{corollary}

\begin{proof}
This is a particular case of \sref{II.6.1.5}[(iii)].
\end{proof}

\begin{corollary}[6.1.7]
\label{II.6.1.7}
Let $f:X\to Y$ be a finite morphism.
Then, for all $y\in Y$, the fibre $f^{-1}(y)$ is a finite algebraic scheme over $\kres(y)$, and \emph{a fortiori} its underlying space is discrete and finite.
\end{corollary}

\begin{proof}
Indeed, as a $\kres(y)$-prescheme, $f^{-1}(y)$ can be identified with $X\times_Y\Spec(\kres(y))$ \sref[I]{I.3.6.1}, which is finite over $\Spec(\kres(y))$ \sref{II.6.1.5}[(iii)];
it is thus an affine scheme whose ring is an algebra of finite rank over $\kres(y)$ \sref{II.6.1.4}.
The corollary then follows from \sref[I]{I.6.4.4}.
\end{proof}

\begin{corollary}[6.1.8]
\label{II.6.1.8}
Let $X$ and $S$ be integral preschemes, and $f:X\to S$ a \emph{dominant} morphism.
If $f$ is integral (resp. finite) then the field $R(X)$ of rational functions on $X$ is algebraic (resp. algebraic of finite degree) over the field $R(S)$ of rational functions on $S$.
\end{corollary}

\begin{proof}
Let $s$ be the generic point of $S$;
the $\kres(s)$-prescheme $f^{-1}(s)$ is integral (resp. finite) over $\Spec(\kres(s))$ \sref{II.6.1.5}[(iii)] and contains, by hypothesis, the generic point $x$ of $X$;
since the local ring of $x$ in $f^{-1}(s)$ is equal to $\kres(x)$ \sref[I]{I.3.6.5}, and is thus a local ring of an integral (resp. finite) algebra over $\kres(s)$ \sref{II.6.1.4}, whence the corollary.
\end{proof}

\begin{remark}[6.1.9]
\label{II.6.1.9}
The hypothesis that $g$ be \emph{separated} is essential for the validity of \sref{II.6.1.5}[(v)]: if $Y$ is not separated over $S$, then the identity $1_Y$ is the composite morphism $Y\xrightarrow{\Delta_Y}Y\times_Z Y\xrightarrow{p_1}Y$, but $\Delta_Y$ is not an integral morphism, as follows from \sref{II.6.1.10}:
\end{remark}

\begin{proposition}[6.1.10]
\label{II.6.1.10}
Every integral morphism is universally closed.
\end{proposition}

\begin{proof}
Let $f:X\to Y$ be an integral morphism;
by \sref{II.6.1.5}[(iii)], it suffices to show that $f$ is \emph{closed}.
Let $Z$ be a closed subset of $X$;
\oldpage[II]{113}
then there exists a subprescheme of $X$ whose underlying space is $Z$ \sref[I]{I.5.4.1}, and it thus follows from \sref{II.6.1.5}[(i) and (ii)] that it suffices to prove that $f(X)$ is \emph{closed} in $Y$.
By \sref{II.6.1.5}[(vi)], we can suppose that $X$ and $Y$ are \emph{reduced};
further, if $T$ is the closed reduced subprescheme of $Y$ whose underlying space is $\overline{f(X)}$ \sref[I]{I.5.2.1} then we know that $f$ factors as $X\to T\xrightarrow{j}Y$, where $j$ is the injection morphism \sref[I]{I.5.2.2}, and since $j$ is separated \sref[I]{I.5.5.1}[(i)], it follows from \sref{II.6.1.5}[(v)] that $g$ is an integral morphism.
We can thus suppose that $f(X)$ is \emph{dense} in $Y$.
Finally, since the question is local on $Y$, we can restrict to the case where $Y=\Spec(A)$.
Then $X=\Spec(B)$, where $B$ is an $A$-algebra that is integral over $A$ \sref{II.6.1.4};
furthermore, $A$ is reduced \sref[I]{I.5.1.4}, and the hypothesis that $f(X)$ be dense in $Y$ implies that the homomorphism $\vphi:A\to B$ corresponding to $f$ is \emph{injective} \sref[I]{I.1.2.7}.
Under these conditions, saying that $f(X)=Y$ implies that every prime ideal of $A$ is the intersection with $A$ of a prime ideal of $B$, which is exactly the first theorem of Cohen--Seidenberg (\cite[t.~I, p.~257, th.~3]{II-13}).
\end{proof}

\begin{corollary}[6.1.11]
\label{II.6.1.11}
Every finite morphism $f:X\to Y$ is projective.
\end{corollary}

\begin{proof}
Since $f$ is affine, $\sh{O}_X$ is a \emph{very ample} $\sh{O}_X$-module with respect to $f$ \sref{II.5.1.2};
furthermore, $f_*(\sh{O}_X)$ is a quasi-coherent $\sh{O}_Y$-module \emph{of finite type} \sref{II.6.1.2};
finally, $f$ is separated, of finite type, and universally closed \sref{II.6.1.10}, and thus satisfies the conditions of criterion~\sref{II.5.5.4}[(i)].
\end{proof}

\begin{proposition}[6.1.12]
\label{II.6.1.12}
Let $f:X'\to X$ be a finite morphism, and let $\sh{B}=f_*(\sh{O}_{X'})$ (which is a quasi-coherent $\sh{O}_X$-algebra, and a $\sh{O}_X$-module of finite type).
Let $\sh{F}'$ be a quasi-coherent $\sh{O}_{X'}$-module;
for $\sh{F}'$ to be locally free of rank~$r$, it is necessary and sufficient that $f_*(\sh{F}')$ be a locally free $\sh{B}$-module of rank~$r$.
\end{proposition}

\begin{proof}
It is clear that, if $f_*(\sh{F}')|U$ is isomorphic to $\sh{B}^r|U$ (where $U\subset X$ is open), then $\sh{F}'|f^{-1}(U)$ is isomorphic to $\sh{O}_{X'}^r|f^{-1}(U)$ \sref{II.1.4.2}.
Conversely, suppose that $\sh{F}'$ is locally free of rank~$r$; we will show that $f_*(\sh{F}')$ is locally isomorphic to $\sh{B}^r$ as a $\sh{B}$-module.
Let $x$ be a point of $X$;
as $U$ runs over a fundamental system of affine neighbourhoods of $x$, $f^{-1}(U)$ runs over a fundamental system of affine neighbourhoods \sref{II.1.2.5} of the finite set $f^{-1}(x)$, since $f$ is closed \sref{II.6.1.10}.
The proposition then follows from the following lemma:
\end{proof}

\begin{lemma}[6.1.12.1]
\label{II.6.1.12.1}
Let $Y$ be a prescheme, $\sh{E}$ a locally free $\sh{O}_Y$-module of rank~$r$, and $Z$ a finite subset of $Y$ contained inside some affine open $V$.
Then there exists a neighbourhood $U\subset V$ of $Z$ such that $\sh{E}|U$ is isomorphic to $\sh{O}_Y^r|U$.
\end{lemma}

\begin{proof}
We can evidently assume that $Y$ is affine;
for all $z_i\in Z$, there exists in the closure $\overline{\{z_i\}}$ at least one closed point $z'_i$ \sref[0]{0.2.1.3};
if $Z'$ is the set of the $z'_i$ then every neighbourhood of $Z'$ is a neighbourhood of $Z$, and we can thus assume that $Z$ is discrete and closed in $Y$.
Consider the closed reduced subprescheme of $Y$ that has $Z$ has its underlying space \sref[I]{I.5.2.1} and let $j:Z\to Y$ be the canonical injection;
$j^*(\sh{E})=\sh{E}\otimes_Y\sh{O}_Z$ is locally free of rank~$r$ on the discrete scheme $Z$, and is thus isomorphic to $\sh{O}_Z^r$;
in other words, there exist $r$ sections $s_i$ (for $1\leq i\leq r$) of $\sh{E}\otimes_Y\sh{O}_Z$ over $Z$ such that the homomorphism $\sh{O}_Z^r\to\sh{E}\otimes_Y\sh{O}_Z$ defined by these sections is bijective.
But $Y=\Spec(A)$ is affine, $Z$ is defined by an ideal $\mathfrak{J}$ of $A$, and we have $\sh{E}=\widetilde{M}$, where $M$ is an $A$-module;
the $s_i$ are elements of $M\otimes_A(A/\mathfrak{J})$ and are thus images of $r$ elements $t_i\in M=\Gamma(Y,\sh{E})$.
\oldpage[II]{114}
For all $z_j\in Z$ there is thus a neighbourhood $V_j$ of $z_j$ such that the restrictions of the $t_i$ to $V_j$ defined an isomorphism $\sh{O}_Y^r|V_j\to\sh{E}|V_j$ \sref[0]{0.5.5.4};
the neighbourhood $U$ given by the union of the $V_j$ is the desired neighbourhood.
\end{proof}

\begin{proposition}[6.1.13]
\label{II.6.1.13}
Let $g:X'\to X$ be an integral morphism of preschemes, $Y$ a normal locally integral prescheme, and $f$ a rational map from $Y$ to $X'$ such that $g\circ f$ is an everywhere defined rational map \sref[I]{I.7.2.1}.
Then $f$ is everywhere defined.
\end{proposition}

\begin{proof}
If $f_1$ and $f_2$ are morphisms (from dense open subsets of $Y$ to $X'$) in the class $f$, then it is clear that $g\circ f_1$ and $g\circ f_2$ are equivalent morphisms, which justifies the notation $g\circ f$ for their equivalence class.
We recall also that, if we further suppose $Y$ to be \emph{locally Noetherian}, then the hypothesis that $Y$ is normal already implies that $Y$ is locally integral \sref[I]{I.6.1.13}.

To prove the proposition, note first of all that the question is local on $Y$, and so we can suppose that there exists in the class $g\circ f$ a \emph{morphism} $h:Y\to X$.
Consider the inverse image $Y'=X'_{(h)}=X'_{(Y)}$, and note that the morphism $g'=g_{(Y)}:Y'\to Y$ is \emph{integral} \sref{II.6.1.5}[(iii)].
Given the correspondence between rational maps from $Y$ to $X'$ and rational $Y$-sections of $Y'$ \sref[I]{I.7.1.2}, we see that it suffices to prove the specific case of \sref{II.6.1.13} where $X=Y$; in other words, the following:
\end{proof}

\begin{corollary}[6.1.14]
\label{II.6.1.14}
Let $X$ be a normal locally integral prescheme, $g:X'\to X$ an \emph{integral} morphism, and $f$ a rational $X$-section of $X'$.
Then $f$ is everywhere defined.
\end{corollary}

\begin{proof}
Since the question is local on $X$, we can assume that $X$ is integral, and then $f$ is identified with a morphism from an open $U$ of $X$ to $X'$ \sref[I]{I.7.2.2} that is a $U$-section of $g^{-1}(U)$.
Since $g$ is separated, $f$ is a closed immersion from $U$ into $g^{-1}(U)$ \sref[I]{I.5.4.6};
let $Z$ be the closed subprescheme of $g^{-1}(U)$ associated to $f$ \sref[I]{I.4.2.1}, which is isomorphic to $U$, and thus integral;
let $X_1$ be the reduced subprescheme of $X'$ whose underlying space is the closure $\overline{Z}$ of $Z$ in $X'$ \sref[I]{I.5.2.1};
then $Z$ is an induced subprescheme on an open of $X_1$ \sref[I]{I.5.2.3}, and, since it is irreducible, so too is $X_1$, which is thus integral.
The morphism $f$ can then be considered as a rational $X$-section of $X_1$;
since the restriction of $g$ to $X_1$ is an integral morphism \sref{II.6.1.5}[(i) and (ii)], we can finally reduce to proving \sref{II.6.1.14} in the specific case where $X'=X_1$;
in other words, the following:
\end{proof}

\begin{corollary}[6.1.15]
\label{II.6.1.15}
Let $X$ be a normal integral prescheme, $X'$ an integral prescheme, and $g:X'\to X$ an \emph{integral} morphism.
If there exists a rational $X$-section $f$ of $X'$, then $g$ is an isomorphism.
\end{corollary}

\begin{proof}
Since the question is local on $X$, we can assume that $X$ is affine of integral ring $A$, and then $X'$ is affine of ring $A'$ with $A'$ integral over $A$ \sref{II.6.1.4} and integral;
furthermore, the argument of \sref{II.6.1.14} shows that there exists a dense open of $X$ that is isomorphic to a dense open of $X'$, and so $A$ and $A'$ have the same field of fractions.
Also, by \sref[I]{I.8.2.1.1}, and the hypothesis that the $\sh{O}_x$ are integrally closed, the ring $A$ is integrally closed, and so $A'=A$, which finishes the proof of \sref{II.6.1.13}
\end{proof}


\subsection{Quasi-finite morphisms}
\label{subsection:II.6.2}

\begin{proposition}[6.2.1]
\label{II.6.2.1}
Let $f:X\to Y$ be morphism locally of finite type, and $x$ a point of $X$.
Then the following conditions are equivalent:
\oldpage[II]{115}
\begin{enumerate}
  \item[a)] The point $x$ is isolated in its fibre $f^{-1}(f(x))$.
  \item[b)] The ring $\sh{O}_x$ is a quasi-finite $\sh{O}_{f(x)}$-module \sref[0]{0.7.4.1}.
\end{enumerate}
\end{proposition}

\begin{proof}
Since the question is clearly local on $X$ and on $Y$, we can assume that $X=\Spec(A)$ and $Y=\Spec(B)$ are affine, with $A$ a $B$-algebra of finite type \sref[I]{I.6.3.3}.
Furthermore, we can replace $X$ by $X\times_Y\Spec(\sh{O}_{f(x)})$ without changing either the fibre $f^{-1}(f(x))$ or the local ring $\sh{O}_x$ \sref[I]{I.3.6.5};
we can thus assume that $B$ is a local ring (equal to $\sh{O}_{f(x)}$);
if $\mathfrak{n}$ is the maximal ideal of $B$, then $f^{-1}(f(x))$ is an affine scheme of ring $A/\mathfrak{n}A$, of finite type over $\kres(f(x))=B/\mathfrak{n}$ \sref[I]{I.6.4.11}.
With this, if (a) is satisfied then we can further suppose that $f^{-1}(f(x))$ consists of the single point $x$;
thus $A/\mathfrak{n}A$ is of finite rank over $B/\mathfrak{n}$ \sref[I]{I.6.4.4}, or, in other words, $A$ is a quasi-finite $B$-module.
Conversely, if $A$ is a quasi-finite $B$-module, then $f^{-1}(f(x))$ is an Artinian affine scheme, and thus discrete \sref[I]{I.6.4.4};
so $x$ is isolated in its fibre, and this shows that (b) implies (a).
\end{proof}

\begin{corollary}[6.2.2]
\label{II.6.2.2}
Let $f:X\to Y$ be a morphism of finite type.
Then the following conditions are equivalent:
\begin{enumerate}
  \item[a)] Every point $x\in X$ is isolated in its fibre $f^{-1}(f(x))$ (or, in other words, the subspace $f^{-1}(f(x))$ is \emph{discrete}).
  \item[b)] For all $x\in X$, the prescheme $f^{-1}(f(x))$ is a finite $\kres(f(x))$-prescheme.
  \item[c)] For all $x\in X$, the ring $\sh{O}_x$ is a quasi-finite $\sh{O}_{f(x)}$-module.
\end{enumerate}
\end{corollary}

\begin{proof}
The equivalence of (a) and (c) follows from \sref{II.6.2.1}.
Since $f^{-1}(f(x))$ is an algebraic $\kres(f(x))$-prescheme \sref[I]{I.6.4.11}, the equivalence of (a) and (b) follows from \sref[I]{I.6.4.4}.
\end{proof}

\begin{definition}[6.2.3]
\label{II.6.2.3}
If $f:X\to Y$ is a morphism of finite type satisfying the equivalent conditions of \sref{II.6.2.2}, we say that $f$ is \emph{quasi-finite}, or that $X$ is \emph{quasi-finite over $Y$}.
\end{definition}

It is clear that every \emph{finite} morphism is quasi-finite \sref{II.6.1.8}.

\begin{proposition}[6.2.4]
\label{II.6.2.4}
\medskip\noindent
\begin{enumerate}
  \item[(i)] An immersion $X\to Y$ that is closed, or such that $X$ is Noetherian, is a quasi-finite morphism.
  \item[(ii)] If $f:X\to Y$ and $g:Y\to Z$ are quasi-finite morphisms, then $g\circ f$ is quasi-finite.
  \item[(iii)] If $X$ and $Y$ are $S$-preschemes, and $f:X\to Y$ a quasi-finite $S$-morphism, then $f_{(S')}:X_{(S')}\to Y_{(S')}$ is quasi-finite for any base extension $g:S'\to S$.
  \item[(iv)] If $f:X\to Y$ and $g:X'\to Y'$ are quasi-finite $S$-morphisms, then
    \[
      f\times_S g : X\times_S Y \to X'\times_S Y'
    \]
    is quasi-finite.
  \item[(v)] Let $f:X\to Y$ and $g:Y\to Z$ be morphisms such that $g\circ f$ is quasi-finite; if, further, $g$ is separated, or $X$ is Noetherian, or $X\times_Z Y$ is locally Noetherian, then $f$ is quasi-finite.
  \item[(vi)] If $f$ is quasi-finite, then $f_\red$ is quasi-finite.
\end{enumerate}
\end{proposition}

\begin{proof}
If $f:X\to Y$ is an immersion, then every fibre consists of a single point, and claim (i) then follows from (\sref[I]{I.6.3.4}[(i)] and \sref[I]{I.6.3.5}).
To prove (ii), note first of all that $h=g\circ f$ is of finite type \sref[I]{I.6.3.4}[(ii)];
furthermore, if $z=h(x)$ and $y=f(x)$, then $y$ is isolated in $g^{-1}(z)$, and so there exists an open neighbourhood $V$ of $y$ in $Y$ that does not meet any point of $g^{-1}(z)$ apart from $y$;
thus $f^{-1}(V)$ is an open neighbourhood of $x$ that does not meet any of the $f^{-1}(y')$, where $y'\neq y$ is in $g^{-1}(z)$;
\oldpage[II]{116}
since $x$ is isolated in $f^{-1}(y)$, it is isolated in $h^{-1}(z)=f^{-1}(g^{-1}(z))$.
To prove (iii), we can reduce to the case where $Y=S$ \sref[I]{I.3.3.11};
we again note first of all that $f'=f_{(S')}$ is of finite type \sref[I]{I.6.3.4}[(iii)];
also, if $x'\in X'=X_{(S')}$, and if we set $y'=f'(x')$ and $y=g(y')$, then ${f'}^{-1}(y')$ can be identified with $f^{-1}(y)\otimes_{\kres(y)}\kres(y')$ \sref[I]{I.3.6.5};
since $f^{-1}(y)$ is of finite rank over $\kres(y)$ by hypothesis, ${f'}^{-1}(y')$ is of finite rank over $\kres(y')$, and thus discrete.
Claims (iv), (v), and (vi) follows from (i), (ii), and (iii) by the general method \sref[I]{I.5.5.12}, except for when the hypotheses in (v) are not ``g is separated'';
in these cases, we remark first of all that, if $x$ is isolated in $f^{-1}(g^{-1}(g(f(x))))$, then it is \emph{a fortiori} isolated in $f^{-1}(f(x))$;
the fact that $f$ is of finite type then follows from \sref[I]{I.6.3.6}.
\end{proof}

\begin{proposition}[6.2.5]
\label{II.6.2.5}
Let $A$ be a \emph{complete} local Noetherian ring, $X$ an $A$-scheme locally of finite type, and $x$ a point of $X$ over the closed point $y$ of $Y=\Spec(A)$.
Suppose that $x$ is isolated in its fibre $f^{-1}(y)$ (where $f$ is the structure morphism $X\to Y$).
Then $\sh{O}_x$ is an $A$-module of finite type, and $X$ is $Y$-isomorphic to the sum \sref[I]{I.3.1} of $X'=\Spec(\sh{O}_x)$ (which is a finite $Y$-scheme) and an $A$-scheme $X''$.
\end{proposition}

\begin{proof}
It follows from \sref{II.6.2.1} that $\sh{O}_x$ is a quasi-finite $A$-module.
Since $\sh{O}_x$ is Noetherian \sref[I]{I.6.3.7}, and the homomorphism $A\to\sh{O}_x$ is local, the hypothesis that $A$ is \emph{complete} implies that $\sh{O}_x$ is an $A$-module \emph{of finite type} \sref[0]{0.7.4.3}.
Let $X'=\Spec(\sh{O}_x)$ be the local scheme of $X$ at the point $x$ \sref[I]{I.2.4.1}, and let $g:X'\to X$ be the canonical morphism.
Since the composite $X'\xrightarrow{g}X\xrightarrow{f}Y$ is finite \sref{II.6.1.1}, and since $f$ is separated, $g$ is finite \sref{II.6.1.5}[(v)], and so $g(X')$ is \emph{closed} in $X$ \sref{II.6.1.10};
also, since $g$ is of finite type, $g$ is a local isomorphism at the closed point $x'$ of $X'$ by the definition of $g$ and \sref[I]{I.6.5.4};
but since $X'$ is the only open neighbourhood of $x'$, this implies that $g$ is an open immersion, and so $g(X')$ is also \emph{open} in $X$, which finishes the proof.
\end{proof}

\begin{corollary}[6.2.6]
\label{II.6.2.6}
Let $A$ be a \emph{complete} local Noetherian ring, $Y=\Spec(A)$, and $f:X\to Y$ a separated quasi-finite morphism.
Then $X$ is $Y$-isomorphic to a sum $X'\sqcup X''$, where $X'$ is a finite $Y$-scheme, and $X''$ is a quasi-finite $Y$-scheme such that, if $y$ is the closed point of $Y$, then $X''\cap f^{-1}(y)=\emp$.
\end{corollary}

\begin{proof}
Indeed, the fibre $f^{-1}(y)$ is finite and discrete by hypothesis, and the corollary then follows, by induction on the number of points in this fibre, from \sref{II.6.2.5}.
\end{proof}

\begin{remark}[6.2.7]
\label{II.6.2.7}
In Chapter~V, we will see that, if $Y$ is \emph{locally Noetherian}, then a separated quasi-finite morphism $X\to Y$ is necessarily \emph{quasi-affine}.
\end{remark}


\subsection{Integral closure of a prescheme}
\label{subsection:II.6.3}

\begin{proposition}[6.3.1]
\label{II.6.3.1}
Let $(X,\sh{A})$ be a ringed space, $\sh{B}$ a (commutative) $\sh{A}$-algebra, and $f$ a section of $\sh{B}$ over $X$.
Then the following properties are equivalent:
\begin{enumerate}
  \item[a)] The sub-$\sh{A}$-module of $\sh{B}$ generated by the $f^n$ for $n\geq$ \sref[0]{0.5.1.1} is of finite type.
  \item[b)] There exists a sub-$\sh{A}$-algebra $\sh{C}$ of $\sh{B}$ that is an $\sh{A}$-module of finite type and such that $f\in\Gamma(X,\sh{C})$.
  \item[c)] For all $x\in X$, $f_x$ is integral over the fibre $\sh{A}_x$.
\end{enumerate}
\end{proposition}

\oldpage[II]{117}
\begin{proof}
Since the sub-$\sh{A}$-module of $\sh{B}$ generated by the $f^n$ is an $\sh{A}$-algebra, it is clear that (a) implies (b).
On the other hand, (b) implies that, for all $x\in X$, the $\sh{A}_x$-module $\sh{C}_x$ is of finite type, which implies that every element of the algebra $\sh{C}_x$, and, in particular, $f_x$, is integral over $\sh{A}_x$.
Finally, if we have, for some $x\in X$, a relation of the form
\[
  f_x^n + (a_1)_xf_x^{n-1} + \ldots + (a_n)_x = 0
\]
where the $a_i$ (for $1\leq i\leq n$) are sections of $\sh{A}$ over an open neighbourhood $U$ of $x$, then the section $f^n|U+a_1\cdot f^{n-1}|U+\ldots+a_n$ is zero over a neighbourhood $V\subset U$ of $x$, from which it immediately follows that the $f^k|V$ (for $k\geq0$) are linear combinations of the $f^j|V$ (for $0\leq j\leq n-1$) with coefficients in $\Gamma(V,\sh{A})$;
we thus conclude that (c) implies (a).
\end{proof}

If the equivalent conditions of \sref{II.6.3.1} are satisfied then we say that the section $f$ is \emph{integral over $\sh{A}$}.

\begin{corollary}[6.3.2]
\label{II.6.3.2}
Under the hypotheses of \sref{II.6.3.1}, there exists a (unique) sub-$\sh{A}$-module $\sh{A}'$ of $\sh{B}$ such that, for all $x\in X$, $\sh{A}'_x$ is the set of germs $f_x\in\sh{B}_x$ that are integral over $\sh{A}_x$.
For every open $U\subset X$, the sections of $\sh{A}'$ over $U$ are the sections $f\in\Gamma(U,\sh{B})$ that are integral over $\sh{A}|U$.
\end{corollary}

\begin{proof}
The existence of $\sh{A}'$ is immediate, by taking $\Gamma(U,\sh{A}')$ to be the set of $f\in\Gamma(U,\sh{B})$ such that $f_x$ is integral over $\sh{A}_x$ for all $x\in U$.
The second claim follows immediately from \sref{II.6.3.1}.
\end{proof}

It is clear that $\sh{A}'$ is a sub-$\sh{A}$-algebra of $\sh{B}$;
we say that it is the \emph{integral closure of $\sh{A}$ in $\sh{B}$}.

\begin{env}[6.3.3]
\label{II.6.3.3}
Let $(X,\sh{A})$ and $(Y,\sh{B})$ be ringed spaces, and let
\[
  g = (\psi,\theta):X\to Y
\]
be a morphism.
Let $\sh{C}$ (resp. $\sh{D}$) be an $\sh{A}$-algebra (resp. $\sh{B}$-algebra), and let
\[
  u:\sh{D}\to\sh{C}
\]
be a $g$-morphism \sref[0]{0.4.4.1}.
Then, if $\sh{A}'$ (resp. $\sh{B}'$) is the integral closure of $\sh{A}$ (resp. $\sh{B}$) in $\sh{C}$ (resp. $\sh{D}$), the \emph{restriction} of $u$ to $\sh{B}'$ is a $g$-morphism
\[
  u':\sh{B}'\to\sh{A}'.
\]

Indeed, if $j$ is the canonical injection $\sh{B}'\to\sh{D}$, then it suffices to show that
\[
  v = u^\sharp\circ g^*(j) : j^*(\sh{B}')\to\sh{C}'
\]
sends $g^*(\sh{B}')$ to $\sh{A}'$.
But an element of $g^*(\sh{B}')_x=\sh{B}'_{\psi(x)}\otimes_{\sh{B}_{\psi(x)}}\sh{A}_x$ is integral over $\sh{A}_x$ by the definition of $\sh{B}'$, and thus so too is its images under $v_x$, which proves the claim.
\end{env}

\begin{proposition}[6.3.4]
\label{II.6.3.4}
Let $X$ be a prescheme, and $\sh{A}$ a quasi-coherent $\sh{O}_X$-algebra.
Then the integral closure $\sh{O}'_X$ of $\sh{O}_X$ in $\sh{A}$ is a quasi-coherent $\sh{O}_X$-algebra, and, for every affine open $U$ of $X$, $\Gamma(U,\sh{O}'_X)$ is the integral closure of $\Gamma(U,\sh{O}_X)$ in $\Gamma(U,\sh{A})$.
\end{proposition}

\begin{proof}
We can restrict to the case where $X=\Spec(B)$ is affine, and $\sh{A}=\widetilde{A}$, where $A$ is a $B$-algebra;
\oldpage[II]{118}
let $B'$ be the integral closure of $B$ in $A$.
Everything then reduces to proving that, for all $x\in X$, an element of $A_x$ which is integral over $B_x$ necessarily belongs to $B'_x$, which follows from the fact that, for a commutative ring $C$, the operations of taking the integral closure in a $C$-algebra and passing to a ring of fractions (with respect to a multiplicative subset of $C$) commute \cite[t.~I, pp.~261 and 257]{I-13}.
\end{proof}

The $X$-scheme $X'=\Spec(\sh{O}'_X)$ is then called the \emph{integral closure of $X$ with respect to $\sh{A}$} (or \emph{with respect to $\Spec(\sh{A})$});
it is clear that $X'$ is \emph{integral} over $X$ \sref{II.6.1.2}.

We immediately deduce from \sref{II.6.1.4} that, if $f:X'\to X$ is the structure morphism, then, for every open $U$ of $X$, $f^{-1}(U)$ is the \emph{integral closure of the prescheme induced by $X$ on $U$, with respect to $\sh{A}|U$}.

\begin{env}[6.3.5]
\label{II.6.3.5}
Let $X$ and $Y$ be preschemes, $f:X\to Y$ a morphism, $\sh{A}$ (resp. $\sh{B}$) a quasi-coherent $\sh{O}_X$-algebra (resp. quasi-coherent $\sh{O}_Y$-algebra), and let $u:\sh{B}\to\sh{A}$ be an $f$-morphism.
We have seen \sref{II.6.3.3} that we obtain an $f$-morphism $u':\sh{O}'_Y\to\sh{O}'_X$, where $\sh{O}'_X$ (resp. $\sh{O}'_Y$) is the integral closure of $\sh{O}_X$ (resp. $\sh{O}_Y$) in $\sh{A}$ (resp. $\sh{B}$).
Then, if $X'$ (resp. $Y'$) is the integral closure of $X$ (resp. $Y$) with respect to $\sh{A}$ (resp. $\sh{B}$), we canonically obtain from $u$ a morphism $f'=\Spec(u'):X'\to Y'$ \sref{II.1.5.6} such that the diagram
\[
\label{II.6.3.5.1}
  \xymatrix{
    X' \ar[r]^{f'} \ar[d]
    & Y' \ar[d]
  \\X \ar[r]_{f}
    & Y
  }
\tag{6.3.5.1}
\]
commutes.
\end{env}

\begin{env}[6.3.6]
\label{II.6.3.6}
Suppose that $X$ has only a \emph{finite} number of irreducible components $X_i$ (for $1\leq i\leq r$), with generic points $\xi_i$, and consider, in particular, the integral closure of $X$ with respect to some quasi-coherent \emph{$\sh{R}(X)$-algebra} $\sh{A}$ (quasi-coherent as either an $\sh{O}_X$-algebra or as an $\sh{R}(X)$-algebra, since the two are equivalent).
We know \sref[I]{I.7.3.5} that $\sh{A}$ is a direct sum of $r$ quasi-coherent $\sh{O}_X$-algebras $\sh{A}_i$, where the support of $\sh{A}_i$ is contained inside $X_i$, and the sheaf induced by $\sh{A}_i$ on $X_i$ is a simple sheaf whose fibre $A_i$ is an algebra over $\sh{O}_{\xi_i}$.
It is then clear \sref{II.6.3.4} that the integral closure $\sh{O}'_X$ of $\sh{O}_X$ in $\sh{A}$ is the \emph{direct sum} of the integral closures $\sh{O}_X^{(i)}$ of $\sh{O}_X$ in each of the $\sh{A_i}$, and thus the integral closure $X'=\Spec(\sh{O}'_X)$ of $X$ with respect to $\sh{A}$ is the $X$-scheme given by the \emph{sum} of the $\Spec(\sh{O}_X^{(i)})=X'_i$ (for $1\leq i\leq r$).

Suppose further that the $\sh{O}_X$-algebra $\sh{A}$ is \emph{reduced}, or, equivalently, that each of the algebras $A_i$ are reduced, and can thus be considered as an algebra over the \emph{field} $\kres(\xi_i)$ (equal to the field of rational functions of the reduced subprescheme of $X$ that has $X_i$ as its underlying space);
then \sref{II.1.3.8} each of the $X'_i$ is a \emph{reduced} $X$-scheme, and $X'$ is also the integral closure of $X_\red$.
Suppose further that each of the algebras $A_i$ is a \emph{direct sum of a finite number of fields $K_{ij}$} (for $1\leq j\leq s_i$);
if $\sh{K}_{ij}$ is the sub-algebra of $\sh{A}_i$ corresponding to $K_{ij}$, then it is clear that $\sh{O}_X^{(i)}$ is the \emph{direct sum} of the integral closures $\sh{O}_X^{(ij)}$ of $\sh{O}_X$ in each of the $\sh{K}_{ij}$.
Then $X'_i$ is the \emph{sum} of the $X$-schemes $X'_{ij}=\Spec(\sh{O}_X^{(ij)})$ (for $1\leq j\leq s_i$).
Furthermore, under these hypotheses, and with this notation, we have:
\end{env}

\oldpage[II]{119}
\begin{proposition}[6.3.7]
\label{II.6.3.7}
Each of the $X'_{ij}$ is an integral normal $X$-prescheme, and its field of rational functions $R(X'_{ij})$ is canonically identified with the algebraic closure $K'_{ij}$ of $\kres(\xi_i)$ in $K_{ij}$.
\end{proposition}

\begin{proof}
By the above, we can suppose that $X$ is integral, and so $r=1$ and $s_1=1$, so that the unique algebra $A_1$ is a field $K$;
let $\xi$ be the generic point of $X$, and let $f:X\to X'$ be the structure morphism.
For every non-empty affine open $U$ of $X$, $f^{-1}(U)$ can be identified with the spectrum of the integral closure $B'_U$ in the field $K$ of the integral ring $B_U=\Gamma(Y,\sh{O}_X)$ \sref{II.6.3.4};
since the ring $B'_U$ is integral and integrally closed, so too are the local rings of the points of its spectrum, and thus $f^{-1}(U)$ is by definition an \emph{integral} and \emph{normal} scheme (\sref[0]{0.4.1.4} and \sref[I]{I.5.1.4}).
Furthermore, since $(0)$ is the only prime ideal of $B'_U$ over the prime ideal $(0)$ of $B_U$ \cite[t.~1, p.~259]{I-13}, $f^{-1}(\xi)$ consists of a single point $\xi'$, and $\kres(\xi')$ is the field of fractions $K'$ of $B'_U$, which is exactly the algebraic closure of $\kres(\xi)$ in $K$.
Finally, $X'$ is irreducible, since, if $U$ runs over the set of non-empty affine opens of $X$, then the $f^{-1}(U)$ give an open cover of $X'$ by irreducible opens;
furthermore, the intersection $f^{-1}(U\cap V)$ of any two of these opens contains $\xi'$, and is thus non-empty, and we conclude by \sref[0]{0.2.1.4}.
\end{proof}

\begin{corollary}
Let $X$ be a reduced prescheme that has only a finite number of irreducible components $X_i$ (where $1\leq i\leq r$), and let $\xi_i$ be the generic point of $X_i$.
Then the integral closure $X'$ of $X$ with respect to $\sh{R}(X)$ is the sum of the $r$ $X$-schemes $X'_i$, which are integral and normal.
If $f:X'\to X$ is the structure morphism, then $f^{-1}(\xi_i)$ consists only of the generic point $\xi'_i$ of $X'_i$, and we have $\kres(\xi'_i)=\kres(\xi_i)$, or, in other words, that $f$ is birational.
\end{corollary}

In this case, we say that $X'$ is the \emph{normalisation} of the reduced prescheme $X$;
we note that $f$, since it is birational and integral, is \emph{surjective} \sref{II.6.1.10}.
Then in order to have $X'=X$, it is necessary and sufficient that $X$ be \emph{normal}.
If $X$ is an integral prescheme, then it follows from \sref{II.6.3.8} that its normalisation $X'$ is integral.

\begin{env}[6.3.9]
\label{II.6.3.9}
Let $X$ and $Y$ be integral preschemes, $f:X\to Y$ a dominant morphism, and $L=R(X)$ (resp. $K=R(Y)$) the field of rational functions of $X$ (resp. $Y$);
there is a canonical injection $K\to L$ corresponding to $f$, and if we identify $K$ (resp. $L$) with the simple sheaf $\sh{R}(Y)$ (resp. $\sh{R}(X)$), this injection is an $f$-morphism.
Let $K_1$ (resp. $L_1$) be an extension of $K$ (resp. $L$), and suppose we have a morphism $K_1\to L_1$ such that the diagram
\[
  \xymatrix{
    K_1 \ar[r]
    & L_1
  \\K \ar[u] \ar[r]
    & L \ar[u]
  }
\]
commutes;
if $K_1$ (resp. $L_1$) is considered as a simple sheaf on $Y$ (resp. $X$), and thus as an $\sh{R}(Y)$-algebra (resp. $\sh{R}(X)$-algebra), this implies that $K_1\to L_1$ is an $f$-morphism.
With this, if $X'$ (resp. $Y'$) is the integral closure of $X$ (resp. $Y$) with respect to $L_1$ (resp. $K_1$), then $X'$ (resp. $Y'$) is a normal integral prescheme \sref{II.6.3.6} whose field of rational functions is canonically identified with the algebraic closure $L'$ (resp. $K'$) of $L$ (resp. $K$) in $L_1$ (resp $K_1$), and there exists a (necessarily dominant) canonical morphism $f':X'\to Y'$ that makes the diagram \sref{II.6.3.5.1} commute.
\oldpage[II]{120}

The most important case is when we take $L_1=L$, with $K_1$ then being an extension of $K$ contained in $L$, and when we suppose $X$ to be integral and \emph{normal}, so that $X'=X$.
The above then shows that, since $X$ is normal, if $Y'$ is the integral closure of $Y$ with respect to a field $K_1\subset L=R(X)$, then every dominant morphism $f:X\to Y$ factors as
\[
  f:X\xrightarrow{f'}Y'\to Y
\]
where $f'$ is dominant;
also, since the monomorphism $K_1\to L$ is given, $f'$ is necessarily unique, as we see by reducing to the case where $X$ and $Y$ are affine.
We thus see that, given $Y$, $L$, and a $K$-monomorphism $K_1\to L$, the integral closure $Y'$ of $Y$ with respect to $K_1$ is the solution of a \emph{universal problem}.
\end{env}

\begin{remark}[6.3.10]
\label{II.6.3.10}
Consider the hypothesis of \sref{II.6.3.6}, and suppose further that each of the algebras $A_i$ is \emph{of finite rank} over $\kres(\xi_i)$ (which implies that $A_i$ is a direct sum of a finite number of fields);
we can then show, in certain cases, that the structure morphism $X'\to X$ is not just integral, but even \emph{finite}.
Restrict to the case where $X$ is \emph{reduced};
since the question is local on $X$, we can further suppose that $X$ is affine of ring $C$, and that $C$ has only a finite number of minimal ideals $\mathfrak{p}_i$ (for $1\leq i\leq r$) such the that $C_i=C/\mathfrak{p}_i$ are integral;
then $X'$ is finite over $X$ if the integral closure of each $C_i$ in finite-dgree extension of its field of fractions is a \emph{$C$-module of finite type} \sref{II.6.3.4}.
We know that this condition is always satisfied if $C$ is an \emph{algebra of finite type over a field} \cite[t.~I, p.~267, th.~9]{I-13}, or \emph{over $\bb{Z}$} \cite[I,p.~93, th.~3]{I-9}, or \emph{over a complete Noetherian local ring} \cite[p.~298]{II-25}.
We thus conclude that $X'\to X$ will be a finite morphism whenever $X$ is a scheme \emph{of finite type} over a field, or over $\bb{Z}$, or over a complete Noetherian local ring.
\end{remark}


\subsection{Determinant of an endomorphism of $\mathcal{O}_X$-modules}
\label{subsection:II.6.4}

\begin{env}[6.4.1]
\label{II.6.4.1}
Let $A$ be a (commutative) ring, $E$ a free $A$-module of rank~$n$, and $u$ an endomorphism of $E$;
recall that, to define the \emph{characteristic polynomial} of $u$, we consider the endomorphism $u\otimes1$ of the free $A[T]$-module of rank~$n$, $E\otimes_A A[T]$ (where $T$ is an indeterminate), and we set
\[
\label{II.6.4.1.1}
  P(u,T) = \det(T\cdot I-(u\otimes 1))
\tag{6.4.1.1}
\]
(where $I$ is the identity automorphism of $E\otimes_A A[T]$).
We have
\[
\label{II.6.4.1.2}
  P(u,T) = T^n - \sigma_1(u)T^{n-1} + \ldots + (-1)^n\sigma_n(u)
\tag{6.4.1.2}
\]
where $\sigma_i(u)$ is an element of $A$, equal to a homogeneous polynomial of degree~$i$ (with integer coefficients) with respect to the entries of the matrix of $u$ in some arbitrary basis of $E$;
we say that the $\sigma_i(u)$ are the \emph{elementary symmetric functions} of $u$;
in particular, we have that $\sigma_1(u)=\Tr u$ and $\sigma_n(u)=\det u$.
Recall that, by the Hamilton--Cayley theorem, we have
\[
\label{II.6.4.1.3}
  P(u,u) = u^n - \sigma_1(u)u^{n-1} + \ldots + (-1)^n\sigma_n(u) = 0
\tag{6.4.1.3}
\]
\oldpage[II]{121}
which can also be written as
\[
\label{II.6.4.1.4}
  (\det u)\cdot I_E = uQ(u)
\tag{6.4.1.4}
\]
(where $I_E$ is the identity automorphism of $E$), where
\[
\label{II.6.4.1.5}
  Q(u) = (-1)^{n+1}(u^{n-1}-\sigma_1(u)u^{n-2}+\ldots+(-1)^{n-1}\sigma_{n-1}(u)).
\tag{6.4.1.5}
\]

Let $\vphi:A\to B$ be a ring homomorphism, which makes $B$ into an $A$-algebra;
consider the $B$-module $E_{(B)}=E\otimes_A B$, which is free of rank~$n$, and the extension $u\otimes1$ of $u$ to an endomorphism of $E_{(B)}$;
it is immediate that we have $\sigma_i(u\otimes1)=\vphi(\sigma_i(u))$ for all indices $i$.
\end{env}

\begin{env}[6.4.2]
\label{II.6.4.2}
Now suppose that $A$ is an \emph{integral} ring, with field of fractions $K$, and let $E$ be an $A$-module \emph{of finite type} (but not necessarily free).
Let $n$ be the \emph{rank} of $E$, i.e. the rank of the free $K$-module $E\otimes_A K$;
to every endomorphism $u$ of $E$ there canonically corresponds the endomorphism $u\otimes1$ of $E\otimes_A K$;
by an abuse of language, we also define the \emph{characteristic polynomial} of $u$, denoted by $P(u,T)$, to be the polynomial $P(u\otimes1,T)$, whose coefficients $\sigma_i(u\otimes1)$ (which belong to $K$) are also denoted by $\sigma_i(u)$ and called the \emph{elementary symmetric functions of $u$};
in particular, $\det u=\det(u\otimes1)$ by definition.
With this notation, Equations~\sref{II.6.4.1.3} to \sref{II.6.4.1.5} make sense and still hold true, if we interpret $u^j$ as the composite homomorphism $E\to E\otimes_A K$ of the endomorphism $u^j\otimes1=(u\otimes1)^j$ of $E\otimes_A K$ and the canonical homomorphism $x\mapsto x\otimes1$.

If $F$ is the torsion submodule of $E$, and if $E_0=E/F$, then $u(F)\subset F$, and so, by passing to quotients, $u$ gives an endomorphism $u_0$ of $E_0$;
furthermore, $E\otimes_A K$ can be identified with $E_0\otimes_A K$, and $u\otimes1$ with $u_0\otimes1$, and so $\sigma_i(u)=\sigma_i(u_0)$ for $1\leq i\leq n$.

If $E$ is \emph{torsion free}, then $E$ is canonically identified with a sub-$A$-module of $E\otimes_A K$, and the relation $u\otimes1=0$ is equivalent to $u=0$.
If $A$ is a free $A$-module, then the two definitions of $\sigma_i(u)$ given in \sref{II.6.4.1} and \sref{II.6.4.2} coincide, by the above, which justifies the notation used.

We also note that, if $E$ is a \emph{torsion module}, or, in other words, if $E_0=\{0\}$, then the exterior algebra of $E_0$ consists of $K$, and the determinant of the unique endomorphism $u_0$ of $E_0$ is \emph{equal to $1$}.
\end{env}

\begin{proposition}[6.4.3]
\label{II.6.4.3}
Let $A$ be an integral ring, $E$ an $A$-module of finite type, and $u$ an endomorphism of $E$;
then the elementary symmetric functions $\sigma_i(u)$ of $u$ (and, in particular, $\det u$) are elements of $K$ that are integral over $A$.
\end{proposition}

\begin{proof}
Let $A'$ be the integral closure of $A$;
since $A'[T]$ is an integrally closed ring \cite[p.~99]{II-24}, it is the integral closure of $A[T]$ in its field of fractions $K(T)$.
Replacing $u$ by $T\cdot I-u\otimes1$, and $A$ by $A[T]$, we see that we can reduce to proving that $u$ is integral over $A$.
If $n$ is the rank of $E$, then $\det(u)=\det(\vee^n u)$ and $(\vee^n u\otimes1)=\vee^n(u\otimes1)$, so we can suppose that $n=1$.
But then the map $u\mapsto\det u$ is a homomorphism from the $A$-module $\Hom_A(E,E)$ to $K$;
since $E$ is of finite type, $\Hom_A(E,E)$ is isomorphic to a sub-$A$-module of the $A$-module of finite type $E^n$ (if $E$ admits a system of $n$ generators), and so the elements $\det u$ belong to a sub-$A$-module of $K$ of finite type, and are thus integral over $A$.
\end{proof}

\oldpage[II]{122}
\begin{corollary}[6.4.4]
\label{II.6.4.4}
Under the hypotheses of \sref{II.6.4.3}, if we further suppose $A$ to be normal, then the $\sigma_i(u)$ (and, in particular, $\Tr u$ and $\det u$) belong to $A$.
\end{corollary}

\begin{proposition}[6.4.5]
\label{II.6.4.5}
Let $A$ be an integral ring, $E$ an $A$-module of finite type of rank~$n$, and $u$ an endomorphism of $E$ such that the $\sigma_i(u)$ belong to $A$.
For $u$ to be an automorphism of $E$, it is necessary that $\det u$ be invertible in $A$;
this condition is sufficient if $E$ is torsion free.
\end{proposition}

\begin{proof}
The condition is \emph{sufficient} when $E$ is torsion free, since the hypothesis and equations~\sref{II.6.4.1.4} and \sref{II.6.4.1.5} (which hold in $E$, and not just in $E\otimes_A K$, since $E$ is torsion free) prove that $(\det u)^{-1}Q(u)$ is the inverse of $u$.

The condition is \emph{necessary}, since, if $u$ is invertible, then it follows from \sref{II.6.4.3} that $\det(u^{-1})$ belongs to the integral closure $A'$ of $A$ in its field of fractions $K$, and is evidently the inverse of $\det u$ \emph{in $A'$}.
Our claim then follows from:
\end{proof}

\begin{lemma}[6.4.5.1]
\label{II.6.4.5.1}
Let $A$ be a subring of a ring $A'$ such that $A'$ is \emph{integral} over $A$.
If an element $x\in A$ is invertible in $A'$, then it is also invertible in $A$.
\end{lemma}

\begin{proof}
In the contrary case, $x$ would belong to a maximal ideal $\mathfrak{m}$ of $A$, and it follows from the first theorem of Cohen--Seidenberg \cite[t.~I, p.~257, th.~3]{I-13} that there would be a maximal ideal $\mathfrak{m}'$ of $A'$ such that $\mathfrak{m}=\mathfrak{m}'\cap A$;
we would then have that $x\in\mathfrak{m}'$, which is a contradiction.
\end{proof}

\begin{corollary}[6.4.6]
\label{II.6.4.6}
Let $A$ be an integral and integrally closed ring, $E$ a torsion-free $A$-module of finite type, and $u$ an endomorphism of $E$>
For $u$ to be an automorphism of $E$, it is necessary and sufficient that $\det u$ be invertible in $A$.
\end{corollary}

\begin{proof}
This follows from \sref{II.6.4.4} and \sref{II.6.4.5}.
\end{proof}

\begin{remark}[6.4.7]
\label{II.6.4.7}
We will later need a generalisation of the above results.
Consider a \emph{reduced} Noetherian ring $A$;
let $\mathfrak{p}_\alpha$ (for $1\leq\alpha\leq r$) be its minimal ideals, $K_\alpha$ the field of fractions of the integral ring $A/\mathfrak{p}_\alpha$, and $K$ the total ring of fractions of $A$, which is the \emph{direct sum} of the fields $K_\alpha$.
Let $E$ be an $A$-module of finite type, and \emph{suppose} that $E\otimes_A K$ is a \emph{free} $K$-module of rank~$n$ (which here is merely a consequence of the other hypotheses);
equivalently, we can ask that all the $K_\alpha$-vector spaces $E\otimes_A K_\alpha=E_\alpha$ have \emph{the same dimension~$n$};
if then $u$ is an endomorphism of $E$, we again set $P(u,T)=P(u\otimes1,T)$ and $\sigma_j(u)=\sigma_j(u\otimes1)$, and, in particular, $\det u=\det(u\otimes1)$;
the $\sigma_j(u)$ are thus elements of $K$.
It is immediate that $E\otimes_A K$ is the direct sum of the $E_\alpha$, and that the latter are stable under $u\otimes1$;
the restriction of $u\otimes1$ to $E_\alpha$ is exactly the extension $u_\alpha$ of $u$ to this $K_\alpha$-vector space;
we thus conclude that $\sigma_j(u)$ is the element of $K$ who components in the $K_\alpha$ are the $\sigma_j(u_\alpha)$.
Since the integral closure of $A$ in $K$ is the \emph{direct sum} of the integral closures of $A$ in the $K_\alpha$, the $\sigma_j(u)$ are \emph{integral} over $A$.

\begin{lemma}[6.4.7.1]
\label{II.6.4.7.1}
The sub-$A$-algebra of $K$ generated by all the elements $\sigma_j(u)$ (for $1\leq j\leq n$), where $u$ runs over $\Hom_A(E,E)$, is an $A$-module of finite type.
\end{lemma}

\begin{proof}
It suffices to prove that the sub-$A[T]$-algebra of $K[T]$ generated by the $P(u,T)$ is an $A[T]$-module of finite type, since if the $F_i(T)$ (for $1\leq i\leq m$) form a system of generators for this $A[T]$-module, then the coefficients of the $F_i(T)$ are integral over $A$, and thus generate an $A$-algebra that is an $A$-module of finite type \cite[t.~I, p.~255, th.~1]{I-13}.
\oldpage[II]{123}
We can thus replace $A$ by $A[T]$ (which is Noetherian), and $E$ by $E\otimes_A A[T]=E'$, which is such that $E'\otimes_{A[T]}K[T]=E\otimes_A K[T]$ is a free $K[T]$-module of rank~$n$.
Using the initial notation, we thus see that it suffices to prove that the $A$-module generated by the elements $\det u$, where $u$ runs over $\Hom_A(E,E)$, is of finite type;
\emph{a fortiori} (since every submodule of an $A$-module of finite type is of finite type) it suffices to prove that, as $v$ runs over the set of endomorphisms of $\wedge^n E$, the $A$-module generated by the $\det v$ is of finite type;
in other words, we can again reduce to the case where $n=1$.
But then the proposition follows from the fact that $\Hom_A(E,E)$ is an $A$-module of finite type, and that $v\mapsto\det v$ is a homomorphism from this $A$-module into $K$.
\end{proof}

Let $F$ be the kernel of the canonical homomorphism $E\to E\otimes_A K$, and let $E_0=E/F$;
we see, as above, that $E\otimes_A K$ can be identified with $E_0\otimes_A K$, that $u(F)\subset F$, and that, if $u_0$ is the endomorphism of $E_0$ induced from $u$ by passing to the quotient, then $u\otimes1$ can be identified with $u_0\otimes1$, and $\sigma_j(u)=\sigma_j(u_0)$ for all $j$.
\emph{If we have $F=0$}, then equations~\sref{II.6.4.1.3} and \sref{II.6.4.1.5} still make sense and hold true whenever $E$ is identified with a submodule of $E\otimes_A K$, and the $u^j$ with homomorphisms $E\to E\otimes_A K$;
then Proposition~\sref{II.6.4.5} extends also to this case, with the same proof.
\end{remark}

\begin{env}[6.4.8]
\label{II.6.4.8}
Let $(X,\sh{A})$ be a ringed space, $\sh{E}$ a \emph{locally free} $\sh{A}$-module (of finite rank), and $u$ an endomorphism of $\sh{E}$.
By hypothesis, there exists a base $\mathfrak{B}$ for the topology of $X$ such that, for all $V\in\mathfrak{B}$, $\sh{E}|V$ is isomorphic to some $\sh{A}^n|V$ (where $n$ may vary with $V$).
Let $u$ be an endomorphism of $\sh{E}$;
for all $V\in\mathfrak{B}$, $u_V$ is thus an endomorphism of the $\Gamma(V,\sh{A})$-module $\Gamma(V,\sh{E})$, which is free by hypothesis;
the determinant $\det u_V$ is thus defined and belongs to $\Gamma(V,\sh{A})$.
Furthermore, if $e_1,\ldots,e_n$ form a basis of $\Gamma(V,\sh{E})$, then their restrictions to every open $W\subset V$ form a basis of $\Gamma(W,\sh{E})$ over $\Gamma(W,\sh{A})$, and so $\det u_W$ is the restriction of $\det u_V$ to $W$.
There thus exists exactly one section of $\sh{A}$ over $X$, which we denote by $\det u$, and call the \emph{determinant} of $u$, such that the restriction of $\det u$ to each $V\in\mathfrak{B}$ is $\det u_V$.
It is clear that, for all $x\in X$, we have $(\det u)_x=\det u_x$;
for endomorphisms $u$ and $v$ of $\sh{E}$, we have
\[
\label{II.6.4.8.1}
  \det(u\circ v) = (\det u)(\det v)
\tag{6.4.8.1}
\]
as well as
\[
\label{II.6.4.8.1}
  \det(1_{\sh{E}}) = 1_{\sh{A}}
\tag{6.4.8.1}
\]
and, if the rank of $\sh{E}$ is constant (which will be the case \sref[0]{0.5.4.1} if $X$ is connected) and equal to $n$,
\[
\label{II.6.4.8.3}
  \det(s\cdot u) = s^n\det u
\tag{6.4.8.3}
\]
for all $s\in\Gamma(X,\sh{A})$ (note that $\det(0)=0_{\sh{A}}$ if $n\geq1$, but $\det(0)=1_{\sh{A}}$ for $n=0$).
Furthermore, for $u$ to be an \emph{automorphism} of $\sh{E}$, it is necessary and sufficient that $u$ be \emph{invertible} in $\Gamma(X,\sh{A})$.

If the rank of $\sh{E}$ is constant, then we can similarly define the elementary symmetric functions $\sigma_i(u)$, which are elements of $\Gamma(X,\sh{A})$;
we again have the relations in \sref{II.6.4.1.3} and \sref{II.6.4.1.5}.

\oldpage[II]{124}
We have thus defined a homomorphism $\det:\Hom_{\sh{A}}(\sh{E},\sh{E})\to\Gamma(X,\sh{A})$ of multiplicative monoids;
if we note that $\Hom_{\sh{A}}(\sh{E},\sh{E})=\Gamma(X,\shHom_{\sh{A}}(\sh{E},\sh{E}))$ by definition, then we see that we can replace $X$ in this definition by an arbitrary open $U$ of $X$, which immediately defines a homomorphism $\det:\shHom_{\sh{A}}(\sh{E},\sh{E})\to\sh{A}$ of sheaves of multiplicative monoids.
If $\sh{E}$ has constant rank, then we similarly define homomorphisms $\sigma_i:\shHom_{\sh{A}}(\sh{E},\sh{E})\to\sh{A}$ of sheaves of sets;
for $i=1$, the homomorphism $\sigma_1=\Tr$ is a homomorphism of $\sh{A}$-modules.

Let $(Y,\sh{B})$ be another ringed space, and let $f:(X,\sh{A})\to(Y,\sh{B})$ be a morphism of ringed spaces;
if $\sh{F}$ is a locally free $\sh{B}$-module, then $f^*(\sh{F})$ is a locally free $\sh{A}$-module (which is of the same rank as $\sh{F}$ if the latter is of constant rank) \sref[0]{0.5.4.5}.
For every endomorphism $v$ of $\sh{F}$, $f^*(v)$ is an endomorphism of $f^*(\sh{F})$, and it follows immediately from the definitions that $\det f^*(v)$ is the section of $\sh{A}=f^*(\sh{B})$ over $X$ that canonically corresponds to $\det v\in\Gamma(Y,\sh{B})$.
We can further say that the homomorphism $f^*(\det):f^*(\shHom_{\sh{B}}(\sh{F},\sh{F}))\to f^*(\sh{B})=\sh{A}$ is the composition
\[
  f^*(\shHom_{\sh{B}}(\sh{F},\sh{F}))
  \xrightarrow{\gamma^\sharp} \shHom_{\sh{A}}(f^*(\sh{F}),f^*(\sh{F}))
  \xrightarrow{\det} \sh{A}
\]
\sref[0]{0.4.4.6}.
We have analogous results for the $\sigma_i$.
\end{env}

\begin{env}[6.4.9]
Now suppose that $X$ is a \emph{locally integral prescheme}, so that the sheaf $\sh{R}(X)$ of rational functions on $X$ is a locally simple sheaf of fields \sref[I]{I.7.4.3}, and quasi-coherent as an $\sh{O}_X$-module.
If $\sh{E}$ is a quasi-coherent $\sh{O}_X$-module \emph{of finite type}, then $\sh{E}'=\sh{E}\otimes_{\sh{O}_X}\sh{R}(X)$ is a locally free $\sh{R}(X)$-module \sref[I]{I.7.3.6};
for every endomorphism $u$ of $\sh{E}$, $u\otimes1_{\sh{R}(X)}$ is then an endomorphism of $\sh{E}'$, and $\det(u\otimes1)$ is a section of $\sh{R}(X)$ over $X$, which we also call the \emph{determinant} of $u$, and also denote by $\det u$.
It follows from \sref{II.6.4.3} that $\det u$ is a section of the \emph{integral closure} of $\sh{O}_X$ in $\sh{R}(X)$ \sref{II.6.3.2};
if, further, $X$ is \emph{normal}, then $\det u$ is a \emph{section of $\sh{O}_X$} over $X$, and if we further suppose that $\sh{E}$ is \emph{torsion free}, then for $u$ to be an automorphism of $\sh{E}$ it is necessary and sufficient that $\det u$ be \emph{invertible}, by \sref{II.6.4.6}.
Equations~\sref{II.6.4.8.1} to \sref{II.6.4.8.3} still hold true;
from the homomorphism $u\mapsto\det u$, applied to the modules of sections of $\shHom_{\sh{O}_X}(\sh{E},\sh{E})$, we obtain a homomorphism of sheaves $\det:\shHom_{\sh{O}_X}(\sh{E},\sh{E})\to\sh{R}(X)$, which takes values in $\sh{O}_X$ if $X$ is normal.
We have analogous definitions and results for the other elementary symmetric functions $\sigma_j(u)$, if $\sh{E}'$ is of \emph{constant rank};
if, further, $X$ is \emph{normal}, then the $\sigma_j(u)$ are \emph{sections of $\sh{O}_X$} over $X$.

Finally, let $X$ and $Y$ be integral preschemes, and $f:X\to Y$ a \emph{dominant} morphism.
We know that there exists a canonical homomorphism $f^*(\sh{R}(Y))\to\sh{R}(X)$ \sref[I]{I.7.3.8}, whence we obtain, for every quasi-coherent $\sh{O}_Y$-module $\sh{F}$ of finite type, a canonical homomorphism $\theta:f^*(\sh{F}\otimes_{\sh{O}_Y}\sh{R}(Y))\to f^*(\sh{F})\otimes_{\sh{O}_X}\sh{R}(X)$.
\oldpage[II]{125}
If $v$ is an endomorphism of $\sh{F}$, then $f^*(v\otimes1_{\sh{R}(Y)})$ is an endomorphism of $f^*(\sh{F}\otimes_{\sh{O}_Y}\sh{R}(Y))$, and we have a commutative diagram
\[
  \xymatrix{
    f^*(\sh{F}\otimes_{\sh{O}_Y}\sh{R}(Y))
      \ar[r]^{f^*(v\otimes1)}
      \ar[d]_{\theta}
    & f^*(\sh{F}\otimes{\sh{O}_Y}\sh{R}(Y))
      \ar[d]^{\theta}
  \\f^*(\sh{F})\otimes_{\sh{O}_X}\sh{R}(X)
      \ar[r]_{f^*(v)\otimes1}
    & f^*(\sh{F})\otimes_{\sh{O}_X}\sh{R}(X)
  }
\]

We thus easily conclude that $\det f^*(v)$ is the canonical image, under the homomorphism $f^*(\sh{R}(Y))\to\sh{R}(X)$, of the section $\det v$ of $\sh{R}(Y)$;
indeed, we can immediately reduce to the case where $X=\Spec(A)$ and $Y=\Spec(B)$ are affine, with $A$ and $B$ integral rings with fields of fractions $K$ and $L$ (resp.), with the homomorphism $B\to A$ an injection and thus extending to a monomorphism $L\to K$;
if $\sh{F}=\widetilde{M}$, where $M$ is a $B$-module of finite type, then the rank of $M\otimes_B L$ over $L$ is \emph{equal} to that of $(M\otimes_B A)\otimes_A K$ over $K$, and $\det((u\otimes1)\otimes1)$ is the image in $K$ of $\det(u\otimes1)$ for any endomorphism $u$ of $M$, whence the conclusion.
\end{env}

\begin{env}[6.4.10]
\label{II.6.4.10}
Finally, suppose that $X$ is a \emph{locally Noetherian reduced prescheme}, so that the sheaf $\sh{R}(X)$ of rational functions on $X$ is again a quasi-coherent $\sh{O}_X$-module \sref[I]{I.7.3.4};
let $\sh{E}$ be a \emph{coherent} $\sh{O}_X$-module such that $\sh{E}'=\sh{E}\otimes_{\sh{O}_X}\sh{R}(X)$ is \emph{locally free} (of finite rank).
By \sref{II.6.4.7}, if $\sh{E}'$ is of constant rank, then we can, for any endomorphism $u$ of $\sh{E}$, define the $\sigma_j(u)$, which are sections of $\sh{R}(X)$ over $X$.
If we do not suppose $\sh{E}'$ to be of constant rank, we can still define the homomorphism $\det:\shHom_{\sh{O}_X}(\sh{E},\sh{E})\to\sh{R}(X)$.
\end{env}


\subsection{Norm of an invertible sheaf}
\label{subsection:II.6.5}

\begin{env}[6.5.1]
\label{II.6.5.1}
Let $(X,\sh{A})$ be a ringed space and $\sh{B}$ a (commutative) $\sh{A}$-algebra.
The $\sh{A}$-module $\sh{B}$ is canonically identified with a sub-$\sh{A}$-module of $\shHom_{\sh{A}}(\sh{B},\sh{B})$, and a section $f$ of $\sh{B}$ over an open $U$ of $X$ is identified with multiplication by this section.
If $(X,\sh{A})$ and $\sh{B}$ satisfy one of the conditions given in \sref{II.6.4.8}, \sref{II.6.4.9}, or \sref{II.6.4.10}, then we can define $\det(f)$ (and, in certain cases, the $\sigma_j(f)$) as sections of $\sh{A}$ or $\sh{R}(X)$ over $U$, that we call the \emph{norm} of $f$ (resp. the \emph{elementary symmetric functions} of $f$), and denote by $N_{\sh{B}/\sh{A}}(f)$.
We will suppose that \emph{one} of the following conditions is satisfied:
\begin{enumerate}
  \item[(I)] \emph{$\sh{B}$ is a locally free $\sh{A}$-module (of finite rank).}
  \item[(II)] \emph{$(X,\sh{A})$ is a locally Noetherian reduced prescheme, $\sh{B}$ is a coherent $\sh{A}$-module such that $\sh{B}\otimes_{\sh{A}}\sh{R}(X)$ is a locally free $\sh{R}(X)$-module, and, for every section $f\in\Gamma(U,\sh{B})$ over an open $U$ of $X$, the norm $N_{\sh{B}/\sh{A}}(f)$ is a section of $\sh{A}$ over $U$.}
\end{enumerate}

\oldpage[II]{126}
Note that the latter condition is automatically satisfied whenever the locally Noetherian prescheme $X$ is \emph{normal} \sref{II.6.4.9}.

The hypothesis that $\sh{B}\otimes_{\sh{A}}\sh{R}(X)$ is locally free can also be expressed in the following way: denote by $X_\alpha$ the closed reduced subpreschemes of $X$ whose underlying spaces are the irreducible components of $X$ \sref[I]{I.5.2.1}, which are thus locally Noetherian integral preschemes.
Every $x\in X$ belongs to a finite number of the subspaces $X_\alpha$;
on the other hand, $\sh{B}\otimes_{\sh{A}}\sh{R}(X_\alpha)$ is a locally free $\sh{R}(X_\alpha)$-module of constant rank~$k_\alpha$ \sref[I]{I.7.3.6};
to say that $\sh{B}\otimes_{\sh{A}}\sh{R}(X)$ is a locally free $\sh{R}(X)$-module implies that, for all $x\in X$, \emph{the ranks $k_\alpha$ that correspond to the indices such that $x\in X_\alpha$ are equal}.
This is a local statement, and we can reduce to the case $X=\Spec(C)$, where $C$ is a reduced Noetherian ring, and $\sh{B}=\widetilde{D}$, where $D$ is a $C$-algebra that is a $C$-module of finite type;
if $\mathfrak{p}_i$ (for $1\leq i\leq m$) are the minimal prime ideals of $C$, then the total ring of fractions $L$ of $C$ is the direct sum of the fields of fractions $K_i$ of the integral rings $C/\mathfrak{p}_i$, and $D\otimes_C L$ is the direct sum of the $D\otimes_C K_i$, whence the conclusion.

It is clear that, under hypothesis~(I) or (II), we have thus defined a \emph{homomorphism of sheaves of multiplicative monoids} $N_{\sh{B}/\sh{A}}:\sh{B}\to\sh{A}$, also denoted by $N$ if no confusion may arise, and we call this homomorphism the \emph{norm}.
For sections $f$ and $g$ of $\sh{B}$ over the same open $U$, we thus have
\[
\label{II.6.5.1.1}
  N_{\sh{B}/\sh{A}}(fg) = N{\sh{B}/\sh{A}}(f){\sh{B}/\sh{A}}(g)
\tag{6.5.1.1}
\]
for the corresponding sections of $\sh{A}$ over $U$;
\[
\label{II.6.5.1.2}
  N{\sh{B}/\sh{A}}(1_{\sh{B}}) = 1_{\sh{A}};
\tag{6.5.1.2}
\]
finally, for any section $s$ of $\sh{A}$ over $U$,
\[
\label{II.6.5.1.3}
  N{\sh{B}/\sh{A}}(s\cdot1_{\sh{B}}) = s^n
\tag{6.5.1.3}
\]
if the rank of $\sh{B}$ is constant and equal to $n$ (for $s=0_{\sh{A}}$, this formula gives $N(0_{\sh{B}})=0_{\sh{A}}$ if $n\geq1$, and $N(0_{\sh{B}})=N(1_{\sh{B}})=1_{\sh{A}}$ if $n=0$).

In hypothesis~(I), for $f\in\Gamma(U,\sh{B})$ to be invertible it is necessary and sufficient that $N(f)\in\Gamma(U,\sh{A})$ be invertible.
In hypothesis~(II), this condition is necessary;
it is sufficient (by \sref{II.6.4.7}) if we suppose that $\sh{B}\to\sh{B}\otimes_{\sh{A}}\sh{R}(X)$ is \emph{injective} and that the following more restrictive hypothesis is satisfied:
\begin{enumerate}
  \item[(II~\emph{bis})] \emph{$(X,\sh{A})$ is a locally Noetherian reduced prescheme, $\sh{B}$ is a coherent $\sh{A}$-module such that $\sh{B}\otimes_{\sh{A}}\sh{R}(X)$ is a locally free $\sh{R}(X)$-module, and, for any section $f\in\Gamma(U,\sh{B})$ over an open $U$ such that $\sh{B}\otimes_{\sh{A}}\sh{R}(X)|U$ is of constant rank~$n$ on $\sh{R}(X)|U$, the $\sigma_j(f)$ (for $1\leq j\leq n$) are sections of $\sh{A}$ over $U$.}
\end{enumerate}

(We again note that this condition is satisfied if $X$ is \emph{normal}.)
\end{env}

\begin{env}[6.5.2]
\label{II.6.5.2}
Suppose that one of the hypotheses (I) or (II) of \sref{II.6.5.1} are satisfied, and let $\sh{L}'$ be an \emph{invertible} $\sh{B}$-module.
We will canonically associate (up to unique isomorphism) an \emph{invertible} $\sh{A}$-module in the following way.
Denote by $\sh{A}^*$ (resp. $\sh{B}^*$) the subsheaf of $\sh{A}$ (resp. of $\sh{B}$) such that $\Gamma(U,\sh{A}^*)$ (resp. $\Gamma(U,\sh{B}^*)$) is the set of invertible elements of $\Gamma(U,\sh{A})$ (resp. of $\Gamma(U,\sh{B})$) for every open $U\subset X$;
\oldpage[II]{127}
these are sheaves of multiplicative groups, and $N_{\sh{B}/\sh{A}}$ restricted to $\sh{B}^*$ is a \emph{homomorphism} $\sh{B}^*\to\sh{A}^*$ of sheaves of groups \sref{II.6.5.1}.
Let $\mathfrak{L}$ be the set of pairs $(U_\lambda,\nu_\lambda)$ that have the following property: $U_\lambda$ is an open of $X$, and $\nu_\lambda$ is an isomorphism $\sh{L}'|U_\lambda\simto\sh{B}|U_\lambda$ of $(B|U_\lambda)$-modules.
By hypothesis, the $U_\lambda$ form a cover of $X$;
for arbitrary indices $\lambda$ and $\mu$, set $\omega_{\lambda\mu}=(\eta_\lambda|U_\lambda\cap U_\mu)\circ(\nu_\mu|U_\lambda\cap U_\mu)^{-1}$, which is an automorphism of $\sh{B}|U_\lambda\cap U_\mu$ which is canonically identified with a section of $\sh{B}^*$ over $U_\lambda\cap U_\mu$, and $(\omega_{\lambda\mu})$ is a $1$-cocycle for the cover $\mathfrak{U}=(U_\lambda)$ with values in $\sh{B}^*$ \sref[0]{0.5.4.7}.
The fact that $N_{\sh{B}/\sh{A}}:\sh{B}^*\to\sh{A}^*$ is a homomorphism implies that $(N_{\sh{B}/\sh{A}}\omega_{\lambda\mu})$ is a $1$-cocycle of $\mathfrak{U}$ with values in $\sh{A}^*$, and thus corresponds (up to unique isomorphism) to an invertible $\sh{A}$-module $\sh{L}$ \sref[0]{0.5.4.7} which we denote by $N_{\sh{B}/\sh{A}}(\sh{L}')$, and call the \emph{norm} of the invertible $\sh{B}$-module $\sh{L}'$.

Let $\mathfrak{M}$ be a subset of $\mathfrak{L}$ such that the corresponding $U_\lambda$ still form a cover of $X$, and let $\mathfrak{B}$ be the resulting cover;
the restriction of the cocycle $(\omega_{\lambda\mu})$ to $\mathfrak{B}$ defines a $1$-cocycle $(N_{\sh{B}/\sh{A}}\omega_{\lambda\mu})$ of $\mathfrak{B}$ with values in $\sh{A}^*$, the restriction of the $1$-cocycle $(N_{\sh{B}/\sh{A}}\omega_{\lambda\mu})$ of $\mathfrak{U}$;
it is clear that there is a canonical isomorphism from the invertible $\sh{A}$-module defined by this $1$-cocycle of $\mathfrak{B}$ to $N_{\sh{B}/\sh{A}}(\sh{L}')$, which allows us to define the invertible $\sh{A}$-module by an arbitrary sub-cover of $\mathfrak{U}$.
This possibility immediately shows that, if $\sh{L}'_1$ and $\sh{L}'_2$ are invertible $\sh{B}$-modules, then, by \sref{II.6.5.1.1} and \sref{II.6.5.1.2},
\[
\label{II.6.5.2.1}
  N(\sh{L}'_1\otimes_{\sh{B}}\sh{L}'_2) = N(\sh{L}'_1)\otimes_{\sh{A}}N(\sh{L}'_2)
\tag{6.5.2.1}
\]
and
\[
\label{II.6.5.2.2}
  N_{\sh{B}/\sh{A}}(\sh{B}) = \sh{A}
\tag{6.5.2.2}
\]
as well as
\[
\label{II.6.5.2.3}
  N({\sh{L}'}^{-1}) = (N(\sh{L}'))^{-1}
\tag{6.5.2.3}
\]
up to canonical isomorphisms.
Furthermore, it follows from \sref{II.6.5.1.3} that, if $\sh{L}$ is an invertible $\sh{A}$-module = and if $\sh{B}$ is of constant rank~$n$ on $\sh{A}$ in case~(I) (resp. $\sh{B}\otimes_{\sh{A}}\sh{R}(X)$ is of constant rank~$n$ on $\sh{R}(X)$ in case~(II)), we have, up to canonical isomorphism,
\[
\label{II.6.5.2.4}
  N_{\sh{B}/\sh{A}}(\sh{L}\otimes_{\sh{A}}\sh{B}) = \sh{L}^{\otimes n}.
\tag{6.5.2.4}
\]
\end{env}

\begin{env}[6.5.3]
\label{II.6.5.3}
We now show that $N_{\sh{B}/\sh{A}}$ is a covariant \emph{functor} in the category of invertible $\sh{B}$-modules.
Let $h':\sh{L}'_1\to\sh{L}'_2$ be a homomorphism of invertible $\sh{B}$-modules, and let $\mathfrak{B}=(U_\lambda)$ be an open cover of $X$ such that, for all $\lambda$, we have $(\sh{B}|U_\lambda)$-isomorphisms $\eta_\lambda^{(1)}:\sh{L}'_1|U_\lambda\simto\sh{B}|U_\lambda$ and $\eta_\lambda^{(2)}:\sh{L}'_2|U_\lambda\simto\sh{B}|U_\lambda$;
there is thus, for each $\lambda$, an endomorphism $h'_\lambda$ of $\sh{B}|U_\lambda$ such that $h'_\lambda\circ\eta_\lambda^{(1)}=\eta_\lambda^{(2)}\circ(h'|U_\lambda)$, and we can evidently identify $h'_\lambda$ with a section of $\sh{B}$ over $U_\lambda$ \sref[0]{0.5.1.1}.
So, for every pair $(\lambda,\mu)$ of indices, the restrictions to $U_\lambda\cap U_\mu$ of $(\eta_\lambda^{(2)})^{-1}\circ h'_\lambda\circ\eta_\lambda^{(1)}$ and $(\eta_\mu^{(2)})^{-1}\circ h'_\mu\circ\eta_\mu^{(1)}$ agree.
We thus obtain, for $1$-cocycles $(\omega_{\lambda\mu}^{(1)})$ and $(\omega_{\lambda\mu}^{(2)})$ with values in $\sh{B}^*$ corresponding to $\sh{L}'_1$ and $\sh{L}'_2$, the relation
\[
  \omega_{\lambda\mu}^{(2)}h'_\mu = h'_\lambda\omega_{\lambda\mu}^{(1)}.
\]

\oldpage[II]{128}
If we set $h_\lambda=N(h'_\lambda)$, we thus have the analogous relations
\[
  N(\omega_{\lambda\mu}^{(2)})h_\mu = h_\lambda N(\omega_{\lambda\mu}^{(1)})
\]
and so the $h_\lambda$ define a homomorphism $N(\sh{L}'_1)\to N(\sh{L}'_2)$ which we denote by $N_{\sh{B}/\sh{A}}(h')$, or $N(h')$.
In hypothesis~(I), for $h'$ to be an \emph{isomorphism}, it is necessary and sufficient (since it is a local question) for $N_{\sh{B}/\sh{A}}(h')$ to be an isomorphism.
In hypothesis~(II), this condition is again necessary;
it is sufficient if hypothesis~(II~\emph{bis}) is satisfied and $\sh{B}\to\sh{B}\otimes_{\sh{A}}\sh{R}(X)$ is injective.

Take, in particular, $\sh{L}'_1=\sh{B}$;
the homomorphisms $\sh{B}\to\sh{L}'$ can then be identified \sref[0]{0.5.1.1} with the sections of $\sh{L}'$ over $X$, whence a canonical map
\[
  N_{\sh{B}/\sh{A}} : \Gamma(X,\sh{L}')\to\Gamma(X,N_{\sh{B}/\sh{A}}(\sh{L}')).
\]

It again follows from \sref{II.6.5.1.1} that, if $f'_1\in\Gamma(X,\sh{L}'_1)$ and $f'_2\in\Gamma(X,\sh{L}'_2)$, then
\[
\label{II.6.5.3.1}
  N(f'_1\otimes f'_2) = N(f'_1)\otimes N(f'_2).
\tag{6.5.3.1}
\]

For every invertible $\sh{A}$-module $\sh{L}$ and every section $f\in\Gamma(X,\sh{L})$, we have, taking \sref{II.6.5.2.4} into account, that
\[
\label{II.6.5.3.2}
  N_{\sh{B}/\sh{A}}(f\otimes 1_{\sh{B}}) = f^{\otimes n}
\tag{6.5.3.2}
\]
whenever $\sh{B}$ is of constant rank~$n$ in hypothesis~(I) (resp. whenever $\sh{B}\otimes_{\sh{A}}\sh{R}(X)$ is of constant rank~$n$ in hypothesis~(II)).
Finally, for the homomorphism $\sh{B}\to\sh{L}'$ corresponding to a section $f'$ of $\sh{L}'$ over $X$ to be an isomorphism, it is necessary and sufficient for $f'_x$ to be a basis for $\sh{L}'_x$ for all $x\in X$;
in hypothesis~(I), this condition is thus equivalent to saying that $(N(f'))_x$ is a basis for $(N(\sh{L}'))_x$ for all $x$;
in hypothesis~(II), this condition is again necessary, and it is sufficient whenever $\sh{B}$ satisfies hypothesis~(II~\emph{bis}) and $\sh{B}\to\sh{B}\otimes_{\sh{A}}\sh{R}(X)$ is injective.
\end{env}

\begin{env}[6.5.4]
\label{II.6.5.4}
Let $(X,\sh{A})$ and $(X',\sh{A}')$ be ringed spaces, $f:X'\to X$ a morphism, $\sh{B}$ an $\sh{A}$-algebra, and $\sh{B}'=f^*(\sh{B})$ the inverse image $\sh{A}'$-algebra.
Suppose that one of the following hypotheses is satisfied:
\begin{enumerate}
  \item $\sh{B}$ satisfies hypothesis~(I) of \sref{II.6.5.1}.
  \item $(X,\sh{A})$ and $\sh{B}$ satisfy hypothesis~(II) of \sref{II.6.5.1}, $(X',\sh{A}')$ is a locally Noetherian reduced prescheme, and, if we denote by $X_\alpha$ and $X'_\beta$ the closed reduced subpreschemes of $X$ and $X'$ (respectively) that have the irreducible components of these spaces as their underlying spaces, then the restriction of $f$ to each $X'_\beta$ is a \emph{dominant} morphism from $X'_\beta$ to $X_\alpha$.
\end{enumerate}

Under these conditions, $\sh{B}'$ satisfies hypothesis~(I) or hypothesis~(II) (respectively) of \sref{II.6.5.1}.
The first claim is immediate;
to establish the second, it suffices to show that, for all $x'\in X'$, the ranks of the $\sh{B}'\otimes_{\sh{O}_{X'}}\sh{R}(X'_\beta)$ are \emph{the same} for the $\beta$ such that $x'\in X'_\beta$.
But if the restriction of $f$ to $X'_\beta$ is a dominant morphism to $X_\alpha$, then the rank of $\sh{B}'\otimes_{\sh{O}_{X'}}\sh{R}(X'_\beta)$ is equal to that of $\sh{B}\otimes_{\sh{O}_X}\sh{R}(X_\alpha)$ (as we immediately see by reducing to the affine case, as in \sref{II.6.4.9}), whence our claim, by hypothesis~(II) and \sref{II.6.5.1}.

\oldpage[II]{129}
With this in mind, it follows from \sref{II.6.4.8}, \sref{II.6.4.9}, and \sref{II.6.4.10} that, if $s$ is a section of $\sh{B}$ over an open $U\subset X$, and $s'$ the corresponding section of $\sh{B}'$ over $f^{-1}(U)$, then $N_{\sh{B}'/\sh{A}'}(s')$ is the section of $\sh{A}'$ over $f^{-1}(U)$ that corresponds to the section $N_{\sh{B}/\sh{A}}(s)$ of $\sh{A}$ over $U$.

If $\sh{M}$ is an invertible $\sh{B}$-module, then we deduce from the above that, if $\sh{M}'=f^*(\sh{M})$ (which is an invertible $\sh{B}'$-module), we have $N_{\sh{B}'/\sh{A}'}(\sh{M}')=f^*(N_{\sh{B}/\sh{A}}(\sh{M}))$ up to canonical isomorphism.
\end{env}

\begin{env}[6.5.5]
\label{II.6.5.5}
Suppose from now on that $(X,\sh{A})$ is a \emph{prescheme}.
The data of a quasi-coherent $\sh{A}$-algebra $\sh{B}$, which is an \emph{$\sh{A}$-module of finite type}, is then equivalent, as we know, to that of a \emph{finite} morphism $g:X'\to X$ such that $g_*(\sh{O}_{X'})=\sh{B}$, defined up to $X$-isomorphism (\sref{II.6.1.2} and \sref{II.1.3.1}).
Furthermore, the data of a quasi-coherent $\sh{O}_{X'}$-module $\sh{F}'$ is equivalent to that of a quasi-coherent $\sh{B}$-module $\sh{F}$ such that $g_*(\sh{F})=\sh{F}$ \sref{II.1.4.3}, and for $\sh{F}'$ to be invertible it is necessary and sufficient that $\sh{F}$ be invertible \sref{II.6.1.12}.
To translate the above results in terms of finite morphisms $g$, it will be necessary to suppose either that $g_*(\sh{O}_{X'})$ is a \emph{locally free} $\sh{O}_X$-module (of finite type) or that $(X,\sh{O}_X)$ and $g_*(\sh{O}_{X'})$ satisfy hypothesis~(II).
For every invertible $\sh{O}_{X'}$-module $\sh{L}'$, we thus set
\[
\label{II.6.5.5.1}
  N_{X'/X}(\sh{L}') = N_{g_*(\sh{O}_{X'})/\sh{O}_X}(g_*(\sh{L}'))
\tag{6.5.5.1}
\]
and we call this the \emph{norm} (with respect to $g$) of $\sh{L}'$.
Similarly, if $h':\sh{L}'_1\to\sh{L}'_2$ is a homomorphism of invertible $\sh{O}_{X'}$-modules, then we set
\[
\label{II.6.5.5.2}
  N_{X'/X}(h') = N_{g_*(\sh{O}_{X'})/\sh{O}_X}(g_*(h')) : N_{X'/X}(\sh{L}'_1)\to N_{X'/X}(\sh{L}'_2).
\tag{6.5.5.2}
\]

In particular, for $\sh{L}'_1=\sh{O}_{X'}$, we thus obtain a canonical map
\[
\label{II.6.5.5.3}
  N_{X'/X} : \Gamma(X',\sh{L}') \to \Gamma(X,N_{X'/X}(\sh{L}')).
\tag{6.5.5.3}
\]

We leave to the reader the majority of these translations, and we restrict ourselves to spelling out the details of the following:
\end{env}

\begin{proposition}[6.5.6]
\label{II.6.5.6}
Let $g:X'\to X$ be a finite morphism, and suppose that either $g_*(\sh{O}_{X'})$ is a locally free $\sh{O}_X$-module or that $(X,\sh{O}_X)$ and $g_*(\sh{O}_{X'})$ satisfy (II~\emph{bis}) (which will be the case, in particular, if $X$ is locally Noetherian and normal).
For a homomorphism $h':\sh{L}'_1\to\sh{L}'_2$ of invertible $\sh{O}_{X'}$-modules to be an isomorphism, it is necessary and sufficient, in the first hypothesis, that $N_{X'/X}(h')$ be an isomorphism;
in the second hypothesis, this condition is again necessary, and is sufficient if the homomorphism $g_*(\sh{O}_{X'})\to g_*(\sh{O}_{X'})\otimes_{\sh{O}_X}\sh{R}(X)$ is injective.
\end{proposition}

\begin{proof}
Note that we use here the fact that, for $g_*(h')$ to be an isomorphism, it is necessary and sufficient for $h'$ to be an isomorphism \sref{II.1.4.2}.
\end{proof}

\begin{corollary}[6.5.7]
\label{II.6.5.7}
Let $g:X'\to X$ be a finite morphism, and suppose that either $g_*(\sh{O}_{X'})$ is a locally free $\sh{O}_X$-module or that $(X,\sh{O}_X)$ and $g_*(\sh{O}_{X'})$ satisfy (II~\emph{bis}) and $g_*(\sh{O}_{X'})\to g_*(\sh{O}_{X'})\otimes_{\sh{O}_X}\sh{R}(X)$ is injective.
Let $\sh{L}'$ be an invertible $\sh{O}_{X'}$-module, $f'$ a section of $\sh{L}'$ over $X'$, and $f=N_{X'/X}(f')$ the corresponding section of $\sh{L}=N_{X'/X}(\sh{L}')$ over $X$ \sref{II.6.5.5.1}.
\oldpage[II]{130}
Then $g(X'\setminus X'_{f'})=X\setminus X_f$, and $X_f$ is the largest open subset $U$ of $X$ such that $g^{-1}(U)\subset X'_{f'}$.
\end{corollary}

\begin{proof}
Indeed, $g(X'\setminus X'_{f'})$ is closed in $X$ \sref{II.6.1.10};
it thus suffices to prove the last claim.
But the relation $U\subset X_f$ is equivalent to the fact that the homomorphism $\sh{O}_X|U\to\sh{L}|U$ defined by $f|U$ is an isomorphism.
By \sref{II.6.5.6}, this is equivalent to saying that the homomorphism $\sh{O}_{X'}|g^{-1}(U)\to\sh{L}'|g^{-1}(U)$ defined by $f'|g^{-1}(U)$ is an isomorphism, which is equivalent to the relation $g^{-1}(U)\subset X'_{f'}$.
\end{proof}

\begin{proposition}[6.5.8]
\label{II.6.5.8}
Let $g:X'\to X$ be a finite morphism, and $f:Y\to X$ a morphism;
let $Y'=X'_{(Y)}$, $g'=g_{(Y)}$, and $f'=f_{(X')}$, so that the diagram
\[
  \xymatrix{
    X' \ar[d]_{g}
    & Y' \ar[l]_{f'} \ar[d]^{g'}
  \\X
    & Y \ar[l]^{f}
  }
\]
commutes.

Suppose that either $g_*(\sh{O}_{X'})$ is a locally free $\sh{O}_X$-module or that $(X,\sh{O}_X)$ and $g_*(\sh{O}_{X'})$ satisfy (II).
Suppose further that $Y$ is a locally Noetherian reduced prescheme, and that the restriction of $f$ to any irreducible component of $Y$ is a dominant morphism to an irreducible component of $X$.
Then, for every invertible $\sh{O}_{X'}$-module $\sh{L}'$, we have
\[
  N_{Y'/Y}({f'}^*(\sh{L}')) = f^*(N_{X'/X}(\sh{L}'))
\]
up to canonical isomorphism.
\end{proposition}

\begin{proof}
Note that we have $f^*(g_*(\sh{L}'))=g'_*({f'}^*(\sh{L}'))$, by \sref{II.1.5.2}, and, in particular, $g'_*(\sh{O}_{Y'})=f^*(g_*(\sh{O}_{X'}))$;
if $g_*(\sh{O}_{X'})$ is locally free, then so too is $g'_*(\sh{O}_{Y'})$.
The conclusion then follows from the definitions and \sref{II.6.5.4}.
\end{proof}

\begin{remark}[6.5.9]
\label{II.6.5.9}
We later generalise the notion of norm developed above, by placing it relation to the notion of direct image of a divisor.
\end{remark}


\subsection{Application: criteria for ampleness}
\label{subsection:II.6.6}

\begin{proposition}[6.6.1]
\label{II.6.6.1}
Let $Y$ be a prescheme, $f:X\to Y$ a quasi-compact morphism, and $g:X'\to X$ a finite and surjective morphism.
Suppose that either $g_*(\sh{O}_{X'})$ is a locally free $\sh{O}_X$-module or that $(X,\sh{O}_X)$ and $g_*(\sh{O}_{X'})$ satisfy (II~\emph{bis}).
Then, for every invertible $\sh{O}_{X'}$-module $\sh{L}'$ that is ample for $f\circ g$, $\sh{L}=N_{X'/X}(\sh{L}')$ is ample for $f$.
\end{proposition}

\begin{proof}
We can suppose $Y$ to be affine \sref{II.4.6.4}, and then, by \sref{II.4.6.6}, the statement is equivalent to:
\end{proof}

\begin{corollary}[6.6.2]
\label{II.6.6.2}
Let $X$ be a quasi-compact prescheme, $g:X'\to X$ a finite surjective morphism, such that either $g_*(\sh{O}_{X'})$ is a locally free $\sh{O}_X$-module or that $(X,\sh{O}_X)$ and $g_*(\sh{O}_{X'})$ satisfy (II~\emph{bis}).
Then, for every ample $\sh{O}_{X'}$-module $\sh{L}'$, $\sh{L}=N_{X'/X}(\sh{L}')$ is ample.
\end{corollary}

\begin{proof}
In the second hypothesis, we can further suppose that the canonical homomorphism $g_*(\sh{O}_{X'})\to g_*(\sh{O}_{X'})\otimes_{\sh{O}_X}\sh{R}(X)$ is \emph{injective}.
\oldpage[II]{131}
Indeed, if not, let $\sh{T}$ be the kernel of this homomorphism, which is a coherent ideal of $\sh{B}=g_*(\sh{O}_{X'})$ \sref[I]{I.6.1.1}, and set $X''=\Spec(\sh{B}/\sh{T})$;
we thus have a commutative diagram
\[
  \xymatrix{
    X'' \ar[rr]^{h} \ar[dr]_{g'}
    && X' \ar[dl]_{g}
  \\&X
  }
\]
where $h$ is a closed immersion \sref{II.1.4.10}.
Furthermore, we know that the support of $\sh{T}$ is a closed set \sref[0]{0.5.2.2} that is rare in $X$ \sref[I]{I.7.4.6}, whence we conclude that, for the generic point $x$ of an irreducible component of $X$, there is an affine open neighbourhood $U$ of $x$ such that $\sh{B}|U=(\sh{B}/\sh{T})|U$.
Since $g$ is, by hypothesis, surjective, we thus conclude that $x\in g'(X'')$;
$g'$ is thus dominant, and, since it is a finite morphism, it is \emph{surjective} \sref{II.6.1.10};
by definition,
\[
  g'_*(\sh{O}_{X''})\otimes_{\sh{O}_X}\sh{R}(X)
  = (\sh{B}/\sh{T})\otimes_{\sh{O}_X}\sh{R}(X)
  = g_*(\sh{O}_{X'})\otimes_{\sh{O}_X}\sh{R}(X)
\]
thus $(X,\sh{O}_X)$ and $g'_*(\sh{O}_{X''})$ satisfy (II~\emph{bis}), and furthermore $g'_*(\sh{O}_{X''})\to g'_*(\sh{O}_{X''})\otimes_{\sh{O}_X}\sh{R}(X)$ is injective.
Finally, $h^*(\sh{L}')=\sh{L}''$ is an ample $\sh{O}_{X''}$-module \sref{II.4.6.13}[(i~\emph{bis})], and we have that $N_{X''/X}(\sh{L}'')=N_{X'/X}(\sh{L}')$.
Indeed, to define these two invertible $\sh{O}_X$-modules, we can use the same affine open cover $(U_\lambda)$ of $X$ such that the restrictions of $g_*(\sh{L}')$ and $g'_*(\sh{L}'')$ to $U_\lambda$ are isomorphic to $\sh{B}|U_\lambda$ and $(\sh{B}/\sh{T})|U_\lambda$ (respectively).
We immediately see that, for every isomorphism $\eta_\lambda:g_*(\sh{L}')|U_\lambda\to\sh{B}|U_\lambda$, there is a canonically corresponding isomorphism
\[
  \eta'_\lambda : g'_*(\sh{L}'')|U_\lambda \to (\sh{B}/\sh{T})|U_\lambda
\]
so that, if $(\omega_{\lambda\mu})$ and $(\omega'_{\lambda\mu})$ are the $1$-cocycles corresponding to the systems of isomorphisms $(\eta_\lambda)$ and $(\eta'_\lambda)$ \sref{II.6.5.2}, $\omega'_{\lambda\mu}$ is the canonical image in $\Gamma(U_\lambda\cap U_\mu,\sh{B}/\sh{T})$ of $\omega_{\lambda\mu}\in\Gamma(U_\lambda\cap U_\mu,\sh{B})$.
By the definition of $\sh{T}$, we thus conclude that
\[
  N_{\sh{B}/\sh{A}}(\omega_{\lambda\mu}) = N_{(\sh{B}/\sh{T})/\sh{A}}(\omega'_{\lambda\mu})
\]
(where $\sh{A}=\sh{O}_X$), whence the claimed equality.

So suppose that the homomorphism $g_*(\sh{O}_{X'})\to g_*(\sh{O}_{X'})\otimes_{\sh{O}_X}\sh{R}(X)$ is injective when we are in hypothesis~(II~\emph{bis}).
It suffices to prove that, as $f$ runs over the sections of $\sh{L}^{\otimes n}$ (for $n>0$) over $X$, the $X_f$ form a base for the topology of $X$ \sref{II.4.5.2}.
But let $x\in X$, and let $U$ be an arbitrary neighbourhood of $x$;
since $g^{-1}(x)$ is finite \sref{II.6.1.7} and $\sh{L}'$ is ample, there exists an integer $n>0$ and a section $f'$ of ${\sh{L}'}^{\otimes n}$ over $X'$ such that $X'_{f'}$ is a neighbourhood of $g^{-1}(x)$ contained inside $g^{-1}(U)$ \sref{II.4.5.4}.
Since
\[
  \sh{L}^{\otimes n} = N_{X'/X}({\sh{L}'}^{\otimes n})
\]
it suffices to take $f=N_{X'/X}(f')$;
indeed, then $X\setminus X_f=g(X'\setminus X'_{f'})$ \sref{II.6.5.7}, and so $x\in X_f\subset U$.
\end{proof}

\begin{corollary}[6.6.3]
\label{II.6.6.3}
Under the hypotheses of \sref{II.6.6.1}, for an invertible $\sh{O}_X$-module $\sh{L}$ to be ample for $f$, it is necessary and sufficient that $\sh{L}'=g^*(\sh{L})$ be ample for $f\circ g$.
\end{corollary}

\begin{proof}
The condition is necessary, since $g$ is affine \sref{II.5.1.12}.
To prove that the condition is sufficient, we can suppose that $Y$ is affine \sref{II.4.6.4}, and so $X$ and $X'$ are quasi-compact and $\sh{L}'$ is ample \sref{II.4.6.6}, and we need to show that $\sh{L}$ is ample.
\oldpage[II]{132}
But the set of points $x\in X$ that admit a neighbourhood where $g_*(\sh{O}_{X'})$ (resp. $g_*(\sh{O}_{X'})\otimes_{\sh{O}_X}\sh{R}(X)$) has a given rank~$n$ in the first (resp. second) hypothesis is simultaneously open and closed in $X$, and so $X$ is the prescheme given by the sum of a finite number of these opens, and we can thus suppose that it is equal to one of them \sref{II.4.6.17}.
But then $N_{X'/X}(\sh{L}')=\sh{L}^{\otimes n}$, and so $\sh{L}^{\otimes n}$ is ample by \sref{II.6.6.2}, and thus so too is $\sh{L}$ \sref{II.4.5.6}.
\end{proof}

\begin{corollary}[6.6.4]
\label{II.6.6.4}
Suppose that the hypotheses of \sref{II.6.6.1} are satisfied, and suppose further that $f:X\to Y$ is of finite type.
Then, for $f$ to be quasi-projective, it is necessary and sufficient that $f\circ g$ be quasi-projective.
If we further suppose that $Y$ is a quasi-compact scheme, or a prescheme whose underlying space is Noetherian, then for $f$ to be projective, it is necessary and sufficient that $f\circ g$ be projective.
\end{corollary}

\begin{proof}
The hypothesis implies that $f\circ g$ is of finite type.
Taking into account the definition of quasi-projective morphisms \sref{II.5.3.1}, the first claim follows from \sref{II.6.6.1} and \sref{II.6.6.3}.
Taking into account this result, along with \sref{II.5.5.3}[(ii)], it remains to show that, if $f$ is quasi-projective, then for $f$ to be proper, it is necessary and sufficient that $f\circ g$ be proper.
But $f$ is then separated \sref{II.5.3.1} and of finite type;
since $g$ is surjective, our claim follows from \sref{II.5.4.2}[(ii)] and \sref{II.5.4.3}[(ii)].
\end{proof}

In particular:

\begin{corollary}[6.6.5]
\label{II.6.6.5}
Let $X$ be a prescheme of finite type over a field $K$, and $K'$ a finite-degree extension of $K$.
For $X$ to be projective (resp. quasi-projective) over $K$, it is necessary and sufficient that $X'=X\otimes_K K'$ be projective (resp. quasi-projective) over $K'$.
\end{corollary}

\begin{proof}
The condition is necessary (\sref{II.5.3.4}[(iii)] and \sref{II.5.5.5}[(iii)]).
Conversely, suppose that it is satisfied, and let $g:X'\to X$ be the canonical projection.
It is clear that $g$ is a finite morphism \sref{II.6.1.5}[(iii)] and surjective \sref[I]{I.3.5.2}[(ii)].
Furthermore, $g_*(\sh{O}_{X'})$ is a locally free $\sh{O}_X$-module, since it is isomorphic to $\sh{O}_X\otimes_K K'$ \sref{II.1.5.2}.
It then follows from the hypothesis and from \sref{II.6.1.11} and \sref{II.5.5.5}[(ii)] that $X'$ is projective (resp. quasi-projective) \emph{over $K$};
we then deduce from \sref{II.6.6.4} that $X$ is projective (resp. quasi-projective) over $K$.
\end{proof}

In Chapter~V, we will show that the statement of \sref{II.6.6.5} remains true when $K'$ is an \emph{arbitrary} extension of $K$.

The rest of this section is dedicated to the proof of the criterion in \sref{II.6.6.11}, which is a rather technical refinement of \sref{II.6.6.1};
it can be omitted on a first reading.

\begin{lemma}[6.6.6]
\label{II.6.6.6}
Let $X$ be a reduced Noetherian prescheme, and $\sh{E}$ a coherent $\sh{O}_X$-module such that $\sh{E}\otimes_{\sh{O}_X}\sh{R}(X)$ is a locally free $\sh{R}(X)$-module of constant rank~$n$.
Then there exists a reduced Noetherian prescheme $Z$ and a birational finite morphism $h:Z\to X$ that has the following property: the morphisms of sheaves of sets $\sigma_i:\shHom_{\sh{O}_X}(\sh{E},\sh{E})\to\sh{R}(X)$ (for $1\leq i\leq n$) (cf. \sref{II.6.4.10}) send $\shHom_{\sh{O}_X}(\sh{E},\sh{E})$ to the coherent $\sh{O}_X$-algebra $h_*(\sh{O}_Z)$.
\end{lemma}

\begin{proof}
Consider an affine open $U$ of $X$, of ring $A(U)=A$;
let $E=\Gamma(U,\sh{E})$, and let $C_U$ be the subalgebra of $R(U)$ generated by the $\sigma_i(u)$ where $u$ runs over $\Hom_A(E,E)$;
we have seen \sref{II.6.4.7.1} that this $A$-algebra is of \emph{finite rank}.
Furthermore, it is clear that forming the algebras $C_U$ commutes with the restriction operations of an affine open $U$ to an affine open $U'\subset U$.
We have thus defined a \emph{finite} sub-$\sh{O}_X$-algebra $\sh{C}$ of $\sh{R}(X)$ such that $\Gamma(U,\sh{C})=C_U$ for every affine open $U$ of $X$.
We take $Z=\Spec(\sh{C})$, and take $h$ to be the structure morphism, which is thus finite \sref{II.6.1.2};
since $\sh{C}$ is reduced, $Z$ is a reduced Noetherian prescheme \sref{II.1.3.8}.
Finally, the total ring of fractions of $C_U$ is $R(U)$, by definition, and since $C_U$ is contained inside the integral closure of $A(U)$ in $R(U)$, there is a bijective correspondence between minimal prime ideals of $A(U)$ and minimal prime ideals of $C_U$ \cite[t.~I, p.~259]{I-13}, which proves that $h$ is birational and finishes the proof.
\end{proof}

\begin{corollary}[6.6.7]
\label{II.6.6.7}
Under the hypotheses of \sref{II.6.6.6}, let $W$ be an open of $X$ such that, for all $x\in W$, either $X$ is normal at the point $x$, or $\sh{E}_x$ is a free $\sh{O}_x$-module.
Then we can suppose $h$ to be defined such that the restriction of $h$ to $h^{-1}(W)$ is an isomorphism from $h^{-1}(W)$ to $W$.
\end{corollary}

\begin{proof}
Either hypothesis implies that, if $U\subset W$ is an affine open, then, with the notation of \sref{II.6.6.6}, $(\sigma_i(u))_x\in A_x$ for all $x\in U$ \sref{II.6.4.3}, and so $\sigma_i(u)\in A$, and the conclusion follows from the definition of $h$ given in \sref{II.6.6.6}.
\end{proof}

\begin{env}[6.6.8]
\label{II.6.6.8}
Let $X$ be a reduced Noetherian prescheme, and $g:X'\to X$ a finite surjective morphism, so that $\sh{B}=g_*(\sh{O}_{X'})$ is a \emph{coherent} $\sh{O}_X$-algebra;
suppose further that $\sh{B}\otimes_{\sh{O}_X}\sh{R}(X)$ is a \emph{locally free} $\sh{R}(X)$-module \emph{of constant rank~$n$}.
We can then apply Lemma~\sref{II.6.6.6}, taking $\sh{E}=\sh{B}$, whence, with the notation of \sref{II.6.6.6}, we obtain a homomorphism of sheaves of multiplicative monoids $\sigma_n:\shHom_{\sh{O}_X}(\sh{B},\sh{B})\to h_*(\sh{O}_Z)$, and by composing this homomorphism with the canonical homomorphism $\sh{B}\to\shHom_{\sh{O}_X}(\sh{B},\sh{B})$ \sref{II.6.5.1}, we thus obtain a homomorphism of sheaves of multiplicative monoids:
\[
\label{II.6.6.8.1}
  N' : \sh{B} = g_*(\sh{O}_{X'}) \to h_*(\sh{O}_Z) = \sh{C}.
\tag{6.6.8.1}
\]

With this in mind, for every invertible $\sh{O}_{X'}$-module $\sh{L}'$, $g_*(\sh{L}')$ is an invertible $\sh{B}$-module \sref{II.6.1.12}, and the method of \sref{II.6.5.2} allows us to functorially associate to $\sh{L}'$ an invertible $\sh{C}$-module, which we denote by $N'(g_*(\sh{L}'))$.
\end{env}

\begin{lemma}[6.6.9]
\label{II.6.6.9}
Let $X$ be a reduced Noetherian prescheme, and $g:X'\to X$ a finite surjective morphism such that $g_*(\sh{O}_{X'})\otimes_{\sh{O}_X}\sh{R}(X)$ is a locally free $\sh{R}(X)$-module of constant rank.
Then there exists a reduced Noetherian prescheme $Z$ and a finite birational morphism $h:Z\to X$ that has the following property: for every ample $\sh{O}_{X'}$-module $\sh{L}'$, the invertible $\sh{O}_Z$-module $\sh{M}$ such that $h_*(\sh{M})=N'(g_*(\sh{L}'))$ (using the notation of \sref{II.6.6.8}) is ample.
\end{lemma}

\begin{proof}
Suppose first of all that the homomorphism $\sh{B}\to\sh{B}\otimes_{\sh{O}_X}\sh{R}(X)$ is \emph{injective}.
Define $Z$ and $h$ as in \sref{II.6.6.6} (with $\sh{E}=g_*(\sh{O}_{X'})$).
Let $z\in Z$;
we have to show that there exists an integer $m>0$ and a section $t$ of $\sh{M}^{\otimes m}$ over $Z$ such that $Z_t$ is an affine open that contains $z$ \sref{II.4.5.2}.
Let $x=h(z)$, and let $U$ be an affine open of $X$ that contains $x$;
then $h^{-1}(U)$ is an affine open neighbourhood of $z$, and it suffices to find $t$ such that $z\in Z_t\subset h^{-1}(U)$, since $Z_t$ will then necessarily be affine \sref{II.5.5.8}.
There exists, by hypothesis, an integer $n>0$ and a section $s'$ of ${\sh{L}'}^{\otimes n}$ over $X'$ such that
\[
\label{II.6.6.9.1}
  g^{-1}(x) \subset X'_{s'} \subset g^{-1}(U)
\tag{6.6.9.1}
\]
by \sref{II.4.5.4}.
By definition, $s'$ is also a section of $g_*(\sh{L}')$ over $X$, and it corresponds, as in \sref{II.6.5.2}, to a section $s=N'(s')$ of $N'(g_*(\sh{L}'))$ over $X$.
\oldpage[II]{134}
We will show that, if $t$ is the section $s$ considered as a section of $\sh{M}$ over $Z$, then $t$ is the desired section.
Set
\[
\label{II.6.6.9.2}
  V = X\setminus g(X'\setminus X'_{s'})
\tag{6.6.9.2}
\]
which is an open of $X$ that contains $x$ and is contained in $U$, by \sref{II.6.6.9.1} and \sref{II.6.1.10}.
We will show that
\[
\label{II.6.6.9.3}
  h^{-1}(V) \subset Z_t \subset h^{-1}(U)
\tag{6.6.9.3}
\]
which will finish the proof.
It is equivalent to say that the set $T$ of $y\in X$ such that $s_y$ is invertible contains $V$ and is contained in $U$.
For this, consider first of all an affine open $W$ contained in $V$;
then $g^{-1}(W)$ is an affine open in $X'$, and by \sref{II.6.6.9.2} $s'_{y'}$ is invertible for all $y'\in g^{-1}(W)$;
by the hypotheses on $X$ and $\sh{B}$, we can apply the results of \sref{II.6.4.7}, and we see that, if $y=g(y')$, then $s_y$ is invertible;
in other words, $V\subset T$.
On the other hand, it also follows from \sref{II.6.4.7} that, conversely, if $s_y$ is invertible, then so too is $s'_{y'}$, which implies that $y'\in g^{-1}(U)$ by \sref{II.6.6.9.1}, and so $y\in U$, whence $T\subset U$ in this case.

We pass from this to the general case by the same argument as in \sref{II.6.6.2}, replacing $X'$ by $X''$ such that $g'_*(\sh{O}_{X''})\to g'_*(\sh{O}_{X''})\otimes_{\sh{O}_X}\sh{R}(X)$ is injective, and $\sh{L}'$ by an ample $\sh{O}_{X''}$-module $\sh{L}''$ such that $N'(g_*(\sh{L}'))=N'(g'_*(\sh{L}''))$.
Lemma~\sref{II.6.6.9} is then proven in all cases (with a suitable choice of $h$).
\end{proof}

\begin{corollary}[6.6.10]
\label{II.6.6.10}
Suppose that the hypotheses of \sref{II.6.6.9} are satisfied;
for every invertible $\sh{O}_X$-module $\sh{L}$ such that $g^*(\sh{L})$ is ample, $h^*(\sh{L})$ is ample.
\end{corollary}

\begin{proof}
If we set $\sh{L}'=g^*(\sh{L})$, then $g_*(\sh{L}')=\sh{L}\otimes_{\sh{O}_X}\sh{B}$ \sref[0]{0.5.4.10}, so
\[
  N'(g_*(\sh{L}')) = (\sh{L}\otimes_{\sh{O}_X}\sh{C})^{\otimes n}
\]
(by the same argument as for \sref{II.6.5.2.4}).
We thus conclude that $\sh{M}=(h^*(\sh{L}))^{\otimes n}$, and since $\sh{M}$ is ample, so too is $h^*(\sh{L})$ \sref{II.4.5.6}.
\end{proof}

\begin{proposition}[6.6.11]
\label{II.6.6.11}
Let $Y$ be an affine scheme, $X$ an reduced Noetherian prescheme, $f:X\to Y$ a quasi-compact morphism, and $g:X'\to X$ a finite surjective morphism.
Let $W$ be an open subset of $X$ such that, for all $x\in W$, either $X$ is normal at the point $x$ or there exists an open neighbourhood $T\subset W$ of $x$ such that $(g_*(\sh{O}_{X'}))|T$ is a locally free $(\sh{O}_X|T)$-module.
Then there exists a reduced $Y$-prescheme $Z$ and a \emph{finite birational} $Y$-morphism $h:Z\to X$ such that the restriction of $h$ to $h^{-1}(W)$ is an isomorphism from $h^{-1}(W)$ to $W$ and has the following property: for every invertible $\sh{O}_X$-module $\sh{L}$ such that $g^*(\sh{L})$ is ample for $f\circ g$, $h^*(\sh{L})$ is ample for $f\circ h$.
\end{proposition}

\begin{proof}
Since $Y$ is affine, $g^*(\sh{L})$ is ample, and the problem thus reduces to proving that, for a suitable choice of $h$, $h^*(\sh{L})$ is ample \sref{II.4.6.6}.
We will show that we can replace $g$ by a finite surjective morphism $g':X''\to X$ such that ${g'}^*(\sh{L})$ is ample and $g'_*(\sh{O}_{X''})\otimes_{\sh{O}_X}\sh{R}(X)$ is a \emph{locally free} $\sh{R}(X)$-module \emph{of constant rank};
we will then have reduced to the conditions of \sref{II.6.6.10}, and the proposition will thus be proven.

For this, let $\sh{B}=g_*(\sh{O}_{X'})$;
denote by $X_i$ (for $1\leq i\leq n$) the closed reduced subpreschemes of $X$ that have the irreducible components of $X$ as their underlying space \sref[I]{I.5.2.1};
they are integral by hypothesis.
\oldpage[II]{135}
Let $X'_i$ be the closed subprescheme $g^{-1}(X_i)$ of $X'$, and $g_i:X'_i\to X_i$ the morphism $g$ restricted to $X'_i$, which is finite \sref{II.6.1.5}[(iii)] and surjective;
let $k_i$ be the \emph{rank} of the $\sh{O}_{X_i}$-algebra $\sh{B}_i=(g_i)_*(\sh{O}_{X'_i})$.
Since $\sh{B}_i\otimes_{\sh{O}_{X_i}}\sh{R}(X_i)$ is a constant presheaf \sref[I]{I.7.3.5}, the rank $k_i$ is also the rank of the $(\sh{O}_X|U)$-algebra $\sh{B}|U$ for every open $U$ of $X$ that does not meet the \emph{only} irreducible component $X_i$.
If $T$ is an open subset of $X$ such that $\sh{B}|T$ is isomorphic to $\sh{O}_X^m|T$, then it follows from the above remark that the numbers $k_i$ are equal to $m$ for all the indices $i$ such that $T\cap X_i\neq\emp$.
So let $U$ be the open subset of $X$ given by the points that admit a neighbourhood on which $\sh{B}$ is a locally free $\sh{O}_X$-module, and let $U_j$ (for $1\leq j\leq s$) be its connected components, which are open in $X$ and finitely many (since $U$ is Noetherian);
denote by $V_j$ the closed subprescheme of $X'$ given by the \emph{closure} of the subprescheme induced on the open $g^{-1}(U_j)$ \sref[I]{I.9.5.11}.
By the above, for all indices $i$ such that $X_i\cap U_j\neq\emp$, the ranks $k_i$ are all equal to a single integer $m_j$;
note also that a single $X_i$ cannot meet two $U_j$ with different indices $j$.
Let $i_\lambda$ be the indices $i$ such that $X_i\cap U=\emp$.
Consider the product $k$ of all the $k_i$, set $n_i=k/k_i$, and let $X''$ be the prescheme defined as follows.
For each $j$ (for $1\leq j\leq s$), consider $k/m_j$ preschemes isomorphic to $V_j$, and for each $\lambda$, $k/k_{i_\lambda}$ preschemes isomorphic to $X'_{i_\lambda}$;
then $X''$ is the \emph{sum} of all these preschemes.
We define a morphism $g'':X''\to X'$ that reduces to the canonical injection on each of the summands of $X''$;
it is clear that $g''$ is a finite dominant morphism, and thus surjective (since a finite morphism is closed \sref{II.6.1.10});
set $g'=g\circ g''$, which is a finite surjective morphism $X''\to X$;
we have ${g'}^*(\sh{L})={g''}^*(g^*(\sh{L}))$, and so ${g'}^*(\sh{L})$ is an ample $\sh{O}_{X''}$-module \sref{II.5.1.12}.
It is then clear that, for this new prescheme $X''$, the ranks defined as the $k_i$ for $X'$ are all \emph{equal to $k$};
taking \sref[I]{I.7.3.3} into account, we immediately conclude that, for every affine open $T$ of $X$, $(g'_*(\sh{O}_{X''})\otimes_{\sh{O}_X}\sh{R}(X))|T$ is an $(\sh{R}(X)|T)$-module isomorphic to $(\sh{R}(X)|T)^k$.
\end{proof}

\begin{corollary}[6.6.12]
\label{II.6.6.12}
If, in the statement of \sref{II.6.6.11}, we have $W=X$, then for an invertible $\sh{O}_X$-module $\sh{L}$ to be ample for $f$, it is necessary and sufficient that $g^*(\sh{L})$ be ample for $f\circ g$.
\end{corollary}

\begin{remark}[6.6.13]
\label{II.6.6.13}
In Chapter~III, we will see that, if $Y$ is Noetherian, and $f$ of finite type, and if the restriction of $f$ to the closed reduced subprescheme of $X$ that has $X\setminus W$ as its underlying space is \emph{proper}, then the conclusion of \sref{II.6.6.12} still holds true.
But we will give, in Chapter~V, examples of algebraic schemes $X$ over a field $K$ (with the structure morphism $X\to\Spec(K)$ not proper) whose normalisation $X'$ is quasi-affine, but which are not quasi-affine (in that $\sh{O}_X$ is not ample, even though $\sh{O}_{X'}$ is \sref{II.5.1.12} and the morphism $X'\to X$ is finite and surjective \sref{II.6.3.10}).
We will see in the next section that this circumstance cannot happen when we replace ``quasi-affine'' by ``affine''.
\end{remark}


\subsection{Chevalley's theorem}
\label{subsection:II.6.7}

We are going to prove (with the help of Serre's criterion \sref{II.5.2.1}) the following theorem, which was proven by C.~Chevalley by other methods, in the case of \emph{algebraic} schemes.

\oldpage[II]{136}
\begin{theorem}[6.7.1]
\label{II.6.7.1}
Let $X$ be an affine scheme, $Y$ a Noetherian prescheme, and $f:X\to Y$ a finite surjective morphism.
Then $Y$ is an affine scheme.
\end{theorem}

\begin{proof}
It is clear that $f_\red:X_\red\to Y_\red$ is finite \sref{II.6.1.5}[(vi)];
since $X_\red$ is an affine scheme, and since saying that $Y$ is affine is equivalent to saying that $Y_\red$ is affine (since $Y$ is Noetherian \sref[I]{I.6.1.7}), we see that we can suppose $Y$ to be \emph{reduced}.
For every closed subset $Y'$ of $Y$, there exists exactly one reduced subprescheme of $Y$ that has $Y'$ as its underlying space \sref[I]{I.5.1.2};
its inverse image $f^{-1}(Y')$, which is canonically isomorphic to $X\times_Y Y'$ \sref[I]{I.4.4.1}, is affine as a closed subprescheme of $X$, and the restriction of $f$ to $f^{-1}(Y')$, which can be identified with $f\times_Y 1_{Y'}$, is a finite surjective morphism \sref{II.6.1.5}[(iii)].
By the principal of Noetherian induction \sref[0]{0.2.2.2}, we can thus (taking \sref[I]{I.6.1.7} into account) reduce to proving the theorem under the hypothesis that for \emph{every} closed subset $Y'\neq Y$, every closed subprescheme of $Y$ that has $Y'$ as its underlying space is affine.
We thus conclude that, \emph{for every coherent $\sh{O}_Y$-module $\sh{F}$ whose (closed) support $Z$ is distinct from $Y$, we have $\HH^1(Y,\sh{F})=0$}.
Indeed, there exists a closed subprescheme $Y'$ of $Y$ that has $Z$ as its underlying space and is such that, if $j:Y\to Y$ is the canonical injection, then $\sh{F}=j_*(j^*(\sh{F}))$ \sref[I]{I.9.3.5};
then \sref{II.5.2.3} $\HH^1(Y,\sh{F})=\HH^1(Y',j^*(\sh{F}))=0$, by \sref[I]{I.5.1.9.2}.

Suppose first of all that $Y$ is not irreducible, and let $Y'$ be an irreducible component of $Y$, and $Y''=Y\setminus Y'$;
we again denote by $Y'$ the closed reduced subprescheme of $Y$ that has $Y'$ as its underlying space, and by $j$ the canonical injection $Y'\to Y$.
Let $\sh{F}$ be a coherent $\sh{O}_Y$-module, and consider the canonical homomorphism
\[
  \rho : \sh{F}\to\sh{F}' = j_*(j^*(\sh{F}))
\]
\sref[0]{0.4.4.3};
$\sh{F}'$ is a coherent $\sh{O}_Y$-module by \sref[0]{0.5.3.10} and \sref[0]{0.5.3.12}, since $j_*(\sh{O}_{Y'})=\sh{O}_Y/\sh{J}$, where we denote by $\sh{J}$ the sheaf of ideals of $\sh{O}_Y$ that defines the subprescheme $Y'$.
Then $\sh{G}=\Ker\rho$ and $\sh{K}=\Im\rho$ are also coherent $\sh{O}_Y$-modules \sref[0]{0.5.3.4};
but, by definition, the fibre $\sh{F}'_y$ of $\sh{F}'$ at the generic point $y$ of $Y'$ is equal to $\sh{F}_y$, since $y$ is interior to $Y'$ and thus $\sh{J}_y=0$, since $Y$ is reduced.
We thus conclude that $y$ is not contained in the (closed) support of $\sh{G}$;
also, the support of $\sh{F}'$ (and \emph{a fortiori} that of $\sh{K}$) is contained in $Y'$;
in other words, the supports of $\sh{G}$ and $\sh{K}$ are \emph{distinct from $Y$}.
We thus deduce that $\HH^1(Y,\sh{G})=\HH^1(Y,\sh{K})=0$, and the exact sequence of cohomology applied to the exact sequence $0\to\sh{G}\to\sh{F}\to\sh{K}\to0$ implies that $\HH^1(Y,\sh{F})=0$.
We thus conclude by Serre's criterion \sref{II.5.2.1}.

So suppose that $Y$ is irreducible, and thus \emph{integral}.
We can also suppose that $X$ is \emph{integral}:
if we denote by $X_i$ the closed reduced subpreschemes of $X$ that have the irreducible components of $X$ as their underlying space \sref[I]{I.5.2.1}, and by $g_i$ the restriction of $g$ to $X_i$, then at least one of the $g_i$ is dominant, and since it is a finite morphism \sref{II.6.1.5}, it is surjective \sref{II.6.1.10};
since $X_i$ is also an affine scheme, we see that we can replace $X$ by $X_i$ in the statement.

  \begin{lemma}[6.7.1.1]
  \label{II.6.7.1.1}
  Let $X$ and $Y$ be integral Noetherian preschemes, $x$ (resp. $y$) the generic point of $X$ (resp. $Y$), and $f:X\to Y$ a finite surjective morphism.
  Let $\sh{L}$ be an invertible $\sh{O}_X$-module such that there exists an affine open neighbourhood $U$ of $y$ and a section $g\in\Gamma(X,\sh{L})$ such that $x\in X_g\subset f^{-1}(U)$.
\oldpage[II]{137}
  Then there exist integers $m,n>0$, a homomorphism $u:\sh{O}_Y^m\to f_*(\sh{L}^{\otimes n})$, and an open neighbourhood $V$ of $y$ such that the restriction $u|V$ is an isomorphism from $\sh{O}_Y^m|V$ to $f_*(\sh{L}^{\otimes n})|V$.
  \end{lemma}

  \begin{proof}
  Let $C$ be the (integral) ring of $U$, and $k=\sh{O}_y$ its field of fractions:
  since $f$ is finite, $U'=f^{-1}(U)$ is affine \sref{II.1.3.2};
  let $D$ be its (integral) ring with field of fractions $K=\sh{O}_x$;
  by hypothesis, $D$ is a $C$-module of finite type \sref{II.6.1.4}, and so $K$ is an extension of finite rank of $k$.
  The fibre $f^{-1}(y)=X\times_Y\Spec(\kres(y))=X\times_Y\Spec(\sh{O}_y)$ can be identified with $\Spec(K)$ \sref[I]{I.3.6.6};
  let $s_i$ (for $1\leq i\leq m$) be elements of $D$ that form a basis of $K$ over $k$.
  There exists $n>0$ such that the sections $(s_i|X_g)g^{\otimes n}$ of $\sh{L}^{\otimes n}$ over $X_g$ extend to sections $b_i$ (for $1\leq i\leq m$) of $\sh{L}^{\otimes n}$ over $X$ \sref[I]{I.9.3.1}.
  The $b_i$ are also, by definition, sections of $f_*(\sh{L}^{\otimes n})$ over $Y$, and thus define a homomorphism $u:\sh{O}_Y^m\to f_*(\sh{L}^{\otimes n})$ \sref[0]{0.5.1.1};
  we will show that $u$ is the desired homomorphism.
  We have that $\sh{L}^{\otimes n}|U'=\widetilde{M}$, where $M$ is a $D$-module of finite type, so if $\vphi$ is the injection $C\to D$ corresponding to the morphism $f^{-1}(U)\to U$ given by the restriction of $f$, then $M_{[\vphi]}$ is a $C$-module of finite type;
  then
  \[
    f_*(\sh{L}^{\otimes n})|U = (M_{[\vphi]})\supertilde
  \]
  \sref[I]{I.1.6.3} is coherent, and since $U$ is an arbitrary affine open of $Y$, $f_*(\sh{L}^{\otimes n})$ is coherent;
  furthermore, $u|U=\widetilde{\theta}$, where $\theta$ is a $C$-homomorphism $C^m\to M_{[\vphi]}$, and $u_y=\theta_y$ is the homomorphism $\theta\otimes1:K^m=C^m\otimes K\to M_{[\vphi]}\otimes K$;
  but the latter is, by definition, an \emph{isomorphism}, since the $(b_i)_x$ form a basis of $M_{[\vphi]}\otimes K$ over $k$, with $g_x$ being, by hypothesis, a generator of $(\sh{L}^{\otimes n})_x$.
  We thus conclude that the supports of the $\sh{O}_Y$-modules $\Ker u$ and $\Coker u$ do not contain $y$;
  since these $\sh{O}_Y$-modules are coherent \sref[0]{0.5.3.3}, their supports are closed \sref[0]{0.5.2.2}, whence the lemma.
  \end{proof}

With this, the hypotheses of Lemma~\sref{II.6.7.1.1} are satisfied in the case that we are considering, by taking $\sh{L}=\sh{O}_X$, since $X$ is affine \sref[I]{I.1.1.10};
set $\sh{A}=\sh{O}_Y$ and $\sh{B}=f_*(\sh{O}_X)$.
By Serre's criterion \sref{II.5.2.1.1}, it suffices to prove that, for every coherent $\sh{O}_Y$-module $\sh{F}$, we have that $\HH^1(Y,\sh{F})=0$;
it even suffices to prove this in the case where $\sh{F}\subset\sh{O}_Y$, which implies that $\sh{F}$ is torsion free, since $Y$ is integral;
in fact, we will show that $\HH^1(Y,\sh{O}_Y)=0$ for \emph{every} coherent \emph{torsion-free} $\sh{O}_Y$-module $\sh{F}$.
But the homomorphism $u:\sh{A}^m\to\sh{B}$ defines a homomorphism
\[
  v : \sh{G} = \shHom_{\sh{A}}(\sh{B},\sh{F}) \to \shHom_{\sh{A}}(\sh{A}^m,\sh{F}) = \sh{F}^m.
\]

We will first show that $v$ is \emph{injective}:
by hypothesis, $\sh{T}=\Coker u$ has a support that does not meet $V$, and is thus a \emph{torsion} $\sh{O}_Y$-module \sref[I]{I.7.4.6};
the exact sequence
\[
  \sh{A}^m\to\sh{B}\to\sh{T}\to0
\]
gives, by left exactness of the functor $\shHom_{\sh{A}}$, the exact sequence
\[
  0\to\shHom_{\sh{A}}(\sh{T},\sh{F})\to\sh{G}\xrightarrow{v}\sh{F}^m.
\]
But since $\sh{F}$ is torsion free, we have that $\shHom_{\sh{A}}(\sh{T},\sh{F})=0$ \sref[0]{0.5.2.6}, whence our claim.
We thus have the exact sequence
\[
  0\to\sh{G}\to\sh{F}^m\to\Coker v\to0
\]
\oldpage[II]{138}
where $\sh{G}$ and $\Coker v$ are coherent $\sh{O}_Y$-modules (\sref[0]{0.5.3.4} and \sref[0]{0.5.3.5});
by the exact sequence of cohomology, it would suffice to show that $\HH^1(Y,\sh{G})=\HH^1(Y,\Coker v)=0$ in order to deduce that $\HH^1(Y,\sh{F}^m)=(\HH^1(Y,\sh{F}))^m=0$, and thus that $\HH^1(Y,\sh{F})=0$.
But the restriction $v|V$ is an isomorphism, and so the support of $\Coker v$ is distinct from $Y$, whence $\HH^1(Y,\Coker v)=0$ by the hypothesis at the start.
Now, $\sh{G}$ is a coherent \emph{$\sh{B}$-module} \sref[I]{I.9.6.4};
since $X$ is affine over $Y$, there exists a quasi-coherent $\sh{O}_X$-module $\sh{K}$ such that $\sh{G}$ is isomorphic to $f_*(\sh{K})$ \sref{II.1.4.3};
since $X$ is affine, we have that $\HH^1(X,\sh{K})=0$ \sref[I]{I.5.1.9.2}, and so $\HH^1(Y,\sh{G})=0$ by \sref{II.5.2.3}, which finishes the proof of Theorem~\sref{II.6.7.1}.
\end{proof}

\begin{corollary}[6.7.2]
\label{II.6.7.2}
Let $X$ be a Noetherian prescheme, $(X_i)_{1\leq i\leq n}$ a finite cover of the space $X$ consisting of closed subsets.
For $X$ to be affine, it is necessary and sufficient that, for each $i$, there exist a closed subprescheme of $X$ that has $X_i$ as its underlying space and is affine.
\end{corollary}

\begin{proof}
Indeed, if this is the case, then let $X'$ be the scheme given by the \emph{sum} of the $X_i$;
it is clear that $X'$ is affine, and we define a surjective morphism $f:X'\to X$ by taking the restriction of $f$ to $X_i$ to be the canonical injection.
Everything reduces to showing that $f$ is \emph{finite}, by \sref{II.6.7.1}, and this we have already seen in \sref{II.6.1.5}.
\end{proof}

\begin{corollary}[6.7.3]
\label{II.6.7.3}
For a Noetherian prescheme $X$ to be affine, it is necessary and sufficient that the closed reduced subpreschemes whose underlying spaces are the irreducible components of $X$ be affine.
\end{corollary}

\section{Valuative criteria}
\label{section:II.7}

In this section, we give valuative criteria for separation and properness for a given morphism, that is, criteria which introduce a variable auxiliary scheme of the form $\Spec(A)$, where $A$ is a valuation ring.
Under certain suitable ``Noetherian'' hypotheses, we can refine our criteria and restrict to the case where $A$ is a \emph{discrete} valuation ring.
This will be the only case that we need to concern ourselves with in all that follows, and we introduce arbitrary valuation rings, in the general case, only to discuss the links with the classical study of such objects.


\subsection{Reminder on valuation rings}
\label{subsection:II.7.1}

\begin{env}[7.1.1]
\label{II.7.1.1}
Amongst the many diverse equivalent properties that characterise valuation rings, we will use the following: a ring $A$ is said to be a \emph{valuation ring} if it is an integral ring which is not a field, and $A$ is \emph{maximal} in the set of local rings strictly contained in the field of fractions $K$ of $A$ under the domination relation \sref[I]{I.8.1.1}.
Recall that a valuation ring is \emph{integrally closed}.
If $A$ is a valuation ring, then so too is $A_\mathfrak{p}$ for any prime ideal $\mathfrak{p}\neq0$ of $A$.
\end{env}

\begin{env}[7.1.2]
\label{II.7.1.2}
Let $K$ be a field, and $A$ a local subring of $K$ that is not a field;
\oldpage[II]{139}
then there exists a valuation ring that both dominates $A$ and has $K$ as its field of fractions (\cite[p.~1-07, lemma~2]{I-1}).

Now let $B$ be a valuation ring, $k$ its residue field, $K$ its field of fractions, and $L$ an extension of $k$.
Then there exists a \emph{complete} valuation ring $C$ that dominates $B$ and whose residue field is $L$.
Indeed, $L$ is the algebraic extension of a pure transcendental extension $L'=k(T_\mu)_{\mu\in M}$;
we know that we can extend the valuation of $B$ corresponding to $B$ to a valuation of $K'=K(T_\mu)_{\mu\in M}$ in such a way that $L'$ is the residue field of this valuation (\cite[p.~98]{II-24});
replacing $B$ by the completion of the ring of this extended valuation, we see that that we can restrict to the case where $B$ is complete and $L$ is an algebraic closure of $k$.
If $\overline{K}$ is an algebraic closure of $K$, we can then extend the valuation that defines $B$ to $\overline{K}$, and the corresponding residue field is an algebraic closure of $k$, as we can see by lifting to $\overline{K}$ the coefficients of a unitary polynomial of $k[T]$.
We are thus finally led to the case where $L=k$, and it then suffices to take $C$ to be the completion of $B$ in order to satisfy our claim.
\end{env}

\begin{env}[7.1.3]
\label{II.7.1.3}
Let $K$ be a field, and $A$ a subring of $K$;
the integral closure $A'$ of $A$ in $K$ is the intersection of the valuation rings that contain $A$ and have $K$ as their field of fractions (\cite[p.~51, th.~2]{I-11}).
Proposition~\sref{II.7.1.2} can then be expressed geometrically in an equivalent form:
\end{env}

\begin{proposition}[7.1.4]
\label{II.7.1.4}
Let $Y$ be a prescheme, $p:X\to Y$ a morphism, $x$ a point of $X$, $y=p(x)$, and $y'\neq y$ a specialisation \sref[0]{0.2.1.2} of $y$.
Then there exists a local scheme $Y'$ which is the spectrum of some valuation ring, and a separated morphism $f:Y'\to Y$ such that, denoting the unique closed point of $Y'$ by $a$ and the generic point of $Y'$ by $b$, we have $f(a)=y'$ and $f(b)=y$.
We can furthermore suppose that one of the two additional following properties are satisfied:
\begin{enumerate}
    \item[\rm{(i)}] $Y'$ is the spectrum of a complete valuation ring whose residue field is algebraically closed, and there exists a $\kres(y)$-homomorphism $\kres(x)\to\kres(b)$.
    \item[\rm{(ii)}] There exists a $\kres(y)$-isomorphism $\kres(x)\simto\kres(b)$.
\end{enumerate}
\end{proposition}

\begin{proof}
Let $Y_1$ be the closed reduced subprescheme of $Y$ that has $\overline\{y\}$ as its underlying space \sref[I]{I.5.2.1}, and let $X_1$ be the closed subprescheme given by the inverse image $p^{-1}(Y_1)$;
since $y'\in\overline{\{y\}}$ by hypothesis, and since $\kres(x)$ is the same in $X$ and in $X_1$, we can assume that $Y$ is \emph{integral}, with generic point $y$;
$\sh{O}_{y'}$ is then an integral local ring that is not a field, and whose field of fractions is $\sh{O}_y=\kres(y)$, and $\kres(x)$ is then an extension of $\kres(y)$.
To satisfy the conditions $f(a)=y'$ and $f(b)=y$ as well as the additional condition (i) (resp. (ii)), we take $Y'=\Spec(A')$, where $A'$ is a valuation ring that dominates $\sh{O}_{y'}$ (resp. a valuation ring that dominates $\sh{O}_{y'}$ and whose field of fractions is $\kres(x)$);
the existence such an of $A'$ is guaranteed by \sref{II.7.1.2}.
\end{proof}

\begin{env}[7.1.5]
\label{II.7.1.5}
Recall that a local ring $A$ is said to be \emph{of dimension 1} if there exists a prime ideal distinct from the maximal ideal $\mathfrak{m}$, and if every prime ideal of $A$ distinct from $\mathfrak{m}$ is a \emph{minimal} prime ideal;
when $A$ is \emph{integral}, it is equivalent to ask that $\mathfrak{m}$ and $(0)$ be the only prime ideals, with $\mathfrak{m}\neq(0)$;
in other words, $Y=\Spec(A)$ consists of two
\oldpage[II]{140}
points $a$ and $b$: $a$ is the unique \emph{closed} point, we have $\mathfrak{j}_a=\mathfrak{m}$, and $\kres(a)=k$ is the \emph{residue field} $k=A/\mathfrak{m}$;
$b$ is the \emph{generic point} of $Y$, $\mathfrak{j}_b=(0)$, with the set $\{b\}$ being the unique open subset of $Y$ distinct from both $\emp$ and $Y$ (an open subset which is thus \emph{everywhere dense}), and $\kres(b)=K$ is the \emph{field of fractions} of $A$.
\end{env}

\begin{env}[7.1.6]
\label{II.7.1.6}
For a local ring $A$, Noetherian and of dimension 1, we know (\cite[pp.~2-08 and 17-01]{I-1}) that the following conditions are equivalent:
\begin{enumerate}
    \item[\rm{(a)}] $A$ is normal;
    \item[\rm{(b)}] $A$ is regular;
    \item[\rm{(c)}] $A$ is a valuation ring;
\end{enumerate}
furthermore, $A$ is then a \emph{discrete valuation ring}.
Propositions~\sref{II.7.1.2} and \sref{II.7.1.3} then have the following analogues for discrete valuation rings:
\end{env}

\begin{proposition}[7.1.7]
\label{II.7.1.7}
Let $A$ be an integral local Noetherian ring that is not a field, $K$ its field of fractions, and $L$ an extension of finite type of $K$;
then there exists a discrete valuation ring that dominates $A$ and has $L$ as its field of fractions.
\end{proposition}

\begin{proof}
Suppose first of all that $L=K$.
Let $\mathfrak{m}$ be the maximal ideal of $A$, $(x_1,\ldots,x_n)$ a system of non-null generators of $\mathfrak{m}$, and $B$ the subring $A[x_2/x_1,\ldots,x_n/x_1]$ of $K$, which is Noetherian.
It is immediate that the ideal $\mathfrak{m}B$ of $B$ is identical to the principal ideal $x_1B$;
if $\mathfrak{p}$ is a minimal prime ideal of $x_1B$, then $\mathfrak{p}$ is of rank 1 (\cite[t.~I, p.~277]{I-13});
in other words, $B_\mathfrak{p}$ is a local Noetherian ring \emph{of dimension 1};
it is clear that $\mathfrak{p}B_\mathfrak{p}\cap A$ is an ideal of $A$ that contains $\mathfrak{m}$ and that does not contain $1$, and is thus equal to $\mathfrak{m}$, and so $B_\mathfrak{p}$ \emph{dominates} $A$ \sref[I]{I.8.1.1}.
It follows from the Krull-Akizuki Theorem (\cite[p.~293]{II-25}) that the integral closure $C$ of $B_\mathfrak{p}$ is a Noetherian ring (even though $C$ is not necessarily a $B_\mathfrak{p}$-module of finite type);
if $\mathfrak{n}$ is a maximal ideal of $C$, then $C_\mathfrak{n}$ is a normal local Noetherian ring of dimension 1 (\cite[p.~295]{II-25}), and thus a discrete valuation ring that dominates $B_\mathfrak{p}$ and \emph{a fortiori} $A$.

Now, if $L$ is an extension of finite type of $K$, we can, by the above, restrict to the case where $A$ is already a discrete valuation ring.
Let $w$ be a valuation of $K$ associated to $A$;
there exists a discrete valuation $w'$ of $L$ that \emph{extends} $w$: we can restrict, by induction on the number of generators of $L$, to the case where $L=K(\alpha)$, and then the proposition is classical (\cite[p.~106]{II-24}).
\end{proof}

\begin{corollary}[7.1.8]
\label{II.7.1.8}
Let $A$ be a Noetherian integral ring, $K$ its field of fractions, and $L$ an extension of finite type of $K$.
Then the integral closure of $A$ in $L$ is the intersection of the discrete valuation rings that have $L$ as their field of fractions and that contain $A$.
\end{corollary}

\begin{proof}
Indeed, such a discrete valuation ring, being normal, contains \emph{a fortiori} every element of $L$ that is integral over $A$.
It thus suffices to prove that, if $x\in L$ is not integral over $A$, then there exists a discrete valuation ring $C$ that has $L$ as its field of fractions, contains $A$, and does not contain $x$.
The hypothesis on $x$ implies that $x\not\in B=A[1/x]$, or, in other words, that $1/x$ is not invertible in the Noetherian ring $B$.
There is thus a prime ideal $\mathfrak{p}$ of $B$ that contains $1/x$.
The integral local ring $B_\mathfrak{p}$ is Noetherian and contained in $L$, which is an extension  of finite type of the field of fractions of $B_\mathfrak{p}$ (with the latter containing $K$).
By \sref{II.7.1.7}, there thus exists a discrete valuation ring $C$ that dominates $B_\mathfrak{p}$ and has $L$ as its field of fractions;
since $1/x\in\mathfrak{p}B_\mathfrak{p}$ belongs to the maximal ideal of $C$, we have that $x\not\in C$, which concludes the proof.
\end{proof}

The geometric form of \sref{II.7.1.7} is the following:

\oldpage[II]{141}
\begin{proposition}[7.1.9]
\label{II.7.1.9}
Let $Y$ be a locally Noetherian prescheme, $p:X\to Y$ a morphism locally of finite type, $x$ a point of $X$, $y=p(x)$, and $y'\neq y$ a specialisation of $y$.
Then there exists a local scheme $Y'$, spectrum of a discrete valuation ring, a separated morphism $f:Y'\to Y$, and a rational $Y$-map $g$ from $Y'$ to $X$, such that, denoting the closed point of $Y'$ by $a$, and the generic point of $Y'$ by $b$, we have $f(a)=y'$, $f(b)=y$, $g(b)=x$, and such that, in the commutative diagram
\[
\label{II.7.1.9.1}
    \xymatrix{
        & \kres(x) \ar[dl]_{\gamma}
    \\  \kres(b)
        & \kres(y) \ar[u]_{\pi} \ar[l]^{\vphi}
    }
\tag{7.1.9.1}
\]
(where $\pi$, $\vphi$, and $\gamma$ are the homomorphisms corresponding to $p$, $f$, and $g$, respectively) the morphism $\gamma$ is a bijection.
\end{proposition}

\begin{proof}
As in \sref{II.7.1.4}, we can restrict to the case where $Y$ is integral with generic point $y$ (taking \sref[I]{I.6.4.3}[iv] into account), and, since the question is local on $X$ and $Y$, we can assume that $p$ is of finite type;
we are then in the situation of \sref{II.7.1.4}, with the additional property that $\kres(x)$ is an extension \emph{of finite type} of $\kres(y)$ \sref[I]{I.6.4.11} and that $\sh{O}_{y'}$ is Noetherian;
this lets us apply \sref{II.7.1.7} and take $Y'=\Spec(A')$, where $A'$ is a discrete valuation ring that dominates $\sh{O}_{y'}$ and whose field of fractions is $\kres(x)$.
We have thus defined a commutative diagram \sref{II.7.1.9.1} where $\gamma$ is a bijection, with $\pi$ and $\vphi$ corresponding to the morphisms $p$ and $f$.
Furthermore, since $X$ and $Y$ are locally Noetherian \sref[I]{I.6.6.2} and since $Y'$ is integral, there exists exactly one rational $Y$-map $g$ from $Y'$ to $X$ to which corresponds the isomorphism $\gamma$ \sref[I]{I.7.1.15}, which finishes the proof.
\end{proof}

\subsection{Valuative criterion for separatedness}
\label{subsection:II.7.2}

\begin{proposition}[7.2.1]
\label{II.7.2.1}
Let $X$ and $Y$ be preschemes, and $f:X\to Y$ a quasi-compact morphism.
The following two conditions are equivalent:
\begin{enumerate}
    \item[\rm{(a)}] The morphism $f$ is closed.
    \item[\rm{(b)}] For all $x\in X$, and every specialisation $y'\neq y$ of $y=f(x)$, there exists a specialisation $x'$ of $x$ such that $f(x')=x$.
\end{enumerate}
\end{proposition}

\begin{proof}
Condition~(b) implies that $f(\overline{\{x\}})=\overline{\{y\}}$, and is thus a consequence of (a).
To show that (b) implies (a), consider a closed subset $X'$ of the underlying space $X$;
let $Y'=\overline{f(X')}$, and show that $Y'=f(X')$ as follows.
Consider the closed reduced subpreschemes of $X$ and $Y$ whose underlying spaces are $X'$ and $Y'$ (respectively) \sref[I]{I.5.2.1};
there then exists a morphism $f':X'\to Y'$ such that the diagram
\[
    \xymatrix{
        X' \ar[r]^{f'} \ar[d]
        & Y' \ar[d]
    \\  X \ar[r]_{f}
        & Y
    }
\]
commutes \sref[I]{I.5.2.2}, and, since $f$ is quasi-compact, so too is $f'$.
We are thus led to proving that, if $f$ is a quasi-compact and \emph{dominant} morphism, then
\oldpage[II]{142}
condition~(b) implies that $f(X)=Y$.
But let $y'$ be a point of $Y$, and let $y$ be the generic point of an irreducible component of $Y$ that contains $y'$;
by (b), it suffices to show that $f^{-1}(y)$ is not empty.
But we know that this property is a consequence of the fact that $f$ is quasi-compact and dominant \sref[I]{I.6.6.5}.
\end{proof}

\begin{corollary}[7.2.2]
\label{II.7.2.2}
Let $f:X\to Y$ be a quasi-compact immersion.
For the underlying space $X$ to be closed in $Y$, it is necessary and sufficient for it to contain every specialisation (in $Y$) of all of its points.
\end{corollary}

\begin{proposition}[7.2.3]
\label{II.7.2.3}
Let $Y$ be a prescheme (resp. a locally Noetherian prescheme), and $f:X\to Y$ a morphism (resp. a morphism locally of finite type).
The following conditions are equivalent:
\begin{enumerate}
    \item[\rm{(a)}] $f$ is separated.
    \item[\rm{(b)}] The diagonal morphism $X\to X\times_Y X$ is quasi-compact, and, for every $Y$-prescheme of the form $Y'=\Spec(A)$, with $A$ some valuation ring (resp. some discrete valuation ring), any two $Y$-morphisms from $Y'$ to $X$ that agree on the generic point of $Y'$ are equal.
    \item[\rm{(c)}] The diagonal morphism $X\to X\times_Y X$ is quasi-compact, and, for every $Y$-prescheme of the form $Y'=\Spec(A)$, with $A$ some valuation ring (resp. some discrete valuation ring), any two $Y'$-sections of $X'=X_{(Y')}$ that agree on the generic point of $Y'$ are equal.
\end{enumerate}
\end{proposition}

\begin{proof}
The equivalence of (b) and (c) follows from the bijective correspondence between $Y$-morphisms from $Y'$ to $X$ and $Y'$-sections of $X'$ \sref[I]{I.3.3.14}.
If $X$ is separated over $Y$, condition~(b) is satisfied, by \sref[I]{I.7.2.2.1}, since $Y'$ is integral.
It remains to show that (b) implies that the diagonal morphism $\Delta:X\to X\times_Y X$ is closed, and it is equivalent to show that it satisfies the criteria of \sref{II.7.2.2}.
But let $z$ be a point of the diagonal $\Delta(X)$, and $z'\neq z$ a specialisation of $z$ in $X\times_Y X$.
There then exists \sref{II.7.1.4} a valuation ring $A$ and a morphism $f$ from $Y'=\Spec(A)$ to $X\times_Y X$ such that $f$ sends the closed point $a$ of $Y'$ to $z'$, and the generic point $b$ of $Y'$ to $z$;
this morphism makes $Y'$ an $(X\times_Y X)$-prescheme, and \emph{a fortiori} a $Y$-prescheme.
If we compose the two projections of $X\times_Y X$ with $f$, then we obtain two $Y$-morphisms, $g_1$ and $g_2$, from $Y'$ to $X$, which, by hypothesis, agree on the point $b$;
they are thus equal to one single morphism $g$, which implies \sref[I]{I.5.3.1} that $f$ factors as $f=\Delta\circ g$, and thus $z'\in\Delta(X)$.
If we suppose that $Y$ is locally Noetherian and $f$ is locally of finite type, then $X\times_Y X$ is locally Noetherian \sref[I]{I.6.6.7};
we can thus follow the same argument as before by supposing that $A$ is a discrete valuation ring, by \sref{II.7.1.9}.
\end{proof}

\begin{remark}[7.2.4]
\label{II.7.2.4}
\begin{enumerate}
    \item[\rm{(i)}] The hypothesis that the morphism $\Delta$ is quasi-compact is always satisfied whenever $Y$ is locally Noetherian and $f$ is locally of finite type, because $X\times_Y X$ is then locally Noetherian \sref[I]{I.6.6.4}[i].
        In the general case, this also implies that, for every cover $(U_\alpha)$ of $X$ by affine opens, the sets $U_\alpha\cap U_\beta$ are \emph{quasi-compact}.
    \item[\rm{(ii)}] For $f$ to be separated, it is \emph{sufficient} for condition~(b) or (c) to be satisfied for some valuation ring $A$ that is \emph{complete} and whose residue field is \emph{algebraically closed};
        this follows from the proofs of \sref{II.7.2.3} and \sref{II.7.2.4}.
\end{enumerate}
\end{remark}


\oldpage[II]{143}

\subsection{Valuative criterion for properness}
\label{subsection:II.7.3}

\begin{proposition}[7.3.1]
\label{II.7.3.1}
Let $A$ be a valuation ring, $Y=\Spec(A)$, $b$ the generic point of $Y$, $X$ an integral \emph{scheme}, and $f:X\to Y$ a \emph{closed} morphism such that $f^{-1}(b)$ consists of a single point $x$ and such that the corresponding homomorphism $\kres(b)\to\kres(x)$ is bijective.
Then $f$ is an isomorphism.
\end{proposition}

\begin{proof}
Since $f$ if closed and dominant, we have that $f(X)=Y$;
it suffices \sref[I]{I.4.2.2} to prove that, for all $y'\neq b$ in $Y$, there exists \emph{exactly one} point $x'$ such that $f(x')=y'$, and that the corresponding homomorphism $\sh{O}_{y'}\to\sh{O}_{x'}$ is bijective, since then $f$ will be a homeomorphism.
But if $f(x')=y'$ then $\sh{O}_{x'}$ is a local ring contained in $K=\kres(x)=\kres(b)$ and dominating $\sh{O}_{y'}$;
the latter is the local ring $A_{y'}$, and is thus a valuation ring \sref{II.7.1.1} that has $K$ as its field of fractions.
Also, $\sh{O}_{x'}\neq K$, since $x'$ is not the generic point of $X$ \sref[0]{0.2.1.3};
we thus conclude that $\sh{O}_{x'}=\sh{O}_{y'}$.
Since $X$ is an integral scheme, the fact that $\sh{O}_{x'}=\sh{O}_{x''}$ implies that $x'=x''$ \sref[I]{I.8.2.2}, which finishes the proof.
\end{proof}

\begin{env}[7.3.2]
\label{II.7.3.2}
Let $A$ be a valuation ring, $K$ its field of fractions, $Y=\Spec(A)$, and $b$ the generic point of $Y$, such that $\sh{O}_b=\kres(b)$ is equal to $K$;
let $f:X\to Y$ be a morphism.
We know \sref[I]{I.7.1.4} that the \emph{rational $Y$-sections} of $X$ are in bijective correspondence with the \emph{germs} of $Y$-sections (defined in a neighbourhood of $b$) at the point $b$, whence we have a canonical map
\[
\label{II.7.3.2.1}
    \Gamma_\mathrm{rat}(X/Y) \to \Gamma(f^{-1}(b)/\Spec(K))
\tag{7.3.2.1}
\]
with the elements of $\Gamma(f^{-1}(b)/\Spec(K))$ being identified, by definition \sref[I]{I.3.4.5}, with the \emph{points of $f^{-1}(b)=X\otimes_A K$ that are rational over $K$}.
When $f$ is \emph{separated}, it follows from \sref[I]{I.5.4.7} that the map \sref{II.7.3.2.1} is \emph{injective}, since $Y$ is an integral scheme.

Composing \sref{II.7.3.2.1} with the canonical map $\Gamma(X/Y)\to\Gamma_\mathrm{rat}(X/Y)$ \sref[I]{I.7.1.2}, we obtain a canonical map
\[
\label{II.7.3.2.2}
    \Gamma(X/Y) \to \Gamma(f^{-1}(b)/\Spec(K)).
\tag{7.3.2.2}
\]
When $f$ is \emph{separated}, this map is again \emph{injective} \sref[I]{I.5.4.7}.
\end{env}

\begin{proposition}[7.3.3]
\label{II.7.3.3}
Let $A$ be a valuation ring with field of fractions $K$, $Y=\Spec(A)$, $b$ the generic point of $Y$, and $f:X\to Y$ a \emph{separated} and \emph{closed} morphism.
Then the canonical map \sref{II.7.3.2.2} is bijective (which is equivalent to saying that it is \emph{surjective}, and implies that the rational $Y$-sections of $X$ are \emph{everywhere defined}).
\end{proposition}

\begin{proof}
So let $x$ be a point of $f^{-1}(b)$ that is \emph{rational} over $K$.
Since $f$ is separated, so too is the morphism $f^{-1}(b)\to\Spec(K)$ corresponding to $f$ \sref[I]{I.5.5.1}[iv], and, since every section of $f^{-1}(b)$ is a closed immersion \sref[I]{I.5.4.6}, $\{x\}$ is closed \emph{in $f^{-1}(b)$}.
Consider the closed reduced subprescheme $X'$ of $X$ that has the closure $\overline{\{x\}}$ of $\{x\}$ \emph{in $X$} as its underlying space.
It is clear that the restriction of $f$ to $X'$ satisfies the hypotheses of \sref{II.7.3.1}, and is thus an \emph{isomorphism} from $X'$ to $Y$, whose inverse isomorphism is the desired $Y$-section of $X$.
\end{proof}

\begin{env}[7.3.4]
\label{II.7.3.4}
To state the two following results, we use a terminology that will be justified and discussed in chapter~IV: if $F$ is a subset
\oldpage[II]{144}
of a prescheme $Y$, we define the \emph{codimension} of $F$ in $Y$, denoted $\codim_Y F$, to be the lower bound of the integers $\dim(\sh{O}_z)$ over all $z$ in $F$.
\end{env}

\begin{corollary}[7.3.5]
\label{II.7.3.5}
Let $Y$ be a locally Noetherian reduced prescheme, and $N$ the set of points $y\in Y$ where $Y$ is not regular \sref[0]{0.4.1.4};
suppose that $\codim_Y N\geq2$.
Let $f:X\to Y$ be a morphism of finite type, both \emph{separated} and \emph{closed}, and let $g$ be a rational $Y$-section of $X$;
if $Y'$ is the set of points of $Y$ where $g$ is not defined (a set which is \emph{closed} \sref[I]{I.7.2.1}), then $\codim_Y Y'\geq2$.
\end{corollary}

\begin{proof}
It suffices to prove that $g$ is defined at every point $z\in Y$ such that $\dim\sh{O}_z\leq1$.
If $\dim\sh{O}_z=0$, then $z$ is the generic point of an irreducible component of $Y$ \sref[I]{I.1.1.14}, and so belongs to every everywhere-dense open subset of $Y$, and, in particular, to the domain of definition of $g$.
So suppose that $\dim\sh{O}_z=1$; by hypothesis, $\sh{O}_z$ is then a regular Noetherian local ring, and thus \sref{I.7.1.6} a \emph{discrete valuation ring}.
Let $Z=\Spec(\sh{O}_z)$;
since $U=Y\setmin Y'$ is everywhere dense, $U\cap Z$ is nonempty \sref[I]{I.2.4.2};
let $g'$ be the rational map from $Z$ to $X$ induced by $g$ \sref[I]{I.7.2.8};
it suffices to show that $g'$ is a \emph{morphism} \sref[I]{I.7.2.9}.
But $g'$ can be considered as a rational $Z$-section of the $Z$-prescheme $f^{-1}(Z)=X\times_Y Z$;
it is clear that the morphism $f^{-1}(Z)\to Z$ corresponding to $f$ is closed, and it follows from \sref[I]{I.5.5.1}[i] that it is separated;
we thus conclude from \sref{II.7.3.3} that $g'$ is everywhere defined;
since $Z$ is reduced, and $X$ is separated over $Y$, $g'$ is a morphism \sref[I]{I.7.2.2}.
\end{proof}

\begin{corollary}[7.3.6]
\label{II.7.3.6}
Let $S$ be a locally Noetherian prescheme, and $X$ and $Y$ both $S$-preschemes;
suppose that $Y$ is reduced, and further that the set $N$ of points $y\in Y$ where $Y$ is not regular is such that $\codim_Y N\geq2$;
suppose finally that the structure morphism $X\to S$ is proper.
Let $f$ be a rational $S$-map from $Y$ to $X$, and let $Y'$ be the points of $Y$ where $f$ is not defined;
then $\codim_Y Y'\geq2$.
\end{corollary}

\begin{proof}
We know \sref[I]{I.7.1.2} that we can identify the rational $S$-maps from $Y$ to $X$ with the rational $Y$-sections of $X\times_S Y$;
since the structure morphism $X\times_S Y\to Y$ is closed \sref{II.5.4.1}, we can apply \sref{II.7.3.5}, whence the corollary.
\end{proof}

\begin{remark}[7.3.7]
\label{II.7.3.7}
The hypotheses on $Y$ in \sref{II.7.3.5} and \sref{II.7.3.6} will be satisfied in particular when $Y$ is \emph{normal} \sref[0]{0.4.1.4}, by \sref{II.7.1.6}.
\end{remark}

We can characterise the universally closed morphisms (resp. proper morphisms) by a converse of \sref{II.7.3.3}:
\begin{theorem}[7.3.8]
\label{II.7.3.8}
Let $Y$ be a prescheme (resp. a locally Noetherian prescheme), and $f:X\to Y$ a quasi-compact separated morphism (resp. a morphism of finite type).
The following conditions are equivalent:
\begin{enumerate}
    \item[\rm{(a)}] $f$ is universally closed (resp. proper).
    \item[\rm{(b)}] For every $Y$-scheme of the form $Y'=\Spec(A)$, where $A$ is a valuation ring (resp. a discrete valuation ring) with field of fractions $K$, the canonical map
        \[
            \Hom_Y(Y',X) \to \Hom_Y(\Spec(K),X)
        \]
        corresponding to the canonical injection $A\to K$ is surjective (resp. bijective).
\oldpage[II]{145}
    \item[\rm{(c)}] For every $Y$-scheme of the form $Y'=\Spec(A)$, where $A$ is a valuation ring (resp. a discrete valuation ring), the canonical map \sref{II.7.3.2.2} with respect to the $Y'$-prescheme $X_{(Y')}$ is surjective (resp. bijective).
\end{enumerate}
\end{theorem}

\begin{proof}
The equivalence of (b) and (c) follows immediately from \sref[I]{I.3.3.14};
(a) implies (b), since (a) implies, in either case, that $f_(Y')$ is separated \sref[I]{I.5.5.1}[iv] and closed, and it suffices to apply \sref{II.7.3.3}.
It remains to prove that (b) implies (a).
We first consider the case where $Y$ is arbitrary, and $f$ is separated and quasi-compact.
If condition (b) is satisfied by $f$, then it is also satisfied by $f_{(Y'')}:X_{(Y'')}\to Y''$, where $Y''$ is an arbitrary $Y$-prescheme, thanks to the equivalence between (b) and (c), and the fact that $X_{(Y'')}\times_{Y''}Y' = X\times_Y Y'$ for every morphism $Y'\to Y''$ \sref[I]{I.3.3.9.1};
since, further, $f_{(Y'')}$ is separated and quasi-compact whenever $f$ is (\sref[I]{I.5.5.1}[iv] and \sref[I]{I.6.6.4}[iii]), we are led to proving that (b) implies that $f$ is \emph{closed}.
For this, it suffices to verify condition (b) of \sref{II.7.2.1}.
So let $x\in X$, and $y'$ be a specialisation of $y=f(x)$, distinct from $y$;
by \sref{II.7.1.4}, there exists a scheme $Y'$, the spectrum of some valuation ring, and a separated morphism $g:Y'\to Y$ such that, letting $a$ denote the closed point and $b$ the generic point of $Y'$, we have that $g(a)=y'$, $g(b)=y$, and that there exists a $\kres(y)$-homomorphism $\kres(x)\to\kres(b)$.
The latter corresponds canonically to a $Y$-morphism $\Spec(\kres(b))\to X$ \sref[I]{I.2.4.6}, and it thus follows from (b) that there exists a $Y$-morphism $h:Y'\to X$ to which the previous morphism corresponds.
We then have that $h(b)=x$;
if we set $h(a)=x'$, then $x'$ is a specialisation of $x$, and we have that $f(x')=f(h(a))=g(a)=y'$.

If now $Y$ is locally Noetherian and $f$ of finite type, then hypothesis (b) implies, first of all, that $f$ is \emph{separated}, by \sref{II.7.2.3}, with the diagonal morphism $X\to X\times_Y X$ being quasi-compact \sref{II.7.2.4}.
Further, to show that $f$ is proper, it suffices to show that $f_{(Y'')}:X_{(Y'')}\to Y''$ is \emph{closed} for every $Y$-prescheme $Y''$ \emph{of finite type}, taking \sref{II.5.6.3} into account.
Since $Y''$ is then locally Noetherian, we can follow the same reasoning as in the first case by taking $Y'$ to be the spectrum of a discrete valuation ring, and applying \sref{II.7.1.9} instead of \sref{II.7.1.4}.
\end{proof}

\begin{remarks}[7.3.9]
\label{II.7.3.9}
\begin{enumerate}
    \item[\rm{(i)}] Whenever $Y$ is an arbitrary prescheme and $f$ a separated morphism, for $f$ to be universally closed, it \emph{suffices} that condition (b) or (c) be satisfied for the \emph{complete} valuation rings $A$ whose residue field is \emph{algebraically closed};
        this follows from the above proof and from \sref{II.7.1.4}.
    \item[\rm{(ii)}] From criterion (c) of \sref{II.7.3.8} we obtain a new proof of the fact that a projective morphism $X\to Y$ is closed \sref{II.5.5.3}, and it is closer to the classical approach.
        We can indeed assume that $Y$ is affine, and thus that $X$ can be identified with a closed subprescheme of a projective bundle $\bb{P}_Y^n$ \sref{II.5.3.3};
        to prove that $X\to Y$ is closed, it suffices to verify that the structure morphism $\bb{P}_Y^n\to Y$ is closed, and criteria (c) of \sref{II.7.3.8}, combined with \sref{II.4.1.3.1}, tells us that we can reduce to proving the following fact:
        \emph{if $Y$ is the spectrum of a valuation ring $A$, with field of fractions $K$, then every point of $\bb{P}_Y^n$ with values in $K$ comes from (by restriction to the generic point of $Y$) a point of $\bb{P}_Y^n$ with values in $A$.}
        But every invertible $\sh{O}_Y$-module is trivial \sref[I]{I.2.4.8};
        so it follows from \sref{II.4.2.6} that a point of $\bb{P}_Y^n$ with values in $K$ can be identified with a class of elements $(\zeta c_0,\zeta c_1,\ldots,\zeta c_n)$ of $K$, where $\zeta\neq0$ and the $c$ are elements of $K$ that are not all zero.
        However, by multiplying the $c_i$ by an element of $A$ of
\oldpage[II]{146}
        suitable valuation, we can suppose that the $c_i$ all belong to $A$, and that at least one of them is invertible.
        But then \sref{II.4.2.6} the system $(c_0,\ldots,c_n)$ also defines a point of $\bb{P}_Y^n$ with values in $A$, which proves our claim.
    \item[\rm{(iii)}] Criteria \sref{II.7.2.3} and \sref{II.7.3.8} are particularly simple when we consider the data of a $Y$-prescheme $X$ as being equivalent to the data of the functor
        \[
            X(Y') = \Hom_Y(Y',X)
        \]
        for $Y$-preschemes $Y'$;
        these criteria allow us, for example, to prove that, under certain conditions, the ``Picard schemes'' are proper.
\end{enumerate}
\end{remarks}

\begin{corollary}[7.3.10]
\label{II.7.3.10}
Let $Y$ be an integral scheme (resp. a locally Noetherian integral scheme), $X$ an integral scheme, and $f:X\to Y$ a dominant morphism.
\begin{enumerate}
    \item[\rm{(i)}] If $f$ is quasi-compact and universally closed, then every valuation ring whose field of fractions is the field $R(X)$ of rational functions on $X$, and which is dominated by a local ring $Y$, also dominates by a local ring of $X$.
    \item[\rm{(ii)}] Conversely, suppose that $f$ is of finite type, and that the property described in (i) is verified by every valuation ring (resp. every discrete valuation ring) that has $R(X)$ as its field of fractions.
        Then $f$ is proper.
\end{enumerate}
\end{corollary}

\begin{proof}
Note first of all that the hypotheses imply, in any case, that $f$ is separated \sref[I]{I.5.5.9}.
\begin{enumerate}
    \item[\rm{(i)}] Let $K=R(Y)$, $L=R(X)$, $y$ a point of $Y$, and $A$ a valuation ring that dominates $\sh{O}_y$ and has $L$ as its field of fractions;
        the injection $\sh{O}_y\to A$ then defines a morphism $h$ from $Y'=\Spec(A)$ to $Y$ \sref[I]{I.2.4.4} such that $h(a)=y$, where we write $a$ to denote the closed point of $Y'$;
        furthermore, if $\eta$ is the generic point of $Y$, which is also the generic point of $\Spec(\sh{O}_y)$, then we have $h(b)=\eta$, writing $b$ to denote the generic point of $Y'$ (since $K\subset L$ by hypothesis).
        If $\xi$ is the generic point of $X$, then $\kres(\xi)=\kres(b)=L$ by hypothesis, whence we have a $Y$-morphism $g:\Spec(L)\to X$ such that $g(b)=\xi$;
        by \sref{II.7.3.8}, $g$ comes from a $Y$-morphism $g':Y'\to X$.
        If $x=g'(a)$, it is clear that $A$ dominates $\sh{O}_x$.
    \item[\rm{(ii)}] Since the questions is local on $Y$, we can always suppose that $Y$ is affine (resp. affine and Noetherian).
        Since $f$ is of finite type, we can apply, in either case, Chow's lemma \sref{II.5.6.1}.
        There is thus a projective morphism $p:P\to Y$, an immersion morphism $j:X'\to P$, and a projective morphism $g:X'\to X$ that is both surjective and birational, with $X$ integral, such that the diagram
        \[
            \xymatrix{
                P \ar[d]_p
                & X' \ar[l]_j \ar[d]^g
            \\  Y
                & X \ar[l]^f
            }
        \]
        commutes.
        It suffices to prove that $j$ is a \emph{closed} immersion, since then $f\circ g=p\circ j$ will be a projective morphism, and thus closed, and, since $g$ is surjective, $f$ will also be proper \sref{II.5.4.3}.
        Let $Z$ be the closed reduced subprescheme of $P$ that has $\overline{j(X')}$ as its underlying space \sref[I]{I.5.2.1};
        since $X'$ is integral, $j$ factors as $i\circ h$, where $i:Z\to P$ is the canonical injection, $h:X'\to Z$ a dominant open immersion \sref[I]{I.5.2.3}, and $Z$ is integral;
\oldpage[II]{147}
        furthermore, $Z$ is projective over $Y$, and we see that we can restrict to the case where $P$ is \emph{integral} and $j$ is \emph{dominant} and \emph{birational}, and everything then reduces to showing that $j$ is \emph{surjective}.
        But let $z\in P$;
        then $\sh{O}_z$ is an integral (resp. integral and Noetherian) local ring whose field of fractions is
        \[
            L = R(P) = R(X') = R(X).
        \]
        We can restrict to the case where $z$ is not the generic point of $P$.
        There then exists (\sref{II.7.1.2} and \sref{II.7.1.7}) a valuation ring (resp. a discrete valuation ring) $A$ which dominates $\sh{O}_z$ and has $L$ as its field of fractions.
        \emph{A fortiori}, $A$ dominates $\sh{O}_y$, where $y=p(z)$, and, by hypothesis, there thus exists some $x\in X$ such that $A$ dominates $\sh{O}_x$.
        Since $g$ is proper, the first part of the proof shows that $A$ also dominates $\sh{O}_{x'}$ for some $x'\in X'$;
        it then follows that $\sh{O}_z$ and $\sh{O}_{j(x')}=\sh{O}_{x'}$ are allied \sref[I]{I.8.1.4}, and, since $P$ is a scheme, this implies that $z=j(x')$ \sref[I]{I.8.2.2} and finishes the proof.
\end{enumerate}
\end{proof}

\begin{corollary}[7.3.11]
\label{II.7.3.11}
Let $X$ and $Y$ be integral schemes, and $f:X\to Y$ a dominant, quasi-compact, and universally closed morphism.
Suppose further that $Y$ is affine of (integral) ring $B$.
Then $\Gamma(X,\sh{O}_X)$ is canonically isomorphic to a subring of the integral closure of $B$ in $R(X)$.
\end{corollary}

\begin{proof}
Indeed \sref[I]{I.8.2.1.1}, $\Gamma(X,\sh{O}_X)$ can be identified with the intersection of the $\sh{O}_x$ over $x\in X$;
by \sref{II.7.3.10}, \sref{II.7.1.2}, and \sref{II.7.1.3}, $\Gamma(X,\sh{O}_X)$ is then contained in the intersection of the valuation rings that contain $B$ and that have $R(X)$ as their field of fractions;
the conclusion then follows from \sref{II.7.1.3}.
\end{proof}

\begin{remarks}[7.3.12]
\label{II.7.3.12}
Under the hypotheses of \sref{II.7.3.11}, and when we suppose that $R(X)$ is an extension of finite type of $R(Y)$, we can, in many cases, conclude that $\Gamma(X,\sh{O}_X)$ is a \emph{module of finite type} over the ring $B=\Gamma(Y,\sh{O}_X)$.
For example, this will be the case whenever $B$ is an \emph{algebra of finite type over a field}, since we then know that the integral closure of $B$ in an extension of finite type of its field of fractions is a $B$-module of finite type (\cite[t.~I, p.~267, th.~9]{I-13});
the conclusion then follows from \sref{II.7.3.11} and the fact that $B$ is Noetherian.

In particular, \emph{a proper affine scheme $X$ over a field $K$ is finite}.
Indeed, by \sref{II.1.6.4}, \sref{II.5.4.6}, and \sref[I]{I.6.4.4}[(c)], we can restrict to the case where $X$ is \emph{reduced}.
Furthermore, it suffices to prove that each of the closed subpreschemes of $X$ that have an irreducible component of $X$ as their underlying space (of which there are finitely many) is finite over $K$, which means (taking \sref{II.5.4.5} into account) that we are finally reduced to the case where $X$ is \emph{integral}.
But then the result follows from the above remarks.

In chapter~III, we will again prove this above proposition by other methods, and as a consequence of more general results, by showing that, if $f:X\to Y$ is proper and $Y$ is locally Noetherian, $f_*(\sh{F})$ is coherent for any coherent $\sh{O}_X$-module $\sh{F}$ \sref[III]{III.4.4.2}.

Finally, note that criterion~\sref{II.7.3.10} is taken as the \emph{definition} of proper morphisms in classical algebraic geometry.
We only mention this here as a remark, since criterion~\sref{II.7.3.8} seems more manageable in all the applications with which we are familiar.
\end{remarks}


\oldpage[II]{148}

\subsection{Algebraic curves and function fields of dimension 1}
\label{subsection:II.7.4}

The aim of this section is to show how to formulate the classical notion of algebraic curves (as introduced, by example, in the book of C.~Chevalley \cite{II-23}) in the language of schemes.
All throughout this section, \emph{we write $k$ to mean a field, all the schemes in question are $k$-schemes of finite type, and all the morphisms are $k$-morphisms}.

\begin{proposition}[7.4.1]
\label{II.7.4.1}
Let $X$ be a prescheme of finite type over $k$ (and thus Noetherian);
let $x_i$ ($1\leq i\leq n$) be the generic points of the irreducible components $X_i$ of $X$, and let $K_i=\kres(x_i)$ ($1\leq i\leq n$).
Then the following conditions are equivalent:
\begin{enumerate}
    \item[\rm{(a)}] Each of the $K_i$ is an extension of $k$ with transcendence degree equal to $1$.
    \item[\rm{(b)}] For every closed point $x$ of $X$, the local ring $\sh{O}_x$ is of dimension $1$ \sref{II.7.1.5}.
    \item[\rm{(c)}] The closed irreducible subsets of $X$ that are distinct from the $X_i$ are exactly the closed points of $X$.
\end{enumerate}
\end{proposition}

\begin{proof}
Since $X$ is quasi-compact, every closed irreducible subset $F$ of $X$ contains a closed point \sref[0]{0.2.1.3}.
By \sref[I]{I.2.4.2}, there is a bijective correspondence between the prime ideals of $\sh{O}_x$ and the closed irreducible subsets of $X$ that contain $x$ \sref[I]{I.1.1.14};
the equivalence between (b) and (c) follows immediately from this.
Now, if $\mathfrak{p}_\alpha$ ($1\leq\alpha\leq r$) are the minimal prime ideals of the local Noetherian ring $\sh{O}_x$, then the local rings $\sh{O}_x/\mathfrak{p}_\alpha$ are integral, and have the $K_i$ such that $x\in X_i$ as their fields of fractions.
Furthermore, we know (\cite[p.~4-06, th.~2]{I-1}) that the dimension of a local integral $k$-algebra of finite type is equal to the transcendence degree over $k$ of its field of fractions.
Finally, the dimension of $\sh{O}_x$ is bounded above by the dimensions of the $\sh{O}_x/\mathfrak{p}_\alpha$;
but condition~(a) implies that these dimensions are equal to $1$, and so (a) implies (b);
conversely, if $\sh{O}_x$ is of dimension $1$, then none of the $\mathfrak{p}_\alpha$ can be equal to the maximal ideal of $\sh{O}_x$, otherwise $\sh{O}_x$ would be of dimension $0$;
thus each of the $\sh{O}_x/\mathfrak{p}_\alpha$ are of dimension $1$, which shows that (b) implies (a).
\end{proof}

We note that, under the conditions of \sref{II.7.4.1}, the set $X$ is either \emph{empty} or \emph{infinite}, as an immediate result of \sref[I]{I.6.4.4}.

\begin{definition}[7.4.2]
\label{II.7.4.2}
We define an \emph{algebraic curve over $k$} to be a non-empty algebraic \emph{scheme} over $k$ that satisfies the conditions of \sref{II.7.4.1}.
\end{definition}

In the language of dimensions, which will be introduced in Chapter~IV, this can be expressed by saying that an algebraic curve over $k$ is a non-empty algebraic $k$-scheme \emph{whose irreducible components are all of dimension $1$}.

We note that, if $X$ is an algebraic curve over $k$, then the closed reduced subpreschemes $X_i$ ($1\leq i\leq n$) of $X$ that have the irreducible components of $X$ as their underlying space are also algebraic curves over $k$.

\begin{corollary}[7.4.3]
\label{II.7.4.3}
Let $X$ be an irreducible algebraic curve.
The only non-closed point of $X$ is its generic point.
The closed subsets of $X$ that are distinct from $X$ are the finite sets of closed points;
these are also the only subsets of $X$ that are not everywhere dense.
\end{corollary}

\begin{proof}
If a point $x\in X$ is not closed, then its closure in $X$ is an irreducible closed subset of $X$, and thus necessarily the whole of $X$, by \sref{II.7.4.1}, and thus $x$ is the generic point of $X$.
A closed subset $F$ of $X$ that is distinct from $X$ cannot contain
\oldpage[II]{149}
the generic point of $X$, and so all its points are closed (in $X$, and \emph{a fortiori} in $F$);
by considering the closed reduced subpreschemes of $X$ that have $F$ as their underlying space \sref[I]{I.5.2.1}, it thus follows from \sref[I]{I.6.2.2} that $F$ is finite and discrete.
The closure in $X$ of any infinite subset of $X$ is thus necessarily equal to $X$ itself.
\end{proof}

If $X$ is an arbitrary algebraic curve, by applying \sref{II.7.4.3} to the irreducible components of $X$, we see that the only non-closed points of $X$ are the generic points of these components.

\begin{corollary}[7.4.4]
\label{II.7.4.4}
Let $X$ and $Y$ be irreducible algebraic curves over $k$, and $f:X\to Y$ a $k$-morphism.
For $f$ to be dominant, it is necessary and sufficient for $f^{-1}(y)$ to be finite for all $y\in Y$.
\end{corollary}

\begin{proof}
Indeed, if $f$ is not dominant, then $f(X)$ is necessarily a \emph{finite} subset of $Y$, by \sref{II.7.4.3}, and so it is not possible for $f^{-1}(y)$ to be finite for every point of $Y$, since otherwise $X$ would be finite, which is a contradiction \sref{II.7.4.1}.
Conversely, if $f$ is dominant, then for any $y\in Y$ distinct from the generic point $\eta$ of $Y$, we have that $f^{-1}(y)$ is closed in $X$, since $\{y\}$ is closed in $Y$ \sref{II.7.4.3};
also, by hypothesis, $f^{-1}(y)$ does not contain the generic point $\xi$ of $X$, and is thus finite, by \sref{II.7.4.3}.
Finally, to see that, when $f$ is dominant, $f^{-1}(\eta)$ is finite, we note that the fibre $f^{-1}(\eta)$ is an irreducible scheme of finite type over $\kres(\eta)$, and with generic point $\xi$ (\sref[I]{I.6.3.9} and \sref[I]{6.4.11}).
Since $\kres(\xi)$ and $\kres(\eta)$ are extensions of finite type of $k$, both of transcendence degree $1$, we have that $\kres(\xi)$ is necessarily an extension of finite degree of $\kres(\eta)$, and so $\xi$ is closed in $f^{-1}(\eta)$ \sref[I]{I.6.4.2}, and $f^{-1}(\eta)$ thus consists of a single point $\xi$.
\end{proof}

We will see, in Chapter~III, that, if $f:X\to Y$ is a \emph{proper} morphism of Noetherian preschemes such that $f^{-1}(y)$ is finite for all $y\in Y$, then $f$ is necessarily \emph{finite};
it will thus follow from \sref{II.7.4.4} that a proper dominant morphism from an irreducible algebraic curve to an algebraic curve is \emph{finite}.

\begin{corollary}[7.4.5]
\label{II.7.4.5}
Let $X$ be an algebraic curve over $k$.
For $X$ to be regular, it is necessary and sufficient for $X$ to be normal, or for the local rings of its closed points to be discrete valuation rings.
\end{corollary}

\begin{proof}
This follows immediately from \sref{II.7.4.1}[(b)] and \sref{II.7.1.6}.
\end{proof}

\begin{corollary}[7.4.6]
\label{II.7.4.6}
Let $X$ be a reduced algebraic curve, and $\sh{A}$ a reduced coherent $\sh{R}(X)$-algebra;
then the integral closure $X'$ of $X$ with respect to $\sh{A}$ \sref{II.6.3.4} is a normal algebraic curve, and the canonical morphism $X'\to X$ is finite.
\end{corollary}

\begin{proof}
The fact that $X'\to X$ is finite follows from \sref{II.6.3.10};
$X'$ is thus an algebraic $k$-scheme;
furthermore, if $x_i$ ($1\leq i\leq n$) are the generic points of the irreducible components of $X$, and $x'_j$ ($1\leq j\leq m$) the generic points of the irreducible components of $X'$, then each of the $\kres(x'_j)$ is a finite algebraic extension of one of the $\kres(x_i)$ \sref{II.6.3.6}, and thus of transcendence degree $1$ over $k$.
So $X'$ is indeed an algebraic curve over $k$, and, furthermore, we know that $X'$ is a finite sum of normal integral schemes (\sref{II.6.3.6} and \sref{II.6.3.7}).
\end{proof}

\begin{env}[7.4.7]
\label{II.7.4.7}
We say that an algebraic curve $X$ over $k$ is \emph{complete} if it is \emph{proper} over $k$.
\end{env}

\begin{corollary}[7.4.8]
\label{II.7.4.8}
For a reduced algebraic curve $X$ over $k$ to be complete, it is necessary and sufficient for its normalisation $X'$ to be complete.
\end{corollary}

\oldpage[II]{150}
\begin{proof}
The canonical morphism $f:X'\to X$ is finite \sref{II.7.4.6}, and thus proper \sref{II.6.1.11} and surjective \sref{II.6.3.8};
if $g:X\to\Spec(k)$ is the structure morphism, then $g$ and $\circ f$ are both proper, by \sref{II.5.4.2}[(ii)] and \sref{II.5.4.3}[(ii)], since $g$ is separated by hypothesis.
\end{proof}

\begin{proposition}[7.4.9]
\label{II.7.4.9}
Let $X$ be a \emph{normal} algebraic curve over $k$, and $Y$ a \emph{proper} algebraic $k$-scheme over $k$.
Then every rational $k$-map from $X$ to $Y$ is everywhere defined, or, in other words, is a morphism.
\end{proposition}

\begin{proof}
It follows from \sref{II.7.3.7} that, at the points $x\in X$ where such a map is not defined, the dimension of $\sh{O}_x$ must be $\geq2$, and so the set of such points is empty;
the final claim follows from \sref[I]{I.7.2.3}.
\end{proof}

\begin{corollary}[7.4.10]
\label{II.7.4.10}
A normal algebraic curve over $k$ is quasi-projective over $k$.
\end{corollary}

\begin{proof}
Since $X$ is a finite sum of normal integral algebraic curves \sref{II.6.3.8}, we can restrict to the case where $X$ is integral \sref{II.5.3.6}.
Since $X$ is quasi-compact, it is covered by a finite number of affine open subsets $U_i$ ($1\leq i\leq n$), and, since each of these $U_i$ is of finite type over $k$, for each $i$ there exists some integer $n_i$ along with a $k$-immersion $f_i:U_i\to\bb{P}_k^{n_i}$ (\sref{II.5.3.3} and \sref{II.5.3.4}[(i)]).
Since $U_i$ is dense in $X$, it follows from \sref{II.7.4.9} that $f_i$ can be extended to a $k$-morphism $g_i:X\to\bb{P}_k^{n_i}$, whence we obtain a $k$-morphism $g=(g_1,\ldots,g_n)_k$ from $X$ to the product $P$ of the $\bb{P}_k^{n_i}$ over $k$.
Furthermore, for each $i$, since the restriction of $g_i$ to $U_i$ is an immersion, so too is the restriction of $g$ to $U_i$ \sref[I]{I.5.3.14}.
Since the $U_i$ cover $X$, and since $g$ is separated \sref[I]{I.5.5.1}[(v)], $g$ is an immersion from $X$ into $P$ \sref[I]{I.8.2.8}.
Since the Segre morphism \sref{II.4.3.3} gives an immersion of $P$ into $\bb{P}_k^N$, this proves that $X$ is quasi-projective.
\end{proof}

\begin{corollary}[7.4.11]
\label{II.7.4.11}
Any normal algebraic curve $X$ is isomorphic to the scheme induced by some complete normal algebraic curve $\widehat{X}$ on some everywhere dense open subset, and this $\widehat{X}$ is unique up to unique isomorphism.
\end{corollary}

\begin{proof}
If $X_1$ and $X_2$ are complete normal curves, then it follows from \sref{II.7.4.9} that every isomorphism from any dense open $U_1$ in $X_1$ to any dense open $U_2$ in $X_2$ can be uniquely extended to an isomorphism from $X_1$ to $X_2$;
whence the uniqueness claim.
To prove the existence of $\widehat{X}$, it suffices to note that we can consider $X$ as a subscheme of a projective bundle $\bb{P}_k^n$ \sref{II.7.4.10}.
Let $\overline{X}$ be the closure of $X$ in $\bb{P}_k^n$ \sref[I]{I.9.5.11};
since $X$ is induced by $\overline{X}$ on a dense open subset of $\overline{X}$ \sref[I]{I.9.5.10}, the generic points $x_i$ of the irreducible components of $X$ are also the generic points of the irreducible components of $\overline{X}$, and the $\kres(x_i)$ are the same for both of these schemes, and so \sref{II.7.4.1} $\overline{X}$ is an algebraic curve over $k$ that is reduced \sref[I]{I.9.5.9} and projective over $k$ \sref{II.5.5.1}, whence complete \sref{II.5.5.3}.
So we take for $\widehat{X}$ the \emph{normalisation} of $\overline{X}$, which is again complete \sref{II.7.4.8};
furthermore, if $h:\widehat{X}\to\overline{X}$ is the canonical morphism, then the restriction of $h$ to $h^{-1}(X)$ is an isomorphism to $X$, since $X$ is normal \sref{II.6.3.4}, and since $h^{-1}(X)$ contains the generic points of the irreducible components of $\widehat{X}$ \sref{II.6.3.8}, it is dense in $\widehat{X}$, which finishes the proof.
\end{proof}

\oldpage[II]{151}

\begin{remark}[7.4.12]
\label{II.7.4.12}
We will show, in Chapter~V, that the conclusion of \sref{II.7.4.10} still holds true without the assumption that the curve is normal (or even reduced);
we will also show that, for an algebraic curve (reduced or not) to be affine, it is necessary and sufficient for its (reduced) irreducible components to \emph{not} be complete.
\end{remark}

\begin{corollary}[7.4.13]
\label{II.7.4.13}
Let $X$ be a \emph{normal} irreducible curve over the field $K=R(X)$, and $Y$ a \emph{complete} integral curve over the field $L=R(Y)$.
Then there is a canonical bijective correspondence between \emph{dominant} $k$-morphisms $X\to Y$ and $k$-monomorphisms $L\to K$.
\end{corollary}

\begin{proof}
By \sref{II.7.4.9}, rational $k$-map from $X$ to $Y$ can be identified with $k$-morphisms $u:X\to Y$.
Since the dominant morphisms $u:X\to Y$ are characterised by being those such that $u(x)=y$ (writing $x$ and $y$ to denote the generic points of $X$ and $Y$, respectively), the corollary follows from these remarks and from \sref[I]{I.7.1.13}.
\end{proof}

\begin{env}[7.4.14]
\label{II.7.4.14}
We can refine the result of \sref{II.7.4.13} in the case where $Y$ is the \emph{projective line} $\bb{P}_k^1=\Proj(k[T_0,T])$, where $T_0$ and $T$ are indeterminates.
Then $Y$ is an integral scheme \sref{II.2.4.4}, and the scheme induced on the open subset $D_+(T_0)$ of $Y$ is isomorphic to $\Spec(k[T])$ \sref{II.2.3.6}, and so the generic point of $Y$ is the ideal $(0)$ of $k[T]$, and the field of rational functions of $Y$ is $k(T)$, which proves that $Y$ is a complete algebraic curve over $k$.
Furthermore, the only graded prime ideal of $S=k[T_0,T]$ that contains $T_0$ and is distinct from $S_+$ is the principal ideal $(T_0)$, and so the complement of $D_+(T_0)$ in $Y=\bb{P}_k^1$ consists of \emph{one closed point}, called the ``point at infinity'', which we denote by $\infty$ (for a general study of the links between vector bundles and projective bundles, see \sref{subsection:II.8.4}).
With these notations:
\end{env}

\begin{corollary}[7.4.15]
\label{II.7.4.15}
Let $X$ be a \emph{normal} irreducible curve over the field $K=R(X)$.
Then there exists a canonical bijective correspondence between the set $K$ and the set of morphisms $u$ from $X$ to $\bb{P}_k^1$ that are distinct from the constant morphism with value $\infty$.
For such a rational map to be dominant, it is necessary and sufficient for the corresponding element of $K$ to be transcendent over $k$.
\end{corollary}

\begin{proof}
This claim follows immediately from \sref{II.7.4.9} and the following:
\begin{lemma}
\label{II.7.4.15.1}
Let $X$ be an integral prescheme over $k$, and let $K=R(X)$ be its field of rational functions.
Then there exists a canonical bijective correspondence between the set $K$ and the set of rational maps $u$ from $X$ to $\bb{P}_k^1$ that are distinct from the constant morphism with value $\infty$.
For such a rational map to be dominant, it is necessary and sufficient for the corresponding element of $K$ to be transcendent over $k$.
\end{lemma}
First of all, rational maps from $X$ to $\bb{P}_k^1$ correspond bijectively to points of $\bb{P}_k^1$ with values in the extension $K$ of $k$ \sref[I]{I.7.1.12}.
If such a point is located \sref[I]{I.3.4.5} at the generic point of $\bb{P}_k^1$, then the corresponding rational map is clearly dominant.
In the converse case, since every point of $\bb{P}_k^1$ that is distinct from the generic point is closed \sref{II.7.4.3}, the image of the domain of definition $U$ of $u$ by the unique morphism $U\to\bb{P}_k^1$ of the class $u$ \sref[I]{I.7.2.2} consists of one closed point $y$ of $\bb{P}_k^1$, and this morphism (which is not necessarily everywhere defined on $X$) is thus not dominant;
as an abuse of language, we thus say that the rational map $u$ is ``constant, of value $y$''.
It remains to place in bijective correspondence the points of $\bb{P}_k^1$ with value in $K$ that are located \sref[I]{I.3.4.5} \emph{not} at $\infty$, and the elements of $K$, and then to verify that the location of such a point is the generic point of $\bb{P}_k^1$ if and only if it corresponds to an
\oldpage[II]{152}
element that is transcendental over $k$.
But this is immediate \sref{II.4.2.6}[example~1].
\end{proof}

\begin{corollary}[7.4.16]
\label{II.7.4.16}
Let $X$ and $Y$ be algebraic curves over $k$ that are \emph{normal}, \emph{complete}, and irreducible;
let $K=R(X)$ and $L=R(Y)$ be their fields.
Then there exists a canonical bijective correspondence between the set of $k$-isomorphisms $X\simto Y$ and the set of $k$-isomorphisms $L\simto K$.
\end{corollary}

\begin{proof}
This is an evident consequence of \sref{II.7.4.13}.
\end{proof}

\begin{env}[7.4.17]
\label{II.7.4.17}
This corollary \sref{II.7.4.16} shows that an algebraic curve over $k$ that is normal, complete, and irreducible, is \emph{determined by its field of rational functions $K$ up to unique isomorphism};
by definition, $K$ is an extension of finite type of $k$, of transcendence degree $1$ (we classically call this a \emph{field of algebraic functions of one variable}).
Furthermore:
\end{env}

\begin{proposition}[7.4.18]
\label{II.7.4.18}
For every extension $K$ of $k$ of finite type and of transcendence degree $1$, there exists an algebraic curve $X$ (determined up to unique isomorphism) that is normal, complete, and irreducible, and such that $R(X)=K$.
The set of local rings of $X$ can be identified \sref[I]{I.8.2.1} with the set consisting of the elements of $K$ and the elements of the valuation rings that contain $k$ and have $K$ as their field of fractions.
\end{proposition}

\begin{proof}
Indeed, $K$ is an extension of finite degree of a pure transcendental extension $k(T)$ of $k$, which can be identified, as we have seen, with the field of rational functions of the projective line $Y=\bb{P}_k^1$.
Let $X$ be the \emph{integral closure} of $Y$ with respect to $K$ \sref{II.6.3.4};
then $X$ is a normal algebraic curve over the field $K$ \sref{II.6.3.7}, and it is complete, since the morphism $X\to Y$ is finite \sref{II.7.4.6}.
The local rings $\sh{O}_x$ of $X$ are either the field $K$, when $x$ is the generic point, or discrete valuation rings that contain $k$ and have $K$ as their field of fractions, when $x$ is distinct from the generic point \sref{II.7.4.5}.
Conversely, let $A$ be such a ring;
since the morphism $X\to\Spec(k)$ is proper, the fact that $A$ dominates $k$ implies that $A$ also dominates a local ring $\sh{O}_x$ of $X$ \sref{II.7.3.10};
since the latter is a valuation ring that has $K$ as a field of fractions, it is necessarily equal to $A$.
\end{proof}

\begin{remarks}[7.4.19]
\label{II.7.4.19}
It follows from \sref{II.7.4.16} and \sref{II.7.4.18} that \emph{the data of an algebraic curve over $k$ that is normal, complete, and irreducible, is essentially equivalent to the data of an extension $K$ of $k$ that is of finite type and of transcendence degree $1$}.
We note that, if $k'$ is an extension of the base field $k$, then $X\otimes_k k'$ will again be a complete algebraic curve over $k'$ \sref{II.5.4.2}[(iii)], but, in general, it will be neither reduced nor irreducible.
It will, however, be both reduced and irreducible if $K$ is a separable extension of $k$, and $k$ is algebraically closed in $K$ (this can be expressed, in classical terminology, which we will not use, by saying that $K$ is a ``regular extension'' of $k$).
But even in this case, it is possible for $X\otimes_k k'$ to not be normal.
The reader will find details on these questions in Chapter~IV.
\end{remarks}

\section{Blowup schemes; based cones; projective closure}
\label{section:II.8}

\subsection{Blowup preschemes}
\label{subsection:II.8.1}

\begin{env}[8.1.1]
\label{II.8.1.1}
Let $Y$ be a prescheme, and, for every integer $n\geq 0$, let $\sh{I}_n$ be a quasi-coherent sheaf of ideals of $\sh{O}_Y$; suppose that the following conditions are satisfied:
\[
\label{II.8.1.1.1}
  \sh{I}_0=\sh{O}_Y,\ \sh{I}_n\subset\sh{I}_m\text{ for }m\leq n,
\tag{8.1.1.1}
\]
\[
\label{II.8.1.1.2}
  \sh{I}_m\sh{I}_n\subset\sh{I}_{m+n}\text{ for any }m,n.
\tag{8.1.1.2}
\]

\oldpage[II]{153}
We note that these hypotheses imply
\[
\label{II.8.1.1.3}
  \sh{I}_1^n\subset\sh{I}_n.
\tag{8.1.1.3}
\]

Set
\[
\label{II.8.1.1.4}
  \sh{S}=\bigoplus_{n\geq 0}\sh{I}_n.
\tag{8.1.1.4}
\]

It follows from \sref{II.8.1.1.1} and \sref{II.8.1.1.2} that $\sh{S}$ is a quasi-coherent graded $\sh{O}_Y$-algebra, and thus defines a $Y$-scheme $X=\Proj(\sh{S})$.
If $\sh{J}$ is an \emph{invertible} sheaf of ideals of $\sh{O}_Y$, then $\sh{I}_n\otimes_{\sh{O}_Y}\sh{J}^{\otimes n}$ is canonically identified with $\sh{I}_n\sh{J}^n$.
If we then replace the $\sh{I}_n$ by the $\sh{I}_n\sh{J}^n$, and, in doing so, replace $\sh{S}$ by a quasi-coherent $\sh{O}_Y$-algebra $\sh{S}_{(\sh{J})}$, then $X_{(\sh{J})}=\Proj(\sh{S}_{(\sh{J})})$ is canonically isomorphic to $X$ \sref{II.3.1.8}.
\end{env}

\begin{env}[8.1.2]
\label{II.8.1.2}
Suppose that $Y$ is \emph{locally integral}, so that the sheaf $\sh{R}(Y)$ of rational functions is a quasi-coherent $\sh{O}_Y$-algebra \sref[1]{I.7.3.7}.
We say that an $\sh{O}_Y$-submodule $\sh{I}$ of $\sh{R}(Y)$ is a \emph{fractional ideal} of $\sh{R}(Y)$ if it is of \emph{finite type} \sref[0]{0.5.2.1}.
Suppose we have, for all $n\geq0$, a quasi-coherent fractional ideal $\sh{I}_n$ of $\sh{R}(Y)$, such that $\sh{I}_0 = \sh{O}_Y$, and such that condition \sref{II.8.1.1.2} (but not necessarily the second condition \sref{II.8.1.1.1}) is satisfied;
we can then again define a quasi-coherent graded $\sh{O}_Y$-algebra by Equation~\sref{II.8.1.1.4}, and the corresponding $Y$-scheme $X = \Proj(\sh{S})$;
we will again have a canonical isomorphism from $X$ to $X_{{\sh{J}}}$ for every \emph{invertible} fractional ideal $\sh{J}$ of $\sh{R}(Y)$.
\end{env}

\begin{definition}[8.1.3]
\label{II.8.1.3}
Let $Y$ be a prescheme (resp. a locally integral prescheme), and $\sh{I}$ a quasi-coherent ideal of $\sh{O}_Y$ (resp. a quasi-coherent fractional ideal of $\sh{R}(Y)$).
We say that the $Y$-scheme $X = \Proj(\bigoplus_{n\geq0}\sh{I}^n)$ is obtained by blowing up the ideal $\sh{I}$, or is the blow-up prescheme of $Y$ relative to $\sh{I}$.
When $\sh{I}$ is a quasi-coherent ideal of $\sh{O}_Y$, and $Y'$ is the closed subprescheme of $Y$ defined by $\sh{I}$, we also say that $X$ is the $Y$-scheme obtained by blowing up $Y'$.
\end{definition}

By definition, $\sh{S} = \bigoplus_{n\geq0}\sh{I}^n$ is then generated by $\sh{S}_1 = \sh{I}$;
if $\sh{I}$ is an $\sh{O}_Y$-module of \emph{finite type}, then $X$ is \emph{projective} over $Y$ \sref{II.5.5.2}.
Without any hypotheses on $\sh{I}$, the $\sh{O}_X$-module $\sh{O}_X(1)$ is \emph{invertible} \sref{II.3.2.5} and \emph{very ample}, by \sref{II.4.4.3} applied to the structure morphism $X\to Y$.

We note that, if $j:X\to Y$ is the structure morphism, then the restriction of $f$ to $f^{-1}(Y\setmin Y')$ is an \emph{isomorphism} to $Y\setmin Y'$ whenever $\sh{I}$ is an \emph{ideal of $\sh{O}_Y$} and $Y'$ is the closed subprescheme that it defines: indeed, since the questions is local on $Y$, it suffices to assume that $\sh{I} = \sh{O}_Y$, and our claim then follows from \sref{II.3.1.7}.

If we replace $\sh{I}$ by $\sh{I}^d$ ($d>0$), then the blow-up $Y$-scheme $X$ is replaced by a canonically isomorphic $Y$-scheme $X'$ \sref{II.8.1.1};
similarly, for every \emph{invertible} ideal (resp. \emph{invertible} fractional ideal) $\sh{J}$, the blow-up prescheme $X_{(\sh{J})}$ relative to the ideal $\sh{I}\sh{J}$ is canonically isomorphic to $X$ \sref{II.8.1.1}.

In particular, whenever $\sh{I}$ is an \emph{invertible} ideal (resp. \emph{invertible} fractional ideal), the $Y$-scheme obtained by blowing up $\sh{I}$ is \emph{isomorphic to $Y$} \sref{II.3.1.7}.

\begin{proposition}[8.1.3]
\label{II.8.1.4}
Let $Y$ be an integral prescheme.
\begin{enumerate}
    \item[\rm{(i)}] For every sequence $(\sh{I}_n)$ of quasi-coherent fractional ideals of $\sh{R}(Y)$ that satisfies \sref{II.8.1.1.2}
\oldpage[II]{154}
        and such that $\sh{I}_0 = \sh{O}_Y$, the $Y$-scheme $X=\Proj(\bigoplus_{n\geq0}\sh{I}^n)$ is integral, and the structure morphism $f:X\to Y$ is dominant.
    \item[\rm{(ii)}] Let $\sh{I}$ be a quasi-coherent fractional ideal of $\sh{R}(Y)$, and let $X$ be the $Y$-scheme given by the blow up of $Y$ relative to $\sh{I}$.
        If $\sh{I} \neq 0$, then the structure morphism $f:X\to Y$ is then birational and surjective.
\end{enumerate}
\end{proposition}

\begin{proof}
\medskip\noindent
\begin{enumerate}
    \item[\rm{(i)}] This follows from the fact that $\sh{S} = \bigoplus_{n\geq0}\sh{I}_n$ is an \emph{integral} $\sh{O}_Y$-algebra (\sref{II.3.1.12} and \sref{II.3.1.14}), since, for all $y\in Y$, $\sh{O}_y$ is an integral ring \sref[I]{I.5.1.4}.
    \item[\rm{(ii)}] By (i), $X$ is integral;
        if, furthermore, $x$ and $y$ are the generic points of $X$ and $Y$ (respectively), then we have $f(x) = y$, and it remains to show that $\kres(x)$ is of rank 1 over $\kres(y)$.
        But $x$ is also the generic point of the fibre $f^{-1}(y)$;
        if $\psi$ is the canonical morphism $Z\to Y$, where $Z=\Spec(\kres(y))$, then the prescheme $f^{-1}(y)$ can be identified with $\Proj(\sh{S}')$, where $\sh{S}' = \psi^*(\sh{S})$ \sref{II.3.5.3}.
        But it is clear that $\sh{S}' = \bigoplus_{n\geq0}(\sh{I}_y)^n$, and, since $\sh{I}$ is a quasi-coherent fractional ideal of $\sh{R}(Y)$ that is not zero, $\sh{I}_y \neq 0$ \sref[I]{I.7.3.6}, whence $\sh{I}_y = \kres(y)$;
        then $\Proj(\sh{S}')$ can be identified with $\Spec(\kres(y))$ \sref{II.3.1.7}, whence the conclusion.
\end{enumerate}
\end{proof}

We show a \emph{converse} of \sref{II.8.1.4} in \sref[III]{III.2.3.8}.

\begin{env}[8.1.5]
\label{II.8.1.5}
We return to the setting and notation of \sref{II.8.1.1}.
By definition, the injection homomorphisms $\sh{I}_{n+1}\to\sh{I}_n$ \sref{II.8.1.1.1} define, for every $k\in\bb{Z}$, an injective homomorphism of degree zero of graded $\sh{S}$-modules
\[
\label{II.8.1.5.1}
  u_k: \sh{S}_+(k+1) \to \sh{S}(k);
\tag{8.1.5.1}
\]
since $\sh{S}_+(k+1)$ and $\sh{S}(k+1)$ are canonically \textbf{(TN)}-isomorphic, they give a canonical correspondence between $u_k$ and an injective homomorphism of $\sh{O}_X$-modules \sref{II.3.4.2}:
\[
\label{II.8.1.5.2}
  \widetilde{u}_k: \sh{O}_X(k+1) \to \sh{O}_X(k).
\tag{8.1.5.2}
\]

Recall as well \sref{II.3.2.6} that we have defined canonical homomorphisms
\[
\label{II.8.1.5.3}
  \lambda: \sh{O}_X(h) \otimes_{\sh{O}_X} \sh{O}_X(k) \to \sh{O}_X(h+k)
\tag{8.1.5.3}
\]
and, since the diagram
\[
  \xymatrix{
    \sh{S}(h) \otimes_{\sh{S}} \sh{S}(k) \otimes_{\sh{S}} \sh{S}(l)
      \ar[r]
      \ar[d]
  & \sh{S}(h+k) \otimes_{\sh{S}} \sh{S}(l)
      \ar[d]
  \\\sh{S}(h) \otimes_{\sh{S}} \sh{S}(k+l)
      \ar[r]
  & \sh{S}(h+k+l)
  }
\]
commutes, it follows from the functoriality of the $\lambda$ \sref{II.3.2.6} that the homomorphisms \sref{II.8.1.5.3} define the structure of a \emph{quasi-coherent graded $\sh{O}_X$-algebra} on
\[
\label{II.8.1.5.4}
  \sh{S}_X = \bigoplus_{n\in\bb{Z}}\sh{O}_X(n).
\tag{8.1.5.4}
\]
Furthermore, the diagram
\[
  \xymatrix{
    \sh{S}(h) \otimes_{\sh{S}} \sh{S}(k+1)
      \ar[r]
      \ar[d]_{1\otimes u_k}
  & \sh{S}(h+k+1)
      \ar[d]^{u_{k+h}}
  \\\sh{S}(h) \otimes_{\sh{S}} \sh{S}(k)
      \ar[r]
  & \sh{S}(h+k)
  }
\]
commutes; the functoriality of the $\lambda$ then implies that we have a commutative diagram
\[
  \xymatrix{
    \sh{O}_X(h) \otimes_{\sh{O}_X} \sh{O}_X(k+1)
      \ar[r]^{\lambda}
      \ar[d]_{1\otimes \widetilde{u}_k}
  & \sh{O}_X(h+k+1)
      \ar[d]^{\widetilde{u}_{k+h}}
  \\\sh{O}_X(h) \otimes_{\sh{O}_X} \sh{O}_X(k)
      \ar[r]^{\lambda}
  & \sh{O}_X(h+k)
  }
\]
where the horizontal arrows are the canonical homomorphisms.
We can thus say that the $\widetilde{u}_k$ define an \emph{injective homomorphism} (of degree zero) \emph{of graded $\sh{S}_X$-modules}
\[
\label{II.8.1.5.6}
  \widetilde{u}: \sh{S}_X(1) \to \sh{S}_X.
\tag{8.1.5.6}
\]
\end{env}

\begin{env}[8.1.6]
\label{II.8.1.6}
Keeping the notation from \sref{II.8.1.5}, we now note that, for $n\geq0$, the composite homomorphism $\widetilde{v}_n = \widetilde{u}_{n-1} \circ \widetilde{u}_{n-2} \circ \ldots \circ \widetilde{u}_0$ is an \emph{injective} homomorphism $\sh{O}_X(n) \to \sh{O}_X$;
we denote by $\sh{I}_{n,X}$ its image, which is thus a quasi-coherent ideal of $\sh{O}_X$, \emph{isomorphic} to $\sh{O}_X(n)$.
Furthermore, the diagram
\[
  \xymatrix{
    \sh{O}_X(m) \otimes_{\sh{O}_X} \sh{O}_X(n)
      \ar[r]^{\lambda}
      \ar[d]_{\widetilde{v}_m \otimes \widetilde{v}_n}
  & \sh{O}_X(m+n)
      \ar[d]^{\widetilde{v}_{m+n}}
  \\\sh{O}_X
      \ar[r]_{\id}
  & \sh{O}_X
  }
\]
commutes for $m\geq0$, $n\geq0$.
We thus deduce the following inclusions:
\[
\label{II.8.1.6.1}
  \sh{I}_{0,X} = \sh{O}_X, \quad \sh{I}_{n,X} \subset \sh{I}_{m,X}
  \qquad\mbox{for $0\leq m\leq n$;}
\tag{8.1.6.1}
\]
\[
\label{II.8.1.6.2}
  \sh{I}_{m,X}\sh{I}_{n,X} \subset \sh{I}_{m+n,X}
  \qquad\qquad\mbox{for $m\geq0$, $n\geq0$.}
\tag{8.1.6.2}
\]
\end{env}

\oldpage[II]{156}

\begin{proposition}[8.1.7]
\label{II.8.1.7}
Let $Y$ be a prescheme, $\sh{I}$ a quasi-coherent ideal of $\sh{O}_Y$, and $X = \Proj(\bigoplus_{n\geq0}\sh{I}^n)$ the $Y$-scheme given by blowing up $\sh{I}$.
We then have, for all $n>0$, a canonical isomorphism
\[
\label{II.8.1.7.1}
  \sh{O}_X(n) \simto \sh{I}^n\sh{O}_X = \sh{I}_{n,X}
\tag{8.1.7.1}
\]
(cf. \sref[0]{0.4.3.5}), and thus that $\sh{I}^n\sh{O}_X$ is a very-ample invertible $\sh{O}_X$-module if $n>0$.
\end{proposition}

\begin{proof}
The last claim is immediate, since $\sh{O}_X(1)$ is invertible \sref{II.3.2.5} and very ample for $Y$ by definition (\sref{II.4.4.3} and \sref{II.4.4.9}).
Also by definition, the image of $v_n$ is exactly $\sh{I}^n\sh{S}$, and \sref{II.8.1.7.1} then follows from the exactness of the functor $\widetilde{\sh{M}}$ \sref{II.3.2.4} and from Equation~\sref{II.3.2.4.1}.
\end{proof}

\begin{corollary}[8.1.8]
\label{II.8.1.8}
Under the hypotheses of \sref{II.8.1.7}, if $f:X\to Y$ is the structure morphism, and $Y'$ the closed subprescheme of $Y$ defined by $\sh{I}$, then the closed subprescheme $X' = f^{-1}(Y')$ of $X$ is defined by $\sh{I}\sh{O}_X$ (which is canonically isomorphic to $\sh{O}_X(1)$), from which we obtain a canonical short exact sequence
\[
\label{II.8.1.8.1}
  0 \to \sh{O}_X(1) \to \sh{O}_X \to \sh{O}_{X'} \to 0.
\tag{8.1.8.1}
\]
\end{corollary}

\begin{proof}
This follows from \sref{II.8.1.7.1} and from \sref[I]{I.4.4.5}.
\end{proof}

\begin{env}[8.1.9]
\label{II.8.1.9}
Under the hypotheses of \sref{II.8.1.7}, we can be more precise about the structure of the $\sh{I}_{n,X}$.
Note that the homomorphism
\[
  \widetilde{u}_{-1}: \sh{O}_X \to \sh{O}_X(-1)
\]
canonically corresponds to a section $s$ of $\sh{O}_X(-1)$ over $X$, which we call the \emph{canonical section} (relative to $\sh{I}$) \sref[0]{0.5.1.1}.
In the diagram in \sref{II.8.1.5.5}, the horizontal arrows are isomorphisms \sref{II.3.2.7}; by replacing $h$ with $k$, and $k$ with $-1$ in this diagram, we obtain that $\widetilde{u}_k = 1_k \otimes \widetilde{u}_{-1}$ (where $1_k$ denotes the identity on $\sh{O}_X(h)$), or, equivalently, that the homomorphism $\widetilde{u}_k$ is given exactly by \emph{tensoring with the canonical section $s$} (for all $k\in\bb{Z}$).
The homomorphism $\widetilde{u}$ \sref{II.8.1.5.6} can then be understood in the same way.

Thus, for all $n\geq0$, the homomorphism $\widetilde{v}_n: \sh{O}_X(n)\to\sh{O}_X$ is given exactly by tensoring with $s^{\otimes n}$;
we thus deduce:
\end{env}

\begin{corollary}[8.1.10]
\label{II.8.1.10}
With the notation of \sref{II.8.1.8}, the underlying space of $X'$ is the set of $x\in X$ such that $s(x)=0$, where $s$ denotes the canonical section of $\sh{O}_X(-1)$.
\end{corollary}

\begin{proof}
Indeed, if $c_x$ is a generator of the fibre $(\sh{O}_X(1))_x$ at a point $x$, then $s_x\otimes c_x$ is canonically identified with a generator of the fibre of $\sh{I}_{1,X}$ at the point $x$, and is thus invertible if and only if $s_x\not\in\mathfrak{m}_x(\sh{O}_x(-1))_x$, or, equivalently, if and only if $s(x)\neq0$.
\end{proof}

\begin{proposition}
\label{II.8.1.11}
Let $Y$ be an integral prescheme, $\sh{I}$ a quasi-coherent fractional ideal of $\sh{R}(Y)$, and $X$ the $Y$-scheme given by blowing up $\sh{I}$.
Then $\sh{I}\sh{O}_X$ is an invertible $\sh{O}_X$-module that is very ample for $Y$.
\end{proposition}

\begin{proof}
Since the questions is local on $Y$ \sref{II.4.4.5}, we can reduce to the case where $Y=\Spec(A)$, with $A$ some integral ring of ring of fractions $K$, and $\sh{I}=\widetilde{\mathfrak{I}}$, with $\mathfrak{I}$ some fractional ideal of $K$;
there then exists an element $a\neq0$ of $A$ such that $a\mathfrak{I}\subset A$.
Let $S = \bigoplus_{n\geq0}\mathfrak{I}^n$;
the map $x\mapsto ax$ is an $A$-isomorphism from $\mathfrak{I}^{n+1} = (S(1))_n$ to $a\mathfrak{I}^{n+1} = a\mathfrak{I}S_n \subset \mathfrak{I}^n = S_n$,
\oldpage[II]{157}
and thus defines a (TN)-isomorphism of degree zero of graded $S$-modules $S_+(1)\to a\mathfrak{I}S$.
On the other hand, $x\mapsto a^{-1}x$ is an isomorphism of degree zero of graded $S$-modules $a\mathfrak{I}S \simto \mathfrak{I}S$.
We thus obtain, by composition \sref{II.3.2.4}, an isomorphism of $\sh{O}_X$-modules $\sh{O}_X(1) \simto \sh{I}\sh{O}_X$, and, since $S$ is generated by $S_1=\mathfrak{I}$, $\sh{O}_X(1)$ is invertible \sref{II.3.2.5} and very ample (\sref{II.4.4.3} and \sref{II.4.4.9}), whence our claim.
\end{proof}


\subsection{Preliminary results on the localisation of graded rings}
\label{subsection:II.8.2}

\begin{env}[8.2.1]
\label{II.8.2.1}
Let $S$ be a graded ring, but not assumed (for the moment) to be only in positive degree.
We define
\[
\label{II.8.2.1.1}
  S^\geq = \bigoplus_{n\geq0} S_n,
  \qquad
  S^\leq = \bigoplus_{n\leq0} S_n
\tag{8.2.1.1}
\]
which are both graded subrings of $S$, in only positive and negative degrees (respectively).
If $f$ is a homogeneous elements of degree $d$ (positive or negative) of $S$, then the ring of fraction $S_f=S'$ is again endowed with the structure of a graded ring, by taking $S'_n$ ($n\in\bb{Z}$) to be the set of the $x/f^k$ for $x\in S_{n+kd}$ ($k\geq0$);
we define $S_{(f)}=S'_0$, and will write $S_f^\geq$ and $S_f^\leq$ for $S^{'\geq}$ and $S^{'\leq}$ (respectively).
If $d>0$, then
\[
\label{II.8.2.1.2}
  (S^\geq)_f = S_f
\tag{8.2.1.2}
\]
since, if $x\in S_{n+kd}$ with $n+kd<0$, then we can write $x/f^k = xf^h/f^{h+k}$, and we also have that $n+(h+k)d>0$ for $h$ sufficiently large and $>0$.
We thus conclude, by definition, that
\[
\label{II.8.2.1.3}
  (S^\geq)_{(f)} = (S_f^\geq)_0 = S_{(f)}.
\tag{8.2.1.3}
\]

If $M$ is a graded $S$-module, then we similarly define
\[
\label{II.8.2.1.4}
  M^\geq = \bigoplus_{n\geq0} M_n,
  \qquad
  M^\leq = \bigoplus_{n\leq0} M_n
\tag{8.2.1.4}
\]
which are (respectively) a graded $S^\geq$-module and a graded $S^\leq$-module, and their intersection is the $S_0$ module $M_0$.
If $f\in S_d$, then we define $M_f$ to be the graded $S_f$-module whose elements of degree $n$ are the $z/f^k$ for $z\in M_{n+kd}$ ($k\geq0$);
we denote by $M_{(f)}$ the set of elements of degree zero of $M_f$, and this is an $S_{(f)}$-module, and we will write $M_f^\geq$ and $M_f^\leq$ to mean $(M_f)^\geq$ and $(M_f)^\leq$ (respectively).
If $d>0$, then we see, as above, that
\[
\label{II.8.2.1.5}
  (M^\geq)_f = M_f
\tag{8.2.1.5}
\]
and
\[
\label{II.8.2.1.6}
  (M^\geq)_{(f)} = (M_f^\geq)_0 = M_{(f)}.
\tag{8.2.1.6}
\]
\end{env}

\begin{env}[8.2.2]
\label{II.8.2.2}
Let $\bb{z}$ be an indeterminate, we we will call the \emph{homogenisation variable}.
If $S$ is a graded ring (in positive or negative degrees), then the polynomial algebra\footnote{This should not be confused with the use of the notation $\widehat{S}$ to denote the completed separation of a ring.}
\[
\label{II.8.2.2.1}
  \widehat{S} = S[\bb{z}]
\tag{8.2.2.1}
\]
\oldpage[II]{158}
is a graded $S$-algebra, where we define the degree of $f\bb{z}^n$ ($n\geq0$), with $f$ homogeneous, as
\[
\label{II.8.2.2.2}
  \deg(f\bb{z}^n) = n+\deg f.
\tag{8.2.2.2}
\]
\end{env}

\begin{lemma}[8.2.3]
\label{II.8.2.3}
\begin{enumerate}
  \item[\rm{(i)}] There are canonical isomorphisms of (non-graded) rings
    \[
    \label{II.8.2.3.1}
      \widehat{S}_{(\bb{z})}
      \simto
      \widehat{S}/(\bb{z}-1)\widehat{S}
      \simto
      S.
    \tag{8.2.3.1}
    \]
  \item[\rm{(ii)}] There is a canonical isomorphism of (non-graded) rings
    \[
    \label{II.8.2.3.2}
      \widehat{S}_{(f)} \simto S_f^\leq
    \tag{8.2.3.2}
    \]
    for all $f\in S_d$ with $d>0$.
\end{enumerate}
\end{lemma}

\begin{proof}
The first of the isomorphisms in \sref{II.8.2.3.1} was defined in \sref{II.2.2.5}, and the second is trivial;
the isomorphism $\widehat{S}_{(\bb{z})} \simto S$ thus defined thus gives a correspondence between $x\bb{z}^n/\bb{z}^{n+k}$ (where $\deg(x) = k$ for $k\geq -n$) and the element $x$.
The homomorphism \sref{II.8.2.3.2} gives a correspondence between $x\bb{z}^n/f^k$ (where $\deg(x) = kd-n$) and the element $x/f^k$ of degree $-n$ in $S_f^\leq$, and it is again clear that this does indeed give an isomorphism.
\end{proof}

\begin{env}[8.2.4]
\label{II.8.2.4}
Let $M$ be a graded $S$-module.
It is clear that the $S$-module
\[
\label{II.8.2.4.1}
  \widehat{M} = M \otimes_S \widehat{S} = M \otimes_S S[\bb{z}]
\tag{8.2.4.1}
\]
is the direct sum of the $S$-modules $M\otimes S\bb{z}^n$, and thus of the abelian groups $M_k\otimes S\bb{z}^n$ ($k\in\bb{Z}$, $n\geq0$);
we define on $\widehat{M}$ the structure of a graded $\widehat{S}$-module by setting
\[
\label{II.8.2.4.2}
  \deg(x\otimes\bb{z}^n) = n+\deg x
\tag{8.2.4.2}
\]
for all homogeneous $x$ in $M$.
We leave it to the reader to prove the analogue of \sref{II.8.2.3}:
\end{env}

\begin{lemma}[8.2.5]
\label{II.8.2.5}
\begin{enumerate}
  \item[\rm{(i)}] There is a canonical di-isomorphism of (non-graded) modules
    \[
    \label{II.8.2.5.1}
      \widehat{M}_{(\bb{z})} \simto M.
    \tag{8.2.5.1}
    \]
  \item[\rm{(ii)}] For all $f\in S_d$ ($d>0$), there is a di-isomorphism of (non-graded) modules
    \[
    \label{II.8.2.5.2}
      \widehat{M}_{(f)} \simto M_f^\leq.
    \tag{8.2.5.2}
    \]
\end{enumerate}
\end{lemma}

\begin{env}[8.2.6]
\label{II.8.2.6}
Let $S$ be a \emph{positively}-graded ring, and consider the decreasing sequence of graded ideals of $S$
\[
\label{II.8.2.6.1}
  S_{[n]} = \bigoplus_{m\geq n} S_m
  \qquad\mbox{($n\geq0$)}
\tag{8.2.6.1}
\]
(so, in particular, we have $S_{[0]}=S$ and $S_{[1]}=S_+$).
Since it is evident that $S_{[m]}S_{[n]} \subset S_{[m+n]}$, we can define a \emph{graded ring} $S^\natural$ by setting
\[
\label{II.8.2.6.2}
  S^\natural = \bigoplus_{n\geq0} S_n^\natural
  \quad
  \text{with}
  \quad
  S_n^\natural = S_{[n]}.
\tag{8.2.6.2}
\]
$S_0^\natural$ is then the ring $S$ considered as a \emph{non-graded} ring, and $S^\natural$ is thus an $S_0^\natural$-algebra.
For every homogeneous element $f\in S_d$ ($d>0$), we denote by $f^\natural$ the element $f$ considered as belonging to $S_{[d]} = S_d^\natural$.
With this notation:
\end{env}

\oldpage[II]{159}
\begin{lemma}[8.2.7]
\label{II.8.2.7}
Let $S$ be a positively-graded ring, and $f$ a homogeneous element of $S_d$ ($d>0$).
There are canonical ring isomorphisms
\[
\label{II.8.2.7.1}
  S_f \simto \bigoplus_{n\in\bb{Z}} S(n)_{(f)}
\tag{8.2.7.1}
\]
\[
\label{II.8.2.7.2}
  (S_f^\geq)_{f/1} \simto S_f
\tag{8.2.7.2}
\]
\[
\label{II.8.2.7.3}
  S_{(f^\natural)}^\natural \simto S_f^\geq
\tag{8.2.7.3}
\]
where the first two are isomorphisms of graded rings.
\end{lemma}

\begin{proof}
It is immediate, by definition, that we have $(S_f)_n = (S(n)_f)_0$, whence the isomorphism in \sref{II.8.2.7.1}, which is exactly the identity.
Next, since $f/1$ is invertible in $S_f$, there is a canonical isomorphism $S_f \simto (S_f^\geq)_{f/1} = (S_f)_{f/1}$, by \sref{II.8.2.1.2} applied to $S_f$;
the inverse isomorphism is, by definition, the isomorphism in \sref{II.8.2.7.2}.
Finally, if $x = \sum_{m\geq n}y_m$ is an element of $S_{[n]}$ with $n=kd$, then the element $x/(f^\natural)^k$ corresponds to the element $\sum_m y_m/f^k$ of $S_f^\geq$, and we can quickly verify that this defines an isomorphism \sref{II.8.2.7.3}.
\end{proof}

\begin{env}[8.2.8]
\label{II.8.2.8}
If $M$ is a graded $S$-module, then we similarly define, for all $n\in\bb{Z}$,
\[
\label{II.8.2.8.1}
  M_{[n]} = \bigoplus_{m\geq n} M_m
\tag{8.2.8.1}
\]
and, since $S_{[m]}M_{[n]} \subset M_{[m+n]}$ ($m\geq0$), we can define a graded $S^\natural$-module $M^\natural$ by setting
\[
\label{II.8.2.8.2}
  M^\natural = \bigoplus_{n\in\bb{Z}}
  \quad
  \text{with}
  \quad
  M_n^\natural = M_{[n]}.
\tag{8.2.8.2}
\]
\end{env}

We leave to the reader the proof of:
\begin{lemma}[8.2.9]
\label{II.8.2.9}
With the notation of \sref{II.8.2.7} and \sref{II.8.2.8}, there are canonical di-isomorphisms of modules
\[
\label{II.8.2.9.1}
  M_f \simto \bigoplus_{n\in\bb{Z}} M(n)_{(f)}
\tag{8.2.9.1}
\]
\[
\label{II.8.2.9.2}
  (M_f^\geq)_{f/1} \simto M_f
\tag{8.2.9.2}
\]
\[
\label{II.8.2.9.3}
  M_{(f^\natural)}^\natural \simto M_f^\geq
\tag{8.2.9.3}
\]
where the first two are di-isomorphisms of graded modules.
\end{lemma}

\begin{lemma}[8.2.10]
\label{II.8.2.10}
Let $S$ be a positively-graded ring.
\begin{enumerate}
  \item[\rm{(i)}] For $S^\natural$ to be an $S_0^\natural$-algebra of finite type (resp. a Noetherian $S_0^\natural$-algebra), it is necessary and sufficient for $S$ to be an $S_0^\natural$-algebra of finite type (resp. a Noetherian $S_0^\natural$-algebra).
  \item[\rm{(ii)}] For $S_{n+1}^\natural = S_1^\natural S_n^\natural$ ($n\geq n_0$), it is necessary and sufficient for $S_{n+1}=S_1S_n$ ($n\geq n_0$).
  \item[\rm{(iii)}] For $S_n^\natural = S_1^\natural$ ($n\geq n_0$), it is necessary and sufficient for $S_n=S_1^n$ ($n\geq n_0$).
  \item[\rm{(iv)}] If $(f_\alpha)$ is a set of homogeneous elements of $S_+$ such that $S_+$ is the radical in $S_+$ of the ideal of $S_+$ generated by the $f_\alpha$, then $S_+^\natural$ is the radical in $S_+^\natural$ of the ideal of $S_+^\natural$ generated by the $f_\alpha^\natural$.
\end{enumerate}
\end{lemma}

\begin{proof}
\medskip\noindent
\begin{enumerate}
  \item[\rm{(i)}] If $S^\natural$ is an $S_0^\natural$-algebra of finite type, then $S_+=S_1^\natural$ is a module of finite type over $S=S_0^\natural$, by \sref{II.2.1.6}[i], and so $S$ is an $S_0$-algebra of finite type \sref{II.2.1.4};
    if $S^\natural$ is a Noetherian ring, then so too is $S_0^\natural=S$ \sref{II.2.1.5}.
    Conversely, if $S$ is an $S_0$-algebra
\oldpage[II]{160}
    of finite type, then we know \sref{II.2.1.6}[ii] that there exist $h>0$ and $m_0>0$ such that $S_{n+h}=S_hS_n$ for $n\geq m_0$;
    we can clearly assume that $m_0\geq h$.
    Furthermore, the $S_m$ are $S_0$-modules of finite type \sref{II.2.1.6}[i].
    So, if $n\geq m_0+h$, then $S_n^\natural = S_hS_{n-h}^\natural = S_h^\natural S_{n-h}^\natural$;
    and if $m<m_0+h$ then, letting $E = S_{m_0}+\ldots+S_{m_0+h-1}$, we have that
    \[
      S_m^\natural = S_m + \ldots + S_{m_0+h-1} + S_hE + S_h^2E + \ldots.
    \]
    For $1\leq m\leq m_0$, let $G_m$ be the union of the finite systems of generators of the $S_0$-modules $S_i$ for $m\leq i\leq m_0+h-1$, considered as a subset of $S_{[m]}$.
    For $m_0+1\leq m\leq m_0+h-1$, let $G_m$ be the union of the finite system of generators of the $S_0$-modules $S_i$ for $m\leq i\leq m_0+h-1$ and of $S_hE$, considered as a subset of $S_{[m]}$.
    It is clear that $S_m^\natural=S_0^\natural G_m$ for $1\leq m\leq m_0+h-1$, and thus the union $G$ of the $G_m$ for $1\leq m\leq m_0+h-1$ is a system of generators of the $S_0^\natural$-algebra $S^\natural$.
    We thus conclude that, if $S=S_0^\natural$ is a Noetherian ring, then so too is $S^\natural$.
  \item[\rm{(ii)}] It is clear that, if $S_{n+1}=S_1S_n$ for $n\geq n_0$, then $S_{n+1}^\natural=S_1S_n^\natural$, and \emph{a fortiori} $S_{n+1}^\natural=S_1^\natural S_n^\natural$ for $n\geq n_0$.
    Conversely, this last equality can be written as
    \[
      S_{n+1} + S_{n+2} + \ldots
      =
      (S_1 + S_2 + \ldots)(S_n + S_{n+1} + \ldots)
    \]
    and comparing terms of degree $n+1$ (in $S$) on both sides gives that $S_{n+1}=S_1S_n$.
  \item[\rm{(iii)}] If $S_n=S_1^n$ for $n\geq n_0$, then $S_n^\natural=S_1^n+S_1^{n+1}+\ldots$;
    since $S_1^\natural$ contains $S_1+S_1^2+\ldots$, we have that $S_n^\natural\subset S_1^{\natural n}$, and thus $S_n^\natural=S_1^{\natural n}$ for $n\geq n_0$.
    Conversely, the only terms of $S_1^{\natural n}=(S_1+S_2+\ldots)^n$ that are of degree $n$ in $S$ are those of $S_1^n$;
    the equality $S_n^\natural=S_1^{\natural n}$ thus implies that $S_n=S_1^n$.
  \item[\rm{(iv)}] It suffices to show that, if an element $g\in S_{k+h}$ is considered as an element of $S_k^\natural$ ($k>0$, $h\geq0$), then there exists an integer $n>0$ such that $g^n$ is a linear combination (in $S_{kn}^\natural$) of the $f_\alpha^\natural$ with coefficients in $S^\natural$.
    By hypothesis, there exists an integer $m_0$ such that, for $m\geq m_0$, we have, \emph{in $S$}, that $g^m = \sum_\alpha c_{\alpha m}f_\alpha$, where the indices $\alpha$ here are \emph{independent of $m$};
    furthermore, we can clearly assume that the $c_{\alpha m}$ are homogeneous, with
    \[
      \deg(c_{\alpha m}) = m(k+h)-\deg f_\alpha
    \]
    in $S$.
    So take $m_0$ sufficiently large enough to ensure that $km_0>\deg f_\alpha$ for all the $f_\alpha$ that appear in $g^{m_0}$;
    for all $\alpha$, let $c'_{\alpha m}$ be the element $c_{\alpha m}$ considered as having degree $km-\deg f_\alpha$ \emph{in $S^\natural$};
    we then have, in $S^\natural$, that $g^m = \sum_\alpha c'_{\alpha m}f_\alpha^\natural$, which finishes the proof.
\end{enumerate}
\end{proof}

\begin{env}[8.2.11]
\label{II.8.2.11}
Consider the graded $S_0$-algebra
\[
\label{II.8.2.11.1}
  S^\natural \otimes_S S_0
  =
  S^\natural/S_+S^\natural
  =
  \bigoplus_{n\geq0} S_{[n]}/S_+S_{[n]}.
\tag{8.2.11.1}
\]

Since $S_n$ is a quotient $S_0$-module of $S_{[n]}/S_+S_{[n]}$, there is a canonical homomorphism of graded $S_0$-algebras
\[
\label{II.8.2.11.2}
  S^\natural \otimes_S S_0 \to S
\tag{8.2.11.2}
\]
which is clearly \emph{surjective}, and thus corresponds \sref{II.2.9.2} to a canonical \emph{closed immersion}
\[
\label{II.8.2.11.3}
  \Proj(S) \to \Proj(S^\natural \otimes_S S_0).
\tag{8.2.11.3}
\]
\end{env}

\oldpage[II]{161}
\begin{proposition}[8.2.12]
\label{II.8.2.12}
The canonical morphism \sref{II.8.2.11.3} is bijective.
For the homomorphism \sref{II.8.2.11.2} to be (TN)-bijective, it is necessary and sufficient for there to exist some $n_0$ such that $S_{n+1}=S_1S_n$ for $n\geq n_0$.
If this latter condition is satisfied, then \sref{II.8.2.11.3} is an isomorphism;
the converse is true whenever $S$ is Noetherian.
\end{proposition}

\begin{proof}
To prove the first claim, it suffices \sref{II.2.8.3} to show that the kernel $\mathfrak{I}$ of the homomorphism \sref{II.8.2.11.2} consists of \emph{nilpotent} elements.
But if $f\in S_{[n]}$ is an element whose class modulo $S_+S_{[n]}$ belongs to this kernel, then this implies that $f\in S_{[n+1]}$;
then $f^{n+1}$, considered as an element of $S_{[n(n+1)]}$, is also an element of $S_+S_{[n(n+1)]}$, since it can be written as $f\cdot f^n$;
so the class of $f^{n+1}$ modulo $S_+S_{[n(n+1)]}$ is zero, which proves our claim.
Since the hypothesis that $S_{n+1}=S_1S_n$ for $n\geq n_0$ is equivalent to $S_{n+1}^\natural=S_1^\natural S_n^\natural$ for $n\geq n_0$ \sref{II.8.2.10}[ii], this hypothesis is equivalent, by definition, to the fact that \sref{II.8.2.11.2} is (TN)-injective, and thus (TN)-bijective, and so \sref{II.8.2.11.3} is an isomorphism, by \sref{II.2.9.1}.
Conversely, if \sref{II.8.2.11.3} is an isomorphism, then the sheaf $\widetilde{\mathfrak{I}}$ on $\Proj(S^\natural\otimes_S S_0)$ is zero \sref{II.2.9.2}[i];
since $S^\natural\otimes_S S_0$ is Noetherian, as a quotient of $S^\natural$ \sref{II.8.2.10}[i], we conclude from \sref{II.2.7.3} that $\mathfrak{I}$ satisfies condition (TN), and so $S_{n+1}^\natural=S_1^\natural S_n^\natural$ for $n\geq n_0$, and this finishes the proof, by \sref{II.8.2.10}[ii].
\end{proof}

\begin{env}[8.2.13]
\label{II.8.2.13}
Consider now the canonical injections $(S_+)^n\to S_{[n]}$, which define an injective homomorphism of degree zero of graded rings
\[
\label{II.8.2.13.1}
  \bigoplus_{n\geq0} (S_+)^n \to S^\natural.
\tag{8.2.13.1}
\]
\end{env}

\begin{proposition}[8.2.14]
\label{II.8.2.14}
For the homomorphism \sref{II.8.2.13.1} to be a (TN)-isomorphism, it is necessary and sufficient for there to exist some $n_0$ such that $S_n=S_1^n$ for all $n\geq n_0$.
Whenever this is the case, the morphisms corresponding to \sref{II.8.2.13.1} is everywhere defined and also an isomorphism
\[
  \Proj(S^\natural) \simto \Proj(\bigoplus_{n\geq0}(S_+)^n);
\]
the converse is true whenever $S$ is Noetherian.
\end{proposition}

\begin{proof}
The first two claims are evident, given \sref{II.8.2.10}[iii] and \sref{II.2.9.1}.
The third will follow from \sref{II.8.2.10}[i and iii] and the following lemma:
\begin{lemma}[8.2.14.1]
\label{II.8.2.14.1}
Let $T$ be a positively-graded ring that is also a $T_0$-algebra of finite type.
If the morphism corresponding to the injective homomorphism $\bigoplus_{n\geq0}T_1^n\to T$ is everywhere defined and also an isomorphism $\Proj(T)\to\Proj(\bigoplus_{n\geq0}T_1^n)$, then there exists some $n_0$ such that $T_n=T_1^n$ for $n\geq n_0$.
\end{lemma}

Let $g_i$ ($1\leq i\leq r$) be generators of the $T_0$-module $T_1$.
The hypothesis implies first of all that the $D_+(g_i)$ cover $\Proj(T)$ \sref{II.2.8.1}.
Let $(h_j)_{1\leq j\leq s}$ be a system of homogeneous elements of $T_+$, with $\deg(h_j)=n_j$, that form, with the $g_i$, a system of generators of the ideal $T_+$, or, equivalently \sref{II.2.1.3}, a system of generators of $T$ as a $T_0$-algebra;
if we set $T'=\bigoplus_{n\geq0}T_1^n$, then the element $h_j/g_i^{n_j}$ of the ring $T_{(g_i)}$ must, by hypothesis, belong to the subring $T'_{(g_i)}$, and so there exists some integer $k$ such that $T_1^k h_j\subset T_1^{k+n_j}$ for all $j$.
We thus conclude, by induction on $r$, that $T_1^k h_j^r \subset T'$ for all $r\geq1$, and, by definition of the $h_j$, we thus have that $T_1^k T\subset T'$.
Also, there exists, for all $j$, an integer $m_j$ such that $h_j^{m_j}$ belongs to the ideal of $T$ generated by the $g_i$ \sref{II.2.3.14}, so $h_j^{m_j}\in T_1 T$, and
\oldpage[II]{162}
$h_j^{m_j k}\in T_1^k T\subset T'$.
There is thus an integer $m_0\geq k$ such that $h_j^m\in T_1^{mn}$ for $m\geq m_0$.
So, if $q$ is the largest of the integers $n_j$, then $n_0=qsm_0+k$ is the required number.
Indeed, an element of $S_n$, for $n\geq n_0$, is the sum of monomials belonging to $T_1^\alpha u$, where $u$ is a product of powers of the $h_j$;
if $\alpha\geq k$, then it follows from the above that $T_1^\alpha u\subset T_1^n$;
in the other case, one of the exponents of the $h_j$ is $\geq m_0$, so $u\in T_1^\beta v$, where $\beta\geq k$ and $v$ is again a product of powers of the $h_j$;
we can then reduce to the previous case, and so we conclude that $T_1^\alpha u\subset T_1^n$ in all cases.
\end{proof}

\begin{remark}[8.2.15]
\label{II.8.2.15}
The condition $S_n=S_1^n$ for $n\geq n_0$ clearly implies that $S_{n+1}=S_1S_n$ for $n\geq n_0$, but the converse is not necessarily true, even if we assume that $S$ is Noetherian.
For example, let $K$ be a field, $A=K[\bb{x}]$, and $B=K[\bb{y}]/\bb{y}^2K[\bb{y}]$, where $\bb{x}$ and $\bb{y}$ are indeterminates, with $\bb{x}$ taken to have degree 1 and $\bb{y}$ to have degree 2, and let $S=A\otimes_K B$, so that $S$ is a graded algebra over $K$ that has a basis given by the elements $1$, $\bb{x}^n$ ($n\geq1$), and $\bb{x}^n\bb{y}$ ($n\geq0$).
It is immediate that $S_{n+1}=S_1S_n$ for $n\geq2$, but $S_1^n=K\bb{x}^n$ while $S_n=K\bb{x}^n+K\bb{x}^n\bb{y}$ for $n\geq2$.
\end{remark}


\subsection{Based cones}
\label{subsection:II.8.3}

\begin{env}[8.3.1]
\label{II.8.3.1}
Let $Y$ be a prescheme;
in all of this section, we will consider only \emph{$Y$-preschemes} and \emph{$Y$-morphisms}.
Let $\sh{S}$ be a quasi-coherent \emph{positively}-graded $\sh{O}_Y$-algebra;
\emph{we further assume that $\sh{S}_0=\sh{O}_Y$}.
Following the notation introduced in \sref{II.8.2.2}, we let
\[
\label{II.8.3.1.1}
  \widehat{\sh{S}} = \sh{S}[\bb{z}] = \sh{S} \otimes_{\sh{O}_Y} \sh{O}_Y[\bb{z}]
\tag{8.3.1.1}
\]
which we consider as a positively-graded $\sh{O}_Y$-algebra by defining the degrees as in \sref{II.8.2.2.2}, so that, for every affine open subset $U$ of $Y$, we have
\[
  \Gamma(U,\widehat{\sh{S}}) = (\Gamma(U,\sh{S}))[\bb{z}].
\]
In what follows, we write
\[
\label{II.8.3.1.2}
  X = \Proj(\sh{S}),
  \quad
  C = \Spec(\sh{S}),
  \quad
  \widehat{C} = \Proj(\widehat{\sh{S}})
\tag{8.3.1.2}
\]
(where, in the definition of $C$, we consider $\sh{S}$ as a non-graded $\sh{O}_Y$-algebra), and we say that $C$ (resp. $\widehat{C}$) is the \emph{affine cone} (resp. \emph{projective cone}) defined by $\sh{S}$;
we will sometimes say ``cone'' instead of ``affine cone''.
By an abuse of language, we also say that $C$ (resp. $\widehat{C}$) is the \emph{affine cone \unsure{based at $X$}} (resp. the \emph{projective cone \unsure{based at $X$}})\footnote{\emph{[Trans.] A more literal translation of the French (\emph{c\^one projetant (affine/projectif)}) would be the \emph{projecting (affine/projective) cone}, but it seems that this terminology already exists to mean something else.}}, with the implicit understanding that the prescheme $X$ is given in the form $\Proj(\sh{S})$;
finally, we say that $\widehat{C}$ is the \emph{projective closure} of $C$ (with the data of $\sh{S}$ being implicit in the structure of $C$).
\end{env}

\begin{proposition}[8.3.2]
\label{II.8.3.2}
There exist canonical $Y$-morphisms
\[
\label{II.8.3.2.1}
  Y \xrightarrow{\varepsilon} C \xrightarrow{i} \widehat{C}
\tag{8.3.2.1}
\]
\[
\label{II.8.3.2.2}
  X \xrightarrow{j} \widehat{C}
\tag{8.3.2.2}
\]
such that $\varepsilon$ and $j$ are closed immersions, and $i$ is an affine morphism, which is a dominant open immersion, for which
\[
\label{II.8.3.2.3}
  i(C) = \widehat{C}\setmin j(X);
\tag{8.3.2.3}
\]
furthermore, $\widehat{C}$ is the smallest closed subprescheme of $\widehat{C}$ containing $i(C)$.
\end{proposition}

\begin{proof}
To define $i$, consider the open subset of $\widehat{C}$ given by
\[
\label{II.8.3.2.4}
  \widehat{C}_{\bb{z}} = \Spec(\widehat{\sh{S}}/(\bb{z}-1)\widehat{\sh{S}})
\tag{8.3.2.4}
\]
\sref{II.3.1.4}, where $\bb{z}$ is canonically identified with a section of $\sh{S}$ over $Y$.
The isomorphism $i: C\simto\widehat{C}_{\bb{z}}$ then corresponds to the canonical isomorphism \sref{II.8.2.3.1}
\[
  \widehat{\sh{S}}/(\bb{z}-1)\widehat{\sh{S}} \simto \sh{S}.
\]

The morphism $\varepsilon$ corresponds to the augmentation homomorphism $\sh{S}\to\sh{S}_0=\sh{O}_Y$, which has kernel $\sh{S}_+$ \sref{II.1.2.7}, and, since the latter is surjective, $\varepsilon$ is a closed immersion \sref{II.1.4.10}.
Finally, $j$ corresponds \sref{II.3.5.1} to the surjective homomorphism of degree zero $\widehat{\sh{S}}\to\sh{S}$, which restricts to the identity on $\sh{S}$ and is zero on $\bb{z}\widehat{\sh{S}}$, which is its kernel;
$j$ is everywhere defined, and is a closed immersion, by \sref{II.3.6.2}.

To prove the other claims of \sref{II.8.3.2}, we can clearly restrict to the case where $Y=\Spec(A)$ is affine, and $\sh{S}=\widetilde{S}$, with $S$ a graded $A$-algebra, whence $\widehat{\sh{S}}=(\widehat{S})\supertilde$;
the homogeneous elements $f$ of $S_+$ can then be identified with sections of $\widehat{\sh{S}}$ over $Y$, and the open subset of $\widehat{C}$, denoted $D_+(f)$ in \sref{II.2.3.3}, can then be written as $\widehat{C}_f$ \sref{II.3.1.4};
similarly, the open subset of $C$ denoted $D(f)$ in \sref[I]{I.1.1.1} can be written as $C_f$ \sref[0]{0.5.5.2}.
With this in mind, it follows from \sref{II.2.3.14} and from the definition of $\widehat{S}$ that, in this case, the open subsets $\widehat{C}_{\bb{z}}=i(C)$ and $\widehat{C}_f$ (with $f$ homogeneous in $S_+$) form a \emph{cover} of $\widehat{C}$.
Furthermore, with this notation,
\[
\label{II.8.3.2.5}
  i^{-1}(\widehat{C}_f) = C_f;
\tag{8.3.2.5}
\]
indeed, $\widehat{C}_f\cap i(C) = \widehat{C}_f\cap\widehat{C}_{\bb{z}} = \widehat{C}_{f\bb{z}} = \Spec(\widehat{S}_{(f\bb{z})})$.
But, if $d=\deg(f)$, then $\widehat{S}_{(f\bb{z})}$ is canonically isomorphic to $(\widehat{S}_{(\bb{z})})_{f/\bb{z}^d}$ \sref{II.2.2.2}, and it follows from the definition of the isomorphism in \sref{II.8.2.3.1} that the image of $(\widehat{S}_{(\bb{z})})_{f/\bb{z}^d}$ under the corresponding isomorphism of rings of fractions is exactly $S_f$.
Since $C_f=\Spec(S_f)$, this proves \sref{II.8.3.2.5} and shows, at the same time, that the morphism $i$ is affine;
furthermore, the restriction of $i$ to $C_f$, considered as a morphism to $\widehat{C}_f$, corresponds \sref[I]{I.1.7.3} to the canonical homomorphism $\widehat{S}_{(f)}\to\widehat{S}_{(f\bb{z})}$, and, by the above and \sref{II.8.2.3.2}, we can claim the following result:
\begin{env}[8.3.2.6]
\label{II.8.3.2.6}
If $Y=\Spec(A)$ is affine, and $\sh{S}=\widetilde{S}$, then, for every homogeneous $f$ in $S_+$, $\widehat{C}_f$ is canonically identified with $\Spec(S_f^\leq)$, and the morphism $C_f\to\widehat{C}_f$ given by restricting $i$ then corresponds to the canonical injection $S_f^\leq\to S_f$.
\end{env}

Now note that (for $Y$ affine) the complement of $\widehat{C}_{\bb{z}}$ in $\widehat{C}=\Proj(\widehat{S})$
\oldpage[II]{164}
is, by definition, the set of graded prime ideals of $\widehat{S}$ containing $\bb{z}$, which is exactly $j(X)$, by definition of $j$, which proves \sref{II.8.3.2.3}.

Finally, to prove the last claim of \sref{II.8.3.2}, we can assume that $Y$ is affine.
With the above notation, note that, in the ring $\widehat{S}$, $\bb{z}$ is not a zero divisor;
since $i(C)=\widehat{C}$, it suffices to prove the following lemma:
\begin{lemma}[8.3.2.7]
\label{II.8.3.2.7}
Let $T$ be a positively-graded ring, $Z=\Proj(T)$, and $g$ a homogeneous element of $T$ of degree $d>0$.
If $g$ is not a zero divisor in $T$, then $Z$ is the smallest closed subprescheme of $Z$ that contains $Z_g=D_+(g)$.
\end{lemma}

By \sref[I]{I.4.1.9}, the question is local on $Z$;
for every homogeneous element $h\in T_e$ ($e>0$), it thus suffices to prove that $Z_h$ is the smallest closed subprescheme of $Z_h$ that contains $Z_{gh}$;
it follows from the definitions and from \sref[I]{I.4.3.2} that this condition is equivalent to asking for the canonical homomorphism $T_{(h)}\to T_{(gh)}$ to be \emph{injective}.
But this homomorphism can be identified with the canonical homomorphism $T_{(h)}\to(T_{(h)})_{g^e/h^d}$ \sref{II.2.2.3}.
But since $g^e$ is not a zero divisor in $T$, $g^e/h^d$ is not a zero divisor in $T_h$ (nor \emph{a fortiori} in $T_{(h)}$), since the fact that $(g^e/h^d)(t/h^m)=0$ (for $t\in T$ and $m>0$) implies the existence of some $n>0$ such that $h^ng^et=0$, whence $h^nt=0$, and thus $t/h^m=0$ in $T_h$.
This thus finishes the proof \sref[0]{0.1.2.2}.
\end{proof}

\begin{env}[8.3.3]
\label{II.8.3.3}
We will often identify the affine cone $C$ with the subprescheme induced by the projective cone $\widehat{C}$ on the open subset $i(C)$ by means of the open immersion $i$.
The closed subprescheme of $C$ associated to the closed immersion $\varepsilon$ is called the \unsure{\emph{vertex prescheme}} of $C$;
we also say that $\varepsilon$, which is a $Y$-section of $C$, is the \unsure{\emph{vertex section}}, or the \emph{null section},  or $C$;
we can identify $Y$ with the \unsure{vertex prescheme} of $C$ by means of $\varepsilon$.
Also, $i\circ\varepsilon$ is a $Y$-section of $\widehat{C}$, and thus also a closed immersion \sref[I]{I.5.4.6}, corresponding to the canonical surjective homomorphism of degree zero $\widehat{\sh{S}} = \sh{S}[\bb{z}] \to \sh{O}_Y[\bb{z}]$ \sref{II.3.1.7}, whose kernel is $\sh{S}_+[\bb{z}] = \sh{S}_+\widehat{\sh{S}}$;
the subprescheme of $\widehat{C}$ associated to this closed immersion is also called the \unsure{\emph{vertex prescheme}} of $\widehat{C}$, and $i\circ\varepsilon$ the \unsure{\emph{vertex section}} of $\widehat{C}$;
it can be identified with $Y$ by means of $i\circ\varepsilon$.
Finally, the closed subprescheme of $\widehat{C}$ associated to $j$ is called the \emph{part at infinity} of $\widehat{C}$, and can be identified with $X$ by means of $j$.
\end{env}

\begin{env}[8.3.4]
\label{II.8.3.4}
The subpreschemes of $C$ (resp. $\widehat{C}$) induced on the \emph{open} subsets
\[
\label{II.8.3.4.1}
  E = C \setmin \varepsilon(Y),
  \qquad
  \widehat{E} = \widehat{C} \setmin i(\varepsilon(Y))
\tag{8.3.4.1}
\]
are called (by an abuse of language) the \emph{pointed affine cone} and the \emph{pointed projective cone} (respectively) defined by $\sh{S}$;
we note that, despite this nomenclature, \emph{$E$ is not necessarily affine} over $Y$, nor $\widehat{E}$ projective over $Y$ \sref{II.8.4.3}.
When we identify $C$ with $i(C)$, we thus have the underlying spaces
\[
\label{II.8.3.4.2}
  C \cup \widehat{E} = \widehat{C},
  \qquad
  C \cap \widehat{E} = E
\tag{8.3.4.2}
\]
so that $\widehat{C}$ can be considered as being obtained by \emph{gluing} the open subpreschemes $C$ and $\widehat{E}$;
furthermore, by \sref{II.8.3.2.3},
\[
\label{II.8.3.4.3}
  E = \widehat{E} \setmin j(X).
\tag{8.3.4.3}
\]

If $Y=\Spec(A)$ is affine, then, with the notation of \sref{II.8.3.2},
\[
\label{II.8.3.4.4}
  E = \bigcup C_f,
  \qquad
  \widehat{E} = \bigcup \widehat{C}_f,
  \qquad
  C_f = C \cap \widehat{C}_f
\tag{8.3.4.4}
\]
where $f$ runs over the set of homogeneous elements of $S_+$ (or only a subset $M$ of this set, with $M$ generating an ideal of $S_+$ whose radical in $S_+$ is $S_+$ itself, or, equivalently, such that the $X_f$ for $f\in M$ cover $X$ \sref{II.2.3.14}).
The gluing of $C$ and $\widehat{C}_f$ along $C_f$ is thus determined by the injection morphisms $C_f\to C$ and $C_f\to\widehat{C}_f$, which, as we have seen \sref{II.8.3.2.6}, correspond (respectively) to the canonical homomorphisms $S\to S_f$ and $S_f^\leq\to S_f$.
\end{env}

\begin{proposition}[8.3.5]
\label{II.8.3.5}
With the notation of \sref{II.8.3.1} and \sref{II.8.3.4}, the morphism associated \sref{II.3.5.1} to the canonical injection $\vphi:\sh{S} \to \widehat{\sh{S}} = \sh{S}[\bb{z}]$ is a surjective affine morphism (called the canonical retraction)
\[
\label{II.8.3.5.1}
  p:\widehat{E} \to X
\tag{8.3.5.1}
\]
such that
\[
\label{II.8.3.5.2}
  p \circ j = 1_X.
\tag{8.3.5.2}
\]
\end{proposition}

\begin{proof}
To prove the proposition, we can restrict to the case where $Y$ is affine.
Taking into account the expression in \sref{II.8.3.4.4} for $\widehat{E}$, the fact that the domain of definition $G(\vphi)$ of $p$ is equal to $\widehat{E}$ will follow from the first of the following claims:
\begin{env}[8.3.5.3]
\label{II.8.3.5.3}
If $Y=\Spec(A)$ is affine, and $\sh{S}=\widetilde{S}$, then, for all homogeneous $f\in S_+$,
\[
\label{II.8.3.5.4}
  p^{-1}(X_f) = \widehat{C}_f
\tag{8.3.5.4}
\]
and the restriction of $p$ to $\widehat{C}_f=\Spec(S_f^\leq)$, considered as a morphism from $\widehat{C}_f$ to $X_f$, corresponds to the canonical injection $S_{(f)}\to S_f^\leq$.
If, further, $f\in S_1$, then $\widehat{C}_f$ is isomorphic to $X_f\otimes_{\bb{Z}}\bb{Z}[T]$ (where $T$ is an indeterminate).
\end{env}

Indeed, Equation~\sref{II.8.3.5.4} is exactly a particular case of \sref{II.2.8.1.1}, and the second claim is exactly the definition of $\Proj(\vphi)$ whenever $Y$ is affine \sref{II.2.8.1}.
Then Equation~\sref{II.8.3.5.2} and the fact that $p$ is surjective show that the composition $\sh{S}\to\widehat{\sh{S}}\to\sh{S}$ of the canonical homomorphisms is the identity on $\sh{S}$.
Finally, the last claim of \sref{II.8.3.5.3} follows from the fact that $S_f^\leq$ is isomorphic to $S_{(f)}[T]$ whenever $f\in S_1$ \sref{II.2.2.1}.
\end{proof}

\begin{corollary}[8.3.6]
\label{II.8.3.6}
The restriction
\[
\label{II.8.3.6.1}
  \pi: E \to X
\tag{8.3.6.1}
\]
of $p$ to $E$ is a surjective affine morphism.
If $Y$ is affine and $f$ homogeneous in $S_+$, then
\[
\label{II.8.3.6.2}
  \pi^{-1}(X_f) = C_f
\tag{8.3.6.2}
\]
and the restriction of $\pi$ to $C_f$ corresponds to the canonical injection $S_{(f)}\to S_f$.
If, further, $f\in S_1$, then $C_f$ is isomorphic to $X_f\otimes_{\bb{Z}}\bb{Z}[T,T^{-1}]$ (where $T$ is an indeterminate).
\end{corollary}

\begin{proof}
Equation~\sref{II.8.3.6.2} follows immediately from \sref{II.8.3.5.3} and \sref{II.8.3.2.5}, and shows the surjectivity of $\pi$;
we have already seen that the immersion $i$, restricted to $C_f$, corresponds
\oldpage[II]{166}
to the injection $S_f^\leq\to S_f$ \sref{II.8.3.2}.
Finally, the last claim is a consequence of the fact that, for $f\in S_1$, $S_f$ is isomorphic to $S_{(f)}[T,T^{-1}]$ \sref{II.2.2.1}.
\end{proof}

\begin{remark}[8.3.7]
\label{II.8.3.7}
Whenever $Y$ is affine, the elements of the underlying space of $E$ are the (not-necessarily-graded) prime ideals $\mathfrak{p}$ of $S$ not containing $S_+$, by definition of the immersion $\varepsilon$ \sref{II.8.3.2}.
For such an ideal $\mathfrak{p}$, the $\mathfrak{p}\cap S_n$ clearly satisfy the conditions of \sref{II.2.1.9}, and so there exists exactly one \emph{graded} prime ideal $\mathfrak{q}$ of $S$ such that $\mathfrak{q}\cap S_n=\mathfrak{p}\cap S_n$ for all $n$;
the map $\pi:E\to X$ of underlying spaces can then be understood via the equation
\[
\label{II.8.3.7.1}
  \pi(\mathfrak{p}) = \mathfrak{q}.
\tag{8.3.7.1}
\]

Indeed, to prove this equation, it suffices to consider some homogeneous $f$ in $S_+$ such that $\mathfrak{p}\in D(f)$, and to note that $\mathfrak{q}_{(f)}$ is the inverse image of $\mathfrak{p}_f$ under the injection $S_{(f)}\to S_f$.
\end{remark}

\begin{corollary}[8.3.8]
\label{II.8.3.8}
If $\sh{S}$ is generated by $\sh{S}_1$, then the morphisms $p$ and $\pi$ are of finite type;
for all $x\in X$, the fibre $p^{-1}(x)$ is isomorphic to $\Spec(\kres(x)[T])$, and the fibre $\pi^{-1}$ isomorphic to $\Spec(\kres(x)[T,T^{-1}])$
\end{corollary}

\begin{proof}
This follows immediately from \sref{II.8.3.5} and \sref{II.8.3.6} by noting that, whenever $Y$ is affine and $S$ is generated by $S_1$, the $X_f$, for $f\in S_1$, form a cover of $X$ \sref{II.2.3.14}.
\end{proof}

\begin{remark}[8.3.9]
\label{II.8.3.9}
The pointed affine cone corresponding to the graded $\sh{O}_Y$-algebran $\sh{O}_Y[T]$ (where $T$ is an indeterminate) can be identified with $G_m=\Spec(\sh{O}_Y[T,T^{-1}])$, since it is exactly $C_T$, as we have seen in \sref{II.8.3.2} (see \sref{II.8.4.4} for a more general result).
This prescheme is canonical endowed with the structure of a ``\emph{$Y$-scheme in commutative groups}''.
This idea will be explained in detail later on, but, for now, can be quickly summarised as follows.
A $Y$-scheme in groups is a $Y$-scheme $G$ endowed with two $Y$-morphisms, $p:G\times_Y G\to G$ and $s:G\to G$, that satisfy conditions formally analogous to the axioms of the composition law and the symmetry law of a group: the diagram
\[
  \xymatrix{
    G \times G \times G
      \ar[r]^{p \times 1}
      \ar[d]_{1 \times p}
  & G \times G
      \ar[d]^{p}
  \\G \times G
      \ar[r]_{p}
  & G
  }
\]
should commute (``associativity''), and there should be a condition which corresponds to the fact that, for groups, the maps
\[
  (x,y)
  \mapsto
  (x,x^{-1},y)
  \mapsto
  (x,x^{-1}y)
  \mapsto
  x(x^{-1}y)
\]
and
\[
  (x,y)
  \mapsto
  (x,x^{-1},y)
  \mapsto
  (x,yx^{-1})
  \mapsto
  (yx^{-1})x
\]
should both reduce to $(x,y)\mapsto y$;
the sequence of morphisms corresponding, for example, to the first composite map is
\[
  G \times G
  \xrightarrow{(1,s) \times 1}
  G \times G \times G
  \xrightarrow{1 \times p}
  G\times G
  \xrightarrow{p}
  G
\]
and the reader should write down the second sequence.

\oldpage[II]{167}
It is immediate \sref[I]{I.3.4.3} that the data of a structure of a $Y$-scheme in groups on a $Y$-scheme $G$ is equivalent to the data, for \emph{every} $Y$-prescheme $Z$, of a \emph{group} structure on the set $\Hom_Y(Z,G)$, where these structures should be such that, for every $Y$-morphism $Z\to Z'$, the corresponding map $\Hom_Y(Z',G)\to\Hom_Y(Z,G)$ is a group homomorphism.
In the particular case of $G_m$ that we consider here, $\Hom_Y(Z,G)$ can be identified with the set of $Z$-sections of $Z\times_Y G_m$ \sref[I]{I.3.3.14}, and thus with the set of $Z$-sections of $\Spec(\sh{O}_Z[T,T^{-1}])$;
finally, the same reasoning as in \sref[I]{I.3.3.15} shows that this set is canonically identified with the set of \emph{invertible} elements of the ring $\Gamma(Z,\sh{O}_Z)$, and the group structure on this set is the structure coming from the multiplication in the ring $\Gamma(Z,\sh{O}_Z)$.
The reader can verify that the morphisms $p$ and $s$ from above are obtained in the following way: they correspond, by \sref{II.1.2.7} and \sref{II.1.4.6}, to the homomorphisms of $\sh{O}_Y$-algebras
\begin{align*}
  \pi: &\sh{O}_Y[T,T^{-1}] \to \sh{O}_Y[T,T^{-1},T',T^{'-1}]
\\\sigma: &\sh{O}_Y[T,T^{-1}] \to \sh{O}_Y[T,T^{-1}]
\end{align*}
and are entirely defined by the data of $\pi(T)=TT'$ and $\sigma(T)=T^{-1}$.

With this in mind, $G_m$ can be considered as a ``\emph{universal domain of operators}'' for every \emph{affine cone} $C=\Spec(\sh{S})$, where $\sh{S}$ is a quasi-coherent positively-graded $\sh{O}_Y$-algebra.
This means that we can canonically define a $Y$-morphism $G_m\times_Y C\to C$ which has the formal properties of an external law of a set endowed with a group of operators;
or, again, as above for schemes in groups, we can give, for every $Y$-prescheme $Z$, an external law on $\Hom_Y(Z,C)$, having the group $\Hom_Y(Z,G_m)$ as its set of operators, with the usual axioms of sets endowed with a group of operators, and a compatibility condition with respect to the $Y$-morphisms $Z\to Z'$.
In the current case, the morphism $G_m\times_Y C\to C$ is defined by the data of a homomorphism of $\sh{O}_Y$-algebras $\sh{S} \to \sh{S}\otimes_{\sh{O}_Y}\sh{O}_Y[T,T^{-1}] = \sh{S}[T,T^{-1}]$, which associates, to each section $s_n\in\Gamma(U,\sh{S}_n)$ (where $U$ is an open subset of $Y$), the section $s_n T^n\in\Gamma(U,\sh{S}\otimes_{\sh{O}_Y}\sh{O}_Y[T,T^{-1}])$.

Conversely, suppose that we are given a quasi-coherent, \emph{a priori non-graded}, $\sh{O}_Y$-algebra, and, on $C=\Spec(\sh{S})$, a structure of a ``\emph{$Y$-scheme in sets endowed with a group of operators}'' that has the $Y$-scheme in groups $G_m$ as its domain of operators;
then we canonically obtain a \emph{grading} of $\sh{O}_Y$-algebras on $\sh{S}$.
Indeed, the data of a $Y$-morphism $G_m\times_Y C\to C$ is equivalent to that of a homomorphism of $\sh{O}_Y$-algebras $\psi:\sh{S}\to\sh{S}[T,T^{-1}]$, which can be written as $\psi=\sum_{n\in\bb{Z}}\psi_n T^n$, where the $\psi_n:\sh{S}\to\sh{S}$ are homomorphisms of $\sh{O}_Y$-modules (with $\psi_n(s)=0$ except for finitely many $n$ for every section $s\in\Gamma(U,\sh{S})$, for any open subset $U$ of $Y$).
We can then prove that the axioms of sets endowed with a group of operators imply that the $\psi_n(\sh{S})=\sh{S}_n$ define a grading (in positive or negative degree) of $\sh{O}_Y$-algebras on $\sh{S}$, with the $\psi_n$ being the corresponding projectors.
We also have the notation of a structure of an ``\emph{affine cone}'' on every affine $Y$-scheme, defined in a ``geometric'' way without any reference to any prior grading.
\oldpage[II]{168}
We will not further develop this point of view here, and we leave the work of precisely formulating the definitions and results corresponding to the information given above to the reader.
\end{remark}


\subsection{Projective closure of a vector bundle}
\label{subsection:II.8.4}

\begin{env}[8.4.1]
\label{II.8.4.1}
Let $Y$ be a prescheme, and $\sh{E}$ a quasi-coherent $\sh{O}_Y$-module.
If we take $\sh{S}$ to be the graded $\sh{O}_Y$-algebra $\bb{S}_{\sh{O}_Y}(\sh{E})$, then Definition~\eref{eq:2.8.3.1.1} shows that $\widehat{\sh{S}}$ can be identified with $\bb{S}_{\sh{O}_Y}(\sh{E}\oplus\sh{O}_Y)$.
With the affine cone $\Spec(\sh{S})$ defined by $\sh{S}$ being, by definition, $\bb{V}(\sh{E})$, and $\Proj(\sh{S})$ being, by definition, $\bb{P}(\sh{E})$, we see that:
\end{env}
\begin{proposition}[8.4.2]
\label{II.8.4.2}
The projective closure of a vector bundle $\bb{V}(\sh{E})$ on $Y$ is canonically isomorphic to $\bb{P}(\sh{E}\oplus\sh{O}_Y)$, and the part at infinity of the latter is canonically isomorphic to $\bb{P}(\sh{E})$.
\end{proposition}

\begin{remark}[8.4.3]
\label{II.8.4.3}
Take, for example, $\sh{E}=\sh{O}_Y^r$ with $r\geq2$;
then the pointed cones $E$ and $\widehat{E}$ defined by $\sh{S}$ are nether affine nor projective on $Y$ if $Y\neq\emp$.
The second claim is immediate, because $\widehat{C}=\bb{P}(\sh{O}_Y^{r+1})$ is projective on $Y$, and the underlying spaces of $E$ and $\widehat{E}$ are non-closed open subsets of $\widehat{C}$, and so the canonical immersions $E\to\widehat{C}$ and $\widehat{E}\to\widehat{C}$ are not projective \sref{II.5.5.3}, and we conclude by appealing to \sref{II.5.5.5}[v].
Now, supposing, for example, that $Y=\Spec(A)$ is affine, and $r=2$, then $C=\Spec(A[T_1,T_2])$, and $E$ is then the prescheme induced by $C$ on the open subset $D(T_1)\cup D(T_2)$;
but we have already seen that the latter is not affine \sref[I]{I.5.5.11};
\emph{a fortiori} $\widehat{E}$ cannot be affine, since $E$ is the open subset where the section $\bb{z}$ over $\widehat{E}$ does not vanish \sref{II.8.3.2}.

However:
\end{remark}
\begin{proposition}[8.4.4]
\label{II.8.4.4}
If $\sh{L}$ is an invertible $\sh{O}_Y$-module, then there are canonical isomorphisms for both the pointed cones $E$ and $\widehat{E}$ corresponding to $C=\bb{V}(\sh{L})$:
\[
\label{eq:2.8.4.4.1}
  \Spec\left(
    \bigoplus_{n\in\bb{Z}}\sh{L}^{\otimes n}
  \right)
  \simto
  E
\tag{8.4.4.1}
\]
\[
\label{eq:2.8.4.4.2}
  \bb{V}(\sh{L}^{-1})
  \simto
  \widehat{E}.
\tag{8.4.4.2}
\]

Furthermore, there exists a canonical isomorphism from the projective closure of $\bb{V}(\sh{L})$ to the projective closure of $\bb{V}(\sh{L}^{-1})$ that sends the null section (resp. the part at infinity) of the former to the part at infinity (resp. the null section) of the second.
\end{proposition}

\begin{proof}
We have here that $\sh{S}=\bigoplus_{n\geq0}\sh{L}^{\otimes n}$;
the canonical injection
\[
  \sh{S} \to \bigoplus_{n\in\bb{Z}} \sh{L}^{\otimes n}
\]
defines a canonical dominant morphism
\[
\label{eq:2.8.4.4.3}
  \Spec\left(
    \bigoplus_{n\in\bb{Z}} \sh{L}^{\otimes n}
  \right)
  \to
  \bb{V}(\sh{L})
  =
  \Spec\left(
    \bigoplus_{n\geq0} \sh{L}^{\otimes n}
  \right)
\tag{8.4.4.3}
\]
and it suffices to prove that this morphism is an isomorphism from the scheme $\Spec(\bigoplus_{n\in\bb{Z}} \sh{L}^{\otimes n})$ to $E$.
Since the questions is local on $Y$, we can assume that $Y=\Spec(A)$ is affine
\oldpage[II]{169}
and that $\sh{L}=\sh{O}_Y$, and so $\sh{S}=(A[T])\supertilde$ and $\bigoplus_{n\in\bb{Z}}\sh{L}^{\times n} = (A[T,T^{-1}])\supertilde$.
But $A[T,T^{-1}]$ is the ring of fractions $A[T]_T$ of $A[T]$, and thus \eref{eq:2.8.4.4.3} identifies \unsure{$\bigoplus_{n\in\bb{Z}}\sh{L}^{\otimes n}$} with the prescheme induced by $C=\bb{V}(\sh{L})$ on the open subset $D(T)$;
the complement $V(T)$ of this open subset in $C$ is the underlying space of the closed subprescheme of $C$ defined by the ideal $TA[T]$, which is exactly the null section of $C$, and so $E=D(T)$.

The isomorphism in \eref{eq:2.8.4.4.2} will be a consequence of the last claim, since $\bb{V}(\sh{L}^{-1})$ is the complement of the part at infinity of its projective closure, and $\widehat{E}$ is the complement of the null section of the projective closure $C=\bb{V}(\sh{L})$.
But these projective closures are $\bb{P}(\sh{L}^{-1}\oplus\sh{O}_Y)$ and $\bb{P}(\sh{L}\oplus\sh{O}_Y)$ (respectively);
but we can write $\sh{L}\oplus\sh{O}_Y = \sh{L}\otimes(\sh{L}^{-1}\oplus\sh{O}_Y)$.
The existence of the desired canonical isomorphism then follows from \sref{II.4.1.4}, and everything reduces to showing that this isomorphism swaps the null sections and the parts at infinity.
For this, we can reduce to the case where $Y=\Spec(A)$ is affine, $L=Ac$, and $L^{-1}=Ac'$, with the canonical isomorphism $L\otimes L^{-1}\to A$ sending $c\otimes c'$ to the element $1$ of $A$.
Then $\bb{S}(L\oplus A)$ is the tensor product of $A[\bb{z}]$ with $\bigoplus_{n\geq0}Ac^{\otimes n}$, and $\bb{S}(L^{-1}\oplus A)$ is the tensor product of $A[\bb{z}]$ with $\bigoplus_{n\geq0}Ac^{'\otimes n}$, and the isomorphism defined in \sref{II.4.1.4} sends $\bb{z}^h\otimes c^{'\otimes(n-h)}$ to the element $\bb{z}^{n-h}\otimes c^{\otimes h}$.
But, in $\bb{P}(\sh{L}^{-1}\oplus\sh{O}_Y)$, the part at infinity is the set of points where the section $\bb{z}$ vanishes, and the null section is the section of points where the section $c'$ vanishes;
since we have analogous definitions for $\bb{P}(\sh{L}\oplus\sh{O}_Y)$, the conclusion follows immediately from the above explanation.
\end{proof}


\subsection{Functorial behaviour}
\label{subsection:II.8.5}

\begin{env}[8.5.1]
\label{II.8.5.1}
Let $Y$ and $Y'$ be prescheme, $q:Y'\to Y$ a morphism, and $\sh{S}$ (resp. $\sh{S}'$) a quasi-coherent \emph{positively}-graded $\sh{O}_Y$-algebra (resp. quasi-coherent \emph{positively}-graded $\sh{O}_{Y'}$-algebra).
Consider a $q$-morphism of graded algebras
\[
\label{eq:2.8.5.1.1}
  \vphi: \sh{S} \to \sh{S}'.
\tag{8.5.1.1}
\]

We know \sref{II.1.5.6} that this corresponds, canonically, to a morphism
\[
  \Phi = \Spec(\vphi): \Spec(\sh{S}') \to \Spec(\sh{S})
\]
such that the diagram
\[
\label{eq:2.8.5.1.2}
  \xymatrix{
    C'
      \ar[r]^{\Phi}
      \ar[d]
  & C
      \ar[d]
  \\Y'
      \ar[r]_{q}
  & Y
  }
\tag{8.5.1.2}
\]
commutes, where we write $C=\Spec(\sh{S})$ and $C'=\Spec(\sh{S}')$.
\emph{Suppose, further, that $\sh{S}_0=\sh{O}_Y$ and $\sh{S}'_0=\sh{O}_{Y'}$};
let $\varepsilon:Y\to C$ and $\varepsilon:Y'\to C'$ be the canonical immersions \sref{II.8.3.2};
we then have a commutative diagram
\[
\label{eq:2.8.5.1.3}
  \xymatrix{
    Y'
      \ar[r]^{q}
      \ar[d]_{\varepsilon'}
  & Y
      \ar[d]^{\varepsilon}
  \\C'
      \ar[r]_{\Phi}
  & C
  }
\tag{8.5.1.3}
\]
\oldpage[II]{170}
which corresponds to the diagram
\[
  \xymatrix{
    \sh{S}
      \ar[r]^{\vphi}
      \ar[d]
  & \sh{S}'
      \ar[d]
  \\\sh{O}_Y
      \ar[r]
  & \sh{O}_{Y'}
  }
\]
where the vertical arrows are the augmentation homomorphisms, and so the commutativity follows from the hypothesis that $\vphi$ is assumed to be a homomorphism of \emph{graded} algebras.
\end{env}

\begin{proposition}[8.5.2]
\label{II.8.5.2}
If $E$ (resp. $E'$) is the pointed affine cone defined by $\sh{S}$ (resp. $\sh{S}'$), then $\Phi^{-1}(E)\subset E'$;
if, further, $\Proj(\vphi):G(\vphi)\to\Proj(\sh{S})$ is everywhere defined (or, equivalently, if $G(\vphi)=\Proj(\sh{S}')$), then $\Phi^{-1}(E)=E'$, and conversely.
\end{proposition}

\begin{proof}
The first claim follows from the commutativity of \eref{eq:2.8.5.1.3}.
To prove the second, we can restrict to the case where $Y=\Spec(A)$ and $Y'=\Spec(A')$ are affine, and $\sh{S}=\widetilde{S}$ and $\sh{S}'=\widetilde{S'}$.
For every homogeneous $f$ in $S_+$, writing $f'=\vphi(f)$, we have that $\Phi^{-1}(C_f)=C'_{f'}$ \sref[I]{I.2.2.4.1};
saying that $G(\vphi)=\Proj(S')$ implies that the radical (in $S'_+$) of the ideal generated by the $f'=\vphi(f)$ is $S'_+$ itself (\sref{II.2.8.1} and \sref{II.2.3.14}), and this is equivalent to saying that the $C'_{f'}$ cover $E'$ \eref{II.8.3.4.4}.
\end{proof}

\begin{env}[8.5.3]
\label{II.8.5.3}
The $q$-morphism $\vphi$ canonically extends to a $q$-morphism of graded algebras
\[
\label{II.8.5.3.1}
  \widehat{\vphi}: \widehat{\sh{S}} \to \widehat{\sh{S}'}
\tag{8.5.3.1}
\]
by letting $\widehat{\vphi}(\bb{z})=\bb{z}$.
This induces a morphism
\[
  \widehat{\Phi} = \Proj(\widehat{\vphi}) : G(\widehat{\vphi}) \to \widehat{C} = \Proj(\widehat{\sh{S}})
\]
such that the diagram
\[
  \xymatrix{
    G(\widehat{\vphi})
      \ar[r]^{\widehat{\Phi}}
      \ar[d]
  & \widehat{C}
      \ar[d]
  \\Y'
      \ar[r]_{q}
  & Y
  }
\]
commutes \sref{II.3.5.6}.
It follows immediately from the definitions that, if we write $i:C\to\widehat{C}$ and $i':C'\to\widehat{C'}$ to mean the canonical open immersions \sref{II.8.3.2}, then $i'(C')\subset G(\widehat{\vphi})$, and the diagram
\[
\label{eq:2.8.5.3.2}
  \xymatrix{
    C'
      \ar[r]^{\Phi}
      \ar[d]_{i}
  & C
      \ar[d]^{i'}
  \\G(\widehat{\vphi})
      \ar[r]_{\widehat{\Phi}}
  & \widehat{C}
  }
\tag{8.5.3.2}
\]
commutes.
Finally, if we let $X=\Proj(\sh{S})$ and $X'=\Proj(\sh{S}')$, and if $j:X\to\widehat{C}$ and $j':X'\to\widehat{C'}$ are the canonical closed immersions \sref{II.8.3.2}, then it follows from the definition of these immersions that $j'(G(\vphi))\subset G(\widehat{\vphi})$, and that the diagram
\oldpage[II]{171}
\[
\label{II.8.5.3.3}
  \xymatrix{
    G(\vphi)
      \ar[r]^{\Proj(\vphi)}
      \ar[d]_{j'}
  & X
      \ar[d]^{j}
  \\G(\widehat{\vphi})
      \ar[r]_{\widehat{\Phi}}
  & \widehat{C}
  }
\tag{8.5.3.3}
\]
commutes.
\end{env}

\begin{proposition}[8.5.4]
\label{II.8.5.4}
If $\widehat{E}$ (resp. $\widehat{E'}$) is the pointed projective cone defined by $\sh{S}$ (resp. by $\sh{S}'$), then $\widehat{\Phi}^{-1}(\widehat{E}) \subset \widehat{E'}$;
furthermore, if $p:\widehat{E}\to X$ and $p':\widehat{E'}\to X'$ are the canonical retractions, then $p'(\widehat{\Phi}^{-1}(\widehat{E})) \subset G(\widehat{\vphi})$, and the diagram
\[
\label{II.8.5.4.1}
  \xymatrix{
    \widehat{\Phi}^{-1}(\widehat{E})
      \ar[r]^{\widehat{\Phi}}
      \ar[d]_{p'}
  & \widehat{E}
      \ar[d]^{p}
  \\G(\vphi)
      \ar[r]_{\Proj(\vphi)}
  & X
  }
\tag{8.5.4.1}
\]
commutes.
If $\Proj(\vphi)$ is everywhere defined, then so too is $\widehat{\Phi}$, and we have that $\widehat{\Phi}^{-1}(\widehat{E}) = \widehat{E'}$
\end{proposition}

\begin{proof}
The first claim follows from the commutativity of Diagrams \sref{II.8.5.1.3} and \sref{II.8.5.3.2}, and the two following claims from the definition of the canonical retractions \sref{II.8.3.5} and the definition of $\widehat{\vphi}$.
To see that $\widehat{\Phi}$ is everywhere defined whenever $\Proj(\vphi)$ is, we can restrict to the case where $Y=\Spec(A)$ and $Y'=\Spec(A')$ are affine, and where $\sh{S}=\widetilde{S}$ and $\sh{S}'=\widetilde{S'}$;
the hypothesis is that, when $f$ runs over the set of homogeneous elements of $S_+$, the radical in $S'_+$ of the ideal generated in $S'_+$ by the $\vphi(f)$ is $S'_+$ itself;
we thus immediately conclude that the radical in $(S'[\bb{z}])_+$ of the ideal generated by $\bb{z}$ and the $\vphi(f)$ is $(S'[\bb{z}])_+$ itself, whence our claim;
this also shows that $\widehat{E'}$ is the union of the $\widehat{C'}_{\vphi(f)}$, and hence equal to $\widehat{\Phi}^{-1}(\widehat{E})$.
\end{proof}

\begin{corollary}[8.5.5]
\label{II.8.5.5}
Whenever $\Proj(\vphi)$ is everywhere defined, the inverse image under $\widehat{\Phi}$ of the underlying space of the part at infinity (resp. of the vertex prescheme) of $\widehat{C'}$ is the underlying space of the part at infinity (resp. of the vertex prescheme) of $\widehat{C}$.
\end{corollary}

\begin{proof}
This follows immediately from \sref{II.8.5.4} and \sref{II.8.5.2}, taking into account the equalities \sref{II.8.3.4.1} and \sref{II.8.3.4.2}.
\end{proof}


\subsection{A canonical isomorphism for pointed cones}
\label{subsection:II.8.6}

\begin{env}[8.6.1]
\label{II.8.6.1}
Let $Y$ be a prescheme, $\sh{S}$ a quasi-coherent positively-graded $\sh{O}_Y$-algebra \emph{such that $\sh{S}_0=\sh{O}_Y$}, and let $X$ be the $Y$-scheme $\Proj(\sh{S})$.
We are going to apply the results of \sref{subsection:II.8.5} to the case where $Y'=X$, and $q:X\to Y$ is the structure morphism;
let
\[
\label{II.8.6.1.1}
  \sh{S}_X=\bigoplus_{n\in\bb{Z}}\sh{O}_X(n)
  \tag{8.6.1.1}
\]
\oldpage[II]{172}
which is a quasi-coherent graded $\sh{O}_X$-algebra, with multiplication defined by means of the canonical homomorphisms \sref{II.3.2.6.1}
\[
  \sh{O}_X(m)\otimes_{\sh{O}_X}\sh{O}_X(n)\to\sh{O}_X(m+n)
\]
whose associativity is ensured by the commutative diagram in \sref{II.2.5.11.4}.
Let $\sh{S}'$ be the quasi-coherent positively-graded $\sh{O}_X$-subalgebra $\sh{S}_X^\geq=\bigoplus_{n\geq0}\sh{O}_X(n)$ of $\sh{S}_X$.

Finally, consider the canonical $q$-morphism
\[
\label{II.8.6.1.2}
  \alpha: \sh{S} \to \sh{S}_X^\geq
\tag{8.6.1.2}
\]
defined in \sref{II.3.3.2.3} as a homomorphism $\sh{S}\to q_*(\sh{S}_X)$, but which clearly sends $\sh{S}$ to $q_*(\sh{S}_X^\geq)$.
Write
\[
\label{II.8.6.1.3}
  C_X = \Spec(\sh{S}_X^\geq),
  \quad
  \widehat{C}_X = \Proj(\sh{S}_X^\geq[\bb{z}]),
  \quad
  X' = \Proj(\sh{S}_X^\geq)
\tag{8.6.1.3}
\]
and denote by $E_X$ and $\widehat{E}_X$ the corresponding pointed affine and pointed projective cones (respectively);
denote the canonical morphisms defined in \sref{subsection:II.8.3} by $\varepsilon_X:X\to C_X$, $i_X:C\to\widehat{C}_X$, $j_X:X'\to\widehat{C}_X$, $p_X:\widehat{E}_X\to X'$, and $\pi_X:E_X\to X'$.
\end{env}

\begin{proposition}[8.6.2]
\label{II.8.6.2}
The structure morphism $u:X'\to X$ is an \emph{isomorphism}, and the morphism $\Proj(\alpha)$ is everywhere defined and identical to $u$.
The morphism $\Proj(\widehat{\alpha}):\widehat{C}_X\to\widehat{C}$ is everywhere defined, and its restrictions to $\widehat{E}_X$ and $E_X$ are \emph{isomorphisms} to $\widehat{E}$ and $E$ (respectively).
Finally, if we identify $X'$ with $X$ via $u$, then the morphisms $p_X$ and $\pi_X$ are identified with the structure morphisms of the $X$-preschemes $\widehat{E}_X$ and $E_X$.
\end{proposition}

\begin{proof}
We can clearly restrict to the case where $Y=\Spec(A)$ is affine, and $\sh{S}=\widetilde{S}$;
then $X$ is the union of affine open subsets $X_f$, where $f$ runs over the set of homogeneous elements of $S_+$, with the ring of each $X_f$ being $S_{(f)}$.
It follows from \sref{II.8.2.7.1} that
\[
\label{II.8.6.2.1}
  \Gamma(X_f, \sh{S}_X^\geq) = S_f^\geq.
\tag{8.6.2.1}
\]

So $u^{-1}(X_f)=\Proj(S_f^\geq)$.
But if $f\in S_d$ ($d>0$), then $\Proj(S_f^\geq)$ is canonically isomorphic to $\Proj((S_f^\geq)^{(d)})$ \sref{II.2.4.7}, and we also know that $(S_f^\geq)^{(d)}=(S^{(d)})_f^\geq$ can be identified with $S_{(f)}[T]$ \sref{II.2.2.1} by the map $T\mapsto f/1$;
we thus conclude \sref{II.3.1.7} that the structure morphism $u^{-1}(X_f)\to X_f$ is an isomorphism, whence the first claim.
To prove the second, note that the restriction $u^{-1}(X_f)\cap G(\alpha)\to X=\Proj(S)$ of $\Proj(\alpha)$ corresponds to the canonical map $x\mapsto x/1$ from $S$ to $S_f^\geq$ \sref{II.2.6.2};
we thus deduce, first of all, that $G(\alpha)=X'$, and then, taking into account the fact that $u^{-1}(X_f)=(u^{-1}(X_f))_{f/1}$, that it follows from \sref{II.2.8.1.1} that the image of $u^{-1}(X_f)$ under $\Proj(\alpha)$ is contained in $X_f$, and the restriction of $\Proj(\alpha)$ to $u^{-1}(X_f)$, considered as a morphism to $X_f=\Spec(S_{(f)})$, is indeed identical to that of $u$.
Finally, applying \sref{II.8.3.5.4} to $p_X$ instead of $p$, we see that $p_X^{-1}(u^{-1}(X_f)) = \Spec((S_f^\geq)_{f/1}^\leq)$, and this open subset is, by \sref{II.8.5.4.1}, the inverse image under $\Proj(\widehat{\alpha})$ of $p^{-1}(X_f)=\Spec(S_f^\leq)$ \sref{II.8.3.5.3}.
Taking \sref{II.8.2.3.2} into account, the restriction of $\Proj(\widehat{\alpha})$ to $p_X^{-1}(u^{-1}(X_f))$ corresponds to the isomorphism inverse to \sref{II.8.2.7.2}, restricted to $S_f^\leq$, whence the third claim;
the last claim is evident by definition.

\oldpage[II]{173}
We note also that it follows from the commutative diagram in \sref{II.8.5.3.2} that \emph{the restriction to $C_X$ of $\Proj(\widehat{\alpha})$ is exactly the morphism $\Spec(\alpha)$}.
\end{proof}

\begin{corollary}[8.6.3]
\label{II.8.6.3}
Considered as $X$-schemes, $\widehat{E}_X$ is canonically isomorphic to $\Spec(\sh{S}_X^\leq)$, and $E_X$ to $\Spec(\sh{S}_X)$.
\end{corollary}

\begin{proof}
Since we know that the morphisms $p_X$ and $\pi_X$ are affine (\sref{II.8.3.5} and \sref{II.8.3.6}), it suffices (given \sref{II.1.3.1}) to prove the corollary in the case where $Y=\Spec(A)$ is affine and $\sh{S}=\widetilde{S}$.
The first claim follows from the existence of the canonical isomorphisms \sref{II.8.2.7.2} $(S_f^\geq)_{f/1}^\leq \simto S_f^\leq$ and from the fact that these isomorphisms are compatible with the map sending $f$ to $fg$ (where $f$ and $g$ are homogeneous in $S_+$).
Similarly, applying \sref{II.8.3.6.2} to $\pi_X$ instead of $\pi$, we see that $\pi_X^{-1}(u^{-1}(X_f)) = \Spec((S_f^\geq)_{f/1})$ for $f$ homogeneous in $S_+$, and the second claim then follows from the existence of the canonical isomorphisms \sref{II.8.2.7.2} $(S_f^\geq)_{f/1} \simto S_f$.

We can then say that $\widehat{C}_X$, considered as an $X$-scheme, is given by \emph{gluing} the affine $X$-schemes $C_X=\Spec(\sh{S}_X^\geq)$ and $\widehat{E}_X=\Spec(\sh{S}_X^\leq)$ over $X$, where the intersection of the two affine $X$-schemes is the open subset $E_X=\Spec(\sh{S}_X)$.
\end{proof}

\begin{corollary}[8.6.4]
\label{II.8.6.4}
Assume that $\sh{O}_X(1)$ is an invertible $\sh{O}_X$-module, and that $\sh{S}_X$ is isomorphic to $\bigoplus_{n\in\bb{Z}}(\sh{O}_X(1))^{\otimes n}$ (which will be the case, in particular, whenever $\sh{S}$ is generated by $\sh{S}_1$ (\sref{II.3.2.5} and \sref{II.3.2.7})).
Then the pointed projective cone $\widehat{E}$ can be identified with the rank-1 vector bundle $\bb{V}(\sh{O}_X(-1))$ on $X$, and the pointed affine cone $E$ with the subprescheme of this vector bundle induced on the complement of the null section.
With this identification, the canonical retraction $\widehat{E}\to X$ is identified with the structure morphism of the $X$-scheme $\bb{V}(\sh{O}_X(-1))$.
Finally, there exists a canonical $Y$-morphism $\bb{V}(\sh{O}_X(1))\to C$, whose restriction to the complement of the null section of $\bb{V}(\sh{O}_X(1))$ is an isomorphism from this complement to the pointed affine cone $E$.
\end{corollary}

\begin{proof}
If we write $\sh{L}=\sh{O}_X(1)$, then $\sh{S}_X^\geq$ is identical to $\bb{S}_{\sh{O}_X}(\sh{L})$, and so $\widehat{E}_X$ is canonically identified with $\bb{V}(\sh{L}^{-1})$, by \sref{II.8.6.3}, and $C_X$ with $\bb{V}(\sh{L})$.
The morphism $\bb{V}(\sh{L})\to C$ is the restriction of $\Proj(\widehat{\alpha})$, and the claims of the corollary are then particular cases of \sref{II.8.6.2}.
\end{proof}

We note that the inverse image under the morphism $\bb{V}(\sh{O}_X(1))\to C$ of the underlying space of the vertex prescheme of $C$ is the underlying space of the null section of $\bb{V}(\sh{O}_X(1))$ \sref{II.8.5.5};
but, in general, the corresponding subpreschemes of $C$ and of $\bb{V}(\sh{O}_X(1))$ are not isomorphic.
This problem will be studied below.


\subsection{Blowing up based cones}
\label{subsection:II.8.7}

\begin{env}[8.7.1]
\label{II.8.7.1}
Under the conditions of \sref{II.8.6.1}, we have, writing $r=\Proj(\widehat{\alpha})$, a commutative diagram
\[
\label{II.8.7.1.1}
  \xymatrix{
    X
      \ar[r]^{i_X\circ\varepsilon_X}
      \ar[d]_{q}
  & \widehat{C}_X
      \ar[d]^{r}
  \\Y
      \ar[r]^{i\circ\varepsilon}
  & \widehat{C}
  }
\tag{8.7.1.1}
\]
\oldpage[II]{174}
by \sref{II.8.5.1.3} and \sref{II.8.5.3.2};
furthermore, the restriction of $r$ to the complement $\widehat{C}_X\setmin i_X(\varepsilon_X(X))$ of the null section is an \emph{isomorphism} to the complement $\widehat{C}\setmin i(\varepsilon(Y))$ of the null section, by \sref{II.8.6.2}.
If we suppose, to simplify things, that $Y$ is affine, that $\sh{S}$ is of finite type and generated by $\sh{S}_1$, and that $X$ is projective over $Y$ and $\widehat{C}_X$ projective over $X$ \sref{II.5.5.1}, then $\widehat{C}_X$ is projective over $Y$ \sref{II.5.5.5}[ii], and \emph{a fortiori} over $\widehat{C}$ \sref{II.5.5.5}[v].
We then have a projective $Y$-morphism $r:\widehat{C}_X\to\widehat{C}$ (whose restriction to $C_X$ is a projective $Y$-morphism $C_X\to C$) that \unsure{\emph{contracts $X$ to $Y$}} and that induces an \emph{isomorphism} when we restrict to the \emph{complements of $X$ and $Y$}.
We thus have a connection between $C_X$ and $C$, analogous to that which exists between a blow-up prescheme and the original prescheme \sref{II.8.1.3}.
We will effectively show that $C_X$ can be identified with the homogeneous spectrum of a graded $\sh{O}_C$-algebra.
\end{env}

\begin{env}[8.7.2]
\label{II.8.7.2}
Keeping the notation of \sref{II.8.6.1}, consider, for all $n\geq0$, the quasi-coherent ideal
\[
\label{II.8.7.2.1}
  \sh{S}_{[n]} = \bigoplus_{m\geq n}\sh{S}_m
\tag{8.7.2.1}
\]
of the graded $\sh{O}_Y$-algebra $\sh{S}$.
It is clear that
\[
\label{II.8.7.2.2}
  \sh{S}_{[0]} = \sh{S},
  \qquad
  \sh{S}_{[n]} \subset \sh{S}_{[m]}
  \qquad\qquad
  \mbox{for $m\leq n$}
\tag{8.7.2.2}
\]
\[
\label{II.8.7.2.3}
  \sh{S}_{n} \sh{S}_{[m]} \subset \sh{S}_{[m+n]}.
\tag{8.7.2.3}
\]

Consider the $\sh{O}_C$-module associated to $\sh{S}_{[n]}$, which is a quasi-coherent ideal of $\sh{O}_C = \widetilde{\sh{S}}$ \sref{II.1.4.4}
\[
\label{II.8.7.2.4}
  \sh{I}_n = (\sh{S}_{[n]})\supertilde.
\tag{8.7.2.4}
\]

We thus deduce, from \sref{II.8.7.2.2} and \sref{II.8.7.2.3}, using \sref{II.1.4.4} and \sref{II.1.4.8.1}, the analogous formulas
\[
\label{II.8.7.2.5}
  \sh{I}_{[0]} = \sh{O}_C,
  \qquad
  \sh{I}_{[n]} \subset \sh{I}_{[m]}
  \qquad
  \text{for }m\leq n
\tag{8.7.2.5}
\]
\[
\label{II.8.7.2.6}
  \sh{I}_{n} \sh{I}_{[m]} \subset \sh{I}_{[m+n]}.
\tag{8.7.2.6}
\]

We are thus in the setting of \sref{II.8.1.1}, which leads us to introduce the quasi-coherent graded $\sh{O}_C$-algebra
\[
\label{II.8.7.2.7}
  \sh{S}^\natural
  =
  \bigoplus_{n\geq0}\sh{I}_n
  =
  \left(
    \bigoplus_{n\geq0} \sh{S}_{[n]}
  \right)\supertilde.
\tag{8.7.2.7}
\]
\end{env}

\begin{proposition}[8.7.3]
\label{II.8.7.3}
There is a canonical $C$-isomorphism
\[
\label{II.8.7.3.1}
  h: C_X \simto \Proj(\sh{S}^\natural).
\tag{8.7.3.1}
\]
\end{proposition}

\begin{proof}
Suppose first of all that $Y=\Spec(A)$ is affine, so that $\sh{S}=\widetilde{S}$, with $S$ a positively-graded $A$-algebra, and $C=\Spec(S)$.
Definition~\sref{II.8.2.7.4} then shows, with the notation of \sref{II.8.2.6}, that $\sh{S}^\natural=(S^\natural)\supertilde$.
To define \sref{II.8.7.3.1}, consider a homogeneous element $f\in S_d$ ($d>0$) and the corresponding element $f^\natural\in S^\natural$ \sref{II.8.2.6};
the $S$-isomorphism in \sref{II.8.2.7.3} then defines a $C$-isomorphism
\[
\label{II.8.7.3.2}
  \Spec(S_f^\geq) \simto \Spec(S_{(f^\natural)}^\natural).
\tag{8.7.3.2}
\]

\oldpage[II]{175}
But with the notation of \sref{II.8.6.2}, if $v:C_X\to X$ is the structure morphism, then it follows from \sref{II.8.6.2.1} that $v^{-1}(X_f)=\Spec(S_f^\geq)$.
We also have that $\Spec(S_{(f^\natural)}^\natural)=D_+(f^\natural)$, which means that \sref{II.8.7.3.2} defines an isomorphism $v^{-1}(X_f)\to D_+(f^\natural)$.
Furthermore, if $g\in S_e$ ($e>0$), then the diagram
\[
  \xymatrix{
    v^{-1}(X_{fg})
      \ar[r]^{\sim}
      \ar[d]
  & D_+(f^\natural g^\natural)
      \ar[d]
  \\v^{-1}(X_f)
      \ar[r]^{\sim}
  & D_+(f^\natural)
  }
\]
commutes, by definition of the isomorphism in \sref{II.8.2.7.3}.
Finally, by definition, $S_+$ is generated by the homogeneous $f$, and so it follows from \sref{II.8.2.10}[iv] and from \sref{II.2.3.14} that the $D_+(f^\natural)$ form a cover of $\Proj(S^\natural)$, and that the $v^{-1}(X_f)$ form a cover of $C_X$, since the $X_f$ form a cover of $X$;
in this case, we have thus defined the isomorphism \sref{II.8.7.3.1}.

To prove \sref{II.8.7.3} in the general case, it suffices to show that, if $U$ and $U'$ are affine open subsets of $Y$, given by rings $A$ and $A'$ (respectively), and such that $U'\subset U$, then, setting $\sh{S}|U=\widetilde{S}$ and $\sh{S}|U'=\widetilde{S'}$, the diagram
\[
\label{II.8.7.3.3}
  \xymatrix{
    C_{U'}
      \ar[r]
      \ar[d]
  & \Proj(S^{'\natural})
      \ar[d]
  \\C_U
      \ar[r]
  & \Proj(S^\natural)
  }
\tag{8.7.3.3}
\]
commutes.
But $S$ is canonically identified with $S\otimes_A A'$, and so $S^{'\natural}$ is canonically identified with
\[
  S^\natural \otimes_S S' = S^\natural \otimes_A A';
\]
thus $\Proj(S^{'\natural}) = \Proj(S^\natural)\times_U U'$ \sref{II.2.8.10};
similarly, if $X=\Proj(S)$ and $X'=\Proj(S')$, then $X'=X\times_U U'$ and $\sh{S}_{X'}=\sh{S}_X\otimes_{\sh{O}_U}U'$ \sref{II.3.5.4}, or, equivalently, $\sh{S}_{X'}=j^*(\sh{S}_X)$, where $j$ is the projection $X'\to X$.
We then \sref{II.1.5.2} have that $C_{U'} = C_U\times_X X' = C_U\times_U U'$, and the commutativity of \sref{II.8.7.3.3} is then immediate.
\end{proof}

\begin{remark}[8.7.4]
\label{II.8.7.4}
\begin{enumerate}
  \item[\rm{(i)}] The end of the proof of \sref{II.8.7.3} can be immediately generalised in the following way.
    Let $g:Y'\to Y$ be a morphism, $\sh{S}'=g^*(\sh{S})$, and $X'=\Proj(\sh{S}')$;
    then we have a commutative diagram
    \[
    \label{II.8.7.4.1}
      \xymatrix{
        C_{X'}
          \ar[r]
          \ar[d]
      & \Proj(\sh{S}^{'\natural})
          \ar[d]
      \\C_X
          \ar[r]
      & \Proj(\sh{S}^\natural)
      }
    \tag{8.7.4.1}
    \]

    Now let $\vphi:\sh{S}''\to\sh{S}$ be a homomorphism of graded $\sh{O}_Y$-algebras such that, if we write $X''=\Proj(\sh{S}'')$, then $u=\Proj(\vphi):X\to X''$ is everywhere defined;
    we also have
\oldpage[II]{176}
    a $Y$-morphism $v:C\to C''$ (with $C''=\Spec(\sh{S}'')$) such that $\sh{A}(v)=\vphi$, and, since $\vphi$ is a homomorphism of graded algebras, $\vphi$ induces a $v$-morphism of graded algebras $\psi:\sh{S}^{\prime\prime\natural}\to\sh{S}^\natural$ \sref{II.1.4.1}.
    Furthermore, it follows from \sref{II.8.2.10}[iv] and from the hypothesis on $\vphi$ that $\Proj(\psi)$ is everywhere defined.
    Finally, taking \sref{II.3.5.6.1} into account, there is a canonical $u$-morphism $\sh{S}_{X''}\to\sh{S}_X$, whence \sref{II.1.5.6} a morphism $w:C_{X''}\to C_X$.
    With this in mind, the diagram
    \[
    \label{II.8.7.4.2}
      \xymatrix{
        C_{X''}
          \ar[r]^-{\sim}
          \ar[d]_{w}
      & \Proj(\sh{S}^{\prime\prime\natural})
          \ar[d]^{\Proj(\psi)}
      \\C_X
          \ar[r]^-{\sim}
      & \Proj(\sh{S}^\natural)
      }
    \tag{8.7.4.2}
    \]
    is commutative, as we can immediately verify by restricting to the case where $Y$ is affine.
  \item[\rm{(ii)}] Note that, by \sref{II.8.7.2.5} and \sref{II.8.7.2.6}, we have $\sh{I}_1^m\subset\sh{I}_m\subset\sh{I}_1$ for all $m>0$.
    But, by definition, $\sh{I}_1=(\sh{S}_+)\supertilde$, and so $\sh{I}_1$ defines the closed subprescheme $\varepsilon(Y)$ in $C$ (\sref{II.1.4.10} and \sref{II.8.3.2});
    we thus conclude that, for all $m>0$, \emph{the support of $\sh{O}_C/\sh{I}_m$ is contained in the underlying space of the vertex prescheme $\varepsilon(Y)$};
    on the inverse image of the pointed affine cone $E$, the structure morphism $\Proj(\sh{S}^\natural)\to C$ thus restricts to an \emph{isomorphism} (by \sref{II.8.7.3} and \sref{II.8.7.1}).
    Furthermore, by canonically identifying $C$ with an open subset of $\widehat{C}$ \sref{II.8.3.3}, we can clearly extend the ideals $\sh{I}_m$ of $\sh{O}_C$ to ideals $\sh{J}_m$ of $\sh{O}_{\widehat{C}}$, by asking for it to agree with $\sh{O}_{\widehat{C}}$ on the open subset $\widehat{E}$ of $\widehat{C}$.
    If we define $\sh{T}=\bigoplus_{n\geq 0}\sh{J}_m$, which is a quasi-coherent graded $\sh{O}_{\widehat{C}}$-algebra, we can extend the isomorphism \sref{II.8.7.3.1} to a $\widehat{C}$-isomorphism
    \[
    \label{II.8.7.4.3}
      \widehat{C}_X \simto \Proj(\sh{T}).
    \tag{8.7.4.3}
    \]

    Indeed, over $\widehat{E}$, it follows from the above that $\Proj(\sh{T})$ is canonically identified with $\widehat{E}$, and we thus define the isomorphism \sref{II.8.7.4.3} over $\widehat{E}$ by asking for it to agree with the canonical isomorphism $\widehat{E}_X\to\widehat{E}$ \sref{II.8.6.2};
    it is clear that this isomorphism and \sref{II.8.7.3.1} then agree over $\widehat{E}$.
\end{enumerate}
\end{remark}

\begin{corollary}[8.7.5]
\label{II.8.7.5}
Suppose that there exists some $n_0>0$ such that
\[
\label{II.8.7.5.1}
  \sh{S}_{n+1} = \sh{S}_1\sh{S}_n
  \qquad
  \text{for }n\geq n_0.
\tag{8.7.5.1}
\]

Then the \unsure{vertex subprescheme} of $C_X$ (isomorphic to $X$) is the inverse image under the canonical morphism $r:C_X\to C$ of the vertex subprescheme of $C$ (isomorphic to $Y$).
Conversely, if this property is true, and if we further assume that $Y$ is Noetherian and that $\sh{S}$ is of finite type, then there exists some $n_0>0$ such that \sref{II.8.7.5.1} holds true.
\end{corollary}

\begin{proof}
Since the first claim is local on $Y$, we can assume that $Y=\Spec(A)$ is affine, so that $\sh{S}=\widetilde{S}$, with $S$ a positively-graded $A$-algebra.
The claim then follows from \sref{II.8.2.12}, since $\Proj(S^\natural\otimes_S S_0) = C_X\times_C\varepsilon(Y)$ (by the identification in \sref{II.8.7.3.1}), or, in other words, since this prescheme is the inverse image of $\varepsilon(Y)$ in $C_X$ \sref[I]{I.4.4.1}.
The converse also follows from \sref{II.8.2.12} whenever $Y$ is Noetherian affine and $S$ is of finite type.
\oldpage[II]{177}
If $Y$ is Noetherian (but not necessarily affine) and $\sh{S}$ is of finite type, then there exists a finite cover of $Y$ by Noetherian affine open subsets $U_i$, and we then deduce from the above that, for all $i$, there exists an integer $n_i$ such that $\sh{S}_{n+1}|U_i = (\sh{S}_1|U_i)(\sh{S}_n|U_i)$ for $n\geq n_i$;
the largest of the $n_i$ then ensures that \sref{II.8.7.5.1} holds true.
\end{proof}

\begin{env}[8.7.6]
\label{II.8.7.6}
Now consider the $C$-prescheme $Z$ given by \emph{blowing up} the \emph{vertex subprescheme $\varepsilon(Y)$} in the affine cone $C$;
by Definition~\sref{II.8.1.3}, it is exactly the prescheme $\Proj(\bigoplus_{n\geq0}\sh{S}_+^n)$;
the canonical injection
\[
\label{II.8.7.6.1}
  \iota: \bigoplus_{n\geq0} \sh{S}_+^n \to \sh{S}^\natural
\tag{8.7.6.1}
\]
defines (by the identification in \sref{II.8.7.3}) a canonical dominant $C$-morphism
\[
\label{II.8.7.6.2}
  G(\iota)\to Z
  \tag{8.7.6.2}
\]
where $G(\iota)$ is an open subset of $C_X$ \sref{II.3.5.1};
note that it could be the case that $G(\iota)\neq C_X$, as shown by the example where $Y=\Spec(K)$, with $K$ a field, and $\sh{S}=\widetilde{S}$, with $S=K[\bb{y}]$, where $\bb{y}$ is an indeterminate \emph{of degree 2};
if $R_n$ denotes the set $(S_+)^n$, considered as a subset of $S_{[n]}=S_n^\natural$, then $S_+^\natural$ is not the radical in $S_+^\natural$ of the ideal generated by the union of the $R_n$ (cf. \sref{II.2.3.14}).
\end{env}

\begin{corollary}[8.7.7]
\label{II.8.7.7}
Assume that there exists some $n_0>0$ such that
\[
\label{II.8.7.7.1}
  \sh{S}_n=\sh{S}_1^n
  \quad
  \text{for }n\geq n_0.
  \tag{8.7.7.1}
\]

Then the canonical morphism \sref{II.8.7.6.2} is everywhere defined, and is an isomorphism $C_X\simto Z$.
Conversely, if this property is true, and if we further assume that $Y$ is Noetherian and that $\sh{S}$ is of finite type, then there exists some $n_0$ such that \sref{II.8.7.7.1} holds true.
\end{corollary}

\begin{proof}
\label{II.8.7.7}
The first claim is local on $Y$, and thus follows from \sref{II.8.2.14};
the converse follows similarly, arguing as in \sref{II.8.7.5}.
\end{proof}

\begin{remark}[8.7.8]
\label{II.8.7.8}
Since condition \sref{II.8.7.7.1} implies \sref{II.8.7.5.1}, we see that, whenever it holds true, not only can $C_X$ be identified with the prescheme given by blowing up the vertex (identified with $Y$) of the affine cone $C$, but also the vertex (identified with $X$) of $C_X$ can be identified with the closed subprescheme given by the inverse image of the vertex $Y$ of $C$.
Furthermore, hypothesis~\sref{II.8.7.7.1} implies that, on $X=\Proj(\sh{S})$, the $\sh{O}_X$-modules $\sh{O}_X(n)$ are invertible (\sref{II.3.2.5} and \sref{II.3.2.9}), and that $\sh{O}_X(n)=\sh{L}^{\otimes n}$ with $\sh{L}=\sh{O}_X(1)$ (\sref{II.3.2.7} and \sref{II.3.2.9});
by Definition~\sref{II.8.6.1.1}, $C_X$ is thus the \emph{vector bundle} $\bb{V}(\sh{L})$ on $X$, and its vertex is the \emph{null section} of this vector bundle.
\end{remark}

\subsection{Ample sheaves and contractions}
\label{subsection:II.8.8}

\begin{env}[8.8.1]
\label{II.8.8.1}
Let $Y$ be a prescheme, $f:X\to Y$ a \emph{separated} and \emph{quasi-compact} morphism, and $\sh{L}$ an invertible $\sh{O}_X$-module that is \emph{ample relative to $f$}.
Consider the positively-graded $\sh{O}_Y$-algebra
\[
\label{II.8.8.1.1}
  \sh{S} = \sh{O}_Y \oplus \bigoplus_{n\geq1}f_*(\sh{L}^{\otimes n})
\tag{8.8.1.1}
\]
\oldpage[II]{178}
which is quasi-coherent \sref[I]{I.9.2.2}[a].
There is a canonical homomorphisms of graded $\sh{O}_X$-algebras
\[
\label{II.8.8.1.2}
  \tau: f^*(\sh{S}) \to \bigoplus_{n\geq0}\sh{L}^{\otimes n}
\tag{8.8.1.2}
\]
which, in degrees $\geq1$, agrees with the canonical homomorphism $\sigma:f^*(f_*(\sh{L}^{\otimes n})) \to \sh{L}^{\otimes n}$ \sref[0]{0.4.4.3}, and is the identity in degree 0.
The hypothesis that $\sh{L}$ is $f$-ample then implies (\sref{II.4.6.3} and \sref{II.3.6.1}) that the corresponding $Y$-morphism
\[
\label{II.8.8.1.3}
  r = r_{\sh{L},\tau} : X \to P = \Proj(\sh{S})
\tag{8.8.1.3}
\]
is everywhere defined and is a \emph{dominant open immersion}, and that
\[
\label{II.8.8.1.4}
  r^*(\sh{O}_P(n)) = \sh{L}^{\otimes n}
  \qquad\qquad
  \mbox{for all $n\in\bb{Z}$.}
\tag{8.8.1.4}
\]
\end{env}

\begin{proposition}[8.8.2]
\label{II.8.8.2}
Let $C=\Spec(\sh{S})$ be the affine cone defined by $\sh{S}$;
if $\sh{L}$ is $f$-ample, then there exists a canonical $Y$-morphism
\[
\label{II.8.8.2.1}
  g : V = \bb{V}(\sh{L}) \to C
\tag{8.8.2.1}
\]
such that the diagram
\[
\label{II.8.8.2.2}
  \xymatrix{
    X
      \ar[r]^{j}
      \ar[d]_{f}
  & \bb{V}(\sh{L})
      \ar[r]^{\pi}
      \ar[d]^{g}
  & X
      \ar[d]^{f}
  \\Y
      \ar[r]^{\varepsilon}
  & C
      \ar[r]^{\psi}
  & Y
  }
  \tag{8.8.2.2}
\]
commutes, where $\psi$ and $\pi$ are the structure morphisms, and $j$ and $\varepsilon$ the canonical immersions sending $X$ and $Y$ (respectively) to the null section of $\bb{V}(\sh{L})$ and the vertex prescheme of $C$ (respectively).
Furthermore, the restriction of $g$ to $\bb{V}(\sh{L})\setmin j(X)$ is an open immersion
\[
\label{II.8.8.2.3}
  \bb{V}(\sh{L})\setmin j(X)\to E=C\setmin\varepsilon(Y)
  \tag{8.8.2.3}
\]
into the pointed affine cone $E$ corresponding to $\sh{S}$.
\end{proposition}

\begin{proof}
With the notation of \sref{II.8.8.1}, let $\sh{S}_P^\geq = \bigoplus_{n\geq0}\sh{O}_P(n)$ and $C_p=\Spec(\sh{S}_P^\geq)$.
We know \sref{II.8.6.2} that there is a canonical morphism $h=\Spec(\alpha):C_p\to C$ such that the diagram
\[
\label{II.8.8.2.4}
  \xymatrix{
    C_P
      \ar[r]
      \ar[d]_{h}
  & P
      \ar[d]^{p}
  \\C
      \ar[r]^{\psi}
  & Y
  }
\tag{8.8.2.4}
\]
commutes; furthermore, if $\varepsilon_P:P\to C_P$ is the canonical immersion, then the diagram
\[
\label{II.8.8.2.5}
  \xymatrix{
    P
      \ar[r]^{p}
      \ar[d]_{\varepsilon_P}
  & C_P
      \ar[d]^{h}
  \\Y
      \ar[r]^{\varepsilon}
  & C
  }
\tag{8.8.2.5}
\]
commutes \sref{II.8.7.1.1}, and, finally, the restriction of $H$ to the pointed affine cone $E_P$ is an \emph{isomorphism} $E_P\simto E$ \sref{II.8.6.2}.
It follows from \sref{II.8.8.1.4} that
\[
  r^*(\sh{S}_P^\geq)=\bb{S}_{\sh{O}_X}(\sh{L})
\]
\oldpage[II]{179}
and so we have a canonical $P$-morphism $q:\bb{V}(\sh{L})\to C_P$, with the commutative diagram
\[
\label{II.8.8.2.6}
  \xymatrix{
    \bb{V}(\sh{L})
      \ar[r]^{\pi}
      \ar[d]_{q}
  & X
      \ar[d]^{r}
  \\C_P
      \ar[r]
  & P
  }
\tag{8.8.2.6}
\]
identifying $\bb{V}(\sh{L})$ with the product $C_P\times_P X$ \sref{II.1.5.2};
since $r$ is an open immersion, so too is $q$ \sref[I]{I.4.3.2}.
Furthermore, the restriction of $q$ to $\bb{V}(\sh{L})\setmin j(X)$ sends this prescheme to $E_p$, by \sref{II.8.5.2}, and the diagram
\[
\label{II.8.8.2.7}
  \xymatrix{
    X
      \ar[r]^{j}
      \ar[d]_{r}
  & \bb{V}(\sh{L})
      \ar[d]^{q}
  \\P
      \ar[r]^{\varepsilon_P}
  & C_P
  }
\tag{8.8.2.7}
\]
is commutative (since it is a particular case of \sref{II.8.5.1.3}).
The claims of \sref{II.8.8.2} immediately follow from these facts, by taking $g$ to be the composite morphism $h\circ q$.
\end{proof}

\begin{remark}[8.8.3]
\label{II.8.8.3}
Assume further that $Y$ is a \emph{Noetherian} prescheme, and that $f$ is a \emph{proper} morphism.
Since $r$ is then \emph{proper}\sref{II.5.4.4}, and thus closed, and since it is also a dominant open immersion, $r$ is necessarily an \emph{isomorphism} $X\simto P$.
Furthermore, we will see, in Chapter~III \sref[III]{III.2.3.5.1}, that $\sh{S}$ is then necessarily an $\sh{O}_Y$-algebra \emph{of finite type}.
It then follows that $\sh{S}^\natural$ is an $\sh{S}_0^\natural$-algebra \emph{of finite type} (\sref{II.8.2.10}[i] and \sref{II.8.7.2.7});
since $C_P$ is $C$-isomorphic to $\Proj(\sh{S}^\natural)$ \sref{II.8.7.3}, we see that the morphism $h:C_P\to C$ is \emph{projective};
since the morphism $r$ is an isomorphism, so too is $q:\bb{V}(\sh{L})\to C_P$, and we thus conclude that the morphism $g:\bb{V}(\sh{L})\to C$ is \emph{projective}.
Furthermore, since the restriction of $h$ to $E_P$ is an isomorphism to $E$, and since $q$ is an isomorphism, the restriction \sref{II.8.8.2.3} of $g$ is an isomorphism $\bb{V}(\sh{L})\setmin j(X)\simto E$.

If we further assume that $L$ is \emph{very ample} for $f$, then, as we will also see in Chapter~III \sref[III]{III.2.3.5.1}, there exists some integer $n_0>0$ such that $\sh{S}_n=\sh{S}_1^n$ for $n\geq n_0$.
We then conclude, by \sref{II.8.7.7}, that $\bb{V}(\sh{L})$ can be identified with the prescheme $Z$ given by \emph{blowing up the vertex prescheme \emph{(identified with $Y$)} in the affine cone $C$}, and that the \emph{null section} of $\bb{V}(\sh{L})$ (identified with $Y$) is the \emph{inverse image} of the vertex subprescheme $Y$ of $C$.

Some of the above results can in fact be proven even without the Noetherian hypothesis:
\end{remark}
\begin{corollary}[8.8.4]
\label{II.8.8.4}
Let $Y$ be a prescheme (resp. a quasi-compact scheme), $f:X\to Y$ a proper morphism, and $\sh{L}$ an invertible $\sh{O}_X$-module that is ample relative to $f$.
Then the morphism in \sref{II.8.8.2.1} is proper (resp. projective), and its restriction \sref{II.8.8.2.3} is an isomorphism.
\end{corollary}

\begin{proof}
To prove that $g$ is proper, we can restrict to the case where $Y$ is affine, and it then suffices to consider the case where $Y$ is a quasi-compact scheme.
The same arguments as in \sref{II.8.8.3} first of all show that $r$ is an \emph{isomorphism} $X\simto P$;
then $q$ is also an isomorphism, and, since the restriction of $h$ to $E_P$ is an isomorphism $E_P\simto E$, we have already seen that \sref{II.8.8.2.3} is an isomorphism.
It remains only to prove that $g$ is \emph{projective}.

Since $f$ is of finite type, by hypothesis, we can apply \sref{II.3.8.5} to the homomorphism
\oldpage[II]{180}
$\tau$ from \sref{II.8.8.1.2}:
there is an integer $d>0$ and a quasi-coherent $\sh{O}_Y$-submodule $\sh{E}$ of finite type of $\sh{S}_d$ such that, if $\sh{S}'$ is the $\sh{O}_Y$-subalgebra of $\sh{S}$ generated by $\sh{E}$, and $\tau'=\tau\circ q^*(\vphi)$ (where $\vphi$ is the canonical injection $\sh{S}'\to\sh{S}$), then $r'=r_{\sh{L},\tau'}$ is an immersion
\[
  X\to P'=\Proj(\sh{S}').
\]
Furthermore, since $\vphi$ is injective, $r'$ is also a \emph{dominant immersion} \sref{II.3.7.6};
the same argument as for $r$ then shows that $r'$ is a \emph{surjective closed immersion};
since $r'$ factors as $X\xrightarrow{r}\Proj(\sh{S})\xrightarrow{\Phi}\Proj(\sh{S}')$, where $\Phi=\Proj(\vphi)$, we thus conclude that $\Phi$ is also a \emph{surjective closed immersion}.
But this implies that $\Phi$ is an \emph{isomorphism};
we can restrict to the case where $Y=\Spec(A)$ is affine, and $\sh{S}=\widetilde{S}$ and $\sh{S}'=\widetilde{S'}$, with $S$ a graded $A$-algebra and $S'$ a graded subalgebra of $S$.
For every homogeneous element $t\in S'$, we have that $S'_{(t)}$ is a subring of $S_{(t)}$;
if we return to the definition of $\Proj(\vphi)$ \sref{II.2.8.1}, we see that it suffices to prove that, if $B'$ is a subring of a ring $B$, and if the morphism $\Spec(B)\to\Spec(B')$ corresponding to the canonical injection $B'\to B$ is a closed immersion, then this morphism is necessarily an \emph{isomorphism};
but this follows from \sref[I]{I.4.2.3}.
Furthermore, $\Phi^*(\sh{O}_{P'}(n))=\sh{O}_P(n)$ (\sref{II.3.5.2}[ii] and \sref{II.3.5.4}), and so $r^{'*}(\sh{O}_{P'}(n))$ is isomorphic to $\sh{L}^{\otimes n}$ \sref{II.4.6.3}.
Let $\sh{S}''=\sh{S}^{'(d)}$, so that \sref{II.3.1.8}[i] $X$ is canonically identified with $P''=\Proj(\sh{S}'')$, and $\sh{L}''=\sh{L}^{\otimes d}$ with $\sh{O}_{P''}(1)$ \sref{II.3.2.9}[ii].

Now, if $C''=\Spec(\sh{S}'')$, then $\sh{S}_{P''}^\geq=\bigoplus_{n\geq 0}\sh{O}_{P''}(n)$ can be identified with $\bigoplus_{n\geq 0}\sh{L}^{\prime\prime\otimes n}$, and thus $C_{P''}=\Spec(\sh{S}_{P''}^\geq)$ with $\bb{V}(\sh{L}'')$;
we also know \sref{II.8.7.3} that $C_{P''}$ is $C''$-isomorphic to $\Proj(\sh{S}^{\prime\prime\natural})$;
by the definition of $\sh{S}''$, we know that $\sh{S}^{\prime\prime\natural}$ is generated by $\sh{S}_1^{\prime\prime\natural}$, and that $\sh{S}_1^{\prime\prime\natural}$ is of finite type over $\sh{S}_0^{\prime\prime\natural}=\sh{S}''$ (\sref{II.8.2.10}[i and iii]), and so $\Proj(\sh{S}^{\prime\prime\natural})$ is \emph{projective} over $C''$ \sref{II.5.5.1}.
Consider the diagram
\[
\label{II.8.8.4.1}
  \xymatrix{
    \bb{V}(\sh{L})
      \ar[r]^-{g}
      \ar[d]_{u}
  & \Spec(\sh{S}) = C
      \ar[d]^{v}
  \\\bb{V}(\sh{L}'')
      \ar[r]^-{g''}
  & \Spec(\sh{S}'') = C''
  }
  \tag{8.8.4.1}
\]
where $g$ and $g''$ correspond, by \sref{II.1.5.6}, to the canonical $j$-morphisms
\[
  \sh{S}\to\bigoplus_{n\geq0}\sh{L}^{\otimes n}
  \quad\mbox{and}\quad
  \sh{S}''\to\bigoplus_{n\geq0}\sh{L}^{\prime\prime\otimes n}
\]
\sref{II.3.3.2.3} (see \sref{II.8.8.5} below), and $v$ and $u$ to the inclusion morphisms $\sh{S}''\to\sh{S}$ and $\bigoplus_{n\geq0}\sh{L}^{\otimes nd}\to\bigoplus_{n\geq0}\sh{L}^{\otimes n}$ (respectively);
it is immediate \sref{II.3.3.2} that this diagram is commutative.
We have just seen that $g''$ is a projective morphism;
we also know that $u$ is a \emph{finite} morphism.
Since the question is local on $X$, we can assume that $X$ is affine of ring $A$, and that $\sh{L}=\sh{O}_X$;
everything then reduces to noting that the ring $A[T]$ is a module of finite type over its subring $A[T^d]$ (with $T$ an indeterminate).
Since $Y$ is a quasi-compact scheme, and since $C''$ is affine over $Y$, we know that $C''$ is also a quasi-compact scheme,
\oldpage[II]{181}
and so $g''\circ u$ is a projective morphism \sref{II.5.5.5}[ii];
by commutativity of \sref{II.8.8.4.1}, $v\circ g$ is also projective, and, since $v$ is affine, thus separated, we finally conclude that $g$ is projective \sref{II.5.5.5}[v].
\end{proof}

\begin{env}[8.8.5]
Consider again the situation in \sref{II.8.8.1}.
We will see that the morphism $g:\bb{V}(\sh{L})\to C$ can be also be defined in a way that works for any invertible (but not necessarily ample) $\sh{O}_X$-module $\sh{L}$.
For this, consider the $f$-morphism
\[
\label{II.8.8.5.1}
  \tau^\flat:\sh{S}\to\bigoplus_{n\geq0}\sh{L}^{\otimes n}
  \tag{8.8.5.1}
\]
corresponding to the morphism $\tau$ from \sref{II.8.8.1.2}.
This induces \sref{II.1.5.6} a morphism $g':V\to C$ such that, if $\pi:V\to X$ and $\psi:C\to Y$ are the structure morphisms, the diagrams
\[
\label{II.8.8.5.2}
  \xymatrix{
    X
      \ar[d]_{f}
  & V
      \ar[l]_{\pi}
      \ar[d]^{g'}
  \\Y
  & C
      \ar[l]_{\psi}
  }
  \qquad
  \xymatrix{
    X
      \ar[r]^{j}
      \ar[d]_{f}
  & V
      \ar[d]^{g'}
  \\Y
      \ar[r]^{\varepsilon}
  & C
  }
\tag{8.8.5.2}
\]
commute (\sref{II.8.5.1.2} and \sref{II.8.5.1.3}).
We will show that (if we assume that $\sh{L}$ is $f$-ample) \emph{the morphisms $g$ and $g'$ are identical}.

Since the questions is local on $Y$, we can assume that $Y=\Spec(A)$ is affine, and (by \sref{II.8.8.1.3}) identify $X$ with an open subset of $P=\Proj(S)$, where $S=A\oplus\bigoplus_{n\geq0}\Gamma(X,\sh{L}^{\otimes n})$;
we then deduce, by \sref{II.8.8.1.4}, that $\Gamma(X,\sh{O}_P(n))=\Gamma(X,\sh{L}^{\otimes n})$ for all $n\in\bb{Z}$.
Taking into account the definition of $h=\Spec(\alpha)$, where $\alpha$ is the canonical $p$-morphism $\widetilde{S}\to\sh{S}_P^\geq$ \sref{II.8.6.1.2}, we have to show that the restriction to $X$ of $\alpha^\sharp:p^*(\widetilde{S})\to\sh{S}_P^\geq$ is identical to $\tau$.
Taking \sref[0]{0.4.4.3} into account, it suffices to show that, if we compose the canonical homomorphism $\alpha_n:S_n\to\Gamma(P,\sh{O}_P(n))$ with the restriction homomorphism $\Gamma(P,\sh{O}_P(n))\to\Gamma(X,\sh{O}_P(n))=\Gamma(X,\sh{L}^{\otimes n})$, then we obtain the identity, for all $n>0$;
but this follows immediately from the definition of the algebra $S$ and of $\alpha_n$ \sref{II.2.6.2}.
\end{env}

\begin{proposition}[8.8.6]
\label{II.8.8.6}
Assume (with the notation of \sref{II.8.8.5}) that, if we write $f=(f_0,\lambda)$, then the homomorphism $\lambda:\sh{O}_Y\to j_*(\sh{O}_X)$ is bijective;
then:
\begin{enumerate}
  \item[\rm{(i)}] if we write $g=(g_0,\mu)$, then $\mu:\sh{O}_C\to g_*(\sh{O}_V)$ is an isomorphism; and
  \item[\rm{(ii)}] if $X$ is integral (resp. locally integral and normal), then $C$ is integral (resp. normal).
\end{enumerate}
\end{proposition}

\begin{proof}
Indeed, the $f$-morphism $\tau^\flat$ is then an \emph{isomorphism}
\[
  \tau^\flat:\sh{S}=\psi_*(\sh{O}_C)\to f_*(\pi_*(\sh{O}_V))=\psi_*(g_*(\sh{O}_V))
\]
and the $Y$-morphism $g$ can be considered as that for which the homomorphism $\sh{A}(g)$ \sref{II.1.1.2} is equal to $\tau^\flat$.
To see that $\mu$ is an isomorphism of $\sh{O}_C$-modules, it suffices \sref{II.1.4.2} to see that $\sh{A}(\mu):\psi_*(\sh{O}_C)\to\psi_*(g_*(\sh{O}_V))$ is an isomorphism.
But, by Definition~\sref{II.1.1.2}, we have that $\sh{A}(\mu)=\sh{A}(g)$, whence the conclusion of (i).

To prove (ii), we can restrict to the case where $Y$ is affine, and so $\sh{S}=\widetilde{S}$, with
\oldpage[II]{182}
$S=\bigoplus_{n\geq0}\Gamma(X,\sh{L}^{\otimes n})$;
the hypothesis that $X$ is integral implies that the ring $S$ is integral \sref[I]{I.7.4.4}, and thus so too is $C$ \sref[I]{I.5.1.4}.
To show that $C$ is normal, we will use the following lemma:

\begin{lemma}[8.8.6.1]
\label{II.8.8.6.1}
Let $Z$ be a normal integral prescheme.
Then the ring $\Gamma(Z,\sh{O}_Z)$ is integral and integrally closed.
\end{lemma}

\begin{proof}
It follows from \sref[I]{I.8.2.1.1} that $\Gamma(Z,\sh{O}_Z)$ is the intersection, in the field of rational functions $R(Z)$, of the integrally closed rings $\sh{O}_z$ over all $z\in Z$.
\end{proof}

With this in mind, we first show that $V$ is \emph{locally integral} and \emph{normal};
for this, we can restrict to the case where $X=\Spec(A)$ is affine, with ring $A$ integral and integrally closed \sref{II.6.3.8}, and where $\sh{L}=\sh{O}_X$.
Since then $V=\Spec(A[T])$, and $A[T]$ is integral and integrally closed \cite[p.~99]{II-24}, this proves our claim.
For every affine open subset $U$ of $C$, $g^{-1}(U)$ is quasi-compact, since the morphism $g$ is quasi-compact;
since $V$ is locally integral, the connected components of $g^{-1}(U)$ are open integral preschemes in $g^{-1}(U)$, and thus finite in number, and, since $V$ is normal, these preschemes are also normal \sref{II.6.3.8}.
Then $\Gamma(U,\sh{O}_C)$, which is equal to $\Gamma(g^{-1}(U),\sh{O}_V)$, by (i), is the \unsure{direct sum} of finitely-many integral and integrally closed rings \sref{II.8.8.6.1}, which proves that $C$ is normal \sref{II.6.3.4}.
\end{proof}

\subsection{Grauert's ampleness criterion: statement}
\label{subsection:II.8.9}

We intend to show that the properties proven in \sref{II.8.8.2} \emph{characterise} $f$-ample $\sh{O}_X$-modules, and, more precisely, to prove the following criterion:
\begin{theorem}[8.9.1]
\label{II.8.9.1}
\emph{(Grauert's criterion).}
Let $Y$ be a prescheme, $p:X\to Y$ a separated and quasi-compact morphism, and $\sh{L}$ an invertible $\sh{O}_X$-module.
For $\sh{L}$ to be ample relative to $p$, it is necessary and sufficient for there to exist a $Y$-prescheme $C$, a $Y$-section $\varepsilon:Y\to C$ of $C$, and a $Y$-morphism $q:\bb{V}(\sh{L})\to C$, satisfying the following properties:
\begin{enumerate}
  \item[\rm{(i)}] the diagram
    \[
      \label{II.8.9.1.1}
        \xymatrix{
          X
            \ar[r]^{j}
            \ar[d]_{p}
        & \bb{V}(\sh{L})
            \ar[d]^{q}
        \\Y
            \ar[r]^{\varepsilon}
        & C
        }
      \tag{8.9.1.1}
    \]
    commutes, where $j$ is the null section of the vector bundle $\bb{V}(\sh{L})$; and
  \item[\rm{(ii)}] the restriction of $q$ to $\bb{V}(\sh{L})\setmin j(X)$ is a quasi-compact open immersion
    \[
      \bb{V}(\sh{L})\setmin j(X)\to X
    \]
    whose image does not intersect $\varepsilon(Y)$.
\end{enumerate}
\end{theorem}

Note that, if $C$ is \emph{separated} over $Y$, we can, in condition~(ii), remove the hypothesis that the open immersion is quasi-compact;
to see that this property (of quasi-compactness) is in fact a consequence of the other conditions, we can restrict to the case where $Y$ is affine, and the claim then follows from \sref[I]{I.5.5.1}{i} and \sref[I]{I.5.5.10}.
We can also remove
\oldpage[II]{183}
the same hypothesis if we assume that $X$ is Noetherian, since then $V$ is also Noetherian, and the claim follows from \sref[I]{I.6.3.5}.

\begin{corollary}[8.9.2]
\label{II.8.9.2}
If the morphism $p:X\to Y$ is proper, then we can, in the statement of Theorem~\sref{II.8.9.1}, assume that $q$ is proper, and replace ``open immersion'' by ``isomorphism''.
\end{corollary}

In a more suggestive manner, we can say (whenever $p:X\to Y$ is proper) that \emph{$\sh{L}$ is ample relative to $p$ if and only if we can ``\emph{contract}'' the null section of the vector bundle $\bb{V}(\sh{L})$ to the base prescheme $Y$.}
An important particular case is that where $Y$ is the spectrum of a field, and where the operation of ``contraction'' consists of contract the null section $\bb{V}(\sh{L})$ \emph{to a single point}.

\begin{env}[8.9.3]
\label{II.8.9.3}
The necessity of the conditions in Theorem~\sref{II.8.9.1} and Corollary~\sref{II.8.9.2} follow immediately from \sref{II.8.8.2} and \sref{II.8.8.4}.

To show that the conditions of \sref{II.8.9.1} suffices, consider a slightly more general situation.
For this, let (with the notation of \sref{II.8.8.2})
\[
  \sh{S}'=\bigoplus_{n\geq 0}\sh{L}^{\otimes n}
\]
and
\[
  V=\bb{V}(\sh{L})=\Spec(\sh{S}').
\]

The closed subprescheme $j(X)$, null section of $\bb{V}(\sh{L})$, is defined by the quasi-coherent sheaf of ideals $\sh{J}=(\sh{S}'_+)\supertilde$ of $\sh{O}_V$ \sref{II.1.4.10}.
This $\sh{O}_V$-module is \emph{invertible}, since this property is local on $X$, and this reduces to remarking that the ideal $TA[T]$ in a ring of polynomials $A[T]$ is a free cyclic $A[T]$-module.
Furthermore, it is immediate (again, because the question is local on $X$) that
\[
  \sh{L}=j^*(\sh{J})
\]
and
\[
  j_*(\sh{L})=\sh{J}/\sh{J}^2.
\]

Now, if
\[
  \pi:\bb{V}(\sh{L})\to X
\]
is the structure morphism, then $\pi_*(\sh{J})=\sh{S}'_+$ and $\pi_*(\sh{J}/\sh{J}^2)=\sh{L}$;
there are thus canonical homomorphisms $\sh{L}\to\pi_*(\sh{J})\to\sh{L}$, the first being the canonical injection $\sh{L}\to\sh{S}'_+$, and the second the canonical projection from $\sh{S}'_+$ to $\sh{S}'_1=\sh{L}$, and their composition being the identity.
We can also canonically embed $\pi_*(\sh{J}) = \sh{S}'_+ = \bigoplus_{n\geq1}\sh{L}^{\otimes n}$ into the \emph{product} $\prod_{n\geq1}\sh{L}^{\otimes n} = \varprojlim_n \pi_*(\sh{J}/\sh{J}^{n+1})$ (since $\pi_*(\sh{J}/\sh{J}^{n+1}) = \sh{L}\oplus\sh{L}^{\otimes 2}\oplus\ldots\oplus\sh{L}^{\otimes n}$), and we thus have canonical homomorphisms
\[
\label{II.8.9.3.1}
  \sh{L}\to\varprojlim\pi_*(\sh{J}/\sh{J}^{n+1})\to\sh{L}
  \tag{8.9.3.1}
\]
whose composition is the identity.

With this in mind, the generalisation of \sref{II.8.9.1} that we are going to prove is the following:
\end{env}

\begin{proposition}[8.9.4]
\label{II.8.9.4}
Let $Y$ be a prescheme, $V$ a $Y$-prescheme, and $X$ a closed subprescheme of $V$ defined by an ideal $\sh{J}$ of $\sh{O}_V$, which is an \emph{invertible} $\sh{O}_V$-module;
if $j:X\to V$ is
\oldpage[II]{184}
the canonical injection, then let $\sh{L} = j^*(\sh{J}) = \sh{J}\otimes_{\sh{O}_V}\sh{O}_X$, so that $j_*(\sh{L})=\sh{J}/\sh{J}^2$.
Assume that the structure morphism $p:X\to Y$ is separated and quasi-compact, and that the following conditions are satisfied:
\begin{enumerate}
  \item[\rm{(i)}] there exists a $Y$-morphism $\pi:V\to X$ of finite type such that $\pi\circ j=1_X$, and so $\pi_*(\sh{J}/\sh{J}^2)=\sh{L}$;
  \item[\rm{(ii)}] there exists a homomorphism of $\sh{O}_X$-modules $\vphi:\sh{L}\to\varprojlim\pi_*(\sh{J}/\sh{J}^{n+1})$ such that the composition
    \[
      \sh{L}
      \xrightarrow{\vphi}
      \varprojlim \pi_*(\sh{J}/\sh{J}^{n+1})
      \xrightarrow{\alpha}
      \pi_*(\sh{J}/\sh{J}^2) = \sh{L}
    \]
    (where $\alpha$ is the canonical homomorphism) is the identity;
  \item[\rm{(iii)}] there exists a $Y$-prescheme $C$, a $Y$-section $\varepsilon$ of $C$, and a $Y$-morphism $q:V\to C$ such that the diagram
    \[
    \label{II.8.9.4.1}
      \xymatrix{
        X
          \ar[r]^{j}
          \ar[d]_{p}
      & V
          \ar[d]^{q}
      \\Y
          \ar[r]_{\varepsilon}
      & C
      }
    \tag{8.9.4.1}
    \]
    commutes; and
  \item[\rm{(iv)}] the restriction of $q$ to $W=V\setmin j(X)$ is a quasi-compact open immersion into $C$, whose image does not intersect $\varepsilon(Y)$.
\end{enumerate}
Then $\sh{L}$ is ample relative to $p$.
\end{proposition}


\subsection{Grauert's ampleness criterion: proof}
\label{subsection:II.8.10}

\begin{lemma}[8.10.1]
\label{II.8.10.1}
Let $\pi:V\to X$ be a morphism, $j:X\to V$ an $X$-section of $V$ that is also a closed immersion, and $\sh{J}$ a quasi-coherent ideal of $\sh{O}_V$ that defines the closed subprescheme of $V$ associated to $j$.
Then the following all hold true.
\begin{enumerate}
  \item[\rm{(i)}] For all $n\geq0$, $\pi_*(\sh{O}_V/\sh{J}^{n+1})$ and $\pi_*(\sh{J}/\sh{J}^{n+1})$ are quasi-coherent $\sh{O}_X$-modules, and $\pi_*(\sh{O}_V/\sh{J})=\sh{O}_X$ and $\pi_*(\sh{J}/\sh{J}^2)=j^*(\sh{J})$.
  \item[\rm{(ii)}] If $X=\{\xi\}=\Spec(k)$, where $k$ is a field, then $\varprojlim\pi_*(\sh{O}_V/\sh{J}^{n+1})$ is isomorphic to the separated completion of the local ring $\sh{O}_{j(\xi)}$ for the $\mathfrak{m}_{j(\xi)}$-preadic topology.
  \item[\rm{(iii)}] Assume that $\sh{J}$ is an invertible $\sh{O}_V$-module (which implies that
    \[
      \sh{L} = j^*(\sh{J}) = \pi_*(\sh{J}/\sh{J}^2)
    \]
    is an invertible $\sh{O}_X$-module), and that there exists a homomorphism $\vphi:\sh{L}\to\varprojlim\pi_*(\sh{J}/\sh{J}^{n+1})$ such that the composition $\sh{L} \xrightarrow{\vphi} \varprojlim\pi_*(\sh{J}/\sh{J}^{n+1}) \xrightarrow{\alpha} \pi_*(\sh{J}/\sh{J}^2)$ (where $\alpha$ is the canonical homomorphism) is the identity.
    If we write $\sh{S}=\bigoplus_{n\geq0}\sh{L}^{\otimes n}$, then $\vphi$ canonically induces an isomorphism of $\sh{O}_X$-algebras from the completion $\widehat{\sh{S}}$ of $\sh{S}$ relative to its canonical filtration (the completion being isomorphic to the product $\prod_{n\geq0}\sh{L}^{\otimes n}$) to $\varprojlim\pi_*(\sh{O}_V/\sh{J}^{n+1})$.
\end{enumerate}
\end{lemma}

\begin{proof}
Note first of all that the support of the $\sh{O}_V$-module $\sh{O}_V/\sh{J}^{n+1}$ is $j(X)$, and the support of $\sh{J}/\sh{J}^{n+1}$ is contained in $j(X)$.
In the case of (ii), $j(X)$ is a closed point $j(\xi)$ of $V$,
\oldpage[II]{185}
and, by definition, $\pi_*(\sh{O}_V/\sh{J}^{n+1})$ is the fibre of $\sh{O}_V/\sh{J}^{n+1}$ at the point $j(\xi)$, or, equivalently, setting $C=\sh{O}_{j(\xi)}$, and denoting by $\mathfrak{m}$ the maximal ideal of $C$, the $C$-module $C/\mathfrak{m}^{n+1}$;
claim (ii) is then evident.

To prove (i), note that the question is local on $X$;
we can thus restrict to the case where $X$ is affine.
Let $U$ be an affine open subset of $V$;
then $j(X)\cap U$ is an affine open subset of $j(X)$, so $U_0=\pi(j(X)\cap U)$, which is isomorphic to it, is an affine open subset of $X$;
for every affine open subset $W_0\subset U_0$ in $X$, $W=\pi^{-1}(W_0)\cap U$ is an affine open subset of $V$, since $X$ is a scheme \sref[I]{I.5.5.10};
in particular, $U'=U\cap\pi^{-1}(U_0)$ is an affine open subset of $V$, and clearly $\pi(U')=U_0$ and $j(U_0)=j(X)\cap U$.
Then, by definition, $\Gamma(W_0,\pi_*(\sh{O}_V/\sh{J}^{n+1})) = \Gamma(\pi^{-1}(W_0),\sh{O}_V/\sh{J}^{n+1})$;
but since every point of $\pi^{-1}(W_0)$ not belonging to $j(W_0)$ has an open neighbourhood in $\pi^{-1}(W_0)$ not intersecting $j(X)$, and in which $\sh{O}_V/\sh{J}^{n+1}$ is thus zero, it is clear that the sections of $\sh{O}_V/\sh{J}^{n+1}$ over $\pi^{-1}(W_0)$ and over $W$ are in bijective correspondence.
In other words, if $\pi'$ is the restriction of $\pi$ to $U'$, then the $(\sh{O}_X|U_0)$-modules $\pi_*(\sh{O}_V/\sh{J}^{n+1})|U_0$ and $\pi'_*((\sh{O}_V/\sh{J}^{n+1})|U')$ are identical.
Since $U'$ and $U_0$ are affine, and since the $U_0$ cover $X$, we thus conclude \sref[I]{I.1.6.3} that $\pi_*(\sh{O}_V/\sh{J}^{n+1})$ is quasi-coherent, and the proof is identical for $\pi_*(\sh{J}/\sh{J}^{n+1})$.

Finally, to prove (iii), note that $\sh{S}$ is exactly $\bb{S}_{\sh{O}_X}(\sh{L})$;
so $\vphi$ canonically induces a homomorphism of $\sh{O}_X$-algebras $\psi:\sh{S}\to\varprojlim\pi_*(\sh{O}_V/\sh{J}^{n+1})$ \sref{II.1.7.4};
furthermore, this homomorphism sends $\sh{L}^{\otimes n}$ to $\varprojlim_m\pi_*(\sh{J}^n/\sh{J}^{n+1})$, and is thus continuous for the topologies considered, and indeed then extends to a homomorphism $\widehat{\psi}:\widehat{\sh{S}}\to\varprojlim\pi_*(\sh{O}_V/\sh{J}^{n+1})$.
To see that this is indeed an isomorphism, we can, as in the proof of (i), restrict to the case where $X=\Spec(A)$ and $V=\Spec(B)$ are affine, with $\sh{J}=\widetilde{\mathfrak{J}}$, where $\mathfrak{J}$ is an ideal of $B$;
there is an injection $A\to B$ corresponding to $\pi$ that identifies $A$ with a subring of $B$ that is \emph{complementary} to $B$, and $\sh{L}$ (resp. $\pi_*(\sh{O}_V/\sh{J}^{n+1})$) is the quasi-coherent $\sh{O}_X$-module associated to the $A$-module $L=\mathfrak{J}/\mathfrak{J}^2$ (resp. $B/\mathfrak{J}^{n+1}$).
Since $\sh{J}$ is an \emph{invertible} $\sh{O}_V$-module, we can further assume that $\mathfrak{J}=Bt$, where $t$ is not a zero divisor in $B$.
From the fact that $B=A\oplus Bt$, we deduce that, for all $n>0$,
\[
  B = A \oplus At \oplus At^2 \oplus \ldots \oplus At^n \oplus Bt^{n+1}
\]
and so there exists a canonical $A$-isomorphism from the ring of formal series $A[[T]]$ to $C=\varprojlim B/\mathfrak{J}^{n+1}$ that sends $T$ to $t$.
We also have that $L=A\overline{t}$, where $\overline{t}$ is the class of $t$ modulo $Bt^2$, and the homomorphism $\vphi$ sends, by hypothesis, $\overline{t}$ to an element $t'\in C$ that is congruent to $t$ modulo $Ct^2$.
We thus deduce, by induction on $n$, that
\[
  A \oplus At' \oplus \ldots \oplus At^{'n} \oplus Ct^{n+1}
  =
  A \oplus At \oplus \ldots \oplus At^n \oplus Ct^{n+1}
\]
which proves that the homomorphism $\widehat{\psi}$ does indeed correspond to an isomorphism from $\prod_{n\geq0}L^{\otimes n}$ to $C$.
\end{proof}

\begin{lemma}[8.10.2]
\label{II.8.10.2}
Under the hypotheses of Lemma~\sref{II.8.10.1}, let $g:X'\to X$ be a morphism,
\oldpage[II]{186}
write $V'=V\times_X X'$, and let $\pi':V'\to X'$ and $g:V'\to V$ be the canonical projections, so that we have the commutative diagram
\[
  \xymatrix{
    V
      \ar[d]_{\pi}
  & V'
      \ar[l]_{g'}
      \ar[d]^{\pi'}
  \\X
  & X'
      \ar[l]^{g}
  }
\]

Then $j'=j\times1_{X'}$ is an $X'$-section of $V'$ that is also a closed immersion, and $\sh{J}'=g^{'*}(\sh{J})\sh{O}_{V'}$ is the quasi-coherent ideal of $\sh{O}_{V'}$ that defines the closed subprescheme of $V'$ associated to $j'$.
Furthermore, $\pi'_*(\sh{O}_{V'}/\sh{J}{'n+1}) = g^*(\pi_*(\sh{O}_V/\sh{J}^{n+1}))$.
Finally, $\sh{J}'$ is an $\sh{O}_{V'}$-module that is canonically isomorphic to $g^{'*}(\sh{J})$, and is, in particular, invertible if $\sh{J}$ is an invertible $\sh{O}_V$-module.
\end{lemma}

\begin{proof}
The fact that $j'$ is a closed immersion follows from \sref[I]{I.4.3.1}, and it is an $X'$-section of $V'$ by functoriality of extension of the base prescheme.
Furthermore, if $Z$ (resp. $Z'$) is the closed subprescheme of $V$ (resp. $V'$) associated to $j$ (resp. $j'$), then $Z'=g^{'-1}(Z)$ \sref[I]{I.4.3.1}, and the second claim then follows from \sref[I]{I.4.4.5}.
To prove the other claims, we see, as in \sref{II.8.10.1}, that we can restrict to the case where $X$, $V$, and $X'$ (and thus also $V'$) are affine;
we keep the notation from the proof of \sref{II.8.10.1}, and let $X'=\Spec(A')$.
Then $V'=\Spec(B')$, where $B'=B\otimes_A A'$, and $\sh{J}'=\widetilde{\mathfrak{J}''}$, where $\mathfrak{J}'=\Im(\mathfrak{J}\otimes_A A')$.
Then $B'/\mathfrak{J}^{'n+1}=(B/\mathfrak{J}^{n+1})\otimes_A A'$;
furthermore, since $\mathfrak{J}$ is a direct factor (as an $A$-module) of $B$, $\mathfrak{J}\otimes_A A'$ is a direct factor (as an $A'$-module) of $B'$, and is thus canonically identified with $\mathfrak{J}'$.
\end{proof}

\begin{corollary}[8.10.3]
\label{II.8.10.3}
Assume that the hypotheses of Lemma~\sref{II.8.10.1} are satisfied, and assume further that $\pi$ is of finite type, and that $\sh{J}$ is an invertible $\sh{O}_V$-module.
Then, for all $x\in X$, the local ring at the point $j(x)$ of the fibre $\pi^{-1}(x)$ is a regular (thus integral) ring of dimension 1, whose completion is isomorphic to the formal series ring $\kres(x)[[T]]$ (where $T$ is an indeterminate);
furthermore, there exists exactly one irreducible component of $\pi^{-1}(x)$ that contains $j(x)$.
\end{corollary}

\begin{proof}
Since $\pi^{-1}(x)=V\times_X\Spec(\kres(x))$, we are led, by \sref{II.8.10.2}, to the case where $X$ is the spectrum of a field $K$.
Since $\pi$ is of finite type \sref[I]{I.6.4.3}[iv], $\sh{O}_{j(x)}$ is a Noetherian local ring, and thus separated for the $\mathfrak{m}_{j(x)}$-preadic topology \sref[0]{0.7.3.5};
it follows from \sref{II.8.10.1}[ii and iii] that the completion of this ring is isomorphic to $K[[T]]$, and so $\sh{O}_{j(x)}$ is regular and of dimension 1 (\cite[p.~17-01, th.~1]{I-1});
finally, since $\sh{O}_{j(x)}$ is integral, $j(x)$ belongs to exactly one of the (finitely many) irreducible components of $V$ \sref[I]{I.5.1.4}.
\end{proof}

\begin{corollary}[8.10.4]
\label{II.8.10.4}
Suppose that the hypotheses of Lemma~\sref{II.8.10.1} are satisfied, and assume further that $\sh{J}$ is an invertible $\sh{O}_V$-module.
Let $W=V\setmin j(X)$;
for every quasi-coherent ideal $\sh{K}$ of $\sh{O}_X$, let $\sh{K}_V=\pi^*(\sh{K})\sh{O}_V$ and $\sh{K}_W=\sh{K}_V|W$.
Then $\sh{K}_V$ is the largest quasi-coherent ideal of $\sh{O}_V$ whose restriction to $W$ is $\sh{K}_W$.
\end{corollary}

\begin{proof}
Indeed, we see as in \sref{II.8.10.1} that the question is local on $X$ and $V$;
we can thus reuse the notation from the proof of \sref{II.8.10.1}, with $\mathfrak{J}=Bt$, where $t$ is not a zero divisor in $B$.
Furthermore, we have $W=\Spec(B_t)$ and $\sh{K}=\widetilde{\mathfrak{K}}$, where $\mathfrak{K}$ is an ideal of $A$;
whence $\pi^*(\sh{K})\sh{O}_V=(\mathfrak{K}.B)\supertilde$ \sref[I]{I.1.6.9}, $\sh{K}_W=(\mathfrak{K}.B_t)\supertilde$, and the largest ideal
\oldpage[II]{187}
of $B$ whose canonical image in $B_t$ is $\mathfrak{K}.B_t$ is the inverse image of $\mathfrak{K}.B_t$, that is, the set of $s\in B$ such that, for some integer $n>0$, we have $t^ns\in\mathfrak{K}.B$.
We have to show that this last relation implies that $s\in\mathfrak{K}.B$, or again that the canonical image of $t$ is not a zero divisor in $B/\mathfrak{K}B=(A/\mathfrak{K})\otimes_AB$, which follows from \sref{II.8.10.2} applied to $X'=\Spec(A/\mathfrak{K})$.
\end{proof}

\begin{corollary}[8.10.5]
\label{II.8.10.5}
Suppose that the hypotheses of \sref{II.8.10.3} are satisfied;
let $W=V\setmin j(X)$, $x$ be a point of $X$, $\sh{K}$ a quasi-coherent ideal of $\sh{O}_X$, and $z$ the generic point of the irreducible component of $\pi^{-1}(x)$ that contains $j(x)$ \sref{II.8.10.3}.
\begin{enumerate}
    \item[\rm{(i)}] Let $g$ be a section of $\sh{O}_V$ over $V$ such that $g|W$ is a section of $\sh{K}_W$ over $W$ (using the notation from \sref{II.8.10.4}).
        Then $g$ is a section of $\sh{K}_V$;
        if further $g(z)\neq0$, and if, for every integer $m>0$, we denote by $g_m^x$ the germ at the point $x$ of the canonical image $g_m$ of $g$ in $\Gamma(X,\pi_*(\sh{O}_V/\sh{J}^{m+1}))$, then there exists an integer $m>0$ such that the image of $g_m^x$ in
        \[
            (\pi_*(\sh{O}_V/\sh{J}^{m+1}))_x \otimes_{\sh{O}_x} \kres(x)
        \]
        is $\neq0$.
    \item[\rm{(ii)}] Suppose further that the conditions of \sref{II.8.10.1}[iii] are fulfilled.
        Then, if there exists a section $g$ of $\sh{K}_V$ over $V$ such that $g(z)\neq0$, then there exists an integer $n\geq0$ and a section $f$ of $\sh{K}.\sh{L}^{\otimes n}=\sh{K}\otimes\sh{L}^{\otimes n}\subset\sh{L}^{\otimes n}$ such that $f(x)\neq0$.
        If $g$ is a section of $\sh{J}$, we can take $n>0$.
\end{enumerate}
\end{corollary}

\begin{proof}
\medskip\noindent
\begin{enumerate}
    \item[\rm{(i)}] Since the ideal of $\sh{O}_W$ generated by $g|W$ is contained in $\sh{K}_W$ by hypothesis, the ideal of $\sh{O}_V$ generated by $g$ is contained in $\sh{K}_V$ by \sref{II.8.10.4}, or, in other words, $g$ is a section of $\sh{K}_V$.
        To prove the second claim of (i), we can again assume that $X$ and $V$ are affine, and reuse the notation from \sref{II.8.10.1};
        the fibre $\pi^{-1}(x)$ is then affine of ring $B'=B\otimes_A\kres(x)$, and there exists in $B'$ an element $t'$ which is not a zero divisor and is such that $B'=\kres(x)\oplus B't'$.
        Since $j(x)$ is a specialisation of $z$ and since $g(z)\neq0$, we necessarily have that $g_{(j)x}\neq0$.
        But $\sh{O}_{j(x)}$ is a separated local ring \sref{II.8.10.3}, and thus embeds into its completion, and the image of $g$ in this completion is thus not null.
        But this completion is isomorphic to $\varprojlim_n(B'/B't^{'n+1})$ \sref{II.8.10.3};
        if $g'=g\otimes1\in B'$, there then exists an integer $m$ such that $g'\not\in B't^{'m+1}$, or, again, the image $g'_m$ of $g'$ in $B'/B't^{'m+1}$ is not null.
        But since $g'_m$ is exactly the image of $g_m^x$, our claim is proved.
    \item[\rm{(ii)}] By \sref{II.8.10.1}[iii], $\pi_*(\sh{O}_V/\sh{J}^{m+1})$ is isomorphic to the direct sum of the $\sh{L}^{\otimes k}$ for $0\leq k\leq m$;
        we denote by $f_k$ the section of $\sh{L}^{\otimes k}$ over $X$ that is the component of the element of $\bigoplus_{k=0}^m\Gamma(X,\sh{L}^{\otimes k})$ which corresponds to $g_m$ by this isomorphism.
        Choosing $m$ as in (i), there is thus an index $k$ such that $f_k(x)\neq0$, by (i).
        To see that $f_k$ is a section of $\sh{K}\sh{L}^{\otimes k}$, it suffices to consider, as above, the case where $X$ and $V$ are affine, and this follows immediately from the fact that $g\in\mathfrak{K}.B$ (with the notation from \sref{II.8.10.4}).
        The final claim follows from the fact that the hypothesis $g\in\Gamma(V,\sh{J})$ implies that $f_0=0$.
\end{enumerate}
\end{proof}

\begin{env}[8.10.6]
\label{II.8.10.6}
\emph{Proof of \sref{II.8.9.4}.}
The question is local on $Y$ \sref{II.4.6.4};
since $\varepsilon$ is a $Y$-section, we can thus replace $C$ by an affine open neighbourhood $U$ of a point of $\varepsilon(Y)$ such that $\varepsilon(Y)\cap U$ is closed in $U$.
In other words, we can assume that $C$ is affine, and that $Y$ is a closed subprescheme of $C$ (and thus also affine) defined by a quasi-coherent sheaf
\oldpage[II]{188}
$\sh{I}$ of ideals of $\sh{O}_C$.
Since $p$ is separated and quasi-compact, $X$ is thus a quasi-compact scheme, and we are reduced to proving that $\sh{L}$ is \emph{ample} \sref{II.4.6.4}.
By criterion~\sref{II.4.5.2}[a)], we must thus prove the following:
for every quasi-coherent ideal $\sh{K}$ of $\sh{O}_X$ and every point $x\in X$ not belonging to the support of $\sh{O}_X/\sh{K}$, there exists an integer $n>0$ and a section $f$ of $\sh{K}\otimes\sh{L}^{\otimes n}$ over $X$ such that $f(x)\neq0$.

For this, set
\begin{align*}
  \sh{K}_V &= \pi^*(\sh{K})\sh{O}_V\\
  \sh{K}_W &= \sh{K}_V|W
\end{align*}
where $W=V\setmin j(X)$;
since the restriction of $q$ to $W$ is a quasi-compact immersion to $C$, it follows from \sref[I]{I.9.4.2} that $\sh{K}_W$ is the restriction to $W$ of a quasi-coherent ideal $\sh{K}'_V$ of $\sh{O}_V$ of the form
\[
    \sh{K}'_V = q^*(\sh{K}_C)\sh{O}_V
\]
where $\sh{K}_C$ is a quasi-coherent ideal of $\sh{O}_C$.
Furthermore, since, by hypotheses, $q^{-1}(Y)\subset j(X)$, and since $Y$ is defined by the ideal $\sh{I}$, the restriction to $W$ of $q^*(\sh{I})\sh{O}_V$ is identical to that of $\sh{O}_V$, and so $\sh{K}_W$ is also the restriction to $W$ of $q^*(\sh{I}\sh{K}_C)\sh{O}_V$, and we can thus suppose that $\sh{K}_C\subset\sh{I}$, whence
\[
\label{II.8.10.6.1}
    \sh{K}'_V \subset q^*(\sh{I})\sh{O}_V \subset \sh{J}
\tag{8.10.6.1}
\]
taking into account \sref[I]{I.4.4.6} and the commutativity of \sref{II.8.9.4.1}.
Furthermore, we deduce from \sref{II.8.10.4} that
\[
\label{II.8.10.6.2}
    \sh{K}'_V \subset \sh{K}_V.
\tag{8.10.6.2}
\]
With this in mind, it follows from \sref{II.8.10.3} that $j(x)$ belongs to exactly one irreducible component of $\pi^{-1}(x)$;
let $z$ be the generic point of this component, and let $z'=q(z)$.
By \sref{II.8.10.5}, the proof will be finished (taking \sref{II.8.10.6.1} and \sref{II.8.10.6.2} into account) if we show the existence of a section $g$ of $\sh{K}'_V$ over $V$ such that $g(z)\neq0$.
But, by hypothesis, $\sh{K}$ has a restriction equal to that of $\sh{O}_X$ in an open neighbourhood of $x$;
also, it follows from \sref{II.8.10.3} that $z\neq j(x)$, and so $z\in W$, and thus $(\sh{K}_W)_z=\sh{O}_{V,z}$, whence, by definition, $(\sh{K}_C)_{z'}=\sh{O}_{C,z}$.
Since $C$ is affine, there is thus a section $g'$ of $\sh{K}_C$ over $C$ such that $g'(z')\neq0$, and by taking $g$ to be the section of $\sh{K}'_V$ corresponding canonically to $g'$, we indeed have $g(z)\neq0$, which finishes the proof.
\end{env}

\begin{remark}[8.10.7]
\label{II.8.10.7}
We ignore the question of whether or not condition (ii) in \sref{II.8.9.4} is superfluous or not.
In any case, the conclusion does not hold if we do not assume the existence of a $Y$-morphism $\pi:V\to X$ such that $\pi\circ j=1_X$;
we briefly point out how we can indeed construct a counterexample, whose details will not be developed until later on.
We take $Y=\Spec(k)$, where $k$ is a field, and $C=\Spec(A)$, where $A=k[T_1,T_2]$, and the $Y$-section $\varepsilon$ corresponding to the augmentation homomorphism $A\to k$.
We denote by $C'$ the scheme induced by $C$ by blowing up the closed point $a=\varepsilon(Y)$ of $C$;
if $D$ is the inverse image of $a$ in $C'$, we consider in $D$ a closed point $b$,
\oldpage[II]{189}
and we denote by $V$ the scheme induced by $C'$ by blowing up $b$;
$X$ is the closed subprescheme of $V$ given by the inverse image of $a$ by the structure morphism $q:V\to C$.
We now show that $X$ is the union of two irreducible components, $X_1$ and $X_2$, where $X_1$ is the inverse image of $b$ in $V$.
It is immediate that the ideal $\sh{J}$ of $\sh{O}_V$ that defines $X$ is again invertible, we we can show that $j^*(\sh{J})=\sh{L}$ (where $j$ is the canonical injection $X\to V$) is not ample, by considering the ``degree'' of the inverse image of $\sh{L}$ in $X_1$, which would be $>0$ if $\sh{L}$ were ample, but we can show (by an elementary intersection calculation) that it is in fact equal to $0$.
\end{remark}


\subsection{Uniqueness of contractions}
\label{subsection:II.8.11}

\begin{lemma}[8.11.1]
\label{II.8.11.1}
Let $U$ and $V$ be preschemes, and $h=(h_0,\lambda):U\to V$ a surjective morphism.
Suppose that
\begin{enumerate}
    \item $\lambda:\sh{O}_V\to h_*(\sh{O}_U) = (h_0)_*(\sh{O}_U)$ is an isomorphism;
    \item the underlying space of $V$ can be identified with the quotient of the underlying space of $U$ by the relation $h_0(x)=h_0(y)$ (\emph{a condition which always holds whenever the morphism $h$ is \emph{open} or \emph{closed}, or, \emph{a fortiori} when $h$ is \emph{proper}.})
\end{enumerate}
Then, for every prescheme $W$, the map
\[
\label{II.8.11.1.1}
    \Hom(V,W) \to \Hom(U,W)
\tag{8.11.1.1}
\]
that, to each morphism $v=(v_0,\nu)$ from $V$ to $W$, associates the morphism $u=v\circ h=(u_0,\mu)$, is a bijection from $\Hom(V,W)$ to the set of $u$ such that $u_0$ is constant on every fibre $h_0^{-1}(x)$.
\end{lemma}

\begin{proof}
It is clear that, if $u=v\circ h$, so that $u_0=v_0\circ h_0$, then $u_0$ is constant on every set $h_0^{-1}(x)$.
Conversely, if $u$ has this property, we will show that there exists exactly one $v\in\Hom(V,W)$ such that $u=v\circ h$.
The existence and uniqueness of the continuous map $v_0:V\to W$ such that $u_0=v_0\circ h_0$ follows from the hypotheses, since $h_0$ can be identified with the canonical map from $U$ to $U/R$.
We can also, replacing $V$ by some isomorphic prescheme if necessary, suppose that $\lambda$ is the identity;
by hypothesis, $\mu$ is then a homomorphism $\mu:\sh{O}_W\to(u_0)_*(\sh{O}_U) = (v_0)_*((h_0)_*(\sh{O}_U))$ such that the corresponding homomorphism $\mu^\sharp:u_0^*(\sh{O}_W)\to\sh{O}_U$ is \emph{local} on every fibre.
Since $(v_0)_*((h_0)_*(\sh{O}_U))=(v_0)_*(\sh{O}_V)$, we necessarily have that $\nu=u$, and everything then reduces to showing that the corresponding homomorphism $\nu^\sharp:v_0^*(\sh{O}_W)\to\sh{O}_V$ is local on every fibre.
But every $y\in V$ is of the form $h_0(x)$ for some $x\in U$;
let $z=v_0(y)=u_0(x)$.
Then \sref[0]{0.3.5.5} the homomorphism $\mu_x^\sharp$ factors as
\[
    \mu_x^\sharp: \sh{O}_z \xrightarrow{\nu_y^\sharp} \sh{O}_y \xrightarrow{\lambda_x^\sharp} \sh{O}_x.
\]
By hypothesis, $\lambda_x^\sharp$ and $\mu_x^\sharp$ are local homomorphisms;
thus $\lambda_x^\sharp$ sends every invertible element of $\sh{O}_y$ to an invertible element of $\sh{O}_x$;
if $\nu_y^\sharp$ sent a non-invertible element of $\sh{O}_z$ to an invertible element of $\sh{O}_y$, then $\mu_x^\sharp$ would send this element of $\sh{O}_z$ to an invertible element of $\sh{O}_x$, contradicting the hypothesis, whence the lemma.
\end{proof}

\begin{corollary}[8.11.2]
\label{II.8.11.2}
Let $U$ be an integral prescheme, and $V$ a normal prescheme;
then every morphism $h:U\to V$ that is universally closed, birational, and radicial, is also an isomorphism.
\end{corollary}

\begin{proof}
If $h=(h_0,\lambda)$, then it follows from the hypotheses that $h_0$ is injective and closed, and that $h_0(U)$
\oldpage[II]{190}
is dense in $V$, and so $h_0$ is a \emph{homeomorphism} from $U$ to $V$.
To prove the corollary, it will suffice to show that $\lambda:\sh{O}_V\to(h_0)_*(\sh{O}_U)$ is an isomorphism: we can then apply \sref{II.8.11.1}, which proves that the map \sref{II.8.11.1.1} is bijective (the fibres $h_0^{-1}(x)$ each consisting of a single point);
thus $h$ will be an isomorphism.
The question clearly being local on $V$, we can suppose that $V=\Spec(A)$ is affine, of an integral and integrally closed ring \sref{II.8.8.6.1};
$h$ then corresponds \sref[I]{I.2.2.4} to a homomorphism $\vphi:A\to\Gamma(U,\sh{O}_U)$, and everything reduces to showing that $\vphi$ is an isomorphism.
But, if $K$ is the field of fractions of $A$, then $\Gamma(U,\sh{O}_U)$ has, by hypothesis, $K$ as its field of fractions, and $A$ is a subring of $\Gamma(U,\sh{O}_U)$, with $\vphi$ being the canonical injection \sref[I]{I.8.2.7}.
Since the morphism $h$ satisfies the hypotheses of \sref{II.7.3.11}, $\Gamma(U,\sh{O}_U)$ is a subring of the integral closure of $A$ in $K$, and is thus identical to $A$ by hypothesis.
\end{proof}

\begin{remark}[8.11.3]
\label{II.8.11.3}
We will see in chapter~III \sref[III]{III.4.4.11} that, whenever $V$ is a \emph{locally Noetherian} prescheme, every morphism $h:U\to V$ that is proper and quasi-finite (in particular, every morphism satisfying the hypotheses of \sref{II.8.11.2}) is necessarily \emph{finite}.
The conclusion of \sref{II.8.11.2} then follows in this case from \sref{II.6.1.15}.
\end{remark}

\begin{env}[8.11.4]
\label{II.8.11.4}
We will now see that, in Grauert's criterion, we can often prove that the prescheme $C$ and the ``contraction'' $q$ are determined in an \emph{essentially unique} manner.
\end{env}

\begin{lemma}[8.11.5]
\label{II.8.11.5}
Let $Y$ be a prescheme, $p:X\to Y$ a proper morphism, $\sh{L}$ a $p$-ample invertible $\sh{O}_X$-module, $C$ a $Y$-prescheme, $\varepsilon:Y\to C$ a $Y$-section, and $q:V=\bb{V}(\sh{L})\to C$ a $Y$-morphism, all such that the diagram in \sref{II.8.9.1.1} commutes.
Suppose further that, if $p=(p_0,\theta)$, then $\theta:\sh{O}_Y\to p_*(\sh{O}_X)$ is an isomorphism.
Let $\sh{S}'=\bigoplus_{n\geq0}p_*(\sh{L}^{\otimes n})$ and $C'=\Spec(\sh{S}')$, and let $q':\bb{V}(\sh{L})\to C'$ be the canonical $Y$-morphism \sref{II.8.8.5}.
Then there exists exactly one $Y$-morphism $u:C'\to C$ such that $q=u\circ q'$.
\end{lemma}

\begin{proof}
The hypothesis on $\theta$ implies, in particular, that $p$ is surjective;
since, by \sref{II.8.8.4}, the restriction of $q'$ to $\bb{V}(\sh{L})\setmin j(X)$ is an \emph{isomorphism} to $C'\setmin\varepsilon'(Y)$ (where $\varepsilon$ is the vertex section of $C'$), it follows from \sref{II.8.8.4} that $q'$ is \emph{proper} and \emph{surjective};
furthermore, by \sref{II.8.8.6}, if we let $q'=(q'_0,\tau)$, then $\tau:\sh{O}_{C'}\to q'_*(\sh{O}_V)$ is an isomorphism.
We are thus in a situation where we can apply \sref{II.8.11.1}, and we will have proven the lemma if we show that $q$ is constant on every fibre $q^{'-1}(z')$, where $z'\in C'$.
But this condition is trivially satisfied for $z'\not\in\varepsilon'(Y)$.
If $z'\in\varepsilon'(Y)$, then there exists exactly one $y\in Y$ such that $z'=\varepsilon'(y)$, and, by commutativity of \sref{II.8.8.5.2} and the fact that $q'$ sends $\bb{V}(\sh{L})\setmin j(X)$ to $C'\setmin\varepsilon'(Y)$, $q^{'-1}(z')=j(p^{-1}(y))$;
the commutativity of the diagram in \sref{II.8.9.1.1} then proves our claim.
\end{proof}

\begin{corollary}[8.11.6]
\label{II.8.11.6}
Under the hypotheses of \sref{II.8.11.5}, suppose further that $q$ is proper, and that the restriction of $q$ to $\bb{V}(\sh{L})\setmin j(X)$ is an isomorphism to $C\setmin\varepsilon(Y)$.
Then the morphism $u$ is universally closed, surjective, and radicial, and its restriction to $C'\setmin\varepsilon'(Y)$ is an isomorphism to $C\setmin\varepsilon(Y)$.
\end{corollary}

\begin{proof}
Since $q'$ is an isomorphism from $\bb{V}(\sh{L})\setmin j(X)$ to $C'\setmin\varepsilon'(Y)$ \sref{II.8.8.4}, the last claim follows immediately from the fact that $q=u\circ q'$.
Furthermore, the commutativity of the diagrams
\oldpage[II]{191}
in \sref{II.8.8.5.2} and \sref{II.8.9.1.1} shows that the restriction of $u$ to the closed subprescheme $\varepsilon'(Y)$ of $C'$ is an isomorphism to the closed subprescheme $\varepsilon(Y)$ of $C$, from which we immediately deduce that, for all $z'\in\varepsilon'(Y)$, if $z=u(z')$, then $u$ defines an isomorphism from $\kres(z)$ to $\kres(z')$.
These remarks prove that $u$ is bijective and radicial;
furthermore, if $\psi:C\to Y$ and $\psi':C'\to Y$ are the structure morphisms, then $\psi'=\psi\circ u$, and, since $\psi'$ is separated \sref{II.1.2.4}, so too is $u$ \sref[I]{I.5.5.1}[v].
We have already seen, in the proof of \sref{II.8.11.5}, that $q'$ is surjective;
since $q=u\circ q'$ is proper, we finally conclude, from \sref{II.5.4.3} and \sref{II.5.4.9}, that $u$ is universally closed.
\end{proof}

\begin{proposition}[8.11.7]
\label{II.8.11.7}
Let $Y$ be a prescheme, $X$ an \emph{integral} prescheme, $p:X\to Y$ a proper morphism, $\sh{L}$ a $p$-ample invertible $\sh{O}_X$-module, $C$ a \emph{normal} $Y$-prescheme, $\varepsilon:Y\to C$ a $Y$-section, and $q:V=\bb{V}(\sh{L})\to C$ a $Y$ morphism, all such that the diagram in \sref{II.8.9.1.1} commutes.
Suppose further that, if $p=(p_0,\theta)$, then $\theta:\sh{O}_Y\to p_*(\sh{O}_X)$ is an isomorphism.
Let $\sh{S}'=\bigoplus_{n\geq0}p_*(\sh{L}^{\otimes n})$ and $C'=\Spec(\sh{S}')$, and let $q':\bb{V}(\sh{L})\to C'$ be the canonical $Y$-morphism \sref{II.8.8.5}.
Then the unique $Y$-morphism $u:C'\to C$ such that $q=u\circ q'$ is an \emph{isomorphism}.
\end{proposition}

\begin{proof}
It follows from \sref{II.8.8.6} that $C'$ is integral;
since $u$ is a homeomorphism of the underlying subspaces $C'\to C$ ($u$ being bijective and closed, by \sref{II.8.11.6}), $C$ is irreducible, thus integral, and, since the restriction of $u$ to a non-empty open subset of $C'$ is an isomorphism to an open subset of $C$, $u$ is birational.
Since $C$ is assumed to be normal, it suffices to apply \sref{II.8.11.2} to obtain the conclusion.
\end{proof}

\begin{remark}[8.11.8]
\label{II.8.11.8}
\begin{itemize}
    \item[\rm{(i)}] The hypothesis that $C$ is normal implies that $X$ is also normal.
        Indeed, $C'=\Spec(\sh{S}')$ is then normal, being isomorphic to $C$, and integral, by \sref{II.8.8.6};
        we thus conclude that $\Proj(\sh{S}')$ is \emph{normal}.
        Indeed, the question is local on $Y$;
        if $Y$ is affine, with $\sh{S}'=\widetilde{S'}$, then the ring $S'=\Gamma(C',\sh{S}')$ is integral and integrally closed \sref{II.8.8.6.1}, and so, for every homogeneous element $f\in S'_+$, the graded ring $S'_f$ is integral and integrally closed \cite[t.~I, p.~257 and 261]{I-13}, and thus so too is the ring $S'_{(f)}$ of its degree-zero terms, because the intersection of $S'_f$ with the field of fractions of $S'_{(f)}$ is equal to $S'_{(f)}$;
        this proves our claim \sref{II.6.3.4}.
        Finally, since $X$ is isomorphic to an open subprescheme of $\Proj(\sh{S}')$ \sref{II.8.8.1}, $X$ is indeed normal.
        We can thus express \sref{II.8.11.7} in the following form:
        \emph{If $X$ is integral and normal, and $p=(p_0,\theta):X\to Y$ is a proper morphism such that $\theta:\sh{O}_Y\to p_*(\sh{O}_X)$ is an isomorphism, then, for every $p$-ample $\sh{O}_X$-module $\sh{L}$, there exists exactly one way of contracting the null section of $V=\bb{V}(\sh{L})$ to obtain a normal $Y$-scheme $C$ and a proper $Y$-morphism $q:V\to C$.}
    \item[\rm{(ii)}] When $p$ is proper, the hypothesis $p_*(\sh{O}_X)=\sh{O}_Y$ can be considered as an auxiliary hypothesis, not really restricting the generality of the result.
        Indeed, if it is not satisfied, then it suffices to replace $Y$ with the $Y$-scheme $Y'=\Spec(p_*(\sh{O}_X))$, and to consider $X$ as a $Y'$-scheme.
        We will return to this general method in chapter~III, \textsection~4.
\end{itemize}
\end{remark}


\subsection{Quasi-coherent sheaves on based cones}
\label{subsection:II.8.12}

\begin{env}[8.12.1]
\label{II.8.12.1}
Let us use the hypotheses and notation of \sref{II.8.3.1}.
Let $\sh{M}$ be a \emph{quasi-coherent graded $\sh{S}$-module}; to avoid any confusion, we denote by $\widetilde{\sh{M}}$ the quasi-coherent $\sh{O}_C$-module
\oldpage[II]{192}
associated to $\sh{M}$ \sref{II.1.4.3} when $\sh{M}$ is considered as a \emph{non-graded} $\sh{S}$-module, and by $\shProj_0(\sh{M})$ the quasi-coherent $\sh{O}_X$-module associated to $\sh{M}$, $\sh{M}$ being considered this time as a graded $\sh{S}$-module (in other words, the $\sh{O}_X$-module denoted by $\widetilde{\sh{M}}$ in \sref{II.3.2.2}).
In addition, we set
\[
\label{II.8.12.1.1}
  \sh{M}_X=\shProj_0(\sh{M})=\bigoplus_{n\in\bb{Z}}\shProj_0(\sh{M}(n));
\tag{8.12.1.1}
\]
the quasi-coherent graded $\sh{O}_X$-algebra $\sh{S}_X$ being defined by \sref{II.8.6.1.1}, $\shProj(\sh{M})$ is equipped with a structure of a \emph{(quasi-coherent) graded $\sh{S}_X$-module}, by means of the canonical homomorphisms \sref{II.3.2.6.1}
\[
\label{II.8.12.1.2}
  \sh{O}_X(m)\otimes_{\sh{O}_X}\shProj_0(\sh{M}(n))\to\shProj_0(\sh{S}(m)\otimes_\sh{S}\sh{M}(n))\to\shProj_0(\sh{M}(m+n)),
\tag{8.12.1.2}
\]
the verification of the axioms of sheaves of modules being done using the commutative diagram in \sref{II.2.5.11.4}.

If $Y=\Spec(A)$ is affine, $\sh{S}=\widetilde{S}$, and $\sh{M}=\widetilde{M}$, where $S$ is a graded $A$-algebra and $M$ is a graded $S$-module, then, for every homogeneous element $f\in S_+$, we have
\[
\label{II.8.12.1.3}
  \Gamma(X_f,\shProj(\widetilde{M}))=M_f
\tag{8.12.1.3}
\]
by the definitions and \sref{II.8.2.9.1}.

Now consider the quasi-coherent graded $\widehat{\sh{S}}$-module
\[
\label{II.8.12.1.4}
  \widehat{\sh{M}}=\sh{M}\otimes_\sh{S}\widehat{\sh{S}}
\tag{8.12.1.4}
\]
($\widehat{\sh{S}}$ being defined by \sref{II.8.3.1.1}); this induces a quasi-coherent graded $\sh{O}_{\widehat{C}}$-module $\shProj_0(\widehat{\sh{M}})$, which we will also denote by
\[
\label{II.8.12.1.5}
  \sh{M}^\square=\shProj_0(\widehat{\sh{M}}).
\tag{8.12.1.5}
\]

It is clear \sref{II.3.2.4} that $\sh{M}^\square$ is an additive functor which is \emph{exact} in $\sh{M}$, commuting with direct sums and with inductive limits.
\end{env}

\begin{proposition}[8.12.2]
\label{II.8.12.2}
With the notation of \sref{II.8.3.2}, we have canonical functorial isomorphisms
\[
\label{II.8.12.2.1}
    i^*(\sh{M}^\square)\simto\widetilde{\sh{M}},
    \quad
    j^*(\sh{M}^\square)\simto\shProj_0(\sh{M}).
\tag{8.12.2.1}
\]
Indeed, $i^*(\sh{M}^\square)$ is canonically identified with $(\widehat{\sh{M}}/(\bb{z}-1)\widehat{\sh{M}})\supertilde$ on $\Spec(\widehat{\sh{S}}/(\bb{z}-1)\widehat{\sh{S}})$ by \sref{II.3.2.3};
the first of the canonical isomorphisms \sref{II.8.12.2.1} is then immediately induced \sref{II.1.4.1} by the canonical isomorphism $\widehat{\sh{M}}/(\bb{z}-1)\widehat{\sh{M}}\simto\sh{M}$.
The canonical immersion $j:X\to C$ corresponds to the canonical homomorphism $\widehat{\sh{S}}\to\sh{S}$ with kernel $\bb{z}\widehat{\sh{S}}$ \sref{II.8.3.2};
the second homomorphism \sref{II.8.12.2.1} is the particular case of the canonical homomorphism \sref{II.3.5.2}[ii], since here we have $\widehat{\sh{M}}\otimes_{\widehat{\sh{S}}}\sh{S}=\sh{M}$;
to verify that this is an isomorphism, we can restrict to the case where $Y=\Spec(A)$ is affine, $\sh{S}=\widetilde{S}$, and $\sh{M}=\widetilde{M}$;
by appealing to \sref{II.2.8.8}, the proof that, for all homogeneous $f$ in $S_+$, the preceding homomorphism, restricted to $X_f$, restricts to an isomorphism, is then immediate.
\end{proposition}

\oldpage[II]{193}

By an abuse of language, we again say, thanks to the existence of the first isomorphism \sref{II.8.12.2.1}, that $\sh{M}^\square$ is the \emph{projective closure} of the $\sh{O}_X$-module $\widetilde{\sh{M}}$ (it being implicit that the data of the $\sh{O}_C$-module $\widetilde{\sh{M}}$ includes the grading of the $\sh{S}$-module $\sh{M}$).

\begin{env}[8.12.3]
\label{II.8.12.3}
With the notation of \sref{II.8.3.5}, we have a canonical functorial homomorphism
\[
\label{II.8.12.3.1}
    p^*(\shProj(\sh{M}))\to\sh{M}^\square|\widehat{E}.
\tag{8.12.3.1}
\]

Indeed, this is a particular case of the homomorphism $\nu^\sharp$ defined more generally in \sref{II.3.5.6}.
If $Y=\Spec(A)$ is affine, $\sh{S}=\widetilde{S}$, and $\sh{M}=\widetilde{M}$, then, by appealing to \sref{II.2.8.8}, the restriction of \sref{II.8.12.3.1} to $p^{-1}(X_f)=\widehat{C}_f$ (for some homogeneous $f$ in $S_+$) corresponds to the canonical homomorphism
\[
\label{II.8.12.3.2}
    M_{(f)}\otimes_{S_{(f)}}S_f^\leq \to M_f^\leq
\tag{8.12.3.2}
\]
taking into account \sref{II.8.2.3.2} and \sref{II.8.2.5.2}.
\end{env}

\begin{env}[8.12.4]
\label{II.8.12.4}
Let us place ourselves in the settings of \sref{II.8.5.1}, and assume its hypotheses and keep its notation.
It follows from \sref{II.1.5.6} that, for every quasi-coherent graded $\sh{S}$-module $\sh{S}$, we have, on one hand, a canonical isomorphism
\[
\label{II.8.12.4.1}
    \Phi^*(\widetilde{\sh{M}}) \simto (q^*(\sh{M})\otimes_{q^*(\sh{S})}\sh{S}')\supertilde
\tag{8.12.4.1}
\]
of $\sh{O}_{C'}$-modules;
on the other hand, \sref{II.3.5.6} implies the existence of a canonical $\Proj(\vphi)$-morphism
\[
\label{II.8.12.4.2}
    \shProj_0\sh{M} \to (\shProj_0(q^*(\sh{M}))\otimes_{q^*(\sh{S})}\sh{S}')|G(\vphi)
\tag{8.12.4.2}
\]
and also of a canonical $\widehat{\Phi}$-morphism
\[
\label{II.8.12.4.3}
    \shProj_0\widehat{\sh{M}} \to (\shProj_0(q^*(\widehat{\sh{M}}))\otimes_{q^*(\widehat{\sh{S}})}\widehat{\sh{S}}')|G(\widehat{\vphi}).
\tag{8.12.4.3}
\]
\end{env}

\begin{env}[8.12.5]
\label{II.8.12.5}
Consider now the setting of \sref{II.8.6.1}, with the same notation;
we thus take $Y'=X$, the morphism $q:X\to Y$ being the structure morphism, and $\vphi$ the canonical $q$-morphism \sref{II.8.6.1.2}.
We then have a canonical isomorphism
\[
\label{II.8.12.5.1}
    q^*(\sh{M})\otimes_{q^*(\sh{S})}\sh{S}_X^\geq \simto \sh{M}_X^\geq
\tag{8.12.5.1}
\]
by setting $\sh{M}_X^\geq=\bigoplus_{n\geq0}\shProj_0(\sh{M}(n))$.
We can indeed restrict to the case where $Y=\Spec(A)$ is affine, $\sh{S}=\widetilde{S}$, and $\sh{M}=\widetilde{M}$, and define the isomorphism \sref{II.8.12.5.1} on each of the affine open subsets $X_f$ (where $f$ is homogeneous in $S_+$), by verifying the compatibility with taking a homogeneous multiple of $f$.
But the restriction to $X_f$ of the left-hand side of \sref{II.8.12.5.1} is $\widetilde{M}'=((M\otimes_A S_{(f)})\otimes_{S\otimes_A S_{(f)}}S_f^\geq)\supertilde$ by \sref{II.8.6.2.1};
since we have a canonical isomorphism from $M\otimes_A S_{(f)}$ to $M\otimes_S(S\otimes_A S_{(f)})$, we have an induced isomorphism from $\widetilde{M}'$ to $(M\otimes_S S_f^\geq)\supertilde$, and the latter is canonically isomorphic, by \sref{II.8.2.9.1}, to the restriction to $X_f$ of the right-hand side of \sref{II.8.12.5.1}, and satisfies the required compatibility conditions.
\oldpage[II]{194}

Replacing $\sh{M}$ by $\widehat{\sh{M}}$, $\sh{S}$ by $\widehat{\sh{S}}$, and $\sh{S}_X$ by $(\sh{S}_X^\geq)^\wedge$ in the previous argument, we similarly have a canonical isomorphism
\[
\label{II.8.12.5.2}
    q^*(\widehat{\sh{M}})\otimes_{q^*(\widehat{\sh{S}})}(\sh{S}_X^\geq)^\wedge \simto (\sh{M}_X^\geq)^\wedge.
\tag{8.12.5.2}
\]

If we recall \sref{II.8.6.2} that the structure morphism $u:\Proj(\sh{S}_X^\geq)\to X$ is an isomorphism, then we deduce, first of all, from the above, that we have a canonical $u$-isomorphism
\[
\label{II.8.12.5.3}
    \shProj_0\sh{M} \simto \shProj_0(\sh{M}_X^\geq)
\tag{8.12.5.3}
\]
as a particular case of \sref{II.8.12.4.2}.
We note that, with the notation from the proof of \sref{II.8.6.2}, this reduces to seeing that the canonical homomorphism $M_{(f)}\otimes_{S_{(f)}}(S_f^\geq)^{(d)}\to(M_f^\geq)^{(d)}$ is an isomorphism whenever $f\in S_d$, which is immediate.

Secondly, the isomorphism \sref{II.8.12.5.2} gives us, by this time applying \sref{II.8.12.4.3} to the canonical morphism $r=\Proj(\widehat{\alpha}):\widehat{C}_X\to\widehat{C}$, a canonical $r$-morphism
\[
\label{II.8.12.5.4}
    \sh{M}^\square \to (\sh{M}_X^\geq)^\square.
\tag{8.12.5.4}
\]

Recall now \sref{II.8.6.2} that the restrictions of $r$ to the pointed cones $\widehat{E}_X$ and $E_X$ are \emph{isomorphisms} to $\widehat{E}$ and $E$ (respectively).
Furthermore:
\end{env}

\begin{proposition}[8.12.6]
\label{II.8.12.6}
The restrictions to $\widehat{E}_X$ and $E_X$ of the canonical $r$-morphism \sref{II.8.12.5.4} are isomorphisms
\[
\label{II.8.12.6.1}
    \sh{M}^\square|\widehat{E} \simto (\sh{M}_X^\geq)^\square|\widehat{E}_X
\tag{8.12.6.1}
\]
\[
\label{II.8.12.6.2}
    \sh{M}^\sim|\widehat{E} \simto (\sh{M}_X^\geq)\supertilde|\widehat{E}_X.
\tag{8.12.6.2}
\]
\end{proposition}

\begin{proof}
We restrict to the case where $Y$ is affine, as in the proof of \sref{II.8.6.2} (whose notation we adopt);
by reducing to definitions \sref{II.2.8.8}, we have to show that the canonical homomorphism
\[
    \widehat{M}_{(f)}\otimes_{\widehat{S}_{(f)}}(S_f^\geq)_{(f/1)}^\wedge \to (M\otimes_S S_f^\geq)_{(f/1)}^\wedge
\]
is an isomorphism;
but, by \sref{II.8.2.3.2} and \sref{II.8.2.5.2}, the left-hand side is canonically identified with $M_f^\leq\otimes_{S_f^\leq}(S_f^\geq)_{f/1}^\leq$, and thus with $M_f^\leq$, by \sref{II.8.2.7.2}, and the right-hand side with $(M_f^\geq)_{f/1}^\leq$, and thus also with $M_f^\leq$, by \sref{II.8.2.9.2}, whence the conclusion concerning \sref{II.8.12.6.1};
\sref{II.8.12.6.2} then follows from \sref{II.8.12.6.1} and \sref{II.8.12.2.1}.
\end{proof}

\begin{corollary}[8.12.7]
\label{II.8.12.7}
With the identifications of \sref{II.8.6.3}, the restriction of $(\sh{M}_X^\geq)^\square$ to $\widehat{E}_X$ can be identified with $(\sh{M}_X^\leq)\supertilde$, and the restriction of $(\sh{M}_X^\geq)^\square$ to $E_x$ with $\widetilde{\sh{M}}_X$.
\end{corollary}

\begin{proof}
We can restrict to the affine case, and this follows from the identification of $(M_f^\geq)_{f/1}^\leq$ with $M_f^\leq$, and of $(M_f^\geq)_{f/1}$ with $M_f$ \sref{II.8.2.9.2}.
\end{proof}

\begin{proposition}[8.12.8]
\label{II.8.12.8}
Under the hypotheses of \sref{II.8.6.4}, the canonical homomorphism \sref{II.8.12.3.1} is an isomorphism.
\end{proposition}

\begin{proof}
Taking into account the fact that $\Proj(\sh{S}_X^\geq)\to X$ is an isomorphism \sref{II.8.6.2}, and the
\oldpage[II]{195}
isomorphisms \sref{II.8.12.5.4} and \sref{II.8.12.6.1}, we are led to proving the corresponding proposition for the canonical homomorphism $p_X^*(\shProj_0(\sh{M}_X^\geq))\to(\sh{M}_X^\geq)^\square|E_X$, or, in other words, we can restrict to the case where $\sh{S}_1$ is an invertible $\sh{O}_Y$-module, and where $\sh{S}$ is generated by $\sh{S}_1$.
With the notation of \sref{II.8.12.3}, we then have, for some $f\in S_1$, that $S_f^\leq=S_{(f)}[1/f]$, and the canonical homomorphism $M_{(f)}\otimes_{S_{(f)}}S_f^\leq\to M_f^\leq$ is an isomorphism, by the definition of $M_f^\leq$.
\end{proof}

\begin{env}[8.12.9]
\label{II.8.12.9}
Consider now the quasi-coherent $\sh{S}$-modules
\[
    \sh{M}_{[n]}=\bigoplus_{m\geq n}\sh{M}_m
\]
and (with the notation of \sref{II.8.7.2}) the quasi-coherent graded $\sh{S}^\natural$-module
\[
\label{II.8.12.9.1}
    \sh{M}^\natural=\left(\bigoplus_{n\geq0}\sh{M}_{[n]}\right)\supertilde.
\tag{8.12.9.1}
\]

We have seen \sref{II.8.7.3} that there exists a canonical $C$-isomorphism $h:C_X\simto\Proj(\sh{S}^\natural)$.
Furthermore:
\end{env}

\begin{proposition}[8.12.10]
\label{II.8.12.10}
There exists a canonical $h$-isomorphism
\[
\label{II.8.12.10.1}
    \shProj_0(\sh{M}^\natural) \simto \widetilde{\sh{M}}_X.
\tag{8.12.10.1}
\]
\end{proposition}

\begin{proof}
We argue as in \sref{II.8.7.3}, this time using the existence of the di-isomorphism \sref{II.8.2.9.3} instead of \sref{II.8.2.7.3}.
We leave the details to the reader.
\end{proof}


\subsection{Projective closures of subsheaves and closed subschemes}
\label{subsection:II.8.13}

\begin{env}[8.13.1]
\label{II.8.13.1}
With hypotheses and notation as in \sref{II.8.12.1}, consider a \emph{not-necessarily graded} quasi-coherent sub-$\sh{S}$-module $\sh{N}$ of $\sh{M}$.
We can then consider the quasi-coherent $\sh{O}_C$-module $\widetilde{\sh{N}}$ associated to $\sh{N}$, which is a sub-$\sh{O}_C$-module of $\widetilde{\sh{M}}$.
We have seen elsewhere \sref{II.8.12.2.1} that $\widetilde{\sh{M}}$ can be identified with the restriction of $\sh{M}^\square$ to $C$.
Since the canonical injection $i:C\to\widehat{C}$ is an affine morphism \sref{II.8.3.2}, and \emph{a fortiori} quasi-compact, the \emph{canonical extension} $(\widetilde{\sh{N}})^-$, the largest sub-$\sh{O}_{\widehat{C}}$-module contained in $\sh{M}^\square$ and inducing $\widetilde{\sh{N}}$ on $C$, is a \emph{quasi-coherent} $\sh{O}_{\widehat{C}}$-module \sref[I]{I.9.4.2}.
We will give a more explicit description by using a graded $\widehat{\sh{S}}$-module.
\end{env}

\begin{env}[8.13.2]
\label{II.8.13.2}
For this, consider, for every integer $n\geq0$, the homomorphism $\bigoplus_{i\leq n}\sh{M}_i\to\sh{M}$ which, for every open $U$ of $Y$, sends the family
\[
    (s_i) \in \bigoplus_{i\leq n}\Gamma(U,\sh{M}_i)
\]
to the section $\sum_i s_i\in\Gamma(U,\sh{M})$.
Denote by $\sh{N}'_n$ the inverse image of $\sh{N}$ by this homomorphism, which is a quasi-coherent sub-$\sh{S}$-module of $\bigoplus_{i\leq n}\sh{M}_i$.
Now consider the homomorphism $\bigoplus_{i\leq n}\sh{M}_i\to\widehat{\sh{M}}=\sh{M}[\bb{z}]$ which sends $(s_i)$ to the section $\sum_{i\leq n}s_i\bb{z}^{n-i}\in\Gamma(U,\widehat{\sh{M}}_n)$, and let $\sh{N}_n$ be the image of $\sh{N}'_n$ under this homomorphism;
we immediately have that $\overline{\sh{N}}=\bigoplus_{n\geq0}\sh{N}_n$ is a (quasi-coherent) sub-$\widehat{\sh{S}}$-module of $\widehat{\sh{M}}$;
we say that $\overline{\sh{N}}$ is induced from $\sh{N}$ by \emph{homogenisation}, via the ``homogenising variable'' $\bb{z}$.
We note
\oldpage[II]{196}
that, if $\sh{N}$ is already a \emph{graded} sub-$\sh{S}$-module of $\sh{M}$, then $\overline{\sh{N}}$ can be identified with the direct sum of the components $\widehat{\sh{N}}_n$ of degree $n\geq0$ in $\widehat{\sh{N}}=\sh{N}[\bb{z}]$.
\end{env}

\begin{proposition}[8.13.3]
\label{II.8.13.3}
The $\sh{O}_{\widehat{C}}$-module $\shProj_0(\overline{\sh{N}})$ is the canonical extension $(\widetilde{\sh{N}})^-$ of $\widetilde{\sh{N}}$ to $\widehat{C}$.
\end{proposition}

\begin{proof}
The question is local on $Y$ and $\widehat{C}$ by the definition of the canonical extension \sref[I]{I.9.4.1}.
We can thus already suppose that $Y=\Spec(A)$ is affine, with $\sh{S}=\widetilde{S}$, $\sh{M}=\widetilde{M}$, and $\sh{N}=\widetilde{N}$, where $N$ is a non-necessarily-graded sub-$S$-module of $M$.
Furthermore \sref{II.8.3.2.6}, $\widehat{C}$ is a union of affine opens $\widehat{C}_z=C$ and $\widehat{C}_f=\Spec(S_f^\leq)$ (with $f$ homogeneous in $S_+$).
It thus suffices to show that: (1) the restriction of $\shProj_0(\overline{\sh{N}})$ to $C$ is $\widetilde{\sh{N}}$; (2) the restriction of $\shProj_0(\overline{\sh{N}})$ to each $\widehat{C}_f$ is the canonical extension of the restriction of $\sh{N}$ to $C\cap\widehat{C}_f=\Spec(S_f)$ \sref{II.8.3.2.6}.
For the first point, note that $\shProj_0(\overline{\sh{N}})|C$ can be identified with $(\overline{N}_{(\bb{z})})\supertilde$ \sref{II.8.3.2.4};
but $\overline{N}_{(\bb{z})}$ is canonically identified \sref{II.2.2.5} with the image of $\overline{N}$ in $\widehat{M}/(\bb{z}-1)\widehat{M}$, and by the canonical isomorphism of the latter with $M$ \sref{II.8.2.5}, this image can be identified with $N$, by the definition of $\overline{N}$ given in \sref{II.8.13.2}.

To prove the second point, note that the injection $i:C\cap\widehat{C}_f\to\widehat{C}$ corresponds to the canonical injection $S_f^\leq\to S_f$ \sref{II.8.3.2.6};
we also have that $\Gamma(\widehat{C}_f,\sh{M}^\square)=M_f^\leq$, that $\Gamma(\widehat{C}_f,i_*(\widetilde{\sh{N}}))=N$, and, by \sref{II.8.12.2.1}, that $\Gamma(\widehat{C}_f,i_*(i^*(\sh{M}^\square)))=M_f$.
Taking \sref[I]{I.9.4.2} into account, we are thus led to showing that $\overline{N}_{(f)}\subset\widehat{M}_{(f)}=M_f^\leq$ is canonically identified with the inverse image of $N_f$ under the canonical injection $M_f^\leq\to M_f$.
Indeed, let $d=\deg(f)>0$, and suppose that an element $(\sum_{k\leq md}x_k)/f^m$ of $M_f$ (with $x_k\in M_k$) is of the form $y/f^m$ with $y\in N$.
By multiplying $y$ and the $x_k$ by one single suitable $f^h$, we can already assume that $\sum_{k\leq md}x_k=y$.
But in the identification of \sref{II.8.2.5.2}, $(\sum_{k\leq md}x_k)/f^m$ corresponds to $\sum_{k\leq md}x_k\bb{z}^{md-k}/f^m$, and this is indeed an element of $\overline{N}_{(f)}$, since $\sum_{k\leq md}x_k\in N$;
the converse is evident.
\end{proof}

\begin{remark}[8.13.4]
\label{II.8.13.4}
\begin{enumerate}
    \item[\rm{(i)}] The most important case of application of \sref{II.8.13.3} is that where $\sh{M}=\sh{S}$, with $\widetilde{\sh{N}}$ then being an \emph{arbitrary} quasi-coherent sheaf of ideals $\sh{J}$ of $\sh{O}_C$ \sref{II.1.4.3}, corresponding bijectively to a \emph{closed subprescheme} $Z$ of $C$.
        Then the canonical extension $\overline{\sh{J}}$ of $\sh{J}$ is the quasi-coherent sheaf of ideals of $\sh{O}_{\widehat{C}}$ that defines the \emph{closure} $\overline{Z}$ of $Z$ in $\widehat{C}$ \sref[I]{I.9.5.10};
        Proposition~\sref{II.8.13.3} gives a canonical way of defining $\overline{Z}$ by using a graded ideal in $\widehat{\sh{S}}=\sh{S}[\bb{z}]$.
    \item[\rm{(ii)}] Suppose, to simplify things, that $Y$ is affine, and adopt the notation from the proof of \sref{II.8.13.3}.
        For every non-zero $x\in N$, let $d(x)$ be the largest degree of the homogeneous components $x_i$ of $x$ in $M$;
        by definition, $\overline{N}$ is the submodule of $\widehat{M}$ consisting of $0$ and elements of the form $h(x,k)=\bb{z}^k\sum_{i\leq d(x)}x_i\bb{z}^{d(x)-i}$ (for integral $k\geq0$);
        it is thus generated, as a module over $\widehat{S}=S[\bb{z}]$, by the elements of the form
        \[
            h(x,0) = \sum_{i\leq d(x)}x_i\bb{z}^{d(x)-i}.
        \]
        \oldpage[II]{197}
        We say that $h(x,0)$ is induced from $x$ by \emph{homogenisation} via the ``homogenising variable'' $\bb{z}$.
        But since $h(x,0)$ does not depend additively on $x$ (nor \emph{a fortiori} $S$-linearly), \emph{we will refrain from believing} (even when $M=S$) that the $h(x,0)$ form a \emph{system of generators} of the graded $\widehat{S}$-module $\overline{N}$ when we let $x$ run over a \emph{system of generators} of the $S$-module $N$.
        This is, however, the case (considered only in elementary algebraic geometry) when $N$ is a \emph{free cyclic} $S$-module, since, if $t$ is a basis of $N$, then $h(t,0)$ generates the $\widehat{S}$-module $\overline{N}$.
\end{enumerate}
\end{remark}


\subsection{Supplement on sheaves associated to graded $\mathcal{S}$-modules}
\label{subsection:II.8.14}

\begin{env}[8.14.1]
\label{II.8.14.1}
Let $Y$ be a prescheme, $\sh{S}$ a \emph{positively-graded} quasi-coherent $\sh{O}_Y$-algebra, $X=\Proj(\sh{S})$, and $q:X\to Y$ the structure morphism (which is separated, by \sref{II.3.1.3}).
Using the notation of \sref{II.8.12.1}, we have defined a functor $\sh{M}_X=\shProj(\sh{M})$ in $\sh{M}$, from the category of quasi-coherent graded $\sh{S}$-modules to the category of quasi-coherent graded $\sh{S}_X$-modules;
it is further clear \sref{II.3.2.4} that this is an \emph{additive} and \emph{exact} functor, commuting with inductive limits.

Note, furthermore, that it follows immediately from the definition \sref{II.8.12.1.1} that we have
\[
\label{II.8.14.1.1}
    \shProj(\sh{M}(n)) = (\shProj(\sh{M}))(n)
    \quad\mbox{for all $n\in\bb{Z}$.}
\tag{8.14.1.1}
\]
\end{env}

\begin{env}[8.14.2]
\label{II.8.14.2}
We will first extend the canonical homomorphisms $\lambda$ and $\mu$, defined in \sref{II.3.2.6}, to $\sh{S}_X$-modules of the form $\shProj(\sh{M})$.
For this, note that, for any $m\in\bb{Z}$ and $n\in\bb{Z}$, we have, by \sref{II.2.1.2.1}, a canonical homomorphism of $\sh{O}_X$-modules
\[
\label{II.8.14.2.1}
    \lambda_{mn}:
    \shProj_0((\shHom_{\sh{S}}(\sh{M},\sh{N}))(n-m))
    \to
    \shHom_{\sh{O}_X}(\shProj_0(\sh{M}(m)),\shProj_0(\sh{N}(n)))
\tag{8.14.2.1}
\]
for any quasi-coherent graded $\sh{S}$-modules $\sh{M}$ and $\sh{N}$.
This induces a homomorphism
\[
    \mu_k:
    \shProj_0((\shHom_{\sh{S}}(\sh{M},\sh{N}))(k))
    \to
    (\shHom_{\sh{S}_X}(\shProj(\sh{M}),\shProj(\sh{N})))_k
\]
given by sending every $u\in\Gamma(U,\shProj_0((\shHom_{\sh{S}}(\sh{M},\sh{N}))(k)))$ to the homomorphism $\mu_k(u)$, of degree $k$, of graded $\bb{Z}$-modules $\Gamma(U,\shProj(\sh{M}))\to\Gamma(U,\shProj(\sh{N}))$ (where $U$ is open in $X$) which, in each $\Gamma(U,\shProj_0(\sh{M}(m)))$, agrees with $\mu_{m,m+k}(u)$;
furthermore, by returning to the definition of the $\mu_{mn}$ \sref{II.2.5.12.1}, we immediately see that $\mu_k(u)$ is in fact a homomorphism of degree $k$ of graded $\Gamma(U,\sh{S}_X)$-modules, and, furthermore, that the $\mu_k$ define a homomorphism of \emph{graded $\sh{S}_X$-modules}
\[
\label{II.8.14.2.3}
    \shProj(\shHom_{\sh{S}}(\sh{M},\sh{N}))
    \to
    \shHom_{\sh{S}_X}(\shProj(\sh{M}),\shProj(\sh{N})).
\tag{8.14.2.3}
\]

Similarly, taking the associativity diagram \sref{II.2.5.11.4} into account, the homomorphisms \sref{II.8.14.2.1} give a homomorphism of \emph{graded $\sh{S}_X$-modules}
\[
\label{II.8.14.2.4}
    \lambda:
    \shProj(\sh{M})\otimes_{\sh{S}_X}\shProj(\sh{N})
    \to
    \shProj(\sh{M}\otimes_{\sh{S}}\sh{N}).
\tag{8.14.2.4}
\]
\end{env}

\oldpage[II]{198}

\begin{proposition}[8.14.3]
\label{II.8.14.3}
The homomorphism \sref{II.8.14.2.4} is bijective;
so too is \sref{II.8.14.2.3} whenever the graded $\sh{S}$-module $\sh{M}$ admits a finite presentation \sref{II.3.1.1}.
\end{proposition}

\begin{proof}
The question is clearly local on $X$ and $Y$;
we can thus suppose that $Y=\Spec(A)$ is affine, with $\sh{S}=\widetilde{S}$, $\sh{M}=\widetilde{M}$, and $\sh{N}=\widetilde{N}$, where $S$ is a positively-graded $A$-algebra, and $M$ and $N$ are graded $S$-modules.
If $f$ is a homogeneous element of $S_+$, then the homomorphisms \sref{II.8.14.2.1} and \sref{II.8.14.2.2}, restricted to the affine open $D_+(f)$, correspond to the canonical homomorphisms \sref{II.2.5.11.1} and \sref{II.2.5.12.1}:
\begin{align*}
    M(m)_{(f)}\otimes_{S_{(f)}}N(n)_{(f)}
    &\to
    (M\otimes_S N)(m+n)_{(f)}
\\  (\Hom_S(M,N))(n-m)_{(f)}
    &\to
    \Hom_{S_{(f)}}(M(m)_{(f)},N(n)_{(f)}).
\end{align*}

If we refer to the definitions of these homomorphisms, we thus see (taking \sref{II.8.2.9.1} into account) that the restriction of \sref{II.8.14.2.4} to $D_+(f)$ corresponds to the canonical homomorphism
\[
    M_f\otimes_{S_f}N_f \to (M\otimes_S N)_f
\]
defined in \sref[0]{0.1.3.4}, and we know that this latter homomorphism is an isomorphism.
Similarly, the restriction of \sref{II.8.14.2.3} to $D_+(f)$ corresponds to the canonical homomorphism \sref[0]{0.1.3.5}
\[
    (\Hom_S(M,N))_f \to \Hom_{S_f}(M_f,N_f)
\]
taking into account the fact that, since $M$ is of finite type, the module $\Hom_S(M,N)$, the direct sum of the subgroups consisting of \emph{homogeneous} homomorphisms of $S$-modules \sref{II.2.1.2}, agrees with the set of \emph{all} homomorphisms $M\to N$ of $S$-modules.
The hypothesis that $M$ admits a finite presentation then implies \sref[0]{0.1.3.5} that the canonical homomorphism in question is indeed an isomorphism.
\end{proof}

\begin{proposition}[8.14.4]
\label{II.8.14.4}
If $U$ is a quasi-compact open of $X$, then there exists an integer $d$ such that, for every integer $n$ that is a multiple of $d$, $\sh{O}_X(n)|U$ is invertible, with its inverse being $\sh{O}_X(-n)|U$.
\end{proposition}

\begin{proof}
Since $q(U)$ is quasi-compact, it is covered by a finite number of affine opens $V_i$, and so every $x\in U$ is contained in some affine open of the form $D_+(f)$, where $f$ is a homogeneous element of degree $>0$ of one of the rings $\Gamma(V_i,\sh{S})$.
Since $U$ is quasi-compact, we can cover it by a finite number of such opens $D_+(f_j)$;
let $d$ be a common multiple of the degrees of the $f_j$.
This $d$ satisfies the desired property, by \sref{II.2.5.17}.
\end{proof}

\begin{env}[8.14.5]
\label{II.8.14.5}
With the hypotheses and notation of \sref{II.8.14.1}, we defined, in \sref{II.3.3.2}, canonical homomorphisms of $\sh{O}_Y$-modules
\[
\label{II.8.14.5.1}
    \alpha_n: \sh{M}_n \to q_*(\shProj_0(\sh{M}(n)))
    \qquad (n\in\bb{Z}).
\tag{8.14.5.1}
\]

Generalising the notation of \sref{II.3.3.1}, we set, for every \emph{graded $\sh{S}_X$-module $\sh{F}$},
\[
\label{II.8.14.5.2}
    \boldsymbol{\Gamma}_*(\sh{F}) = \bigoplus_{n\in\bb{Z}}q_*(\sh{F}_n).
\tag{8.14.5.2}
\]

In particular, $\boldsymbol{\Gamma}(\sh{S}_X)=\bigoplus_{n\in\bb{Z}}q_*(\sh{O}_X(n))$ is the graded $\sh{O}_Y$-algebra denoted by $\boldsymbol{\Gamma}_*(\sh{O}_X)$ in \sref{II.3.3.1.2};
it is clear that $\boldsymbol{\Gamma}(\sh{F})$ is a \emph{graded $\boldsymbol{\Gamma}_*(\sh{S}_X)$-algebra} \sref[0]{0.4.2.2}.
When
\oldpage[II]{199}
we take $\sh{M}=\sh{S}$ in the homomorphisms \sref{II.8.14.5.1}, we obtain the homomorphism of graded $\sh{O}_Y$-algebras
\[
\label{II.8.14.5.3}
    \alpha:\sh{S}\to\boldsymbol{\Gamma}(\sh{S}_X)
\tag{8.14.5.3}
\]
previously defined in \sref{II.3.3.2}, and which makes $\boldsymbol{\Gamma}_*(\sh{F})$ a \emph{graded $\sh{S}$-module};
the homomorphisms \sref{II.8.14.5.1} then define a homomorphism (of degree $0$) of \emph{graded $\sh{S}$-modules}
\[
\label{II.8.14.5.4}
    \alpha:\sh{M}\to\boldsymbol{\Gamma}_*(\shProj(\sh{M})).
\tag{8.14.5.4}
\]
\end{env}

\begin{env}[8.14.6]
\label{II.8.14.6}
In general, for a quasi-coherent graded $\sh{S}_X$-module $\sh{F}$, it is not certain that the graded $\sh{S}$-module $\boldsymbol{\Gamma}_*(\sh{F})$ will necessarily be quasi-coherent.
Consider an open $X'$ of $X$ such that the restriction $q':X'\to Y$ of $q$ to $X'$ is a \emph{quasi-compact} morphism.
Since $q'$ is further separated, $q'_*(\sh{F}')$ is then a quasi-coherent $\sh{O}_Y$-module for every quasi-coherent $\sh{O}_{X'}$ module $\sh{F}'$ \sref[I]{I.9.2.2}[b].
We set
\[
\label{II.8.14.6.1}
    \sh{S}_{X'} = \sh{S}_X|X' = \bigoplus_{n\in\bb{Z}}\sh{O}_X(n)|X'
\tag{8.14.6.1}
\]
and, for every graded $\sh{S}_{X'}$-module $\sh{F}'$,
\[
\label{II.8.14.6.2}
    \boldsymbol{\Gamma}'_*(\sh{F}') = \bigoplus_{n\in\bb{Z}}q'_*(\sh{F}'_n).
\tag{8.14.6.2}
\]

The previous remark then shows that, if $\sh{F}'$ is a quasi-coherent $\sh{S}_{X'}$-module, then $\boldsymbol{\Gamma}'_*(\sh{F}')$ is a graded \emph{quasi-coherent} $\sh{S}$-module \sref[I]{I.9.6.1}.

We note also that the canonical injection $j:X'\to X$ is \emph{quasi-compact}, because $q'=q\circ j$ is quasi-compact and $q$ is separated \sref[I]{I.6.6.4}[v].
Then $\sh{F}=j_*(\sh{F}')$ is a quasi-coherent graded $\sh{S}_X$-module for every quasi-coherent graded $\sh{S}_{X'}$-module $\sh{F}$', and it follows from the previous definitions that
\[
\label{II.8.14.6.3}
    \boldsymbol{\Gamma}'_*(\sh{F}') = \boldsymbol{\Gamma}_*(\sh{F}).
\tag{8.14.6.3}
\]

With the same hypotheses on $X'$, for every quasi-coherent graded $\sh{S}$-module $\sh{M}$, we set
\[
\label{II.8.14.6.4}
    \shProj'(\sh{M}) = \shProj(\sh{M})|X'
\tag{8.14.6.4}
\]
which is a quasi-coherent graded $\sh{S}_{X'}$-module.
The canonical homomorphism
\[
    \shProj(\sh{M}) \to j_*(\shProj'(\sh{M}))
\]
\sref[0]{0.4.4.3} thus gives a canonical homomorphism $\boldsymbol{\Gamma}_*(\shProj(\sh{M}))\to\boldsymbol{\Gamma}'_*(\shProj'(\sh{M}))$ of graded $\sh{S}$-modules, and, by composition with \sref{II.8.14.5.4}, we obtain a functorial canonical homomorphism (of degree $0$) of quasi-coherent graded $\sh{S}$-modules
\[
\label{II.8.14.6.5}
    \alpha': \sh{M} \to \boldsymbol{\Gamma}'_*(\shProj'(\sh{M})).
\tag{8.14.6.5}
\]
\end{env}

\begin{env}[8.14.7]
\label{II.8.14.7}
Keeping the hypotheses on $X'$ from \sref{II.8.14.6}, let $\sh{F}'$ be a \emph{quasi-coherent graded $\sh{S}_{X'}$-module} such that $\shProj'(\boldsymbol{\Gamma}'_*(\sh{F}'))$ is also a graded \emph{quasi-coherent} $\sh{S}_{X'}$-module.
\oldpage[II]{200}
We will define a functorial canonical homomorphism (of degree $0$) of graded $\sh{S}_{X'}$-modules
\[
\label{II.8.14.7.1}
    \beta': \shProj'(\boldsymbol{\Gamma}'_*(\sh{F}')) \to \sh{F}'.
\tag{8.14.7.1}
\]

Suppose first of all that $Y=\Spec(A)$ is affine, and that $\sh{S}=\widetilde{S}$, where $S$ is a positively-graded $A$-algebra;
then $\boldsymbol{\Gamma}'_*(\sh{F}')=\widetilde{M}$, where $M=\bigoplus{n\in\bb{Z}}\Gamma(X',\sh{F}'_n)$ is a graded $S$-module.
Let $f\in S_d$ be such that $D_+(f)\subset X'$;
by definition \sref{II.2.6.2}, $\alpha_d(f)$ restricted to $D_+(f)$ is the section of $\sh{O}_X(d)$ over $D_+(f)$ corresponding to the element $f/1$ of $(S(d))_{(f)}$, and is thus invertible;
thus so too is $\alpha_d(f^n)$ for every $n>0$.
From this, we immediately conclude that we have defined an $S_f$-homomorphism (of degree $0$) of graded modules $\beta_f:M_f\to\Gamma(D_+(f),\sh{F}')$ by sending each element $z/f^n\in M_f$ (where $z\in M$) to the section $(z|D_+(f))(\alpha_d(f^n)|D_+(f))^{-1}$ of $\sh{F}'$ over $D_+(f)$.
Furthermore, we have a commutative diagram corresponding to \sref{II.2.6.4.1}, whence the definition of $\beta'$ in this case.
To pass to the general case, we must consider an $A$-algebra $A'$, the graded $A'$-algebra $S'=S\otimes_A A'$, and use the commutative diagram analogous to \sref{II.2.8.13.2};
we leave the details to the reader.
\end{env}

\begin{proposition}[8.14.8]
\label{II.8.14.8}
If $X'$ is an open of $X=\Proj(\sh{S})$ such that $q':X'\to Y$ is quasi-compact, then the homomorphism $\beta'$ defined in \sref{II.8.14.7} is bijective.
\end{proposition}

\begin{proof}
We can clearly restrict to the case where $Y$ is affine, and everything then reduces to proving (with the notation of \sref{II.8.14.7}) that the homomorphism $\beta_f:M_f\to\Gamma(D_+(f),\sh{F}')$ is an isomorphism.
But replacing $f$ by one of its powers changes neither $D_+(f)$ nor $\beta_f$;
since $X'$ is \emph{quasi-compact} by hypothesis, we can always assume, by \sref{II.8.14.4}, that the sheaf $\sh{O}_X(d)$ is \emph{invertible}.
Since $X'$ is a scheme (because $q'$ is separated), the proposition is then exactly \sref[I]{I.9.3.1}.
\end{proof}

\begin{corollary}[8.14.9]
\label{II.8.14.9}
Under the hypotheses of \sref{II.8.14.8}, every quasi-coherent graded $\sh{S}_{X'}$-module is isomorphic to a graded $\sh{S}_{X'}$-module of the form $\shProj'(\sh{M})$, where $\sh{M}$ is a quasi-coherent graded $\sh{S}$-module.
Further, if $\sh{F}'$ is of finite type, and if we assume that $Y$ is a quasi-compact scheme, or a prescheme whose underlying space is Noetherian, then we can assume that $\sh{M}$ is of finite type.
\end{corollary}

\begin{proof}
The proof starting from \sref{II.8.14.8} follows exactly the same route as the proof of \sref{II.3.4.5} starting from \sref{II.3.4.4}, and we leave the details to the reader.
\end{proof}

\begin{proposition}[8.14.10]
\label{II.8.14.10}
Under the hypotheses of \sref{II.8.14.7}, let $\sh{M}$ be a quasi-coherent graded $\sh{S}$-module, and $\sh{F}'$ a quasi-coherent graded $\sh{S}_{X'}$-module;
the composite homomorphisms
\[
\label{II.8.14.10.1}
    \shProj'(\sh{M})
    \xrightarrow{\shProj'(\alpha')}
    \shProj'(\boldsymbol{\Gamma}'_*(\shProj'(\sh{M})))
    \xrightarrow{\beta'}
    \shProj'(\sh{M})
\tag{8.14.10.1}
\]
\[
\label{II.8.14.10.2}
    \boldsymbol{\Gamma}'_*(\sh{F}')
    \xrightarrow{\alpha'}
    \boldsymbol{\Gamma}'_*(\shProj'(\boldsymbol{\Gamma}'_*(\sh{F}')))
    \xrightarrow{\boldsymbol{\Gamma}'_*(\beta')}
    \boldsymbol{\Gamma}'_*(\sh{F}')
\tag{8.14.10.2}
\]
are the identity isomorphisms.
\end{proposition}

\begin{proof}
The question is local on $Y$, and the proof follows as in \sref{II.2.6.5};
we leave the details to the reader.
\end{proof}

\begin{remark}[8.14.11]
\label{II.8.14.11}
In chapter~III \sref[III]{III.2.3.1}, we will see that, when $Y$ is \emph{locally Noetherian}, and $\sh{S}$ is a quasi-coherent graded $\sh{O}_Y$-algebra \emph{of finite type} (in which case
\oldpage[II]{201}
we can take $X'=X$), then the homomorphism $\alpha$ \sref{II.8.14.5.4} is \textbf{(TN)}-\emph{bijective} for every quasi-coherent graded $\sh{S}$-module $\sh{M}$ satisfying condition~\textbf{(TF)}.
\end{remark}

\begin{remark}[8.14.12]
\label{II.8.14.12}
The situation described in \sref{II.8.14.4} is a particular case of the following.
Let $X$ be a ringed space, and $\sh{S}$ a (positively- and negatively-) graded $\sh{O}_X$-algebra;
suppose that there exists an integer $d>0$ such that $\sh{S}_d$ and $\sh{S}_{-d}$ are \emph{invertible}, with the canonical homomorphism
\[
\label{II.8.14.12.1}
    \sh{S}_d\otimes_{\sh{O}_X}\sh{S}_{-d} \to \sh{O}_X
\tag{8.14.12.1}
\]
being an \emph{isomorphism} (such that $\sh{S}_{-d}$ is identified with $\sh{S}_d^{-1}$).
We then say that the graded $\sh{O}_X$-algebra $\sh{S}$ is \emph{periodic}, \emph{of period $d$}.
This nomenclature stems from the following property:
\emph{under the preceding hypotheses, for every graded $\sh{S}$-module $\sh{F}$, the canonical homomorphism}
\[
\label{II.8.14.12.2}
    \sh{S}_d\otimes\sh{F}_n \to \sh{F}_{n+d}
\tag{8.14.12.1}
\]
\emph{is an isomorphism for all $n\in\bb{Z}$.}
Indeed, the question is local on $X$, and we can assume that $\sh{S}_d$ has an \emph{invertible} section $s$ over $X$, with its inverse $s'$ being a section of $\sh{S}_{-d}$.
The homomorphism $\sh{F}_{n+d}\to\sh{S}_d\otimes\sh{F}_n$, which sends each section $z\in\Gamma(U,\sh{F}_{n+d})$ to the section $(s|U)\otimes(s'|U)z$ of $\sh{S}_d\otimes\sh{F}_n$ over $U$, is then the inverse of \sref{II.8.14.12.2}, whence our claim.
This induces, for all $k\in\bb{Z}$, a canonical isomorphism
\[
    (\sh{S}_d)^{\otimes k}\otimes\sh{F}_n \simto \sh{F}_{n+kd}.
\]
Then \emph{the data of a graded $\sh{S}$-module $\sh{F}$ is equivalent to the data of $\sh{S}_0$-modules $\sh{F}_i$ ($0\leq i\leq d-1$) and canonical homomorphisms}
\[
    \sh{S}_i\otimes\sh{F}_j \to \sh{F}_{i+j}
    \qquad
    \mbox{for $0\leq i,j\leq d-1$}
\]
(setting $\sh{F}_{i+j}=\sh{S}_d\otimes_{\sh{S}_0}\sh{F}_{i+j-d}$ whenever $i+j\geq d$).
Of course, for theses homomorphisms to give a well-defined $\sh{S}$-module structure on the direct sum of the $(\sh{S}_d)^{\otimes k}\otimes\sh{F}_i$ ($k\in\bb{Z}$, $0\leq i\leq d-1$), they should satisfy some associativity conditions that we will not explain.

In the case where $d=1$ (which is the one considered in \sref{subsection:II.3.3}), we can thus say that the category of graded $\sh{S}$-modules (resp. quasi-coherent $\sh{S}$-modules if $X$ is a prescheme and $\sh{S}$ is quasi-coherent) is \emph{equivalent} to the category of arbitrary $\sh{S}_0$-modules (resp. quasi-coherent $\sh{S}_0$-modules);
it is in this way that we can think of the results of this paragraph as generalising those of §3.
Furthermore, we see that, under suitable finiteness conditions, the results of this paragraph (along with \sref{II.8.14.11}) reduces, in some sense, the study of quasi-coherent graded algebras on a prescheme, and graded modules ``modulo \textbf{(TN)}'' on such algebras, to the study of the particular case where the algebras in question are \emph{periodic} (and where condition~\textbf{(TN)} for $\sh{M}$ \sref{II.3.4.2} thus implies that $\sh{M}=0$).
\end{remark}

\begin{remark}[8.14.13]
\label{II.8.14.13}
Under the hypotheses of \sref{II.8.14.1}, let $d$ be an integer $>0$;
we have defined a canonical $Y$-isomorphism $h$ from $X$ to $X^{(d)}=\Proj(\sh{S}^{(d)})$ \sref{II.3.1.8}.
For every
\oldpage[II]{202}
quasi-coherent graded $\sh{S}$-module $\sh{M}$ and every integer $k$ such that $0\leq k\leq d-1$, we also have (with the notation of \sref{II.3.1.1}) a canonical $h$-isomorphism
\[
\label{II.8.14.13.1}
    (\shProj(\sh{M}))^{(d,k)} \xleftarrow{\sim} \shProj(\sh{M}^{(d,k)}).
\tag{8.14.13.1}
\]

Suppose, first of all, that $Y=\Spec(A)$ is affine, $\sh{S}=\widetilde{S}$, and $\sh{M}=\widetilde{M}$, where $S$ is a positively-graded $A$-algebra, and $M$ a graded $S$-module.
We know, for every $f\in S_e$ ($e>0$), that $h$ sends $D_+(f)$ to $D_+(f^d)$, and corresponds to the canonical isomorphism $S_{(f^d)}\to S_{(f)}$ \sref{II.2.2.2}.
The restriction of \sref{II.8.14.13.1} to $D_+(f^d)$ then corresponds to the canonical di-isomorphism $M_{f^d}\to M_f$ restricted to the elements of $M_{f^d}$ whose degree is congruent to $k$ (modulo $d$).
We leave to the reader the task of showing that these isomorphisms are compatible with passing from $f$ to some homogeneous multiple $fg$, and then that there is an analogous compatibility with passing from $S$ to a graded $A'$-algebra $S'=S\otimes_A A'$, where $A'$ is some $A$-algebra.
In particular, this gives us an $h$-isomorphism
\[
\label{II.8.14.13.2}
    (\sh{S}^{(d)})_{X^{(d)}} \simto (\sh{S}_X)^{(d)}
\tag{8.14.13.2}
\]
that respects the multiplicative structures of both the source and the target, and that, thanks to \sref{II.8.14.13.1}, becomes an $h$-di-isomorphism from a graded $(\sh{S}^{(d)})_{X^{(d)}}$-module to a graded $(\sh{S}_X)^{(d)}$-module.
Similarly, we have an $h$-isomorphism
\[
\label{II.8.14.13.3}
    \shProj_0(\sh{S}^{(d,k)}(n)) \simto \sh{O}_X(nd+k),
\tag{8.14.13.3}
\]
which completes the result of \sref{II.3.2.9}[ii].

The isomorphism in \sref{II.8.14.13.1} immediately induces an isomorphism of graded $\sh{S}^{(d)}$-modules
\[
\label{II.8.14.13.4}
    \boldsymbol{\Gamma}_*^{(d)}(\shProj(\sh{M}^{(d,k)})) \simto \boldsymbol{\Gamma}_*((\shProj(\sh{M}))^{(d,k)})
\tag{8.14.13.4}
\]
where $\boldsymbol{\Gamma}_*^{(d)}$ corresponds to the structure morphism $q^{(d)}:X^{(d)}\to Y$;
it can be immediately verified that the canonical homomorphism $\alpha$ \sref{II.8.14.5.4}, and the analogous homomorphism $\alpha^{(d)}$ for $X^{(d)}$, make the following diagram commute:
\[
\label{II.8.14.13.5}
    \xymatrix{
        &\sh{M}^{(d,k)} \ar[dl]_{\alpha^{(d)}} \ar[dr]^{\alpha}&
    \\  \boldsymbol{\Gamma}_*^{(d)}(\shProj(\sh{M}^{(d,k)})) \ar[rr]^{\sim}
        && \boldsymbol{\Gamma}_*((\shProj(\sh{M}))^{(d,k)})
    }
\tag{8.14.13.5}
\]
where we proceed by supposing that $Y$ is affine and then calculating the restrictions of the images under $\alpha^{(d)}$ and $\alpha$ of some single element of $M^{(d,k)}$ to the open subsets $D_+(f^d)$ and $D_+(f)$ (using the same notation as above).
\end{remark}

\begin{proposition}[8.14.14]
\label{II.8.14.14}
Let $Y$ be a quasi-compact prescheme, $\sh{S}$ a quasi-coherent graded $\sh{O}_Y$-algebra of finite type, and $\sh{M}$ a quasi-coherent graded $\sh{S}$-module satisfying condition~\textbf{(TF)};
let $X=\Proj(\sh{S})$.
Then $\sh{S}_X$ is a periodic graded $\sh{O}_X$-algebra \sref{II.8.14.12}, and there exists
\oldpage[II]{203}
some period $d$ of $\sh{S}_X$ such that the $(\shProj(\sh{M}))^{(d,k)}$ ($0\leq k\leq d-1$) are $(\sh{S}_X)^{(d)}$-modules of finite type.
\end{proposition}

\begin{proof}
Indeed, \sref{II.3.1.10} proves that there exists some $d$ such that $\sh{S}^{(d)}$ is generated by $\sh{S}_d=(\sh{S}^{(d)})_1$, with the latter being an $\sh{S}_0$-module of finite type.
To prove the first claim, we can thus, by \sref{II.8.14.13.2}, restrict to the case where $d=1$, and the proposition then follows from \sref{II.3.2.7}.
Furthermore, taking \sref{II.8.14.13.1} into account, the second claim is a consequence of \sref{II.2.1.6}[iii] and \sref{II.3.4.3}.
\end{proof}


\bibliography{the}
\bibliographystyle{amsalpha}

\end{document}


\section{Differential invariants. Differentially smooth morphisms}
\label{section:IV.16}

\oldpage[IV-4]{5}
In this paragraph we will present, in global form, some notions of differential calculus particularly useful in algebraic geometry.
We will ignore many classic developments in differential geometry (connections, infinitesimal transformations associated to vector fields, jets, etc.), although these notions are translated in a particularly natural way for schemes.
We will similarly ignore phenomena exclusive to characteristic $p>0$ (some of which are seen, in the affine case, in \sref[0]{0.21}.
For certain complements to the differential formalism for preschemes the reader may consult Expos\'es~II and VII of \cite{IV-42} as well as subsequent chapters of this treatise. 

\subsection{Normal invariants of an immersion}
\label{IV.16.1}

\begin{env}[16.1.1]
\label{IV.16.1.1}
Let $(X, \sh{O}_X), (Y, \sh{O}_Y)$ be two ringed spaces and $f = (\psi, \theta): Y \to X$ a morphism of ringed spaces \sref[0]{0.4.1.1} such that the homomorphism
\[
  \theta^\#: \psi^*(\sh{O}_X) \to \sh{O}_Y
\]
is surjective, so that $\sh{O}_Y$ is identified with a sheaf of quotient rings $\psi^*(\sh{O}_X)/\sh{I}_f$. 
We can then endow $\psi^*(\sh{O}_X)$ with the $\sh{I}_f$-preadic filtration.
\end{env}

\begin{definition}[16.1.2]
\label{IV.16.1.2}
The $\sh{O}_Y$-augmented sheaf of rings $\psi^*(\sh{O}_X)/\sh{I}_f^{n+1}$ is called the $n$'th \emph{normal invariant} of $f$;
the ringed space $(Y, \psi^*(\sh{O}_X)/\sh{I}_f^{n+1})$ is called the $n$'th \emph{infinitesimal neighbourhood} of $Y$ along $f$ and is denoted by $Y^{(n)}_f$ or simply $Y^{(n)}$.
The sheaf of graded rings associated to the sheaf of filtered rings $\psi^*(\sh{O}_X)$
\[
  \label{IV.16.1.2.1}
  \shGr_\bullet(f) = \bigoplus_{n \geq 0}(\sh{I}_f^{n}/\sh{I}_f^{n+1} )
  \tag{16.1.2.1}
\]
is called the sheaf of graded rings \emph{associated to} $f$. The sheaf $\shGr_1(f) = \sh{I}_f/\sh{I}_f^{2}$ is called the \emph{conormal sheaf} of $f$ (that will be denoted by $\sh{N}_{Y/X}$ when there is no risk of confusion). 
\end{definition}

It is clear that the $\sh{O}_{Y^{(n)}} = \psi^*(\sh{O}_X)/\sh{I}_f^{n+1}$ (that we also denote $\sh{O}_{Y_f^{(n)}})$ form a
\oldpage[IV-4]{6}
projective system of sheaves of rings on $Y$, the transition homomorphism $\vphi_{nm}:\sh{O}_{Y^{(m)}} \to \sh{O}_{Y^{(n)}}$ for $n \leq m$ identifies $\sh{O}_{Y^{(n)}}$ with the quotient of $\sh{O}_{Y^{(m)}}$ by the power $(\sh{I}_f/\sh{I}_f^{n+1} )^m$ of the \emph{augmentation ideal} of $\sh{O}_{Y^{(n)}}$, kernel of $\vphi_{0n}: \sh{O}_{Y^{(n)}} \to \sh{O}_{Y}$.
The $Y^{(n)}$ therefore form a inductive system of ringed spaces, all having underlying space $Y$, and we have canonical morphisms of ringed spaces $h_n: Y^{(n)} \to X$ equal to $(\psi, \theta_n)$, where $\theta^\#_n$ is the canonical morphism $\psi^*(\sh{O}_X) \to \psi^*(\sh{O}_X)/\sh{I}_f^{n+1}$.
It is clear that the sheaf $\shGr_\bullet(f)$ is a sheaf of graded algebras over the sheaf of rings $\sh{O}_Y = \shGr_0(f)$ and the $\shGr_k(f)$ of $\sh{O}_Y$-modules.

As with every sheaf of filtered rings, we have a \emph{canonical surjective homomorphism} of graded $\sh{O}_Y$-algebras
\[
  \label{IV.16.1.2.2}
  \bb{S}_{\sh{O}_Y}^\bullet(\shGr_1(f)) \to \shGr_\bullet(f)
  \tag{16.1.2.2}
\]
which coincide in degrees $0$ and $1$ with the identities.

\begin{examples}[16.1.3]
\label{IV.16.1.3}
\medskip\noindent
\begin{enumerate}
  \item[(i)] Suppose that $X$ is a locally ringed space, $Y$ is reduced to a single point $y$ (endowed with a ring $\sh{O}_y$) and that, if $x = \psi(y)$, $\theta^\#:\sh{O}_x \to \sh{O}_y$ is a \emph{surjective} homomorphism of rings having as kernel the maximal ideal $\mathfrak{m}_x$ of $\sh{O}_x$.
  So the $\sh{O}_{Y^{(n)}}$ are identified with the rings $\sh{O}_x/\mathfrak{m}_x^{n+1}$ and $\shGr_\bullet(f)$ with the graded ring associated with the local ring $\sh{O}_x$ endowed with the $\mathfrak{m}_x$-preadic filtration.
  \item[(ii)] Suppose that $Y$ is a closed subset of an open subspace $U$ of $X$ and that the $\sh{O}_Y$ is induced on $Y$ by a quotient sheaf $\sh{O}_U/\sh{I}$, where $\sh{I}$ is an ideal of $\sh{O}_U$ such that $\sh{I}_x = \sh{O}_x$ for every $x \not\in Y$;
  if $X$ is a locally ringed space we also suppose that $\sh{I}_x \neq \sh{O}_x$ for $y \in Y$ so that $(Y, \sh{O}_Y)$ is a locally ringed space.
  
  Let $\psi_0: Y \to U$ be the canonical injection and denote by $\theta_0: \sh{O}_U \to (\psi_0)_*(\sh{O}_Y)$ the homomorphism such that $\theta_0^\#$ is the canonical homomorphism $\psi^*_0(\sh{O}_U) = \sh{O}_U|Y \to (\sh{O}_U/\sh{I})|Y$, so that $j_0=(\psi_0, \theta_0):Y \to U$ is a morphism of ringed spaces (and of locally ringed spaces if $X$ is a locally ringed space);
  if $i:U \to X$ is the canonical injection (morphism of ringed spaces), $j = i\circ j_0$ is the morphism $(\psi, \theta)$ of $Y$ to $X$ where $\psi: Y \to X$ is the canonical injection and $\theta:\sh{O}_X \to \psi_*(\sh{O}_Y)$ is the homomorphism such that $\theta^\# = \theta_0^\#$.
  Since $\theta^\#$ is surjective we can apply the previous definitions;
  $\sh{O}_{Y^{(n)}}$ is equal to $\psi^*_0(\sh{O}_U/\sh{I}^{n+1})$, and we have $(\psi_0)_*(\sh{O}_{Y^{(n)}} ) = \sh{O}_U/\sh{I}^{n+1}$, and $\shGr_n(j) = \shGr_n(j_0) = \psi^*_0(\sh{I}^n/\sh{I}^{n+1}) = j^*_0(\sh{I}^n/\sh{I}^{n+1})$.
\end{enumerate}
\end{examples}

\begin{env}[16.1.4]
\label{IV.16.1.4}
The example \sref{IV.16.1.3}[(ii)] shows that in general the $\sh{O}_{Y^{(n)}}$ are \emph{not canonically endowed with a structure of an $\sh{O}_Y$-module}, or \emph{a fortiori} with a structure of an $\sh{O}_Y$-algebra.
The data of such structure is equivalent to the data of a homomorphism of sheaves of rings $\lambda_n:\sh{O}_Y \to \sh{O}_{Y^{(n)}}$, right inverse to the augmentation morphism $\vphi_{0n}$;
it is also equivalent to the data of a morphism of ringed spaces $(1_Y, \lambda_n): Y^{(n)} \to Y$ left inverse to the canonical morphism $(1_Y, \vphi_{0n}): Y \to Y^{(n)}$.
\end{env}

\begin{proposition}[16.1.5]
\label{IV.16.1.5}
Let $f = (\psi, \theta): Y \to X$ be an immersion of preschemes.
Then:
\begin{enumerate}
  \item[{\rm(i)}] $\shGr_\bullet(f)$ is a quasi-coherent graded $\sh{O}_Y$-algebra.
\oldpage[IV-4]{7}
  \item[{\rm(ii)}] The $Y^{(n)}$ are preschemes, canonically isomorphic to subpreschemes of $X$.
  \item[{\rm(iii)}] Every homomorphism of sheaves of rings $\lambda_n: \sh{O}_Y \to \sh{O}_{Y^{(n)}}$, right inverse to the augmentation homomorphism $\vphi_{0n}$, makes the $\sh{O}_{Y^{(n)}}$ and $\sh{O}_{Y^{(k)}}$ for $k\leq n$ quasi-coherent $\sh{O}_Y$-algebras;
  the $\sh{O}_Y$-module structures induced from the above structures on the $\shGr_k(f)$ for $k \leq n$ coincide with the ones defined in \sref{IV.16.1.2}.
\end{enumerate}
\end{proposition}

\begin{proof}
(i) Since the question is local on $X$ and $Y$, we can reduce to the case where $Y$ is a closed subpreschemes of $X$ defined by an quasi-coherent ideal $\sh{I}$ of $\sh{O}_X$;
since $\sh{O}_Y$ is the restriction to $Y$ of $\sh{O}_X/\sh{I}$ the assertion (i) is evident, and $Y^{(n)}$ is the closed subprescheme of $X$ defined by the quasi-coherent ideal $\sh{I}^{n+1}$ of $\sh{O}_X$.
Finally, to prove (iii) we notice that the data of $\lambda_n$ makes the ideal $\sh{I}/\sh{I}^n$ of the augmentation $\vphi_{0n}$ and their quotients $\sh{I}/\sh{I}^{k+1} (1\leq k \leq n)$ $\sh{O}_Y$-modules, and it suffices to prove by induction on $k$ that the $\sh{I}/\sh{I}^{k+1}$ are quasi-coherent $\sh{O}_Y$-modules and the structure of quotient $\sh{O}_Y$-module induced on $\sh{I}^k/\sh{I}^{k+1}$ is the same as defined on \sref{IV.16.1.2}.
The second assertion is immediate, $\sh{I}^k/\sh{I}^{k+1}$ being killed by $\sh{I}/\sh{I}^{n+1}$;
the first result, by induction on $k$, is trivial for $k=1$ and for $\sh{I}/\sh{I}^{k+1}$ being an extension of $\sh{I}/\sh{I}^{k}$ by $\sh{I}^k/\sh{I}^{k+1}$ \sref{III.1.4.17}.
\end{proof}

\begin{corollary}[16.1.6]
\label{IV.16.1.6}
Under the general hypotheses of \sref{IV.16.1.5}, if the immersion $f$ is locally of finite presentation then the $\shGr_n(f)$ are quasi-coherent $\sh{O}_Y$-modules of finite type.
\end{corollary}

\begin{proof}
Indeed, with the notation from the proof of \sref{IV.16.1.5}, $\sh{I}$ is an ideal of finite type of $\sh{O}_X$ \sref{IV.1.4.7}, therefore the $\sh{I}^n/\sh{I}^{n+1}$ are $\sh{O}_Y$-modules of finite type, hence the conclusion.
\end{proof}

\begin{corollary}[16.1.7]
\label{IV.16.1.7}
Under the general hypotheses of \sref{IV.16.1.5}, let $g:X \to Y$ be a morphism of preschemes, left inverse to $f$.
Therefore, for every $n$, the composite morphism $(1, \lambda_n): Y^{(n)}\xrightarrow{h_n} X \xrightarrow{g} Y$ defines a homomorphism of sheaves of rings $\lambda_n: \sh{O}_Y \to \sh{O}_{Y^{(n)}}$ right inverse to the augmentation $\vphi_{0n}$, making $\sh{O}_{Y^{(n)}}$ a quasi-coherent $\sh{O}_Y$-algebra;
via these homomorphisms, the transition homomorphism $\vphi_{nm}:\sh{O}_{Y^{(m)}} \to \sh{O}_{Y^{(n)}}$ ($n\leq m$) are homomorphisms of $\sh{O}_Y$-algebras. 
Also, if $g$ is locally of finite type, then the $\sh{O}_{Y^{(n)}}$ are quasi-coherent $\sh{O}_Y$-modules of finite type.
\end{corollary}

\begin{proof}
The first assertion is an immediate result from the definitions and \sref{IV.16.1.5}.
On the other hand, if $g$ is locally of finite type, then $f$ is locally of finite presentation \sref{IV.1.4.3}[(v)];
the $\shGr_n(f)$ being then quasi-coherent $\sh{O}_Y$-modules of finite type by \sref{IV.16.1.6}, the same goes for the $\sh{O}_Y$-modules $\sh{I}/\sh{I}^{n+1}$, being extensions of a finite number of the $\shGr_k(f)$ \sref[III]{III.1.4.17}.
\end{proof}

\begin{proposition}[16.1.8]
\label{IV.16.1.8}
Let $X$ be a locally Noetherian prescheme, $j:Y \to X$ an immersion;
Then the $Y^{(n)}$ are locally Noetherian preschemes, the $\shGr_n(j)$ are coherent $\sh{O}_Y$-modules and the $\shGr_\bullet(j)$ is a coherent sheaf of rings over the space $Y$.
\end{proposition}

\begin{proof}
Everything is local on $X$ and $Y$, so we reduce to the case where $X$ is affine and $j$ is a closed immersion and therefore all the assertions are evident except for the last, which follows from the fact that if $A$ is a Noetherian ring and $\mathfrak{I}$ is an ideal of $A$, then $\gr_\mathfrak{I}^\bullet(A)$ is a Noetherian ring, taking into account the exactness of the functor $\psi^*$ and \sref[0]{0.5.3.7}.
\end{proof}

\begin{proposition}[16.1.9]
\label{IV.16.1.9}
\oldpage[IV-4]{8}
Let $X$ be a prescheme, $j: Y \to X$ an immersion locally of finite presentation, $y$ a point of $Y$. The following conditions are equivalent:
\begin{enumerate}
  \item[(a)] There exists an open neighbourhood $U$ of y in $Y$ such that $j|U$ is a homeomorphism of $U$ onto an open set of $X$.
  \item[(b)] There is an integer $n>0$ such that the canonical homomorphism
  \[
    (\vphi_{n-1,n})_y: \sh{O}_{Y^{(n)},y} \to \sh{O}_{Y^{(n-1)},y}
  \]
  is bijective.
  \item[(c)] There is an integer $n>0$ such that $(\shGr_n(j))_y = 0$.
  
  In addition, if the integer $n$ satisfies \emph{(b)} or \emph{(c)}, then there is a neighbourhood $V$ of $y$ in $Y$ such that $\shGr_m(j)|V = 0$ for $m \geq n$ and that $\vphi_{nm}|V: \sh{O}_{Y^{(m)}}|V \to \sh{O}_{Y^{(n)}}|V$ is bijective for $m \geq n$. 
\end{enumerate}
\end{proposition}

\begin{proof}
Since the questions is local on $Y$, we can restrict ourselves to the case where $j$ is a closed immersion, $Y$ being defined by a quasi-coherent ideal \emph{of finite type} $\mathfrak{I}$ of $\sh{O}_X$.
The equivalence of (b) and (c), for a given $n$, is immediate;
also, since $\sh{I}^n/\sh{I}^{n+1}$ is an $\sh{O}_X$-module of finite type, there is an open neighbourhood $U$ of $y$ in $X$ such that $\sh{I}^n|U = \sh{I}^{n+1}|U$ \sref[0]{0.5.2.2}, so we also have $\sh{I}^n|U = \sh{I}^m|U$ for $m \geq n$ proving the last assertions.
To prove that (a) implies (b), we can restrict ourselves to the cases where the underlying space of $Y$ is equal to the underlying space of $X$ and where $\sh{I}$ is generated by a finite number of sections over $X$:
since $\sh{I}$ is contained in the nilradical $\sh{N}$ of $\sh{O}_X$ \sref[I]{I.5.1.2}, it is now nilpotent which proves b).
Finally, to prove that (b) implies (a), we can restrict ourselves to the case where $\sh{I}^n = \sh{I}^m$; 
therefore, for every $y \in Y$, since $\sh{I}_y \subset \mathfrak{m}_y$, maximal ideal of $\sh{O}_{X,y}$, we must have $\sh{I}^n_y = 0$ because of Nakayama's lemma, since $\sh{I}_y$ is an ideal of finite type.
The set of $x \in X$ such that $\sh{I}^n_x = 0$ is an open $U$ of $X$ contained in $Y$ \sref[0]{0.5.2.2};
since on the other hand $\sh{I}_x \neq 0$ for $x \notin Y$, we must have $U = Y$.
\end{proof}

\begin{corollary}[16.1.10]
\label{IV.16.1.10}
For a restriction of the immersion $j$ to an open neighbourhood of $y$ in $Y$ to be an open immersion (in other words, for $j$ to be a \emph{local isomorphism} on the point $y$), it is necessary and sufficient that $(\shGr_1(j))_y = (\sh{N}_{Y/X})_y = 0$.
\end{corollary}

\begin{proof}
The condition is clearly necessary, and the previous reasoning applied to $n=1$ proves that it is sufficient.
\end{proof}

\begin{remark}[16.1.11]
\label{IV.16.1.11}
\medskip\noindent
\begin{enumerate}
  \item[(i)] Under the conditions of the definition \sref{IV.16.1.1}, the projective limit of the projective system $(\sh{O}_{Y^{(n)}}, \vphi_{nm})$ of sheaves of rings over $Y$ is called the \emph{normal invariant of infinite order} of $f$, and sometimes denoted by $\sh{O}_{Y^{(\infty)}}$.
  When $X$ is a locally noetherian prescheme, $j:Y \to X$ a closed immersion, $Y$ then is a closed subprescheme of $X$ defined by a coherent ideal $\sh{I}$ and $\sh{O}_{Y^{(\infty)}}$ is exactly the \emph{formal completion} of $\sh{O}_X$ along $Y$ \sref[I]{I.10.8.4}, and $Y^{(\infty)} = (Y, \sh{O}_{Y^{(\infty)}})$ is the formal prescheme that is the \emph{completion} of $X$ along $Y$ \sref[I]{I.10.8.5}.
  In all cases, we could say that $Y^{(\infty)}$ is the \emph{formal neighbourhood} of $Y$ in $X$ (via the morphism $f$).
  In the particular case we have just considered, it is the formal prescheme that is the inductive limit of the infinitesimal neighbourhoods of order $n$.
  \item[(ii)] Note that for a morphism of preschemes $f=(\psi, \theta): Y \to X$, it can happen that the homomorphism $\theta^\#:\psi^*(\sh{O}_X) \to \sh{O}_Y$ is surjective without $f$ being a local 
\oldpage[IV-4]{9}
  immersion and without $f$ being injective.
  We have an example by taking $Y$ to be a sum of preschemes $Y_\lambda$ all isomorphic to $\Spec(\sh{O}_x)$, where $x \in X$, ad taking $f$ to be the morphism equal to the canonical morphism in each of the $Y_\lambda$.
\end{enumerate}
\end{remark}

\subsection{Functorial properties of the normal invariants of an immersion}
\label{IV.16.2}

\begin{env}[16.2.1]
\label{IV.16.2.1}
Let $f = (\psi, \theta): Y \to X$ and $f' = (\psi', \theta'): Y' \to X'$ by two morphisms of ringed spaces such that $\theta^\#$ and $\theta'^\#$ are surjective;
consider a commutative diagram of morphisms of ringed spaces
\[
  \label{IV.16.2.1.1}
  \xymatrix{
    Y \ar[r]^f & X \\
    Y'\ar[r]_{f'} \ar[u]^u & X'\ar[u]_v  
  }
  \tag{16.2.1.1}
\]

Let $u = (\rho, \lambda), v = (\sigma, \mu)$. 
We have $\rho^*(\psi^*(\sh{O}_X)) = \psi'^*(\sigma^*(\sh{O}_X))$ and as a result a commutative diagram of homomorphisms of sheaves of rings over $Y'$
\[
  \xymatrix{
    \rho^*(\psi^*(\sh{O}_X)) = \psi'^*(\sigma^*(\sh{O}_X)) \ar[r]^-{\psi'^*(\mu^\#)}\ar[d]_{\rho^*(\theta^\#)} & \psi'^*(\sh{O}_{X'}) \ar[d]^{\theta'^\#} \\
    \rho^*(\sh{O}_Y)\ar[r]_-{\lambda^\#}  & \sh{O}_{Y'}  
  }
\]
from which we conclude, if $\sh{I}$ and $\sh{I'}$ are the kernels of $\theta^\#$ and $\theta'^\#$, that we have $\psi'^*(\mu^\#)(\rho^*(\sh{I})) \subset \sh{I'}$, having in mind the exactness of the functor $\rho^*$.
We deduce that, for every integer $n$, $\psi'^*(\mu^\#)(\rho^*(\sh{I}^n)) \subset \sh{I'}^n$, which shows that $\psi'^*(\mu^\#)$ defines, passing to the quotients, a homomorphism of sheaves of rings
\[
  \label{IV.16.2.1.2}
  \nu_n: \rho^*(\psi^*(\sh{O}_X)/\sh{I}^{n+1}) \to \psi'^*(\sh{O}_{X'})/\sh{I'}^{n+1}
  \tag{16.2.1.2}
\]
and therefore a morphism of ringed spaces $w_n = (\rho, \nu_n): Y'^{(n)} \to Y^{(n)}$ (which, for $n = 0$, is none other than $u$).
It follows immediately from this definition that the diagrams
\[
  \xymatrix@R=1pc{
    Y^{(n)} \ar[r]^-{h_{mn}} & Y^{(m)} \ar[r]^-{h_m} & X \\
    & & & (n \leq m) \\
    Y'^{(n)} \ar[r]_-{h'_{mn}} \ar[uu]^-{w_n} & Y'^{(m)} \ar[r]_-{h'_m} \ar[uu]^-{w_m} & X' \ar[uu]_-v \\
  }
\]
(where the horizontal arrows are the canonical morphisms \sref{IV.16.1.2}) are commutative.

By passage to the quotients via the morphisms \sref{IV.16.2.1.2}, and taking into
\oldpage[IV-4]{10}
account the exactness of the functor $\rho^*$, we obtain a di-homomorphism of graded algebras (relative to the morphism $\lambda^\#: \rho^*(\sh{O}_Y) \to \sh{O}_{Y'}$)
\[
  \label{IV.16.2.1.3}
  \gr(u): \rho^*(\shGr_\bullet(f)) \to \shGr_\bullet(f')
  \tag{16.2.1.3}
\]
(or, if you like, a $\rho$-morphism \sref[0]{0.3.5.1} $\shGr_\bullet(f) \to \shGr_\bullet(f')$), and in particular a di-homomorphism of conormal sheaves
\[
  \gr_1(u): \rho^*(\shGr_1(f)) \to \shGr_1(f').
\]

It is also immediate that these homomorpisms give rise to a commutative diagram
\[
  \label{IV.16.2.1.4}
  \xymatrix{
    \rho^*(\bb{S}_{\sh{O}_Y}^\bullet(\shGr_1(f))) \ar[r] \ar[d]_-{\bb{S}(\gr_1(u))} & \rho^*(\shGr_\bullet(f)) \ar[d]^-{\gr(u)}\\
    \bb{S}_{\sh{O}_Y}^\bullet(\shGr_1(f')) \ar[r] & \shGr_\bullet(f')
  }
  \tag{16.2.1.4}
\]
where the horizontal arrow are the canonical morphisms \sref{IV.16.1.2.2}.

Finally, if we have a commutative diagram of morphisms of ringed spaces
\[
  \xymatrix{
    Y \ar[r]^{f} & X \\
    Y' \ar[r]_{f'} \ar[u]^u & X' \ar[u]_v\\
    Y'' \ar[r]_{f''} \ar[u]^{u'} & X'' \ar[u]_{v'}\\
  }
\]
where $f'' = (\psi'', \theta'')$ is such that $\theta''^\#$ is surjective, and if $w_n'$ and $w_n''$ are defined from $u'$, $v'$ for one and $u'' = u \circ u'$, $v'' = v \circ v'$ for the other, we have $w_n'' = w_n \circ w_n'$, which follows immediately from the definitions and from \sref[0]{0.3.5.5};
we have also $\gr(u'') = \gr(u') \circ \rho'^*(\gr(u))$ if $u' = (\rho', \lambda')$.
Therefore we can say that $Y^{(n)}$ and $\shGr_\bullet(f)$ \emph{depend functorially} on $f$. 
\end{env}

\begin{proposition}[16.2.2]
\label{IV.16.2.2}
With the notation and hypotheses of \sref{IV.16.2.1}, suppose also that $f$, $f'$, $u$, and $v$ are morphisms of preschemes. We have:
\begin{enumerate}
  \item[{\rm(i)}] The morphisms $w_n:Y'^{(n)} \to Y^{(n)}$ are morphisms of preschemes.
  \item[{\rm(ii)}] If $Y' = Y \times_X X'$, $u$ and $f'$ the canonical projections, and if $f$ is an immersion or if $v$ is flat, we have $Y'^{(n)} = Y^{(n)} \times_X X'$.
  \item[{\rm(iii)}] If $Y' = Y \times_X X'$ and if $v$ is flat (resp. if $f$ is an immersion), the homomorphism 
  \[
    \Gr(u) = \gr(u)\otimes I : \shGr_\bullet(f)\otimes_{\sh{O}_Y}\sh{O}_{Y'} \to \shGr_\bullet(f')
  \]
  is bijective (resp. surjective).
\end{enumerate}
\end{proposition}

\begin{proof}
\medskip\noindent
\begin{enumerate}
  \item[(i)] The hypotheses immediately imply that, for every $y' \in Y'$, $\rho_{y'}^*(\theta_{\psi'(y')}^\#)$ is a \emph{local} homomorphism \sref[I]{I.1.6.2}, so $w_n$ is a morphism of preschemes \sref[I]{I.2.2.1}.
  \oldpage[IV-4]{11}
  \item[(ii) and (iii)] If $f$ is an immersion, we can restrict ourselves to the case where $f$ is a closed immersion, $Y$ being defined by a quasi-coherent ideal $\sh{I}$ of $\sh{O}_X$ and $Y^{(n)}$ by the ideal $\sh{I}^{n+1}$;
  the assertions follows from \sref[I]{I.4.4.5}.

  Second, suppose that $v$ is flat;
  we can restrict ourselves to the case where $X = \Spec(A)$, $Y = \Spec(B)$, $X' = \Spec(A')$ are affines, $A'$ being a flat $A$-module;
  so $Y' = \Spec(B')$ where $B' = B \otimes_A A'$;
  in addition, if $\mathfrak{I}$ is the kernel of the homomorphism $A \to B$, the kernel $\mathfrak{I'}$ of $A' \to B'$ is identified with $\mathfrak{I}\otimes_A A'$ by flatness, and $\sh{I}'^n/\sh{I'}^{n+1}$ is equal to
  \begin{align*}
    \psi'^*(\sigma^*((\mathfrak{I}^n/\mathfrak{I}^{n+1})\supertilde) \otimes_{\sigma^*(\sh{O}_X)} \sh{O}_{X'}) =& \\
    \psi'^*(\sigma^*((\mathfrak{I}^n/\mathfrak{I}^{n+1} ))\supertilde) \otimes_{\psi'^*(\sigma^*(\sh{O}_X))} &\psi'^*(\sh{O}_{X'}) = \rho^*(\sh{I}^n/\sh{I}^{n+1})\otimes_{\rho^*(\psi^*(\sh{O}_X))} \psi'^*(\sh{O}_{X'}) 
  \end{align*}
  and in particular for $n = 0$, we have
  \[
    \sh{O}_{Y'} = \rho^*(\sh{O}_Y) \otimes_{\rho^*(\psi^*(\sh{O}_X))} \psi'^*(\sh{O}_{X'})
  \]
  from which we have canonical isomorphism of $\sh{I}'^n/\sh{I'}^{n+1}$ with
  \[
    \rho^*(\sh{I}^n/\sh{I}^{n+1})\otimes_{\rho^*(\sh{O}_Y)} \sh{O}_{Y'} = (\sh{I}^n/\sh{I}^{n+1}) \otimes_{\sh{O}_Y} \sh{O}_{Y'}
  \]
  which proves (iii).
  Let now $C_n = \Gamma(Y, \sh{O}_{Y^{(n)}}), C'_n = \Gamma(Y', \sh{O}_{Y'^{(n)}})$.
  As $Y^{(n)}$ and $Y'^{(n)}$ are affine schemes \sref{IV.16.1.5}, the kernel $\mathfrak{K}_n$ (resp. $\mathfrak{K}'_n$) of the homomorphism $C_n \to C_{n-1}$ (resp. $C'_n \to C'_{n-1}$) is $\Gamma(Y, \sh{I}^n/\sh{I}^{n+1})$ (resp. $\Gamma(Y, \sh{I}'^n/\sh{I'}^{n+1})$);
  therefore we can deduce from the above results that $\mathfrak{K}'_n = \mathfrak{K}_n \otimes_A A'$.
  Now, we have a commutative diagram
  \[
    \xymatrix{
      0 \ar[r] & \mathfrak{K}_n \ar[d]^-r\ar[r] \otimes_A A' & C_n \otimes_A A' \ar[d]^-{s_n}\ar[r] & C_{n-1} \otimes_A A' \ar[d]^-{s_{n-1}}\ar[r] & 0 \\
      0 \ar[r] & \mathfrak{K}'_n \ar[r] & C'_n \ar[r] & C'_{n-1} \ar[r] & 0
    }
  \]
  where the vertical arrow of the left is bijective and the two lines are exact ($A'$ being a flat $A$-module).
  We deduce by induction that $s_n$ is bijective for every $n$, because it is true by hypothesis for $n = 0$, and is deduced by application of the five lemma for all $n$.
  That proves the second assertion of (ii).
\end{enumerate}
\end{proof}

\begin{corollary}[16.2.3]
\label{IV.16.2.3}
Let $g: X \to Y$, $u: Y' \to Y$ be two morphisms of preschemes, $X' = X \times_Y Y'$, $g': X' \to Y'$ and $v: X' \to X$ by the canonical projections. Let $f: Y \to X$ by a $Y$-section of $X$ (and therefore an immersion), $f' = f_{(Y')}: Y' \to X'$ the $Y'$-section of $X'$ deduced from $f$ by the base change $u$.
We have:
\begin{enumerate}
  \item[{\rm(i)}] The morphism $w_n:{Y'_{f'}}^{(n)} \to Y_f^{(n)}$ corresponding to $f$, $f'$, $u$, $v$ \sref{IV.16.2.1} and the canonical morphism $h'_n: {Y'_{f'}}^{(n)} \to X'$ identifies $ {Y'_{f'}}^{(n)}$ with the product $Y_f^{(n)} \times_X X'$.
  \item[{\rm(ii)}] If we endow $\sh{O}_{Y_f^{(n)}}$ (resp. $\sh{O}_{{Y'_{f'}}^{(n)}}$) with the structure of an $\sh{O}_Y$-algebra defined by $g$ (resp. with the structure of an $\sh{O}_{Y'}$-algebra defined by $g'$ ) \sref{IV.16.1.5}[(iii)],
  % The original citation is IV.16.1.6, but he clearly meant 16.1.5 item (iii)
  then the homomorphism of $\sh{O}_{Y'}$-algebras
  \[
    \label{IV.16.2.3.1}
    \rho^*(\sh{O}_{Y_f^{(n)}})\otimes_{\sh{O}_Y} \sh{O}_{Y'} \to \sh{O}_{{Y'_{f'}}^{(n)}}
    \tag{16.2.3.1}
  \]
\oldpage[IV-4]{12}
  induced by the homomorphism $\nu_n$ \sref{IV.16.2.1.2} is bijective.
  Also, the homomorphism of $\sh{O}_{Y'}$-modules
  \[
    \label{IV.16.2.3.2}
    \Gr_1(u): \shGr_1(f)\otimes_{\sh{O}_Y} \sh{O}_{Y'} \to \shGr_1(f')
    \tag{16.2.3.2}
  \]
  is bijective.
 \end{enumerate} 
\end{corollary}

\begin{proof}
\medskip\noindent
\begin{enumerate}
  \item[(i)] Let us first note that $f': Y' \to X'$ and $u: Y' \to Y$  identifies $Y'$ with the product $Y \times_X X'$ (via the structure morphisms $f:Y \to X$ and $v: X' \to X$) \sref{IV.14.5.12.1}.
  The conclusion of (i) now follows from \sref{IV.16.2.2}[(ii)], the morphism $g$ being an immersion.
  \item[(ii)] The commutative diagram
  \[
  \xymatrix{
    Y_f^{(n)} \ar[d]^{h_n}  & {Y'_{f'}}^{(n)} \ar[d]^-{h'_n} \ar[l]^{w_n}\\  
    X         \ar[d]^{g}    & X'              \ar[d]^{g'} \ar[l]^v \\  
    Y                       & Y' \ar[l]^u \\  
  }
  \]
  identifies ${Y'_{f'}}^{(n)}$ with the product $Y_f^{(n)} \times_X X'$, so \sref[I]{I.3.3.9} it identifies (via the morphisms $g'\circ h'_n$ and $w_n$) ${Y'_{f'}}^{(n)}$ to the product $Y_f^{(n)} \times_Y Y'$.
  Since $Y_f^{(n)}$ (resp. ${Y'_{f'}}^{(n)}$) is the affine prescheme over $Y$ (resp. over $Y'$) associated with the $\sh{O}_Y$-algebra $\sh{O}_{Y_f^{(n)}}$ (resp. to the $\sh{O}_{Y'}$-algebra $\sh{O}_{{Y'_{f'}}^{(n)}}$), the fact that the canonical homomorphism \sref{IV.16.2.3.1} is bijective follows from \sref[II]{II.1.5.2}.
  Finally, the canonical homomorphism \sref{IV.16.2.3.1} is compatible with the augmentations $\sh{O}_{Y_f^{(n)}} \to \sh{O}_Y$ and $\sh{O}_{{Y'_{f'}}^{(n)}} \to \sh{O}_{Y'}$;
  since $\sh{O}_{Y_f^{(n)}}$ is a direct sum (as an $\sh{O}_Y$-module) of $\sh{O}_Y$ and the augmentation ideal $\sh{I}/\sh{I}^{n+1}$, we can therefore see that the canonical homomorphism \sref{IV.16.2.3.1}, restricted to $\sh{I}/\sh{I}^{n+1} \otimes_{\sh{O}_Y} \sh{O}_{Y'}$, is a bijection of the latter onto $\sh{I}'/\sh{I}'^{n+1}$. For $n=1$ this shows that $\Gr_1(u)$ is bijective.
\end{enumerate}
\end{proof}

We note that, under the hypotheses of \sref{IV.16.2.3}, the homomorphisms $\Gr_n(u)$ are \emph{surjective} in view of the above, but are not bijective in general for $n \geq 2$.
However:

\begin{corollary}[16.2.4]
\label{IV.16.2.4}
Under the hypotheses of \sref{IV.16.2.3}, suppose that $u: Y' \to Y$ is a flat morphism (resp. that the $\shGr_n(f)$ are flat $\sh{O}_Y$-modules for $n \leq m$).
Then the homomorphism 
\[
  \Gr_n(u):\shGr_n(f) \otimes_{\sh{O}_Y}\sh{O}_{Y'} \to \shGr_n(f')
\]
is bijective for all $n$ (resp. for $n \leq m$).
\end{corollary}

\begin{proof}
If $u$ is flat, then we deduce by base change that the same is true for $v:X' \to X$, and we already know in this case that $\Gr(u)$ is bijective \sref{IV.16.2.2}[(iii)].
If the $\shGr_n(f)$ are flat for $n\leq m$, then we first see by induction on $n$ that the same holds for $\sh{I}/\sh{I}^{n+1}$ for $n\leq m$, because of the exact sequences
  \[
    \xymatrix{
      0 \ar[r] & \sh{I}^n/\sh{I}^{n+1} \ar[r] & \sh{I}/\sh{I}^{n+1} \ar[r] & \sh{I}/\sh{I}^{n} \ar[r] & 0
    }
  \]
\oldpage[IV-4]{13}
  \sref[0]{0.6.1.2};
  in addition, we have the commutative diagram
  \[
    \xymatrix{
      0 \ar[r] & (\sh{I}^n/\sh{I}^{n+1}) \otimes_{\sh{O}_Y}\sh{O}_{Y'} \ar[d]\ar[r] &( \sh{I}/\sh{I}^{n+1}) \otimes_{\sh{O}_Y}\sh{O}_{Y'} \ar[d]\ar[r] & (\sh{I}/\sh{I}^{n}) \otimes_{\sh{O}_Y}\sh{O}_{Y'} \ar[d]\ar[r] & 0 \\
      0 \ar[r] & \sh{I'}^n/\sh{I'}^{n+1} \ar[r] & \sh{I'}/\sh{I'}^{n+1} \ar[r] & \sh{I'}/\sh{I'}^{n} \ar[r] & 0
    }
  \]
  in which the lines are exact (the first by flatness \sref[0]{0.6.1.2}) and the two last vertical arrows are bijective by virtue of \sref{IV.16.2.2}[(ii)];
  hence the conclusion.
\end{proof}

\begin{remarks}[16.2.5]
\label{IV.16.2.5}
\medskip\noindent
\begin{enumerate}
  \item[(i)] The reasoning of \sref{IV.16.2.2}[(i)] still applies to \sref{IV.16.2.1.1} when these are morphisms of \emph{locally ringed spaces} \sref[I]{I.1.8.2}.
  \item[(ii)] In \sref{IV.16.2.2}[(ii)], the conclusion is no longer necessarily valid if we only suppose that $v$ and $f$ are morphisms of preschemes ($f$ satisfying the condition of \sref{IV.16.1.1}).
  For example (with the notation of the proof of \sref{IV.16.2.2}[(ii)]), it can happen that $\mathfrak{I} = 0$ but the kernel $\mathfrak{I}'$ of $A' \to B' = B \otimes_A A'$ is not zero and that $B' \neq 0$, in which case we have $Y^{(n)} = Y$ for all $n$, but ${Y'}^{(n)} \neq Y'$.
  We have an example of this by taking $A = \bb{Z}$, $B = \bb{Q}$, $A' = \prod_{h = 1}^\infty (\bb{Z}/m^h\bb{Z})$ where $m>1$.
\end{enumerate}
\end{remarks}

\begin{env}[16.2.6]
\label{IV.16.2.6}
Consider the particular case of the diagram \sref{IV.16.2.1.1} where $X' = X$, $v$ is the identity, $X$ a prescheme, $Y$ a subprescheme of $X$, $Y'$ a subprescheme of $Y$, $f$, $u$, and $f' = f \circ u$ the canonical injections;
the di-homomorphism \sref{IV.16.2.1.3} gives us, by tensoring with $\sh{O}_{Y'}$ over $\rho^*(\sh{O}_Y)$, a homomorphism of graded $\sh{O}_{Y'}$-algebras
\[
  \label{IV.16.2.6.1}
  u^*(\shGr_\bullet(f)) \to \shGr_\bullet(f').
  \tag{16.2.6.1}
\]
On the other hand, we identify $\sh{O}_Y$ to $\psi^*(\sh{O}_X)/\sh{I}_f$ and $\sh{O}_{Y'}$ to $\rho^*(\sh{O}_Y)/\sh{I}_u$;
since $\rho^*$ is an exact functor, we have $\rho^*(\sh{O}_Y) = \rho^*(\psi^*(\sh{O}_X))/\rho^*(\sh{I}_f) = \psi'^*(\sh{O}_X)/\rho^*(\sh{I}_f)$, and since $\sh{O}_{Y'}$ is moreover identified with $\psi'^*{\sh{O}_X}/\sh{I}_{f'}$, we see that $\sh{I}_u = \sh{I}_{f'}/\rho^*(\sh{I}_f)$.
We deduce that for every integer $n$ there is a canonical homomorphism $\sh{I}_{f'}^n/\sh{I}_{f'}^{n+1} \to \sh{I}_{u}^n/\sh{I}_{u}^{n+1}$, from which we have a canonical morphism of graded $\sh{O}_{Y'}$-algebras
\[
  \label{IV.16.2.6.2}
  \shGr_\bullet(f') \to \shGr_\bullet(u).
  \tag{16.2.6.2}
\]
\end{env}

\begin{proposition}[16.2.7]
\label{IV.16.2.7}
Let $X$ be a prescheme, $Y$ a subprescheme of $X$, $Y'$ a subprescheme of $Y$, $j:Y' \to Y$ the canonical injection.
We then have an exact sequence of conormal sheaves ($\sh{O}_{Y'}$-modules)
\[
  \label{IV.16.2.7.1}
  \xymatrix{
    j^*(\sh{N}_{Y/X}) \ar[r] & \sh{N}_{Y'/X} \ar[r] & \sh{N}_{Y'/Y} \ar[r] & 0
  }
  \tag{16.2.7.1}
\]
where the arrows are the degree $1$ components of the canonical homomorphisms \sref{IV.16.2.6.1} and \sref{IV.16.2.6.2}.
\end{proposition}

\begin{proof}
The problem being local, we can restrict to the case where $X = \Spec(A)$, $Y = \Spec(A/\mathfrak{I})$ and $Y' = \Spec(A/\mathfrak{K})$, $\mathfrak{I}$ and $\mathfrak{K}$ being ideals of $A$ such that $\mathfrak{I} \subset \mathfrak{K}$;
everything reduces to seeing 
\oldpage[IV-4]{14}
that the sequence of canonical morphisms $\mathfrak{I}/\mathfrak{K}\mathfrak{I} \to \mathfrak{K}/\mathfrak{K}^2 \to (\mathfrak{K}/\mathfrak{I})/(\mathfrak{K}/\mathfrak{I})^2 \to 0$ is exact, which is immediate given that the image of $\mathfrak{I}/\mathfrak{K}\mathfrak{I}$ in $\mathfrak{K}/\mathfrak{K}^2$ is $(\mathfrak{I} + \mathfrak{K}^2)/\mathfrak{K}^2$ and that $(\mathfrak{K}/\mathfrak{I})/(\mathfrak{K}/\mathfrak{I})^2$ is identified with $\mathfrak{K}/(\mathfrak{I} + \mathfrak{K}^2)$.
\end{proof}

It is easy to give examples where the sequence \sref{IV.16.2.7.1} extended on the left by $0$ is not exact;
with the above notation, it suffices to take $A = k[T]$, $\mathfrak{I} = AT^2$, $\mathfrak{K} = AT$, because then $(\mathfrak{I} + \mathfrak{K}^2)/\mathfrak{K}^2 = 0$ and $\mathfrak{I}/\mathfrak{K}\mathfrak{I} \neq 0$.
See however \sref{IV.16.9.13} and \sref{IV.19.1.5} for some cases where the extended sequence is indeed exact.

\subsection{Fundamental differential invariants of morphisms of preschemes}
\label{IV.16.3}

\begin{definition}[16.3.1]
\label{IV.16.3.1}
Let $f:X \to S$ be a morphism of preschemes, $\Delta_f: X \to X \times_S X$ the corresponding diagonal morphism, which is an immersion \sref[I]{I.5.3.9}.
We denote by $\sh{P}_f^n$ or $\sh{P}_{X/S}^n$, and call the \emph{sheaf of principal parts of order $n$ of the $S$-prescheme $X$}, the $\sh{O}_X$-augmented sheaf of rings, $n$-th normal invariant of $\Delta_f$ \sref{IV.16.1.2}.
We will also write $\sh{P}_f^\infty = \sh{P}_{X/S}^\infty = \varprojlim_n \sh{P}_{X/S}^n$, $\shGr_n(\sh{P}_f) = \shGr_n(\sh{P}_{X/S}) = \shGr_n(\Delta_f)$ \sref{IV.16.1.2};
the $\sh{O}_X$-module $\shGr_1(\Delta_f)$, augmentation sheaf of ideals of $\sh{P}_{X/S}^1$, is denoted by $\Omega_f^1$ or $\Omega_{X/S}^1$, and is called the $\sh{O}_X$-module of \emph{$1$-differentials} of $f$, or of $X$ with respect to $S$, or of the $S$-prescheme $X$.
\end{definition}

It follows from this definition that $\sh{P}_{X/S}^0$ is canonically identified with $\sh{O}_X$ \sref{IV.16.1.2}.

We have \sref{IV.16.1.2.2} a canonical surjective morphism of graded $\sh{O}_X$-algebras 
\[
  \label{IV.16.3.1.1}
  \bb{S}_{\sh{O}_X}^\bullet(\Omega_{X/S}^1) \to \shGr_\bullet(\sh{P}_{X/S}).
  \tag{16.3.1.1}
\]
And it follows from Definition~\sref{IV.16.3.1} that for every open $U$ of $X$ we have $\sh{P}_{f|U}^n = \sh{P}_f^n|U$, $\sh{P}_{f|U}^\infty = \sh{P}_f^\infty|U$, $\shGr_n(\sh{P}_{f|U}) = \shGr_n(\sh{P}_f)|U$, $\Omega_{f|U}^1 = \Omega_f^1|U$ (in other words, the notions introduced are \emph{local} on $X$).

\begin{env}[16.3.2]
\label{IV.16.3.2}
Denote by $p_1$, $p_2$ the two canonical projections of the product $X \times_S X$;
since $\Delta_f$ is an $X$-section of $X \times_S X$ for both $p_1$ and $p_2$, \emph{each} of these morphisms define, for all $n$, a homomorphism of sheaves of rings $\sh{O}_X \to \sh{P}_{X/S}^n$, right inverse of the augmentation $\sh{P}_{X/S}^n \to \sh{O}_X$ \sref{IV.16.1.7};
we can also say that we thus define on $\sh{P}_{X/S}^n$ \emph{two} \emph{quasi-coherent augmented $\sh{O}_X$-algebra} structures;
the corresponding $\sh{O}_X$-module structures on on $\shGr_n(\sh{P}_{X/S}^n)$ are the same. 
We also have, by passing to the limit, two $\sh{O}_X$-algebra structures on $\sh{P}_{X/S}^\infty$.
\end{env}

\begin{env}[16.3.3]
\label{IV.16.3.3}
The morphism $s = (p_2, p_1)_S: X \times_S X \to X \times_S X$ is an \emph{involutive automorphism} of $X \times_S X$, called the \emph{canonical symmetry}, such that
\[
  \label{IV.16.3.3.1}
  p_1 \circ s = p_2, \qquad p_2 \circ s = p_1, \qquad s \circ \Delta_f = \Delta_f.
  \tag{16.3.3.1}
\]

If we put $s = (\rho, \lambda)$, $p_i = (\pi_i, \mu_i)$ ($i = 1,2$), $\Delta_f = (\delta, \nu)$, $\lambda^\#$ is then an isomorphism of $\rho^*(\pi_1^*(\sh{O}_X))$ onto $\pi_2^*(\sh{O}_X)$, and $\delta^*(\lambda^\#)$ fixes $\delta^*(\sh{O}_{X \times_S X})$ and the kernel $\sh{I}$ of the homomorphism $\nu^\#: \delta^*(\sh{O}_{X \times_S X}) \to \sh{O}_X$.
Therefore:
\end{env}

\begin{proposition}[16.3.4]
\label{IV.16.3.4}
The homomorphism $\sigma = \delta^*(\lambda^\#)$ induced from $s$ (and also called the \emph{canonical symmetry}) is an involutive automorphism of the projective system $(\sh{P}_{X/S}^n)$ of $\sh{O}_X$-augmented 
\oldpage[IV-4]{15}
sheaves of rings, and as a result also of the projective limit $\sh{P}_{X/S}^\infty$.
This automorphism permutes the $\sh{O}_X$-algebra structure on $\sh{P}_{X/S}^n$ and on $\sh{P}_{X/S}^\infty$.
\end{proposition}

\begin{env}[16.3.5]
\label{IV.16.3.5}
In what follows, the two $\sh{O}_X$-algebra structures defined on the $\sh{P}_{X/S}^n$ and on $\sh{P}_{X/S}^\infty$ will play very different roles:
\emph{we will now agree, unless said otherwise, that when $\sh{P}_{X/S}^n$ or $\sh{P}_{X/S}^\infty$ is considered as an $\sh{O}_X$-algebra, it is the algebra structure induced by $p_1$}.
\end{env}

For every open $U$ of $X$ and every section $t \in \Gamma(U, \sh{O}_X)$, we will simply denote by $t.1$ or even $t$ the image of $t$ under the structure morphism $\Gamma(U, \sh{O}_X) \to \Gamma(U, \sh{P}_{X/S}^n)$ (resp. $\Gamma(U, \sh{O}_X) \to \Gamma(U, \sh{P}_{X/S}^\infty)$) (that is to say, the homomorphism corresponding to $p_1$).

\begin{definition}[16.3.6]
\label{IV.16.3.6}
We denote by $d_f^n$, or $d_{X/S}^n$ (resp. $d_f^\infty$, or $d_{X/S}^\infty$), or simply $d^n$ (resp. $d^\infty$), the homomorphism of sheaves of rings $\sh{O}_X \to \sh{P}_{f}^n = \sh{P}_{X/S}^n$ (resp. $\sh{O}_X \to \sh{P}_{f}^\infty = \sh{P}_{X/S}^\infty$) induced by $p_2$.
For every open $U$ of $X$, and every $t \in \Gamma(U, \sh{O}_X)$, $d^n t$ (resp. $d^\infty t$) is called the \emph{principal part of order $n$} (resp. \emph{principal part of infinite order}) of $t$.
We set $dt = d^1 t - t$, and we say that $dt$ is the differential of $t$ (an element of $\Gamma(U, \Omega_{X/S}^1)$, also denoted $d_{X/S}(t)$).
\end{definition}  

It follows immediately
\footnote{
[Trans.] This is, locally we have \sref[0]{0.20.1.1}.
}
from this definition that we have
\[
  \label{IV.16.3.6.1}
  d(t_1 t_2) = t_1 dt_2 + t_2 dt_1
  \tag{16.3.6.1} 
\]
for every $t_1$, $t_2$ in $\Gamma(U \sh{O}_X)$, that is, $d$ is a \emph{derivation} of the ring $\Gamma(U, \sh{O}_X)$ in the $\Gamma(U, \sh{O}_X)$-module $\Gamma(U, \Omega_{X/S}^1)$.

In all notation introduced in \sref{IV.16.3.1} and \sref{IV.16.3.6}, we will sometimes replace $S$ by $A$ when $S = \Spec(A)$.

\begin{env}[16.3.7]
\label{IV.16.3.7}
Suppose in particular that $S = \Spec(A)$ and $X = \Spec(B)$ are affine schemes, $B$ then being an $A$-algebra.
Then $\Delta_f$ corresponds to the canonical surjective homomorphism $\pi: B \otimes_A B \to B$ such that $\pi(b\otimes b') = bb'$, with kernel $\mathfrak{I} = \mathfrak{I}_{B/A}$ \sref[0]{0.20.4.1};
$\sh{P}_{f}^n$ is the structure sheaf of the prescheme $\Spec(P_{B/A}^n)$, where
\[
  P_{B/A}^n = (B \otimes_A B)/\mathfrak{I}^{n+1};
\]
$\shGr_\bullet(\sh{P}_f)$ is the quasi-coherent $\sh{O}_X$-module corresponding to the graded $B$-module
\[
  \gr_\mathfrak{I}^\bullet(B \otimes_A B) = \bigoplus_{n \geq 0} (\mathfrak{I}^n/\mathfrak{I}^{n+1});
\]
in particular $\Omega_f^1 = \Omega_{X/S}^1$ is the quasi-coherent $\sh{O}_X$-module corresponding to the $B$-module of $1$-differentials of $B$ over $A$, $\Omega_{B/A}^1$ \sref[0]{0.20.4.3}.
The projection morphisms $p_1: X \times_S X \to X$, $p_2: X \times_S X \to X$ corresponding to the two homomorphisms of rings $j_1: B \to B \otimes_A B$, $j_2: B \to B \otimes_A B$ such that $j_1(b) = b \otimes 1$, $j_2(b) = 1 \otimes b$, so that (by the convention of \sref{IV.16.3.5}), $P_{B/A}^n$ is always considered as a $B$-algebra via the composite homomorphism $B \xrightarrow{j_1} B \otimes_A B \to P_{B/A}^n$;
the ring homomorphism $B \xrightarrow{j_2} B \otimes_A B \to P_{B/A}^n$ is denoted by $d_{B/A}^n$ and corresponds to $d_{X/S}^n$ acting on $\Gamma(X, \sh{O}_X)$;
for every $t \in B$, $dt$ is equal to $d_{B/A}t$, defined in \sref[0]{0.20.4.6}.

If $\pi_n: B \otimes_A B \to P_{B/A}^n$ is the canonical homomorphism, so we have, in light of the preceding definitions,
\[
  \label{IV.16.3.7.1}
  \pi_n(b\otimes b') = b \cdot \pi_n(1\otimes b') = b \cdot d_{B/A}^n(b') \quad \text{for } b \in B, b' \in B.
  \tag{16.3.7.1}
\]
\end{env}

\begin{proposition}[16.3.8]
\label{IV.16.3.8}
\oldpage[IV-4]{16}
The image of the canonical homomorphism $d_{X/S}^n: \sh{O}_X \to \sh{P}_{X/S}^n$ generates the $\sh{O}_X$-module $\sh{P}_{X/S}^n$.
\end{proposition}

\begin{proof}
We immediately reduce to the case where $X = \Spec(B)$ and $S = \Spec(A)$ are affine and the proposition follows from \sref{IV.16.3.7.1} since $\pi_n$ is surjective.
We note that in general $d_{X/S}^n$ \emph{is not surjective} (even for $n = 1$).
\end{proof}

\begin{proposition}[16.3.9]
\label{IV.16.3.9}
Suppose that $f:X \to S$ is a morphism locally of finite type.
Then the $\sh{P}_{f}^n$ and the $\shGr_n(\sh{P}_{f})$ are quasi-coherent $\sh{O}_X$-modules of finite type.
\end{proposition}

\begin{proof}
This follows from \sref{IV.16.1.6} and from the fact that $\Delta_f$ is locally of finite presentation \sref[1]{1.4.3.1}.
\end{proof}

\subsection{Functorial properties of differential invariants}
\label{IV.16.4}

\begin{env}[16.4.1]
\label{IV.16.4.1}
Consider a commutative diagram of morphisms of preschemes
\[
  \label{IV.16.4.1.1}
  \xymatrix{
    X \ar[d]_-{f} & X' \ar[l]_-u \ar[d]^-{f'}\\
    S & S' \ar[l]^-w
  }
  \tag{16.4.1.1}
\]
We deduce a commutative diagram
\[
  \xymatrix{
    X \ar[d]_-{\Delta_f} & X' \ar[l]_-u \ar[d]^-{\Delta_{f'}}\\
    X \times_S X & X' \times_{S'} X' \ar[l]^-v
  }
\]
where $v$ is the composite homomorphism \sref[I]{I.5.3.5} and \sref[I]{I.5.3.15}.
\[
  \label{IV.16.4.1.2}
  X' \times_{S'} X' \xrightarrow{(p'_1, p'_2)_S} X' \times_{S} X' \xrightarrow{u \times_S u} X \times_S X.
  \tag{16.4.1.2}
\]

So we induce from $u$ and $v$, as explained in \sref{IV.16.2.1}, homomorphisms of augmented sheaves of rings
\[
  \label{IV.16.4.1.3}
  \nu_n: \rho^*(\sh{P}_{X/S}^n) \to \sh{P}_{X'/S'}^n
  \tag{16.4.1.3}
\]
(where we put $u = (\rho,\lambda)$);
these homomorphisms form a projective system, and therefore give at the limit a homomorphism of sheaves of graded rings
\[
  \label{IV.16.4.1.4}
  \nu_\infty: \rho^*(\sh{P}_{X/S}^\infty) \to \sh{P}_{X'/S'}^\infty;
  \tag{16.4.1.4}
\]
on the other hand, by passing to the quotient, the homomorphisms $\nu_n$ give rise to a di-homomorphism of graded algebras (relative to $\lambda^\#$):
\[
  \label{IV.16.4.1.5}
  \gr(u): \rho^*(\shGr_\bullet(\sh{P}_{X/S})) \to \shGr_\bullet(\sh{P}_{X'/S'}).
  \tag{16.4.1.5}
\]
\end{env}

\begin{env}[16.4.2]
\label{IV.16.4.2}
\oldpage[IV-4]{17}
If we have a commutative diagram
\[
  \xymatrix{
    X \ar[d]_-{f} & X' \ar[l]_-u \ar[d]^-{f'} & X'' \ar[l]_-{u'} \ar[d]^-{f''}\\
    S & S' \ar[l]^-w & S'' \ar[l]^-{w'} 
  }
\]
we deduce a commutative diagram
\[
  \xymatrix{
    X \ar[d]_-{\Delta_f} & X' \ar[l]_-u \ar[d]^-{\Delta_{f'}} & X'' \ar[l]_-{u'} \ar[d]^-{\Delta_{f''}}\\
    X \times_S X & X' \times_{S'} X' \ar[l]^-v & X'' \times_{S''} X'' \ar[l]^-{v'}
  }
\]
where $v'$ is defined from $u'$, $w'$, $f'$, $f''$ as $v$ is from $u$, $w$, $f$, $f'$.
We verify immediately that if $u'' = u \circ u'$, $w'' = w \circ w'$, then the composite homomorphism $v \circ v'$ is equal to the homomorphism $v''$ deduced from $u''$, $v''$, $f$, $f''$ as $v$ is from $u$, $w$, $f$, $f'$.
If we put $u' = (\rho', \lambda')$, $u'' = (\rho'', \lambda'')$ it follows \sref{IV.16.2.1} that the homomorphism $\nu''_n: {\rho''}^*(\sh{P}_{X/S}^n) \to \sh{P}_{X''/S''}^n$ is equal to the composite
\[
  {\rho'}^*(\rho^*(\sh{P}_{X/S}^n)) \xrightarrow{{\rho'}^*(\nu_n^\#)} {\rho'}^*(\sh{P}_{X'/S'}^n) \xrightarrow{\nu'_n} \sh{P}_{X''/S''}^n,
\]
and we have analogous transitivity properties for the homomorphisms \sref{IV.16.4.1.4} and \sref{IV.16.4.1.5}, which lets us say that the $\sh{P}_{X/S}^n$, $\sh{P}_{X/S}^\infty$ and $\shGr_\bullet(\sh{P}_{X/S})$ \emph{depend functorially on $f$}. 
\end{env}

\begin{env}[16.4.3]
\label{IV.16.4.3}
We verify immediately (for example, by restricting ourselves to the affine case with help of \sref{IV.16.3.7}) that with the notation of \sref{IV.16.4.1}, the diagram
\[
  \label{IV.16.4.3.1}
  \xymatrix{
    \rho^*(\sh{O}_X) \ar[r]^-{\lambda^\#} \ar[d] & \sh{O}_{X'} \ar[d] \\
    \rho^*(\sh{P}_{X/S}^n) \ar[r]_-{\nu_n} & \sh{P}_{X'/S'}^n  
  }
  \tag{16.4.3.1}
\]
where the vertical arrows are the ones defining the algebra structure chosen in \sref{IV.16.3.5} (that is to say, the ones coming from the first projections) is commutative;
the same goes for the diagram
\[
  \label{IV.16.4.3.2}
  \xymatrix{
    \rho^*(\sh{O}_X) \ar[r]^-{\lambda^\#} \ar[d]_-{\rho^*(d_{X/S}^n)} & \sh{O}_{X'} \ar[d]^-{d_{X'/S'}^n} \\
    \rho^*(\sh{P}_{X/S}^n) \ar[r]_-{\nu_n} & \sh{P}_{X'/S'}^n  
  }
  \tag{16.4.3.2}
\]
\oldpage[IV-4]{18}
the vertical arrows defining here the algebra structure from the second projection;
besides, if $\sigma$ and $\sigma'$ are the canonical symmetries corresponding to $f$ and $f'$ \sref{IV.16.3.4}, we have
\[
  \nu_n \circ \rho^*(\sigma) = \sigma' \circ \nu_n 
\] 
which switches one diagram with the other.
We deduce from \sref{IV.16.4.3.1} a canonical homomorphism of \emph{augmented $\sh{O}_{X'}$-algebras}
\[
  \label{IV.16.4.3.3}
  P^n(u): u^*(\sh{P}_{X/S}^n) = \sh{P}_{X/S}^n \otimes_{\sh{O}_X} \sh{O}_{X'} \to \sh{P}_{X'/S'}^n 
  \tag{16.4.3.3}
\]
and it follows from \sref{IV.16.4.3.2} that the diagram 
\[
  \label{IV.16.4.3.4}
  \xymatrix{
    \sh{O}_{X'} \ar[r]^-{\operatorname{id}} \ar[d]_-{u^*(d_{X/S}^n)}  & \sh{O}_{X'} \ar[d]^-{d_{X'/S'}^n} \\
    u^*(\sh{P}_{X/S}^n) \ar[r]_{P^n(u)} & \sh{P}_{X'/S'}^n
  }
  \tag{16.4.3.4}
\]
is commutative.
We deduce a homomorphism of \emph{graded $\sh{O}_{X'}$-algebras}
\[
  \label{IV.16.4.3.5}
  \Gr_\bullet(u):u^*(\shGr_\bullet(\sh{P}_{X/S})) \to \shGr_\bullet(\sh{P}_{X'/S'})
  \tag{16.4.3.5}
\]
and in particular a homomorphism of $\sh{O}_{X'}$-modules 
\[
  \label{IV.16.4.3.6}
  \Gr_1(u):\Omega_{X/S}^1 \otimes_{\sh{O}_X} \sh{O}_{X'} \to \Omega_{X'/S'}^1
  \tag{16.4.3.6}
\]
giving rise to a commutative diagram
\[
  \label{IV.16.4.3.7}
  \xymatrix{
    \sh{O}_{X'} \ar[r]^-{\operatorname{id}} \ar[d]_-{d_{X/S} \otimes 1}  & \sh{O}_{X'} \ar[d]^-{d_{X'/S'}} \\
    \Omega_{X/S}^1 \otimes_{\sh{O}_X} \sh{O}_{X'} \ar[r] & \Omega_{X'/S'}^1
  }
  \tag{16.4.3.7}
\]
\end{env}

\begin{env}[16.4.4]
\label{IV.16.4.4}
When $S = \Spec(A)$, $S' = \Spec(A')$, $X = \Spec(B)$, $X' = \Spec(B')$ are affine, so that we have a commutative diagram of ring homomorphisms
\[
  \xymatrix{
    B \ar[r] & B' \\
    A \ar[u] \ar[r] & A' \ar[u]
  }
\]
the image of $\mathfrak{I}_{B/A}$ in $B' \otimes_{A'} B'$ is contained in $\mathfrak{I}_{B'/A'}$, and the homomorphism $\nu_n$ corresponds to the homomorphism of rings $P_{B/A}^n \to P_{B'/A'}^n$ induced from the homomorphism $B \otimes_A B \to B' \otimes_{A'} B'$ by passing to quotients.
The homomorphism \sref{IV.16.4.3.6} corresponds to the homomorphism defined in \sref[0]{0.20.5.4.1}, and the commutative diagram \sref{IV.16.4.3.7} to the diagram \sref[0]{0.20.5.4.2}.
\end{env}

\begin{proposition}[16.4.5]
\label{IV.16.4.5}
\oldpage[IV-4]{19}
Suppose that $X' = X \times_S S'$, $f'$ and $u$ the canonical projections.
Then the canonical homomorphisms $P^n(u)$ \sref{IV.16.4.3.3} and $\Gr_1(u)$ \sref{IV.16.4.3.6} are bijective.
\end{proposition}

\begin{proof}
We have $X' \times_{S'} X' = (X \times_S X) \times_S S'$, and it suffices to apply \sref{IV.16.2.3}[(ii)] replacing $g$ by the first $p_1:X\times_S X \to X$ and $f$ by the diagonal $\Delta_f$.
\end{proof}

We note that under the hypotheses of \sref{IV.16.4.5} the homomorphism $\Gr_\bullet(u)$ \sref{IV.16.4.3.5} is \emph{surjective}, but not bijective in general. 
However \sref{IV.16.2.4}:

\begin{corollary}[16.4.6]
\label{IV.16.4.6}
Under the hypotheses of \sref{IV.16.4.5}, suppose in addition that $w: S \to S'$ is flat (resp. that $\shGr_n(\sh{P}_{X/S}^n)$ are flat $\sh{O}_X$-modules for $n \leq m$);
then the homomorphism
\[
  \Gr_n(u):u^*(\shGr(\sh{P}_{X/S}^n)) \to \shGr(\sh{P}_{X'/S'}^n)
\]
is bijective for each $n$ (resp. for $n \leq m$).
\end{corollary}

\begin{proof}
Indeed, if $w$ is flat, then so is $v: X' \times_{S'} X' \to X \times_S X$, so the conclusion follows from \sref{IV.16.2.4}.
\end{proof}

\begin{env}[16.4.7]
\label{IV.16.4.7}
Let $S$ be a prescheme, $\sh{E}$ a quasi-coherent $\sh{O}_S$-Module, and set $X = \bb{V}(\sh{E})$ \sref[II]{II.1.7.8}, the vector bundle associated to $\sh{E}$, equal to $\Spec(\bb{S}_{\sh{O}_S}(\sh{E}))$.
Let $f:X \to S$ be the structure morphism.
For every open $U$ of $S$ and every section $t \in \Gamma(U, \sh{E})$, $t$ is identified with a section of $\bb{S}_{\sh{O}_S}(\sh{E})$ over $U$;
let $t'$ be its image in $\Gamma(f^{-1}(U), \sh{O}_X) = \Gamma(U, f_*(\sh{O}_X)) = \Gamma(U, \bb{S}_{\sh{O}_S}(\sh{E}))$, and set
\[
  \label{IV.16.4.7.1}
  \delta(t) = d_{X/S}^n(t') - t' \in \Gamma(f^{-1}(U), \sh{P}_{X/S}^n);
  \tag{16.4.7.1}
\]
it is clear that $\delta$ is a di-homomorphism of modules (corresponding to the homomorphism of rings $\Gamma(U, \sh{O}_S) \to \Gamma(f^{-1}(U), \sh{O}_X)$) of $\Gamma(U, \sh{E})$ into $\Gamma(f^{-1}(U), \sh{P}_{X/S}^n)$, and therefore the image belongs to the augmentation ideal of $\Gamma(f^{-1}(U), \sh{P}_{X/S}^n)$.
We deduce (by varying $U$) a canonical homomorphism of $\sh{O}_X$-algebras
\[
  \label{IV.16.4.7.2}
  f^*(\bb{S}_{\sh{O}_S}(\sh{E})) \to \sh{P}_{X/S}^n
  \tag{16.4.7.2}
\]
and in view of the above remark, if $\sh{K}$ is the ideal kernel of augmentation $\bb{S}_{\sh{O}_S}(\sh{E}) \to \sh{O}_S$, the image of $\sh{K}^{n+1}$ by \sref{IV.16.4.7.2} is zero, so that by factoring by $\sh{K}^{n+1}$, we finally have a canonical homomorphism
\[
  \label{IV.16.4.7.3}
  \delta_n: f^*(\bb{S}_{\sh{O}_S}(\sh{E})/\sh{K}^{n+1}) \to \sh{P}_{X/S}^n.
  \tag{16.4.7.3}
\]
\end{env} 

\begin{proposition}[16.4.8]
\label{IV.16.4.8}
Under the conditions of \sref{IV.16.4.7}, the homomorphisms $\delta_n$ are bijective and form a projective system of isomorphisms;
we deduce an isomorphism of graded $\sh{O}_S$-algebras 
\[
  \label{IV.16.4.8.1}
  f^*(\bb{S}_{\sh{O}_S}^\bullet(\sh{E})) \to \shGr_\bullet(\sh{P}_{X/S}).
  \tag{16.4.8.1}
\]
\end{proposition}

\begin{proof}
The fact that homomorphisms \sref{IV.16.4.7.3} form a projective system follows immediately from their definition.
To prove they are isomorphisms, it suffices to 
\oldpage[IV-4]{20}
prove that \sref{IV.16.4.8.1} is an isomorphism, since both filtrations involved in \sref{IV.16.4.7.3} are finite (Bourbaki, \emph{Alg. comm.}, chap.~III, \textsection2, no 8, cor. 3 of th. ~1).
To do this, consider the split exact sequence of $\sh{O}_S$-modules
\[
  \label{IV.16.4.8.2}
  \xymatrix{
    0 \ar[r] & \sh{E} \ar[r]^-u & \sh{E} \oplus \sh{E} \ar[r]^-v & \sh{E} \ar[r] & 0
  }
  \tag{16.4.8.2}
\] 
where, for every pair of sections $s$, $t$ of $\sh{E}$ over an open $U$ of $S$, we take $u(s) = (-s, s)$ and $v(s,t) = s + t$.
We have
\[
  X \times_S X = \Spec(\bb{S}_{\sh{O}_S}(\sh{E}) \otimes_{\sh{O}_S} \bb{S}_{\sh{O}_S}(\sh{E})) = \Spec(\bb{S}_{\sh{O}_S}(\sh{E} \oplus \sh{E}))
\]
(\sref[II]{II.1.4.6} and \sref[II]{II.1.7.11}), and the diagonal morphism $X \to X \times_S X$ corresponds \sref[II]{II.1.2.7} to the homomorphism of $\sh{O}_X$-algebras $\bb{S}(v): \bb{S}_{\sh{O}_S}(\sh{E} \oplus \sh{E}) \to \bb{S}_{\sh{O}_S}(\sh{E})$ \sref[II]{II.1.7.4}, such that if $\sh{I}$ is the kernel of this homomorphism, then we have
\[
  \sh{P}_{X/S}^n = f^*(\bb{S}_{\sh{O}_S}(\sh{E} \oplus \sh{E})/\sh{I}^{n+1}).
\]
The proposition now will be a consequence of the following lemma:
\end{proof}

\begin{lemma}[16.4.8.3]
\label{IV.16.4.8.3}
Let Y be a ringed space, $0 \to \sh{F}'  \xrightarrow{u} \sh{F} \xrightarrow{v} \sh{F}'' \to 0$ an exact sequence of $\sh{O}_Y$ modules such that each point $y \in Y$ has an open neighbourhood $V$ such that the sequence $0 \to \sh{F}'|V \to \sh{F}|V \to \sh{F}''|V \to 0$ is split.
Let $\sh{I}$ be the kernel ideal of $\bb{S}(v)$:
\[
  \bb{S}_{\sh{O}_Y}(\sh{F}) \to \bb{S}_{\sh{O}_Y}(\sh{F}''),
\]
and let $\gr_{\sh{I}}^\bullet(\bb{S}_{\sh{O}_Y}(\sh{F}))$ be the graded $\sh{O}_Y$-algebra associated to the $\sh{O}_Y$-algebra $\bb{S}_{\sh{O}_Y}(\sh{F})$ \emph{endowed with the $\sh{I}$-preadic filtration}.
Then the homomorphism of graded $\sh{O}_Y$-algebras
\[
  \label{IV.16.4.8.4}
  \bb{S}_{\sh{O}_Y}^\bullet(\sh{F}') \otimes_{\sh{O}_Y} \bb{S}_{\sh{O}_Y}^\bullet(\sh{F}'') \to \gr_{\sh{I}}^\bullet(\bb{S}_{\sh{O}_Y}(\sh{F}))
  \tag{16.4.8.4}
\]
(where the first member is the graded tensor product of symmetric $\sh{O}_Y$-algebras endowed with the canonical gradation \sref[II]{II.1.7.4} and \sref[II]{II.2.1.2}), induced by the canonical injection
\[
  \sh{F}' \to \sh{I} = \gr_{\sh{I}}^1(\bb{S}_{\sh{O}_Y}(\sh{F})),
\]
is bijective.
\end{lemma}

\begin{proof}
The injection $\sh{F}' \to \sh{I}$ indeed canonically gives a homomorphism of graded $\sh{O}_Y$-algebras $\bb{S}_{\sh{O}_Y}^\bullet(\sh{F}') \to \gr_\sh{I}^\bullet(\bb{S}_{\sh{O}_Y}^\bullet(\sh{F}))$, and since the second member is by definition a graded $\bb{S}_{\sh{O}_Y}^\bullet(\sh{F}'')$-algebra, we induce the canonical homomorphism \sref{IV.16.4.8.4} by tensoring the above with $\bb{S}_{\sh{O}_Y}^\bullet(\sh{F}'')$.
To prove the lemma we can, being a local problem, restrict to the case where $\sh{F} = \sh{F}' \oplus \sh{F}''$, $u$ and $v$ the canonical homomorphisms.
Then the graded algebra $\bb{S}_{\sh{O}_Y}^\bullet(\sh{F})$ is canonically identified with the graded tensor product $\bb{S}_{\sh{O}_Y}^\bullet(\sh{F}') \otimes_{\sh{O}_Y} \bb{S}_{\sh{O}_Y}^\bullet(\sh{F}'')$ \sref[II]{II.1.7.4}, and it is immediate that $\sh{I}$ is therefore the ideal $\sh{I}' \otimes_{\sh{O}_Y} \bb{S}_{\sh{O}_Y}^\bullet(\sh{F}'')$, where $\sh{I}'$ is the augmentation ideal of $\bb{S}_{\sh{O}_Y}^\bullet(\sh{F}')$, that is to say the (direct) sum of the $\bb{S}_{\sh{O}_Y}^m(\sh{F}')$ for $m \geq 1$.
We conclude that $\sh{I}^n = \sh{I}'^n \otimes_{\sh{O}_Y} \bb{S}_{\sh{O}_Y}^\bullet(\sh{F}'')$, where this time $\sh{I}'^n$ is the direct sum of the $\bb{S}_{\sh{O}_Y}^m(\sh{F}')$ for $m \geq n$;
we have therefore $\sh{I}^n/\sh{I}^{n+1} = \bb{S}_{\sh{O}_Y}^n(\sh{F}) \otimes_{\sh{O}_Y} \bb{S}_{\sh{O}_Y}^\bullet(\sh{F}'')$, which proves that \sref{IV.16.4.8.4} is bijective.
\end{proof}

\oldpage[IV-4]{21}
Having proved the lemma, it remains to see that the homomorphism \sref{IV.16.4.8.1} is the image by $f^*$ of the homomorphism \sref{IV.16.4.8.4} corresponding to the exact sequence \sref{IV.16.4.8.2};
we can easily see that it follows from the definition of $u$ \sref{IV.16.4.8.2} and of $\delta$ \sref{IV.16.4.7.1}, given the definition of the $\sh{O}_X$-algebra structures of $\sh{P}_{X/S}^n$ and of the $d_{X/S}^n$ \sref{IV.16.3.5} and \sref{IV.16.4.3.6}.

In particular:

\begin{corollary}[16.4.9]
\label{IV.16.4.9}
Under the conditions of \sref{IV.16.4.7}, we have a canonical isomorphism
\[
\label{IV.16.4.9.1}
  \gr_1(\delta): f^*(\sh{E}) \isoto \Omega_{X/S}^1.
  \tag{16.4.9.1}
\]
\end{corollary}

\begin{corollary}[16.4.10]
\label{IV.16.4.10}
If $S = \Spec(A)$, $\sh{E} = \sh{O}_S^m$, so that
\[
  X = \Spec(A[T_1, \dots, T_m]),
\]
then $\sh{P}_{X/S}^n$ is canonically identified with the $\sh{O}_X$-algebra corresponding to the quotient $A[T_1, \dots, T_m]$-algebra\\
$A[T_1, \dots, T_m, U_1, \dots, U_m]/\mathfrak{K}^{n+1}$, where the $U_i$ ($1 \leq i \leq m$) are $m$ new indeterminates and $\mathfrak{K}$ is the ideal generated by $U_1, \dots, U_m$.
\end{corollary}

We thus recover in particular the structure of $\Omega_{X/S}^1$ in this case \sref[0]{0.20.5.13}.

In addition, note that the $d_{X/S}^n$ then corresponds to a polynomial $F(T_1, \dots, T_m)$, the class modulo $\mathfrak{K}^{n+1}$ of $F(T_1 + U_1, \dots, T_m + U_m)$, which follows from the definition \sref{IV.16.4.7.1}.

\begin{proposition}[16.4.11]
\label{IV.16.4.11}  
Let $f:X \to S$ be a morphism, $g: S \to X$ a $S$-section of $X$, $S^{(n)}$ the $n$-th infinitesimal neighbourhood of $S$ by the immersion $g$ \sref{IV.16.1.2}.
Then there exists a unique isomorphism of $\sh{O}_S$-algebras
\[
  \label{IV.16.4.11.1}
  \varpi_n: g^*(\sh{P}_{X/S}^n) \to \sh{O}_{S_g^{(n)}}
  \tag{16.4.11.1}
\]  
(via the $\sh{O}_S$-algebra structure on $\sh{O}_{S_g^{(n)}}$ defined by $f$ \sref{IV.16.1.7}), making the diagram
\[
  \label{IV.16.4.11.2}
  \xymatrix{
    \sh{O}_S = g^*(\sh{O}_X) \ar[rr]^-{\lambda_n} \ar[dr]_-{g^*(d_{X/S}^n)} && \sh{O}_{S_g^{(n)}} \\
    & g^*(\sh{P}_{X/S}^n) \ar[ur]_-{\varpi_n}
  }
  \tag{16.4.11.2}
\]
commutative (where $\lambda_n$ is the structure morphism).
\end{proposition} 

\begin{proof}
In light of \sref[I]{I.5.3.7}, where we replace $X$, $Y$, $S$ by $S$, $X$, $S$ respectively and $f$ by $g$, the diagrams
\[
  \label{IV.16.4.11.3}
  \xymatrix@=3pc{
    S \ar[r]^-g \ar[d]_-g &  X \ar[d]^-{\Delta_f} & S \ar[r]^-g  \ar[d]_-g & X \ar[d]^-{\Delta_f}  \\
    X \ar[r]_{(g\circ f, 1_X)_S} & X\times_S X & X \ar[r]_{(1_X, g\circ f)_S} & X \times_S X
  }
  \tag{16.4.11.3}
\]
\oldpage[IV-4]{22}
identifies $S$ with the product of the $(X \times_S X)$-preschemes $X$ and $X$ by the morphisms $\Delta_f$ and $(g\circ f, 1_X)_S$ (resp. $(1_X, g\circ f)_S$).
On the other hand, the diagrams
\[
  \label{IV.16.4.11.4}
  \xymatrix@=3pc{
    X \ar[r]^{(g\circ f, 1_X)_S} \ar[d]_-f &  X \ar[d]^-{p_1} &  X \ar[r]^{(1_X, g\circ f)_S} \ar[d]_-f &  X \ar[d]^-{p_2} \\
    S \ar[r]_{g} & X & S \ar[r]_{g} & X
  }
  \tag{16.4.11.4}
\]
identify $X$ to the product of $X$-preschemes $S$ and $X \times_S X$ via the morphisms $g$ and $p_1$ (resp. $p_2$) (particular case of the associativity formula \sref[I]{I.3.3.9.1}).
We can say that $\Delta_f$, considered as an $X$-section of $X \times_S X$ (relative to $p_1$ or $p_2$) plays the role of a \emph{universal section} for the $S$-sections of $X$:
each of these sections $g$ in fact are deduced by \emph{base change} $(g \circ f, 1_X)_S: X \to X \times_S X$.
The definition of the homomorphism $\bar\omega_n$ and the fact that it is bijective follows from the remarks of \sref{IV.16.2.3}[(ii)] applied to the first diagram \sref{IV.16.4.11.4}.
The commutativity of the first diagram \sref{IV.16.4.11.4} follows also from \sref{IV.16.2.3}[(ii)] this time applied to the second diagram \sref{IV.16.11.4}.
To explain $\varpi_n$, we can restrict ourselves to the case where $g$ is a closed immersion:
Indeed, for every $s\in S$, there is an open neighbourhood $W$ of $s$ in $S$ such that $g(W)$ is closed in an open set $U$ of $X$, and it is clear that $g|W$ is a $W$-section of the morphism $U \cap f^{-1}(W)$.
We can then suppose that $S$ is a closed subprescheme of $X$ defined by a quasi-coherent ideal $\sh{K}$.
Then the preceding definitions show that if $W$ is an open of $S$, $t$ is a section of $\sh{O}_X$ over $f^{-1}(W)$, $\varpi_n(d^n t|W)$ is equal to the canonical image of $t$ in $\Gamma(W, (\sh{O}_X/\sh{K}^{n+1})|W)$. 
The uniqueness of $\varpi_n$ then follows since the image of $\sh{O}_X$ under $d_{X/S}^n$ generates the $\sh{O}_X$-module $\sh{P}_{X/S}^n$ \sref{IV.16.3.8}.
\end{proof}

\begin{corollary}[16.4.12]
\label{IV.16.4.12}
Let $k$ be a field, $X$ a $k$-prescheme, $x$ a point of $X$ \emph{rational over $k$}.
Then $(\sh{P}_{X/S}^n)_x \otimes_{\sh{O}_x} \kres(x)$ is canonically isomorphic (as an augmented $\kres(x)$-algebra) to $\sh{O}_x/\mathfrak{m}_x^{n+1}$.
\end{corollary}

\begin{proof}
It suffices to consider the unique $k$-section $g$ of $X$ such that $g(\Spec(k)) = \{x\}$.
\end{proof}

\begin{corollary}[16.4.13]
\label{IV.16.4.13}
Let $f: X \to S$ be a morphism, $s$ a point of $S$, $X_s = X \times_S \Spec(\kres(s))$ the fibre of $f$ in $s$.
If $x \in X_s$ is \emph{rational over $\kres(s)$}, $(\sh{P}_{X/S}^n)_x \otimes_{\sh{O}_s} \kres(s)$ is canonically isomorphic to $\sh{O}_{X_s, x}/{\mathfrak{m}'_x}^{n+1}$, is the maximal ideal of $\sh{O}_{X_s, x}$;
more precisely, this isomorphism sends $(d^n t)_x \otimes 1$ \emph{(where $t$ is a section of $\sh{O}_X$ over an open neighbourhood of $x$ in $X$)} to the class of $t_x \otimes 1$ modulo ${\mathfrak{m}'_x}^{n+1}$.
\end{corollary}

\begin{proof}
This follows from \sref{IV.16.4.5} and \sref{IV.16.4.12}.
\end{proof}

The preceding corollaries justify the terminology ``sheaf of principal parts of order $n$''.

\begin{proposition}[16.4.14]
\label{IV.16.4.14}
Let $\rho:A \to B$ be a morphism of rings, $S$ a multiplicative subset of $B$.
Then the canonical homomorphisms 
\[
  \label{IV.16.4.14.1}
  S^{-1}P_{B/A}^n \to P_{S^{-1}B/A}^n
  \tag{16.4.14.1}
\] 
\oldpage[IV-4]{23}
deduced from the canonical homomorphisms $P_{B/A}^n \to P_{S^{-1}B/A}^n$ \sref{IV.16.4.4}, form a projective system and are bijective.
\end{proposition}

\begin{proof}
It suffices to remark that $S^{-1}((B \otimes_A B)/\mathfrak{I}^{n+1}) = S^{-1}(B \otimes_A B)/(S^{-1}\mathfrak{I})^{n+1}$ by flatness, and that $S^{-1}(B \otimes_A B) = (S^{-1}B)\otimes_A (S^{-1}B) $ \sref[I]{I.1.3.4}.
\end{proof}

\begin{corollary}[16.4.15]
\label{IV.16.4.15}
The notation being that of \sref{IV.16.4.14}, let $R$ be a multiplicative subset of $A$ such that $\rho(R) \subset S$.
Then we have canonical isomorphisms 
\[
  \label{IV.16.4.15.1}
  S^{-1}P_{B/A}^n \isoto P_{S^{-1}B/R^{-1}A}^n
  \tag{16.4.15.1}
\]
forming a projective system.
\end{corollary}

\begin{proof}
It evidently suffices to define canonical isomorphisms 
\[
  \label{IV.16.4.15.2}
  P_{S^{-1}B/A}^n \isoto P_{S^{-1}B/R^{-1}A}^n
  \tag{16.4.15.2}
\]
that is to say that we reduce to the case there $\rho(R)$ is consists of \emph{invertible} elements of $B$.
But then the isomorphism \sref{IV.16.4.15.2} is simply induced by the canonical isomorphism $B \otimes_A B \to B \otimes_{R^{-1}A} B$ by passing to quotients \sref[I]{I.1.5.3}.
\end{proof}

\begin{corollary}[16.4.16]
\label{IV.16.4.16}
Let $f:X \to S$ be a morphism of preschemes, $x$ a point of $X$, $s = f(x)$.
Then we have canonical isomorphisms
\[
  \label{IV.14.4.16.1}
  (\sh{P}_{X/S}^n)_x \isoto P_{\sh{O}_x/\sh{O}_s}^n
  \tag{14.4.16.1}
\]
forming a projective system.
\end{corollary}

We deduce from these isomorphisms of the associated graded rings, and in particular a canonical isomorphism
\[
  \label{IV.16.4.16.2}
  (\Omega_{X/S}^1)_x \isoto \Omega_{\sh{O}_x/\sh{O}_s}^1.
  \tag{16.4.16.2}
\]

\begin{corollary}[16.4.17]
\label{IV.16.4.17}
Let $k$ be a field, $K$ the field of rational functions $k(T_1, \dots, T_r)$.
Then, for every integer $n$, the homomorphism of $K[U_1, \dots, U_r]$ ($U_i$ indeterminates) into $P_{K/k}^n$ which sends $U_i$ to $d^nT_i - T_i.1$ is surjective and defines an isomorphism from the quotient $K[U_1, \dots, U_n]/\mathfrak{m}^{n+1}$  (where $\mathfrak{m}$ is the ideal generated by the $U_i$) to $P_{K/k}^n$.
\end{corollary}

\begin{proof}
This follows from \sref{IV.16.4.8}, \sref{IV.16.4.10}  and \sref{IV.16.4.14}, where we take $A = k$, $B = k[T_1, \dots, T_r]$ and $S = B \setmin \{0\}$.
\end{proof}

We thus recover the fact that the $dT_i$ form a basis of the $K$-vector space $\Omega_{K/k}^1$ \sref[0]{0.20.5.10}.

\begin{env}[16.4.18]
\label{IV.16.4.18}
Let $f:X \to Y$, $g:Y \to Z$ be two morphisms of preschemes, and consider the canonical homomorphism of augmented $\sh{O}_X$-algebras \sref{IV.16.4.3.3}
\[
  \label{IV.16.4.18.1}
  g_{X/Y/Z}:\sh{P}_{X/Z}^n \to \sh{P}_{X/Y}^n
  \tag{16.4.18.1}
\]
\[
  \label{IV.16.4.18.2}
  f_{X/Y/Z}:f^*(\sh{P}_{Y/Z}^n) \to \sh{P}_{X/Z}^n.
  \tag{16.4.18.2}
\]
Then $g_{X/Y/Z}$ is surjective, and its kernel is the sheaf of ideals generated by the image under $f_{X/Y/Z}$ of the augmentation ideal of $f^*(\sh{P}_{X/Z}^n)$.
\end{env}

\begin{proof}
\oldpage[IV-4]{24}
First note that $g_{X/Y/Z}$ corresponds to the case in \sref{IV.16.4.3.3} where $X' = X$, $S' = Y$ and $S = Z$, $u = 1_X$, and $f_{X/Y/Z}$ to the case where we replace $X', X, S, S'$ by $X, Y, Z, Z$ respectively and $u$, $f$ by $f$, $g$ respectively.

We have a commutative diagram \sref[I]{5.3.5}
\[
  \label{IV.16.4.18.3}
  \xymatrix{
    X \ar[r]^-{\Delta_f} \ar[dr]_-f & X \times_Y X \ar[d]^-p \ar[r]^-j & X \times_Z X \ar[d]^-{f \times_z f} \\
    &   Y \ar[r]_-{\Delta_g}  & Y \times_Z Y
  }
  \tag{16.4.18.3}
\] 
where $j = (1_X, 1_X)_Z$ is an immersion, $j \circ \Delta_f = \Delta_{g \circ f}$, and $p$ is the structure morphism.
Since we can restrict ourselves to the case where $X$, $Y$ and $Z$ are affine, we can suppose that the immersions $\Delta_f$, $\Delta_g$ and $j$ are closed, so that $\sh{O}_X$ and $\sh{O}_{X \times_Y X}$ are identified respectively with $\sh{O}_{X \times_Z X} / \sh{I}$ and $\sh{O}_{X \times_Z X}/\sh{L}$, where $\sh{L} \supset \sh{I}$ are two quasi-coherent ideals corresponding respectively to the immersions $\Delta_{g \circ f} $ and $j$.
The $\sh{O}_X$-algebra $\sh{P}_{X/Z}^n$ is identified with $\sh{O}_{X \times_Z X} / \sh{I}^{n+1}$, and $\sh{P}_{X/Y}^n$ is identified with $\sh{O}_{X \times_Y X}/(\sh{I}/\sh{L})^{n+1}$, which is to say with $\sh{O}_{X \times_Z X}/(\sh{I}^{n+1} + \sh{L})$, and therefore with the quotient of $\sh{P}_{X/Z}^n$ by $(\sh{I}^{n+1} + \sh{L})/\sh{I}^{n+1}$.
But we know (\emph{loc. cit}) that if $p$ and $j$ make $X \times_Y X$ the product of the $(Y \times_Z Y)$-preschemes $Y$ and $X \times_Z X$, so if $\sh{O}_Y$ is identified to $\sh{O}_{Y \times_Z Y}/\sh{K}$, where $\sh{K}$ is the ideal corresponding to $\Delta_g$, $\sh{L}$ is equal to $(f\times_Z f)^*(\sh{K}).\sh{O}_{X \times_Z X}$ \sref[I]{I.4.4.5}.
Since $(\sh{I}^{n+1} + \sh{L})/\sh{I}^{n+1}$ is the ideal of $\sh{P}_{X/Z}^n$ generated by the image of $\sh{L}$, we deduce the proposition.
\end{proof}

\begin{corollary}[16.4.19]
\label{IV.16.4.19}
With the notation of \sref{IV.16.4.18}, we have an exact sequence of quasi-coherent $\sh{O}_X$-modules 
\[
  \label{IV.16.4.19.1}
  \xymatrix{
    f^*(\Omega_{Y/Z}^1) \ar[r]^-{f_{X/Y/Z}} & \Omega_{X/Z}^1 \ar[r]^-{g_{X/Y/Z}} & \Omega_{X/Y}^1 \ar[r] & 0.
  }
  \tag{16.4.19.1}
\]
\end{corollary}

When $X$, $Y$, $Z$ are affine, we recover the exact sequence \sref[0]{20.5.7.1}.

\begin{proposition}[16.4.20]
\label{IV.16.4.20}
Let $f:Y \to Z$ be a morphism, $j: X \to Y$ a closed immersion, $\sh{K}$ the quasi-coherent sheaf of ideals of $\sh{O}_Y$ corresponding to $j$.
It follows that $\sh{P}_{X/Y}^n = \sh{O}_X = \sh{O}_Y/\sh{K}$, the canonical homomorphism $j_{X/Y/Z}: j^*(\sh{P}_{Y/Z}^n) \to \sh{P}_{X/Z}^n$ is surjective, and its kernel is the ideal of $j^*(\sh{P}_{Y/Z}^n)$ generated by $j^*(\sh{O}_Y \cdot d_{Y/Z}^n(\sh{K}))$ (it should be noted that $d_{Y/Z}^n(\sh{K})$ is a subsheaf of abelian groups of $\sh{P}_{X/Z}^n$, but not an $\sh{O}_Y$-module in general).
\end{proposition}

\begin{proof}
We know \sref[I]{I.5.3.8} that the diagonal $\Delta_j:X \to X \times_Y X$ is an isomorphism, from which the first assertion follows.
If $\varpi_1$ and $\varpi_2$ are the two canonical homomorphisms of algebras $\sh{O}_Y \to \sh{P}_{Y/Z}^n$ corresponding respectively to the two canonical projections $p_1$, $p_2$ of $Y \times_Z Y \to Y$, recall that by definition (\sref{IV.16.3.5} and \sref{IV.16.3.6}) $\varpi_1$ is the structure homomorphism of the $\sh{O}_Y$-algebra $\sh{P}_{Y/Z}^n$ and $\varpi_2 = d_{Y/Z}^n$.
The $\sh{O}_X$-algebra $j^*(\sh{P}_{Y/Z}^n)$ is therefore identified with $\sh{P}_{Y/Z}^n/\varpi_1(\sh{K})\sh{P}_{Y/Z}^n$ and its quotient by the ideal generated by $j^*(d_{Y/Z}^n(\sh{K}))$ to $\sh{P}_{Y/Z}^n/(\varpi_1(\sh{K}) + \varpi_2(\sh{K}))\sh{P}_{Y/Z}^n$.
Now note that we have a commutative diagram
\oldpage[IV-4]{25}
\[
  \xymatrix{
  Y \ar[d]_-{\Delta_f} & X \ar[d]^-{\Delta_{f \circ j}} \ar[l]_-{j}  \\
  Y \times_Z Y & X \times_Z X \ar[l]^-{j \times_Z j} 
  }
\]
identifying $X$ with the product of the $(Y \times_Z Y)$-preschemes $Y$ and $X \times_Z X$ \sref[I]{I.5.3.7}.
Since $j \times_Z j$ is an immersion, we therefore deduce from this remark and from \sref{IV.16.2.2} that if $\Delta_{Y/Z}^n$ and $\Delta_{X/Z}^n$ denote the infinitesimal neighbourhoods of order $n$ of $Y$ and $X$ by the canonical immersions $\Delta_f$ and $\Delta_{f\circ j}$ respectively, then we have a diagram
\[
  \xymatrix{
  \Delta_{Y/Z}^n \ar[d] & \Delta_{X/Z}^n \ar[d] \ar[l]  \\
  Y \times_Z Y & X \times_Z X \ar[l]^-{j \times_Z j} 
  }
\]
making $\Delta_{X/Z}^n$ the product of the $(Y \times_Z Y)$-preschemes $\Delta_{Y/Z}^n$ and $X \times_Z X$.
We can also say that $\sh{P}_{X/Z}^n$ is identified with the sheaf of rings $\sh{P}_{Y/Z}^n \otimes_{\sh{O}_{Y \times_Z Y}} \sh{O}_{X \times_Z X}$.
But we see immediately that (for example, by restricting to the affine case) that $\sh{O}_{X \times_Z X} = \sh{O}_{Y \times_Z Y}/(p_1^*(\sh{K}) + p_2^*(\sh{K}))\sh{O}_{Y \times_Z Y}$.
Therefore $\sh{P}_{X/Z}^n$ is identified with the quotient of $\sh{P}_{Y/Z}^n$ by the ideal generated by the image in $\sh{P}_{Y/Z}^n$ of $p_1^*(\sh{K}) + p_2^*(\sh{K})$.
But by definition this ideal is generated by $\varpi_1(\sh{K}) + \varpi_2(\sh{K})$. 
\end{proof}

\begin{corollary}[16.4.21]
\label{IV.16.4.21}
Let $f:Y \to Z$ be a morphism, $j: X \to Y$ an immersion.
We have an exact sequence of quasi-coherent $\sh{O}_X$-modules
\[
  \label{IV.16.4.21.1}
  \xymatrix{
    \sh{N}_{X/Y} \ar[r] & j^*(\Omega_{Y/Z}^1) \ar[r] & \Omega_{X/Z}^1 \ar[r] & 0.
  }
  \tag{16.4.21.1}
\]
\end{corollary}

When $X$, $Y$, $Z$ are affine, we recover the exact sequence \sref[0]{0.20.5.12.1}.

\begin{corollary}[16.4.22]
\label{IV.16.4.22}
If $f: X \to S$ is a morphism locally of finite presentation, $\sh{P}_{X/S}^n$ and $\Omega_{X/S}^1$ are quasi-coherent $\sh{O}_X$-modules of finite presentation.
\end{corollary}

\begin{proof}
We immediately reduce to the case where $S = \Spec(A)$ is affine, $X = \Spec(B)$, where $B = A[T_1, \dots, T_r]/\mathfrak{K}$, $\mathfrak{K}$ being an ideal of finite type of $C = A[T_1, \dots, T_r]$.
Applying \sref{IV.16.4.20} where $Z = S$, $Y = \Spec(C)$ and $\sh{K} = \widetilde{\mathfrak{K}}$.
Then $j^*(\sh{P}_{Y/Z}^n)$ is a free $\sh{O}_X$-module of finite rank \sref{IV.16.4.10} and the hypothesis on $\mathfrak{K}$ implies that $j^*(\sh{O}_Y.d_{Y/Z}^n(\sh{K}))$ generates a quasi-coherent $\sh{O}_X$-module of finite type;
hence the conclusion.
\end{proof}

\begin{proposition}[16.4.23]
\label{IV.16.4.23}
Let $X$, $Y$ be two $S$-preschemes, $Z = X \times_S Y$ their product, $p:X \times_S Y \to X$ and $q:X \times_S Y \to Y$ the canonical projections.
Then the canonical homomorphism
\[
  \label{IV.16.4.23.1}
  p_{Z/X/S} \oplus q_{Z/Y/S}: p^*(\Omega_{X/S}^1) \oplus  q^*(\Omega_{Y/S}^1) \to \Omega_{(X\times_S Y)/S}^1
  \tag{16.4.23.1}
\]
is bijective.
\end{proposition}   

\begin{proof}
\oldpage[IV-4]{26}
The commutative diagram
\[
  \xymatrix{
    Y \ar[d]_-g & X \times_S Y \ar[l]_-q \ar[d]_-h & X \times_S Y \ar[l]_-{\operatorname{id}} \ar[d]^-p\\
    S & S \ar[l]^-{\operatorname{id}} & X \ar[l]^-f
  }
\]
gives us a factorization of the canonical \emph{isomorphism} $P^n(p)$ \sref{IV.16.4.5}
\[
  p^*(\sh{P}_{X/S}^n) \to \sh{P}_{Z/S}^n \to \sh{P}_{Z/Y}^n
\]
and similarly, switching $X$ with $Y$, we have a factorization of the \emph{isomorphism} $P^n(q)$
\[
  q^*(\sh{P}_{Y/S}^n) \to \sh{P}_{Z/S}^n \to \sh{P}_{Z/Y}^n.
\]
This proves that the canonical homomorphism \sref{IV.16.4.18.1}
\[
  p_{Z/X/S}:p^*(\sh{P}_{X/S}^n) \to \sh{P}_{Z/S}^n \quad \text{(resp. $q_{Z/X/S}:q^*(\sh{P}_{Y/S}^n) \to \sh{P}_{Z/S}^n$)}
\] 
is \emph{injective}, and that the kernel of the canonical surjective homomorphism  \sref{IV.16.4.18.2}
\[
  \sh{P}_{Z/S}^n \to \sh{P}_{Z/Y}^n \quad \text{(resp. $\sh{P}_{Z/S}^n \to \sh{P}_{Z/X}^n$)}
\] 
is direct summand of the image $p_{Z/X/S}$ (resp. $q_{Z/Y/S}$).
On the other hand, this kernel is, by virtue of \sref{IV.16.4.18}, generated by the image by $q_{Z/Y/S}$ (resp. $p_{Z/X/S}$) of the augmentation ideal of $q^*(\sh{P}_{Y/S}^n)$ (resp. $p^*(\sh{P}_{X/S}^n)$).
We conclude the proposition by considering the case $n = 1$.
\end{proof}

We immediately generalize \sref{IV.16.4.23} to the case of a product of any finite number of $S$-preschemes.

\begin{remarks}[16.4.24]
\label{IV.16.4.24}
\medskip\noindent
\begin{enumerate}
  \item[(i)] We will see \sref{IV.17.2.3} that when the morphism $f: X \to Y$ in \sref{IV.16.4.18} is \emph{smooth}, the homomorphism $f_{X/Y/Z}$ in \sref{IV.16.4.19.1} is locally \emph{left invertible} and in particular injective.
  Similarly, when the morphism $f \circ j: X \to Z$ of \sref{IV.16.4.20} is \emph{smooth}, the homomorphism on the left in \sref{IV.16.4.21.1} is locally \emph{left invertible} and \textit{a fortiori} injective \sref{IV.17.2.5}.
  In Chapter~V, we will also give a variant, in the case of modules over a prescheme, of the ``imperfection modules'' studied in \sref[0]{0.20.6}, and the exact sequences where they occur.
  \item[(ii)] Let $X$ be a topological space, $\sh{A}$ a sheaf of  rings over $X$ and $\sh{B}$ a $\sh{A}$-algebra over $X$.
  Then it is clear that 
  \[
    U \mapsto P_{\Gamma(U, \sh{B})/\Gamma(U, \sh{A})}^n \quad \text{($U$ open in $X$)}
  \]
  is a presheaf of augmented $\Gamma(U, \sh{B})$-algebras, and therefore the associated sheaf $\sh{P}_{\sh{B}/\sh{A}}^n$ is an augmented $\sh{B}$-algebra.
  In the particular case where $X$ is a prescheme, $f = (\psi, \theta): X \to S$ a morphism of preschemes, it follows easily from \sref{IV.16.4.16} and from the exactness of the functor $\varinjlim$ that $\sh{P}_{X/S}^n$ is canonically isomorphic to $\sh{P}_{\sh{O}_X/\psi^*(\sh{O}_S)}^n$.
  It follows that the formalism developed in the present paragraph could be considered as a
\oldpage[IV-4]{27}
  particular case of a differential formalism for ringed spaces endowed with a sheaf of algebras over the structure sheaf.
  However, we did not start with this point of view, which is less intuitive and less convenient for applications.
  It also seems that, for various kinds of ``varieties'', the ``global'' constructions of the $\sh{P}^n$ analogous to those we have used here are also better suited for applications.
\end{enumerate}
\end{remarks}

\subsection{Relative tangent sheaves and bundles; derivations.}
\label{IV.16.5}

\begin{env}[16.5.1]
\label{IV.16.5.1}
Let $f = (\psi, \theta) : X \to S$ be a morphism of ringed spaces.
For every $\sh{O}_X$-module $\sh{F}$, we say \emph{$S$-derivation} (or \emph{$(X/S)$-derivation}, or \emph{$f$-derivation}) \emph{of $\sh{O}_X$ to $\sh{F}$} for every homomorphism of \emph{sheaves of additive groups} $D : \sh{O}_X \to \sh{F}$ satisfying the following conditions:
\begin{enumerate}
  \item[(a)] for every open $V$ of $X$, and all pair of sections $(t_1, t_2)$ of $\sh{O}_X$ over $V$, we have
  \[
  \label{IV.16.5.1.1}
    D(t_1 t_2) = t_1 D(t_2) + D(t_1)t_2;
    \tag{16.5.1.1}
  \]
  \item[(b)] for every open $V$ of $X$, every section $t$ of $\sh{O}_X$ over $V$, and every section $s$ of $\sh{O}_S$ over an open $U$ of $S$ such that $V \subset f^{-1}(U)$, we have
  \[
  \label{IV.16.5.1.2}
    D((s|V)t) = (s|V)D(t).
    \tag{16.5.1.2}
  \]
\end{enumerate}
It is clear that this amounts to saying that, for all $x \in X$, the homomorphism of additive groups $D_x : \sh{O}_x \to \sh{F}_x$ is an \emph{$\sh{O}_{f(x)}$-derivation}.

Another interpretation consists of considering the $\sh{O}_X$-algebra $\sh{D}_{\sh{O}_X}(\sh{F})$ as equal to $\sh{O}_X \oplus \sh{F}$, the algebra structure being defined by the condition that for every open $V$ of $X$, the product of two sections of $\sh{O}_X$ (resp. of a section of $\sh{O}_X$ and a section of $\sh{F}$) over $V$ is defined by the ring structure of $\Gamma(V, \sh{O}_X)$ (resp. the $\Gamma(V, \sh{O}_X)$-module structure on $\Gamma(V, \sh{F})$), and the product of two sections of $\sh{F}$ over $V$ is chosen to be zero;
then $\sh{F}$ is an ideal of $\sh{D}_{\sh{O}_X}(\sh{F})$, the kernel of the canonical augmentation $\sh{D}_{\sh{O}_X}(\sh{F}) \to \sh{O}_X$, and to say that $D$ is an $S$-derivation of $\sh{O}_X$ to $\sh{F}$ means that $1_{\sh{O}_X} + D$ is an \emph{$\sh{O}_S$-homomorphism of algebras} from $\sh{O}_X$ to $\sh{D}_{\sh{O}_X}(\sh{F})$, which, composed with the augmentation, gives $1_{\sh{O}_X}$.

The $S$-derivations of $\sh{O}_X$ to $\sh{F}$ clearly form a \emph{$\Gamma(X, \sh{O}_X)$-module} $\Der_{\sh{O}_S}(\sh{O}_X, \sh{F})$.

When $\sh{F} = \sh{O}_X$, an $S$-derivation of $\sh{O}_X$ to itself is simply called an \emph{$S$-derivation of $\sh{O}_X$}.
\end{env}

\begin{proposition}[16.5.2]
\label{IV.16.5.2}
Let $A$ be a ring, $B$ an $A$-algebra, $L$ a $B$-module;
let $S = \Spec(A)$, $X=\Spec(B)$, $\sh{F} = \widetilde{L}$.
Then the map $D \mapsto \Gamma(D)$ which sends every $S$-derivation $D$ of $\sh{O}_X$ to $\sh{F}$ to the map $\Gamma(D) : t \mapsto D(t)$ of $B$ to $L$, is an isomorphism of $B$-modules from $\Der_S(\sh{O}_X, \sh{F})$ to $\Der_A(B, L)$ (cf.~\sref[0]{0.20.1.2}).
\end{proposition}

\begin{proof}
This follows immediately from the given interpretation of $S$-derivations in terms
\oldpage[IV-4]{28}
of homomorphisms of algebras, analogous to the interpretation given in \sref[0]{0.20.1.6}, and from the canonical correspondence between homomorphisms of $\sh{O}_X$-algebras and homomorphisms of $B$-algebras (\sref[I]{I.1.3.13} and \sref[I]{I.1.3.8}).
\end{proof}

\begin{proposition}[16.5.3]
\label{IV.16.5.3}
Let $f = (\psi, \theta) : X \to S$ be a morphism of preschemes.
\begin{enumerate}
  \item[{\rm(i)}] The differential $d_{X/S} : \sh{O}_X \to \Omega_{X/S}^1$ \sref{IV.16.3.6} is an $S$-derivation.
  \item[{\rm(ii)}] For every $\sh{O}_X$-module $\sh{F}$, the map $u \mapsto u \circ d_{X/S}$ is an isomorphism of $\Gamma(X, \sh{O}_X)$-modules
  \[
  \label{IV.16.5.3.1}
    \Hom_{\sh{O}_X}(\Omega_{X/S}^1, \sh{F}) \isoto \Der_S(\sh{O}_X, \sh{F}).
    \tag{16.5.3.1}
  \]
\end{enumerate}
\end{proposition}

\begin{proof}
The assertion (i) has already been written \sref{IV.16.3.6}.
On the other hand, it is immediate (in light of \sref[0]{0.20.4.8}) that $u \mapsto u\circ d_{X/S}$ is injective, considering the restrictions to a fibre $\sh{O}_x$ of the two sides and using \sref{IV.16.4.16.2}.
To see that the homomorphism \sref{IV.16.5.3.1} is surjective, consider an $S$-derivation $D : \sh{O}_X \to \sh{F}$;
for every affine open $V = \Spec(B)$ of $X$, such that $f(V)$ is contained in an affine open $U = \Spec(A)$ of $S$, $D_V : B \to \Gamma(V, \sh{F}) $ is an $A$-derivation, and therefore there exists a \emph{unique} $B$-homomorphism $u_V : \Omega_{B/A}^1 \to \Gamma(V, \sh{F})$ such that $D_V = u_V \circ d_{B/A}$ \sref[0]{0.20.4.8};
in addition, the uniqueness of $u_V$ shows immediately that for an affine open $W \subset V$ we have $u_W = u_V|W$, and therefore the $u_V$ define a homomorphism of $\sh{O}_X$-modules $u : \sh{O}_X \to \sh{F}$ answering the question.
\end{proof}

\begin{env}[16.5.4]
\label{IV.16.5.4}
With the notation of \sref{IV.16.5.1}, for every open $U$ of $X$, $\Der_S(\sh{O}_U, \sh{F}|U)$ is a $\Gamma(U, \sh{O}_X)$-module and it is clear that the map $U \mapsto \Der_S(\sh{O}_U, \sh{F}|U)$ is a presheaf;
in fact, it is even a \emph{sheaf} (and therefore an $\sh{O}_X$-module), in light of the pointwise characterization of $S$-derivations, seen in \sref{IV.16.5.1}.
This $\sh{O}_X$-module is denoted by $\shDer_S(\sh{O}_X, \sh{F})$ and is called the \emph{sheaf of $S$-derivations of $\sh{O}_X$ in $\sh{F}$}, and what we have seen is further expressed in the following corollary:
\end{env}

\begin{corollary}[16.5.5]
\label{IV.16.5.5}
For every $\sh{O}_X$-module $\sh{F}$, the homomorphism of $\sh{O}_X$-modules induced by $u \mapsto u \circ d_{X/S}$
\[
\label{IV.16.5.5.1}
  \shHom_{\sh{O}_X}(\Omega_{X/S}^1, \sh{F}) \to \shDer_S(\sh{O}_X, \sh{F})
  \tag{16.5.5.1}
\]
is bijective.
\end{corollary}

\begin{corollary}[16.5.6]
\label{IV.16.5.6}
\begin{enumerate}
  \item[{\rm(i)}] If the morphism $f : X \to S$ is locally of finite presentation and if $\sh{F}$ is a quasi-coherent $\sh{O}_X$-module, then $\shDer_S(\sh{O}_X, \sh{F})$ is a quasi-coherent $\sh{O}_X$-module.
  \item[{\rm(ii)}] If in addition $S$ is locally Noetherian and if $\sh{F}$ is coherent, then $\shDer_S(\sh{O}_X, \sh{F})$ is a coherent $\sh{O}_X$-module.
\end{enumerate}
\end{corollary}

\begin{proof}
The assertion (i) follows from the isomorphism \sref{IV.16.5.5.1}, from \sref{IV.16.4.22}, and \sref[I]{I.3.12};
the assertion (ii) follows from \sref[0]{0.5.3.5}.
\end{proof}

\begin{env}[16.5.7]
\label{IV.16.5.7}
We set
\[
\label{IV.16.5.7.1}
  \mathfrak{G}_{X/S} = \shHom_{\sh{O}_X}(\Omega_{X/S}^1, \sh{O}_X) = \shDer_S(\sh{O}_X, \sh{O}_X),
  \tag{16.5.7.1}
\]
and say that it is the \emph{sheaf of $S$-derivations of $\sh{O}_X$}, or even the \emph{tangent sheaf of $X$ relative to $S$}:
it is therefore the \emph{dual} of the $\sh{O}_X$-module $\Omega_{X/S}^1$.
If $f$ is locally of finite presentation,
\oldpage[IV-4]{29}
$\mathfrak{G}_{X/S}$ is a quasi-coherent $\sh{O}_X$-module;
if in addition $S$ is locally Noetherian, then $\mathfrak{G}_{X/S}$ is coherent \sref{IV.16.5.6}.
\end{env}

\begin{env}[16.5.8]
\label{IV.16.5.8}
Suppose in particular that $\Omega_{X/S}^1$ is a \emph{locally free} $\sh{O}_X$-module (of finite rank) (which will be the case then $f$ is \emph{smooth} \sref{IV.17.2.3});
then $\mathfrak{G}_{X/S}$ is locally free $\sh{O}_X$-module of the same rank as $\Omega_{X/S}^1$ at each point.
More specifically, suppose that $\Omega_{X/S}^1$ is of rank $n$ at a point $x$;
then there are $n$ sections $s_i$ ($1 \leq i \leq n$) of $\sh{O}_X$ over an affine neighbourhood $U$ of $x$ such that the canonical images of the $ds_i$ in $\Omega_{X/S}^1 \otimes_{\sh{O}_X} \kres(x)$ form a basis if this $\kres(x)$-vector space;
by virtue of Nakayama's lemma, the germs $(ds_i)_x = d(s_i)_x$ of the $ds_i$ at the point $x$ form a basis of the $\sh{O}_x$-module $(\Omega_{X/S}^1)_x$, and therefore, by restricting $U$, we can suppose that the $ds_i$ form a \emph{basis} of the $\Gamma(U, \sh{O}_X)$-module $\Gamma(U, \Omega_{X/S}^1)$.
So the $\Gamma(U, \sh{O}_X)$-module $\Gamma(U, \mathfrak{G}_{X/S})$ is dual to the above;
we denote by $(D_i)_{1 \leq i \leq n}$ or $\left(\frac{\partial}{\partial s_i}\right)_{1 \leq i \leq n}$ the dual basis of $(ds_i)_{1 \leq i \leq n}$, so that, by \sref{IV.16.5.3}, we have
\[
\label{IV.16.5.8.1}
  D_i s_j = \langle D_i, ds_j \rangle = \left\langle \frac{\partial}{\partial s_i}, ds_j \right\rangle = \delta_{ij} \quad \text{(Kronecker's symbol)}.
  \tag{16.5.8.1}
\]
Every $\Gamma(S, \sh{O}_S)$-derivation of the $\Gamma(S, \sh{O}_S)$-algebra $\Gamma(U, \sh{O}_X)$ is therefore written in an unique way as 
\[
  D = \sum_{i = 1}^n a_i D_i = \sum_{i = 1}^n a_i \frac{\partial}{\partial s_i},
\] 
where the $a_i$ ($1 \leq i \leq n$) are sections of $\sh{O}_X$ over $U$.
For every section $g \in \Gamma(U, \sh{O}_X)$, if we put $dg = \sum_{i = 1}^n c_i ds_i$, then we have $c_i = \langle D_i, dg \rangle = D_i g$ by virtue of \sref{IV.16.5.8.1}, in other words,
\[
\label{IV.16.5.8.2}
  dg = \sum_{i = 1}^n (D_i g) ds_i = \sum_{i = 1}^n \frac{\partial g}{\partial s_i} ds_i.
  \tag{16.5.8.2}
\]
\end{env}

\begin{env}[16.5.9]
\label{IV.16.5.9}
Let $D_1$, $D_2$ be two $S$-derivations of $\sh{O}_X$. 
For every open $U$ of $X$, if $D_1^U$, $D_2^U$ are the corresponding derivations of the ring $\Gamma(U, \sh{O}_X)$, the \emph{bracket}
\[
  [D_1^U, D_2^U] = D_1^U \circ D_2^U - D_2^U \circ D_1^U
\]
is also a derivation in this ring, and therefore the $\psi^*(\sh{O}_S)$-endomorphism of $\sh{O}_X$
\[
\label{IV.16.5.9.1}
  [D_1, D_2] = D_1 \circ D_2 - D_2 \circ D_1
  \tag{16.5.9.1}
\]
is also an $S$-derivation;
as we immediately check that this bracket satisfies the Jacobi identity, we have thus defined on $\Der_S(\sh{O}_X, \sh{O}_X)$ a \emph{$\Gamma(S, \sh{O}_S)$-Lie algebra} structure.
Since the definition of this structure commutes with the restriction to an open of $X$, we thus see that $\mathfrak{G}_{X/S}$ is canonically equipped with a \emph{ $\psi^*(\sh{O}_S)$-Lie algebra} structure.
Note that the mapping $(D_1, D_2) \mapsto [D_1, D_2]$ is \emph{not $\Gamma(X, \sh{O}_X)$-bilinear}.
\end{env} 

\begin{env}[16.5.10]
\label{IV.16.5.10}
For every base change $g : S' \to S$, if  we set $X' = X \times_S S'$, then we see \sref{IV.16.4.5} that we have a canonical isomorphism
\[
\label{IV.16.5.10.1}
  \Omega_{X/S}^1 \otimes_S S' \isoto \Omega_{X'/S'}^1,
  \tag{16.5.10.1}
\]
\oldpage[IV-4]{30}  
from which we deduce, by \sref{IV.16.5.10.1}, a canonical homomorphism (Bourbaki, \emph{Alg.}, chap.~II, 3\textsuperscript{rd}ed., \textsection5, n\textsuperscript{o}3)
\[
\label{IV.16.5.10.2}
  \mathfrak{G}_{X/S} \otimes_{\sh{O}_S} \sh{O}_{S'} \to \mathfrak{G}_{X'/S'},
  \tag{16.5.10.2}
\]
which is neither injective nor surjective in general.
However:
\end{env}

\begin{proposition}[16.5.11]
\label{IV.16.5.11}
\begin{enumerate}
  \item[{\rm(i)}] If $g : S' \to S$ is a flat morphism and if $f$ is locally of finite type (resp. locally of finite presentation), then the homomorphism \sref{IV.16.5.10.2} is injective (resp. bijective).
  \item[{\rm(ii)}] If $\Omega_{X/S}^1$ is a locally free $\sh{O}_X$-module of finite type, then the homomorphism \sref{IV.16.5.10.2} is bijective.
\end{enumerate}
\end{proposition}

\begin{proof}
The assertion (ii) follows from Bourbaki, \emph{Alg.}, chap.~II, 3\textsuperscript{rd}~ed., \textsection5, n\textsuperscript{o}3, prop.~7.
The assertion (i) follows similarly from Bourbaki, \emph{Alg. Comm.}, chap.~I, \textsection2, n\textsuperscript{o}10, prop.~11 and from the fact that if $f$ is locally of finite type (resp. locally of finite presentation), then $\Omega_{X/S}^1$ is an $\sh{O}_X$-module of finite type (resp. of finite presentation) (\sref{IV.16.3.9} \sref{IV.16.4.22}).
\end{proof}

\begin{env}[16.5.12]
\label{IV.16.5.12}
Since $\Omega_{X/S}^1$ is a quasi-coherent $\sh{O}_X$-module, we can consider the vector bundle over $X$ defined by $\Omega_{X/S}^1$ \sref[II]{II.1.7.8}
\[
  \label{IV.16.5.12.1}
  T_{X/S} = \bb{V}(\Omega_{X/S}^1)
  \tag{16.5.12.1}
\]
which is called the \emph{tangent bundle of $X$ relative to $S$}.
We have therefore a canonical bijection \sref[II]{II.1.7.9}
\[
  \Gamma(T_{X/S}/S) \isoto \Hom_{\sh{O}_X}(\Omega_{X/S}^1, \sh{O}_X) = \Gamma(X, \mathfrak{G}_{X/S})
\]
by definition of $\mathfrak{G}_{X/S}$, and we can replace $X$ by an open set $U$ of $X$ in this isomorphism;
so we can say that the \emph{tangent sheaf} of $X$ relative to $S$ is isomorphic to the \emph{sheaf of germs of $S$-sections} of the tangent bundle of $X$ relative to $S$.
If $f:X \to Y$ is an $S$-morphism, we saw \sref{IV.16.4.19} that we have a canonical homomorphism $f_{X/Y/S}:f^*(\Omega_{Y/S}^1) \to \Omega_{X/S}^1$, which, having in mind that
\[
  \bb{V}(f^*(\Omega_{Y/S}^1)) = \bb{V}(\Omega_{Y/S}^1) \times_Y X \quad \text{\sref[II]{II.1.7.11} },
\]
gives us an $X$-morphism $T_{X/S}(f): T_{X/S} \to T_{Y/S} \times_Y X$.
If $g:Y \to Z$ is a second $S$-morphism, we have $T_{X/S}(g \circ f) = (T_{Y/S}(g) \times 1_X ) \circ T_{X/S}(f)$ \sref[0]{0.20.5.4.1}.

It follows from \sref{IV.16.5.10.1} and from \sref[II]{II.1.7.11} that for every base change $g:S' \to S$ we have a canonical isomorphism
\[
  \label{IV.16.5.12.2}
  T_{X'/S'} \isoto T_{X/S} \times_S S' = T_{X/S} \times_X X'.
  \tag{16.5.12.2}
\]
\end{env}

\begin{env}[16.5.13]
\label{IV.16.5.13}
For every point $x \in X$, we define the \emph{tangent space of $X$ at the point $x$} (relative to $S$) to be the set of points in the fibre $T_{X/S} \times_X \Spec(\kres(x))$ that are \emph{rational over $\kres(x)$}, that is, the set
\[
\label{IV.16.5.13.1}
  T_{X/S}(x) = \Hom_{\kres(x)}(\Omega_{X/S}^1 \otimes_{\sh{O}_x} \kres(x), \kres(x)),
  \tag{16.5.13.1}
\]
which is the \emph{dual} of the $\kres(x)$-vector space $\Omega_{\sh{O}_x/\sh{O}_s}^1/\mathfrak{m}_x \cdot \Omega_{\sh{O}_x/\sh{O}_s}^1$.
When $\Omega_{X/S}^1$ is an $\sh{O}_X$-module of \emph{finite type}, then $T_{X/S}(x)$ is a vector space of finite rank over $\kres(x)$, and for every base change
\oldpage[IV-4]{31}
$g : S \to S'$, and every point $x' \in X' = X \times_S S'$ over $x$, we have a canonical isomorphism
\[
  \label{IV.16.5.13.2}
  T_{X'/S'}(x') \isoto T_{X/S} \otimes_{\kres(x)} \kres(x').
  \tag{16.5.13.2}
\]
If $x$ is \emph{rational over $\kres(s)$}, where $s = f(x)$ (so that $\kres(s) \to \kres(x)$ is an isomorphism), it follows from \sref{IV.16.4.13} that we have a canonical isomorphism
\[
\label{IV.16.5.13.3}
  T_{X/S}(x) = T_{X_s/\kres(s)}(x) = \Hom_{\kres(s)}(\mathfrak{m}'_x/\mathfrak{m}_x^{'2}, \kres(x)),
  \tag{16.5.13.3}
\]
where $\mathfrak{m}'_x$ is the maximal ideal of $\sh{O}_{X_s,x} = \sh{O}_{X, x}/\mathfrak{m}_s \sh{O}_{X,x}$.
In the case where $S$ is the spectrum of a field $k$, we recover the definition of the Zariski tangent space of a point $x \in X$ \emph{rational over $k$}, as the dual of $\mathfrak{m}_x / \mathfrak{m}_x^2$.

Let $Y$ be a second $S$-prescheme and let $g : Y \to X$ be an $S$-morphism;
then we have a canonical homomorphism of $\sh{O}_Y$-modules \sref{IV.16.4.19}
\[
\label{IV.16.5.13.4}
  g_{Y/X/S} : g^*(\Omega_{X/S}^1) \to \Omega_{Y/S}^1.
  \tag{16.5.13.4}
\]
Now note that if $y \in Y$ and $x = g(y)$, then we have
\[
  g^*(\Omega_{X/S}^1) \otimes_{\sh{O}_Y} \kres(y) = (\Omega_{X/S}^1 \otimes_{\sh{O}_X} \kres(x)) \otimes_{\kres(x)} \kres(y) 
\]
and consequently, if $\Omega_{X/S}^1$ is an $\sh{O}_X$-module \emph{of finite type}, then we can identify
\[
  \Hom_{\kres(y)}(g^*(\Omega_{X/S}^1) \otimes_{\sh{O}_Y} \kres(y), \kres(y))
\]
with $T_{X/S}(x) \otimes_{\kres(x)} \kres(y)$.
We therefore deduce from the homomorphism \sref{IV.16.5.13.4} a homomorphism of \emph{$\kres(y)$-vector spaces}
\[
\label{IV.16.5.13.5}
  T_y(g) : T_{Y/S}(y) \to T_{X/S}(x) \otimes_{\kres(x)} \kres(y)
  \tag{16.5.13.5}
\]
called the \emph{linear map tangent to $g$ at the point $y$}.
When $y$ is \emph{rational over $\kres(s)$}, we can identify $\kres(s)$, $\kres(y)$, and $\kres(x)$, and $T_y(g)$ is then a homomorphism of $\kres(s)$-vector spaces $T_{Y/S}(y) \to T_{X/S}(x)$;
also note that in this case, $g^*(\Omega_{X/S}^1) \otimes_{\sh{O}_Y} \kres(y)$ is identified with $\Omega_{X/S}^1 \otimes_{\sh{O}_X} \kres(x)$, and the above homomorphism is therefore defined without any finiteness conditions on $\Omega_{X/S}^1$ and it is none other than the homomorphism $T_{Y/S}(g)$ \sref{IV.16.5.12} restricted to the fibre of $T_{Y/S}$ at the point $y$.
\end{env}

\begin{env}[16.5.14]
\label{IV.16.5.14}
The interpretation of derivations of an $A$-algebra $B$ to a $B$-module $L$, given in \sref[0]{0.20.1.1}, translates to the language of preschemes in the following way.

Consider two morphisms of preschemes $f : X \to S$, $g : Y \to S$, and a closed subprescheme $Y_0$ of $Y$ defined by a \emph{zero-square} ideal $\sh{J}$ of $\sh{O}_Y$ (so that $Y$ and $Y_0$ have the \emph{same underlying subspace}).
Suppose we are given an $S$-morphism $u_0 : Y_0 \to X$, so that we have a commutative diagram
\[
\label{IV.16.5.14.1}
  \xymatrix{
    X \ar[d]_-f & Y_0 \ar[l]_-{u_0}\ar[d]^j \\
    S & Y \ar@{-->}[ul]_-u \ar[l]^-g
  }
  \tag{16.5.14.1}
\]
\oldpage[IV-4]{32}
and we suggest looking for an \emph{$S$-morphism} $u : Y \to X$ such that $u_0 = u \circ j$ (in other words, if it is possible to complete the diagram above by the dotted arrow $u$, keeping it \emph{commutative}).

For that, consider an affine open $U = \Spec(C)$ of $Y$;
its inverse image $j^{-1}(U)$ is the affine open $U_0 = \Spec(C/\mathfrak{L})$, where $\mathfrak{L} = \Gamma(U, \sh{J})$, a \emph{zero-square} ideal in $C$;
suppose that $U$ is small enough so that $u_0(U_0)$ is contained in an affine open $V = \Spec(B)$ of $X$ and that $g(U) = f(u_0(U_0))$ is contained in an affine open $W = \Spec(A)$ of $S$, so that $B$ and $C$ are $A$-algebras and $u_0|U_0$ corresponds to an $A$-homomorphism $\psi$ from $B$ to $C/\mathfrak{L}$;
Let $P(U_0)$ be the set of restrictions $u|U$ of the sought homomorphisms, which corresponds canonically to $A$-homomorphisms of algebras $\vphi : B \to C$ such that the \emph{composite $B \xrightarrow{\vphi} C \to C/\mathfrak{L}$ is equal to $\psi$}.
So we know \sref[0]{0.20.1.1} that the set of such homomorphisms is either empty or of the form $\vphi_1 + \Der_A(B, \mathfrak{L})$;
when $P(U_0)$ is not empty, the additive group $\Der_A(B, \mathfrak{L})$ acts by addition on $P(U_0)$, which is therefore an \emph{affine space} for the additive group $\Der_A(B, \mathfrak{L})$ (or even a \emph{principal homogeneous space} (or \emph{torsor}) \emph{under} $\Der_A(B, \mathfrak{L})$).

Now notice that, since $\mathfrak{L}$ is equipped with a $B$-module structure via $\psi$, we have an \emph{isomorphism} $v \mapsto v \circ d_{B/A}$ of $\Hom_B(\Omega_{B/A}^1, \mathfrak{L})$ onto $\Der_A(B, \mathfrak{L})$ \sref[0]{0.20.4.8}.
Besides, as $\mathfrak{L}$ is square-zero, therefore a $(C/\mathfrak{L})$-module, every $B$-homomorphism $v : \Omega_{B/A}^1 \to \mathfrak{L}$ can be considered as a $(C/\mathfrak{L})$-homomorphism $\Omega_{B/A}^1 \otimes_B (C/\mathfrak{L}) \to \mathfrak{L}$.
As $\sh{I}$ is square-zero, it can be considered as a quasi-coherent $\sh{O}_{Y_0}$-module;
let's introduce the $\sh{O}_{Y_0}$-module
\[
\label{IV.16.5.14.2}
  \sh{G} = \shHom(u_0^*(\Omega_{X/S}^1), \sh{I});
  \tag{16.5.14.2}
\]
it follows from the fact that $\Omega_{B/A}^1 = \Gamma(V, \Omega_{X/S}^1)$ \sref{IV.16.3.7} that we can write $\Der_A(B, \mathfrak{L}) = \Gamma(U_0, \sh{G})$.

As $P(U_0)$ is defined as a set of $S$-morphisms $U \to X$, it is clear that $U_0 \mapsto P(U_0)$ is a \emph{sheaf of sets} $\sh{P}$ on $Y_0$.
We can use this fact to prove that the map $h : \Gamma(U_0, \sh{G}) \times P(U_0) \to P(U_0)$ defining the torsor structure on $P(U_0)$ is independent of choice of $V$ and $W$ and also that, if $U' \subset U$ is a second affine open of $Y$, $U'_0$ its inverse image in $Y_0$, then the diagram
\[
\label{IV.16.5.14.3}
  \xymatrix{
    \Gamma(U_0, \sh{G}) \times P(U_0) \ar[d] \ar[r]^-h & P(U_0) \ar[d] \\
    \Gamma(U'_0, \sh{G}) \times P(U'_0) \ar[r]_-{h'} & P(U'_0)
  }
  \tag{16.5.14.3}
\]
is commutative (the vertical arrows being the restrictions).
In light of the above remark, we reduce to proving the commutativity of the above diagram when $h$ is defined as such from affine opens $V$, $W$ and $h'$ from affine
\oldpage[IV-4]{33}
opens $V' \subset V$ and $W' \subset W$.
But because of the preceding description of $h$, this follows from the commutativity of the diagram \sref[0]{0.20.5.4.2}.

The mapping $\Gamma(U_0, \sh{G}) \times P(U_0) \to P(U_0)$ therefore define a homomorphism of \emph{sheaf of sets} $m:\sh{G} \times \sh{P} \times \sh{P}$ such that, for all open sets $U_0$ for which $\Gamma(U_0, \sh{P}) \neq \emp$, $m_{U_0}:\Gamma(U_0, \sh{G}) \times \Gamma(U_0, \sh{P}) \to \Gamma(U_0, \sh{P})$ is an external law defining in $\Gamma(U_0, \sh{P})$ a torsor structure for the group $\Gamma(U_0, \sh{G})$.
\end{env}

\begin{env}[16.5.15]
\label{IV.16.5.15}
In general, when we are given a sheaf of sets $\sh{P}$ over a topological space $Z$, a sheaf of groups $\sh{G}$ (not necessarily commutative), and a homomorphism of sheaves of sets $m : \sh{G} \times \sh{P} \to \sh{P}$ such that, for every open $U \subset Z$ such that $\Gamma(U, \sh{P}) \neq \emp$, $m_U : \Gamma(U, \sh{G}) \times \Gamma(U, \sh{P}) \to \Gamma(U, \sh{P})$ makes $\Gamma(U, \sh{P})$ a \emph{torsor} under the group $\Gamma(U,\sh{G})$, then we say that $\sh{P}$ is a \emph{pseudo-torsor} (or \emph{formally principal homogeneous sheaf}) under the sheaf of groups $\sh{G}$.
We say that $\sh{P}$ is a \emph{torsor} (or \emph{principal homogeneous sheaf}) under $\sh{G}$
\footnote{
[Trans.] This is nowadays more commonly called a $\sh{G}$-torsor rather then a torsor \emph{under} $\sh{G}$.
}
if in addition $\Gamma(U, \sh{P}) \neq \emp$ for every open $U \neq \emp$ in a suitable basis for the topology of $Z$.

For the general theory of torsors, we refer to \cite{IV-42};
we will limit ourselves to recalling the canonical correspondence between isomorphism classes of torsors (for a \emph{given} $\sh{G}$) and elements from the cohomology set $\HH^1(Z, \sh{G})$.
Consider a torsor $\sh{P}$ under $\sh{G}$ and an open cover $(U_\lambda)$ of $Z$ such that $\Gamma(U_\lambda, \sh{P}) \neq \emp$ for every $\lambda$;
denote by $p_\lambda$ an element of $\Gamma(U_\lambda, \sh{P})$.
For every pair of indices $\lambda$, $\mu$ such that $U_\lambda \cap U_\mu \neq \emp$, there then exists a unique element $\gamma_{\lambda\mu}$ of $\Gamma(U_\lambda \cap U_\mu, \sh{G})$ such that $\gamma_{\lambda\mu} \cdot (p_\mu|U_\lambda \cap U_\mu) = p_\lambda|U_\lambda \cap U_\mu$;
in addition, if $\lambda$, $\mu$, $\nu$ are three indices such that $U_\lambda \cap U_\mu \cap U_\nu \neq \emp$, then the restrictions $\gamma'_{\lambda\mu}$, $\gamma'_{\mu\nu}$, $\gamma'_{\lambda\nu}$ of $\gamma_{\lambda\mu}$, $\gamma_{\mu\nu}$, $\gamma_{\lambda\nu}$ to $U_\lambda \cap U_\mu \cap U_\nu$ satisfy the condition $\gamma'_{\lambda\nu} = \gamma'_{\lambda\mu} \gamma'_{\mu\nu}$;
in other words, $(\lambda, \mu) \mapsto \gamma_{\lambda\mu}$ is a \emph{$1$-cocycle} of the cover $(U_\lambda)$ with values in $\sh{G}$.
If, for every $\lambda$, $p'_\lambda$ is a second element of $\Gamma(U_\lambda, \sh{P})$, then there exists a unique element $\beta_\lambda \in \Gamma(U_\lambda, \sh{G})$ such that $p'_\lambda = \beta_\lambda \cdot p_\lambda$, and the $1$-cocycle $(\gamma'_{\lambda\mu})$ corresponding to the family $(p'_\lambda)$ is given by $\gamma'_{\lambda\mu} = \beta_\lambda \gamma_{\lambda\mu} \beta_\mu^{-1}$, that is, it is \emph{cohomologous} to $\gamma_{\lambda\mu}$.
Conversely, the data of a $1$-cocycle $(\gamma_{\lambda\mu})$ defines, for every pair $(\lambda, \mu)$, an automorphism $\theta_{\lambda\mu}$ of the sheaf \emph{of sets} $\sh{G}|U_\lambda \cap U_\mu$, namely the right translation by $\gamma_{\lambda\mu}$, and the fact that it is a cocycle shows that we can \emph{glue} the sheaves of sets $\sh{G}|U_\lambda$ via the automorphisms $\theta_{\lambda\mu}$ \sref[0]{0.3.3.1};
we thus obtain a torsor under $\sh{G}$, denoted $\sh{P}$, and if we take for $p_\lambda$ the unit section over $U_\lambda$, then the corresponding $1$-cocycle is none other than the given $1$-cocycle $(\gamma_{\lambda\mu})$;
in addition, if we replace $(\gamma_{\lambda\mu})$ by a $1$-cocycle $\gamma'_{\lambda\mu} = \beta_\lambda \gamma_{\lambda\mu} \beta_\mu^{-1}$ cohomologous to it, then we check immediately that the torsor obtained is isomorphic to $\sh{P}$.

In particular, if $(\gamma_{\lambda\mu})$ is a \emph{$1$-coboundary}, in other words of the form $\gamma_{\lambda\mu} = \beta_\lambda \beta_\mu^{-1}$, then the torsor $\sh{P}$ obtained is \emph{isomorphic to $\sh{G}$} (considered as a torsor under itself by left translations);
we say in this case that $\sh{P}$ is \emph{trivial}, and the converse is evident.

In particular, it follows from \sref[III]{III.1.3.1} that we have:
\end{env}

\begin{proposition}[16.5.16]
\label{IV.16.5.16}
Let $Z$ be an affine scheme, $\sh{G}$ a quasi-coherent $\sh{O}_Z$-module;
then every torsor over $\sh{G}$ is trivial.
\end{proposition}

\oldpage[IV-4]{34}
Returnning to the problem considered in \sref{IV.16.5.13}, we thus obtain:

\begin{proposition}[16.5.17]
\label{IV.16.5.17}
Let $X$, $Y$ be two $S$-preschemes, $Y_0$ a closed subprescheme of $Y$ defined by a quasi-coherent ideal $\sh{I}$ of $\sh{O}_Y$ such that $\sh{I}^2 = 0$, $j : Y_0 \to Y$ the canonical injection.
Let $u_0:Y_0 \to X$ be an $S$-morphism, and $\sh{P}$ the sheaf of sets on $Y$ such that, for every open $U$ of $Y$, $\Gamma(U, \sh{P})$ is the set of $S$-morphisms $u : U \to X$ such that $u_0|U_0 = u \circ (j|U_0)$, where $U_0 = j^{-1}(U)$.
Then there exists on $\sh{P}$ the structure of a pseudo-torsor over the $\sh{O}_{Y_0}$-module $\sh{G} = \shHom_{\sh{O}_{Y_0}}(u_0^*(\Omega_{X/S}^1), \sh{I})$.
\end{proposition}

In particular:

\begin{corollary}[16.5.18]
\label{IV.16.5.18}
With the notation of \sref{IV.16.5.16}, suppose that $Y$ is affine and $\Omega_{X/S}^1$ is of finite presentation;
if there is a open cover $(U_\alpha)$ of $Y$, and, for every index $\alpha$, an $S$-morphism $v_\alpha : U_\alpha \to X$ such that, if $U_\alpha^0 = j^{-1}(U_\alpha)$, we have $v_\alpha \circ (j|U_\alpha^0) = u_0|U_\alpha^0$, then there is an $S$-morphism $u : Y \to X$ such that $u \circ j = u_0$.
\end{corollary}

\begin{proof}
Indeed, $\sh{G}$ is a quasi-coherent $\sh{O}_{Y_0}$-module \sref[I]{I.1.3.12};
by \sref{IV.16.5.16} and the fact that $Y_0$ is then affine, the sheaf $\sh{P}$, which is by hypothesis a torsor over $\sh{G}$, and not only a pseudo-torsor, is \emph{trivial};
but if $w$ is an isomorphism from $\sh{G}$ to $\sh{P}$ (as it is a torsor over $\sh{G}$), the image under $w$ of the zero section of $\sh{G}$ is the $S$-morphism we want.
\end{proof}

\subsection{Sheaf of $p$-differentials and exterior differentials.}
\label{IV.16.6}

\begin{env}[16.6.1]
\label{IV.16.6.1}
Let $f:X \to S$ be a morphism of preschemes.
We define the \emph{sheaf of $p$-differentials of $X$ relative to $S$} ($p$ integer) to be the \emph{$p$'th exterior power} \sref[0]{0.4.1.5} of the $\sh{O}_X$-module $\Omega_{X/S}^1$, denoted by
\[
  \label{IV.16.6.1.1}
  \Omega_{X/S}^p = \bigwedge^p(\Omega_{X/S}^1).
  \tag{16.6.1.1}
\] 
So we have $\Omega_{X/S}^0 = \sh{O}_X$, and $\Omega_{X/S}^p = 0$ for $p < 0$;
the $\Omega_{X/S}^p$ are the homogeneous components of the exterior algebra of $\Omega_{X/S}^1$
\[  
  \label{IV.16.6.1.2}
  \Omega_{X/S}^\bullet = \bigwedge(\Omega_{X/S}^1) = \bigoplus_{p \in \bb{Z}}\bigwedge^p(\Omega_{X/S}^1),  
  \tag{16.6.1.2}
\] 
which is therefore a quasi-coherent graded anti-commutative $\sh{O}_X$-algebra whose elements of degree $1$ are square-zero.
For every affine $U$ of $X$, we have $\Gamma(U, \Omega_{X/S}^\bullet) = \bigwedge (\Gamma(U, \Omega_{X/S}^1))$, where $\Gamma(U, \Omega_{X/S}^1)$ is considered as a $\Gamma(U, \sh{O}_X)$-module.

When $S = \Spec(A)$ and $X = \Spec(B)$ are affines, $B$ being then an $A$-algebra, we have \sref[0]{0.4.1.5} $\Omega_{X/S}^p = (\Omega_{B/A}^p)^\sim$, by putting $\Omega_{B/A}^p = \bigwedge^p\Omega^1_{B/A}$. 
\end{env}

\begin{theorem}[16.6.2]
\label{IV.16.6.2}
There is one and only one endomorphism $d$ of the sheaf of additive groups $\Omega_{X/S}^\bullet$ with the following properties:
\begin{enumerate}
  \item[\rm{(i)}] $d \circ d = 0$.
  \item[\rm{(ii)}] For every open set $U$ of $X$ and every section $f \in \Gamma(U, \sh{O}_X)$ we have $df = d_{X/S}f$. 
\oldpage[IV-4]{35}
  \item[\rm{(iii)}] For every open set $U$ of $X$, every pair of integers $p$, $q$ and every pair of sections $\omega'_p \in \Gamma(U, \Omega_{X/S}^p)$, $\omega''_q \in \Gamma(U, \Omega_{X/S}^q)$, we have
  \[
    \label{IV.16.6.2.1}
    d(\omega'_p \wedge \omega''_q) = (d\omega'_p) \wedge \omega''_q + (-1)^p \omega'_p \wedge d\omega''_q.
    \tag{16.6.2.1}
  \]
\end{enumerate}
Also, $d$ is an endomorphism of graded $\psi^*(\sh{O}_X)$-modules of degree $+1$.
\end{theorem}

\begin{proof}
Suppose that we have proved the existence of an endomorphism $d$.
For every affine open $U$ of $X$, every section of $\Omega_{X/S}^p$ over $U$ is (because of \rm{(ii)}) a linear combination if a finite number of elements of the form $g(df_1 \wedge df_2 \wedge \dots \wedge df_p)$, where $g$ and the $f_i$ are sections of $\sh{O}_X$ over $U$ \sref[0]{0.20.4.7}.
The conditions \rm{(i)} and \rm{(iii)} then show, by induction on $p$, that we necessarily have
\[
  \label{IV.16.6.2.2}
  d(g(df_1 \wedge df_2 \wedge \dots \wedge df_p)) = dg \wedge df_1 \wedge df_2 \wedge \dots \wedge df_p.
  \tag{16.6.2.2}
\]
This therefore proves the \emph{uniqueness} of $d$ and the last claim of the theorem.
By virtue of this uniqueness property, to show the existence of $d$, we can restrict ourselves to the case where $S = \Spec(A)$ and $X = \Spec(B)$ are affines.
Now (Bourbaki, \emph{Alg.}, chap.~III, 3\textsuperscript{rd}ed., \textsection10) to define an $A$-derivation $D$ of degree $+1$ of an exterior algebra $\bigwedge(M)$ (where $M$ is a $B$-module and an $A$-algebra), such derivation taking its values in a \emph{graded anti-commutative} $A$-algebra $C = \bigoplus_{n=0}^\infty C_n$, whose elements of degree $1$ are square-zero, it suffices to give arbitrarily an $A$-derivation  $D_0$ of $B$ in $C_1$ and an $A$-homomorphism $D_1$ of $M$ in $C_2$;
then it exists one and only one $A$-anti-derivation $D$ of $\Lambda(M)$ in $C$ coinciding with $D_0$ in $B$ and $D_1$ in $M$. 

In the present case, $D_0$ is necessarily equal to $d_{B/A}$ by \rm{(ii)};
we reduce to seeing, having \sref{IV.16.6.2.2} in mind, that there is an $A$-homomorphism $u$ of $\Omega_{B/A}^1$ in $\Omega_{B/A}^2$ such that 
\[
  \label{IV.16.6.2.3}
  u(g.df) = dg \wedge df
  \tag{16.6.2.3}
\]
for whichever $f$, $g$ in $A$;
it suffices to show that there is an $A$-homomorphism $v:B \otimes_A \Omega_{B/A}^1 \to \Omega_{B/A}^2$ such that
\[
  \label{IV.16.6.2.4}
  v(g.\omega) = dg \wedge \omega
  \tag{16.6.2.4}
\]
for $g \in B$ and $\omega \in \Omega_{B/A}^1$.
Finally, since $\Omega_{B/A}^1 = \mathfrak{I}/\mathfrak{I}^2$ (where $\mathfrak{I} = \mathfrak{I}_{B/A}$ is the kernel of the canonical homomorphism $B \times_A B \to B$) and that $\Omega_{B/A}^1$ is generated by elements of the form $g.df$, it is enough to define an $A$-homomorphism $w: B \otimes_A ( B \otimes_A B ) \to \Omega_{B/A}^2$ such that
\[
  \label{IV.16.6.2.5}
  w(g' \otimes g \otimes f) = dg' \wedge (g.df)
  \tag{16.6.2.5}
\]
and such that $w$ is zero on the image of $B \otimes A \mathfrak{I}^2$.
Or, since the second member of \sref{IV.16.6.2.5} is $A$-trilinear in $g'$, $g$ and $f$, the existence of $w$ verifying \sref{IV.16.6.2.5} is immediate.
Since, on the other hand, $\mathfrak{I}$ is generated by elements of the form $1 \otimes x - x \otimes 1$ ($x \in B$), we reduce to checking that when $z = (1 \otimes x - x \otimes 1)(1 \otimes y - y \otimes 1)$ we have $w(g'\otimes z) = 0$. 
Or, since $z = 1 \otimes (xy) + (xy) \otimes 1 - x \otimes y - y \otimes x$, the formula \sref{IV.16.6.2.4} shows that it is 
\oldpage[IV-4]{36}
enough to see that we have $d(xy) - x.dy - y.dx = 0$, which is to say that $d$ is a derivation.

It remains to be shown that $d$ verifies the condition \rm{(i)}.
Now, the square of an anti-derivation is a derivation (Bourbaki, \emph{loc. cit.}), and since $\Omega_{B/A}^\bullet$ is generated by $\Omega_{B/A}^1$ as a $B$-algebra, it is enough to verify that $d(dz) = 0$ for $z \in B$ and $x \in \Omega_{B/A}^1$;
in the first case, this follows from the formula \sref{IV.16.6.2.3} when $g = 1$;
for the second, we can restrict ourselves to the case where $z = g.df$ with $f$, $g$ in $B$, and then we have, because of \sref{IV.16.6.2.1} and \sref{IV.16.6.2.3}, 
\[
  d(d(g.df)) = d(dg \wedge df) = (d(dg)) \wedge (df) - (dg) \wedge (d(df)) = 0.
\] 
\end{proof}

\begin{definition}[16.6.3]
\label{IV.16.6.3}
The anti-derivation $d$ defined in \sref{IV.16.6.2} (also denoted by $d_{X/S}$) is called the \emph{exterior differential} on $X$ (relative to $S$). 
\end{definition}

\begin{proposition}[16.6.4]
\label{IV.16.6.4}
For every base change $g:S' \to S$, if we put $X' = X \times_S S'$, the canonical morphism
\[
  \label{IV.16.6.4.1}
  \Omega_{X/S}^\bullet \otimes_S S' \to \Omega_{X'/S'}^\bullet
  \tag{16.6.4.1}
\]
deduced from the isomorphism \sref{IV.16.5.10.1} is bijective.
Also, if $s$ is a section of $\Omega_{X/S}^\bullet$ over an open set $U$ of $X$, $s \otimes 1$ its inverse image, section of $\Omega_{X'/S'}^\bullet$ over the inverse image $U'$ of $U$ in $X'$, we have $d_{X'/S'}(s \otimes 1) = d_{X/S}(s) \otimes 1$.
\end{proposition}

\begin{proof}
The first claim is immediate, the formation of the exterior algebra of a module commutes with extending the scalar ring.
To prove the second, we can, because of \sref{IV.16.6.2.2}, restrict ourselves to the case where $s \in \Gamma(U, \sh{O}_X)$, and in this case the claim has already been proven \sref{IV.16.4.3.7}.
\end{proof}

\begin{env}[16.6.5]
\label{IV.16.6.5}
Suppose that $\Omega_{X/S}^1$ is an locally free $\sh{O}_X$-module of rank $n$ in a point $x$, so that we have $n$ sections $s_i \in \Gamma(U, \sh{O}_X)$ such that the $ds_i$ form a basis for the $\Gamma(U, \sh{O}_X)$-module $\Gamma(U, \Omega_{X/S}^1)$ \sref{IV.16.5.8}.
Then, for every integer $p \geq 1$, the $p$-differentials $ds_{i_1} \wedge ds_{i_2} \wedge \dots \wedge ds_{i_p}$ (for $i_1 \leq i_2 \leq \dots \leq i_p$ elements of $[1,n]$) form a basis of $\binom{n}{p}$ elements of $\Gamma(U, \Omega_{X/S}^p)$ over $\Gamma(U, \sh{O}_X)$.
Also the formula \sref{IV.16.6.2.2} shows that for every section $g \in \Gamma(U, \sh{O}_X)$, we have
\[
  \label{IV.16.6.5.1}
  d(g.ds_{i_1} \wedge ds_{i_2} \wedge \dots \wedge ds_{i_p}) = \sum_k (-1)^r \frac{\partial g}{\partial s_k} ds_{i_1} \wedge \dots \wedge ds_{i_r} \wedge ds_k \wedge ds_{i_{r+1}} \wedge \dots \wedge ds_{i_p}
  \tag{16.6.5.1}
\]
where, in the second member, $k$ varies in the set of the $n - p$ indexes different from the $i_h$, $i_r$ being the biggest index $<k$.

We note that the relation $d(dg) = 0$ for every section $g \in \Gamma(U, \sh{O}_X)$ expresses itself in the form
\[
  D_i(D_j g) = D_j(D_i g) \quad \text{for $i \neq j$};
\]
in other words, the derivations $D_i$ defined in \sref{IV.16.5.7} commute with each other.
\end{env}

\subsection{The ~$\mathcal{P}_{X/S}^n(\mathcal{F})$.}
\label{IV.16.7}

\begin{env}[16.7.1]
\label{IV.16.7.1}
Let $f:X \to S$ be a morphism of preschemes, $\sh{F}$ an $\sh{O}_X$-module.
We denote by $X_{\Delta_f}^{(n)}$ the \emph{n'th infinitesimal neighbourhood} of $X$ via the diagonal morphism
\oldpage[IV-4]{37}
$\Delta_f: X \to X \times_S X$, by $h_n:X_{\Delta_f}^{(n)} \to X \times_S X$ the canonical morphism \sref{IV.16.1.2}, and consider the two composite morphisms
\[
  p_1^{(n)}:X_{\Delta_f}^{(n)} \xrightarrow{h_n} X \times_S X \xrightarrow{p_1} X, \quad 
  p_2^{(n)}:X_{\Delta_f}^{(n)} \xrightarrow{h_n} X \times_S X \xrightarrow{p_2} X
\]
so that, by definition, $p_1^{(n)}$ corresponds to the homomorphism of sheaves of rings $\sh{O}_X \to \sh{P}_{X/S}^n$ which we have chosen to define the $\sh{O}_X$-algebra structure on $\sh{P}_{X/S}^n$ \sref{IV.16.3.5}, and $p_2^{(n)}$ to the homomorphism of sheaves of rings $d^n_{X/S}:\sh{O}_X \to \sh{P}_{X/S}^n$ \sref{IV.16.3.6}.
Since $X_{\Delta_f}^{(n)}$ and $X$ have the same underlying subspace, we can write
\[
  \label{IV.16.7.1.1}
  \sh{P}_{X/S}^n = (p_1^{(n)})_* ((p_2^{(n)})^* (\sh{O}_X)).
  \tag{16.7.1.1}
\]
More generally, we define
\[
  \label{IV.16.7.1.2}
  \sh{P}_{X/S}^n(\sh{F}) = (p_1^{(n)})_* ((p_2^{(n)})^* (\sh{F})).
  \tag{16.7.1.2}
\]
so that $\sh{P}_{X/S}^n = \sh{P}_{X/S}^n(\sh{O}_X)$;
by definition, $\sh{P}_{X/S}^n(\sh{F})$ is an $\sh{O}_X$-module.
\end{env}

\begin{env}[16.7.2]
\label{IV.16.7.2}
If we come back to the definition of the inverse image of modules on ringed spaces \sref[0]{0.4.3.1} and having in mind that $X_{\Delta_f}^{(n)}$ and $X$ have the same underlying space, we see that we can write the definition \sref{IV.16.7.1.2} in the form
\[
  \label{IV.16.7.2.1}
  \sh{P}_{X/S}^n(\sh{F}) = \sh{P}_{X/S}^n \otimes_{\sh{O}_X} \sh{F},
  \tag{16.7.2.1}
\]
but where you have to be careful that, in the interpretation of the symbol $\otimes$, $\sh{P}_{X/S}^n$ is endowed with the structure of $\sh{O}_X$-module defined by \emph{the homomorphism of sheaves of rings $d_{X/S}^n:\sh{O}_X \to \sh{P}_{X/S}^n$}.
It follows immediately from such formula (or directly from \sref{IV.16.7.1.2}) that $\sh{P}_{X/S}^n(\sh{F})$ is canonically equipped with a $\sh{P}_{X/S}^n$-module structure. 
\end{env}

\begin{proposition}[16.7.3]
\label{IV.16.7.3}
\begin{enumerate}
  \item[\rm{(i)}] The functor $\sh{F} \mapsto \sh{P}_{X/S}^n(\sh{F})$ from the category of $\sh{O}_X$-modules to the category of $\sh{P}_{X/S}^n$-modules is right exact, and commutes with arbitrary inductive limits;
  it is exact when $\sh{P}_{X/S}^n$ is flat.
  \item[\rm{(ii)}] If $\sh{F}$ is a quasi-coherent $\sh{O}_X$-module (resp. of finite type, resp. of finite presentation), then $\sh{P}_{X/S}^n(\sh{F})$ is quasi-coherent (resp. of finite type, resp. of finite presentation).
\end{enumerate}
\end{proposition}

\begin{proof}
The claims from \rm{(i)} follow from \sref{IV.16.7.2.1} and the consideration of the symmetry of $\sh{P}_{X/S}^n$ \sref{IV.16.3.4}.
The claims from \rm{(ii)} follow from the right exactness of the functor $\sh{F} \mapsto \sh{P}_{X/S}^n(\sh{F})$.
\end{proof}

\begin{env}[16.7.4]
\label{IV.16.7.4}
The two structures of $\sh{O}_X$-module on $\sh{P}_{X/S}^n$ define in $\sh{P}_{X/S}^n(\sh{F})$ two structures of $\sh{O}_X$-modules, which happen to be permutable, and therefore a $\sh{O}_X$-bimodule structure.
It is convenient to denote the structure coming from the structure homomorphism $\sh{O}_X \to \sh{P}_{X/S}^n$ (chosen in \sref{IV.16.3.5}) on the \emph{left} and the one coming from the homomorphism $d_{X/S}^n: \sh{O}_X \to \sh{P}_{X/S}^n$ on the \emph{right}.
On other words, for every open $U$ of $X$, and every triplet $a \in \Gamma(U, \sh{O}_X)$, $b \in \Gamma(U, \sh{P}_{X/S}^n)$, $t \in \Gamma(U, \sh{F})$, we have by definition
\[
  \label{IV.16.7.4.1}
  a(b \otimes t) = (ab) \otimes t, \quad (b \otimes t)a = (b.d^n a) \otimes t = b \otimes (at) = (d^na).(b \otimes t).
  \tag{16.7.4.1}
\]
The $\sh{O}_X$-module structure coming from the definition \sref{IV.16.7.1.2} is therefore, under these conventions, the \emph{left} $\sh{O}_X$-module structure.
\oldpage[IV-4]{38}
If $\sh{F}$ is a quasi-coherent $\sh{O}_X$-module, then the same is true for $\sh{P}_{X/S}^n(\sh{F})$ for any one of its $\sh{O}_X$-module structures.
If also $\sh{F}$ is of finite type (resp. of finite presentation) and $f:X \to S$ is locally of finite type (resp. locally of finite presentation), $\sh{P}_{X/S}^n(\sh{F})$ is (for any one of its $\sh{O}_X$-module structures) of finite type (resp. of finite presentation), which is a consequence of \sref{IV.16.3.9} and \sref{IV.16.4.22}.
\end{env}

\begin{env}[16.7.5]
\label{IV.16.7.5}
The definition \sref{IV.16.7.2.1} entails the existence of a homomorphism of sheaves of commutative groups
\[
  \label{IV.16.7.5.1}
  d_{X/S, \sh{F}}^n: \sh{F} \to \sh{P}_{X/S}^n(\sh{F}) \quad (\text{also denoted } d_{X/S}^n)
  \tag{16.7.5.1}
\]
such that, in the notations of \sref{IV.16.7.4}, we have
\[
  \label{IV.16.7.5.2}
  d_{X/S, \sh{F}}^n(t) = 1 \otimes t
  \tag{16.7.5.2}
\]
and consequently, because of \sref{IV.16.7.4.1}
\[
  \label{IV.16.7.5.3}
  d_{X/S, \sh{F}}^n(at) = (1 \otimes t) a = (d_{X/S, \sh{F}}^n(t)).a
  \tag{16.7.5.3}
\]
\[
  \label{IV.IV.16.7.5.4}
  d_{X/S, \sh{F}}^n(at) = (d_{X/S, \sh{F}}^n(a)).(1 \otimes t) = (d_{X/S, \sh{F}}^n(a))(d_{X/S, \sh{F}}^n(t)).
  \tag{IV.16.7.5.4}
\]
Therefore it is $\sh{O}_X$-linear for the structure of \emph{right} $\sh{O}_X$-module on $\sh{P}_{X/S}^n(\sh{F})$, and \emph{semilinear} (relative to the homomorphism $\sigma$ \sref{IV.16.3.4}) for the \emph{left} $\sh{O}_X$-module structure.
\end{env}

\begin{proposition}[16.7.6]
\label{IV.16.7.6}
The right $\sh{O}_X$-module $\sh{P}_{X/S}^n(\sh{F})$ is generated by the image of $\sh{F}$ by the homomorphism $d_{X/S, \sh{F}}$.
\end{proposition}

\begin{proof}
This is an immediate consequence of \sref{IV.16.7.5.3} and of the particular case $\sh{F} = \sh{O}_X$ \sref{IV.16.3.8}.
\end{proof}

\begin{env}[16.7.7]
\label{IV.16.7.7}
The canonical homomorphisms of sheaves of rings
\[
  \vphi_{nm}:\sh{P}_{X/S}^m \to \sh{P}_{X/S}^n 
\]
for $n \leq m$ \sref{IV.16.1.2} define, because of \sref{IV.16.7.2.1}, canonical homomorphisms 
\[
  \sh{P}_{X/S}^m(\sh{F}) \to \sh{P}_{X/S}^n(\sh{F}) \quad (n \leq m)
\]
which are homomorphisms of $\sh{O}_X$-bimodules in light of \sref{IV.16.1.6} and \sref{16.7.4.1};
also we have commutative diagrams
\[
  \xymatrix{
    \sh{P}_{X/S}^m(\sh{F}) \ar[rr] && \sh{P}_{X/S}^n(\sh{F}) \\
      & \sh{F} \ar[ul]^-{d_{X/S,\sh{F}}^n} \ar[ur]_-{d_{X/S,\sh{F}}^n}
  }
\]
We have therefore a projective system of $\sh{O}_X$-bimodules ($\sh{P}_{X/S}^n(\sh{F})$), and we define
\[
  \label{IV.16.7.7.1}
  \sh{P}_{X/S}^\infty(\sh{F}) = \varprojlim \sh{P}_{X/S}^n(\sh{F}).
  \tag{16.7.7.1}
\]
Also, this shows that the homomorphisms \sref{IV.16.7.5.1} form a projective system of homomorphisms, and therefore define a canonical homomorphism
\[
  \label{IV.16.7.7.2}
  \xymatrix{
    d_{X/S,\sh{F}}^\infty: \sh{F} \to \sh{P}_{X/S}^\infty(\sh{F}).
  }
  \tag{16.7.7.2}
\]
\end{env}

\oldpage[IV-4]{39}

\begin{env}[16.7.8]
\label{IV.16.7.8}
Let $\sh{F}$, $\sh{G}$ be two $\sh{O}_X$-modules;
it follows immediately from the definition \sref{IV.16.7.2.1} that we have a canonical isomorphism of $\sh{P}_{X/S}^n$-modules
\[
  \label{IV.16.7.8}
  \sh{P}_{X/S}^n(\sh{F} \otimes_{\sh{O}_X} \sh{G}) \isoto \sh{P}_{X/S}^n(\sh{F}) \otimes_{\sh{P}_{X/S}^n} \sh{P}_{X/S}^n(\sh{G})
  \tag{16.7.8}
\]
(Bourbaki, \emph{Alg.}, chap.~II, 3\textsuperscript{rd}ed., \textsection5, n.1, prop. 3).

We conclude in particular (or we see directly from the definition \sref{IV.16.7.2.1}) that if $\sh{F}$ has an $\sh{O}_X$-algebra structure (not necessarily associative), $\sh{P}_{X/S}^n(\sh{F})$ has a canonical $\sh{O}_X$-algebra structure;
the latter is associative (resp. commutative, res. unital, resp. a Lie algebra) if $\sh{F}$ is so.
Also the canonical homomorphisms $\sh{P}_{X/S}^m(\sh{F}) \to \sh{P}_{X/S}^n(\sh{F})$ for $n \leq m$ \sref{IV.16.7.7} are then algebra di-homomorphisms;
similarly, \sref{IV.16.7.5.1} is then an $\sh{O}_X$-algebra homomorphisms when $\sh{P}_{X/S}^n(\sh{F})$ is equipped with the $\sh{O}_X$-algebra structure from its structure of \emph{right} $\sh{O}_X$-module.

With the same notations, we equally have a canonical homomorphisms of $\sh{P}_{X/S}^n$-modules
\[
  \label{IV.16.7.8.2}
  \sh{P}_{X/S}^n(\shHom_{\sh{O}_X}(\sh{F}, \sh{G})) \to \shHom_{\sh{P}_{X/S}^n}(\sh{P}_{X/S}^n(\sh{F}), \sh{P}_{X/S}^n(\sh{G}))
  \tag{16.7.8.2}
\]
(Bourbaki, \emph{Alg.}, chap.~II, 3\textsuperscript{rd}ed., \textsection5, n.3), which is bijective when $\sh{P}_{X/S}^n$ is \emph{locally free of finite type} (\emph{loc. cit.}, prop. 7).
\end{env}

\begin{env}[16.7.9]
\label{IV.16.7.9}
Suppose we are in the situation described in \sref{IV.16.4.1};
then from the canonical homomorphism $P^n(u)$ \sref{IV.16.4.3.3} we deduce immediately a canonical homomorphism of $\sh{O}_{X'}$-bimodules 
\[
  \label{IV.16.7.9.1}
  u^*(\sh{P}_{X/S}^n(\sh{F})) \to \sh{P}_{X'/S'}^n(u^*(\sh{F})).
  \tag{16.7.9.1}
\]
We leave it to the reader to extend the properties seen in \sref{IV.16.4} in the case $\sh{F} = \sh{O}_X$.
\end{env}

\begin{remark}[16.7.10]
\label{IV.16.7.10}
The definition of $\sh{P}_{X/S}^n(\sh{F})$ in the form \sref{IV.16.7.1.2} still makes sense when $\sh{F}$ is a sheaf of sets (the inverse image of a sheaf of sets by $p_2^{(n)}$ being defined in \sref[0]{0.3.7.1});
a variant of this definition allows us to define the ``jet schemes'' (relatively to $S$) for any prescheme $X$.
\end{remark}


\subsection{Differential operators.}
\footnote{
  For a more general formalism, see the expos\'e VII of \cite{IV-42} (due to P. Gabriel).
}
\label{IV.16.8}

\begin{definition}[16.8.1]
\label{IV.16.8.1}
Let $f = (\psi, \theta): X \to S$ be a morphism of preschemes, $\sh{F}$, $\sh{G}$ two $\sh{O}_X$-modules, $n$ an integer $\geq 0$.
We say that a morphism $D: \sh{F} \to \sh{G}$ of sheaves of additive groups is a differential operator of order $\leq n$ (relative to $S$) if there is a homomorphism of $\sh{O}_X$-modules $u: \sh{P}_{X/S}^n(\sh{F}) \to \sh{G}$ (where $\sh{P}_{X/S}^n(\sh{F})$ is equipped with the structure of left $\sh{O}_X$-modules \sref{IV.16.7.4}) such that we have $D = u \circ d_{X/S, \sh{F}}^n$.
\end{definition}

It is clear, because of the existence of canonical morphisms
\[
  \sh{P}_{X/S}^m(\sh{F}) \to \sh{P}_{X/S}^n(\sh{F})
\] 
\oldpage[IV-4]{40}
for $n \leq m$ \sref{IV.16.7.7}, that a differential operator of order $\leq n$ is a differential operator of order $\leq m$ for $n \leq m$.
If $D:\sh{F} \to \sh{G}$ is a differential operator of order $\leq n$, then, for every open set $U$ of $X$, $D|U: \sh{F}|U \to \sh{G}|U$ is a differential operator of order $\leq n$.

We say that a homomorphism $D: \sh{F} \to \sh{G}$ is a \emph{differential operator} (relative to $S$) if, for every $x \in X$, there is an open neighbourhood $U$ of $x$ and an integer $n \geq 0$ such that $D|U: \sh{F}|U \to \sh{G}|U$ is a differential operator of order $\leq n$.
The \emph{order} of a differential operator is the least upper bound of all integers $n$ so that $D$ is a differential operator of order $\leq n$ (and therefore $+\infty$ if there is no such integer);
such order is always finite if $X$ is \emph{quasi-compact}.
The differential operator of order $0$ are exactly the homomorphisms of $\sh{O}_X$-modules $\sh{F} \to \sh{G}$;
the operators of order $<0$ are \emph{zero} by convention.
For $n \geq 0$ a differential operator of order $n$ is not in general a homomorphism of $\sh{O}_X$-modules but always a homomorphism of $\psi^*(\sh{O}_S)$-modules.

When $\sh{F} = \sh{O}_X$, a differential operator of order $\leq 1$ of $\sh{O}_X$ to $\sh{G}$ can be put in the form of $v + D$, where $v: \sh{O}_X \to \sh{G}$ is an $\sh{O}_X$-homomorphism, and $D$ is an \emph{$S$-derivation} \sref{IV.16.5.1} of $\sh{O}_X$ to $\sh{G}$: this results from the structure of $P_{B/A}$ \sref[0]{0.20.4.8}.

\begin{env}[16.8.2]
\label{IV.16.8.2}
To describe in a more precise manner a differential operator of order $\leq n$, $D: \sh{F} \to \sh{G}$, it suffices, for every open set $U$ of $X$ whose image in $S$ is contained in an affine open set $V$, to characterize the homomorphism $D = D_U: \Gamma(U, \sh{F}) \to \Gamma(U, \sh{G})$.
If we put $\Gamma(V, \sh{O}_S) = A$, $\Gamma(U, \sh{O}_X) = B$, so that $B$ is an $A$-algebra, we have $\Gamma(U, \sh{P}_{X/S}^n(\sh{F}) ) = (B \otimes_A B)/\mathfrak{I}^{n+1}$, where we abbreviate $\mathfrak{I} = \mathfrak{I}_{B/A}$. 
Also put $M = \Gamma(U, \sh{F})$, $N = \Gamma(U, \sh{G})$;
then the definition of $D$ means that for every pair $(U,V)$ satisfying the above, the $A$-homomorphism $D:M \to N$ factors through
\[
  M \to ((B \otimes_A B)/\mathfrak{I}^{n+1}) \otimes_B M \xrightarrow{v} N
\]
where the first arrow is the canonical morphism $t \mapsto 1 \otimes t$, and $v$ is a $B$-homomorphism, the structure of $B$-module coming from the first factor (whereas we recall that in the formation of the tensor product over $B$, the structure of $B$-module on $(B \otimes_A B)/\mathfrak{I}^{n+1}$ comes from the second factor $B$).
Note also that the $B$-module $((B \otimes_A B)/\mathfrak{I}^{n+1}) \otimes_B M$ is isomorphic to $(B \otimes_A M)/\mathfrak{I}^{n+1}(B \otimes_A M)$, where $(B \otimes_A M)$ is considered as a $(B \otimes_A B)$ module and its structure of $B$-module comes from $t \mapsto 1 \otimes t$ of $B$ in $B \otimes_A B$.  
Let then $D'$ be the $B$-homomorphism of $B \otimes_A M$ to $N$ such that $D'(b \otimes t) = b D(t)$; then condition of factorization of $D$ is to say that $D'$ must be zero on the $B$-module $\mathfrak{I}^{n+1}(B \otimes_A M)$.
\end{env}

\begin{env}[16.8.3]
\label{IV.16.8.3}
It is clear that the set of differential operators of order $\leq n$ from $\sh{F}$ to $\sh{G}$ forms an additive group, denoted by $\Diff_{X/S}^n(\sh{F}, \sh{G})$;
when $\sh{F} = \sh{G} = \sh{O}_X$, we also write $\Diff_{X/S}^n$ instead of $\Diff_{X/S}^n(\sh{O}_X, \sh{O}_X)$.

We have seen \sref{IV.16.8.1}, that given two open sets $U \supset V$ of $X$, we have a restriction homomorphism
\[
  \Diff_{X/S}^n(\sh{F}|U, \sh{G}|U) \to \Diff_{X/S}^n(\sh{F}|V, \sh{G}|V)
\]
\oldpage[IV-4]{41}
from which we deduce that $U \mapsto \Diff_{U/S}^n(\sh{F}|U, \sh{G}|U)$ is a presheaf of additive groups;
in fact, it is actually a \emph{sheaf}, since for an open set $U$ varying in $X$, the homomorphisms $u \mapsto u \circ d_{U/S, \sh{F}|U}^n$ are \emph{isomorphisms} of sheaves of additive groups
\[
  \label{IV.16.8.3.1}
  \Hom_{\sh{O}_U}(\sh{P}_{U/S}^n(\sh{F}|U), \sh{G}|U) \isoto \Diff_{U/S}^n(\sh{F}|U, \sh{G}|U),
  \tag{16.8.3.1}
\]
because of the fact that the image of $\sh{F}$ by $d_{X/S, \sh{F}}^n$ generates $\sh{P}_{X/S}^n(\sh{F})$ \sref{IV.16.7.6}.
We denote this sheaf by $\shDiff_{X/S}^n(\sh{F}, \sh{G})$, and we have:
\end{env}

\begin{proposition}[16.8.4]
\label{IV.16.8.4}
The isomorphisms \sref{IV.16.8.3.1} define an isomorphism of sheaves of additive groups
\[
  \label{IV.16.8.4.1}
  \shHom_{\sh{O}_X}(\sh{P}_{X/S}^n(\sh{F}), \sh{G}) \isoto \shDiff_{X/S}^n(\sh{F}, \sh{G}).
  \tag{18.4.1}
\]
\end{proposition}

When $\sh{F} = \sh{G} = \sh{O}_X$, we also write $\shDiff_{X/S}^n$ instead of $\shDiff_{X/S}^n(\sh{O}_X, \sh{O}_X)$;
it results from \sref{IV.16.6.4} that $\shDiff_{X/S}^n$ is the \emph{dual} of $\sh{P}_{X/S}^n$;
so we also write $\langle t, D \rangle$ instead of $u(t)$ if $t$ is a section of $\sh{P}_{X/S}^n$ over an open set and if $u$ is a homomorphism from $\sh{P}_{X/S}^n$ to $\sh{O}_X$ corresponding to $D$.

\begin{env}[16.8.5]
\label{IV.16.8.5}
When $\sh{P}_{X/S}^n(\sh{F})$ has a $\sh{O}_X$-bimodule structure \sref{IV.16.7.4}, we deduce canonicallly a $\sh{O}_X$-bimodule structure on $\shHom_{\sh{O}_X}(\sh{P}_{X/S}^n(\sh{F}), \sh{G})$, and therefore on $\shDiff_{X/S}^n(\sh{F}, \sh{G})$ because of \sref{IV.16.8.4.1}.
More precisely, to the left $\sh{O}_X$-module structure on $\sh{P}_{X/S}^n(\sh{F})$ corresponds, because of \sref{IV.16.8.1}, the left $\sh{O}_X$-module structure on $\shDiff_{X/S}^n(\sh{F}, \sh{G})$ explained as follows:
for every open set $U$ of $X$, every section $a \in \Gamma(U, \sh{O}_X)$ and every differential operator $D: \sh{F}|U \to \sh{G}|U$, $aD$ is the differential operator which, for every section $t \in \Gamma(U, \sh{F})$, makes correspond the section
\[
  \label{IV.16.8.5.1}
  (aD)(t) = a(D(t))
  \tag{16.8.5.1}
\]
of $\Gamma(U, \sh{G})$.
Similarly, the right $\sh{O}_X$-module structure on $\shDiff_{X/S}^n(\sh{F}, \sh{G})$ is made explicit as follows:
under the same notations as above, $Da$ is the operator which, to every $t \in \Gamma(U, \sh{F})$, makes correspond the section
\[
  \label{IV.16.8.5.2}
  (Da)(t) = D(at).
  \tag{16.8.5.2}
\]
\end{env}

\begin{proposition}[16.8.6]
\label{IV.16.8.6}
If $f:X \to S$ is a morphism locally of finite presentation, $\sh{F}$ a quasi-coherent $\sh{O}_X$-module of finite presentation and $\sh{G}$ a quasi-coherent $\sh{O}_X$-module, then $\shDiff_{X/S}^n(\sh{F}, \sh{G})$ is a quasi-coherent $\sh{O}_X$-module for any of the structures defined in \sref{IV.16.8.5}.
\end{proposition}

\begin{proof}
The proposition follows from the fact that, under these hypothesis, $\sh{P}_{X/S}^n$ is a quasi-coherent $\sh{O}_X$-module of finite presentation \sref{IV.16.7.4} and of \sref[I]{I.3.12}
\end{proof}

\begin{env}[16.8.6]
\label{IV.16.8.6}
The set of differential operators (of unspecified order \sref{IV.16.8.1}) is denoted by $\Diff_{X/S}(\sh{F}, \sh{G})$;
we also see as in \sref{IV.16.8.3} that $U \mapsto \Diff_{U/S}(\sh{F}|U, \sh{G}|U)$ is a sheaf of additive groups, which we will denote by $\shDiff_{X/S}(\sh{F}, \sh{G})$.
It is immediate that $\shDiff_{X/S}(\sh{F}, \sh{G})$ is the reunion of the increasing filtered family of its subsheaves $\shDiff_{X/S}^n(\sh{F}, \sh{G})$;
if $X$ is quasi-compact, $\Diff_{X/S}(\sh{F}, \sh{G})$ 
\oldpage[IV-4]{42}
is similarly the union of its subgroups $\Diff_{X/S}^n(\sh{F}, \sh{G})$ \sref{IV.16.8.1}.
The $\sh{O}_X$-bimodule structure on the $\shDiff_{X/S}^n(\sh{F}, \sh{G})$ induce therefore a $\sh{O}_X$-bimodule structure on $\shDiff_{X/S}(\sh{F}, \sh{G})$, further explained in \sref{IV.16.8.5.1} and \sref{IV.16.8.5.2}.

Note that, for $n \leq m$, we have a commutative diagram
\[
  \label{IV.16.8.7.1}
  \xymatrix{
    \shHom_{\sh{O}_X}(\sh{P}_{X/S}^n(\sh{F}), \sh{G}) \ar[r]^-{\sim} \ar[d] & \shDiff_{X/S}^n(\sh{F}, \sh{G}) \ar[d] \\
    \shHom_{\sh{O}_X}(\sh{P}_{X/S}^m(\sh{F}), \sh{G}) \ar[r]^-{\sim} & \shDiff_{X/S}^m(\sh{F}, \sh{G})
  }
  \tag{16.8.7.1}
\] 
where the horizontal arrows are the isomorphisms \sref{IV.16.8.4.1} and the horizontal arrow on the left comes from the canonical morphism $\sh{P}_{X/S}^m(\sh{F}) \to \sh{P}_{X/S}^n(\sh{F})$ \sref{IV.16.7.7}.
For every open set $U$ of $X$, we then endow $\Gamma(U, \sh{P}_{X/S}^\infty(\sh{F})) = \varprojlim \Gamma(U, \sh{P}_{X/S}^n(\sh{F}))$ of the projective limit topology of the discrete topologies on $\Gamma(U, \sh{P}_{X/S}^n(\sh{F}))$, which defines on $\Gamma(U, \sh{P}_{X/S}^\infty(\sh{F}))$ a topological $\Gamma(U, \sh{O}_X)$-bimodule structure, so that $\sh{P}_{X/S}^\infty(\sh{F})$ shows itself as a sheaf valued in the category of topological commutative groups \sref[0]{0.3.2.6}.
So \cite[II, 1.11]{I-4}, the limit of the inductive system of sheaves of commutative groups ($\shHom_{\sh{O}_X}(\sh{P}_{X/S}^n(\sh{F}), \sh{G})$) is precisely the sheaf of \emph{continuous} germs of homomorphisms from $\sh{P}_{X/S}^\infty(\sh{F})$ to $\sh{G}$ (the latter equipped with the discrete topology):
the continuous homomorphisms $\Gamma(U, \sh{P}_{X/S}^\infty(\sh{F}))$ into the discrete group $\sh{G}$ indeed correspond bijectively to the inductive systems of group homomorphisms $\Gamma(U, \sh{P}_{X/S}^n(\sh{F})) \to \Gamma(U, \sh{G})$.
We can furthermore express \sref{IV.16.8.4} by saying there is a canonical isomorphism
\[
  \shHomcont_{\sh{O}_X} (\sh{P}_{X/S}^\infty(\sh{F}), \sh{G}) \isoto \shDiff_{X/S}(\sh{F}, \sh{G})
\]
where the first member denotes the sheaf of germs of continuous homomorphisms from $\sh{P}_{X/S}^\infty(\sh{F})$ to $\sh{G}$.
\end{env}

\begin{proposition}[16.8.8]
\label{IV.16.8.8}
Let $\sh{F}$, $\sh{G}$ be two $\sh{O}_X$-modules, $D: \sh{F} \to \sh{G}$ a homomorphism of $\psi^*(\sh{O}_S)$-modules, $n$ an integer $\geq 0$.
The following conditions are equivalent:
\begin{enumerate}
  \item[(a)] $D$ is a differential operator of order $\leq n$.
  \item[(b)] For all sections $a$ of $\sh{O}_X$ over an open set $U$, the homomorphism $D_a: \sh{F}|U \to \sh{G}|U$ such that, for every section $t$ of $\sh{F}$ over an open set $V \subset U$, we have
  \[  
    \label{IV.16.8.8.1}
    D_a(t) = D(at) - a D(t)
    \tag{16.8.8.1}
  \]
  is a differential operator of order $\leq n-1$.
  \item[(c)] For every open set $U$ of $X$, every family $(a_i)_{1 \leq i \leq n+1}$ of $n+1$ sections of $\sh{O}_X$ over $U$ and every section $t$ of $\sh{F}$ over $U$, we have the identity
  \[
    \label{IV.16.8.8.2}
    \sum_{H \subset I_{n+1}} (-1)^{\operatorname{Card}(H)} (\prod_{i \in H} a_i) D ((\prod_{i \notin H} a_i ) t ) = 0
    \tag{16.8.8.2}
  \]
  (where $I_{n+1}$ is the interval $1 \leq i \leq n+1$ of $\mathbb{N}$).
\end{enumerate}
\oldpage[IV-4]{43}
\end{proposition}

\begin{proof}
Lets prove first the equivalence of (a) and (b).
By definition, to prove that $D$ is a differential operator of order $\leq n$, it suffices to prove so for the restriction $D|U: \sh{F}|U \to \sh{G}|U$ to any affine open set $U$ of $X$, and on the other hand the property (c) is valid for every open set $U$ if it is so on every affine open set.
We can therefore restrict ourselves to the case where $S = \Spec(A)$ and $X = \Spec(B)$ are affines.
Because of \sref{IV.16.8.2} (where we use the same notations), the condition (a) means that the $A$-homomorphism $D': B \otimes_A M \to N$ such that $D'(b \otimes t) = b D(t)$ is zero on $\mathfrak{I}^{n+1}(B \otimes_A M)$, which is, because of \sref[0]{0.20.4.4}, equivalent to saying that $D'$ is annihilates every element of the form
\[
  (\prod_{i=1}^{n+1} (a_i \otimes 1 - 1 \otimes a_i) ) . (1 \otimes t)
\]
 where $a_i \in B$ and $t \in M$.
Now, this element can be written as $\sum_{H \subset I_{n+1}} (\prod_{i \in H} a_i) \otimes ((\prod_{i \notin H} a_i ) t )$, and the image under $D'$ of this element is the first member of \sref{IV.16.8.8.2}, which proves the equivalence of (a) and (c).

Let us prove now the equivalence of (b) and (c).
Let us reason by induction on $n$, the statement being trivial for $n=0$.
Writing $a_{n+1}$ instead of $a$ in the condition (b), we see, by the induction hypothesis, that condition (b) means that for every family $(a_i)_{1 \leq i \leq n}$ of $n$ sections of $\sh{O}_X$ over $U$ and every section $t$ of $\sh{F}$ over U, that
\[
  \sum_{H' \subset I_{n+1}} (-1)^{\operatorname{Card}(H')}  (\prod_{i \in H'} a_i) D_{a_{n+1}} ((\prod_{i \notin H'} a_i ) t ) = 0.
\]
But if we replace on this relation $D_{a_{n+1}}$ by the definition \sref{IV.16.8.8.1}, we check immediately that we have, up to sign, the first member of \sref{IV.16.8.8.2};
from which we conclude.
\end{proof}

\begin{proposition}[16.8.9]
\label{IV.16.8.9}
If $D: \sh{F} \to \sh{G}$ is a differential operator of order $\leq n$, and $D': \sh{G} \to \sh{H}$ a differential operator of order $\leq n'$, then $D' \circ D: \sh{F} \to \sh{H}$ is a differential operator of order $\leq n + n'$.
\end{proposition}

\begin{proof}
By hypothesis, we can write $D = u \circ d_{X/S, \sh{F}}^n$ and $D' = v \circ d_{X/S, \sh{G}}^n$, where $u: \sh{P}_{X/S}^n \otimes_{\sh{O}_X} \sh{F} \to \sh{G}$ and $v: \sh{P}_{X/S}^{n'} \otimes_{\sh{O}_X} \sh{G} \to \sh{H} $ are $\sh{O}_X$-homomorphisms.
It all comes down to showing that the composite homomorphism of sheaves of additive groups
\[
  \xymatrix{
    \sh{F} \ar[r]^-{d_{X/S, \sh{F}}^n} & \sh{P}_{X/S}^n \otimes_{\sh{O}_X} \sh{F} \ar[r]^-u & \sh{G} \ar[r]^-{d_{X/S, \sh{G}}^{n'}} & \sh{P}_{X/S}^{n'} \otimes_{\sh{O}_X} \sh{G}
  }
\]
factors as
\[
  \xymatrix{
    \sh{F} \ar[r]^-{d_{X/S, \sh{F}}^{n + n'}} & \sh{P}_{X/S}^{n + n'} \otimes_{\sh{O}_X} \sh{F} \ar[r]^-w &  \sh{P}_{X/S}^{n'} \otimes_{\sh{O}_X} \sh{G}
  }
\]
where $w$ is an $\sh{O}_X$-homomorphism.
It suffices to prove the 
\begin{lemma}[16.8.9.1]
\label{IV.16.8.9.1}
There is one and only one $\sh{O}_X$-homomorphism
\[
  \label{IV.16.8.9.2}
  \delta: \sh{P}_{X/S}^{n + n'} \to \sh{P}_{X/S}^{n'}(\sh{P}_{X/S}^n) = \sh{P}_{X/S}^{n'} \otimes_{\sh{O}_X} \sh{P}_{X/S}^n 
  \tag{16.8.9.2}
\]
\oldpage[IV-4]{44}
making the following diagram commute
\[
  \label{IV.16.8.9.3}
  \xymatrix{
    \sh{O}_X \ar[r]^-{d_{X/S}^{n + n'}} \ar[d]_-{d_{X/S}^{n}} & \sh{P}_{X/S}^{n + n'} \ar[d]^-\delta \\
    \sh{P}_{X/S}^n \ar[r]_-{d_{X/S, \sh{P}_{X/S}^n}^{n'}} & \sh{P}_{X/S}^{n'} (\sh{P}_{X/S}^n)
  }
  \tag{16.8.9.3}
\]
\end{lemma}

We will then have, indeed, a commutative diagram deduced from \sref{IV.16.8.9.3} by tensorization with $\sh{F}$
\[
  \xymatrix{
    \sh{F} \ar[r]^-{d_{X/S, \sh{F}}^{n + n'}} \ar[d]_-{d_{X/S, \sh{F}}^{n}} & \sh{P}_{X/S}^{n + n'}(\sh{F}) \ar[d]^-{\delta \otimes 1} \\
    \sh{P}_{X/S}^n(\sh{F}) \ar[r]_-{d_{X/S, \sh{P}_{X/S}^n(\sh{F})}^{n'}} & \sh{P}_{X/S}^{n'} (\sh{P}_{X/S}^n((\sh{F})))
  }
\]
and on the other hand, we verify immediately the from definition \sref{IV.16.7.5} that the diagram
\[
  \xymatrix{
    \sh{P}_{X/S}^n(\sh{F}) \ar[r]^-{u} \ar[d]_-{d_{X/S, \sh{P}_{X/S}^n(\sh{F})}^{n'}} & \sh{G} \ar[d]^-{d_{X/S, \sh{G}}^{n'}} \\
    \sh{P}_{X/S}^{n'} (\sh{P}_{X/S}^n((\sh{F}))) \ar[r]_-{1 \otimes u} & \sh{P}_{X/S}^{n'}(\sh{G})
  }
\]
is commutative.
We finish the proof by taking $w$ to be the composite $\sh{O}_X$-homomorphism
\[
  \xymatrix{
    \sh{P}_{X/S}^{n+n'}(\sh{F}) \ar[r]^-{\delta \otimes 1} & \sh{P}_{X/S}^{n'}(\sh{P}_{X/S}^n(\sh{F})) \ar[r]^-{1 \otimes u} & \sh{P}_{X/S}^{n'}(\sh{G}).
  }
\]
\end{proof}

\begin{proof}[Proof (16.8.9.3)]
It remains to prove the lemma \sref{IV.16.8.9.3}.
Considering \sref{IV.16.7.6}, which proves the uniqueness of $\delta$, we are brought back to the case where $S = \Spec(A)$ and $X = \Spec(B)$ are affines;
letting $\mathfrak{I} = \mathfrak{I}_{B/A}$, it suffices to define a canonical homomorphism of $B$-modules
\[
  \vphi: (B \otimes_A B)/\mathfrak{I}^{n+n'+1} \to ((B \otimes_A B)/\mathfrak{I}^{n' + 1}) \otimes_B ((B \otimes_A B)/\mathfrak{I}^{n + 1})
\]
the $B$-module structure of the two members coming from the first $B$ factor;
recall that on tensor product of the second member, $(B \otimes_A B)/\mathfrak{I}^{n' + 1}$ must be considered
\oldpage[IV-4]{45}
as a right $B$-module by its second $B$ factor , and $(B \otimes_A B)/\mathfrak{I}^{n + 1}$ as a left $B$-module by its first $B$ factor \sref{IV.16.7.2}.
It is the same to define a homomorphism of $B$-modules
\[
  \vphi_0: B \otimes_A B \to ((B \otimes_A B)/\mathfrak{I}^{n' + 1}) \otimes_B ((B \otimes_A B)/\mathfrak{I}^{n + 1})
\]
and prove it is zero on $\mathfrak{I}^{n+n'+1}$.
Now, we immediately define a homomorphism by the condition that
\[
  \vphi_0(b \otimes b') = \pi_{n'}(b \otimes 1) \otimes \pi_n(1 \otimes b') \quad \text{for $b$, $b'$ in $B$}
\]
under the notations of \sref{IV.16.3.7}.
Also, it is immediate that $\vphi_0$ is a homomorphism of \emph{rings}.
Now, we can write 
\[
  \vphi_0(b \otimes 1 - 1 \otimes b) = \pi_{n'}(b \otimes 1 - 1 \otimes b) \otimes \pi_n(1 \otimes 1) + \pi_{n'}(1 \otimes b) \otimes \pi_n(1 \otimes 1) - \pi_{n'}(1 \otimes 1) \otimes \pi_n(1 \otimes b)
\]
and we have 
\[
  \pi_{n'}(1 \otimes b) \otimes \pi_n(1 \otimes 1) = \pi_{n'}(1 \otimes 1) b \otimes \pi_n(1 \otimes 1) = \pi_{n'}(1 \otimes 1) \otimes b \pi_n(1 \otimes 1) = \pi_{n'}(1 \otimes 1) \otimes \pi_n(b \otimes 1)
\]
from which, finally
\[
  \label{IV.16.8.9.4}
  \vphi_0(b \otimes 1 - 1 \otimes b) = \pi_{n'}(b \otimes 1 - 1 \otimes b) \otimes \pi_n(1 \otimes 1) + \pi_{n'}(1 \otimes 1) \otimes \pi_n(b \otimes 1 - 1 \otimes b). 
  \tag{16.8.9.4}
\]
A product of $n + n' + 1$ of terms of the form \sref{IV.16.8.9.4} is therefore necessarily zero, because the same is true for the product of $n+1$ terms of the form $\pi_n(b \otimes 1 - 1 \otimes b)$ and of $n' + 1$ terms of the form $\pi_{n'}(b \otimes 1 - 1 \otimes b)$.
The conclusion therefore results from \sref[0]{0.20.4.4}.
\end{proof}

\begin{corollary}[16.8.10]
\label{IV.16.8.10}
The sheaf $\shDiff_{X/S}(\sh{O}_X, \sh{O}_X)$ (also denoted $\shDiff_{X/S}$) is canonically endowed with the structure of sheaf of rings, and the $\shDiff_{X/S}^n$ form an increasing filtration compatible with such structure.
\end{corollary}

In particular, $\shDiff_{X/S}^0$ is a sheaf of subrings of $\shDiff_{X/S}$, which is canonically identified with $\sh{O}_X$ \sref{IV.16.8.1}.
The formulas \sref{IV.16.8.5.1} and \sref{IV.16.8.5.2} show that the structure of $\sh{O}_X$-bimodule of $\shDiff_{X/S}$ comes from the multiplication on the left and on the right by sections of $\sh{O}_X$ considered as a sheaf of subrings of $\shDiff_{X/S}$.

\begin{remarks}[16.8.11]
\label{IV.16.8.11}
\begin{enumerate}
  \item[\rm{(i)}] Suppose that $\sh{F} = \oplus_{\lambda \in L} \sh{F}_\lambda$;
  then it is clear \sref{IV.16.7.2.1} that $\sh{P}_{X/S}^n(\sh{F}) = \oplus_{\lambda \in L} \sh{P}_{X/S}^n(\sh{F}_\lambda)$;
  since the functor $U \mapsto \Gamma(U, \sh{F})$ commutes with the formation of arbitrary direct sums, $d_{X/S, \sh{F}}^n$ is the homomorphism whose restriction to each $\sh{F}_\lambda$ is $d_{X/S, \sh{F}_\lambda}^n: \sh{F}_\lambda \to \sh{P}_{X/S}^n(\sh{F}_\lambda)$;
  then we conclude immediately that we have 
  \[
    \Diff_{X/S}^n (\sh{F}, \sh{G}) = \prod_{\lambda \in L} \Diff_{X/S}^n (\sh{F}_\lambda, \sh{G}),
  \]
  and therefore also \sref[0]{0.3.2.6}
  \[
    \shDiff_{X/S}^n (\sh{F}, \sh{G}) = \prod_{\lambda \in L} \shDiff_{X/S}^n (\sh{F}_\lambda, \sh{G}).
  \]
  Moreover, if $\sh{G} = \prod_{\mu \in M} \sh{G}_\mu$ \sref[0]{0.3.2.6}, we have
  \[
    \Hom_{\sh{O}_X}(\sh{P}_{X/S}^n(\sh{F}), \sh{G}) = \prod_{\mu \in M} \Hom_{\sh{O}_X}(\sh{P}_{X/S}^n(\sh{F}), \sh{G}_\mu),
  \]
  \oldpage[IV-4]{46}
  every homomorphism $u$ from $\sh{P}_{X/S}^n(\sh{F})$ to $\sh{G}$ corresponds bijectively to the family of its composites $u_\mu: \sh{P}_{X/S}^n(\sh{F}) \to \sh{G} \to \sh{G}_\mu$.
  We have therefore
  \[
    \Diff_{X/S}^n(\sh{F}, \sh{G}) = \prod_{\mu \in M} \Diff_{X/S}^n(\sh{F}, \sh{G}_\mu),
  \]
  and consequently also
  \[
    \shDiff_{X/S}^n(\sh{F}, \sh{G}) = \prod_{\mu \in M} \shDiff_{X/S}^n(\sh{F}, \sh{G}_\mu).
  \]
  \item[\rm{(ii)}]
  So far, we have hardly encountered differential operators $\sh{F} \to \sh{G}$ where $\sh{F}$ and $\sh{G}$ are not locally free of finite rank, in which case the structure is reduced locally, because of \rm{(i)}, to the case of the sheaf $\shDiff_{X/S}$;
  the latter will be studied later \sref{IV.16.11} in a particular case.
\end{enumerate}
\end{remarks}

\subsection{Regular and quasi-regular immersions}
\label{IV.16.9}

\begin{definition}[16.9.1]
\label{IV.16.9.1}
Let $X$ be a ringed space.
We say that an ideal $\sh{I}$ of $\sh{O}_X$ is regular (resp. quasi-regular) if, for every point $x \in \Supp(\sh{O}_X/\sh{I})$, there is an open neighbourhood of $x$ in $X$ and a regular sequence \sref[0]{0.15.2.2} (resp. quasi-regular \sref[0]{0.15.2.2}) of elements of $\Gamma(U, \sh{O}_X)$ which generates $\sh{I}|U$. 
\end{definition}

We say that a regular (resp. quasi-regular) sequence of sections of $\sh{O}_X$ over $U$ which generates $\sh{I}|U$ is called a \emph{regular system} (resp. \emph{quasi-regular system}) of generators of $\sh{I}|U$.

\begin{definition}[16.9.2]
\label{IV.16.9.2}
Let $j: Y \to X$ be an immersion of preschemes and let $U$ be an open set such that $j(Y) \subset U$ and that $j$ is a closed immersion of $Y$ in $U$.
We say that $j$ is regular (resp. quasi-regular) if the closed subprescheme $j(Y)$ of $U$ associated to $j$ is defined by a regular (resp. quasi-regular) ideal of $\sh{O}_U$ (condition independent of the choice of $U$).
\end{definition}

We say that a subprescheme $Y$ of a prescheme $X$ is \emph{regularly immersed} (resp. \emph{quasi-regularly immersed}) if the canonical injection $j:Y \to X$ is a regular immersion (resp. quasi-regular immersion). 
If $Y$ is a closed subprescheme and $\sh{I}$ is the ideal of $\sh{O}_X$ that defines $Y$, it is the same as asking $\sh{I}$ to be \emph{regular} (resp. \emph{quasi-regular}).

For example, if $A$ is an \emph{integral} ring, $f$ and element $\neq 0$ of $A$, the closed subprescheme $V(f)$ of $\Spec(A)$ (isomorphic to $\Spec(A/(f))$) is \emph{regularly immersed} in $\Spec(A)$.

Every regular ideal is quasi-regular \sref[0]{0.15.2.2};
every regular immersion is quasi-regular (cf. \sref{IV.16.9.11} for a converse).  

\begin{proposition}[16.9.3]
\label{IV.16.9.3}
Let $X$ be a ringed space, $\sh{I}$ an ideal of $\sh{O}_X$, $(f_i)_{1 \leq i \leq m}$ a finite sequence of sections of $\sh{O}_X$ over $X$ generating $\sh{I}$.
For the $(f_i)$ to be a quasi-regular sequence $\sref[0]{0.15.2.2}$, it is necessary and sufficient that the following conditions are verified:
\begin{enumerate}
  \item[\rm{(i)}] The canonical images of $f_i$ in $\sh{I}/\sh{I}^2$ form a basis of this $\sh{O}_X/\sh{I}$-module.
  \item[\rm{(ii)}] The canonical surjective homomorphism \sref{IV.16.1.2.2}
  \[
    \bb{S}_{\sh{O}_X/\sh{I}}^\bullet (\sh{I}/\sh{I}^2) \to \shGr_{\sh{I}}^\bullet(\sh{O}_X) 
  \]
  is bijective.
\end{enumerate}

Also, if this is true, then every sequence $(f'_i)_{1 \leq i \leq n}$ of $n$ sections of $\sh{I}$ over $X$ which generates $\sh{I}$ is quasi-regular.
\end{proposition}

\oldpage[IV-4]{47}

\begin{proof}
The two conditions stated above only translate the ones in \sref[0]{0.15.2.2}, given the definition of the canonical homomorphisms \sref{IV.16.2.1.1}.
The last claim follows from the fact that, if a module $M$ over a commutative ring $A$ admits a basis of $n$ elements, every system of $n$ generators of $M$ is a basis (Bourbaki, \emph{Alg. comm.}, chap.~II, \textsection3, cor. 5 of th. 1).
\end{proof}

\begin{corollary}[16.9.4]
\label{IV.16.9.4}
Let $X$ be a locally ringed space, $\sh{I}$ an ideal of $\sh{O}_X$.
For $\sh{I}$ to be quasi-regular, it is necessary and sufficient that it verifies the following conditions:
\begin{enumerate}
  \item[\rm{(i)}] $\sh{I}$ is of finite type.
  \item[\rm{(ii)}] $\sh{I}/\sh{I}^2$ is a locally free $\sh{O}_X/\sh{I}$-module.
  \item[\rm{(iii)}] The canonical homomorphism
  \[
    \label{IV.16.9.4.1}
    \bb{S}_{\sh{O}_X/\sh{I}}^\bullet (\sh{I}/\sh{I}^2) \to \shGr_{\sh{I}}^\bullet(\sh{O}_X) 
    \tag{16.9.4.1}
  \]
  is bijective.
\end{enumerate}
\end{corollary}

\begin{proof}
The necessity of the conditions follows immediately from \sref{IV.16.9.3}.
To see that they are sufficient, it is enough to show, because of \sref{IV.16.9.3}, that if, in a point $x \in \Supp(\sh{O}_X/\sh{I})$, there is a neighbourhood $U$ of $x$ in $X$ and $n$-sections $f_i$ $(1 \leq i \leq n)$ of $\sh{I}$ over $U$ such that the image in $\sh{I}/\sh{I}^2$ form a basis of $(\sh{I}/\sh{I}^2)|U$ over $(\sh{O}_X/\sh{I})|U$, so there will be a neighbourhood $V \subset U$ of $x$ such that the $f_i|V$ generate $\sh{I}|V$.
Now, by hypothesis, we have $\sh{I}_x \neq \sh{O}_x$, so that $\sh{I}_x$ is contained in the maximal ideal of $\sh{O}_X$; 
since $\sh{I}_x$ is a $\sh{O}_x$ module of finite type and the classes of $(f_i)_x$ in $\sh{I}_x/\sh{I}_x^2$ generate this $(\sh{O}_x/\sh{I}_x)$-module, Nakayama's lemma shows that the $(f_i)_x$ generate $\sh{I}_x$.
Since $\sh{I}$ is of finite type, we conclude by \sref[0]{0.5.2.2}.
\end{proof} 

\begin{corollary}[16.9.5]
\label{IV.16.9.5}
Let $X$ be a locally ringed space, $\sh{I}$ a quasi-regular ideal of $\sh{O}_X$, $(f_i)_{1 \leq i \leq n}$ a sequence of sections of $\sh{I}$ over $X$, $x$ a point of $\Supp(\sh{O}_X/\sh{I})$. 
The following conditions are equivalent:
\begin{enumerate}
  \item[(a)] There is a neighbourhood of $x$ in $X$ such that $f_i|U$ form a quasi-regular sequence of elements $\Gamma(U, \sh{O}_X)$ generating $\sh{I}|U$.
  \item[(b)] The $(f_i)_x$ form a system of generators of $\sh{I}_x$ whose size is as small as possible.
  \item[(b')] The $(f_i)_x$ form a minimal set of generators of $\sh{I}_x$.
  \item[(c)] If $\bar f_i$ is the canonical image of $f_i$ in $\Gamma(X, \sh{I}/\sh{I}^2)$, the $(\bar f_i)_x$ form a basis for the $(\sh{O}_x/\sh{I}_x)$-module $\sh{I}_x/\sh{I}_x^2$.
\end{enumerate}
\end{corollary}

\begin{proof}
By hypothesis, $\sh{O}_x$ is a local ring, $\sh{I}_x$ an ideal of finite type of $\sh{O}_x$ contained in the maximal ideal of $\sh{O}_x$;
the equivalence of (b), (b') and (c) results from Nakayama's lemma (Bourbaki, \emph{Alg. comm.}, chap.~II, \textsection3, n\textsuperscript{o}2, prop. 5).
It is clear that (a) implies (c) because of \sref{IV.16.9.3};
on the other hand, from \sref[0]{0.5.2.2} it follows that if condition (c) is verified (and therefore so is (b)), there is a neighbourhood $U$ of $x$ in $X$ such that $(\sh{I}/\sh{I}^2)|U$ has constant rank equal to $n$, and that the $f_i|U$ generate $\sh{I}|U$;
it suffices now to apply the last assertion of \sref{IV.16.9.3} to U.
\end{proof}

\begin{remarks}[16.9.6]
\label{IV.16.9.6}
\begin{enumerate}
  \item[\rm{(i)}] Under the general hypothesis of \sref{IV.16.9.5}, for the sequence $(f_i)$ to generate $\sh{I}$, it is not enough that the $(\bar f_i)_y$ form a basis of the $(\sh{O}_y/\sh{I}_y)$-module $(\sh{I}_y/\sh{I}_y^2)$ for all $y \in X$.
  We have an example by 
  \oldpage[IV-4]{48}
  taking  $X = \Spec(A)$, where $A$ is a Dedekind ring, and $\sh{I} = \widetilde{\mathfrak{I}}$, where $\mathfrak{I}$ is a \emph{non-principal} ideal of $A$;
  then indeed $\sh{I}_y/\sh{I}_y^2 = 0$ in every point $y$ different from $x \in X$ corresponding to $\mathfrak{I}$, and $\sh{I}_x/\sh{I}_x^2$ has rank $1$ over the field $\sh{O}_x/\sh{I}_x$;
  also, $\sh{I}$ is evidently a regular ideal.

  \item[\rm{(ii)}] In \sref{IV.16.9.5}, one cannot replace ``quasi-regular'' by ``regular'', even when $X$ is a prescheme (cf. \sref{IV.16.9.12}).
  Indeed, denote by $B$ the ring of germs of infinitely differentiable functions on the point $0$ on $\mathbb{R}$; 
  it has a maximal ideal $\mathfrak{m}$ generated by the germ of $t$ of the identity mapping to $0$, and the intersection $\mathfrak{n}$ of the $\mathfrak{m}^k$ for $k>0$ is not reduced to $0$.
  Now let $A$ be the quotient ring $B[T]/\mathfrak{n}TB[T]$, and let $f_1$, $f_2$ be the canonical images in $A$ of the elements $t$ and $T$ of $B[T]$.
  The sequence $(f_1, f_2)$ is \emph{regular} in $A$: indeed, $f_1$ is not a zero divisor in $A$, because the relation $tP[T] \in \mathfrak{n}TB[T]$, for a polynomial $P \in B[T]$, implies that the products of $t$ by the coefficients of $P$ belongs to the ideal $\mathfrak{n}$, and it results immediately that the coefficients are the same in $\mathfrak{n}$, so $P[T] \in \mathfrak{n}TB[T]$.
  Since $B/tB$ is isomorphic to $\mathbb{R}$, $A/f_1A$ is isomorphic to the ring of polynomials $\mathbb{R}[T]$, therefore integral, and the image of $f_2$ in $A/f_1A$, being equal to $T$, is not a zero divisor, so that our claim is true.
  However, $f_2$ is a zero divisor in $A$, since for every non-zero element $x \in \mathfrak{n}$, the image of $x$ in $A$ is $\neq 0$, but the image of $xT$ is zero.
  We conclude that the sequence $(f_2, f_1)$ is \emph{not regular} in $A$;
  on the other hand, the ideal $\mathfrak{I} = f_1A + f_2A$ is distinct from $A$, so the conditions (b), (b') and (c) of \sref{IV.16.9.5} do not imply the condition (a) when we replace ``quasi-regular'' by ``regular''.
\end{enumerate}
\end{remarks}

\begin{env}[16.9.7]
\label{IV.16.9.7}
If $X = \Spec(A)$ is an affine scheme, we'll say that the ideal $\mathfrak{I}$ of $A$ is \emph{regular} (resp. \emph{quasi-regular}) if the ideal $\sh{J} = \widetilde{\mathfrak{I}}$ of $\sh{O}_X$ is regular (resp. quasi-regular);
we note that this notion is \emph{local} and does not imply the existence of a \emph{system of generators} of $\mathfrak{I}$ forming in $A$ a regular (resp. quasi-regular) sequence as the example \sref{IV.16.9.5} shows;
however this is true if $A$ is \emph{local} \sref{IV.16.9.5}.

The proposition \sref{IV.16.9.4} can be translated in terms of quasi-regular immersions in the following manner:
\end{env}

\begin{proposition}[16.9.8]
\label{IV.16.9.8}
Let $j:Y \to X$ be a morphism of preschemes;
for $j$ to be a quasi-regular immersion, it is necessary and sufficient that $j$ satisfies the following conditions:
\begin{enumerate}
  \item[(i)] $j$ is an immersion locally of finite presentation.
  \item[(ii)] The conormal sheaf $\shGr^1(j) = \sh{N}_{Y/X}$ \sref{IV.16.1.2} is a locally free $\sh{O}_Y$-module.
  \item[(iii)] The canonical homomorphism
  \[
    \bb{S}_{\sh{O}_Y}^\bullet(\shGr^1(j)) \to \shGr^\bullet(j)
  \]
  \sref{IV.16.1.2.2} is bijective.
\end{enumerate}
\end{proposition}

\begin{proof}
The problem being local on $Y$, we can restrict ourselves to the case where $j$ is the canonical injection of a closed subprescheme $Y$ of $X$, so the translation of \sref{IV.16.9.4} into \sref{IV.16.9.8} results from the description of $\shGr^1(j)$ and $\shGr^\bullet(j)$ in terms of the ideal $\sh{I}$ of $\sh{O}_X$ defining the subprescheme $Y$ \sref{IV.16.1.3}[(ii)].
\end{proof}

\begin{corollary}[16.9.9]
\label{IV.16.9.9}
Let $Y$ be a prescheme, $X$ an $Y$-prescheme, $j:Y \to X$ a $Y$-section of $X$, so that the $n$'th normal invariant $\sh{A}^{(n)}$ of $j$ \sref{IV.16.1.2} is an augmented $\sh{O}_Y$-algebra \sref{IV.16.1.7};
take $\sh{A}^{(\infty)} = \varprojlim \sh{A}^{(n)}$.
For $j$ to be a quasi-regular immersion it is necessary and sufficient that $j$ is locally of finite presentation, and that every $y \in Y$ admits an affine open neighbourhood $U$ of ring $C$ such that $\sh{A}^{(\infty)}|U$ is isomorphic as an augmented $\sh{O}_U$-algebra to $\sh{O}_U[[T_1, \dots, T_n]]$.
\end{corollary}

\begin{proof}
We can reduce to the case where $j$ is a closed immersion by restricting to a small neighbourhood of $y$ (see the argument on \sref{IV.16.4.11}), and therefore $\sh{O}_Y$ is identified with the quotient algebra $\sh{O}_X/\sh{I}$ and the canonical surjective homomorphism $\sh{O}_X \to \sh{O}_Y$ admits a
\oldpage[IV-4]{49}
right inverse \sref{IV.16.1.7}.
We can therefore suppose $X = \Spec(B)$, $Y = \Spec(A)$ are affines, $B$ being then an augmented $A$-algebra, and the augmentation ideal $\mathfrak{I}$ being of finite type.
Since $\sh{A}^{(n)}$ is identified with $(B/\mathfrak{I}^{n+1})^{\sim}$, the  corollary follows from the equivalence of (b) and (c) in \sref[0]{0.19.5.4} when $B/\mathfrak{I} = A$.
\end{proof}

We note that, on the affine case considered, the fact that $j$ is an quasi-regular immersion is further equivalent, because of \sref[0]{0.19.5.4}, to saying that $B$ is a \emph{formally smooth} $A$-algebra for the $\mathfrak{I}$-preadic topology.

We also note that the condition that $j$ is locally of finite presentation is always satisfied when $X \to Y$ is locally of finite type \sref{IV.1.4.3}[(v)].

\begin{proposition}[16.9.10]
\label{IV.16.9.10}
Let $X$ be a locally Noetherian prescheme, $Y$ a subprescheme of $X$, $j:Y \to X$ the canonical injection, $y$ a point of $Y$.
\begin{enumerate}
  \item[\rm{(i)}] For there to exist an open neighbourhood $U$ of $y$ in $X$ such that the restriction $Y \cap U \to U$ of $j$ be a regular immersion, it is necessary and sufficient that the kernel $\sh{I}_y$ of the surjective homomorphism $\sh{O}_{X,y} \to \sh{O}_{Y,y}$ is generated by a regular sequence of elements of $\sh{O}_{X,y}$.

  \item[\rm{(ii)}] For the immersion $j$ to be regular, it is necessary and sufficient that it is quasi-regular.
\end{enumerate}
\end{proposition}

\begin{proof}
  \begin{enumerate}
    \item[\rm{(i)}] We can reduce to the case where $Y$ is a subprescheme defined by a \emph{coherent} ideal $\sh{I}$ of $\sh{O}_X$.
    The condition is clearly necessary.
    Conversely, if $\sh{I}_y$ is generated by a regular sequence $(s_i)_y$, where the $s_i$ are sections of $\sh{I}$ over an open neighbourhood $U$ of $y$ in $X$, we can suppose that the $s_i$ generate $\sh{I}|U$ \sref[0]{0.5.5.2} and form a regular sequence \sref[0]{0.15.2.4}, from which the claim follows.
    \item[\rm{(ii)}] The fact that a quasi-regular immersion is regular follows from (i) and the identification of quasi-regular and regular sequences of $\sh{O}_{X,y}$, formed from elements of the maximal ideal \sref[0]{0.15.1.11}
  \end{enumerate}
\end{proof}

\begin{corollary}[16.9.11]
\label{IV.16.9.11}
Let $X$ be a locally Noetherian prescheme;
then every quasi-regular ideal of $\sh{O}_X$ is regular.
\end{corollary}

\begin{remarks}[16.9.12]
\label{IV.16.9.12}
\begin{enumerate}
  \item[\rm{(i)}] We note that a regular immersion is not in general a flat morphism, and therefore \textit{a fortiori} neither are quasi-regular morphisms in the sense of \sref{IV.6.8.1}.
  \item[\rm{(ii)}] Let $A$ be a local Noetherian ring;
  it follows immediately from \sref{IV.16.9.4} and from \sref[0]{0.17.1.1} that for $A$ to be \emph{regular}, it is necessary and sufficient that its maximal ideal $\mathfrak{m}$ is \emph{quasi-regular} (or regular, which amounts to the same thing  given that $A$ is Noetherian).
  For an affine Noetherian scheme $X$ to be \emph{regular}, it is necessary and sufficient that for every closed point $x \in X$, the canonical injection $\Spec(\kres(x)) \to X$ to be a \emph{regular} immersion.
\end{enumerate}
\end{remarks}

\begin{proposition}[16.9.13]
\label{IV.16.9.13}
Let $X$ be a locally Noetherian prescheme, $Y$ a subprescheme of $X$, $Y'$ a subprescheme of $Y$, such that the canonical injection $j:Y' \to Y$ is regular. 
Then the sequence of $\sh{O}_{Y'}$-modules
\[
  \label{IV.16.9.13.1}
  \xymatrix{
    0 \ar[r] & j^*(\sh{N}_{Y/X}) \ar[r] & \sh{N}_{Y'/X} \ar[r] & \sh{N}_{Y'/Y} \ar[r] & 0
  }
  \tag{16.9.13.1}
\]
\oldpage[IV-4]{50}
is exact;
furthermore, for every $x \in X$, there is an open neighbourhood $U$ of $x$ such that the restrictions to $U$ of the homomorphisms of \sref{IV.16.9.13.1} form a split exact sequence.
\end{proposition}

Let us first prove the following lemma:

\begin{lemma}[16.9.13.2]
\label{IV.16.9.13.2}
Let $A$ be a ring, $\mathfrak{I}$ an ideal of $A$, $A' = A/\mathfrak{I}$, $(f_i)_{1 \leq i \leq r}$ a sequence of elements of $A$ which is $A'$-regular, $\mathfrak{K} = \sum_i f_i A$, $\mathfrak{L} = \mathfrak{I} + \mathfrak{K}$, $\mathfrak{K}' = \sum_i f_i A'$, so that $C = A/\mathfrak{L}$ is isomorphic to $A'/\mathfrak{K}'$.
For every integer $n > 0$, and every integer $N \geq n$, we have the relation
\[
  \label{IV.16.9.13.3}
  \mathfrak{I} \cap \mathfrak{K}^n = \mathfrak{I}\mathfrak{K}^n + \mathfrak{I}\mathfrak{K}^N.
  \tag{16.9.13.3}
\]
\end{lemma}

\begin{proof}
It is clearly sufficient to prove that every element of the first is contained in the second, and, by induction on $n$, we reduce to the case $N = n + 1$.
An element of the first member of $\sref{IV.16.9.13.3}$, being in $\mathfrak{K}^n$, is written as $P(f_1, \dots, f_r)$, where $P \in A[T_1, \dots, T_r]$ is homogeneous of degree $n$.
If $f'_i$ is the canonical image of $f_i$ in $A'$, the hypothesis $P(f_1, \dots, f_n) \in \mathfrak{I}$ means that $P(f'_1, \dots, f'_n) = 0$.
But $P(f'_1, \dots, f'_n) \in \mathfrak{K}'^n$, so the canonical image of $P(f'_1, \dots, f'_n)$ in $\mathfrak{K}'^n/\mathfrak{K}'^{n+1}$ is zero.
Now the hypothesis that $(f_i)$ is $A'$-regular implies that the canonical homomorphism $\bb{S}_C^n(\mathfrak{K}'/\mathfrak{K}'^2) \to \mathfrak{K}'^n/\mathfrak{K}'^{n+1}$ is bijective \sref[0]{0.15.1.9};
we conclude that the coefficients of $P$ are in $\mathfrak{L} = \mathfrak{I} + \mathfrak{K}$.
It follows immediately that $P(f_1, \dots, f_n) \in \mathfrak{I}\mathfrak{K}^n + \mathfrak{K}^{n+1}$, and since $P(f_1, \dots, f_n) \in \mathfrak{I}$, we finally have $P(f_1, \dots, f_n) \in \mathfrak{I}\mathfrak{K}^n + \mathfrak{I}\mathfrak{K}^{n+1}$ which proves the lemma.
\end{proof}

By taking the quotient of both side of \sref{IV.16.9.13.3} by $\mathfrak{I}\mathfrak{K}^{n}$, we see that the relations \sref{IV.16.9.13.3} for $N \geq n$ entail
\[
  \label{IV.16.9.13.4}
  (\mathfrak{I} \cap \mathfrak{K}^n)/\mathfrak{I}\mathfrak{K}^n \subset \bigcap_{N \geq n} \mathfrak{K}^N .(A/(\mathfrak{I}\mathfrak{K}^n)).  
  \tag{16.9.13.4}
\]

We deduce the

\begin{corollary}[16.9.13.5]
\label{IV.16.9.13.5}
Suppose that the hypothesis of \sref{IV.16.9.13.2} are verified and also that the ring $A$ is Noetherian and $\mathfrak{K}$ is contained in the radical of $A$.
Then for every integer $n > 0$,
\[
  \label{IV.16.9.13.6}
  \mathfrak{I} \cap \mathfrak{K}^n = \mathfrak{I}\mathfrak{K}^n.
  \tag{16.9.13.6}
\]
\end{corollary}

\begin{proof}
Indeed, the second member of \sref{IV.16.9.13.4} is then zero, given that $A/\mathfrak{I}\mathfrak{K}^n$ is an $A$-module of finite type (Bourbaki, \emph{Alg. comm.}, chap.~III, \textsection3, n\textsuperscript{o}3, prop. 6).
\end{proof}

Taking in particular $n = 2$ in \sref{IV.16.9.13.6}, and remarking that we have $\mathfrak{L}^2 = \mathfrak{I}^2 + \mathfrak{I}\mathfrak{K} + \mathfrak{K}^2 = \mathfrak{I}\mathfrak{L} + \mathfrak{K}^2$;
since $\mathfrak{I}\mathfrak{L} \subset \mathfrak{L}^2$, we deduce that
\[
  \mathfrak{I} \cap \mathfrak{L}^2 = \mathfrak{I}\mathfrak{L} + (\mathfrak{I} \cap \mathfrak{K}^2) = \mathfrak{I}\mathfrak{L} + \mathfrak{I}\mathfrak{K}^2 = \mathfrak{I}\mathfrak{L},
\]
in other words, 
\[
  \label{IV.16.9.13.7}
  \mathfrak{I} \cap \mathfrak{L}^2 = \mathfrak{I}\mathfrak{L},
  \tag{16.9.13.7}
\]
which we can also express in saying that the canonical homomorphism
\[
  \mathfrak{I}/\mathfrak{I}\mathfrak{L} \to (\mathfrak{I} + \mathfrak{J}^2)/\mathfrak{J}^2
\]
is bijective.

\begin{proof}[Proof (16.9.13)]
Having demonstrated the lemmas, let us prove the first claim of \sref{IV.16.9.13}:
It is clearly
\oldpage[IV-4]{51}
enough to prove that the sequence of stalks of the sheaves appearing in \sref{IV.16.9.13.1}, in a point $x \in Y'$, is exact.
Now, if we take $A = \sh{O}_{X,x}$, we can write $\sh{O}_{Y, x} = A' = A/\mathfrak{I}$, where $\mathfrak{I}$ is an ideal contained in the maximal ideal of $A$, then $\sh{O}_{Y', x} = A'/\mathfrak{K}'$, where $\mathfrak{K}'$ is generated by an \emph{$A'$-regular} sequence of elements of $A'$, which themselves are images of elements from a $A'$-regular sequence of elements of $A$ belonging to the maximal ideal of $A$.
If $\mathfrak{K}$ is the ideal generated by the latter and $\mathfrak{L} = \mathfrak{I} + \mathfrak{K}$, we have $\sh{O}_{Y', x} = A/\mathfrak{L}$, and since we are in the situation of \sref{IV.16.9.13.5}, the canonical homomorphism $\mathfrak{I}/\mathfrak{I}\mathfrak{L} \to (\mathfrak{I} + \mathfrak{J}^2)/\mathfrak{J}^2$ is bijective.
But this shows that the sequence
\[
  \xymatrix{
    0 \ar[r] & \mathfrak{I}/\mathfrak{I}\mathfrak{L} \ar[r] & \mathfrak{L}/\mathfrak{L}^2 \ar[r] &  (\mathfrak{I}/\mathfrak{L})/(\mathfrak{I}/\mathfrak{L})^2 \ar[r] & 0
  }
\]
is exact (see the demonstration of \sref{IV.16.2.7}), and the modules making up this sequence are precisely the stalks in $x$ of the sheaves of \sref{IV.16.9.13.1}.
The second claim follows from the fact that $\sh{N}_{X/Y}$ is a locally free $\sh{O}_{Y'}$-module \sref{IV.16.9.8} and Bourbaki, \emph{Alg.}, chap.~II, 3\textsuperscript{rd}ed., \textsection1, n\textsuperscript{o}11, prop.\ 21.
\end{proof}

\subsection{Differentially smooth morphisms.}
\label{IV.16.10}

\begin{definition}[16.10.1]
\label{IV.16.10.1}
We say that a morphism of preschemes $f: X \to S$ is differentially smooth (or that $X$ is differentially smooth over $S$) if it satisfies the following conditions:
\begin{enumerate}
  \item[\rm{(i)}] $\Omega_{X/S}^1$ is a locally projective $\sh{O}_X$-module, which is to say that every point $x \in X$ admits an affine open neighbourhood $U$ such that $\Gamma(U, \Omega_{X/S}^1)$ is a projective $\Gamma(U, \sh{O}_X)$-module (not necessarily of finite type).
  \item[\rm{(ii)}] The canonical morphism \sref{IV.16.3.1.1}
  \[
    \bb{S}_{\sh{O}_X}^\bullet(\Omega_{X/S}^1) \to \shGr_\bullet(\sh{P}_{X/S})
  \]
  is bijective.
\end{enumerate}

In particular, if $\Omega_{X/S}^1$ is locally free of finite rank, the $\sh{P}_{X/S}^n$ are locally free of finite rank $\sh{O}_X$-modules (being extensions of such modules).
\end{definition}

We say that $f$ is \emph{differentially smooth at a point $x \in X$} if there is an open neighbourhood $U$ of $x$ in $X$ such that $f|U$ is differentially smooth.

We will see later \sref{IV.17.12.4} that a smooth morphism is differentially smooth, which justifies the terminology;
but the converse is not true;
indeed, a \emph{monomorphism} $f:X \to S$ is differentially smooth, since $\Omega^1_{X/S} = 0$ because of \sref[I]{I.5.3.8}, and consequently the surjective homomorphism \sref{IV.16.3.1.1} is clearly bijective;
however a monomorphism is not even necessarily flat, neither \textit{a fortiori} smooth. 
Lets limit ourselves to proving the following proposition:

\begin{proposition}[16.10.2]
\label{IV.16.10.2}
Let $A$ be a ring, $B$ a formally smooth $A$-algebra for the discrete topology \sref[0]{0.19.3.1}.
Then $\Spec(B)$ is differentially smooth over $\Spec(A)$.
\end{proposition}

\begin{proof}
Indeed, $B \otimes_A B$ is then (with the discrete topology) a formally smooth $B$-algebra (by on or the other canonical homomorphisms $b \mapsto b \otimes 1$, $b \mapsto 1 \otimes b$ of $B$
\oldpage[IV-4]{52}
to $B \otimes_A B$
) \sref[0]{0.19.3.5}[(iii)];
therefore $B \otimes_A B$ is a formally smooth $A$-algebra with the discrete topology \sref[0]{0.19.3.5}[(ii)].
Taking $\mathfrak{I} = \mathfrak{I}_{B/A}$, it follows that $B \otimes_A B$ is a formally smooth $A$-algebra for the $\mathfrak{I}$-preadic topology \sref[0]{0.19.3.8};
since by hypothesis $B = (B \otimes_A B)/\mathfrak{I}$ is a formally smooth $A$-algebra for the discrete topology, the proposition follows from the equivalence of (a) and (b) in \sref[0]{0.19.5.4}.
\end{proof}

\begin{proposition}[16.10.3]
\label{IV.16.10.3}
For a morphism $f:X \to S$ to be differentially smooth, it is necessary and sufficient that for every $x \in X$, there is an affine open neighbourhood of $x$, of ring $A$, such that $\Gamma(U, \sh{P}_{X/S}^\infty)$ is an augmented topological $A$-algebra isomorphic to the completion $\widehat{B}$, where $B = \bb{S}_A(V)$, $V$ being a projective $A$-module and $B$ being endowed with the $B^+$-preadic topology (where $B^+$ is the augmentation ideal).
If $\Omega_{X/S}^1$ is locally free of finite rank, we can replace $\widehat{B}$ with the ring of formal series $A[[T_1, \dots, T_n]]$.
\end{proposition}

\begin{proof}
The notion of a differentially smooth morphism is clearly local on $X$, so we reduce to the case where $S = \Spec(B)$, $X = \Spec(C)$.
Consider $C \otimes_B C$ as a $C$-algebra (by the first factor);
take $\mathfrak{I} = \mathfrak{I}_{C/B}$ and endow $C$ with the $\mathfrak{I}$-preadic topology;
we can apply to the topological $C$ algebra $C \otimes_B C$ and to the ideal $\mathfrak{I}$ of $C \otimes_B C$ the equivalence of (b) and (c) of \sref[0]{0.19.5.4}, given that $(C \otimes_B C)/\mathfrak{I} = C$ is clearly a formally smooth $C$-algebra for the discrete topologies.
The topology of $\Gamma(U, \sh{P}_{X/S}^\infty)$ is clearly the projective limit topology of this ring \sref{IV.16.1.11}.
\end{proof}

We note that the integer $n$ of the proposition \sref{IV.16.10.3} is the \emph{rank} of $\Omega_{X/S}^1$ in the point $x$.
We shall see \sref{IV.17.13.5} that when $f$ is differentially smooth and locally of finite type, $n$ is equal to the dimension of the fiber $f^{-1}(f(x))$. 

\begin{proposition}[16.10.4]
\label{IV.16.10.4}
Let $f:X \to S$, $g:S' \to S$ be two morphisms, and take $X' = X \times_S S'$, $f' = f_{(S')}:X' \to S'$.
\begin{enumerate}
  \item[\rm{(i)}] If $f$ is differentially smooth, the same is true for $f'$. 
  \item[\rm{(ii)}] Conversely, if $g$ is faithfully flat and quasi-compact, and if $f'$ is differentially smooth and $\Omega_{X'/S'}^1$ is a finite type $\sh{O}_{X'}$-module, $f$ is differentially smooth and $\Omega_{X/S}^1$ is a finite type $\sh{O}_X$-module.
\end{enumerate}
\end{proposition}

\begin{proof}
Indeed, if $f$ is differentially smooth, the $\shGr_n(\sh{P}_{X/S}^n)$ are \emph{flat} $\sh{O}_X$-modules;
therefore by \sref{IV.16.4.6}, the homomorphism $\shGr_n(\sh{P}_{X/S}^n) \otimes_{\sh{O}_X} \sh{O}_{X'} \to \shGr_n(\sh{P}_{X'/S'}^n)$ are bijective for every $n$, because of the commutative of the diagram \sref{IV.16.2.1.3}, it follows from the definition \sref{IV.16.10.1} that $f'$ is differentially smooth.
On the other hand, if $g$ is faithfully flat and quasi-compact, it follows also from \sref{IV.16.4.6} that $\shGr_n(\sh{P}_{X/S}^n) \otimes_{\sh{O}_X} \sh{O}_{X'} \to \shGr_n(\sh{P}_{X'/S'}^n)$ is bijective for every $n$.
Suppose also that $f'$ is differentially smooth and $\Omega_{X'/S'}^1$ is of finite rank.
Since the canonical projection $X' \to X$ is a faithfully flat and quasi-compact morphism, it results first from \sref{IV.2.5.2} that $\Omega_{X/S}^1$ is an $\sh{O}_X$-module locally free of finite rank, then from \sref{IV.2.2.7} that the canonical homomorphism \sref{IV.16.3.1.1} is bijective, and therefore $f$ is differentially smooth.
\end{proof}

\begin{proposition}[16.10.5]
\label{IV.16.10.5}
For a morphism locally of finite type $f:X \to S$ to be differentially smooth, it is necessary and sufficient that the diagonal immersion $\Delta_f:X \to X \times_S X$ to be quasi-regular.
\end{proposition}
  
\oldpage[IV-4]{53}

\begin{proof}
Being a local problem, we can reduce to the case where $S$ and $X$ are affines, and therefore the diagonal subprescheme of $X \times_S X$ is closed.
The hypothesis that $f$ is locally of finite type implies that $\Delta_f$ is locally of finite presentation \sref[I]{I.4.3.1}, therefore the diagonal prescheme of $X \times_S X$ is defined by an ideal $\sh{I}$ of finite type, and $\Omega_{X/S}^1 = \sh{I}/\sh{I}^2$ is an $\sh{O}_X$-module of finite type.
The proposition is now immediate from the comparison of the conditions of \sref{IV.16.10.1} and \sref{IV.16.9.4}. 
\end{proof}

\begin{remark}[16.10.6]
\label{IV.16.10.6}
Let $f:X \to S$ be a morphism such that the $\sh{O}_X$-module $\Omega_{X/S}^1$ is locally free of finite rank.
It results from \sref[0]{0.20.4.7} that every $x \in X$ has an open neighbourhood such that there is a finite family $(z_\lambda)_{\lambda \in L}$ of sections of $\sh{O}_X$ over $U$ for which $(dz_\lambda)_{\lambda \in L}$ forms a \emph{basis} of the $\Gamma(U, \sh{O}_X)$-module $\Gamma(U, \Omega_{X/S}^1)$.
\end{remark}

\subsection{Differential operators on a differentially smooth $S$-prescheme}
\label{IV.16.11}

\begin{env}[16.11.1]
\label{IV.16.11.1}
Let $f:X \to S$ be a morphism, $U$ an open set of $X$, $(z_\lambda)_{\lambda \in L}$ a family of sections of $\sh{O}_X$ over $U$ such that the $dz_\lambda$ form system of generators of $\Omega_{X/S}^1|U = \Omega_{U/S}^1$.
Let $m$ be an integer or the symbol $\infty$, and take, for every $\lambda$
\[
  \label{IV.16.11.1.1}
  \zeta_\lambda = \delta z_\lambda = d^m z_\lambda - z_\lambda \in \Gamma(U, \sh{P}_{X/S}^m).
  \tag{16.11.1.1}
\]

We will also use the usual notations from analysis;
for every $\bb{p} = (p_\lambda) \in \mathbb{N}^{(L)}$ (so that $p_\lambda = 0$ except for a finite number of indices), we put
\[
  \label{IV.16.11.1.2}
  |\bb{p}| = \sum_\lambda p_\lambda, \quad \bb{p} = \prod_\lambda (p_\lambda!)
  \tag{16.11.1.2}
\]
\[
  \label{IV.16.11.1.3}
  \binom{\bb p}{\bb q} = \bb{p}!/(\bb{q}!(\bb{p} - \bb{q})!), \quad \text{for $\bb{p}$, $\bb{q}$ in $\mathbb{N}^{(L)}$, $\bb{q} \leq \bb{p}$}
  \tag{16.11.1.3}
\]
and we adopt the convention that $\binom{\bb p}{\bb q} = 0$ when $\bb{q} \not\leq \bb{p}$,
\[
  \label{IV.16.11.1.4}
  \bb{z}^\bb{p} = \prod_\lambda (z_\lambda)^{p_\lambda}, \quad {\pmb\zeta}^{\bb p} = \prod_\lambda (\zeta_\lambda)^{p_\lambda}.
  \tag{16.11.1.4}
\]

We therefore have, with this notation
\[
  \label{IV.16.11.1.5}
  d^m(\bb{z}^{\bb p}) = (d^m(\bb{z}))^{\bb p} = (\pmb{\zeta} + \bb{z})^{\bb p} = \sum_{\bb{q} \leq \bb{p}} \binom{\bb p}{\bb q} \bb{z}^{\bb{p} - \bb{q}} \pmb{\zeta}^{\bb q}
  \tag{16.11.1.5}
\]
\[
  \label{IV.16.11.1.6}
  \pmb{\zeta}^{\bb p} = (d^m \bb{z} - \bb{z})^{\bb p} =  \sum_{\bb{q} \leq \bb{p}} (-1)^{|\bb{p} - \bb{q}|} \binom{\bb p}{\bb q} \bb{z}^{\bb{p} - \bb{q}} d^m(\bb{z}^{\bb q}).
  \tag{16.11.1.6}
\]
  
Since the $dz_\lambda$ generate $\Omega_{X/S}^1$, and are the images of $\delta z_\lambda$, and as the canonical morphism \sref{IV.16.3.1.1} is surjective, we conclude that for finite $m$, the $\delta z_\lambda$ generate the $\sh{O}_U$-algebra $\sh{P}_{U/S}^n$ (Bourbaki, \emph{Alg. comm.}, chap.~III, \textsection2, n\textsuperscript{o}8, cor.\ 2 du th.\ 1).
Therefore the $\pmb{\zeta}^{\bb p}$ (for $|\bb{p}| \leq m$) generate the $\sh{O}_U$-module $\sh{P}_{U/S}^n$.
A differential operator $D \in \Diff_{U/S}^m$ is consequently entirely determined by the values of $\langle \pmb{\zeta}^{\bb p}, D \rangle $for $|\bb{p}| \leq m$, or, which amounts to the same by \sref{IV.16.11.1.5} and \sref{IV.16.11.1.6}, by the values
\oldpage[IV-4]{54}
of the $\langle d^m(\bb{z}), D \rangle = D(\bb{z}^{\bb p})$ for $|\bb{p}| \leq m$;
more precisely, it follows from \sref{IV.16.11.5} that we have
\[
  \label{IV.16.11.1.7}
  D(\bb{z}^{\bb p}) = \langle d^m({\bb z}^{\bb p}), D \rangle = \sum_{\bb{q} \leq \bb{p}} \binom{\bb p}{\bb q} \langle \pmb{\zeta}^{\bb q} , D \rangle \bb{z}^{\bb{p} - \bb{q}}.
  \tag{16.11.1.7}
\]
\end{env}

\begin{theorem}[16.11.2]
\label{IV.16.11.2}
Let $f:X \to S$ be a morphism, $U$ an open set of $X$, $(z_\lambda)_{\lambda \in L}$ a family of sections of $\sh{O}_X$ over $U$ such that the family $(dz_\lambda)_{\lambda \in L}$ generates $\Omega_{X/S}^1|U = \Omega_{U/S}^1$.
The following conditions are equivalent:
\begin{enumerate}
  \item[(a)] $f|U$ is differentially smooth and $(dz_\lambda)$ is a basis of the $\sh{O}_U$-module $\Omega_{U/S}^1$.
  \item[(b)] There is a family $(D_{\bb p})_{\bb{p} \in \mathbb{N}^{(L)}}$ of differential operators of $\sh{O}_U$ to itself verifying the conditions
  \[
    \label{IV.16.11.2.1}
    D_{\bb p} (\bb{z}^{\bb q}) = \binom{\bb q}{\bb p} \bb{z}^{\bb{q} - \bb{p}}, \quad (\bb{p}, \bb{q} \text{ in } \mathbb{N}^{(L)}).
    \tag{16.11.2.1}
  \]
\end{enumerate}

Also, when these conditions are verified, the family $(D_\bb{p})$ is uniquely determined by the conditions \sref{IV.16.11.2.1} and satisfies the relations
\[
  \label{IV.16.11.2.2}
  D_{\bb q} \circ D_{\bb p} = D_{\bb p} \circ D_{\bb q} = \frac{(\bb{p} + \bb{q})!}{\bb{p}!\bb{q}!}D_{\bb{p} + \bb{q}} \quad (\bb{p}, \bb{q} \text{ in }  \mathbb{N}^{(L)}).
  \tag{16.11.2.2}
\]

Finally, if $L$ is finite, for every integer $m$, the $D_{\bb p}$ such that $|\bb{p}| \leq m$ form a basis of the $\sh{O}_U$-module $\shDiff_{U/S}^m$, in other words, every differential operator of order $\leq m$ on $U$ can be written uniquely as
\[
  D = \sum_{|\bb{p}| \leq m} a_{\bb p} D_{\bb p}
\] 
where the $a_{\bb p}$ are sections of $\sh{O}_X$ over $U$.
\end{theorem}

\begin{proof}
Note first that because of \sref{IV.16.11.1.6} and \sref{IV.16.11.1.5}, we verify immediately that the conditions \sref{IV.16.11.2.1} are equivalent to 
\[
  \label{IV.16.11.2.3}
  \langle \pmb{\zeta}^{\bb p}, D_{\bb q} \rangle = \delta_{\bb{p} \bb{q}} \quad (\text{Kronecker's symbol}).
  \tag{16.11.2.3}
\]

The existence of the family $(D_\bb{p})$ verifying these conditions implies first (by taking $|\bb{p}| = 1$) that the $dz_\lambda$ are linearly independent, and therefore form a basis of the $\sh{O}_U$-module $\Omega_{U/S}^1$.
Then, for every integer $m \geq 1$, we deduce similarly from \sref{IV.16.11.2.3} that the $\pmb{\zeta}^{\bb p}$ such that $|\bb{p}| \leq m$ are linearly independent;
it follows that the canonical homomorphism \sref{IV.16.3.1.1} is injective, and therefore bijective, which proves that (b) implies (a).
The converse follows immediately from the definition \sref{IV.16.10.1}, the fact that $\pmb{\zeta}^{\bb p}$ form a basis of $\sh{P}_{U/S}^n$ for $|\bb{p}| \leq m$ implies the existence and uniqueness of a family of homomorphisms $u_{\bb{q}, m}: \sh{P}_{U/S}^n \to \sh{O}_U (|\bb{q}| \leq m)$ such that $\langle \zeta^{\bb p}, u_{\bb{q}, m} \rangle = \delta{\bb{p}, \bb{q}}$ for $|\bb{p}| \leq m$, $|\bb{q}| \leq m$.
For a given value of $\bb{q}$, the differential operators corresponding to $u_{\bb{q}, m}$ for $m \geq |\bb{q}|$ are identified with the same operator $D_\bb{q}$.
This proves that (a) implies (b), and also that the family $(D_\bb{q})$ is uniquely determined, and that, if $L$ is finite, for $|\bb{p}| \leq m$, the $D_\bb{p}$ form a basis of the dual $\shDiff_{U/S}^m$ of $\sh{P}_{U/S}^n$.
Finally, the relations \sref{IV.16.11.2.2} follows immediately from the expression of the values of the three operators considered on the $\bb{z}^\bb{r}$, and of the fact that the $\pmb{\zeta}^\bb{r}$ for $|\bb{r}| \leq m$ generate $\sh{P}_{U/S}^n$.
\end{proof}

\oldpage[IV-4]{55}

\begin{remarks}[16.11.3]
\label{IV.16.11.3}
\begin{enumerate}
  \item[\rm{(i)}] The fact that, because of \sref{IV.16.11.2.2}, the $D_\bb{p}$ are pairwise permutable naturally do not imply that the $\sh{O}_U$-algebra $\shDiff_{U/S}$ is commutative, the $D_\bb{p}$ do not commute with the sections of $\sh{O}_U$ unless $n = 0$.
  \item[\rm{(ii)}] The indices $\bb{p}$ such that $|\bb{p}| = 1$ are the $\pmb{\epsilon}_\lambda =  (\epsilon_{\lambda\mu})_{\mu \in L}$ where $\epsilon_{\lambda\mu} = 0$ if $\mu \neq \lambda$ and $\epsilon_{\lambda\lambda} = 1$;
  when $L$ is finite, the operators $D_{\pmb{\epsilon}_\lambda}$ are exactly the $S$-derivations $D_i$ introduced in \sref{IV.16.5.7}.
  We note that in general (contrary to what happens in classical analysis), it is not true that a differential operator of any order can be written as a linear combination of powers of $D_i$ (cf. \sref{IV.16.12}). 
  \item[\rm{(iii)}] For every integer $r \geq 1$, we can define the notion of \emph{differentially smooth up to order $r$} by replacing in \sref{IV.16.10.1} the condition (ii) by the condition that the homomorphisms
  \[
    \bb{S}_{\sh{O}_X}^m(\Omega_{X/S}^1) \to \shGr_m(\sh{P}_{X/S}^n)
  \]
  are bijective \emph{for every $m \leq r$}.
  The argument of \sref{IV.16.11.2} proves also that if, in the condition (a), we replace ``differentially smooth'' by ``differentially smooth up to order $r$'', this condition is equivalent to (b) by restricting ourselves to $\bb{p} \in \mathbb{N}^{(L)}$, $\bb{q} \in \mathbb{N}^{(L)}$  such that $|\bb{p}| \leq r$, $|\bb{q}| \leq r$.
\end{enumerate}
\end{remarks}

\subsection{Case of characteristic zero: Jacobian criterion for differentially smooth morphisms.}
\label{IV.16.12}

\begin{env}[16.12.1]
\label{IV.16.12.1}
We say that a prescheme $X$ is \emph{of characteristic $p$} ($p$ equal to zero or a prime number) if, for every affine open set $U$ of $X$, the ring $\Gamma(U, \sh{O}_X)$ is of characteristic $p$ \sref[0]{0.21.1.1}.
It follows from \sref[0]{0.21.1.3} that for $X$ to be of characteristic $0$, it is necessary and sufficient that for every \emph{closed} point $x$ of $X$, the residue field $\kres(x)$ is of characteristic $0$, or even that $X$ can be given a structure of $\mathbb{Q}$-prescheme (necessarily unique).
\end{env}

\begin{theorem}[16.12.2]
\label{IV.16.12.2}
Let $X$ be a scheme of characteristic $0$, $f:X \to S$ a morphism.
If $\Omega_{X/S}^1$ is a locally free $\sh{O}_X$-module (not necessarily of finite type), $f$ is differentially smooth.
\end{theorem}

\begin{proof}
The problem being local on $X$, we can suppose there is a family $(z_\lambda)$ of sections of $\sh{O}_X$ over $X$ such that the $(dz_\lambda)$ is a basis for the $\sh{O}_X$-module $\Omega_{X/S}^1$.
Applying the criterion \sref{IV.16.11.2}, it is enough for the operators 
\[
  D_\bb{p} = (\bb{p}!)^{-1} \prod_\lambda D_\lambda^{p_\lambda}
\]
(where the $D_\lambda$ are the coordinate forms corresponding to the basis $(dz_\lambda)$) to verify the relations \sref{IV.16.11.2.1}, which is a consequence of the fact that the $D_\lambda$ are derivations.
\end{proof}

\begin{env}[16.12.3]
\label{IV.16.12.3}
The theorem above is not true if we discard the hypothesis that $X$ is of characteristic $0$.
For example, if $S = \Spec(k)$, where $k$ is a field of characteristic $p > 0$, $X = \Spec(K)$ where $K = k(\alpha)$ where $\alpha \notin k$, $\alpha^p \in k$, we verify immediately 
\oldpage[IV-4]{56}
that $\Omega_{X/S}^1$ has rank $1$, and that the morphism $X \to S$ has rank $1$, and that the morphism $X \to S$ is differentially smooth up to order $p - 1$ \sref{IV.16.11.3}[(iii)], but not of order $p$.
However, the proof of \sref{IV.16.12.2} proves that if $\Omega_{X/S}^1$ is locally free, and if $n! 1_{\sh{O}_X}$ is inversible in $\Gamma(X, \sh{O}_X)$, then $X$ is differentially smooth over $S$ up to order $n$.
\end{env}

\documentclass[oneside]{amsart}

\usepackage[all]{xy}
\usepackage[T1]{fontenc}
\usepackage{xstring}
\usepackage{xparse}
\usepackage{xr-hyper}
\usepackage{xcolor}
\definecolor{brightmaroon}{rgb}{0.76, 0.13, 0.28}
\usepackage[linktocpage=true,colorlinks=true,hyperindex,citecolor=blue,linkcolor=brightmaroon]{hyperref}
\usepackage[left=1.25in,right=1.25in,top=0.75in,bottom=0.75in]{geometry}
%\usepackage[charter,greekfamily=didot]{mathdesign}
%\usepackage{Baskervaldx}
\usepackage{amssymb}
\usepackage{mathrsfs}
\usepackage{mathpazo}
\linespread{1.05}

\usepackage[nobottomtitles]{titlesec}
\usepackage{marginnote}
\usepackage{enumerate}
\usepackage{longtable}
\usepackage{aurical}
\usepackage{microtype}

\externaldocument[what-]{what}
\externaldocument[intro-]{intro}
\externaldocument[ega0-]{ega0}
\externaldocument[ega1-]{ega1}
\externaldocument[ega2-]{ega2}
\externaldocument[ega3-]{ega3}
\externaldocument[ega4-]{ega4}

\newtheoremstyle{ega-env-style}%
  {}{}{\rmfamily}{}{\bfseries}{.}{ }{\thmnote{(#3)}}%

\newtheoremstyle{ega-thm-env-style}%
  {}{}{\itshape}{}{\bfseries}{. --- }{ }{\thmname{#1}\thmnote{ (#3)}}%

\newtheoremstyle{ega-defn-env-style}%
  {}{}{\rmfamily}{}{\bfseries}{. --- }{ }{\thmname{#1}\thmnote{ (#3)}}%

\theoremstyle{ega-env-style}
\newtheorem*{env}{---}

\theoremstyle{ega-thm-env-style}
\newtheorem*{theorem}{Theorem}
\newtheorem*{proposition}{Proposition}
\newtheorem*{lemma}{Lemma}
\newtheorem*{corollary}{Corollary}

\theoremstyle{ega-defn-env-style}
\newtheorem*{definition}{Definition}
\newtheorem*{example}{Example}
\newtheorem*{examples}{Examples}
\newtheorem*{remark}{Remark}
\newtheorem*{remarks}{Remarks}
\newtheorem*{notation}{Notation}

% indent subsections, see https://tex.stackexchange.com/questions/177290/.
% also make section titles bigger.
% also add § to \thesection, https://tex.stackexchange.com/questions/119667/ and https://tex.stackexchange.com/questions/308737/.
\makeatletter
\def\l@subsection{\@tocline{2}{0pt}{2.5pc}{2.2pc}{}}
\def\section{\@startsection{section}{1}%
  \z@{.7\linespacing\@plus\linespacing}{.5\linespacing}%
  {\normalfont\bfseries\Large\scshape\centering}}
\renewcommand{\@seccntformat}[1]{%
  \ifnum\pdfstrcmp{#1}{section}=0\textsection\fi%
  \csname the#1\endcsname.~}
\makeatother

%\allowdisplaybreaks[1]
%\binoppenalty=9999
%\relpenalty=9999

% for Chapter 0, Chapter I, etc.
% credit for ZeroRoman https://tex.stackexchange.com/questions/211414/
% added into scripts/make_book.py
%\newcommand{\ZeroRoman}[1]{\ifcase\value{#1}\relax 0\else\Roman{#1}\fi}
%\renewcommand{\thechapter}{\ZeroRoman{chapter}}

\def\mathcal{\mathscr}
\def\sh{\mathcal}                   % sheaf font
\def\bb{\mathbf}                    % bold font
\def\cat{\mathtt}                   % category font
\def\leq{\leqslant}                 % <=
\def\geq{\geqslant}                 % >=
\def\setmin{-}                      % set minus
\def\rad{\mathfrak{r}}              % radical
\def\nilrad{\mathfrak{N}}           % nilradical
\def\emp{\varnothing}               % empty set
\def\vphi{\phi}                     % for switching \phi and \varphi, change if needed
\def\HH{\mathrm{H}}                 % cohomology H
\def\CHH{\check{\HH}}               % Čech cohomology H
\def\RR{\mathrm{R}}                 % right derived R
\def\LL{\mathrm{L}}                 % left derived L
\def\dual#1{{#1}^\vee}              % dual
\def\kres{k}                        % residue field k
\def\C{\cat{C}}                     % category C
\def\op{^\cat{op}}                  % opposite category
\def\Set{\cat{Set}}                 % category of sets
\def\CHom{\cat{Hom}}                % functor category
\def\supertilde{{\,\widetilde{\,}\,}}   % use \supertilde instead of ^\sim
\def\GL{\bb{GL}}
\def\red{\mathrm{red}}
\def\rg{\operatorname{rg}}
\def\gr{\operatorname{gr}}
\def\Gr{\operatorname{Gr}}
\def\Hom{\operatorname{Hom}}
\def\Proj{\operatorname{Proj}}
\def\Tor{\operatorname{Tor}}
\def\Ext{\operatorname{Ext}}
\def\Supp{\operatorname{Supp}}
\def\Ker{\operatorname{Ker}}
\def\Im{\operatorname{Im}}
\def\Coker{\operatorname{Coker}}
\def\Spec{\operatorname{Spec}}
\def\Spf{\operatorname{Spf}}
\def\grad{\operatorname{grad}}
\def\dimc{\operatorname{dimc}}
\def\codim{\operatorname{codim}}
\def\id{\operatorname{id}}
\def\Der{\operatorname{Der}}
\def\Diff{\operatorname{Diff}}
\def\Hyp{\operatorname{Hyp}}

\renewcommand{\to}{\mathchoice{\longrightarrow}{\rightarrow}{\rightarrow}{\rightarrow}}
\newcommand{\from}{\mathchoice{\longleftarrow}{\leftarrow}{\leftarrow}{\leftarrow}}
\let\mapstoo\mapsto
\renewcommand{\mapsto}{\mathchoice{\longmapsto}{\mapstoo}{\mapstoo}{\mapstoo}}
\def\isoto{\simeq} % isomorphism

\renewcommand{\hat}[1]{\widehat{#1}}
\renewcommand{\tilde}[1]{\widetilde{#1}}


\def\shHom{\sh{H}\textup{\kern-2.2pt{\Fontauri\slshape om}}}   % sheaf Hom
\def\shProj{\sh{P}\textup{\kern-2.2pt{\Fontauri\slshape roj}}} % sheaf Proj
\def\shExt{\sh{E}\textup{\kern-2.2pt{\Fontauri\slshape xt}}}   % sheaf Ext
\def\shGr{\sh{G}\textup{\kern-2.2pt{\Fontauri\slshape r}}}     % sheaf Gr
\def\shDer{\sh{D}\,\textup{\kern-2.2pt{\Fontauri\slshape er}}} % sheaf Der
\def\shDiff{\sh{D}\,\textup{\kern-2.2pt{\Fontauri\slshape if{}f}}\,} % sheaf Diff
\def\shHomcont{\sh{H}\textup{\kern-2.2pt{\Fontauri\slshape om.\,cont}}}   % sheaf Hom.cont
\def\shAut{\sh{A}\textup{\kern-2.2pt{\Fontauri\slshape ut}}}   % sheaf Aut

% if unsure of a translation
%\newcommand{\unsure}[2][]{\hl{#2}\marginpar{#1}}
%\newcommand{\completelyunsure}{\unsure{[\ldots]}}
\def\unsure#1{#1 {\color{red}(?)}}
\def\completelyunsure{{\color{red}(???)}}

% use to mark where original page starts
\newcommand{\oldpage}[2][]{{\marginnote{\normalfont{\textbf{#1}~|~#2}}}\ignorespaces}
\def\sectionbreak{\begin{center}***\end{center}}

% for referencing environments.
% use as \sref{chapter-number.x.y.z}, with optional args
% for volume and indices, e.g. \sref[volume]{chapter-number.x.y.z}[i].
\NewDocumentCommand{\sref}{o m o}{%
  \IfNoValueTF{#1}%
    {\IfNoValueTF{#3}%
      {\hyperref[#2]{\normalfont{(\StrBehind{#2}{.})}}}%
      {\hyperref[#2]{\normalfont{(\StrBehind{#2}{.},~{#3})}}}}%
    {\IfNoValueTF{#3}%
      {\hyperref[#2]{\normalfont{(\textbf{#1},~\StrBehind{#2}{.})}}}%
      {\hyperref[#2]{\normalfont{(\textbf{#1},~\StrBehind{#2}{.},~{#3})}}}}%
}

% for marking changes following errata
% use as \erratum[volume]{correction} to say where the erratum is given
\newcommand{\erratum}[2][]{{#2}\marginpar{\footnotesize\textbf{Err}\textsubscript{#1}}}

% for referencing equations, use as \eref{eq:chapter-number.x.y.z}.
\newcommand{\eref}[1]{\hyperref[#1]{\normalfont{(\StrBehind{#1}{.})}}}



\begin{document}
\title{What this is}
\maketitle

\phantomsection
\label{section:what}

\noindent
\emph{This whole chapter is written by the translators.}

\bigskip

\noindent
This is a community translation of Alexander Grothendieck's and Jean Dieudonn\'e's \emph{\'El\'ements de g\'eom\'etrie alg\'ebrique}~(EGA).
As it is a work in progress by multiple people, there will probably be a few mistakes---if you spot any then please do \href{https://github.com/ryankeleti/ega/issues}{let us know}\footnote{\url{https://github.com/ryankeleti/ega/issues}}.

\noindent
To contribute, please visit
\begin{center}
  \url{https://github.com/ryankeleti/ega}.
\end{center}

\noindent
\emph{On est d\'esol\'es, Grothendieck.}

\section*{In defense of a translation}

From \href{https://en.wikipedia.org/wiki/Alexander_Grothendieck\#Retirement_into_reclusion_and_death}{Wikipedia}\footnote{\url{https://en.wikipedia.org/wiki/Alexander_Grothendieck\#Retirement_into_reclusion_and_death}}:

\begin{quote}
In January 2010, Grothendieck wrote the letter ``D\'eclaration d'intention de non-publication'' to Luc Illusie, claiming that all materials published in his absence have been published without his permission.
He asks that none of his work be reproduced in whole or in part and that copies of this work be removed from libraries.\footnote{\href{https://sbseminar.wordpress.com/2010/02/09/grothendiecks-letter/}{Grothendieck's letter}. \emph{Secret Blogging Seminar.} 9~February 2010. Retrieved 3~September 2019.}
[...]
This order may have been reversed later in 2010.\footnote{\href{https://web.archive.org/web/20160629235119/http://www.math.u-psud.fr/~laszlo/sga4.html}{R\'e\'edition des SGA}. Archived from the original on 29~June 2016. Retrieved 12~November 2013.}
\end{quote}

It is a matter of often heated contention as to whether or not any translation of Grothendieck's work should take place, given his extremely explicit views on the matter.
By no means do we mean to argue that somehow Grothendieck's wishes should be invalidated or ignored, nor do we wish to somehow twist his earlier words around in order to justify what we have done: we fully accept that he himself would probably have branded this project ``an abomination''.
With this in mind, it remains to explain why we have gone ahead anyway.

First, and possibly foremost, it does not make sense (to us) for an individual to own the rights to knowledge.
Arguments can be made about how the EGA is the product of years and years of intense work by Grothendieck, and so \emph{this} is something that he `owns' and has full control over.
Indeed, it is true that there are almost innumerably many sentences in these works that only Grothendieck himself could have engineered, but, in translation, we have never improved anything, but only (regrettably, but almost certainly) worsened.
The work in these pages is that of Grothendieck; we have been not much more than typesetters and eager readers.
However, there is some important point to be made about the fact that Grothendieck collaborated and worked with many other incredibly proficient mathematicians during the writing of this treatise; although it is impossible to pinpoint which parts exactly others may have contributed (and by no means do we wish to imply that any of this work is derivative or fraudulent in any way whatsoever---EGA was written \emph{by Grothendieck}) it seems fair that, in some amount, there are bits of the EGAs that `belong' to a broader collection of minds.

It is a very good idea here to repeat the oft-quoted aphorism: ``the work here is not ours, but any mistakes are''---it is very understandable for an author to not want their name on something that they have not themselves written, or, at the very least, read.
This may be, in part, a reason for Grothendieck's wishes, but that is pure speculation.
Even so, we include this above disclaimer.

Secondly, then, we note that the French version of EGA is still entirely readily accessible.
Anybody reading these copies who is not a native French speaker, will probably be translating at least some part of EGA into English in their head, or into their notebooks, as they read.
This document is just the product of a few people doing exactly that, but then passing on their efforts to make things just that little bit easier for anyone else who follows.

Lastly, to quote another adage, ``the guilty person is often the loudest''.
If it seems like we are over-eager to defend ourselves because we know that we are somehow in the wrong, it is because we are, at least partially.
Working on this translation has meant going against Grothendieck's explicit requests, and for that we are sorry.
We only hope that the freedom of knowledge is an excusable defense.

\section*{Notes from the translators}

Grothendieck's writing style in EGA is quite particular, most notably for its long sentence structure.
As translators, we have tried to give the best possible approximation of this style in English, resisting the temptation to ``streamline'' things in places where the language is more dense than usual.

\sectionbreak

\unsure{Any translations about which we are not entirely sure will be marked with a}.

\sectionbreak

Whenever a note is made by the translators, it will be prefaced by ``[Trans.]''.

\sectionbreak

Along the margins we have provided the page numbers corresponding to the original text (of the first edition), as published by \emph{Publications math\'ematiques de l'I.H.\'E.S.}, where the EGA were published as the following volumes:\footnote{PDFs of which can be found online, hosted by the \emph{Grothendieck circle}.}
\begin{enumerate}
  \item[--] EGA~I (\emph{tome 4, 1960})
  \item[--] EGA~II (\emph{tome 8, 1961})
  \item[--] EGA~III, part 1 (\emph{tome 11, 1961})
  \item[--] EGA~III, part 2 (\emph{tome 17, 1963})
  \item[--] EGA~IV, part 1 (\emph{tome 20, 1964})
  \item[--] EGA~IV, part 2 (\emph{tome 24, 1965})
  \item[--] EGA~IV, part 3 (\emph{tome 28, 1966})
  \item[--] EGA~IV, part 4 (\emph{tome 32, 1967}).
\end{enumerate}

Due to EGA being a collection of volumes (one non-preliminary chapter, or part of a chapter, per volume), the page numbers reset at every new chapter.
In addition, the preliminary section is stretched out over multiple volumes.
To combat this, we label the pages as
\begin{center}
  \textbf{X}~|~$p$,
\end{center}
referring to Chapter~X, page $p$.
For EGA~III and IV, which are split across multiple chapters, we label the pages as
\begin{center}
  \textbf{X-$n$}~|~$p$,
\end{center}
referring to Chapter~X, part~$n$, page $p$.
In the case of the preliminaries (which are often collectively referred to as EGA~0), the preliminaries from volume~Y are denoted as \textbf{0\textsubscript{Y}}.

\sectionbreak

Later volumes (EGA~II, III, and IV) include errata for earlier chapters.
Where possible, we have used these to `update' our translation, and entirely replace whatever mistakes might have been in the original copies of EGA~I and II.
If the change is minor (e.g. `intersection' replacing `inter-section') then we will not mention it; if it is anything more fundamental (e.g. $X'$ replacing $X$) then we will include some margin note on the relevant line detailing the location of the erratum (e.g. \textbf{Err}\textsubscript{II} to denote that the correction is listed in the Errata section of EGA~II).

\section*{EGA~IV sections}
\label{section:what-ega4-sections}

In EGA~IV-1, the summary included a tentative list of section that EGA~IV would contain.
As EGA~IV was written, \textsection5, \textsection7, and \textsection\textsection11--21 would contain different sections than initially envisaged.
We include the original listing here:
\begin{longtable}{ll}
  \textsection5.  & Dimension and depth for preschemes.\\
  \textsection7.  & Application to the relations between a local Noetherian ring and its completion. Excellent rings.\\
  \textsection11. & Topological properties of finitely presented flat morphisms. Local flatness criteria.\\
  \textsection12. & Study of fibres of finitely presented flat morphisms.\\
  \textsection13. & Equidimensional morphisms.\\
  \textsection14. & Universally open morphisms.\\
  \textsection15. & Study of fibres of a universally open morphism.\\
  \textsection16. & Differential invariants. Differentially smooth morphisms.\\
  \textsection17. & Smooth morphisms, unramified morphisms, and \'etale morphisms.\\
  \textsection18. & Supplement on \'etale morphisms. Henselian local rings.\\
  \textsection19. & Regular immersions and transversely regular immersions.\\
  \textsection20. & Hyperplane sections; generic projections.\\
  \textsection21. & Infinitesimal extensions.
\end{longtable}
\bigskip

\section*{Mathematical warnings}
EGA uses \emph{prescheme} for what is now usually called a scheme, and \emph{scheme} for what is now usually called a separated scheme --- we have decided to translate ``literally'' or ``historically'', and thus continue to use the word \emph{scheme} to mean separated scheme, and \emph{prescheme} to mean scheme.

In some cases, we (the translators) have changed ``$\to$'' to ``$\mapsto$'' where appropriate.

\end{document}


\section{Homogeneous spectrum of a sheaf of graded algebras}
\label{section:II.3}


\subsection{Homogeneous spectrum of a quasi-coherent graded $\mathcal{O}_Y$-algebra}
\label{subsection:II.3.1}

\begin{env}[3.1.1]
\label{II.3.1.1}
Let $Y$ be a prescheme, $\sh{S}$ a graded $\sh{O}_Y$-algebra, and $\sh{M}$ a graded $\sh{S}$-module.
If $\sh{S}$ is \emph{quasi-coherent}, then each of its homogenous components $\sh{S}_n$ is a \emph{quasi-coherent} $\sh{O}_Y$-module, since they are the images of $\sh{S}$ under a homomorphism from $\sh{S}$ to itself (\sref[I]{I.1.3.8} and \sref[I]{I.1.3.9});
similarly, if $\sh{M}$ is quasi-coherent as an $\sh{O}_Y$-module, then its homogenous components $\sh{M}_n$ are also quasi-coherent, and the converse is also true.
For an integer $d>0$, we denote by $\sh{S}^{(d)}$ the direct sum of the $\sh{O}_Y$-modules $\sh{S}_{nd}$ (for $n\in\bb{Z}$), which is quasi-coherent if $\sh{S}$ is \sref[I]{I.1.3.9};
for every integer $k$ such that $0\leq k\leq d-1$, we denote by $\sh{M}^{(d,k)}$ (or $\sh{M}^{(d)}$, for $k=0$) the direct sum of the $\sh{M}_{nd+k}$ (for $n\in\bb{Z}$), which is a graded $\sh{S}^{(d)}$-module, and quasi-coherent if both $\sh{S}$ and $\sh{M}$ are quasi-coherent \sref[I]{I.9.6.1}.
We denote by $\sh{M}(n)$ the graded $\sh{S}$-module such that $(\sh{M}(n))_k=\sh{M}_{n+k}$ for all $k\in\bb{Z}$;
if $\sh{S}$ and $\sh{M}$ are quasi-coherent, then $\sh{M}(n)$ is a quasi-coherent graded $\sh{S}$-module \sref[I]{I.9.6.1}.

We say that $\sh{M}$ is a graded $\sh{S}$-module \emph{of finite type} (resp. admitting a \emph{finite presentation})  if, for all $y\in Y$, there exists an open neighbourhood $U$ of $y$, along with integers $n_i$ (resp. integers $m_i$ and $n_j$) such that there is a surjective degree~$0$ homomorphism $\bigoplus_{i=1}^r(\sh{S}(n_i)|U)\to\sh{M}|U$ (resp. such that $\sh{M}|U$ is isomorphic to the cokernel of a degree~$0$ homomorphism $\bigoplus_{i=1}^r(\sh{S}(m_i)|U)\to\bigoplus_{j 1}^s(\sh{S}(n_J)|U)$).

Let $U$ be an open affine of $Y$, with ring $A=\Gamma(U,\sh{O}_Y)$;
by hypothesis, the graded $(\sh{O}_Y|U)$-algebra $\sh{S}|U$ is isomorphic to $\widetilde{S}$, where $S=\Gamma(U,\sh{S})$ is a graded $A$-algebra \sref[I]{I.1.4.3};
\oldpage[II]{50}
set $X_U=\Proj(\Gamma(U,\sh{S}))$.
Let $U'\subset U$ be another open affine of $Y$, with ring $A'$, and let $j:U'\to U$ be the canonical injection, which corresponds to the restriction homomorphism $A\to A'$;
we have that $\sh{S}|U'=j^*(\sh{S}|U)$, and so $S'=\Gamma(U',\sh{S})$ is canonically identified with $X_U\times_U U'$, and thus also with $f_U^{-1}(U')$, where we denote by $f_U$ the structure morphism $X_U\to U$ \sref[I]{I.4.4.1}.
We denote by $\sigma_{U',U}$ the canonical isomorphism $f_U^{-1}(U')\xrightarrow{\sim}X_{U'}$ thus defined, and by $\rho_{U',U}$ the open immersion $X_{U'}\to X_U$ obtained by composing $\sigma_{U',U}^{-1}$ with the canonical injection $f_U^{-1}(U')\to X_U$.
It is immediate that, if $U''\subset U'$ is another open affine of $Y$, then $\rho_{U'',U}=\rho_{U'',U'}\circ\rho_{U',U}$.
\end{env}

\begin{proposition}[3.1.2]
\label{II.3.1.2}
Let $Y$ be a prescheme.
For every quasi-coherent positively graded $\sh{O}_Y$-algebra, there exists exactly one (up to $Y$-isomorphism) prescheme $X$ over $Y$ satisfying the following property:
if $f:X\to Y$ is the structure morphism, then, for every open affine $U$ of $Y$, there exists an \emph{isomorphism} $\eta_U$ from the induced prescheme $f^{-1}(U)$ to $X_U=\Proj(\Gamma(U,\sh{S}))$ such that, if $V$ is another open affine of $Y$ that is contained in $U$, then the diagram
\[
\label{II.3.1.2.1}
  \xymatrix{
    f^{-1}(V) \ar[r]^{\eta_V} \ar[d]
    & X_V \ar[d]^{\rho_{V,U}}
  \\f^{-1}(U) \ar[r]_{\eta_V}
    & X_U
  }
\tag{3.1.2.1}
\]
commutes.
\end{proposition}

\begin{proof}
Given affine opens $U$ and $V$ of $Y$, let $X_{U,V}$ be the prescheme induced on $f_U^{-1}(U\cap V)$ by $X_U$;
we are going to define a $Y$-isomorphism $\theta_{U,V}:X_{V,U}\xrightarrow{\sim}X_{U,V}$.
For this, we consider an open affine $W\subset U\cap V$:
by composing the isomorphisms
\[
  f_U^{-1}(W)
  \xrightarrow{\sigma_{W,U}} X_W
  \xrightarrow{\sigma_{W,V}^{-1}} f_V^{-1}(W),
\]
we obtain an isomorphism $\tau_W$, and we immediately see that, if $W'\subset W$ is an open affine, then $\tau_{W'}$ is the restriction of $\tau_W$ to $f_U^{-1}(W')$;
the $\tau_W$ are thus indeed the restrictions of a $Y$-isomorphism $\theta_{V,U}$.
Further, if $U$, $V$, and $W$ are open affines of $Y$, and $\theta'_{U,V}$, $\theta'_{V,W}$, and $\theta'_{U,W}$ the restrictions of $\theta_{U,V}$, $\theta_{V,W}$, and $\theta_{U,W}$ (respectively) to the inverse images of $U\cap V\cap W$ in $X_V$, $X_W$, and $X_W$ (respectively), then it follows from the above definitions that we have $\theta'_{U,V}\circ\theta'_{V,W}=\theta'_{U,W}$.
The existence of some $X$ satisfying the properties in the statement thus follows from \sref[I]{I.2.3.1};
its uniqueness up to $Y$-isomorphism is trivial, taking \sref{I.3.1.2.1} into account.
\end{proof}

\begin{env}[3.1.3]
\label{II.3.1.3}
We say that the prescheme $X$ defined in \sref{II.3.1.2} is the \emph{homogeneous spectrum} of the quasi-coherent graded $\sh{O}_Y$-algebra $\sh{S}$, and we denote it by $\Proj(\sh{S})$.
It is immediate that $\Proj(\sh{S})$ is \emph{separated over $Y$} (\sref{II.2.4.2} and \sref[I]{I.5.5.5});
if $\sh{S}$ is an $\sh{O}_Y$-algebra \emph{of finite type} \sref[I]{I.9.6.2}, then $\Proj(\sh{S})$ is \emph{of finite type} over $Y$ (\sref{II.2.7.1}[(ii)] and \sref[I]{I.6.3.1}).

If $f$ is the structure morphism $X\to Y$, then it is immediate that, for every prescheme induced by $Y$ on an open subset $U$ of $Y$, $f^{-1}(U)$ can be identified with the homogeneous spectrum $\Proj(\sh{S}|U)$.
\end{env}

\begin{proposition}[3.1.4]
\label{II.3.1.4}
Let $f\in\Gamma(Y,\sh{S}_d)$ for $d>0$.
Then there exists an open subset $X_f$ of the underlying space of $X=\Proj(\sh{S})$ that satisfies the following property:
for every open affine $U$ of $Y$, we have $X_f\cap\varphi^{-1}(U)=D_+(f|U)$ in $\varphi^{-1}(U)$ identified with $X_U=\Proj(\Gamma(U,\sh{S}))$, where $\varphi$ denotes the structure morphism $X\to Y$.
\oldpage[II]{51}
Furthermore, the prescheme induced on $X_f$ is affine over $Y$, and is canonically isomorphic to $\Spec(\sh{S}^{(d)}/(f-1)\sh{S}^{[d]})$ \sref{II.1.3.1}.
\end{proposition}

\begin{proof}
We have $f|U\in\Gamma(U,\sh{S}_d)=(\Gamma(U,\sh{S}))_d$.
If $U$ and $U'$ are open affines of $Y$ such that $U'\subset U$, then $f|U'$ is the image of $f|U$ under the restriction homomorphism
\[
  \Gamma(U,\sh{S}) \to \Gamma(U',\sh{S})
\]
and so $D_+(f|U')$ is equal (with the notation of \sref{II.3.1.1}) to the prescheme induced on the inverse image $\rho_{U',U}^{-1}(D_+(f|U))$ in $X_{U'}$ \sref{II.2.8.1};
whence the first claim.
Furthermore, the prescheme induced on $D_+(f|U)$ by $X_U$ is canonically identified with $\Spec((\Gamma(U,\sh{S}))_{f|U})$, with these identifications being compatible with the restriction homomorphisms \sref{II.2.8.1};
the second claim then follows from \sref{II.2.2.5} and from the commutativity of the diagram \sref{II.2.8.2.1}.
\end{proof}

We also say that $X_f$ (as an open subset of the underlying space $X$) is the set of $x\in X$ where $f$ \emph{does not vanish}.

\begin{corollary}[3.1.5]
\label{II.3.1.5}
If $f\in\Gamma(Y,\sh{S}_d)$ and $g\in\Gamma(Y,\sh{S}_e)$, then
\[
\label{II.3.1.5.1}
  X_{fg} = X_f\cap X_g.
\tag{3.1.5.1}
\]
\end{corollary}

\begin{proof}
It suffices to consider the intersection of the two sets with a set $\varphi^{-1}(U)$, where $U$ is an open affine in $Y$, and to then apply formula \sref{II.2.3.3.2}.
\end{proof}

\begin{corollary}[3.1.6]
\label{II.3.1.6}
Let $(f_\alpha)$ be a family of sections of $\sh{S}$ over $Y$ such that $f_\alpha\in\Gamma(Y,\sh{S}_{d_\alpha})$;
if the sheaf of ideals of $\sh{S}$ generated by this family \sref[0]{0.5.1.1} contains all the $\sh{S}_n$ starting from a certain rank, then the underlying space $X$ is the union of the $X_{f_\alpha}$.
\end{corollary}

\begin{proof}
  For every open affine $U$ of $Y$, $\varphi^{-1}(U)$ is the union of the $X_{f_\alpha}\cap\varphi^{-1}(U)$ \sref{II.2.3.14.
\end{proof}

\begin{corollary}[3.1.7]
\label{II.3.1.7}
Let $\sh{A}$ be a quasi-coherent $\sh{O}_Y$-algebra;
set
\[
  \sh{S} = \sh{A}[T] = \sh{A}\otimes_{\bb{Z}}\bb{Z}[T]
\]
where $T$ is an indeterminate (and $\bb{Z}$ and $\bb{Z}[T]$ are considered as simple sheaves over $Y$).
Then $X=\Proj(\sh{S})$ is canonically identified with $\Spec(\sh{A})$.
In particular, $\Proj(\sh{O}_Y[T])$ is identified with $Y$.
\end{corollary}

\begin{proof}
By applying \sref{II.3.1.6} to the unique section $f\in\Gamma(Y,\sh{S})$ that is equal to $T$ at each point of $Y$< we see that $X_f=X$.
Further, here we have $d=1$, and $\sh{S}^{(1)}/(f-1)\sh{S}^{(1)}=\sh{S}/(f-1)\sh{S}$ is canonically isomorphic to $\sh{A}$, whence the corollary \sref{II.1.2.2}.
\end{proof}

Let $g\in\Gamma(Y,\sh{O}_Y)$;
if we take $\sh{S}=\sh{O}_Y[T]$, then $g\in\Gamma(Y,\sh{S}_0)$;
let
\[
  h = gT\in\Gamma(Y,\sh{S}_1).
\]
If $X=\Proj(\sh{S})$, then the canonical identification defined in \sref{II.3.1.7} identifies $X_h$ with the open subset $Y_g$ of $Y$ (in the sense of \sref[0]{0.5.5.2}):
indeed, we can restrict to the case where $Y=\Spec(A)$ is affine, and everything then reduces (taking \sref{II.2.2.5} into account) to the fact that the ring of fractions $A_g$ is canonically identified with $A[T]/(gT-1)A[T]$ \sref[0]{0.1.2.3}.

\begin{proposition}[3.1.8]
\label{II.3.1.8}
Let $\sh{S}$ be a quasi-coherent positively-graded $\sh{O}_Y$-algebra.
Then
\begin{enumerate}
  \item[(i)] For all $d>0$, there exists a canonical $Y$-isomorphism from $\Proj(\sh{S})$ to $\Proj(\sh{S}^{(d)})$.
\oldpage[II]{52}
  \item[(ii)] Let $\sh{S}'$ be the graded $\sh{O}_Y$-algebra given by the direct sum of $\sh{O}_Y$ with the $\sh{S}_n$ (for $n\geq0$);
    then $\Proj(\sh{S}')$ and $\Proj(\sh{S})$ are canonically $Y$-isomorphic.
  \item[(iii)] Let $\sh{L}$ be an invertible $\sh{O}_Y$-module \sref[0]{0.5.4.1}, and let $\sh{S}_{(\sh{L})}$ be the graded $\sh{O}_Y$-algebra given by the direct sum of the $\sh{S}_d\otimes\sh{L}^{\otimes d}$ (for $d\geq0$);
    then $\Proj(\sh{S})$ and $\Proj(\sh{S}_{(\sh{L})})$ are canonically $Y$-isomorphic.
\end{enumerate}
\end{proposition}

\begin{proof}
In each of the three cases, it suffices to define the isomorphism locally on $Y$, since the verification of compatibility with the restriction operations from one open subset to a smaller one is trivial.
We can thus suppose that $Y$ is affine, and then (i) follows from \sref{II.2.4.7}[(i)], and (ii) follows from \sref{II.2.4.8}.
As for (iii), if we further suppose that $\sh{L}$ is isomorphic to $\sh{O}_Y$ (which we are allowed to do, since the question is local on $Y$), then the isomorphism between $\Proj(\sh{S})$ and $\Proj(\sh{S}_{(\sh{L})})$ is evident;
to define a \emph{canonical} isomorphism, let $Y=\Spec(A)$ and $\sh{S}=\widetilde{S}$, where $S$ is a graded $A$-algebra, and let $c$ be a generator of the free $A$-module $L$ such that $\sh{L}=\widetilde{L}$;
then, for all $n>0$, $x_n\mapsto x_n\otimes c^{\otimes n}$ is an $A$-isomorphism from $S_n$ to $S_n\otimes L^{\otimes n}$, and these $A$-isomorphisms define an $A$-isomorphism of graded algebras $\varphi_c:S\to S_{(L)}=\bigoplus_{n\geq0}S_n\otimes L^{\otimes n}$.
So let $f\in S_+$ be homogeneous of degree~$d$;
for all $x\in S_{nd}$, we have that $(x\otimes c^{nd})/(f\otimes c^d)^n=(x\otimes(\varepsilon c)^{nd})/(f\otimes(\varepsilon c)^d)^n$ for every invertible element $\varepsilon\in A$, which shows that the isomorphism $S_{(f)}\to(S_{(L)})_{(f\otimes c^d)}$ induced from $\varphi_c$ is \emph{independent} of the generator $c$ of $L$ considered, and thus finishes the proof.
\end{proof}

\begin{env}[3.1.9]
\label{II.3.1.9}
Recall (\sref[0]{0.4.1.3} and \sref[I]{I.1.3.14}) that, for the quasi-coherent graded $\sh{O}_Y$-algebra $\sh{S}$ to be \emph{generated by the $\sh{O}_Y$-module $\sh{S}_1$}, it is necessary and sufficient for there to exist an open affine cover $(U_\alpha)$ of $Y$ such that the graded algebra $\Gamma(U_\alpha,\sh{S})$ over $\Gamma(U_\alpha,\sh{S}_0)$ is generated by the set $\Gamma(U_\alpha,\sh{S}_1)$ of its homogeneous elements of degree~$1$.
For every open $V$ of $Y$, $\sh{S}|V$ is then generated by the $(\sh{O}_Y|V)$-module $\sh{S}_1|V$.
\end{env}

\begin{proposition}[3.1.10]
\label{II.3.1.10}
Suppose that there exists a finite open affine cover $(U_i)$ of $Y$ such that each graded algebra $\Gamma(U_i,\sh{S})$ is of finite type over $\Gamma(U_i\sh{O}_Y)$.
Then there exists $d>0$ such that $\sh{S}^{(d)}$ is generated by $\sh{S}_d$, with $\sh{S}_d$ a $\sh{O}_Y$-module of finite type.
\end{proposition}

\begin{proof}
Indeed, it follows from \sref{II.2.1.6}[(v)] that, for each $i$, there exists an integer $m_i$ such that $\Gamma(U_i,\sh{S}_{nm_i})=(\Gamma(U_i,\sh{S}_{m_i}))^n$ for all $n>0$;
it suffices to take $d$ to be a common multiple of all the $m_i$, taking \sref{II.2.1.6}[(i)] into account.
\end{proof}

\begin{corollary}[3.1.11]
\label{II.3.1.11}
Under the hypotheses of \sref{II.3.1.10}, $\Proj(\sh{S})$ is $Y$-isomorphic to a homogeneous spectrum $\Proj(\sh{S}')$, where $\sh{S}'$ is a graded $\sh{O}_Y$-algebra generated by $\sh{S}'_1$, with $\sh{S}'_1$ an $\sh{O}_Y$-module of finite type.
\end{corollary}

\begin{proof}
It suffices to take $\sh{S}'=\sh{S}^{(d)}$, where $d$ satisfies the property of \sref{II.3.1.10}, and to then apply \sref{II.3.1.8}[(i)]
\end{proof}

\begin{env}[3.1.12]
\label{II.3.12}
If $\sh{S}$ is a quasi-coherent positively-graded $\sh{O}_Y$-algebra, we know \sref[I]{I.5.1.1} that its \emph{nilradical} $\sh{N}$ is a quasi-coherent $\sh{O}_Y$-module;
we say that $\sh{N}_+=\sh{N}\cap\sh{S}_+$ is the \emph{nilradical of $\sh{S}_+$};
this is a quasi-coherent graded $\sh{S}_0$-module, since we can immediately reduce to the case where $Y$ is affine, and the proposition then follows from \sref{II.2.1.10}.
For all $y\in Y$, $(\sh{N}_+)_y$ is then the nilradical of $(\sh{S}_+)_y=(\sh{S}_y)_+$ \sref[I]{I.5.1.1}.
We say that the graded $\sh{O}_Y$-algebra $\sh{S}$ is \emph{essentially reduced} if $\sh{N}_+=0$, which is equivalent
\oldpage[II]{53}
to saying that $\sh{S}_y$ is an essentially reduced graded $\sh{O}_y$-algebra for all $y\in Y$.
For every graded $\sh{O}_Y$-algebra $\sh{S}$, $\sh{S}/\sh{N}_+$ is essentially reduced.

We say that $\sh{S}$ is \emph{integral} if, for all $y\in Y$, $\sh{S}_y$ is an integral ring and, furthermore, $(\sh{S}_y)_+=(\sh{S}_+)_y\neq0$.
\end{env}

\begin{proposition}[3.1.13]
\label{II.3.1.13}
Let $\sh{S}$ be a positively-graded $\sh{O}_Y$-algebra.
If $X=\Proj(\sh{S})$, then the $Y$-scheme $X_\red$ is canonically isomorphic to $\Proj(\sh{S}/\sh{N}_+)$;
in particular, if $\sh{S}$ is essentially reduced, then $X$ is reduced.
\end{proposition}

\begin{proof}
The fact that $X'=\Proj(\sh{S}/\sh{N}_+)$ is reduced follows immediately from \sref{II.2.4.4}[(i)], since the property is local;
further, for every open affine $U\subset Y$, ${\varphi'}^{-1}(U)$ is equal to $(\varphi^{-1}(U))_\red$ (where we denote by $\varphi$ and $\varphi'$ the structure morphisms $X\to Y$ and $X'\to Y$, respectively);
we immediately see that the canonical $U$-morphisms ${\varphi'}^{-1}(U)\to\varphi^{-1}(U)$ are compatible with the restriction operations, and thus define a closed immersion $X'\to X$ that is a homeomorphism of the underlying spaces;
whence the conclusion \sref[I]{I.5.1.2}.
\end{proof}

\begin{proposition}[3.1.14]
\label{II.3.1.14}
Let $Y$ be an integral prescheme, and $\sh{S}$ a quasi-coherent graded $\sh{O}_Y$-algebra such that $\sh{S}_0=\sh{O}_Y$.
\begin{enumerate}
  \item[(i)] If $\sh{S}$ is integral \sref{II.3.1.12}, then $X=\Proj(\sh{S})$ is integral, and the structure morphism $\varphi\colon X\to Y$ is dominant.
  \item[(ii)] Suppose furthermore that $\sh{S}$ is essentially reduced.
    Then, conversely, if $X$ is integral and $\varphi$ is dominant, then $\sh{S}$ is integral.
\end{enumerate}
\end{proposition}

\begin{proof}
\medskip\noindent
\begin{enumerate}
  \item[(i)] If $(U_\alpha)$ is a base of $Y$ consisting of non-empty open affines, then it suffices to prove the proposition in the case where $Y$ is replaced by one of the $U_\alpha$, and $\sh{S}$ by $\sh{S}|U_\alpha$:
    indeed, one one hand it will follow that the underlying space $\varphi^{-1}(U_\alpha)$ are irreducible opens (and thus non-empty) of $X$ such that $\varphi^{-1}(U_\alpha)\cap\varphi^{-1}(U_\beta)\neq\emp$ for any pair of indices (since $U_\alpha\cap U_\beta$ contains some $U_\gamma$), and so $X$ is irreducible \sref[0]{0.2.1.4};
    on the other hand, $X$ will be reduced, since this is a local property, and so $X$ will indeed be integral, with $\varphi(X)$ dense in $Y$.

    So suppose that $Y=\Spec(A)$, where $A$ is integral \sref[I]{I.5.1.4}, and that $\sh{S}=\widetilde{S}$, where $S$ is a graded $A$-algebra;
    the hypothesis is that, for every $y\in Y$, $\widetilde{S}_y=S_y$ is an integral graded ring such that $(S_y)_+\neq0$.
    It suffices to prove that $S$ is an \emph{integral} ring, since then we will have that $S_+\neq0$, and we can then apply \sref{II.2.4.4}[(ii)].
    But let $f,g\neq0$ be elements of $S$, and suppose that $fg=0$;
    for all $y\in Y$ we then have that $(f/1)(g/1)=0$ in $S_y$, and so $f/1=0$ or $g/1=0$ by hypothesis.
    Suppose, for example, that $f/1=0$ in $S_y$;
    this implies that there exists $a\in A$ such that $a\not\in\mathfrak{j}_y$ and $af=0$;
    we then have, for \emph{all} $z\in Y$, that $(a/1)(f/1)=0$ in the \emph{integral} ring $S_z$, and since $a/1\neq0$ (since $A$ is integral), $f/1=0$, which implies that $f=0$.
  \item[(ii)] Since the question is local on $Y$, we can again suppose that $Y=\Spec(A)$, with $A$ integral, and that $\sh{S}=\widetilde{S}$.
    By hypothesis, for all $y\in Y$, $(S_y)_+$ does not contain any non-zero nilpotent elements, and the same is true of $(S_0)_y=A_y$ by hypothesis;
    so $S_y$ is a reduced ring for all $y\in Y$, and we thus conclude first of all that $S$ itself is reduced \sref[I]{I.5.1.1}.
    The hypothesis that $X$ is integral implies that $S$ is essentially integral \sref{II.2.4.4}[(ii)], and everything then reduces to showing that the annihilator $\mathfrak{J}$ of $S_+$ in $A=S_0$ is just $0$ \sref{II.2.1.11}.
    If this were not the case, we would have that $(S_h)_+=0$ for some $h\neq0$ in $\mathfrak{J}$, and thus \sref{II.3.1.1} that $\varphi^{-1}(D(h))=\emp$, and $\varphi(X)$ would not be dense in $Y$, contradicting the hypothesis (since $D(h)\neq\emp$, since $h$ is not nilpotent).
\oldpage[II]{54}
\end{enumerate}
\end{proof}


\subsection{Sheaf on $\operatorname{Proj}(\mathcal{S})$ associated to a graded $\mathcal{S}$-module}
\label{subsection:II.3.2}

\begin{env}[3.2.1]
\label{II.3.2.1}
Let
\end{env}


% \subsection{Graded $\mathcal{S}$-module associated to a sheaf on $\operatorname{Proj}(\mathcal{S})$}
% \label{subsection:II.3.3}


% \subsection{Finiteness conditions}
% \label{subsection:II.3.4}


% \subsection{Functorial behaviour}
% \label{subsection:II.3.5}


% \subsection{Closed subpreschemes of $\operatorname{Proj}(\mathcal{S})$}
% \label{subsection:II.3.6}


% \subsection{Morphisms from a prescheme to a homogeneous spectrum}
% \label{subsection:II.3.7}


% \subsection{Criteria for immersion into a homogeneous spectrum}
% \label{subsection:II.3.8}

\section{Homogeneous prime spectra}
\label{section:II.2}

\subsection{Generalities on graded rings and modules}
\label{subsection:II.2.1}

\begin{notation}[2.1.1]
\label{II.2.1.1}
Given a \emph{positively graded} ring $S$, we denote by $S_n$ the subset of $S$ consisting of homogeneous elements of degree $n$ ($n\geq 0$), by $S_+$ the (direct) sum of the $S_n$ for $n>0$;
we have $1\in S_0$, $S_0$ is a subring of $S$, $S_+$ is a graded ideal of $S$, and $S$ is the direct sum of $S_0$ and $S_+$.
If $M$ is a \emph{graded} module over $S$ (with positive or negative degrees), we similarly denote by $M_n$ the $S_0$-module consisting of homogeneous elements of $M$ of degree $n$ (with $n\in\bb{Z}$).

For every integer $d>0$, we denote by $S^{(d)}$ the direct sum of the $S_{nd}$;
by considering the elements of $S_{nd}$ as homogeneous of degree $n$, the $S_{nd}$ define on $S^{(d)}$ a graded ring structure.

For every integer $k$ such that $0\leq k\leq d-1$, we denote by $M^{(d,k)}$ the direct sum
\oldpage[II]{20}
of the $M_{nd+k}$ ($n\in\bb{Z}$);
this is a graded $S^{(d)}$-module when we consider the elements of $M_{nd+k}$ as homogeneous of degree $n$.
We write $M^{(d)}$ in place of $M^{(d,0)}$.

With the above notation, for every integer $n$ (positive or negative), we denote by $M(n)$ the graded $S$-module defined by $(M(n))_k=M_{n+k}$ for every $k\in\bb{Z}$.
In particular, $S(n)$ will be a graded $S$-module such that $(S(n))_k=S_{n+k}$, by agreeing to set $S_n=0$ for $n<0$.
We say that a graded $S$-module $M$ is \emph{free} if it is isomorphic, considered as a \emph{graded} module, to a direct sum of modules of the form $S(n)$;
as $S(n)$ is a monogeneous $S$-module, generated by the element $1$ of $S$ considered as an element of degree $-n$, it is equivalent to say that $M$ admits a \emph{basis} over $S$ consisting of \emph{homogeneous} elements.

We say that a graded $S$-module $M$ \emph{admits a finite presentation} if there exists an exact sequence $P\to Q\to M\to 0$, where $P$ and $Q$ are finite direct sums of modules of the form $S(n)$ and the homomorphisms are of degree $0$ (cf.~\sref{II.2.1.2}).
\end{notation}

\begin{env}[2.1.2]
\label{II.2.1.2}
Let $M$ and $N$ be two graded $S$-modules;
we define on $M\otimes_S N$ a \emph{graded} $S$-module structure in the following way.
On the tensor product $M\otimes_\bb{Z}N$, we can define a graded $\bb{Z}$-module structure (where $\bb{Z}$ is graded by $\bb{Z}_0=\bb{Z}$, $\bb{Z}_n=0$ for $n\neq 0$) by setting $(M\otimes_\bb{Z}N)_q=\bigoplus_{m+n=q}M_m\otimes_\bb{Z}N_n$ (as $M$ and $N$ are respectively direct sums of the $M_m$ and the $N_n$, we know that we can canonically identify $M\otimes_\bb{Z}N$ with the direct sum of all the $M_m\otimes_\bb{Z}N_n$).
This being so, we have $M\otimes_S N=(M\otimes_\bb{Z}N)/P$, where $P$ is the $\bb{Z}$-submodule of $M\otimes_\bb{Z}N$ generated by the elements $(xs)\otimes y-x\otimes(sy)$ for $x\in M$, $y\in N$, $s\in S$;
it is clear that $P$ is a \emph{graded} $\bb{Z}$-submodule of $M\otimes_\bb{Z}N$, and we see immediately that we obtain a graded $S$-module structure on $M\otimes_S N$ by passing to the quotient.

For two graded $S$-modules $M$ and $N$, recall that a homomorphism $u:M\to N$ of $S$-modules is said to be \emph{of degree $k$} if $u(M_j)\subset N_{j+k}$ for all $j\in\bb{Z}$.
If $H_n$ denotes the set of all the homomorphisms of degree $n$ from $M$ to $N$, then we denote by $\Hom_S(M,N)$ the (direct) \emph{sum} of the $H_n$ ($n\in\bb{Z}$) in the $S$-module $H$ of all the homomorphisms (of $S$-modules) from $M$ to $N$;
in general, $\Hom_S(M,N)$ is not equal to the later.
However, we have $H=\Hom_S(M,N)$ when $M$ is \emph{of finite type};
indeed, we can then suppose that $M$ is generated by a finite number of homogeneous elements $x_i$ ($1\leq i\leq n$), and every homomorphism $u\in H$ can be written in a unique way as $\sum_{k\in\bb{Z}}u_k$, where for each $k$, $u_k(x_i)$ is equal to the homogeneous component of degree $k+\deg(x_i)$ of $u(x_i)$ ($1\leq i\leq n$), which implies that $u_k=0$ except for a finite number of indices;
we have by definition that $u_k\in H_k$, hence the conclusion.

We say that the elements of degree $0$ of $\Hom_S(M,N)$ are the \emph{homomorphisms of graded $S$-modules}.
It is clear that $S_m H_n\subset H_{m+n}$, so the $H_n$ define on $\Hom_S(M,N)$ a graded $S$-module structure.

It follows immediately from these definitions that we have
\[
\label{II.2.1.2.1}
  M(m)\otimes_S N(n)=(M\otimes_S N)(m+n),
\tag{2.1.2.1}
\]
\[
\label{II.2.1.2.2}
  \Hom_S(M(m),N(n))=(\Hom_S(M,N))(n-m),
\tag{2.1.2.2}
\]
for two graded $S$-modules $M$ and $N$.

\oldpage[II]{21}
Let $S$ and $S'$ be two graded rings;
a homomorphism of \emph{graded rings $\vphi:S\to S'$} is a homomorphism of rings such that $\vphi(S_n)\subset S_n'$ for all $n\in\bb{Z}$ (in other words, $\vphi$ must be a homomorphism \emph{of degree $0$} of graded $\bb{Z}$-modules).
The data of such a homomorphism defines on $S'$ a \emph{graded} $S'$-module structure;
equipped with this structure and its graded ring structure, we say that $S'$ is a \emph{graded $S'$-algebra}.

If $M$ is also a graded $S$-module, then the tensor product $M\otimes_S S'$ of \emph{graded} $S$-modules is equipped in a natural way with a \emph{graded} $S'$-module structure, the grading being defined as above.
\end{env}

\begin{lemma}[2.1.3]
\label{II.2.1.3}
Let $S$ be a ring graded in positive degrees.
For a subset $E$ of $S_+$ consisting of homogeneous elements to generate $S_+$ as an $S$-module, it is necessary and sufficient for $E$ to generate $S$ an an $S_0$-algebra.
\end{lemma}

\begin{proof}
The condition is evidently sufficient; we show that it is necessary.
Let $E_n$ (resp. $E^n$) be the set of elements of $E$ equal to $n$ (resp. $\leq n$);
it suffices to show, by induction on $n>0$, that $S_n$ is the $S_0$-module generated by the elements of degree $n$ which are products of elements of $E^n$.
This is evident for $n=1$ by virtue of the hypothesis;
the latter also shows that $S_n=\sum_{p=0}^{n-1}S_p E_{n-p}$, and the induction argument is then immediate.
\end{proof}

\begin{corollary}[2.1.4]
\label{II.2.1.4}
For $S_+$ to be an ideal of finite type, it is necessary and sufficient for $S$ to be an $S_0$-algebra of finite type.
\end{corollary}

\begin{proof}
We can always assume that a finite system of generators of the $S_0$-algebra $S$ (resp. of the $S$-ideal $S_+$) consists of homogeneous elements, by replacing each of the generators considered by its homogeneous components.
\end{proof}

\begin{corollary}[2.1.5]
\label{II.2.1.5}
For $S$ to be Noetherian, it is necessary and sufficient for $S_0$ to be Noetherian and for $S$ to be an $S_0$-algebra of finite type.
\end{corollary}

\begin{proof}
The condition is evidently sufficient;
it is necessary, since $S_0$ is isomorphic to $S/S_+$ and $S_+$ must be an ideal of finite type \sref{II.2.1.4}.
\end{proof}

\begin{lemma}[2.1.6]
\label{II.2.1.6}
Let $S$ be a ring graded in positive degrees, which is an $S_0$-algebra of finite type.
Let $M$ be a graded $S$-module of finite type.
Then:
\begin{enumerate}
  \item[{\rm(i)}] The $M_n$ are $S_0$-modules of finite type, and there exists an integer $n_0$ such that $M_n=0$ for $n\leq n_0$.
  \item[{\rm(ii)}] There exists an integer $n_1$ and an integer $h>0$ such that, for every integer $n\geq n_1$, we have $M_{n+h}=S_h M_n$.
  \item[{\rm(iii)}] For every pair of integers $(d,k)$ such that $d>0$, $0\leq k\leq d-1$, $M^{(d,k)}$ is an $S^{(d)}$-module of finite type.
  \item[{\rm(iv)}] For every integer $d>0$, $S^{(d)}$ is an $S_0$-algebra of finite type.
  \item[{\rm(v)}] There exists an integer $h>0$ such that $S_{mh}=(S_h)^m$ for all $m>0$.
  \item[{\rm(vi)}] For every integer $n>0$, there exists an integer $m_0$ such that $S_m\subset S_+^n$ for all $m\geq m_0$.
\end{enumerate}
\end{lemma}

\begin{proof}
We can assume that $S$ is generated (as an $S_0$-algebra) by homogeneous elements $f_i$, of degrees $h_i$ ($1\leq i\leq r$), and $M$ is generated (as an $S$-module) by homogeneous elements $x_j$ of degrees $k_j$ ($1\leq j\leq s$).
It is clear that $M_n$ is formed by linear combinations,
\oldpage[II]{22}
with coefficients in $S_0$, of elements $f_1^{\alpha_1}\cdots f_r^{\alpha_r}x_j$ such that the $\alpha_i$ are integers $\geq 0$ satisfying $k_j+\sum_i\alpha_i h_i=n$;
for each $j$, there are only finitely many systems $(\alpha_i)$ satisfying this equation, since the $h_i$ are $>0$, hence the first assertion of (i);
the second is evident.
On the other hand, let $h$ be the l.c.m. of the $h_i$ and set $g_i=f_i^{h/h_i}$ ($1\leq i\leq r$) such that all the $g_i$ are of degree $h$;
let $z_\mu$ be the elements of $M$ of the form $f_1^{\alpha_1}\cdots f_r^{\alpha_r}x_j$ with $0\leq\alpha_i<h/h_i$ for $1\leq i\leq r$;
there are finitely many of these elements, so let $n_1$ be the largest of their degrees.
It is clear that for $n\geq n_1$, every element of $M_{n+h}$ is a linear combination of the $z_\mu$ whose coefficients are monomials of degree $>0$ with respect to the $g_i$, so we have $M_{n+h}=S_h M_n$, which establishes (ii).
In a similar way, we see (for all $d>0$) that an element of $M^{(d,k)}$ is a linear combinations, with coefficients in $S_0$, of elements of the form $g^d f_1^{\alpha_1}\cdots f_r^{\alpha_r}x_j$ with $0\leq\alpha_i<d$, $g$ being a homogeneous element of $S$;
hence (iii);
(iv) then follows from (iii) and from Lemma~\sref{II.2.1.3}, by taking $M=S_+$, since $(S_+)^{(d)}=(S^{(d)})_+$.
The assertion of (v) is deduced from (ii) by taking $M=S$.
Finally, for a given $n$, there are finitely many systems $(\alpha_i)$ such that $\alpha_i\geq 0$ and $\sum_i\alpha_i<n$, so if $m_0$ is the largest value of the sum $\sum_i\alpha_i h_i$ of these systems, then we have $S_m\subset S_+^n$ for $m>m_0$, which proves (vi).
\end{proof}

\begin{corollary}[2.1.7]
\label{II.2.1.7}
If $S$ is Noetherian, then so is $S^{(d)}$ for every integer $d>0$.
\end{corollary}

\begin{proof}
This follows from \sref{II.2.1.5} and \sref{II.2.1.6}[iv].
\end{proof}

\begin{env}[2.1.8]
\label{II.2.1.8}
Let $\mathfrak{p}$ be a \emph{graded} prime ideal of the graded ring $S$;
$\mathfrak{p}$ is thus a direct sum of the subgroups $\mathfrak{p}_n=\mathfrak{p}\cap S_n$.
Suppose that \emph{$\mathfrak{p}$ does not contain $S_+$}.
Then if $f\in S_+$ is not in $\mathfrak{p}$, the relation $f^n x\in\mathfrak{p}$ is equivalent to $x\in\mathfrak{p}$;
in particular, if $f\in S_d$ ($d>0$), for all $x\in S_{m-nd}$, then the relation $f^n x\in\mathfrak{p}_m$ is equivalent to $x\in\mathfrak{p}_{m-nd}$.
\end{env}

\begin{proposition}[2.1.9]
\label{II.2.1.9}
Let $n_0$ be an integer $>0$;
for all $n\geq n_0$, let $\mathfrak{p}_n$ be a subgroup of $S_n$.
For there to exist a graded prime ideal $\mathfrak{p}$ of $S$ not containing $S_+$ and such that $\mathfrak{p}\cap S_n=\mathfrak{p}_n$ for all $n\geq n_0$, it is necessary and sufficient for the following conditions to be satisfied:
\begin{enumerate}
  \item[{\rm(1st)}] $S_m\mathfrak{p}_n\subset\mathfrak{p}_{m+n}$ for all $m\geq 0$ and all $n\geq n_0$.
  \item[{\rm(2nd)}] For $m\geq n_0$, $n\geq n_0$, $f\in S_m$, $g\in S_n$, the relation $fg\in\mathfrak{p}_{m+n}$ implies $f\in\mathfrak{p}_m$ or $g\in\mathfrak{p}_n$.
  \item[{\rm(3rd)}] $\mathfrak{p}_n\neq S_n$ for at least one $n\geq n_0$.
\end{enumerate}
In addition, the graded prime ideal $\mathfrak{p}$ is then unique.
\end{proposition}

\begin{proof}
It is evident that the conditions (1st) and (2nd) are necessary.
In addition, if $\mathfrak{p}\not\supset S_+$, then there exists at least one $k>0$ such that $\mathfrak{p}\cap S_k\neq S_k$;
if $f\in S_k$ is not in $\mathfrak{p}$, the relation $\mathfrak{p}\cap S_n=S_n$ implies $\mathfrak{p}\cap S_{n-mk}=S_{n-mk}$ according to \sref{2.2.1.8};
therefore, if $\mathfrak{p}\cap S_n=S_n$ for a certain value of $n$, we would have $\mathfrak{p}\supset S_+$ contrary to the hypothesis, which proves that (3rd) is necessary.
Conversely, suppose that the conditions (1st), (2nd), and (3rd) are satisfied.
Note that if for an integer $d\geq n_0$, $f\in S_d$ is not in $\mathfrak{p}_d$, then, if $\mathfrak{p}$ exists, $\mathfrak{p}_m$, for $m<n_0$, is necessarily equal to the set of the $x\in S_m$ such that $f^r x\in\mathfrak{p}_{m+rd}$, except for a finite number of values of $r$.
This already proves that if $\mathfrak{p}$ exists, then it is unique.
It remains to show that if we define the $\mathfrak{p}_m$ for $m<n_0$ by the previous condition, then $\mathfrak{p}=\sum_{n=0}^\infty\mathfrak{p}_n$ is a prime ideal.
First, note that by virtue of (2nd), for $m\geq n_0$, $\mathfrak{p}_m$ is also defined as the set of the $x\in S_m$ such that $f^r x\in\mathfrak{p}_{m+rd}$ except for a finite number of values of $r$.
This
\oldpage[II]{23}
being so, if $g\in S_m$, $x\in\mathfrak{p}_n$, then we have $f^r gx\in\mathfrak{p}_{m+n+rd}$ except for a finite number of values of $r$, so $gx\in\mathfrak{p}_{m+n}$, which proves that $\mathfrak{p}$ is an ideal of $S$.
To establish that this ideal is prime, in other words that the ring $S/\mathfrak{p}$, graded by the subgroups $S_n/\mathfrak{p}_n$, is an integral domain, it suffices (by considering the components of higher degree of two elements of $S/\mathfrak{p}$) to prove that if $x\in S_m$ and $y\in S_n$ are such that $x\not\in\mathfrak{p}_m$ and $y\not\in\mathfrak{p}_n$, then $xy\not\in\mathfrak{p}_{m+n}$.
If not, for $r$ large enough, we would have $f^{2r}xy\in\mathfrak{p}_{m+n+2rd}$;
but we have $f^r y\not\in\mathfrak{p}_{n+rd}$ for all $r>0$;
it then follows from (2nd) that, except for a finite number of values of $r$, we have $f^r x\in\mathfrak{p}_{m+rd}$, and we conclude that $x\in\mathfrak{p}_m$ contrary to the hypothesis.
\end{proof}

\begin{env}[2.1.10]
\label{II.2.1.10}
We say that a subset $\mathfrak{J}$ of $S_+$ is an \emph{ideal of $S_+$} if it is an ideal of $S$, and $\mathfrak{J}$ is a \emph{graded prime ideal of $S_+$} if it is the intersection of $S_+$ and a graded prime ideal of $S$ \emph{not containing $S_+$} (this prime ideal is also unique according to Proposition~\sref{II.2.1.9}).
If $\mathfrak{J}$ is an ideal of $S_+$, the \emph{radical of $\mathfrak{J}$ in $S_+$} is the set of elements of $S_+$ which have a power in $\mathfrak{J}$, in other words the set $\rad_+(\mathfrak{J})=\rad(\mathfrak{J})\cap S_+$;
in particular, the radical of $0$ in $S_+$ is then called the \emph{nilradical} of $S_+$ and denoted by $\nilrad_+$: this is the set of nilpotent elements of $S_+$.
If $\mathfrak{J}$ is an \emph{graded} ideal of $S_+$, then its radical $\rad_+(\mathfrak{J})$ is a \emph{graded} ideal: by passing to the quotient ring $S/\mathfrak{J}$, we can reduce to the case $\mathfrak{J}=0$, and it remains to see that if $x=x_h+x_{h+1}+\cdots+x_k$ is nilpotent, then so are the $x_i\in S_i$ ($1\leq h\leq i\leq k$);
we can assume $x_k\neq 0$ and the component of highest degree of $x^n$ is then $x_k^n$, hence $x_k$ is nilpotent, and we then argue by induction on $k$.
We say that the graded ring $S$ is \emph{essentially reduced} if $\nilrad_+=0$, in other words, if $S_+$ does not contain nilpotent elements $\neq 0$.
\end{env}

\begin{env}[2.1.11]
\label{II.2.1.11}
We note that if, in the graded ring $S$, an element $x$ is a zero-divisor, then so is its component of highest degree.
We say that a ring $S$ is \emph{essentially integral} if the ring $S_+$ (\emph{without the unit element}) does not contain a zero-divisor and is $\neq 0$;
it suffices that a homogeneous element $\neq 0$ in $S_+$ is not a zero-divisor in this ring.
It is clear that if $\mathfrak{p}$ is a graded prime ideal of $S_+$, then $S/\mathfrak{p}$ is essentially integral.

Let $S$ be an essentially integral graded ring, and let $x_0\in S_0$:
if there then exists \emph{a} homogeneous element $f\neq 0$ of $S_+$ such that $x_0 f=0$, then we have $x_0 S_+=0$, since we have $(x_0 g)f=(x_0 f)g=0$ for all $g\in S_+$, and the hypothesis thus implies $x_0 g=0$.
For $S$ to be integral, it is necessary and sufficient for $S_0$ to be integral and the annihilator of $S_+$ in $S_0$ to be $0$.
\end{env}


\subsection{Rings of fractions of a graded ring}
\label{subsection:II.2.2}

\begin{env}[2.2.1]
\label{II.2.2.1}
Let $S$ be a graded ring, in positive degrees, $f$ a \emph{homogeneous} element of $S$, of degree $d>0$;
then the ring of fractions $S'=S_f$ is graded, taking for $S_n'$ the set of the $x/f^k$, where $x\in S_{n+kd}$ with $k\geq 0$ (we observe here that $n$ can take arbitrary negative values);
we denote the subring $S_0'=(S_f)_0$ of $S'$ consisting of elements \emph{of degree $0$} by the notation $S_{(f)}$.

If $f\in S_d$, then the monomials $(f/1)^h$ in $S_f$ ($h$ a positive or negative integer) form a \emph{free system} over the ring $S_{(f)}$, and the set of their linear combinations is none other than
\oldpage[II]{24}
the ring $(S^{(d)})_f$, which is thus \emph{isomorphic to $S_{(f)}[T,T^{-1}]=S_{(f)}\otimes_\bb{Z}\bb{Z}[T,T^{-1}]$} (where $T$ is an indeterminate).
Indeed, if we have a relation $\sum_{h=-a}^b z_h(f/1)^h=0$ with $z_h=x_h/f^m$, where the $x_h$ are in $S_{md}$, then this relation is equivalent by definition to the existence of a $k>-a$ such that $\sum_{h=-a}^b f^{h+k}x_h=0$, and as the degrees of the terms of this sum are distinct, we have $f^{h+k}x_h=0$ for all $h$, hence $z_h=0$ for all $h$.

If $M$ is a graded $S$-module, then $M'=M_f$ is a graded $S_f$-module, $M_n'$ being the set of the $z/f^k$ with $z\in M_{n+kd}$ ($k\geq 0$);
we denote by $M_{(f)}$ the set of the homomogenous elements of degree $0$ of $M'$;
it is immediate that $M_{(f)}$ is an $S_{(f)}$-module and that we have $(M^{(d)})_f=M_{(f)}\otimes_{S_{(f)}}(S^{(d)})_f$.
\end{env}

\begin{lemma}[2.2.2]
\label{II.2.2.2}
Let $d$ and $e$ be integers $>0$, $f\in S_d$, $g\in S_e$.
There exists a canonical ring isomorphism
\[
  S_{(fg)}\isoto(S_{(f)})_{g^d/f^e};
\]
if we canonically identify these two rings, then there exists a canonical module isomorphism
\[
  M_{(fg)}\isoto(M_{(f)})_{g^d/f^e}.
\]
\end{lemma}

\begin{proof}
Indeed, $fg$ divides $f^e g^d$, and this latter element divides $(fg)^{de}$, so the graded rings $S_{fg}$ and $S_{f^e g^d}$ are canonically identified;
on the other hand, $S_{f^e g^d}$ also identifies with $(S_{f^e})_{g^d/1}$ \sref[0]{0.1.4.6}, and as $f^e/1$ is invertible in $S_{f^e}$, $S_{f^e g^d}$ also identifies with $(S_{f^e})_{g^d/f^e}$.
The element $g^d/f^e$ is of degree $0$ in $S_{f^e}$;
we immediately conclude that the subring of $(S_{f^e})_{g^d/f^e}$ consisting of elements of degree $0$ is $(S_{(f^e)})_{g^d/f^e}$, and as we evidently have $S_{(f^e)}=S_{(f)}$, this proves the first part of the proposition;
the second is established in a similar way.
\end{proof}

\begin{env}[2.2.3]
\label{II.2.2.3}
Under the hypotheses of \sref{II.2.2.2}, it is clear that the canonical homomorphism $S_f\to S_{fg}$ \sref[0]{0.1.4.1}, which sends $x/f^k$ to $g^k x/(fg)^k$, is of degree $0$, thus gives by restriction a \emph{canonical homomorphism $S_{(f)}\to S_{(fg)}$}, such that the diagram
\[
  \xymatrix{
    & S_{(f)}\ar[dl]\ar[dr]\\
    S_{(fg)}\ar[rr]^-{\sim} & &
    (S_{(f)})_{g^d/f^e}
  }
\]
is commutative.
We similarly define a canonical homomorphism $M_{(f)}\to M_{(fg)}$.
\end{env}

\begin{lemma}[2.2.4]
\label{II.2.2.4}
If $f$ and $g$ are two homogeneous elements of $S_+$, then the ring $S_{(fg)}$ is generated by the union of the canonical images of $S_{(f)}$ and $S_{(g)}$.
\end{lemma}

\begin{proof}
By virtue of Lemma~\sref{II.2.2.2}, it suffices to see that $1/(g^d/f^e)=f^{d+e}/(fg)^d$ belongs to the canonical image of $S_{(g)}$ in $S_{(fg)}$, which is evident by definition.
\end{proof}

\begin{proposition}[2.2.5]
\label{II.2.2.5}
Let $d$ be an integer $>0$ and let $f\in S_d$.
Then there exists a canonical ring isomorphisms $S_{(f)}\isoto S^{(d)}/(f-1)S^{(d)}$;
if we identify these two rings by this isomorphism, then there exists a canonical module isomorphism $M_{(f)}\isoto M^{(d)}/(f-1)M^{(d)}$.
\end{proposition}

\begin{proof}
The first of these isomorphisms is defined by sending $x/f^n$, where $x\in S_{nd}$, to the element $\overline{x}$, the class of $x\text{ mod. }(f-1)S^{(d)}$;
this map is well-defined, because we have the congruence $f^h x\equiv x\,(\text{mod.}\,(f-1)S^{(d)})$ for all $x\in S^{(d)}$, so if $f^h x=0$ for an $h>0$,
\oldpage[II]{25}
then we have $\overline{x}=0$.
On the other hand, if $x\in S_{nd}$ is such that $x=(f-1)y$ with $y=y_{hd}+y_{(h+1)d}+\cdots+y_{kd}$ with $y_{jd}\in S_{jd}$ and $y_{hd}\neq 0$, then we necessarily have $h=n$ and $x=-y_{hd}$, as well as the relations $y_{(j+1)d}=fy_{jd}$ for $h\leq j\leq k-1$, $fy_{kd}=0$, which ultimately gives $f^{k-n}x=0$;
we send every class $\overline{x}\text{ mod. }(f-1)S^{(d)}$ of an element $x\in S_{nd}$ to the element $x/f^n$ of $S_{(f)}$, since the preceding remark shows that this map is well-defined.
It is immediate that these two maps thus defined are ring homomorphisms, each the reciprocal of the other.
We proceed exactly the same way for $M$.
\end{proof}

\begin{corollary}[2.2.6]
\label{II.2.2.6}
If $S$ is Noetherian, then so is $S_{(f)}$ for $f$ homogeneous of degree $>0$.
\end{corollary}

\begin{proof}
This follows immediately from Corollary~\sref{II.2.1.7} and Proposition~\sref{II.2.2.5}.
\end{proof}

\begin{env}[2.2.7]
\label{II.2.2.7}
Let $T$ be a multiplicative subset of $S_+$ consisting of \emph{homogeneous} elements;
$T_0=T\cup\{1\}$ is then a multiplicative subset of $S$;
as the elements of $T_0$ are homogeneous, the ring $T_0^{-1}S$ is still graded in the evident way;
we denote by $S_{(T)}$ the subring of $T_0^{-1}S$ consisting of elements of order $0$, that is to say, the elements of the form $x/h$, where $h\in T$ and $x$ is homogeneous of degree equal to that of $h$.
We know \sref[0]{0.1.4.5} that $T_0^{-1}S$ is canonically identified with the inductive limit of the rings $S_f$, where $f$ varies over $T$ (with respect to the canonical homomorphisms $S_f\to S_{fg}$);
as this identification respects the degrees, it identifies $S_{(T)}$ with the \emph{inductive limit} of the $S_{(f)}$ for $f\in T$.
For every graded $S$-module $M$, we similarly define the module $M_{(T)}$ (over the ring $S_{(T)}$) consisting of elements of degree $0$ of $T_0^{-1}M$, and we see that this module is the inductive limit of the $M_{(f)}$ for $f\in T$.

If $\mathfrak{p}$ is a graded prime ideal of $S_+$, then we denote by $S_{(\mathfrak{p})}$ and $M_{(\mathfrak{p})}$ the ring $S_{(T)}$ and the module $M_{(T)}$ respectively, where $T$ is the set of \emph{homogeneous} elements of $S_+$ which do not belong to $\mathfrak{p}$.\end{env}


\subsection{Homogeneous prime spectrum of a graded ring}
\label{subsection:II.2.3}

\begin{env}[2.3.1]
\label{II.2.3.1}
Given a graded ring $S$, in positive degrees, we call the \emph{homogeneous prime spectrum} of $S$ and denote it by $\Proj(S)$ the set of graded prime ideals of $S_+$ \sref{II.2.1.10}, or equivalently the set of graded prime ideals of $S$ \emph{not containing $S_+$};
we will define a \emph{scheme} structure having $\Proj(S)$ as the underlying set.
\end{env}

\begin{env}[2.3.2]
\label{II.2.3.2}
For every subset $E$ of $S$, let $V_+(E)$ be the set of graded prime ideals of $S$ containing $E$ and not containing $S_+$;
this is thus the subset $V(E)\cap\Proj(S)$ of $\Spec(S)$.
From \sref[I]{I.1.1.2} we deduce:
\[
\label{II.2.3.2.1}
  V_+(0)=\Proj(S),\ V_+(S)=V_+(S_+)=\emp,
\tag{2.3.2.1}
\]
\[
\label{II.2.3.2.2}
  V_+\big(\textstyle\bigcup_\lambda E_\lambda\big)=\textstyle\bigcap_\lambda V_+(E_\lambda),
\tag{2.3.2.2}
\]
\[
\label{II.2.3.2.3}
  V_+(EE')=V_+(E)\cup V_+(E').
\tag{2.3.2.3}
\]

We do not change $V_+(E)$ by replacing $E$ with the graded ideal generated by $E$;
in addition, if $\mathfrak{J}$ is a graded ideal of $S$, then we have
\[
  V_+(\mathfrak{J})=V_+\big(\textstyle\bigcup_{q\geq n}(\mathfrak{J}\cap S_q)\big)
\tag{2.3.2.4}
\]
\oldpage[II]{26}
for all $n>0$: indeed, if $\mathfrak{p}\in\Proj(S)$ contains the homogeneous elements of $\mathfrak{J}$ of degree $\geq n$, then as by hypothesis there exists a homogeneous element $f\in S_d$ not contained in $\mathfrak{p}$, for every $m\geq 0$ and every $x\in S_m\cap\mathfrak{J}$, we have $f^r x\in\mathfrak{J}\cap S_{m+rd}$ for all but finitely many values of $r$, so $f^r x\in\mathfrak{p}\cap S_{m+rd}$, which implies that $x\in\mathfrak{p}\cap S_m$ \sref{II.2.1.9}.

Finally, we have, for every graded ideal $\mathfrak{J}$ of $S$,
\[
  V_+(\mathfrak{J})=V_+(\rad_+(\mathfrak{J})).
\tag{2.3.2.5}
\]
\end{env}

\begin{env}[2.3.3]
\label{II.2.3.3}
By definition, the $V_+(E)$ are the closed subsets of $X=\Proj(S)$ for the topology induced by the spectral topology of $\Spec(S)$, which we also call the \emph{spectral topology} on $X$.
For all $f\in S$, we set
\[
\label{II.2.3.3.1}
  D_+(f) = D(f)\cap\Proj(S) = \Proj(S)\setmin V_+(f)
\tag{2.3.3.1}
\]
and so, for any two elements $f$ and $g$ of $S$ \sref[I]{I.1.1.9.1},
\[
\label{II.2.3.3.2}
  D_+(fg) = D_+(f)\cap D_+(g).
\tag{2.3.3.2}
\]
\end{env}

\begin{proposition}[2.3.4]
\label{II.2.3.4}
The $D_+(f)$, as $f$ runs over the set of homogeneous elements of $S_+$, form a base for the topology of $X=\Proj(S)$.
\end{proposition}

\begin{proof}
It follows from \sref{II.2.3.2.2} and \sref{II.2.3.2.4} that every closed subset of $X$ is the intersection of sets of the form $V_+(f)$, where $f$ is homogeneous of degree $>0$.
\end{proof}

\begin{env}[2.3.5]
\label{II.2.3.5}
Let $f$ be a \emph{homogeneous} element of $S_+$, of degree $d>0$;
for every graded prime ideal $\mathfrak{p}$ of $S$ that does not contain $f$, we know that the set of the $x/f^n$, where $x\in\mathfrak{p}$ and $n\geq0$, is a prime ideal of the ring of fractions $S_f$ \sref[0]{0.1.2.6};
its intersection with $S_{(f)}$ is thus a prime ideal of $S_{(f)}$, which we denote by $\psi_f(\mathfrak{p})$:
it is the set of the $x/f^n$ for $n\geq0$ and $x\in\mathfrak{p}\cap S_{nd}$.
We have thus defined a map
\[
  \psi_f: D_+(f)\to\Spec(S_{(f)});
\]
furthermore, if $g\in S_e$ is another homogeneous element of $S_+$, then we have a commutative diagram
\[
\label{II.2.3.5.1}
  \xymatrix{
    D_+(f) \ar[r]^{\psi_f}
    & \Spec(S_{(f)})
  \\D_+(fg) \ar[u] \ar[r]_{\psi_{fg}}
    & \Spec(S_{(fg)}) \ar[u]
  }
\tag{2.3.5.1}
\]
where the vertical arrow on the left is the inclusion, and the vertical arrow on the right is the map ${}^a\!\omega_{fg,f}$ induced by the canonical homomorphism $\omega=\omega_{fg,f}: S_{(f)}\to S_{(fg)}$ \sref[I]{I.1.2.1}.
Indeed, if $x/f^n\in\omega^{-1}(\psi_{fg}(\mathfrak{p}))$, with $fg\not\in\mathfrak{p}$, then, by definition, $g^nx/(fg)^n\in\psi_{fg}(\mathfrak{p})$, so $g^nx\in\mathfrak{p}$, and so $x\in\mathfrak{p}$;
the converse is evident.
\end{env}

\begin{proposition}[2.3.6]
\label{II.2.3.6}
The map $\psi_f$ is a homeomorphism from $D_+(f)$ to $\Spec(S_{(f)})$.
\end{proposition}

\begin{proof}
Firstly, $\psi_f$ is continuous;
this is since, if $h\in S_{nd}$ is such that $h/f^n\in\psi_f(\mathfrak{p})$, then, by definition, $h\in\mathfrak{p}$, and conversely, and so $\psi_f^{-1}(D(h/f^n))=D_+(hf)$, and our claim then follows from \sref{II.2.3.3.2}.
Furthermore, the $D_+(hf)$, where $h$ runs over the sets $S_{nd}$, form a topology of $D_+(f)$, by \sref{II.2.3.4} and \sref{II.2.3.3.2};
the
\oldpage[II]{27}
above thus proves, taking into account the ($T_0$) axiom, which holds in $D_+(f)$ and in $\Spec(S_{(f)})$, that $\psi_f$ is injective and that the inverse map $\psi_f(D_+(f))\to D_+(f)$ is continuous.
Finally, to see that $\psi_f$ is surjective, we note that, if $\mathfrak{q}_0$ is a prime ideal of $S_{(f)}$, and if, for all $n>0$, we denote by $\mathfrak{p}_n$ the set of $x\in S_n$ such that $x^d/f^n\in\mathfrak{q}_0$, then the $\mathfrak{p}_n$ satisfy the conditions of \sref{II.2.1.9}:
if $x,y\in S_n$ are such that $x^d/f^n,y^d/f^n\in\mathfrak{q}_0$, then $(x+y)^{2d}/f^{2n}\in\mathfrak{q}_0$, whence $(x+y)^d/f^n\in\mathfrak{q}_0$, since $\mathfrak{q}_0$ is prime;
this proves that the $\mathfrak{p}_n$ are subgroups of the $S_n$, and the verification of the other conditions of \sref{II.2.1.9} is immediate, taking into account the fact that $\mathfrak{q}_0$ is prime.
If $\mathfrak{p}$ is the graded prime ideal of $S$ thus defined, then indeed $\psi_f(\mathfrak{p})=\mathfrak{q}_0$, since, if $x\in S_{nd}$, then having $x/f^n\in\mathfrak{q}_0$ and $x^d/f^{nd}\in\mathfrak{q}_0$ is equivalent to $\mathfrak{q}_0$ being prime.
\end{proof}

\begin{corollary}[2.3.7]
\label{II.2.3.7}
To have $D_+(f)=\emp$, it is necessary and sufficient for $f$ to be nilpotent.
\end{corollary}

\begin{proof}
To have $\Spec(S_{(f)})=\emp$, it is necessary and sufficient to have $S_{(f)}=0$, or indeed to have $1=0$ in $S_f$, which means, by definition, that $f$ is nilpotent.
\end{proof}

\begin{corollary}[2.3.8]
\label{II.2.3.8}
Let $E$ be a subset of $S_+$.
Then the following conditions are equivalent:
\begin{enumerate}
  \item[{\rm(a)}] $V_+(E) = X = \Proj(S)$.
  \item[{\rm(b)}] Every element of $E$ is nilpotent.
  \item[{\rm(c)}] The homogeneous components of every element of $E$ are nilpotent.
\end{enumerate}
\end{corollary}

\begin{proof}
It is clear that (c) implies (b), and that (b) implies (a).
If $\mathfrak{J}$ is the graded ideal of $S$ generated by $E$, then condition~(a) is equivalent to requiring that $V_+(\mathfrak{J})=X$;
\emph{a fortiori}, (a) implies that every homogeneous element $f\in\mathfrak{J}$ is such that $V_+(f)=X$, and so $f$ is nilpotent by \sref{II.2.3.7}.
\end{proof}

\begin{corollary}[2.3.9]
\label{II.2.3.9}
If $\mathfrak{J}$ is a graded ideal of $S_+$, then $\mathfrak{r}_+(\mathfrak{J})$ is the intersection of the graded prime ideals of $S_+$ that contain $\mathfrak{J}$.
\end{corollary}

\begin{proof}
By considering the graded ring $S/\mathfrak{J}$, we can reduce to the case where $\mathfrak{J}=0$.
We need to prove that, if $f\in S_+$ is not nilpotent, then there exists a graded prime ideal of $S$ that does not contain $f$;
but at least one of the homogeneous components of $f$ is not nilpotent, and we can thus suppose $f$ to be homogeneous;
the claim then follows from \sref{II.2.3.7}.
\end{proof}

\begin{env}[2.3.10]
\label{II.2.3.10}
For every subset $Y$ of $X=\Proj(S)$, let $\mathfrak{j}_+(Y)$ be the set of $f\in S_+$ such that $Y\subset V_+(f)$;
this is equivalent to saying that $\mathfrak{j}_+(Y)=\mathfrak{j}(Y)\cap S_+$;
then $\mathfrak{j}_+(Y)$ is an ideal of $S_+$ that is equal to its radical in $S_+$.
\end{env}

\begin{proposition}[2.3.11]
\label{II.2.3.11}
\begin{enumerate}
  \item[{\rm(i)}] For every subset $E$ of $S_+$, $\mathfrak{j}_+(V_+(E))$ is the radical in $S_+$ of the graded ideal of $S_+$ generated by $E$.
  \item[{\rm(ii)}] For every subset $Y$ of $X$, $V_+(\mathfrak{j}_+(Y))=\overline{Y}$, where $\overline{Y}$ is the closure of $Y$ in $X$.
\end{enumerate}
\end{proposition}

\begin{proof}
\medskip\noindent
\begin{enumerate}
  \item[{\rm(i)}] If $\mathfrak{J}$ is the graded ideal of $S_+$ generated by $E$, then $V_+(E)=V_+(\mathfrak{J})$, and the claim then follows from \sref{II.2.3.9}.
  \item[{\rm(ii)}] Since $V_+(\mathfrak{J})=\bigcap_{f\in\mathfrak{J}}V_+(f)$, having $Y\subset V_+(\mathfrak{J})$ implies that $Y\subset V_+(f)$ for every $f\in\mathfrak{J}$, and thus $\mathfrak{j}_+(Y)\supset\mathfrak{J}$, whence $V_+(\mathfrak{j}_+(Y))\subset V_+(\mathfrak{J})$, which proves (ii) by the definition of the closed subsets.
\end{enumerate}
\end{proof}

\begin{corollary}[2.3.12]
\label{II.2.3.12}
The closed subsets $Y$ of $X=\Proj(S)$ are in bijective correspondence with the graded ideals of $S_+$ that are equal to their radical in $S_+$, via the inclusion-reversing maps $Y\mapsto\mathfrak{j}_+(Y)$ and $\mathfrak{J}\mapsto V_+(\mathfrak{J})$;
the union $Y_1\cup Y_2$ of two closed subsets of $X$ corresponds
\oldpage[II]{28}
to $\mathfrak{j}_+(Y_1)\cap\mathfrak{j}_+(Y_2)$, and the intersection of an arbitrary family $(Y_\lambda)$ of closed subsets corresponds to the radical in $S_+$ of the sum of the $\mathfrak{j}_+(Y_\lambda)$.
\end{corollary}

\begin{corollary}[2.3.13]
\label{II.2.3.13}
Let $\mathfrak{J}$ be a graded ideal of $S_+$;
to have $V_+(\mathfrak{J})=\emp$, it is necessary and sufficient for every element $f$ of $S_+$ to have a power $f^n$ in $\mathfrak{J}$.
\end{corollary}

This above corollary can also be expressed in one of the following equivalent forms:

\begin{corollary}[2.3.14]
Let $(f_\alpha)$ be a family of homogeneous elements of $S_+$.
For the $D_+(f_\alpha)$ to form a cover of $X=\Proj(S)$, it is necessary and sufficient for every element of $S_+$ to have a power in the ideal generated by the $f_\alpha$.
\label{II.2.3.14}
\end{corollary}

\begin{corollary}[2.3.15]
\label{II.2.3.15}
Let $(f_\alpha)$ be a family of homogeneous elements of $S_+$, and $f$ an element of $S_+$.
Then the following are equivalent:
\begin{enumerate}
  \item[{\rm(a)}] $D_+(f)\subset\bigcup_\alpha D_+(f_\alpha)$;
  \item[{\rm(b)}] $V_+(f)\supset\bigcap_\alpha V_+(f_\alpha)$;
  \item[{\rm(c)}] $f$ has a power in the ideal generated by the $f_\alpha$.
\end{enumerate}
\end{corollary}

\begin{corollary}[2.3.16]
\label{II.2.3.16}
For $X=\Proj(S)$ to be empty, it is necessary and sufficient for every element of $S_+$ to be nilpotent.
\end{corollary}

\begin{corollary}[2.3.17]
\label{II.2.3.17}
In the bijective correspondence described in \sref{II.2.3.12}, the \emph{irreducible} closed subsets of $X$ correspond to the graded \emph{prime} ideals of $S_+$.
\end{corollary}

\begin{proof}
If $Y=Y_1\cup Y_2$, where $Y_1$ and $Y_2$ are distinct closed subsets of $Y$, then
\[
  \mathfrak{j}_+(Y) = \mathfrak{j}_+(Y_1)\cap\mathfrak{j}_+(Y_2)
\]
with the ideals $\mathfrak{j}_+(Y_1)$ and $\mathfrak{j}_+(Y_2)$ being distinct from $\mathfrak{j}_+(Y)$, and so $\mathfrak{j}_+(Y)$ is not prime.
Conversely, if $\mathfrak{J}$ is a graded ideal of $S_+$ that is not prime, then there exists elements $f,g\in S_+$ such that $f\not\in\mathfrak{J}$ and $g\not\in\mathfrak{J}$, but $fg\in\mathfrak{J}$;
then $V_+(f)\supset V_+(\mathfrak{J})$ and $V_+(g)\supset V_+(\mathfrak{J})$, but $V_+(\mathfrak{J})\subset V_+(f)\cup V_+(g)$, by \sref{II.2.3.2.3};
we thus conclude that $V_+(\mathfrak{J})$ is the union of the closed subsets $V_+(f)\cap V_+(\mathfrak{J})$ and $V_+(g)\cap V_+(\mathfrak{J})$, which are distinct from $V_+(\mathfrak{J})$.
\end{proof}


\subsection{The scheme structure on $\mathrm{Proj}(S)$}
\label{subsection:II.2.4}

\begin{env}[2.4.1]
\label{II.2.4.1}
Let $f$ and $g$ be homogeneous elements of $S_+$;
consider the affine schemes $Y_f=\Spec(S_{(f)})$, $Y_g=\Spec(S_{(g)})$, and $Y_{fg}=\Spec(S_{(fg)})$.
By \sref{II.2.2.2}, the morphism $w_{fg,f} = ({}^a\!\omega_{fg,f},\widetilde{\omega}_{fg,f})$ from $Y_{fg}$ to $Y_f$, corresponding to the canonical homomorphism $\omega_{fg,f}: S_{(f)}\to S_{(fg)}$, is an \emph{open immersion} \sref[I]{I.1.3.6}.
Using the inverse homeomorphism of $\psi_f: D_+(f)\to Y_f$ \sref{II.2.3.6}, we can transport the affine scheme structure of $Y_f$ to $D_+(f)$;
by the commutativity of diagram~\sref{II.2.3.5.1}, the affine scheme $D_+(fg)$ can thus be identified with the induced scheme on the open subset $D_+(fg)$ of the underlying space of the affine scheme $D_+(f)$.
It is then clear (taking \sref{II.2.3.4} into account) that $X=\Proj(S)$ is endowed with a unique \emph{prescheme} structure, whose restriction to each $D_+(f)$ is the affine scheme that we have just defined.
Furthermore:
\end{env}

\begin{proposition}[2.4.2]
\label{II.2.4.2}
The prescheme $\Proj(S)$ is a scheme.
\end{proposition}

\begin{proof}
It suffices \sref[I]{I.5.5.6} to show, for any homogeneous $f$ and $g$ in $S_+$, that $D_+(f)\cap D_+(g)=D_+(fg)$ is affine, and that its ring is generated by the canonical images of the rings of $D_+(f)$ and $D_+(g)$;
the first point is evident by definition, and the second follows from \sref{II.2.2.4}.
\end{proof}

\oldpage[II]{29}
Whenever we speak of the homogeneous prime spectrum $\Proj(S)$ as a \emph{scheme}, it will always mean with respect to the structure that we have just defined.

\begin{example}[2.4.3]
\label{II.2.4.3}
Let $S=K[T_1,T_2]$, where $K$ is a field, $T_1$ and $T_2$ are indeterminates, and $S$ is graded by total degree.
It follows from \sref{II.2.3.14} that $\Proj(S)$ is the union of $D_+(T_1)$ and $D_+(T_2)$;
we immediately see that these affine schemes are isomorphic to $K[T]$, and that $\Proj(S)$ is obtained by the gluing of these two affine schemes as described in \sref[I]{I.2.3.2} (cf. \sref{II.7.4.14}).
\end{example}

\begin{proposition}[2.4.4]
\label{II.2.4.4}
Let $S$ be a positively graded ring, and let $X$ be the scheme $\Proj(S)$.
\begin{enumerate}
  \item[{\rm(i)}] If $\mathfrak{N}_+$ is the nilradical of $S_+$ \sref{II.2.1.10}, then the scheme $X_\red$ is canonically isomorphic to $\Proj(S/\mathfrak{N}_+)$;
    in particular, if $S$ is essentially reduced, then $\Proj(S)$ is reduced.
  \item[{\rm(ii)}] If $S$ is essentially reduced, then, for $X$ to be integral, it is necessary and sufficient for $S$ to be essentially integral.
\end{enumerate}
\end{proposition}

\begin{proof}
\medskip\noindent
\begin{enumerate}
  \item[{\rm(i)}] Let $\overline{S}$ be the graded ring $S/\mathfrak{N}_+$, and denote by $x\mapsto\overline{x}$ the canonical homomorphism $S\to\overline{S}$ of degree~$0$.
    For all $f\in S_d$ ($d>0$), the canonical homomorphism $S_f\to\overline{S}$ \sref[0]{0.1.5.1} is surjective and of degree~$0$, and thus gives, by restriction, a surjective homomorphism $S_{(f)}\to\overline{S}_{(\overline{f})}$;
    if we suppose that $f\not\in\mathfrak{N}_+$, then we immediately see that $\overline{S}_{(\overline{f})}$  is reduced, and that the kernel of the above homomorphism is the nilradical of $S_{(f)}$, or, in other words, that $\overline{S}_{(\overline{f})}=(S_{(f)})_\red$.
    So to this homomorphism corresponds a closed immersion $D_+(\overline{f})\to D_+(f)$ that identifies $D_+(\overline{f})$ with $(D_+(f))_\red$ \sref[I]{I.5.1.2}, and which is, in particular, a homeomorphism of the underlying spaces of these two affine schemes.
    Furthermore, if $g\not\in\mathfrak{N}_+$ is another homogeneous element of $S_+$, then the diagram
    \[
      \xymatrix{
        S_{(f)} \ar[r] \ar[d]
        & \overline{S}_{(\overline{f})} \ar[d]
      \\S_{(fg)} \ar[r]
        & \overline{S}_{(\overline{fg})}
      }
    \]
    is commutative;
    since, further, the $D_+(f)$, for $f$ homogeneous, of degree $>0$, and $f\not\in\mathfrak{N}_+$, form a cover of $X=\Proj(S)$ \sref{II.2.3.7}, we see that the morphisms $D_+(\overline{f})\to D_+(f)$ are the restrictions of a closed immersion $\Proj(\overline{S})\to\Proj(S)$ which is a homeomorphism of the underlying spaces;
    whence the conclusion \sref[I]{I.5.1.2}.
  \item[{\rm(ii)}] Suppose that $S$ is essentially integral, or, in other words, that $(0)$ is a graded prime ideal of $S_+$ that is distinct from $S_+$;
    then $X$ is reduced, by (i), and irreducible, by \sref{II.2.3.17}.
    Conversely, suppose that $S$ is essentially reduced and that $X$ is integral;
    then, for homogeneous $f\neq0$ in $S_+$, we have that $D_+(f)\neq\emp$ \sref{II.2.3.7};
    the hypothesis that $X$ is irreducible implies that $D_+(f)\cap D_+(g)\neq\emp$ for homogeneous $f,g\neq0$ in $S_+$;
    thus $fg\neq0$, by \sref{II.2.3.3.2}, and we thus conclude that $S_+$ has no zero divisors, whence the first claim.
\end{enumerate}
\end{proof}

\begin{env}[2.4.5]
\label{II.2.4.5}
Given a commutative ring $A$, recall that we say that a graded ring $S$ is a \emph{graded $A$-algebra} if it is endowed with the structure of an $A$-algebra such that each of its subgroups $S_n$ is an $A$-module;
for this, it suffices for $S_0$ to be
\oldpage[II]{30}
an $A$-algebra, or, in other words, we define the structure of a graded $A$-algebra on $S$ by defining the structure of an $A$-algebra on $S_0$ and setting $\alpha\cdot x=(\alpha\cdot1)x$ for $\alpha\in A$ and $x\in S_n$.
\end{env}

\begin{proposition}[2.4.6]
\label{II.2.4.6}
Suppose that $S$ is a graded $A$-algebra.
Then, on $X=\Proj(S)$, the structure sheaf $\sh{O}_X$ is an $A$-algebra (where $A$ is considered as a simple sheaf on $X$);
in other words, $X$ is a scheme over $\Spec(A)$.
\end{proposition}

\begin{proof}
It suffices to note that, for every homogeneous $f$ in $S_+$, $S_{(f)}$ is an algebra over $A$, and that the diagram
\[
  \xymatrix{
    S_{(f)} \ar[rr] && S_{(fg)}
  \\ & A \ar[ul] \ar[ur] &
  }
\]
is commutative, for homogeneous $f,g$ in $S_+$.
\end{proof}

\begin{proposition}[2.4.7]
\label{II.2.4.7}
Let $S$ be a positively graded ring.
\begin{enumerate}
  \item[{\rm(i)}] For every integer $d>0$, there exists a canonical isomorphism from the scheme $\Proj(S)$ to the scheme $\Proj(S^{(d)})$.
  \item[{\rm(ii)}] Let $S'$ be the graded ring such that $S_0=\bb{Z}$ and $S'_n=S_n$ (considered as a $\bb{Z}$-module) for $n>0$.
    Then there exists a canonical isomorphism from the scheme $\Proj(S)$ to the scheme $\Proj(S')$.
\end{enumerate}
\end{proposition}

\begin{proof}
\medskip\noindent
\begin{enumerate}
  \item[{\rm(i)}] We first show that the map $\mathfrak{p}\mapsto\mathfrak{p}\cap S^{(d)}$ is a bijection from the set $\Proj(S)$ to the set $\Proj(S^{(d)})$.
    Indeed, suppose that we have a graded prime ideal $\mathfrak{p}'\in\Proj(S^{(d)})$, and let $\mathfrak{p}_{nd}=\mathfrak{p}'\cap S_{nd}$ ($n\geq0$).
    For all $n>0$ that are not multiples of $d$, define $\mathfrak{p}_n$ as the set of $x\in S_n$ such that $x^d\in\mathfrak{p}_{nd}$;
    if $x,y\in\mathfrak{p}_n$, then $(x+y)^{2d}\in\mathfrak{p}_{2nd}$, and so $(x+y)^d\in\mathfrak{p}_{nd}$, since $\mathfrak{p}'$ is prime;
    it is immediate that the $\mathfrak{p}_n$ thus defined, for $n\geq0$, satisfy the conditions of \sref{II.2.1.9}, and so there exists a unique prime ideal $\mathfrak{p}\in\Proj(S)$ such that $\mathfrak{p}\cap S^{(d)}=\mathfrak{p}'$.
    Since, for every homogeneous $f$ in $S_+$, we have that $V_+(f)=V_+(f^d)$ \sref{II.2.3.2.3}, we see that the above bijection is a \emph{homeomorphism} of topological spaces.
    Finally, with the same notation, $S_{(f)}$ and $S_{(f^d)}$ are canonically identified \sref{II.2.2.2}, and so $\Proj(S)$ and $\Proj(S^{(d)})$ are canonically identified as \emph{schemes}.
  \item[{\rm(ii)}] If, to each $\mathfrak{p}\in\Proj(S)$, we associate the unique prime ideal $\mathfrak{p}'\in\Proj(S')$ such that $\mathfrak{p}'\cap S_n=\mathfrak{p}\cap S_n$ for every $n>0$ \sref{II.2.1.9}, then it is clear that we have defined a canonical homeomorphism $\Proj(S)\simto\Proj(S')$ of the underlying spaces, since $V_+(f)$ is the same set for $S$ and $S'$ when $f$ is a homogeneous element of $S_+$.
  Since, further, $S_{(f)}=S'_{(f)}$, $\Proj(S)$ and $\Proj(S')$ can be identified as \emph{schemes}.
\end{enumerate}
\end{proof}

\begin{corollary}[2.4.8]
\label{II.2.4.8}
If $S$ is a graded $A$-algebra, and $S'_A$ the graded $A$-algebra such that $(S'_A)_0=A$ and $(S'_A)_n=S_n$ for $n>0$, then there exists a canonical isomorphism from $\Proj(S)$ to $\Proj(S'_A)$.
\end{corollary}

\begin{proof}
In fact, these two schemes are both canonically isomorphic to $\Proj(S')$, using the notation of \sref{II.2.4.7}[(ii)].
\end{proof}


\subsection{The sheaf associated to a graded module}
\label{subsection:II.2.5}

\begin{env}[2.5.1]
\label{II.2.5.1}
Let $M$ be a \emph{graded} $S$-module.
The, for every homogeneous $f$ in $S_+$, $M_{(f)}$ is an $S_{(f)}$-module, and thus has a corresponding quasi-coherent associated sheaf $(M_{(f)})\supertilde$ on the affine scheme $\Spec(S_{(f)})$, identified with $D_+(f)$ \sref[I]{I.1.3.4}.
\end{env}

\oldpage[II]{31}
\begin{proposition}[2.5.2]
\label{II.2.5.2}
There exists on $X=\Proj(S)$ exactly one quasi-coherent $\sh{O}_X$-module $\widetilde{M}$ such that, for $f$ homogeneous in $S_+$, we have $\Gamma(D_+(f),\widetilde{M})=M_{(f)}$, with the restriction homomorphism $\Gamma(D_+(f),\widetilde{M})\to\Gamma(D_+(fg),\widetilde{M})$, for $f$ and $g$ homogeneous in $S_+$, being the canonical homomorphism $M_{(f)}\to M_{(fg)}$ \sref{II.2.2.3}.
\end{proposition}

\begin{proof}
Suppose that $f\in S_d$ and $g\in S_e$.
Since $D_+(fg)$ can be identified with the prime spectrum of $(S_{(f)})_{g^d/f^e}$ by \sref{II.2.2.2}, the restriction to $D_+(fg)$ of the sheaf $(M_{(f)})\supertilde$ on $D_+(f)$ is canonically identified with the sheaf associated to the module $(M_{(f)})_{g^d/f^e}$ \sref[I]{I.1.3.6}, and thus also with $(M_{(fg)})\supertilde$ \sref{II.2.2.2};
we thus conclude that there exists a canonical isomorphism
\[
  \theta_{g,f}: (M_{(f)})\supertilde|D_+(fg) \simto (M_{(g)})\supertilde|D_+(fg)
\]
such that, if $h$ is a third homogeneous element of $S_+$, then $\theta_{f,h}=\theta_{f,g}\circ\theta_{g,h}$ in $D_+(fgh)$.
Consequently \sref[0]{0.3.3.1} there exists a quasi-coherent $\sh{O}_X$-module $\sh{F}$ on $X$, and, for every homogeneous $f$ in $S_+$, an isomorphism $\eta_f$ from $\sh{F}|D_+(f)$ to $(M_{f})\supertilde$ such that $\theta_{g,f}=\eta_g\circ\eta_f^{-1}$.
If we then consider the sheaf $\sh{G}$ associated to the presheaf (on the base of the topology of $X$ given by the $D_+(f)$) defined by $D_+(f)\mapsto M_{(f)}$, with the canonical homomorphisms $M_{(f)}\to M_{(fg)}$ as restriction homomorphisms, then the above proves that $\sh{F}$ and $\sh{G}$ are isomorphic (taking \sref[I]{I.1.3.7} into account);
the sheaf $\sh{G}$ is denoted by $\widetilde{M}$, and indeed satisfies the conditions of the statement.
We have, in particular, $\widetilde{S}=\sh{O}_X$.
\end{proof}

\begin{definition}[2.5.3]
\label{II.2.5.3}
We say that the quasi-coherent $\sh{O}_X$-module $\widetilde{M}$ defined in \sref{II.2.5.2} is \emph{associated} to the graded $S$-module $M$.
\end{definition}

Recall that the graded $S$-modules form a category when we restrict from arbitrary homomorphisms of graded modules to homomorphisms \emph{of degree~$0$}.
With this convention:

\begin{proposition}[2.5.4]
\label{II.2.5.4}
The functor $M\mapsto\widetilde{M}$ is an exact additive covariant functor from the category of graded $S$-modules to the category of quasi-coherent $\sh{O}_X$-modules, and it commutes with inductive limits and direct sums.
\end{proposition}

\begin{proof}
Indeed, since these properties are local, it suffices to show that they are satisfied for the sheaves of the form $\widetilde{M}|D_+(f)=(M_{(f)})\supertilde$;
but the functors $M\mapsto M_f$, $N\mapsto N_0$ (to the category of graded $S_f$-modules), and $P\mapsto\widetilde{P}$ (to the category of $S_{(f)}$-modules) all have the three properties of exactness and of commutativity with inductive limits and direct sums (\sref[I]{I.1.3.5} and \sref[I]{I.1.3.9});
whence the proposition.
\end{proof}

We denote by $\widetilde{u}$ the homomorphism $\widetilde{M}\to\widetilde{N}$ corresponding to a homomorphism $u: M\to N$ of degree~$0$.
We immediately deduce from \sref{I.2.5.4} that the results of \sref[I]{I.1.3.9} and \sref[I]{I.1.3.10} still hold for graded $S$-modules and homomorphisms of degree~$0$ (with the sense given here to $\widetilde{M}$), with the proofs being purely formal.

\begin{proposition}[2.5.5]
\label{II.2.5.5}
For all $\mathfrak{p}\in X=\Proj(S)$, we have $\widetilde{M}_\mathfrak{p}=M_{(\mathfrak{p})}$.
\end{proposition}

\begin{proof}
By definition, $\widetilde{M}_\mathfrak{p}=\varinjlim\Gamma(D_+(f),\widetilde{M})$, where $f$ runs over the set of homogeneous elements of $S_+$ such that $f\not\in\mathfrak{p}$;
since $\Gamma(D_+(f),\widetilde{M})=M_{(f)}$, the proposition follows from the definition of $M_{(\mathfrak{p})}$ \sref{II.2.2.7}
\end{proof}

\oldpage[II]{32}
In particular, the \emph{local ring $(\sh{O}_X)_\mathfrak{p}$} is exactly the ring $S_{(\mathfrak{p})}$, the set of fractions $x/f$ with $f$ homogeneous in $S_+$ and not belonging to $\mathfrak{p}$, and with $x$ homogeneous of the same degree as $f$.

Even more particularly, if $S$ is \emph{essential integral}, so that $\Proj(S)=X$ is \emph{integral} \sref{II.2.4.4}, and if $\xi=(0)$ is the generic point of $X$, then the \emph{field of rational functions $R(X)=\sh{O}_\xi$} is field consisting of elements $fg^{-1}$ with $f$ and $g$ homogeneous of the same degree in $S_+$, and with $g\neq 0$.

\begin{proposition}[2.5.6]
\label{II.2.5.6}
If, for all $z\in M$ and all homogeneous $f$ in $S_+$, there exists a power of $f$ that annihilates $z$, then $\widetilde{M}=0$.
This sufficient condition is also necessary if the $S_0$-algebra $S$ is generated by the set $S_1$ of homogeneous elements of degree~$1$.
\end{proposition}

\begin{proof}
The condition $\widetilde{M}=0$ is equivalent to $M_{(f)}=0$ for all homogeneous $f$ in $S_+$.
On the other hand, if $f\in S_d$, to say that $M_{(f)}=0$ implies that, for all homogeneous $z\in M$ whose degree is some multiple of $d$, there exists a power $f^n$ such that $f^nz=0$.
If $M_{(f)}=0$ for all $f\in S_1$, then the condition of the statement is thus satisfied for all $f\in S_1$;
the condition is \emph{a fortiori} satisfied for all homogeneous $f$ in $S_+$ if $S_1$ generates $S$, since every homogeneous element of $S_+$ is then a linear combination of products of elements of $S_1$.
\end{proof}

\begin{proposition}[2.5.7]
\label{II.2.5.7}
Let $d>0$ be an integer, and let $f\in S_d$.
Then, for all $n\in\bb{Z}$, the $(\sh{O}_X|D_+(f))$-module $(S(nd))\supertilde|D_+(f)$ is canonically isomorphic to $\sh{O}_X|D_+(f)$.
\end{proposition}

\begin{proof}
Indeed, multiplication by the invertible element $(f/1)^n$ of $S_f$ gives a bijection from $S_{(f)}=(S_f)_0$ to $(S_f)_{nd}=(S_f(nd))_0=(S(nd)_f)_0=S(nd)_{(f)}$;
in other words, the $S_{(f)}$-modules $S_{(f)}$ and $S(nd)_{(f)}$ are canonically isomorphic, whence the proposition.
\end{proof}

\begin{corollary}[2.5.8]
\label{II.2.5.8}
On the open subset $U=\bigcup_{f\in S_d}D_+(f)$, the restriction of the $\sh{O}_X$-module $(S(nd))\supertilde$ is an invertible $(\sh{O}_X|U)$-module \sref[0]{0.5.4.1}.
\end{corollary}

\begin{corollary}[2.5.9]
\label{II.2.5.9}
If the ideal $S_+$ of $S$ is generated by the set $S_1$ of homogeneous elements of degree~$1$, then the $\sh{O}_X$-module $(S(n))\supertilde$ is invertible for all $n\in\bb{Z}$.
\end{corollary}

\begin{proof}
It suffices to remark that $X=\bigcup_{f\in S_1}D_+(f)$, by the hypothesis \sref{II.2.3.14} and to apply \sref{II.2.5.8} with $U=X$.
\end{proof}

\begin{env}[2.5.10]
\label{II.2.5.10}
We set, for the rest of this section,
\[
\label{II.2.5.10.1}
  \sh{O}_X(n) = (S(n))\supertilde
\tag{2.5.10.1}
\]
for all $n\in\bb{Z}$, and, for every open subset $U$ of $X$, and every $(\sh{O}_X|U)$-module $\sh{F}$,
\[
\label{II.2.5.10.2}
  \sh{F}(n) = \sh{F}\otimes_{\sh{O}_X|U}(\sh{O}_X(n)|U)
\tag{2.5.10.2}
\]
for all $n\in\bb{Z}$.
If the ideal $S_+$ is generated by $S_1$, then the functor $\sh{F}(n)$ is \emph{exact} in $\sh{F}$ for all $n\in\bb{Z}$, since $\sh{O}_X(n)$ is then an \emph{invertible} $\sh{O}_X$-module.
\end{env}

\begin{env}[2.5.11]
\label{II.2.5.11}
Let $M$ and $N$ be graded $S$-modules.
For all $f\in S_d$ ($d>0$), we define a canonical functorial homomorphism of $S_{(f)}$-modules by
\[
\label{II.2.5.11.1}
  \lambda_f: M_{(f)}\otimes_{S_{(f)}}N_{(f)} \to (M\otimes_S N)_{(f)}
\tag{2.5.11.1}
\]
\oldpage[II]{33}
by composing the homomorphism $M_{(f)}\otimes_{S_{(f)}}N_{(f)}\to M_f\otimes_{S_f}N_f$ (coming from the injections $M_{(f)}\to M_f$, $N_{(f)}\to N_f$, and $S_{(f)}\to S_f$) with the canonical isomorphism $M_f\otimes_{S_f}N_f\simto(M\otimes_S N)_f$ \sref[0]{0.1.3.4}, and by noting that, by the definition of the tensor product of two graded modules, this latter isomorphism preserves degrees;
for $x\in M_{md}$ and $y\in N_{nd}$ ($m,n\geq0$), we thus have
\[
  \lambda_f((x/f^m)\otimes(y/f^n)) = (x\otimes y)/f^{m+n}.
\]

It immediately follows from this definition that, if $g\in S_e$ ($e>0$), then the diagram
\[
  \xymatrix{
    M_{(f)}\otimes_{S_{(f)}}N_{(f)} \ar[r]^{\lambda_f} \ar[d]
    & (M\otimes_S N)_{(f)} \ar[d]
  \\M_{(fg)}\otimes_{S_{(fg)}}N_{(fg)} \ar[r]_{\lambda_{fg}}
    & (M\otimes_S N)_{(fg)}
  }
\]
(where the vertical arrow on the right is the canonical homomorphism, and the one on the left comes from the canonical homomorphisms) commutes.
Thus $\lambda$ induces a canonical functorial homomorphism of $\sh{O}_X$-modules
\[
\label{II.2.5.11.2}
  \lambda: \widetilde{M}\otimes_{\sh{O}_X}\widetilde{N} \to (M\otimes_S N)\supertilde.
\tag{2.5.11.2}
\]

Consider, in particular, graded ideals $\mathfrak{J}$ and $\mathfrak{K}$ of $S$;
since $\widetilde{\mathfrak{J}}$ and $\widetilde{\mathfrak{K}}$ are sheaves of ideals of $\sh{O}_X$, we have a canonical homomorphism $\widetilde{\mathfrak{J}}\otimes_{\sh{O}_X}\widetilde{\mathfrak{K}}\to\sh{O}_X$;
the diagram
\[
\label{II.2.5.11.3}
  \xymatrix{
    \widetilde{\mathfrak{J}}\otimes_{\sh{O}_X}\widetilde{\mathfrak{K}} \ar[rr]^\lambda \ar[dr]
    && (\widetilde{\mathfrak{J}}\otimes_S\widetilde{\mathfrak{K}})\supertilde \ar[dl]
  \\&\sh{O}_X&
  }
\tag{2.5.11.3}
\]
then commutes.
Indeed, we can reduce to verifying this on each open subset $D_+(f)$ (for $f$ homogeneous in $S_+$), and this follows immediately from the definition \sref{2.5.11.1} of $\lambda_f$ and from \sref[I]{I.1.3.13}.

Finally, note that, if $M$, $N$, and $P$ are graded $S$-modules, then the diagram
\[
\label{II.2.5.11.4}
  \xymatrix{
    \widetilde{M}\otimes_{\sh{O}_X}\widetilde{N}\otimes_{\sh{O}_X}\widetilde{P} \ar[r]^{\lambda\otimes1} \ar[d]_{1\otimes\lambda}
    & (M\otimes_S N)\supertilde\otimes_{\sh{O}_X}\widetilde{P} \ar[d]^\lambda
  \\\widetilde{M}\otimes_{\sh{O}_X}(N\otimes_S P)\supertilde \ar[r]_\lambda
    & (M\otimes_S N\otimes_S P)\supertilde
  }
\tag{2.5.11.4}
\]
\oldpage[II]{34}
commutes.
It again suffices to verify this on each open subset $D_+(f)$, and this follows immediately from the definitions and from \sref[I]{I.1.3.13}.
\end{env}

\begin{env}[2.5.12]
\label{II.2.5.12}
Under the hypotheses of \sref{II.2.5.11}, we define a functorial canonical homomorphism of $S_{(f)}$-modules
\[
\label{II.2.5.12.1}
  \mu_f: (\Hom_S(M,N))_{(f)} \to \Hom_{S_{(f)}}(M_{(f)},N_{(f)})
\tag{2.5.12.1}
\]
by sending $u/f^n$, where $u$ is a homomorphism of degree~$nd$, to the homomorphism $M_{(f)}\to N_{(f)}$ that sends $x/f^m$ ($x\in M_{md}$) to $u(x)/f^{m+n}$.
For $g\in S_e$ ($e>0$), we again have a commutative diagram:
\[
  \xymatrix{
    (\Hom_S(M,N))_{(f)} \ar[r]^{\mu_f} \ar[d]
    & \Hom_{S_{(f)}}(M_{(f)},N_{(f)}) \ar[d]
  \\(\Hom_S(M,N))_{(fg)} \ar[r]_{\mu_{fg}}
    & \Hom_{S_{(fg)}}(M_{(fg)},N_{(fg)})
  }
\]
(where the vertical arrow on the left is the canonical homomorphism, and the one on the right comes from the canonical homomorphisms).
We thus again conclude (taking \sref[I]{I.1.3.8} into account) that the $\mu_f$ define a functorial canonical homomorphism of $\sh{O}_X$-modules
\[
\label{II.2.5.12.2}
  \mu: (\Hom_S(M,N))\supertilde \to \shHom_{\sh{O}_X}(\widetilde{M},\widetilde{N})
\tag{2.5.12.2}
\]
\end{env}

\begin{proposition}[2.5.13]
\label{II.2.5.13}
Suppose that the ideal $S_+$ is generated by $S_1$.
Then the homomorphism $\lambda$ \sref{II.2.5.11.2} is an isomorphism;
so too is the homomorphism $\mu$ \sref{II.2.5.12.2} if the graded $S$-module $M$ admits a finite presentation \sref{II.2.1.1}
\end{proposition}

\begin{proof}
Since $X$ is the union of the $D_+(f)$ for $f\in S_1$ \sref{II.2.3.14}, we are led to proving that $\lambda_f$ and $\mu_f$ are isomorphisms, under the given hypotheses, whenever $f$ is homogeneous and \emph{of degree~$1$}.
But we can then define a $\bb{Z}$-bilinear map $M_m\times N_n\to M_{(f)}\otimes_{S_{(f)}}N_{(f)}$ by sending $(x,y)$ to the element $(x/f^m)\otimes(y/f^n)$ (if $m<0$, we write $x/f^m$ to mean $f^{-m}x/1$);
these maps define a $\bb{Z}$-linear map $M\otimes_{\bb{Z}}N\to M_{(f)}\otimes_{S_{(f)}}N_{(f)}$, and, if $s\in S_q$, this map sends $(sx)\otimes y$ to $(s/f^q)((x/f^m)\otimes(y/f^n))$ (for $x\in M_m$ and $y\in N_n$).
We thus obtain a di-homomorphism of modules $\gamma_f: M\otimes_S N\to M_{(f)}\otimes_{S_{(f)}}N_{(f)}$, with respect to the canonical homomorphism $S\to S_{(f)}$ (sending $s\in S_q$ to $s/f^q$).
Suppose furthermore that, for an element $\sum_i(x_i\otimes y_i)$ of $M\otimes_S N$ (with $x_i$ and $y_i$ homogeneous of degree $m_i$ and $n_i$, respectively), we have that $f^r\sum_i(x_i\otimes y_i)=0$, or, in other words, that $\sum_i(f^rx_i\otimes y_i)=0$.
We thus deduce, by \sref[0]{0.1.3.4}, that $\sum_i(f^rx_i/f^{m_i+r})\otimes(y_i/f^{n_i})=0$, i.e. $\gamma_f(\sum_i(x_i\otimes y_i))=0$.
Then $\gamma_f$ factors as $M\otimes_S N\to(M\otimes_S N)_f\xrightarrow{\gamma'_f}M_{(f)}\otimes_{S_{(f)}}N_{(f)}$;
if $\lambda'_f$ is the restriction of $\gamma'$
\oldpage[II]{35}
to $(M\otimes_S N)_{(f)}$, then we can immediately show that $\lambda_f$ and $\lambda'_f$ are inverse $S_{(f)}$-homomorphisms, whence the first part of the proposition.

To prove the second part, suppose that $M$ is the cokernel of a homomorphism $P\to Q$ of graded $S$-modules, with $P$ and $Q$ being direct sums of a finite number of modules of the form $S(n)$;
using the left-exactness of $\Hom_S(L,N)$ in $L$, and the exactness of $M_{(f)}$ in $M$, we can immediately reduce to proving that $\mu_f$ is an isomorphism whenever $M=S(n)$.
But, for any homogeneous $z$ in $N$, let $u_z$ be the homomorphism from $S(n)$ to $N$ such that $u_z(1)=z$;
we immediately see that $\eta: z\to u_z$ is an isomorphism of degree~$0$ from $N(-n)$ to $\Hom_S(S(n),N)$.
There is a corresponding isomorphism
\[
  \eta_f: (N(-n))_{(f)} \to (\Hom_S(S(n),N))_{(f)}.
\]

Now let $\eta'_f$ be the isomorphism $N_{(f)}\to\Hom_{S_{(f)}}(S(n)_{(f)},N_{(f)})$ that, to any $z'\in N_{(f)}$, associates the homomorphism $v_{z'}$ that is such that $v_{z'}(s/f^k)=sz'/f^{n+k}$ (for $s\in S_{n+k}=(S(n))_k$).
We easily note that the composed map
\[
  (N(-n))_{(f)}
  \xrightarrow{\eta_f} (\Hom_S(S(n),N))_{(f)}
  \xrightarrow{\mu_f} \Hom_{S_{(f)}}(S(n)_{(f)},N_{(f)})
  \xrightarrow{{\eta'_f}^{-1}} N_{(f)}
\]
is the isomorphism $z/f^h\mapsto z/f^{h-n}$ from $(N(-n))_{(f)}$ to $N_{(f)}$, and thus $\mu_f$ is an isomorphism.
\end{proof}

If the ideal $S_+$ is generated by $S_1$, then we deduce from \sref{II.2.5.13} that, for every graded ideal $\mathfrak{J}$ of $S$, and for every graded $S$-module $M$, we have
\[
\label{II.2.5.13.1}
  \widetilde{\mathfrak{J}}\cdot\widetilde{M} = (\mathfrak{J}\cdot M)\supertilde
\tag{2.5.13.1}
\]
up to canonical isomorphism;
this follows from the commutativity of the diagram
\[
  \xymatrix{
    \widetilde{\mathfrak{J}}\otimes_{\sh{O}_X}\widetilde{M} \ar[rr]^\lambda \ar[dr]
    && (\mathfrak{J}\otimes_S M)\supertilde \ar[dl]
  \\&\widetilde{M}
  }
\]
which we can verify as we did for \sref{II.2.5.11.3}.

\begin{corollary}[2.5.14]
\label{II.2.5.14}
Suppose that $S$ is generated by $S_1$.
For any $m,n\in\bb{Z}$, we then have:
\[
\label{II.2.5.14.1}
  \sh{O}_X(m)\otimes_{\sh{O}_X}\sh{O}_X(n) = \sh{O}_X(m+n)
\tag{2.5.14.1}
\]
\[
\label{II.2.5.14.2}
  \sh{O}_X(n) = (\sh{O}_X(1))^{\otimes n}
\tag{2.5.14.2}
\]
up to canonical isomorphism.
\end{corollary}

\begin{proof}
The first equation follows from \sref{II.2.5.13} and from the existence of the canonical isomorphism $S(m)\otimes_S S(n)\simto S(m+n)$ of degree~$0$ that sends the element $1\otimes1$ (where the first $1$ is in $(S(m))_{-m}$ and the second is in $(S(n))_{-n}$) to the element $1\in(S(m+n))_{-(m+n)}$.
It then suffices to prove the second equation for $n=-1$, and, by \sref{II.2.5.13}, this reduces to seeing that $\Hom_S(S(1),S)$ is canonically isomorphic to $S(-1)$, which can be immediately proven by going back to the definitions \sref{II.2.1.2} and by remembering that $S(1)$ is a monogeneous $S$-module.
\end{proof}

\oldpage[II]{36}
\begin{corollary}[2.5.15]
\label{II.2.5.15}
Suppose that $S$ is generated by $S_1$.
Then, for every graded $S$-module $M$, and for every $n\in\bb{Z}$, we have
\[
\label{II.2.5.15.1}
  (M(n))\supertilde = \widetilde{M}(n)
\tag{2.5.15.1}
\]
up to canonical isomorphism.
\end{corollary}

\begin{proof}
This follows from definitions \sref{II.2.5.10.2} and \sref{II.2.5.10.1}, from Proposition~\sref{II.2.5.13}, and from the existence of a canonical isomorphism $M(n)\simto M\otimes_S S(n)$ of degree~$0$ that, to every $z\in(M(n))_h=M_{n+h}$, associates $z\otimes1\in M_{n+h}\otimes(S(n))_{-n}\subset(M\otimes_S S(n))_h$.
\end{proof}

\begin{env}[2.5.16]
\label{II.2.5.16}
We denote by $S'$ the graded ring such that $S'_0=\bb{Z}$, and $S'_n=S_n$ for $n>0$.
Then, if $f\in S_d$ ($d>0$), we have that $(S(n))_{(f)}=(S'(n))_{(f)}$ for all $n\in\bb{Z}$, since an element of $(S'(n))_{(f)}$ is of the form $x/f^k$, with $x\in S'_{n+kd}$ ($k>0$), and we can always take $k$ to be such that $n+kd\neq0$.
Since $X=\Proj(S)$ and $X'=\Proj(S')$ are canonically identified \sref{II.2.4.7}[(ii)], we see that, for all $n\in\bb{Z}$, $\sh{O}_X(n)$ and $\sh{O}_{X'}(n)$ are the images of one another under the above identification.

Note also that, for all $d>0$ and all $n\in\bb{Z}$, we have
\[
  (S^{(d)}(n))_h = S_{(n+h)d} = (S(nd))_{hd}
\]
for $f\in S_d$, and thus $(S^{(d)}(n))_{(f)}=(S(nd))_{(f)}$.
We know that the schemes $X=\Proj(S)$ and $X^{(d)}=\Proj(S^{(d)})$ are canonically identified \sref{II.2.4.7}[(ii)];
the above shows that, if the $S_0$-algebra $S^{(d)}$ is generated by $S_d$, then $\sh{O}_X(nd)$ and $\sh{O}_{X^{(d)}}(n)$ are the images of one another under this identification, for all $n\in\bb{Z}$.
\end{env}

\begin{proposition}[2.5.17]
\label{II.2.5.17}
Let $d>0$ be an integer, and let $U=\bigcup_{f\in S_d}D_+(f)$.
Then the restriction to $U$ of the canonical homomorphism $\sh{O}_X(nd)\otimes_{\sh{O}_X}\sh{O}_X(-nd)\to\sh{O}_X$ is an isomorphism for every integer $n$.
\end{proposition}

\begin{proof}
By \sref{II.2.5.16}, we can restrict to the case where $d=1$, and the conclusion then follows from the proof of \sref{II.2.5.13}.
\end{proof}


\subsection{The graded $S$-module associated to a sheaf on $\operatorname{Proj}(S)$}
\label{subsection:II.2.6}

\emph{We suppose all throughout this section that the ideal $S_+$ of $S$ is generated by the set $S_1$ of homogeneous elements of degree~$1$.}

\begin{env}[2.6.1]
\label{II.2.6.1}
The $\sh{O}_X$-module $\sh{O}_X(1)$ is \emph{invertible} \sref{II.2.5.9};
we thus define, for every $\sh{O}_X$-module $\sh{F}$ \sref[0]{0.5.4.6},
\[
\label{II.2.6.1.1}
  \Gamma_\bullet(\sh{F})
  = \Gamma_\bullet(\sh{O}_X(1),\sh{F})
  = \bigoplus_{n\in\bb{Z}} \Gamma(X,\sh{F}(n))
\tag{2.6.1.1}
\]
taking \sref{II.2.5.14.2} into account.
Recall \sref[0]{0.5.4.6} that $\Gamma_\bullet(\sh{O}_X)$ is endowed with the structure of a \emph{graded ring}, and $\Gamma_\bullet(\sh{F})$ with the structure of a \emph{graded $\Gamma_\bullet(\sh{O}_X)$-module}.

Since $\sh{O}_X(n)$ is locally free, $\Gamma_\bullet(\sh{F})$ is a \emph{left exact} additive covariant functor in $\sh{F}$;
in particular, if $\sh{J}$ is a sheaf of ideals of $\sh{O}_X$, then $\Gamma_\bullet(\sh{J})$ is canonically identified with a \emph{graded idea} of $\Gamma_\bullet(\sh{O}_X)$.
\end{env}

\begin{env}[2.6.2]
\label{II.2.6.2}
Let $M$ be a graded $S$-module;
for every $f\in S_d$ ($d>0$), $x\mapsto x/1$ is a homomorphism of abelian groups $M_0\to M_{(f)}$, and, since $M_{(f)}$ is canonically identified
\oldpage[II]{37}
with $\Gamma(D_+(f),\widetilde{M})$, we thus obtain a homomorphism of abelian groups $\alpha_0^f: M_0\to\Gamma(D_+(f),\widetilde{M})$.
It is clear that, for every $g\in S_e$ ($e>0$), the diagram
\[
  \xymatrix{
    & \Gamma(D_+(f),\widetilde{M}) \ar[dd]
  \\M_0 \ar[ur]^{\alpha_0^f} \ar[dr]_{\alpha_0^{fg}}
  \\& \Gamma(D_+(fg),\widetilde{M})
  }
\]
commutes;
this implies that, for all $x\in M_0$, the sections $\alpha_0^f(x)$ and $\alpha_0^g(x)$ of $M$ agree on $D_+(f)\cap D_+(g)$, and thus there exists a unique section $\alpha_0(x)\in\Gamma(X,\widetilde{M})$ whose restriction to each $D_+(f)$ is $\alpha_0^f(x)$.
We have thus defined (without using the hypothesis that $S$ be generated by $S_1$) a homomorphism of abelian groups
\[
\label{II.2.6.2.1}
  \alpha_0: M_0\to\Gamma(X,\widetilde{M}).
\tag{2.6.2.1}
\]

Applying this result to the graded $S$-module $M(n)$ (for each $n\in\bb{Z}$), we obtain, for each $n\in\bb{Z}$, a homomorphism of abelian groups
\[
\label{II.2.6.2.2}
  \alpha_n: M_n=(M(n))_0\to\Gamma(X,\widetilde{M}(n))
\tag{2.6.2.2}
\]
(taking \sref{II.2.5.15});
whence we obtain a functorial homomorphism (of degree~$0$) of graded abelian groups
\[
\label{II.2.6.2.3}
  \alpha: M\to\Gamma_\bullet(\widetilde{M})
\tag{2.6.2.3}
\]
(also denoted by $\alpha_M$) which, on each $M_n$, agrees with $\alpha_n$.

If we take, in particular, $M=S$, then we immediately see (taking into account the definition \sref[0]{0.5.4.6} of multiplication in $\Gamma_\bullet(\sh{O}_X)$) that $\alpha: S\to\Gamma_\bullet(\sh{O}_X)$ is a homomorphism of graded rings, and that, for every graded $S$-module $M$, \sref{II.2.6.2.3} is a di-homomorphism of graded modules.
\end{env}

\begin{proposition}[2.6.3]
\label{II.2.6.3}
For every $f\in S_d$ ($d>0$), $D_+(f)$ is identical to the set of $\mathfrak{p}\in X$ on which the section $\alpha_d(f)$ of $\sh{O}_X(d)$ does not vanish \sref[0]{0.5.5.2}.
\end{proposition}

\begin{proof}
Since $X=\bigcup_{g\in S_1}D_+(g)$ by hypothesis, it suffices to show that, for all $g\in S_1$, the set of $\mathfrak{p}\in D_+(g)$ on which $\alpha_d(f)$ does not vanish is identical to $D_+(fg)$.
But the restriction of $\alpha_d(f)$ to $D_+(g)$ is, by definition, the section corresponding to the element $f/1$ of $(S(d))_{(g)}$;
under the canonical isomorphism $(S(d))_{(g)}\simto S_{(g)}$ \sref{II.2.5.7}, this section of $\sh{O}_X(d)$ over $D_+(g)$ is identified with the section of $\sh{O}_X$ over $D_+(g)$ that corresponds to the element $f/g^d$ of $S_{(g)}$;
to say that this section vanishes at $\mathfrak{p}\in D_+(g)$ implies that $f/g^d\in\mathfrak{q}$, where $\mathfrak{q}$ is the prime ideal of $S_{(g)}$ corresponding to $\mathfrak{p}$ \sref{II.2.3.6};
by definition, this implies that $f\in\mathfrak{p}$, whence the proposition.
\end{proof}

\begin{env}[2.4.6]
\label{II.2.4.6}
Now let $\sh{F}$ be an $\sh{O}_X$-modules, and set $M=\Gamma_\bullet(\sh{F})$;
by the existence of the homomorphism of graded rings $\alpha: S\to\Gamma_\bullet(\sh{O}_X)$, we can consider $M$ as a graded $S$-module.
For every $f\in S_d$ ($d>0$), it follows from \sref{II.2.6.3} that the restriction to $D_+(f)$ of the section $\alpha_d(f)$ of $\sh{O}_X(d)$ is invertible;
thus so too is the restriction to $D_+(f)$ of the section $\alpha_d(f^n)$ of $\sh{O}_X(nd)$, for all $n>0$.
So let $z\in M_{nd}=\Gamma(X,\sh{F}(nd)$ ($n>0$);
if there exists an integer $k\geq0$ such that the restriction to $D_+(f)$ of $f^kz$, i.e. the
\oldpage[II]{38}
section $(z|D_+(f))(\alpha_d(f^k)|D_+(f))$ of $\sh{F}((n+k)d)$, is zero, then, by the above remark, we also have that $z|D_+(f)=0$.
This shows that we have defined an $S_{(f)}$-homomorphism $\beta_f: M_{(f)}\to\Gamma(D_+(f),\sh{F})$ by sending the element $z/f^n$ to the section $(z|D_+(f))(\alpha_d(f^n)|D_+(f))^{-1}$ of $\sh{F}$ over $D_+(f)$.
We can further immediately show that the diagram
\[
\label{II.2.6.4.1}
  \xymatrix{
    M_{(f)} \ar[r]^{\beta_f} \ar[d]
    & \Gamma(D_+(f),\sh{F}) \ar[d]
  \\M_{(fg)} \ar[r]_{\beta_{fg}}
    & \Gamma(D_+(fg),\sh{F})
  }
\tag{2.6.4.1}
\]
commutes for $g\in S_e$ ($e>0$).
If we recall that $M_{(f)}$ is canonically identified with $\Gamma(D_+(f),\widetilde{M})$, and that the $D_+(f)$ form a base for the topology of $X$ \sref{II.2.3.4}, then we see that the $\beta_f$ come from a unique canonical homomorphism of $\sh{O}_X$-modules
\[
\label{II.2.6.4.2}
  \beta:(\Gamma_\bullet(\sh{F}))\supertilde\to\sh{F}
\tag{2.6.4.2}
\]
(also denoted by $\beta_{\sh{F}}$) which is evidently functorial.
\end{env}

\begin{proposition}[2.6.5]
\label{II.2.6.5}
Let $M$ be a graded $S$-module, and $\sh{F}$ an $\sh{O}_X$-module;
then the composite homomorphisms
\[
\label{II.2.6.5.1}
  \widetilde{M}
  \xrightarrow{\widetilde{\alpha}} (\Gamma_\bullet(\widetilde{M}))\supertilde
  \xrightarrow{\beta} \widetilde{M}
\tag{2.6.5.1}
\]
\[
\label{II.2.6.5.2}
  \Gamma_\bullet(\sh{F})
  \xrightarrow{\alpha} \Gamma_\bullet((\Gamma_\bullet(\sh{F}))\supertilde)
  \xrightarrow{\Gamma_\bullet(\beta)} \Gamma_\bullet(\sh{F})
\tag{2.6.5.2}
\]
are the identity isomorphisms.
\end{proposition}

\begin{proof}
The proof for \sref{II.2.6.5.1} is local:
in an open subset $D_+(f)$, it follows immediately from the definitions, along with the fact that $\beta$, applied to quasi-coherent sheaves, is determined by its action on the sections over $D_+(f)$ \sref[I]{I.1.3.8}.
The proof for \sref{II.2.6.5.2} is done for each degree separately:
if we set $M=\Gamma_\bullet(\sh{F})$, then $M_n=\Gamma(X,\sh{F}(n))$, and $(\Gamma_\bullet(\widetilde{M}))_n=\Gamma(X,\widetilde{M}(n))=\Gamma(X,(M(n))\supertilde)$.
But if $f\in S_1$ and $z\in M_n$, then $\alpha_n^f(z)$ is the element $z/1$ of $(M(n))_{(f)}$, equal to $(f/1)^n(z/f^n)$;
it corresponds, via $\beta_f$, to the section
\[
  \Big(\big(\alpha_1(f)\big)^n|D_+(f)\Big) \Big(\big(z|D_+(f)\big)\big((\alpha_1(f))^n|D_+(f)\big)^{-1}\Big)
\]
over $D_+(f)$, i.e. the restriction of $z$ to $D_+(f)$, which finishes the proof for \sref{II.2.6.5.2}.
\end{proof}


\subsection{Finiteness conditions}
\label{subsection:II.2.7}

\begin{proposition}[2.7.1]
\label{II.2.7.1}
\begin{enumerate}
  \item[{\rm(i)}] If $S$ is a graded Noetherian ring, then $X=\Proj(S)$ is a Noetherian scheme.
  \item[{\rm(ii)}] If $S$ is a graded $A$-algebra of finite type, then $X=\Proj(S)$ is a scheme of finite type over $Y=\Spec(A)$.
\end{enumerate}
\end{proposition}

\oldpage[II]{39}
\begin{proof}
\medskip\noindent
\begin{enumerate}
  \item[{\rm(i)}] If $S$ is Noetherian, then the ideal $S_+$ admits a finite system of homogeneous generators $f_i$ ($1\leq i\leq p$), thus \sref{II.2.3.14} the underlying space $X$ is the union of the $D_+(f_i)=\Spec(S_{(f_i)})$, and everything then reduces to showing that each of the $S_{(f_i)}$ is Noetherian, which follows from \sref{II.2.2.6}.
  \item[{\rm(ii)}] The hypothesis implies that $S_0$ is an $A$-algebra of finite type, and that $S$ is an $S_0$-algebra of finite type, and so $S_+$ is an ideal of finite type \sref{II.2.1.4}.
    We are thus reduced, as in (i), to showing that $S_{(f)}$ is an $A$-algebra of finite type for all $f\in S_d$.
    By \sref{II.2.2.5}, it suffices to show that $S^{(d)}$ is an $A$-algebra of finite type, which follows from \sref{II.2.1.6}.
\end{enumerate}
\end{proof}

\begin{env}[2.7.2]
\label{II.2.7.2}
In what follows, we consider the following finiteness conditions for a graded $S$-module $M$:
\begin{enumerate}
  \item[{\rm(TF)}] There exists an integer $n$ such that the submodule $\bigoplus_{k\geq n}M_k$ is an $S$-module of finite type.
  \item[{\rm(TN)}] There exists an integer $n$ such that $M_k=0$ for $k\geq n$.
\end{enumerate}

If $M$ satisfies (TN), then $M_{(f)}=0$ for all homogeneous $f$ in $S_+$, and thus $\widetilde{M}=0$.

Let $M$ and $N$ be graded $S$-modules;
we say that a homomorphism $u: M\to N$ of degree~$0$ is (TN)-\emph{injective} (resp. (TN)-\emph{surjective}, (TN)-\emph{bijective}) if there exists an integer $n$ such that $u_k:M_k\to N_k$ is injective (resp. surjective, bijective) for $k\geq n$.
Saying that $u$ is (TN)-injective (resp. (TN)-surjective) thus reduces to saying that $\Ker u$ (resp. $\Coker u$) satisfies (TN).
By \sref{II.2.5.4}, if $u$ is (TN)-injective (resp. (TN)-surjective, (TN)-bijective), then $\widetilde{u}$ is injective (resp. surjective, bijective);
if $u$ is (TN)-bijective, then we also say that $u$ is a (TN)-\emph{isomorphism}.
\end{env}

\begin{proposition}[2.7.3]
\label{II.2.7.3}
Let $S$ be a graded ring such that the ideal $S_+$ is of finite type, and let $M$ be a graded $S$-module.
\begin{enumerate}
  \item[{\rm(i)}] If $M$ satisfies condition (TF) then the $\sh{O}_X$-module $\widetilde{M}$ is of finite type.
  \item[{\rm(ii)}] Suppose that $M$ satisfies (TF);
    for $\widetilde{M}=0$, it is necessary and sufficient for $M$ to satisfy (TN).
\end{enumerate}
\end{proposition}

\begin{proof}
We have just seen that condition (TN) implies that $\widetilde{M}=0$.
If $M$ satisfies (TF), then the graded submodule $M'=\bigoplus_{k\geq n}M_k$, which is, by hypothesis, of finite type, is such that $M/M'$ satisfies (TN);
thus $(M/M')\supertilde$, and the exactness of the functor $\widetilde{M}$ \sref{II.2.5.4} implies that $\widetilde{M}=\widetilde{M'}$;
to prove that $\widetilde{M}$ is of finite type, we can thus reduce to the case where $M$ is \emph{of finite type} .
Since the question is local, it suffices to prove that $M_{(f)}$ is an $S_{(f)}$-module of finite type \sref[I]{I.1.3.9};
but $M^{(d)}$ is an $S^{(d)}$-module of finite type \sref{II.2.1.6}[iii], and our claim then follows from \sref{II.2.2.5}.

Now suppose that $M$ satisfies (TF) and that $\widetilde{M}=0$;
then, with the same notation as above, we have that $\widetilde{M'}=0$, and condition (TN) for $M'$ is equivalent to condition (TN) for $M$, so to prove that $\widetilde{M}=0$ implies that $M$ satisfies (TN), we can again restrict to the case where $M$ is generated by a finite number of homogeneous elements $x_i$ ($1\leq i\leq p$);
let $(f_j)_{1\leq j\leq q}$ be a system of homogeneous generators of the ideal $S_+$.
By hypothesis, $M_{(f_j)}=0$ for all $j$, and so there exists an integer $n$ such that $f_j^nx_i=0$ for any $i$ and $j$.
Let $n_j$ be the degree of $f_j$, and let $m$ be the largest value of $\sum_j r_jn_j$ for the system of finitely many integers $(r_j)$ such that $\sum_j r_j\leq nq$;
it is then clear
\oldpage[II]{40}
that, if $k>m$, then $S_kx_i=0$ for all $i$;
if $h$ is the largest of the degrees of the $x_i$, then we conclude that $M_k=0$ for $k>h+m$, which finishes the proof.
\end{proof}

\begin{corollary}[2.7.4]
\label{II.2.7.4}
Let $S$ be a graded ring such that the ideal $S_+$ is of finite type;
for $X=\Proj(S)=\emp$, it is necessary and sufficient for there to exist $n$ such that $S_k=0$ for $k\geq n$.
\end{corollary}

\begin{proof}
The condition $X=\emp$ is equivalent to $\sh{O}_X=\widetilde{S}=0$, and $S$ is a monogeneous $S$-module.
\end{proof}

\begin{theorem}[2.7.5]
\label{II.2.7.5}
Suppose that the ideal $S_+$ is generated by a finite number of homogeneous elements of degree~$1$;
let $X=\Proj(S)$.
Then, for every quasi-coherent $\sh{O}_X$-module $\sh{F}$, the canonical homomorphism $\beta:(\Gamma_\bullet(\sh{F}))\supertilde\to\sh{F}$ \sref{II.2.6.4} is an isomorphism.
\end{theorem}

\begin{proof}
If $S_+$ is generated by a finite number of elements $f_i\in S_1$, then $X$ is the union of the subspaces $\Spec(S_{(f_i)})$ \sref{II.2.3.6}, which are quasi-compact, and so $X$ is quasi-compact;
furthermore, $X$ is a scheme \sref{II.2.4.2};
by \sref[I]{I.9.3.2}, \sref{II.2.5.14.2}, and \sref{II.2.6.3}, we have, for all $f\in S_d$ ($d>0$), a canonical isomorphism $(\Gamma_\bullet(\sh{F}))_{(\alpha_d(f))}\simto\Gamma(D_+(f),\sh{F})$;
also, by definition, $(\Gamma_\bullet(\sh{F}))_{(\alpha_d(f))}$ (where $\Gamma_\bullet(\sh{F})$ is considered as a $\Gamma_\bullet(\sh{O}_X)$-module) is exactly $(\Gamma_\bullet(\sh{F}))_{(f)}$ (where $\Gamma_\bullet(\sh{F})$ is considered as an $S$-module);
if we refer to the definition \sref[I]{I.9.3.1} of the above canonical isomorphism, then we see that it agrees with the homomorphism $\beta_f$, whence the theorem.
\end{proof}

\begin{remark}[2.7.6]
\label{II.2.7.6}
If we suppose that the graded ring $S$ is \emph{Noetherian}, then the condition of \sref{II.2.7.5} is satisfied \emph{ipso facto} as soon as we suppose that the ideal $S_+$ is generated by the set $S_1$ of homogeneous elements of degree~$1$.
\end{remark}

\begin{corollary}[2.7.7]
\label{II.2.7.7}
Under the hypotheses of \sref{II.2.7.5}, every quasi-coherent $\sh{O}_X$-module $\sh{F}$ is isomorphic to an $\sh{O}_X$-module of the form $\widetilde{M}$, where $M$ is a graded $S$-module.
\end{corollary}

\begin{corollary}[2.7.8]
\label{II.2.7.8}
Under the hypotheses of \sref{II.2.7.5}, every quasi-coherent $\sh{O}_X$-module $\sh{F}$ of finite type is isomorphic to an $\sh{O}_X$-module of the form $\widetilde{N}$, where $N$ is a graded $S$-module of finite type.
\end{corollary}

\begin{proof}
We can suppose that $\sh{F}=\widetilde{M}$, where $M$ is a graded $S$-module \sref{II.2.7.7}.
Let $(f_\lambda)_{\lambda\in L}$ be a system of homogeneous generators of $M$;
for every finite subset $H$ of $L$, let $M_H$ be the graded submodule of $M$ generated by the $f_\lambda$ such that $\lambda\in H$;
it is clear that $M$ is the inductive limit of its submodules $M_H$, and so $\sh{F}$ is the inductive limit of its sub-$\sh{O}_X$-modules $\widetilde{M_H}$ \sref{II.2.5.4}.
But, since $\sh{F}$ is of finite type, and since the underlying space of $X$ is quasi-compact, it follows from \sref[0]{0.5.2.3} that $\sh{F}=\widetilde{M_H}$ for some finite subset $H$ of $L$.
\end{proof}

\begin{corollary}[2.7.9]
\label{II.2.7.9}
Under the hypotheses of \sref{II.2.7.5}, let $\sh{F}$ be a quasi-coherent $\sh{O}_X$-module of finite type.
Then there exists an integer $n_0$ such that, for all $n\geq n_0$, $\sh{F}(n)$ is isomorphic to a quotient of an $\sh{O}_X$-module of the form $\sh{O}_X^k$ (for some $k>0$ depending on $n$), and is thus generated by a finite number of its sections over $X$ \sref[0]{0.5.1.1}.
\end{corollary}

\begin{proof}
By \sref{II.2.7.8}, we can suppose that $\sh{F}=\widetilde{M}$, where $M$ is a quotient of a finite direct sum of $S$-modules of the form $S(m_i)$;
by \sref{II.2.5.4}, we can thus restrict to the case where $M=S(m)$, and so $\sh{F}(n)=(S(m+n))\supertilde=\sh{O}_X(m+n)$ \sref{II.2.5.15}.
It thus suffices to prove

\oldpage{41}
  \begin{lemma}[2.7.9.1]
  \label{II.2.7.9.1}
  Under the hypotheses of \sref{II.2.7.5}, for all $n\geq0$, there exists an integer $k$ (depending on $n$) and a surjective homomorphism $\sh{O}_X^k\to\sh{O}_X(n)$.
  \end{lemma}

It suffices \sref{II.2.7.2} to show that, for suitable $k$, there is a (TN)-surjective homomorphism $u$ \emph{of degree~$0$} from the graded product $S$-module $S^k$ to $S(n)$.
But $(S(n))_0=S_n$, and, by hypothesis, $S_h=S_1^h$ for all $h>0$, and so $SS_n=S_n+S_{n+1}+\ldots$.
Since $S_n$ is an $S_0$-module of finite type (\sref{II.2.1.5} and \sref{II.2.1.6}[i]), consider a system of generators $(a_i)_{1\leq i\leq k}$ of this module;
consider the homomorphism $u$ that sends $a_i$ to the $i$-th element of the canonical basis of $S^k$ ($1\leq i\leq k$);
since $\Coker u$ can then be identified with $(S(n))_{-n}+\ldots+(S(n))_{-1}$, $u$ is indeed the desired homomorphism.
\end{proof}

\begin{corollary}[2.7.10]
\label{II.2.7.10}
Under the hypotheses of \sref{II.2.7.5}, let $\sh{F}$ be a quasi-coherent $\sh{O}_X$-module of finite type.
Then there exists an integer $n_0$ such that, for all $n\geq n_0$, $\sh{F}$ is isomorphic to a quotient of an $\sh{O}_X$-module of the form $(\sh{O}_X(-n))^k$ (for some $k$ depending on $n$).
\end{corollary}

\begin{proposition}[2.7.11]
\label{II.2.7.11}
Suppose that the hypotheses of \sref{II.2.7.5} are satisfied, and let $M$ be a graded $S$-module.
Then:
\begin{enumerate}
  \item[{\rm(i)}] The canonical homomorphism $\widetilde{\alpha}:\widetilde{M}\to(\Gamma_\bullet(\widetilde{M}))\supertilde$ is an isomorphism.
  \item[{\rm(ii)}] Let $\sh{G}$ be a quasi-coherent sub-$\sh{O}_X$-module of $\widetilde{M}$, and let $N$ be the graded sub-$S$-module of $M$ given by the  inverse image of $\Gamma_\bullet(\sh{G})$ under $\alpha$.
    Then $\widetilde{N}=\sh{G}$ (where $\widetilde{N}$ is identified, by \sref{II.2.5.4}, with a sub-$\sh{O}_X$-module of $\widetilde{M}$).
\end{enumerate}
\end{proposition}

\begin{proof}
Since $\beta:(\Gamma_\bullet(\widetilde{M}))\supertilde\to\widetilde{M}$ is an isomorphism (by \sref{II.2.7.5}), $\widetilde{\alpha}$ is the inverse isomorphism (by \sref{II.2.6.5.1}), whence (i).
Let $P$ be the graded submodule $\alpha(M)$ of $\Gamma_\bullet(\widetilde{M})$;
since $\widetilde{M}$ is an exact functor \sref{II.2.5.4}, the image of $\widetilde{M}$ under $\widetilde{\alpha}$ is equal to $\widetilde{P}$, and so, by (i), $\widetilde{P}=(\Gamma_\bullet(\widetilde{M}))\supertilde$.
Set $Q=\Gamma_\bullet(\sh{G})\cap P$;
by the above, and by \sref{II.2.5.4}, we have that $\widetilde{Q}=(\Gamma_\bullet(\sh{G}))\supertilde$, and so the restriction of $\beta$ to $\widetilde{Q}$ is an \emph{isomorphism} from this $\sh{O}_X$-module to $G$ by \sref{II.2.7.5}.
But, by the definition of $N$, and by \sref{II.2.5.4}, the restriction of the isomorphism $\widetilde{\alpha}$ to $\widetilde{N}$ is an isomorphism from $\widetilde{N}$ to $\widetilde{Q}$, whence the conclusion, by \sref{II.2.6.5.1}.
\end{proof}


\subsection{Functorial behaviour}
\label{subsection:II.2.8}

\begin{env}[2.8.1]
\label{II.2.8.1}
Let $S$ and $S'$ be positively graded rings, and $\vphi:S'\to S$ a homomorphism of graded rings.
We denote by $G(\vphi)$ the open subset of $X=\Proj(S)$ given by the complement of $V_+(\vphi(S'_+))$, or, equivalently, the union of the $D_+(\vphi(f'))$ where $f'$ runs over the set of homogeneous elements of $S'_+$.
The restriction to $G(\vphi)$ of the continuous map ${}^a\vphi$ from $\Spec(S)$ to $\Spec(S')$ \sref[I]{I.1.2.1} is thus a continuous map from $G(\vphi)$ to $\Proj(S')$, which we again denote, with an abuse of language, by ${}^a\vphi$.
If $f'\in S'_+$ is homogeneous, then
\[
\label{II.2.8.1.1}
  {}^a\vphi^{-1}(D_+(f')) = D_+(\vphi(f'))
\tag{2.8.1.1}
\]
taking into account the fact that ${}^a\vphi$ sends $G(\vphi)$ to $\Proj(S')$, as well as \sref[I]{I.1.2.2.2}.
The homomorphism $\vphi$ also canonically defines (with the same notation) a homomorphism of graded rings $S'_{f'}\to S_f$, whence, by restriction to the degree~$0$ elements,
\oldpage[II]{42}
a homomorphism $S'_{(f')}\to S_{(f)}$, which we denote by $\vphi_{(f)}$;
there is a corresponding \sref[I]{I.1.6.1} morphism of affine schemes $({}^a\vphi_{(f)},\widetilde{\vphi}_{(f)}):\Spec(S_{(f)})\to\Spec(S'_{(f')})$.
If we canonically identify $\Spec(S_{(f)})$ with the scheme induced by $\Proj(S)$ on $D_+(f)$ \sref{II.2.3.6}, then we have defined a morphism $\Phi_f:D_+(f)\to D_+(f')$, and ${}^a\vphi_{(f)}$ is identified with the restriction of ${}^a\vphi$ to $D_+(f)$.
It is also immediate that, if $g'$ is another homogeneous element of $S'_+$, and $g=\vphi(g')$, then the diagram
\[
  \xymatrix{
    D_+(f) \ar[r]^{\Phi_f}
    & D_+(f')
  \\D_+(fg) \ar[u] \ar[r]_{\Phi_{fg}}
    & D_+(f'g') \ar[u]
  }
\]
commutes, by the fact that the diagram
\[
  \xymatrix{
    S'_{(f')} \ar[r]^{\vphi_{(f)}} \ar[d]_{\omega_{f'g',f'}}
    & S_{(f)} \ar[d]^{\omega_{fg,f}}
  \\S'_{(f'g')} \ar[r]_{\vphi_{(fg)}}
    & S_{(fg)}
  }
\]
commutes.
Taking the definition of $G(\vphi)$, along with \sref{II.2.3.3.2}, we thus see that:
\end{env}

\begin{proposition}[2.8.2]
\label{II.2.8.2}
Given a homomorphism of graded rings $\vphi: S'\to S$, there exists exactly one morphism $({}^a\vphi,\widetilde{\vphi})$ from the induced prescheme $G(\vphi)$ to $\Proj(S')$ (said to be \emph{associated to $\vphi$}, and denoted by $\Proj(\vphi)$) such that, for every homogeneous element $f'\in S'_+$, the restriction of this morphism to $D_+(\vphi(f'))$ agrees with the morphism associated to the homomorphism $S'_{(f')}\to S_{(\vphi(f'))}$ corresponding to $\vphi$.
\end{proposition}

\begin{proof}
With the above notation, if $f'\in S'_d$, then the diagram
\[
\label{II.2.8.2.1}
  \xymatrix{
    S'_{(f')} \ar[r]^{\vphi_{(f)}} \ar[d]_{\sim}
    & S_{(f)} \ar[d]^{\sim}
  \\{S'}^{(d)}/(f'-1){S'}^{(d)} \ar[r]
    & S^{(d)}/(f-1)S^{(d)}
  }
\tag{2.8.2.1}
\]
commutes (the vertical arrows being the isomorphisms \sref{II.2.2.5}).
\end{proof}

\begin{corollary}[2.8.3]
\label{II.2.8.3}
\begin{enumerate}
  \item[{\rm(i)}] The morphism $\Proj(\vphi)$ is affine.
  \item[{\rm(ii)}] If $\Ker(\vphi)$ is nilpotent (and, in particular, if $\vphi$ is injective), then the morphism $\Proj(\vphi)$ is dominant.
\end{enumerate}
\end{corollary}

\begin{proof}
Claim~(i) is an immediate consequence of \sref{II.2.8.2} and \sref{II.2.8.1.1}.
Claim~(ii) follows since, if $\Ker(\vphi)$ is nilpotent, then, for every homogeneous $f'$ in $S'_+$, we immediately see that $\Ker(\vphi_f)$ (with $f=\vphi(f')$) is nilpotent, and thus so too is $\Ker(\vphi_{(f)})$;
we then apply \sref{II.2.8.2} and \sref[I]{I.1.2.7}
\end{proof}

\oldpage[II]{43}
We note that there are, in general, morphisms from $\Proj(S)$ to $\Proj(S')$ that are not affine, and that thus do not come from homomorphisms of graded rings $S'\to S$;
an example is the structure morphism $\Proj(S)\to\Spec(A)$ when $A$ is a field ($\Spec(A)$ thus being identified with $\Proj(A[T])$, cf.~\sref{II.3.1.7});
indeed, this follows from \sref[I]{I.2.3.2}.

\begin{env}[2.8.4]
\label{II.2.8.4}
Let $S''$ be another positively graded ring, and $\vphi':S''\to S'$ a homomorphism of graded rings, and set $\vphi''=\vphi\circ\vphi'$.
By \sref{II.2.8.1.1} and the formula ${}^a\vphi''=({}^a\vphi')\circ({}^a\vphi)$, we immediately see that $G(\vphi'')\subset G(\vphi)$, and that, if $\Phi$, $\Phi'$, and $\Phi''$ are the morphisms associated to $\vphi$, $\vphi'$, and $\vphi''$ (respectively), then $\Phi''=\Phi'\circ(\Phi|G(\vphi''))$.
\end{env}

\begin{env}[2.8.5]
\label{II.2.8.5}
Suppose that $S$ (resp. $S'$) is a graded $A$-algebra (resp. a graded $A'$-algebra), and let $\psi:A'\to A$ be a ring homomorphism such that the diagram
\[
  \xymatrix{
    A' \ar[r]^{\psi} \ar[d]
    & A \ar[d]
  \\S' \ar[r]_{\vphi}
    & S
  }
\]
commutes.
We can then consider $G(\vphi)$ (resp. $\Proj(S')$) as a scheme over $\Spec(A)$ ($resp. \Spec(A')$);
if $\Phi$ (resp. $\Psi$) is the morphism associated to $\vphi$ (resp. $\psi$), then the diagram
\[
  \xymatrix{
    G(\vphi) \ar[r]^{\Phi} \ar[d]
    & \Proj(S') \ar[d]
  \\\Spec(A) \ar[r]_{\Psi}
    & \Spec(A')
  }
\]
commutes: it suffices to prove this for the restriction of $\Phi$ to $D_+(f)$, where $f=\vphi(f')$, with $f'$ homogeneous in $S'_+$;
this then follows from the fact that the diagram
\[
  \xymatrix{
    A' \ar[r]^{\psi} \ar[d]
    & A \ar[d]
  \\S'_{(f')} \ar[r]_{\vphi_{(f)}}
    & S_{(f)}
  }
\]
commutes.
\end{env}

\begin{env}[2.8.6]
\label{II.2.8.6}
Now let $M$ be a graded $S$-module, and consider the $S'$-module $M_{[\vphi]}$, which is evidently graded.
Let $f'$ be homogeneous in $S'_+$, and let $f=\vphi(f')$;
we know \sref[0]{0.1.5.2} that there is a canonical isomorphism $(M_{[\vphi]})_{f'}\simto(M_f)_{[\vphi_f]}$, and it is immediate that this isomorphism preserves degree, whence a canonical isomorphism $(M_{[\vphi]})_{(f')}\simto(M_{(f)})_{[\vphi_{(f)}]}$.
To this isomorphism, there canonically corresponds an isomorphism of sheaves $(M_{[\vphi]})\supertilde|D_+(f')\simto(\Phi_f)_*(\widetilde(M)|D_+(f))$ (\sref{II.2.5.2} and \sref[I]{I.1.6.3}).
Furthermore,
\oldpage{43}
if $g'$ is another homogeneous element of $S'_+$, and $g=\vphi(g')$, then the diagram
\[
  \xymatrix{
    (M_{[\vphi]})_{(f')} \ar[r]^{\sim} \ar[d]
    & (M_{(f)})_{[\vphi_{(f)}]} \ar[d]
  \\(M_{[\vphi]})_{(f'g')} \ar[r]^{\sim}
    & (M_{(fg)})_{[\vphi_{(fg)}]}
  }
\]
commutes, whence we immediately conclude that the isomorphism
\[
  (M_{[\vphi]})\supertilde|D_+(f'g') \simto (\Phi_{fg})_*(\widetilde{M}|D_+(fg))
\]
is the restriction to $D_+(f'g')$ of the isomorphism $(M_{[\vphi]})\supertilde|D_+(f')\simto(\Phi_f)_*(\widetilde{M}|D_+(f))$.
Since $\Phi_f$ is the restriction to $D_+(f)$ of the morphism $\Phi$, we see that, taking \sref{II.2.8.1.1} into account, and setting $X'=\Proj(S)'$:
\end{env}

\begin{proposition}[2.8.7]
\label{II.2.8.7}
There exists a canonical functorial isomorphism from the $\sh{O}_{X'}$-module $(M_{[\vphi]})^\supertilde$ to the $\sh{O}_{X'}$-module $\Phi_*(\widetilde{M}|G(\vphi))$.
\end{proposition}

We thus immediately deduce a canonical functorial map from the set of $\vphi$-morphisms $M'\to M$ from a graded $S'$-module to the graded $S$-module $M$, to the set of $\Phi$-morphisms $\widetilde{M'}\to\widetilde{M}|G(\vphi)$.
With the notation of \sref{II.2.8.4}, if $M''$ is a graded $S''$-module, then, to the composition of a $\vphi$-morphism $M'\to M$ and a $\vphi'$-morphism $M''\to M'$, canonically corresponds the composition of $\widetilde{M'}G(\vphi')\to\widetilde{M}|G(\vphi'')$ and $\widetilde{M''}\to\widetilde{M'}|G(\vphi')$.

\begin{proposition}[2.8.8]
\label{II.2.8.8}
Under the hypotheses of \sref{II.2.8.1}, let $M'$ be a graded $S'$-module.
Then there exists a canonical functorial homomorphism $\nu$ from the $(\sh{O}_X|G(\vphi))$-module $\Phi^*(\widetilde{M'})$ to the $(\sh{O}_X|G(\vphi))$-module $(M'\otimes_{S'}S)\supertilde|G(\vphi)$.
If the ideal $S'_+$ is generated by $S'_1$, then $\nu$ is an isomorphism.
\end{proposition}

\begin{proof}
Indeed, for $f'\in S'_d$ ($d>0$), we define a canonical functorial homomorphism of $S_{(f)}$-modules (where $f=\vphi(f')$)
\[
\label{II.2.8.8.1}
  \nu_f: M'_{(f')}\otimes_{S'_{(f')}}S_{(f)} \to (M'\otimes_{S'}S)_{(f)}
\tag{2.8.8.1}
\]
by composing the homomorphism $M'_{(f')}\otimes_{S'_{(f')}}S_{(f)}\to M'_{f'}\otimes_{S'_{f'}}S_f$ and the canonical isomorphism $M'_{f'}\otimes_{S'_{f'}}S_f\simto(M'\otimes_{S'}S)_f$ \sref[0]{0.1.5.4}, and noting that the latter preserves degrees.
We can immediately verify the compatibility of $\nu_f$ with the restriction operators from $D_+(f)$ to $D_+(fg)$ (for any $g'\in S'_+$ and $g=\vphi(g')$), whence the definition of the homomorphism
\[
  \nu: \Phi^*(\widetilde{M'}) \to (M'\otimes_{S'}S)\supertilde|G(\vphi)
\]
taking \sref[I]{I.1.6.5} into account.
To prove the second claim, it suffices to show that $\nu_f$ is an isomorphism for all $f'\in S_1$, since $G(\vphi)$ is then a union of the $D_+(\vphi(f'))$.
We first define a $\bb{Z}$-bilinear $M'_m\times S_n\to M'_{(f')}\otimes_{S'_{(f')}}S_{(f)}$ by sending $(x',s)$ to the element $(x'/{f'}^m)\otimes(s/f^n)$ (with the convention that $x'/{f'}^m$ is ${f'}^{-m}x'/1$ when $m<0$).
\oldpage[II]{45}
We claim that, in the proof of \sref{II.2.5.13}, this map gives rise to a di-homomorphism of modules
\[
  \eta_f: M'\otimes_{S'}S \to M'_{(f')}\otimes_{S_{(f')}}S_{(f)}.
\]
Furthermore, if, for $r>0$, we have $f^r\sum_i(x'_i\otimes s_i)=0$, then this can also be written as $\sum_i({f'}^rx'_i\otimes s_i)=0$, whence, by \sref[0]{0.1.5.4}, $\sum_i({f'}^rx_i/{f'}^{m_i+r})\otimes(s_i/f^{n_i})=0$, i.e. $\eta_f(\sum_i x_i\otimes y_i)0=$, which proves that $\eta_f$ factors as $M'\otimes_{S'}S\to(M'\otimes_{S'}S)_f\xrightarrow{\eta'_f}M'_{(f')}\otimes_{S'_{(f')}}S_{(f)}$;
we finally can prove that $\eta'_f$ and $\nu_f$ are inverse isomorphisms to one another.

In particular, it follows from \sref{II.2.1.2.1} that we have a canonical homomorphism
\[
\label{II.2.8.8.2}
  \Phi^*(\sh{O}_{X'}(n)) \simto \sh{O}_X(n)|G(\vphi)
\tag{2.8.8.2}
\]
for all $n\in\bb{Z}$.
\end{proof}

\begin{env}[2.8.9]
\label{II.2.8.9}
Let $A$ and $A'$ be rings, and $\psi:A'\to A$ a ring homomorphism, defining a morphism $\Psi:\Spec(A)\to\Spec(A')$.
Let $S'$ be a positively graded $A'$-algebra, and set $S=S'\otimes_{A'}A$, which is evidently an $A$-algebra graded by the $S'_n\otimes_{A'}A$;
the map $\vphi:s'\to s'\otimes1$ is then a graded ring homomorphism that makes the diagram \sref{II.2.8.5.1} commute.
Since $S_+$ is here the $A$-module generated by $\vphi(S'_+)$, we have $G(\vphi)=\Proj(S)=X$;
whence, setting $X'=\Proj(S')$, we have the commutative diagram
\[
\label{II.2.8.9.1}
  \xymatrix{
    X \ar[r]^{\Phi} \ar[d]_p
    & X' \ar[d]
  \\Y \ar[r]_{\Psi}
    & Y'
  }
\tag{2.8.9.1}
\]

Now let $M'$ be a graded $S'$-module, and set $M=M'\otimes_{A'}A=M'\otimes_{S'}S$.
Under these conditions:
\end{env}

\begin{proposition}[2.8.10]
\label{II.2.8.10}
The diagram \sref{II.2.8.9.1} identifies the scheme $X$ with the product $X'\times_{Y'}Y$;
furthermore, the canonical homomorphism $\nu:\Phi^*(\widetilde{M'})\to\widetilde{M}$ \sref{II.2.8.8} is an isomorphism.
\end{proposition}

\begin{proof}
The first claim will be proven if we show that, for every homogenous $f'$ in $S'_+$, setting $f=\vphi(f')$, the restrictions of $\Phi$ and $p$ to $D_+(f)$ identify this scheme with the product $D_+(f')\times_{Y'}Y$ \sref[I]{I.3.2.6.2};
in other words, it suffices \sref[I]{I.3.2.2} to prove that $S_{(f)}$ is canonically identified with $S_f\simto S'_{f'}\otimes_{A'}A$, which is immediate by the existence of the canonical isomorphism $S_f\simto S'_{f'}\otimes_{A'}A$ that preserves degrees \sref[0]{0.1.5.4}.
The second claim then follows from the fact that $M'_{(f')}\otimes_{S'_{(f')}}S_{(f)}$ can be identified, by the above, with $M'_{(f')}\otimes_{A'}A$, and this can be identified with $M_{(f)}$, since $M_f$ is canonically identified with $M'_{f'}\otimes_{A'}A$ by an isomorphism that preserves degrees.
\end{proof}

\begin{corollary}[2.8.11]
\label{II.2.8.11}
For every integer $n\in\bb{Z}$, $\widetilde{M}(n)$ can be identified with $\Phi^*(\widetilde{M'}(n))=\widetilde{M'}(n)\otimes_{Y'}\sh{O}_Y$;
in particular, $\sh{O}_X(n)=\Phi^*(\sh{O}_{X'}(n))=\sh{O}_{X'}(n)\otimes_{Y'}\sh{O}_Y$.
\end{corollary}

\begin{proof}
This follows from \sref{II.2.8.10} and \sref{II.2.5.15}.
\end{proof}

\oldpage[II]{46}

\begin{env}[2.8.12]
\label{II.2.8.12}
Under the hypotheses of \sref{II.2.8.9}, for $f'\in S'_d$ ($d>0$) and $f=\vphi(f')$, the diagram
\[
  \xymatrix{
    M'_{(f')} \ar[r]^-{\sim} \ar[d]
    & {M'}^{(d)}/(f'-1){M'}^{(d)} \ar[d]
  \\M_{(f)} \ar[r]^-{\sim}
    & M^{(d)}/(f-1)M^{(d)}
  }
\]
(cf. \sref{II.2.2.5}) commutes.
\end{env}

\begin{env}[2.8.13]
\label{II.2.8.13}
Keep the notation and hypotheses of \sref{II.2.8.9}, and let $\sh{F}'$ be an $\sh{O}_{X'}$-module;
if we set $\sh{F}=\Phi^*(\sh{F}')$, then, for all $n\in\bb{Z}$, we have $\sh{F}(n)=\Phi^*(\sh{F}'(n))$, by \sref{II.2.8.11} and \sref[0]{0.4.3.3}.
Then \sref[0]{0.3.7.1} we have a canonical homomorphism
\[
  \Gamma(\rho): \Gamma(X',\sh{F}'(n)) \to \Gamma(X,\sh{F}(n))
\]
which gives a canonical di-homomorphism of graded modules
\[
  \Gamma_\bullet(\sh{F}') \to \Gamma_\bullet(\sh{F}).
\]

Suppose that the ideal $S_+$ is generated by $S_1$, and that $\sh{F}'=\widetilde{M'}$, thus $\sh{F}=\widetilde{M}$ with $M=M'\otimes_{A'}A$.
If $f'$ is homogeneous in $S'_+$, and $f=\vphi(f')$, then we have seen that $M_{(f)}=M'_{(f')}\otimes_{A'}A$, and the diagram
\[
  \xymatrix{
    M'_0 \ar[r] \ar[d]
    & M'_{(f')} \ar[d]
    & =\Gamma(D_+(f'),\widetilde{M'})
  \\M_0 \ar[r]
    & M_{(f)}
    & =\Gamma(D_+(f),\widetilde{M})
  }
\]
thus commutes;
we immediately conclude from this remark, and from the definition of the homomorphism $\alpha$ \sref{II.2.6.2}, that the diagram
\[
\label{II.2.8.13.1}
  \xymatrix{
    M' \ar[r]^-{\alpha_{M'}} \ar[d]
    & \Gamma_\bullet(\widetilde{M'}) \ar[d]
  \\M \ar[r]_-{\alpha_M}
    & \Gamma_\bullet(\widetilde{M})
  }
\tag{2.8.13.1}
\]
commutes.
Similarly, the diagram
\[
\label{II.2.8.13.2}
  \xymatrix{
    (\Gamma_\bullet(\sh{F}'))\supertilde \ar[r]^-{\beta_{\sh{F}'}} \ar[d]
    & \sh{F}' \ar[d]
  \\(\Gamma_\bullet(\sh{F}))\supertilde \ar[r]_-{\beta_{\sh{F}}}
    & \sh{F}
  }
\tag{2.8.13.2}
\]
commutes (the vertical arrow on the right being the canonical $\Phi$-morphism $\sh{F}'\to\Phi^*(\sh{F}')=\sh{F}$).
\end{env}

\oldpage[II]{47}

\begin{env}[2.8.14]
\label{II.2.8.14}
Still keeping the notation and hypotheses of \sref{II.2.8.9}, let $N'$ be another graded $S'$-module, and let $N=N'\otimes_{A'}A$.
It is immediate that the canonical di-homomorphisms $M'\to M$ and $N'\to N$ give a di-homomorphism $M'\otimes_{S'}N'\to M\otimes_S N$ (with respect to the canonical ring homomorphism $S'\to S$), and thus also an $S$-homomorphism $(M'\otimes_{S'}N')\otimes_{A'}A\to M\otimes_S N$ of degree~$0$, to which corresponds (taking \sref{II.2.8.10} into account) a homomorphism of $\sh{O}_X$-modules
\[
\label{II.2.8.14.1}
  \Phi^*((M'\otimes_{S'}N')\supertilde) \to (M\otimes_S N)\supertilde.
\tag{2.8.14.1}
\]

Furthermore, we can immediately verify that the diagram
\[
\label{II.2.8.14.2}
  \xymatrix{
    \Phi^*(\widetilde{M'}\otimes_{\sh{O}_{X'}}\widetilde{N'}) \ar[r]^\sim \ar[d]_{\Phi^*(\lambda)}
    & \widetilde{M}\otimes_{\sh{O}_X}\widetilde{N} \ar[d]^\lambda
    & =\Phi^*(\widetilde{M'})\otimes_{\sh{O}_X}\Phi^*(\widetilde{N'})
  \\\Phi^*((M'\otimes_{S'}N')\supertilde) \ar[r]
    & (M\otimes_S N)\supertilde
  }
\tag{2.8.14.2}
\]
commutes, with the first line being the canonical isomorphism \sref[0]{0.4.3.3}.
If the ideal $S'_+$ is generated by $S'_1$, then it is clear that $S_+$ is generated by $S_1$, and the two vertical arrows of \sref{II.2.8.14.2} are then isomorphisms \sref{II.2.5.13};
it is thus also the case for \sref{II.2.8.14.1}.

We similarly have a canonical di-homomorphism $\Hom_{S'}(M',N')\to\Hom_S(M,N)$ by sending a homomorphism $u'$ of degree~$k$ to the homomorphism $u'\otimes1$, which is also of degree~$k$;
from this, we again deduce a $S$-homomorphism of degree~$0$
\[
  (\Hom_{S'}(M',N'))\otimes_{A'}A \to \Hom_S(M,N)
\]
whence a homomorphism of $\sh{O}_X$-modules
\[
\label{II.2.8.14.3}
  \Phi^*((\Hom_{S'}(M',N'))\supertilde) \to (\Hom_S(M,N))^\supertilde.
\tag{2.8.14.3}
\]

Furthermore, the diagram
\[
  \xymatrix{
    \Phi^*((\Hom_{S'}(M',N'))\supertilde) \ar[r] \ar[d]_{\Phi^*(\mu)}
    & (\Hom_S(M,N))^\supertilde \ar[d]^\mu
  \\\Phi^*(\shHom_{\sh{O}_{X'}}(\widetilde{M'},\widetilde{N'})) \ar[r]
    & \shHom_{\sh{O}_X}(\widetilde{M},\widetilde{N})
  }
\]
commutes (the second horizontal line being the canonical homomorphism \sref[0]{0.4.4.6}).
\end{env}

\oldpage[II]{48}

\begin{env}[2.8.15]
\label{II.2.8.15}
With the notation and hypotheses of \sref{II.2.8.1}, it follows from \sref{II.2.4.7} that we do not change the morphism $\Phi$, up to isomorphism, when we replace $S_0$ and $S'_0$ by $\bb{Z}$, and $\vphi_0$ by the identity map, and thus when we replace $S$ and $S'$ by $S^{(d)}$ and ${S'}^{(d)}$ (respectively) ($d>0$), and $\vphi$ by its restriction $\vphi^{(d)}$ to $S^{(d)}$.
\end{env}


\subsection{Closed subschemes of a scheme $\operatorname{Proj}(S)$}
\label{subsection:II.2.9}

\begin{env}[2.9.1]
\label{II.2.9.1}
If $\vphi:S\to S'$ is a homomorphism of graded rings, then we say that $\vphi$ is (TN)-\emph{surjective} (resp. (TN)-\emph{injective}, (TN)-\emph{bijective}) if there exists an integer $n$ such that, for $k\geq n$, $\vphi_k:S_k\to S'_k$ is \emph{surjective} (resp. \emph{injective}, \emph{bijective}).
Instead of saying that $\vphi$ is (TN)-bijective, we sometimes say that it is a (TN)-\emph{isomorphism}.
\end{env}

\begin{proposition}[2.9.2]
\label{II.2.9.2}
Let $S$ be a positively graded ring, and let $X=\Proj(S)$.
\begin{enumerate}
  \item[\rm{(i)}] If $\vphi:S\to S'$ is a (TN)-surjective homomorphism of graded rings, then the corresponding morphism $\Phi$ \sref{II.2.8.1} is defined on the whole of $\Proj(S')$, and is a closed immersion of $\Proj(S')$ into $X$.
    If $\mathfrak{J}$ is the kernel of $\vphi$, then the closed subscheme of $X$ associated to $\Phi$ is defined by the quasi-coherent sheaf of ideals $\widetilde{\mathfrak{J}}$ of $\sh{O}_X$.
  \item[\rm{(ii)}] Suppose further that the ideal $S_+$ is generated by a finite number of homogeneous elements of degree~$1$.
    Let $X'$ be a closed subscheme of $X$ defined by a quasi-coherent sheaf of ideals $\sh{J}$ of $\sh{O}_X$.
    Let $\mathfrak{J}$ be the graded ideal of $S$ given by the inverse image of $\Gamma_\bullet(\sh{J})$ under the canonical homomorphism $\alpha:S\to\Gamma_\bullet(\sh{O}_X)$ \sref{II.2.6.2}, and set $S'=S/\mathfrak{J}$.
    Then $X'$ is the subscheme associated to the closed immersion $\Proj(S')\to X$ corresponding to the canonical homomorphism of graded rings $S\to S'$.
\end{enumerate}
\end{proposition}

\begin{proof}
\medskip\noindent
\begin{enumerate}
  \item[\rm{(i)}] We can suppose that $\vphi$ is surjective \sref{II.2.9.1}.
    Since, by hypothesis, $\vphi(S_+)$ generates $S'_+$, we have $G(\vphi)=\Proj(S')$.
    Now, the second claim can be checked locally on $X$;
    so let $f$ be a homogeneous element of $S_+$, and set $f'=\vphi(f)$.
    Since $\vphi$ is a surjective homomorphism of graded rings, we immediately see that $\vphi_{(f')}:S_{(f)}\to S'_{(f')}$ is surjective, and that its kernel is $\mathfrak{J}_{(f)}$, which proves (i) \sref[I]{I.4.2.3}.
  \item[\rm{(ii)}] By (i), we are led to proving that the homomorphism $\widetilde{j}:\widetilde{\mathfrak{J}}\to\sh{O}_X$ induced by the canonical injection $j:\mathfrak{J}\to S$ is an isomorphism from $\widetilde{\mathfrak{J}}$ to $\sh{J}$, which follows from \sref{II.2.7.11}.
\end{enumerate}
\end{proof}

We note that $\mathfrak{J}$ is the \emph{largest} of the graded ideals $\mathfrak{J}'$ of $S$ such that $\widetilde{j}(\widetilde{\mathfrak{J'}})=\sh{J}$, since we can immediately show, using the definitions \sref{II.2.6.2}, that this equation implies that $\alpha(\mathfrak{J}')\subset\Gamma_\bullet(\sh{J})$.

\begin{corollary}[2.9.3]
\label{II.2.9.3}
Suppose that the hypotheses of \sref{II.2.9.2}[(i)] are satisfied, and further that the ideal $S_+$ is generated by $S_1$;
then $\Phi^*((S(n))\supertilde)$ is canonically isomorphic to $(S'(n))\supertilde$ for all $n\in\bb{Z}$, and so $\Phi^*(\sh{F}(n))$ is canonically isomorphic to $\Phi^*(\sh{F})(n)$ for every $\sh{O}_X$-module $\sh{F}$.
\end{corollary}

\begin{proof}
This is a particular case of \sref{II.2.8.8}, taking \sref{II.2.5.10.2} into account.
\end{proof}

\begin{corollary}[2.9.4]
\label{II.2.9.4}
Suppose that the hypotheses of \sref{II.2.9.2}[(ii)] are satisfied.
For the closed sub-prescheme $X'$ of $X$ to be integral, it is necessary and sufficient for the graded ideal $\mathfrak{J}$ to be prime in $S$.
\end{corollary}

\oldpage[II]{49}

\begin{proof}
Since $X'$ is isomorphic to $\Proj(S/\mathfrak{J})$, the condition is sufficient by \sref{II.2.4.4}.
To see that it is necessary, consider the exact sequence $0\to\sh{J}\to\sh{O}_X\to\sh{O}_X/\sh{J}$, which gives the exact sequence
\[
  0 \to \Gamma_\bullet(\sh{J}) \to \Gamma_\bullet(\sh{O}_X) \to \Gamma_\bullet(\sh{O}_X/\sh{J}).
\]

It suffices to prove that, if $f\in S_m$ and $g\in S_n$ are such that the image in $\Gamma_\bullet(\sh{O}_X/\sh{J})$ of $\alpha_{n+m}(fg)$ is zero, then the image of either $\alpha_m(f)$ or $\alpha_n(g)$ is zero.
But, by definition, these images are sections of invertible $(\sh{O}_X/\sh{J})$-modules $\sh{L}=(\sh{O}_X/\sh{J})(m)$ and $\sh{L}'=(\sh{O}_X/\sh{J})(n)$ over the integral scheme $X'$;
the hypothesis implies that the product of these two sections is zero in $\sh{L}\otimes\sh{L}'$ (\sref{II.2.9.3} and \sref{II.2.5.14.1}), and so one of them is zero by \sref[I]{I.7.4.4}.
\end{proof}

\begin{corollary}[2.9.5]
\label{II.2.9.5}
Let $A$ be a ring, $M$ an $A$-module, $S$ a graded $A$-algebra generated by the set $S_1$ of homogeneous elements of degree~$1$, $u:M\to S_1$ a surjective homomorphism of $A$-modules, and $\overline{u}:\bb{S}(M)\to S$ the homomorphism (of $A$-algebras) from the symmetric algebra $\bb{S}(M)$ of $M$ to $S$ that extends $u$.
Then the morphism corresponding to $\overline{u}$ is a closed immersion of $\Proj(S)$ into $\Proj(\bb{S}(M))$.
\end{corollary}

\begin{proof}
Indeed, $\overline{u}$ is surjective by hypothesis, and so it suffices to apply \sref{II.2.9.2}
\end{proof}

\section{Integral morphisms and finite morphisms}
\label{section:II.6}


\subsection{Preschemes integral over another prescheme}
\label{subsection:II.6.1}

\begin{definition}[6.1.1]
\label{II.6.1.1}
Let $X$ be an $S$-prescheme, with structure morphism $f:X\to S$.
We say that $X$ is \emph{integral over $S$}, or that $f$ is an \emph{integral morphism}, if there exists a cover $(S_\alpha)$ of $S$ by affine opens such that, for all $\alpha$, the induced prescheme $f^{-1}(S_\alpha)$ is an affine scheme whose ring $B_\alpha$ is an integral algebra over the ring $A_\alpha$ of $S_\alpha$.
We say that $X$ is \emph{finite over $S$}, or that $f$ is a \emph{finite morphism} if $X$ is integral and of finite type over $S$.
\end{definition}

If $S$ is affine of ring $A$, then we also say ``integral (resp. finite) over $A$'' instead of ``integral (resp. finite) over $S$''.

\begin{env}[6.1.2]
\label{II.6.1.2}
It is clear that, if $X$ is integral over $S$, then it is \emph{affine} over $S$.
For an affine prescheme $X$ over $S$ to be integral (resp. finite) over $S$ it is necessary and sufficient that the associated quasi-coherent $\sh{O}_S$-algebra $\sh{A}(X)$ be such that there exist a cover $(S_\alpha)$ of $S$ by affine opens having the property that, for all $\alpha$, $\Gamma(S_\alpha,\sh{A}(X))$ is an integral (resp. integral and of finite type) algebra over $\Gamma(S_\alpha,\sh{O}_S)$.
A quasi-coherent $\sh{O}_S$-algebra with this property is said to be \emph{integral} (resp. \emph{finite}) over $\sh{O}_S$.
Giving an integral (resp. finite) prescheme over $S$ is thus \sref{II.1.3.1} the same as giving a quasi-coherent $\sh{O}_S$-algebra that is integral (resp. finite) over $\sh{O}_S$.
Note that a quasi-coherent $\sh{O}_S$-algebra $\sh{B}$ is finite if and only if it is an \emph{$\sh{O}_S$-module of finite type} \sref[I]{I.1.3.9};
it is equivalent to say that $\sh{B}$ is an \emph{integral} $\sh{O}_S$-algebra \emph{of finite type}, since an algebra that is integral and of finite type over a ring $A$ is an $A$-module of finite type.
\end{env}

\begin{proposition}[6.1.3]
\label{II.6.1.3}
Let $S$ be a locally Noetherian prescheme.
For an affine prescheme $X$ over $S$ to be finite over $S$, it is necessary and sufficient that the $\sh{O}_S$-algebra $\sh{A}(X)$ be coherent.
\end{proposition}

\begin{proof}
Taking the preceding remark into account, this reduces to noting that, if $S$ is locally Noetherian, then the quasi-coherent $\sh{O}_S$-modules of finite type are exactly the coherent $\sh{O}_S$-modules \sref[I]{I.1.5.1}.
\end{proof}

\begin{proposition}[6.1.4]
\label{II.6.1.4}
Let $X$ be an integral (resp. finite) prescheme over $S$, with structure morphism $f:X\to S$.
Then, for every affine open $U\subset S$ of ring $A$, $f^{-1}(U)$ is an affine scheme whose ring $B$ is an integral (resp. finite) algebra over $A$.
\end{proposition}

\oldpage[II]{111}
\begin{proof}
We first prove the following lemma:

  \begin{lemma}[6.1.4.1]
  \label{II.6.1.4.1}
  Let $A$ be a ring, $M$ an $A$-module, and $(g_i)_{1\leq i\leq m}$ a finite system of elements of $A$ such that the $D(g_i)$ (for $1\leq i\leq m$) cover $\Spec(A)$.
  If, for all $i$, $M_{g_i}$ is an $A_{g_i}$-module of finite type, then $M$ is an $A$-module of finite type.
  \end{lemma}

  \begin{proof}
  We can assume that $M_{g_i}$ admits a finite system of generators $(m_{ij}/g_i^n)$ with $m_{ij}\in M$, with $n$ the same for all indices $i$.
  We will show that the $m_{ij}$ for a system of generators of $M$.
  By hypothesis, for each $i$, there exist $a_{ij}\in A$ and some integer $p$ (independent of $i$) such that, in $M_{g_i}$, $m/1=(\sum_i a_{ij}m_{ij})/g_i^p$;
  this implies that there exists an integer $r\geq p$ such that, for all $i$, we have $g_i^rm\in M'$.
  But, since the $D(g_i^r)=D(g_i)$ cover $\Spec(A)$, the ideal of $A$ generated by the $g_i^r$ is equal to $A$, or, in other words, there exist elements $a_i\in A$ such that $\sum_i a_ig_i^r$;
  then $m=(\sum_i a_i g_i^r)m\in M'$, whence the lemma.
  \end{proof}

Now we already know \sref{II.1.3.2} that $f^{-1}(U)$ is affine.
If $\vphi$ is the homomorphism $A\to B$ corresponding to $f$, then there exists a finite cover of $U$ by opens $D(g_i)$ (where $g_i\in A$) such that, if $h_i=\vphi(g_i)$, $B_{h_i}$ is an integral (resp. integral and finite) algebra over $A_{g_i}$.
Indeed, there exists a cover of $U$ by affine opens $V_\alpha\subset U$ such that, if $A_\alpha=A(V_\alpha)$, $B_\alpha=A(f^{-1}(V_\alpha))$ is an integral (resp. finite) algebra over $A_\alpha$.
Every $x\in U$ belongs to some $V_\alpha$, so there exists $g\in A$ such that $x\in D(g)\subset V_\alpha$;
if $g_\alpha$ is the image of $g$ in $A_\alpha$, then $A(D(g))=A_g=(A_\alpha)_{g_\alpha}$;
let $h=\vphi(g)$, and let $h_\alpha$ be the image of $g_\alpha$ in $B_\alpha$;
we have
\[
  A(D(h)) = B_h = (B_\alpha)_{h_\alpha}
\]
and, since $B_\alpha$ is integral (resp. finite) over $A_\alpha$, $(B_\alpha)_{h_\alpha}$ is integral (resp. finite) over $(A_\alpha)_{g_\alpha}$.
It now suffices to use the fact that $U$ is quasi-compact to obtain the desired cover.

If we suppose first of all that the $B_{h_i}$ are integral and finite over the $A_{g_i}$, then since $B_{h_i}$ can also be written as $B_{g_i}$ as an $A_{g_i}$-module, Lemma~\sref{II.6.1.4.1} shows that, in this case, $B$ is an $A$-module of finite type.

Now suppose only that each $B_{h_i}$ is integral over $A_{g_i}$;
let $b\in B$, and let $C$ be the sub-$A$-algebra of $B$ generated by $b$.
For all $i$, $C_{h_i}$ is the algebra over $A_{g_i}$ generated by $b/1$ in $B_{h_i}$;
it follows from the hypothesis that each $C_{h_i}$ is an $A_{g_i}$-module of finite type, and so \sref{II.6.1.4.1} $C$ is an $A$-module of finite type, which proves that $B$ is integral over $A$.
\end{proof}

\begin{proposition}[6.1.5]
\label{II.6.1.5}
\medskip\noindent
\begin{enumerate}
  \item[(i)] A closed immersion is finite (and \emph{a fortiori} integral).
  \item[(ii)] The composition of two finite (resp. integral) morphisms is finite (resp. integral).
  \item[(iii)] If $f:X\to Y$ is a finite (resp. integral) $S$-morphism, then $f_{(S')}:X_{(S')}\to Y_{(S')}$ is finite (resp. integral) for any base extension $S'\to S$.
  \item[(iv)] If $f:X\to Y$ and $g:X'\to Y'$ are finite (resp. integral) $S$-morphisms, then $f\times_S g:X\times_S Y\to X'\times_S Y'$ is finite (resp. integral).
  \item[(v)] If $f:X\to Y$ and $g:Y\to Z$ are morphisms such that $g\circ f$ is finite (resp. integral), if $g$ is separated, then $f$ is finite (resp. integral).
  \item[(vi)] If $f:X\to Y$ is a finite (resp. integral) morphism, then $f_\red$ is finite (resp. integral).
\end{enumerate}
\end{proposition}

\oldpage[II]{112}
\begin{proof}
By \sref[I]{I.5.5.12}, it suffices to prove (i), (ii), and (iii).
To prove that a closed immersion $X\to S$ is finite, we can restrict to the case where $S=\Spec(A)$, and everything then follows from noting that a quotient ring $A/\mathfrak{J}$ is a monogeneous $A$-module.
To prove that the composition of two finite (resp. integral) morphism $X\to Y$, $Y\to Z$ is finite (resp. integral), we can again assume that $Z$ (and thus $X$ and $Y$ \sref{II.1.3.4}) is affine, and then the claim is equivalent to saying that, if $B$ is a finite (resp. integral) $A$-algebra and $C$ a finite (resp. integral) $B$-algebra, then $C$ is a finite (resp. integral) $A$-algebra, which is immediate.
Finally, to prove (iii), we can restrict to the case where $S=Y$, since $X_{(S')}$ can be identified with $X\times_Y Y_{(S')}$ \sref[I]{I.3.3.11};
we can further suppose that $S=\Spec(A)$ and $S'=\Spec(A')$;
then $X$ is affine of ring $B$ \sref{II.1.3.4}, and $X_{(S')}$ affine of ring $A'\otimes_A B$, and it suffices to note that, if $B$ is a finite (resp. integral) $A$-algebra, then $A'\otimes_A B$ is a finite (resp. integral) $A'$-algebra.
\end{proof}

We also note that, if $X$ and $Y$ are $S$-preschemes that are finite (resp. integral) over $S$, then their \emph{sum} $X\sqcup Y$ is a finite (resp. integral) prescheme over $S$, since this reduces to showing that, if $B$ and $C$ are finite (resp. integral) $A$-algebras over $A$, then so too is $B\times C$.

\begin{corollary}[6.1.6]
\label{II.6.1.6}
If $X$ is an integral (resp. finite) prescheme over $S$, then, for every open $U\subset S$, $f^{-1}(U)$ is integral (resp. finite) over $U$.
\end{corollary}

\begin{proof}
This is a particular case of \sref{II.6.1.5}[(iii)].
\end{proof}

\begin{corollary}[6.1.7]
\label{II.6.1.7}
Let $f:X\to Y$ be a finite morphism.
Then, for all $y\in Y$, the fibre $f^{-1}(y)$ is a finite algebraic scheme over $\kres(y)$, and \emph{a fortiori} its underlying space is discrete and finite.
\end{corollary}

\begin{proof}
Indeed, as a $\kres(y)$-prescheme, $f^{-1}(y)$ can be identified with $X\times_Y\Spec(\kres(y))$ \sref[I]{I.3.6.1}, which is finite over $\Spec(\kres(y))$ \sref{II.6.1.5}[(iii)];
it is thus an affine scheme whose ring is an algebra of finite rank over $\kres(y)$ \sref{II.6.1.4}.
The corollary then follows from \sref[I]{I.6.4.4}.
\end{proof}

\begin{corollary}[6.1.8]
\label{II.6.1.8}
Let $X$ and $S$ be integral preschemes, and $f:X\to S$ a \emph{dominant} morphism.
If $f$ is integral (resp. finite) then the field $R(X)$ of rational functions on $X$ is algebraic (resp. algebraic of finite degree) over the field $R(S)$ of rational functions on $S$.
\end{corollary}

\begin{proof}
Let $s$ be the generic point of $S$;
the $\kres(s)$-prescheme $f^{-1}(s)$ is integral (resp. finite) over $\Spec(\kres(s))$ \sref{II.6.1.5}[(iii)] and contains, by hypothesis, the generic point $x$ of $X$;
since the local ring of $x$ in $f^{-1}(s)$ is equal to $\kres(x)$ \sref[I]{I.3.6.5}, and is thus a local ring of an integral (resp. finite) algebra over $\kres(s)$ \sref{II.6.1.4}, whence the corollary.
\end{proof}

\begin{remark}[6.1.9]
\label{II.6.1.9}
The hypothesis that $g$ be \emph{separated} is essential for the validity of \sref{II.6.1.5}[(v)]: if $Y$ is not separated over $S$, then the identity $1_Y$ is the composite morphism $Y\xrightarrow{\Delta_Y}Y\times_Z Y\xrightarrow{p_1}Y$, but $\Delta_Y$ is not an integral morphism, as follows from \sref{II.6.1.10}:
\end{remark}

\begin{proposition}[6.1.10]
\label{II.6.1.10}
Every integral morphism is universally closed.
\end{proposition}

\begin{proof}
Let $f:X\to Y$ be an integral morphism;
by \sref{II.6.1.5}[(iii)], it suffices to show that $f$ is \emph{closed}.
Let $Z$ be a closed subset of $X$;
\oldpage[II]{113}
then there exists a subprescheme of $X$ whose underlying space is $Z$ \sref[I]{I.5.4.1}, and it thus follows from \sref{II.6.1.5}[(i) and (ii)] that it suffices to prove that $f(X)$ is \emph{closed} in $Y$.
By \sref{II.6.1.5}[(vi)], we can suppose that $X$ and $Y$ are \emph{reduced};
further, if $T$ is the closed reduced subprescheme of $Y$ whose underlying space is $\overline{f(X)}$ \sref[I]{I.5.2.1} then we know that $f$ factors as $X\to T\xrightarrow{j}Y$, where $j$ is the injection morphism \sref[I]{I.5.2.2}, and since $j$ is separated \sref[I]{I.5.5.1}[(i)], it follows from \sref{II.6.1.5}[(v)] that $g$ is an integral morphism.
We can thus suppose that $f(X)$ is \emph{dense} in $Y$.
Finally, since the question is local on $Y$, we can restrict to the case where $Y=\Spec(A)$.
Then $X=\Spec(B)$, where $B$ is an $A$-algebra that is integral over $A$ \sref{II.6.1.4};
furthermore, $A$ is reduced \sref[I]{I.5.1.4}, and the hypothesis that $f(X)$ be dense in $Y$ implies that the homomorphism $\vphi:A\to B$ corresponding to $f$ is \emph{injective} \sref[I]{I.1.2.7}.
Under these conditions, saying that $f(X)=Y$ implies that every prime ideal of $A$ is the intersection with $A$ of a prime ideal of $B$, which is exactly the first theorem of Cohen--Seidenberg (\cite[t.~I, p.~257, th.~3]{II-13}).
\end{proof}

\begin{corollary}[6.1.11]
\label{II.6.1.11}
Every finite morphism $f:X\to Y$ is projective.
\end{corollary}

\begin{proof}
Since $f$ is affine, $\sh{O}_X$ is a \emph{very ample} $\sh{O}_X$-module with respect to $f$ \sref{II.5.1.2};
furthermore, $f_*(\sh{O}_X)$ is a quasi-coherent $\sh{O}_Y$-module \emph{of finite type} \sref{II.6.1.2};
finally, $f$ is separated, of finite type, and universally closed \sref{II.6.1.10}, and thus satisfies the conditions of criterion~\sref{II.5.5.4}[(i)].
\end{proof}

\begin{proposition}[6.1.12]
\label{II.6.1.12}
Let $f:X'\to X$ be a finite morphism, and let $\sh{B}=f_*(\sh{O}_{X'})$ (which is a quasi-coherent $\sh{O}_X$-algebra, and a $\sh{O}_X$-module of finite type).
Let $\sh{F}'$ be a quasi-coherent $\sh{O}_{X'}$-module;
for $\sh{F}'$ to be locally free of rank~$r$, it is necessary and sufficient that $f_*(\sh{F}')$ be a locally free $\sh{B}$-module of rank~$r$.
\end{proposition}

\begin{proof}
It is clear that, if $f_*(\sh{F}')|U$ is isomorphic to $\sh{B}^r|U$ (where $U\subset X$ is open), then $\sh{F}'|f^{-1}(U)$ is isomorphic to $\sh{O}_{X'}^r|f^{-1}(U)$ \sref{II.1.4.2}.
Conversely, suppose that $\sh{F}'$ is locally free of rank~$r$; we will show that $f_*(\sh{F}')$ is locally isomorphic to $\sh{B}^r$ as a $\sh{B}$-module.
Let $x$ be a point of $X$;
as $U$ runs over a fundamental system of affine neighbourhoods of $x$, $f^{-1}(U)$ runs over a fundamental system of affine neighbourhoods \sref{II.1.2.5} of the finite set $f^{-1}(x)$, since $f$ is closed \sref{II.6.1.10}.
The proposition then follows from the following lemma:
\end{proof}

\begin{lemma}[6.1.12.1]
\label{II.6.1.12.1}
Let $Y$ be a prescheme, $\sh{E}$ a locally free $\sh{O}_Y$-module of rank~$r$, and $Z$ a finite subset of $Y$ contained inside some affine open $V$.
Then there exists a neighbourhood $U\subset V$ of $Z$ such that $\sh{E}|U$ is isomorphic to $\sh{O}_Y^r|U$.
\end{lemma}

\begin{proof}
We can evidently assume that $Y$ is affine;
for all $z_i\in Z$, there exists in the closure $\overline{\{z_i\}}$ at least one closed point $z'_i$ \sref[0]{0.2.1.3};
if $Z'$ is the set of the $z'_i$ then every neighbourhood of $Z'$ is a neighbourhood of $Z$, and we can thus assume that $Z$ is discrete and closed in $Y$.
Consider the closed reduced subprescheme of $Y$ that has $Z$ has its underlying space \sref[I]{I.5.2.1} and let $j:Z\to Y$ be the canonical injection;
$j^*(\sh{E})=\sh{E}\otimes_Y\sh{O}_Z$ is locally free of rank~$r$ on the discrete scheme $Z$, and is thus isomorphic to $\sh{O}_Z^r$;
in other words, there exist $r$ sections $s_i$ (for $1\leq i\leq r$) of $\sh{E}\otimes_Y\sh{O}_Z$ over $Z$ such that the homomorphism $\sh{O}_Z^r\to\sh{E}\otimes_Y\sh{O}_Z$ defined by these sections is bijective.
But $Y=\Spec(A)$ is affine, $Z$ is defined by an ideal $\mathfrak{J}$ of $A$, and we have $\sh{E}=\widetilde{M}$, where $M$ is an $A$-module;
the $s_i$ are elements of $M\otimes_A(A/\mathfrak{J})$ and are thus images of $r$ elements $t_i\in M=\Gamma(Y,\sh{E})$.
\oldpage[II]{114}
For all $z_j\in Z$ there is thus a neighbourhood $V_j$ of $z_j$ such that the restrictions of the $t_i$ to $V_j$ defined an isomorphism $\sh{O}_Y^r|V_j\to\sh{E}|V_j$ \sref[0]{0.5.5.4};
the neighbourhood $U$ given by the union of the $V_j$ is the desired neighbourhood.
\end{proof}

\begin{proposition}[6.1.13]
\label{II.6.1.13}
Let $g:X'\to X$ be an integral morphism of preschemes, $Y$ a normal locally integral prescheme, and $f$ a rational map from $Y$ to $X'$ such that $g\circ f$ is an everywhere defined rational map \sref[I]{I.7.2.1}.
Then $f$ is everywhere defined.
\end{proposition}

\begin{proof}
If $f_1$ and $f_2$ are morphisms (from dense open subsets of $Y$ to $X'$) in the class $f$, then it is clear that $g\circ f_1$ and $g\circ f_2$ are equivalent morphisms, which justifies the notation $g\circ f$ for their equivalence class.
We recall also that, if we further suppose $Y$ to be \emph{locally Noetherian}, then the hypothesis that $Y$ is normal already implies that $Y$ is locally integral \sref[I]{I.6.1.13}.

To prove the proposition, note first of all that the question is local on $Y$, and so we can suppose that there exists in the class $g\circ f$ a \emph{morphism} $h:Y\to X$.
Consider the inverse image $Y'=X'_{(h)}=X'_{(Y)}$, and note that the morphism $g'=g_{(Y)}:Y'\to Y$ is \emph{integral} \sref{II.6.1.5}[(iii)].
Given the correspondence between rational maps from $Y$ to $X'$ and rational $Y$-sections of $Y'$ \sref[I]{I.7.1.2}, we see that it suffices to prove the specific case of \sref{II.6.1.13} where $X=Y$; in other words, the following:
\end{proof}

\begin{corollary}[6.1.14]
\label{II.6.1.14}
Let $X$ be a normal locally integral prescheme, $g:X'\to X$ an \emph{integral} morphism, and $f$ a rational $X$-section of $X'$.
Then $f$ is everywhere defined.
\end{corollary}

\begin{proof}
Since the question is local on $X$, we can assume that $X$ is integral, and then $f$ is identified with a morphism from an open $U$ of $X$ to $X'$ \sref[I]{I.7.2.2} that is a $U$-section of $g^{-1}(U)$.
Since $g$ is separated, $f$ is a closed immersion from $U$ into $g^{-1}(U)$ \sref[I]{I.5.4.6};
let $Z$ be the closed subprescheme of $g^{-1}(U)$ associated to $f$ \sref[I]{I.4.2.1}, which is isomorphic to $U$, and thus integral;
let $X_1$ be the reduced subprescheme of $X'$ whose underlying space is the closure $\overline{Z}$ of $Z$ in $X'$ \sref[I]{I.5.2.1};
then $Z$ is an induced subprescheme on an open of $X_1$ \sref[I]{I.5.2.3}, and, since it is irreducible, so too is $X_1$, which is thus integral.
The morphism $f$ can then be considered as a rational $X$-section of $X_1$;
since the restriction of $g$ to $X_1$ is an integral morphism \sref{II.6.1.5}[(i) and (ii)], we can finally reduce to proving \sref{II.6.1.14} in the specific case where $X'=X_1$;
in other words, the following:
\end{proof}

\begin{corollary}[6.1.15]
\label{II.6.1.15}
Let $X$ be a normal integral prescheme, $X'$ an integral prescheme, and $g:X'\to X$ an \emph{integral} morphism.
If there exists a rational $X$-section $f$ of $X'$, then $g$ is an isomorphism.
\end{corollary}

\begin{proof}
Since the question is local on $X$, we can assume that $X$ is affine of integral ring $A$, and then $X'$ is affine of ring $A'$ with $A'$ integral over $A$ \sref{II.6.1.4} and integral;
furthermore, the argument of \sref{II.6.1.14} shows that there exists a dense open of $X$ that is isomorphic to a dense open of $X'$, and so $A$ and $A'$ have the same field of fractions.
Also, by \sref[I]{I.8.2.1.1}, and the hypothesis that the $\sh{O}_x$ are integrally closed, the ring $A$ is integrally closed, and so $A'=A$, which finishes the proof of \sref{II.6.1.13}
\end{proof}


\subsection{Quasi-finite morphisms}
\label{subsection:II.6.2}

\begin{proposition}[6.2.1]
\label{II.6.2.1}
Let $f:X\to Y$ be morphism locally of finite type, and $x$ a point of $X$.
Then the following conditions are equivalent:
\oldpage[II]{115}
\begin{enumerate}
  \item[a)] The point $x$ is isolated in its fibre $f^{-1}(f(x))$.
  \item[b)] The ring $\sh{O}_x$ is a quasi-finite $\sh{O}_{f(x)}$-module \sref[0]{0.7.4.1}.
\end{enumerate}
\end{proposition}

\begin{proof}
Since the question is clearly local on $X$ and on $Y$, we can assume that $X=\Spec(A)$ and $Y=\Spec(B)$ are affine, with $A$ a $B$-algebra of finite type \sref[I]{I.6.3.3}.
Furthermore, we can replace $X$ by $X\times_Y\Spec(\sh{O}_{f(x)})$ without changing either the fibre $f^{-1}(f(x))$ or the local ring $\sh{O}_x$ \sref[I]{I.3.6.5};
we can thus assume that $B$ is a local ring (equal to $\sh{O}_{f(x)}$);
if $\mathfrak{n}$ is the maximal ideal of $B$, then $f^{-1}(f(x))$ is an affine scheme of ring $A/\mathfrak{n}A$, of finite type over $\kres(f(x))=B/\mathfrak{n}$ \sref[I]{I.6.4.11}.
With this, if (a) is satisfied then we can further suppose that $f^{-1}(f(x))$ consists of the single point $x$;
thus $A/\mathfrak{n}A$ is of finite rank over $B/\mathfrak{n}$ \sref[I]{I.6.4.4}, or, in other words, $A$ is a quasi-finite $B$-module.
Conversely, if $A$ is a quasi-finite $B$-module, then $f^{-1}(f(x))$ is an Artinian affine scheme, and thus discrete \sref[I]{I.6.4.4};
so $x$ is isolated in its fibre, and this shows that (b) implies (a).
\end{proof}

\begin{corollary}[6.2.2]
\label{II.6.2.2}
Let $f:X\to Y$ be a morphism of finite type.
Then the following conditions are equivalent:
\begin{enumerate}
  \item[a)] Every point $x\in X$ is isolated in its fibre $f^{-1}(f(x))$ (or, in other words, the subspace $f^{-1}(f(x))$ is \emph{discrete}).
  \item[b)] For all $x\in X$, the prescheme $f^{-1}(f(x))$ is a finite $\kres(f(x))$-prescheme.
  \item[c)] For all $x\in X$, the ring $\sh{O}_x$ is a quasi-finite $\sh{O}_{f(x)}$-module.
\end{enumerate}
\end{corollary}

\begin{proof}
The equivalence of (a) and (c) follows from \sref{II.6.2.1}.
Since $f^{-1}(f(x))$ is an algebraic $\kres(f(x))$-prescheme \sref[I]{I.6.4.11}, the equivalence of (a) and (b) follows from \sref[I]{I.6.4.4}.
\end{proof}

\begin{definition}[6.2.3]
\label{II.6.2.3}
If $f:X\to Y$ is a morphism of finite type satisfying the equivalent conditions of \sref{II.6.2.2}, we say that $f$ is \emph{quasi-finite}, or that $X$ is \emph{quasi-finite over $Y$}.
\end{definition}

It is clear that every \emph{finite} morphism is quasi-finite \sref{II.6.1.8}.

\begin{proposition}[6.2.4]
\label{II.6.2.4}
\medskip\noindent
\begin{enumerate}
  \item[(i)] An immersion $X\to Y$ that is closed, or such that $X$ is Noetherian, is a quasi-finite morphism.
  \item[(ii)] If $f:X\to Y$ and $g:Y\to Z$ are quasi-finite morphisms, then $g\circ f$ is quasi-finite.
  \item[(iii)] If $X$ and $Y$ are $S$-preschemes, and $f:X\to Y$ a quasi-finite $S$-morphism, then $f_{(S')}:X_{(S')}\to Y_{(S')}$ is quasi-finite for any base extension $g:S'\to S$.
  \item[(iv)] If $f:X\to Y$ and $g:X'\to Y'$ are quasi-finite $S$-morphisms, then
    \[
      f\times_S g : X\times_S Y \to X'\times_S Y'
    \]
    is quasi-finite.
  \item[(v)] Let $f:X\to Y$ and $g:Y\to Z$ be morphisms such that $g\circ f$ is quasi-finite; if, further, $g$ is separated, or $X$ is Noetherian, or $X\times_Z Y$ is locally Noetherian, then $f$ is quasi-finite.
  \item[(vi)] If $f$ is quasi-finite, then $f_\red$ is quasi-finite.
\end{enumerate}
\end{proposition}

\begin{proof}
If $f:X\to Y$ is an immersion, then every fibre consists of a single point, and claim (i) then follows from (\sref[I]{I.6.3.4}[(i)] and \sref[I]{I.6.3.5}).
To prove (ii), note first of all that $h=g\circ f$ is of finite type \sref[I]{I.6.3.4}[(ii)];
furthermore, if $z=h(x)$ and $y=f(x)$, then $y$ is isolated in $g^{-1}(z)$, and so there exists an open neighbourhood $V$ of $y$ in $Y$ that does not meet any point of $g^{-1}(z)$ apart from $y$;
thus $f^{-1}(V)$ is an open neighbourhood of $x$ that does not meet any of the $f^{-1}(y')$, where $y'\neq y$ is in $g^{-1}(z)$;
\oldpage[II]{116}
since $x$ is isolated in $f^{-1}(y)$, it is isolated in $h^{-1}(z)=f^{-1}(g^{-1}(z))$.
To prove (iii), we can reduce to the case where $Y=S$ \sref[I]{I.3.3.11};
we again note first of all that $f'=f_{(S')}$ is of finite type \sref[I]{I.6.3.4}[(iii)];
also, if $x'\in X'=X_{(S')}$, and if we set $y'=f'(x')$ and $y=g(y')$, then ${f'}^{-1}(y')$ can be identified with $f^{-1}(y)\otimes_{\kres(y)}\kres(y')$ \sref[I]{I.3.6.5};
since $f^{-1}(y)$ is of finite rank over $\kres(y)$ by hypothesis, ${f'}^{-1}(y')$ is of finite rank over $\kres(y')$, and thus discrete.
Claims (iv), (v), and (vi) follows from (i), (ii), and (iii) by the general method \sref[I]{I.5.5.12}, except for when the hypotheses in (v) are not ``g is separated'';
in these cases, we remark first of all that, if $x$ is isolated in $f^{-1}(g^{-1}(g(f(x))))$, then it is \emph{a fortiori} isolated in $f^{-1}(f(x))$;
the fact that $f$ is of finite type then follows from \sref[I]{I.6.3.6}.
\end{proof}

\begin{proposition}[6.2.5]
\label{II.6.2.5}
Let $A$ be a \emph{complete} local Noetherian ring, $X$ an $A$-scheme locally of finite type, and $x$ a point of $X$ over the closed point $y$ of $Y=\Spec(A)$.
Suppose that $x$ is isolated in its fibre $f^{-1}(y)$ (where $f$ is the structure morphism $X\to Y$).
Then $\sh{O}_x$ is an $A$-module of finite type, and $X$ is $Y$-isomorphic to the sum \sref[I]{I.3.1} of $X'=\Spec(\sh{O}_x)$ (which is a finite $Y$-scheme) and an $A$-scheme $X''$.
\end{proposition}

\begin{proof}
It follows from \sref{II.6.2.1} that $\sh{O}_x$ is a quasi-finite $A$-module.
Since $\sh{O}_x$ is Noetherian \sref[I]{I.6.3.7}, and the homomorphism $A\to\sh{O}_x$ is local, the hypothesis that $A$ is \emph{complete} implies that $\sh{O}_x$ is an $A$-module \emph{of finite type} \sref[0]{0.7.4.3}.
Let $X'=\Spec(\sh{O}_x)$ be the local scheme of $X$ at the point $x$ \sref[I]{I.2.4.1}, and let $g:X'\to X$ be the canonical morphism.
Since the composite $X'\xrightarrow{g}X\xrightarrow{f}Y$ is finite \sref{II.6.1.1}, and since $f$ is separated, $g$ is finite \sref{II.6.1.5}[(v)], and so $g(X')$ is \emph{closed} in $X$ \sref{II.6.1.10};
also, since $g$ is of finite type, $g$ is a local isomorphism at the closed point $x'$ of $X'$ by the definition of $g$ and \sref[I]{I.6.5.4};
but since $X'$ is the only open neighbourhood of $x'$, this implies that $g$ is an open immersion, and so $g(X')$ is also \emph{open} in $X$, which finishes the proof.
\end{proof}

\begin{corollary}[6.2.6]
\label{II.6.2.6}
Let $A$ be a \emph{complete} local Noetherian ring, $Y=\Spec(A)$, and $f:X\to Y$ a separated quasi-finite morphism.
Then $X$ is $Y$-isomorphic to a sum $X'\sqcup X''$, where $X'$ is a finite $Y$-scheme, and $X''$ is a quasi-finite $Y$-scheme such that, if $y$ is the closed point of $Y$, then $X''\cap f^{-1}(y)=\emp$.
\end{corollary}

\begin{proof}
Indeed, the fibre $f^{-1}(y)$ is finite and discrete by hypothesis, and the corollary then follows, by induction on the number of points in this fibre, from \sref{II.6.2.5}.
\end{proof}

\begin{remark}[6.2.7]
\label{II.6.2.7}
In Chapter~V, we will see that, if $Y$ is \emph{locally Noetherian}, then a separated quasi-finite morphism $X\to Y$ is necessarily \emph{quasi-affine}.
\end{remark}


\subsection{Integral closure of a prescheme}
\label{subsection:II.6.3}

\begin{proposition}[6.3.1]
\label{II.6.3.1}
Let $(X,\sh{A})$ be a ringed space, $\sh{B}$ a (commutative) $\sh{A}$-algebra, and $f$ a section of $\sh{B}$ over $X$.
Then the following properties are equivalent:
\begin{enumerate}
  \item[a)] The sub-$\sh{A}$-module of $\sh{B}$ generated by the $f^n$ for $n\geq$ \sref[0]{0.5.1.1} is of finite type.
  \item[b)] There exists a sub-$\sh{A}$-algebra $\sh{C}$ of $\sh{B}$ that is an $\sh{A}$-module of finite type and such that $f\in\Gamma(X,\sh{C})$.
  \item[c)] For all $x\in X$, $f_x$ is integral over the fibre $\sh{A}_x$.
\end{enumerate}
\end{proposition}

\oldpage[II]{117}
\begin{proof}
Since the sub-$\sh{A}$-module of $\sh{B}$ generated by the $f^n$ is an $\sh{A}$-algebra, it is clear that (a) implies (b).
On the other hand, (b) implies that, for all $x\in X$, the $\sh{A}_x$-module $\sh{C}_x$ is of finite type, which implies that every element of the algebra $\sh{C}_x$, and, in particular, $f_x$, is integral over $\sh{A}_x$.
Finally, if we have, for some $x\in X$, a relation of the form
\[
  f_x^n + (a_1)_xf_x^{n-1} + \ldots + (a_n)_x = 0
\]
where the $a_i$ (for $1\leq i\leq n$) are sections of $\sh{A}$ over an open neighbourhood $U$ of $x$, then the section $f^n|U+a_1\cdot f^{n-1}|U+\ldots+a_n$ is zero over a neighbourhood $V\subset U$ of $x$, from which it immediately follows that the $f^k|V$ (for $k\geq0$) are linear combinations of the $f^j|V$ (for $0\leq j\leq n-1$) with coefficients in $\Gamma(V,\sh{A})$;
we thus conclude that (c) implies (a).
\end{proof}

If the equivalent conditions of \sref{II.6.3.1} are satisfied then we say that the section $f$ is \emph{integral over $\sh{A}$}.

\begin{corollary}[6.3.2]
\label{II.6.3.2}
Under the hypotheses of \sref{II.6.3.1}, there exists a (unique) sub-$\sh{A}$-module $\sh{A}'$ of $\sh{B}$ such that, for all $x\in X$, $\sh{A}'_x$ is the set of germs $f_x\in\sh{B}_x$ that are integral over $\sh{A}_x$.
For every open $U\subset X$, the sections of $\sh{A}'$ over $U$ are the sections $f\in\Gamma(U,\sh{B})$ that are integral over $\sh{A}|U$.
\end{corollary}

\begin{proof}
The existence of $\sh{A}'$ is immediate, by taking $\Gamma(U,\sh{A}')$ to be the set of $f\in\Gamma(U,\sh{B})$ such that $f_x$ is integral over $\sh{A}_x$ for all $x\in U$.
The second claim follows immediately from \sref{II.6.3.1}.
\end{proof}

It is clear that $\sh{A}'$ is a sub-$\sh{A}$-algebra of $\sh{B}$;
we say that it is the \emph{integral closure of $\sh{A}$ in $\sh{B}$}.

\begin{env}[6.3.3]
\label{II.6.3.3}
Let $(X,\sh{A})$ and $(Y,\sh{B})$ be ringed spaces, and let
\[
  g = (\psi,\theta):X\to Y
\]
be a morphism.
Let $\sh{C}$ (resp. $\sh{D}$) be an $\sh{A}$-algebra (resp. $\sh{B}$-algebra), and let
\[
  u:\sh{D}\to\sh{C}
\]
be a $g$-morphism \sref[0]{0.4.4.1}.
Then, if $\sh{A}'$ (resp. $\sh{B}'$) is the integral closure of $\sh{A}$ (resp. $\sh{B}$) in $\sh{C}$ (resp. $\sh{D}$), the \emph{restriction} of $u$ to $\sh{B}'$ is a $g$-morphism
\[
  u':\sh{B}'\to\sh{A}'.
\]

Indeed, if $j$ is the canonical injection $\sh{B}'\to\sh{D}$, then it suffices to show that
\[
  v = u^\sharp\circ g^*(j) : j^*(\sh{B}')\to\sh{C}'
\]
sends $g^*(\sh{B}')$ to $\sh{A}'$.
But an element of $g^*(\sh{B}')_x=\sh{B}'_{\psi(x)}\otimes_{\sh{B}_{\psi(x)}}\sh{A}_x$ is integral over $\sh{A}_x$ by the definition of $\sh{B}'$, and thus so too is its images under $v_x$, which proves the claim.
\end{env}

\begin{proposition}[6.3.4]
\label{II.6.3.4}
Let $X$ be a prescheme, and $\sh{A}$ a quasi-coherent $\sh{O}_X$-algebra.
Then the integral closure $\sh{O}'_X$ of $\sh{O}_X$ in $\sh{A}$ is a quasi-coherent $\sh{O}_X$-algebra, and, for every affine open $U$ of $X$, $\Gamma(U,\sh{O}'_X)$ is the integral closure of $\Gamma(U,\sh{O}_X)$ in $\Gamma(U,\sh{A})$.
\end{proposition}

\begin{proof}
We can restrict to the case where $X=\Spec(B)$ is affine, and $\sh{A}=\widetilde{A}$, where $A$ is a $B$-algebra;
\oldpage[II]{118}
let $B'$ be the integral closure of $B$ in $A$.
Everything then reduces to proving that, for all $x\in X$, an element of $A_x$ which is integral over $B_x$ necessarily belongs to $B'_x$, which follows from the fact that, for a commutative ring $C$, the operations of taking the integral closure in a $C$-algebra and passing to a ring of fractions (with respect to a multiplicative subset of $C$) commute \cite[t.~I, pp.~261 and 257]{I-13}.
\end{proof}

The $X$-scheme $X'=\Spec(\sh{O}'_X)$ is then called the \emph{integral closure of $X$ with respect to $\sh{A}$} (or \emph{with respect to $\Spec(\sh{A})$});
it is clear that $X'$ is \emph{integral} over $X$ \sref{II.6.1.2}.

We immediately deduce from \sref{II.6.1.4} that, if $f:X'\to X$ is the structure morphism, then, for every open $U$ of $X$, $f^{-1}(U)$ is the \emph{integral closure of the prescheme induced by $X$ on $U$, with respect to $\sh{A}|U$}.

\begin{env}[6.3.5]
\label{II.6.3.5}
Let $X$ and $Y$ be preschemes, $f:X\to Y$ a morphism, $\sh{A}$ (resp. $\sh{B}$) a quasi-coherent $\sh{O}_X$-algebra (resp. quasi-coherent $\sh{O}_Y$-algebra), and let $u:\sh{B}\to\sh{A}$ be an $f$-morphism.
We have seen \sref{II.6.3.3} that we obtain an $f$-morphism $u':\sh{O}'_Y\to\sh{O}'_X$, where $\sh{O}'_X$ (resp. $\sh{O}'_Y$) is the integral closure of $\sh{O}_X$ (resp. $\sh{O}_Y$) in $\sh{A}$ (resp. $\sh{B}$).
Then, if $X'$ (resp. $Y'$) is the integral closure of $X$ (resp. $Y$) with respect to $\sh{A}$ (resp. $\sh{B}$), we canonically obtain from $u$ a morphism $f'=\Spec(u'):X'\to Y'$ \sref{II.1.5.6} such that the diagram
\[
\label{II.6.3.5.1}
  \xymatrix{
    X' \ar[r]^{f'} \ar[d]
    & Y' \ar[d]
  \\X \ar[r]_{f}
    & Y
  }
\tag{6.3.5.1}
\]
commutes.
\end{env}

\begin{env}[6.3.6]
\label{II.6.3.6}
Suppose that $X$ has only a \emph{finite} number of irreducible components $X_i$ (for $1\leq i\leq r$), with generic points $\xi_i$, and consider, in particular, the integral closure of $X$ with respect to some quasi-coherent \emph{$\sh{R}(X)$-algebra} $\sh{A}$ (quasi-coherent as either an $\sh{O}_X$-algebra or as an $\sh{R}(X)$-algebra, since the two are equivalent).
We know \sref[I]{I.7.3.5} that $\sh{A}$ is a direct sum of $r$ quasi-coherent $\sh{O}_X$-algebras $\sh{A}_i$, where the support of $\sh{A}_i$ is contained inside $X_i$, and the sheaf induced by $\sh{A}_i$ on $X_i$ is a simple sheaf whose fibre $A_i$ is an algebra over $\sh{O}_{\xi_i}$.
It is then clear \sref{II.6.3.4} that the integral closure $\sh{O}'_X$ of $\sh{O}_X$ in $\sh{A}$ is the \emph{direct sum} of the integral closures $\sh{O}_X^{(i)}$ of $\sh{O}_X$ in each of the $\sh{A_i}$, and thus the integral closure $X'=\Spec(\sh{O}'_X)$ of $X$ with respect to $\sh{A}$ is the $X$-scheme given by the \emph{sum} of the $\Spec(\sh{O}_X^{(i)})=X'_i$ (for $1\leq i\leq r$).

Suppose further that the $\sh{O}_X$-algebra $\sh{A}$ is \emph{reduced}, or, equivalently, that each of the algebras $A_i$ are reduced, and can thus be considered as an algebra over the \emph{field} $\kres(\xi_i)$ (equal to the field of rational functions of the reduced subprescheme of $X$ that has $X_i$ as its underlying space);
then \sref{II.1.3.8} each of the $X'_i$ is a \emph{reduced} $X$-scheme, and $X'$ is also the integral closure of $X_\red$.
Suppose further that each of the algebras $A_i$ is a \emph{direct sum of a finite number of fields $K_{ij}$} (for $1\leq j\leq s_i$);
if $\sh{K}_{ij}$ is the sub-algebra of $\sh{A}_i$ corresponding to $K_{ij}$, then it is clear that $\sh{O}_X^{(i)}$ is the \emph{direct sum} of the integral closures $\sh{O}_X^{(ij)}$ of $\sh{O}_X$ in each of the $\sh{K}_{ij}$.
Then $X'_i$ is the \emph{sum} of the $X$-schemes $X'_{ij}=\Spec(\sh{O}_X^{(ij)})$ (for $1\leq j\leq s_i$).
Furthermore, under these hypotheses, and with this notation, we have:
\end{env}

\oldpage[II]{119}
\begin{proposition}[6.3.7]
\label{II.6.3.7}
Each of the $X'_{ij}$ is an integral normal $X$-prescheme, and its field of rational functions $R(X'_{ij})$ is canonically identified with the algebraic closure $K'_{ij}$ of $\kres(\xi_i)$ in $K_{ij}$.
\end{proposition}

\begin{proof}
By the above, we can suppose that $X$ is integral, and so $r=1$ and $s_1=1$, so that the unique algebra $A_1$ is a field $K$;
let $\xi$ be the generic point of $X$, and let $f:X\to X'$ be the structure morphism.
For every non-empty affine open $U$ of $X$, $f^{-1}(U)$ can be identified with the spectrum of the integral closure $B'_U$ in the field $K$ of the integral ring $B_U=\Gamma(Y,\sh{O}_X)$ \sref{II.6.3.4};
since the ring $B'_U$ is integral and integrally closed, so too are the local rings of the points of its spectrum, and thus $f^{-1}(U)$ is by definition an \emph{integral} and \emph{normal} scheme (\sref[0]{0.4.1.4} and \sref[I]{I.5.1.4}).
Furthermore, since $(0)$ is the only prime ideal of $B'_U$ over the prime ideal $(0)$ of $B_U$ \cite[t.~1, p.~259]{I-13}, $f^{-1}(\xi)$ consists of a single point $\xi'$, and $\kres(\xi')$ is the field of fractions $K'$ of $B'_U$, which is exactly the algebraic closure of $\kres(\xi)$ in $K$.
Finally, $X'$ is irreducible, since, if $U$ runs over the set of non-empty affine opens of $X$, then the $f^{-1}(U)$ give an open cover of $X'$ by irreducible opens;
furthermore, the intersection $f^{-1}(U\cap V)$ of any two of these opens contains $\xi'$, and is thus non-empty, and we conclude by \sref[0]{0.2.1.4}.
\end{proof}

\begin{corollary}
Let $X$ be a reduced prescheme that has only a finite number of irreducible components $X_i$ (where $1\leq i\leq r$), and let $\xi_i$ be the generic point of $X_i$.
Then the integral closure $X'$ of $X$ with respect to $\sh{R}(X)$ is the sum of the $r$ $X$-schemes $X'_i$, which are integral and normal.
If $f:X'\to X$ is the structure morphism, then $f^{-1}(\xi_i)$ consists only of the generic point $\xi'_i$ of $X'_i$, and we have $\kres(\xi'_i)=\kres(\xi_i)$, or, in other words, that $f$ is birational.
\end{corollary}

In this case, we say that $X'$ is the \emph{normalisation} of the reduced prescheme $X$;
we note that $f$, since it is birational and integral, is \emph{surjective} \sref{II.6.1.10}.
Then in order to have $X'=X$, it is necessary and sufficient that $X$ be \emph{normal}.
If $X$ is an integral prescheme, then it follows from \sref{II.6.3.8} that its normalisation $X'$ is integral.

\begin{env}[6.3.9]
\label{II.6.3.9}
Let $X$ and $Y$ be integral preschemes, $f:X\to Y$ a dominant morphism, and $L=R(X)$ (resp. $K=R(Y)$) the field of rational functions of $X$ (resp. $Y$);
there is a canonical injection $K\to L$ corresponding to $f$, and if we identify $K$ (resp. $L$) with the simple sheaf $\sh{R}(Y)$ (resp. $\sh{R}(X)$), this injection is an $f$-morphism.
Let $K_1$ (resp. $L_1$) be an extension of $K$ (resp. $L$), and suppose we have a morphism $K_1\to L_1$ such that the diagram
\[
  \xymatrix{
    K_1 \ar[r]
    & L_1
  \\K \ar[u] \ar[r]
    & L \ar[u]
  }
\]
commutes;
if $K_1$ (resp. $L_1$) is considered as a simple sheaf on $Y$ (resp. $X$), and thus as an $\sh{R}(Y)$-algebra (resp. $\sh{R}(X)$-algebra), this implies that $K_1\to L_1$ is an $f$-morphism.
With this, if $X'$ (resp. $Y'$) is the integral closure of $X$ (resp. $Y$) with respect to $L_1$ (resp. $K_1$), then $X'$ (resp. $Y'$) is a normal integral prescheme \sref{II.6.3.6} whose field of rational functions is canonically identified with the algebraic closure $L'$ (resp. $K'$) of $L$ (resp. $K$) in $L_1$ (resp $K_1$), and there exists a (necessarily dominant) canonical morphism $f':X'\to Y'$ that makes the diagram \sref{II.6.3.5.1} commute.
\oldpage[II]{120}

The most important case is when we take $L_1=L$, with $K_1$ then being an extension of $K$ contained in $L$, and when we suppose $X$ to be integral and \emph{normal}, so that $X'=X$.
The above then shows that, since $X$ is normal, if $Y'$ is the integral closure of $Y$ with respect to a field $K_1\subset L=R(X)$, then every dominant morphism $f:X\to Y$ factors as
\[
  f:X\xrightarrow{f'}Y'\to Y
\]
where $f'$ is dominant;
also, since the monomorphism $K_1\to L$ is given, $f'$ is necessarily unique, as we see by reducing to the case where $X$ and $Y$ are affine.
We thus see that, given $Y$, $L$, and a $K$-monomorphism $K_1\to L$, the integral closure $Y'$ of $Y$ with respect to $K_1$ is the solution of a \emph{universal problem}.
\end{env}

\begin{remark}[6.3.10]
\label{II.6.3.10}
Consider the hypothesis of \sref{II.6.3.6}, and suppose further that each of the algebras $A_i$ is \emph{of finite rank} over $\kres(\xi_i)$ (which implies that $A_i$ is a direct sum of a finite number of fields);
we can then show, in certain cases, that the structure morphism $X'\to X$ is not just integral, but even \emph{finite}.
Restrict to the case where $X$ is \emph{reduced};
since the question is local on $X$, we can further suppose that $X$ is affine of ring $C$, and that $C$ has only a finite number of minimal ideals $\mathfrak{p}_i$ (for $1\leq i\leq r$) such the that $C_i=C/\mathfrak{p}_i$ are integral;
then $X'$ is finite over $X$ if the integral closure of each $C_i$ in finite-dgree extension of its field of fractions is a \emph{$C$-module of finite type} \sref{II.6.3.4}.
We know that this condition is always satisfied if $C$ is an \emph{algebra of finite type over a field} \cite[t.~I, p.~267, th.~9]{I-13}, or \emph{over $\bb{Z}$} \cite[I,p.~93, th.~3]{I-9}, or \emph{over a complete Noetherian local ring} \cite[p.~298]{II-25}.
We thus conclude that $X'\to X$ will be a finite morphism whenever $X$ is a scheme \emph{of finite type} over a field, or over $\bb{Z}$, or over a complete Noetherian local ring.
\end{remark}


\subsection{Determinant of an endomorphism of $\mathcal{O}_X$-modules}
\label{subsection:II.6.4}

\begin{env}[6.4.1]
\label{II.6.4.1}
Let $A$ be a (commutative) ring, $E$ a free $A$-module of rank~$n$, and $u$ an endomorphism of $E$;
recall that, to define the \emph{characteristic polynomial} of $u$, we consider the endomorphism $u\otimes1$ of the free $A[T]$-module of rank~$n$, $E\otimes_A A[T]$ (where $T$ is an indeterminate), and we set
\[
\label{II.6.4.1.1}
  P(u,T) = \det(T\cdot I-(u\otimes 1))
\tag{6.4.1.1}
\]
(where $I$ is the identity automorphism of $E\otimes_A A[T]$).
We have
\[
\label{II.6.4.1.2}
  P(u,T) = T^n - \sigma_1(u)T^{n-1} + \ldots + (-1)^n\sigma_n(u)
\tag{6.4.1.2}
\]
where $\sigma_i(u)$ is an element of $A$, equal to a homogeneous polynomial of degree~$i$ (with integer coefficients) with respect to the entries of the matrix of $u$ in some arbitrary basis of $E$;
we say that the $\sigma_i(u)$ are the \emph{elementary symmetric functions} of $u$;
in particular, we have that $\sigma_1(u)=\Tr u$ and $\sigma_n(u)=\det u$.
Recall that, by the Hamilton--Cayley theorem, we have
\[
\label{II.6.4.1.3}
  P(u,u) = u^n - \sigma_1(u)u^{n-1} + \ldots + (-1)^n\sigma_n(u) = 0
\tag{6.4.1.3}
\]
\oldpage[II]{121}
which can also be written as
\[
\label{II.6.4.1.4}
  (\det u)\cdot I_E = uQ(u)
\tag{6.4.1.4}
\]
(where $I_E$ is the identity automorphism of $E$), where
\[
\label{II.6.4.1.5}
  Q(u) = (-1)^{n+1}(u^{n-1}-\sigma_1(u)u^{n-2}+\ldots+(-1)^{n-1}\sigma_{n-1}(u)).
\tag{6.4.1.5}
\]

Let $\vphi:A\to B$ be a ring homomorphism, which makes $B$ into an $A$-algebra;
consider the $B$-module $E_{(B)}=E\otimes_A B$, which is free of rank~$n$, and the extension $u\otimes1$ of $u$ to an endomorphism of $E_{(B)}$;
it is immediate that we have $\sigma_i(u\otimes1)=\vphi(\sigma_i(u))$ for all indices $i$.
\end{env}

\begin{env}[6.4.2]
\label{II.6.4.2}
Now suppose that $A$ is an \emph{integral} ring, with field of fractions $K$, and let $E$ be an $A$-module \emph{of finite type} (but not necessarily free).
Let $n$ be the \emph{rank} of $E$, i.e. the rank of the free $K$-module $E\otimes_A K$;
to every endomorphism $u$ of $E$ there canonically corresponds the endomorphism $u\otimes1$ of $E\otimes_A K$;
by an abuse of language, we also define the \emph{characteristic polynomial} of $u$, denoted by $P(u,T)$, to be the polynomial $P(u\otimes1,T)$, whose coefficients $\sigma_i(u\otimes1)$ (which belong to $K$) are also denoted by $\sigma_i(u)$ and called the \emph{elementary symmetric functions of $u$};
in particular, $\det u=\det(u\otimes1)$ by definition.
With this notation, Equations~\sref{II.6.4.1.3} to \sref{II.6.4.1.5} make sense and still hold true, if we interpret $u^j$ as the composite homomorphism $E\to E\otimes_A K$ of the endomorphism $u^j\otimes1=(u\otimes1)^j$ of $E\otimes_A K$ and the canonical homomorphism $x\mapsto x\otimes1$.

If $F$ is the torsion submodule of $E$, and if $E_0=E/F$, then $u(F)\subset F$, and so, by passing to quotients, $u$ gives an endomorphism $u_0$ of $E_0$;
furthermore, $E\otimes_A K$ can be identified with $E_0\otimes_A K$, and $u\otimes1$ with $u_0\otimes1$, and so $\sigma_i(u)=\sigma_i(u_0)$ for $1\leq i\leq n$.

If $E$ is \emph{torsion free}, then $E$ is canonically identified with a sub-$A$-module of $E\otimes_A K$, and the relation $u\otimes1=0$ is equivalent to $u=0$.
If $A$ is a free $A$-module, then the two definitions of $\sigma_i(u)$ given in \sref{II.6.4.1} and \sref{II.6.4.2} coincide, by the above, which justifies the notation used.

We also note that, if $E$ is a \emph{torsion module}, or, in other words, if $E_0=\{0\}$, then the exterior algebra of $E_0$ consists of $K$, and the determinant of the unique endomorphism $u_0$ of $E_0$ is \emph{equal to $1$}.
\end{env}

\begin{proposition}[6.4.3]
\label{II.6.4.3}
Let $A$ be an integral ring, $E$ an $A$-module of finite type, and $u$ an endomorphism of $E$;
then the elementary symmetric functions $\sigma_i(u)$ of $u$ (and, in particular, $\det u$) are elements of $K$ that are integral over $A$.
\end{proposition}

\begin{proof}
Let $A'$ be the integral closure of $A$;
since $A'[T]$ is an integrally closed ring \cite[p.~99]{II-24}, it is the integral closure of $A[T]$ in its field of fractions $K(T)$.
Replacing $u$ by $T\cdot I-u\otimes1$, and $A$ by $A[T]$, we see that we can reduce to proving that $u$ is integral over $A$.
If $n$ is the rank of $E$, then $\det(u)=\det(\vee^n u)$ and $(\vee^n u\otimes1)=\vee^n(u\otimes1)$, so we can suppose that $n=1$.
But then the map $u\mapsto\det u$ is a homomorphism from the $A$-module $\Hom_A(E,E)$ to $K$;
since $E$ is of finite type, $\Hom_A(E,E)$ is isomorphic to a sub-$A$-module of the $A$-module of finite type $E^n$ (if $E$ admits a system of $n$ generators), and so the elements $\det u$ belong to a sub-$A$-module of $K$ of finite type, and are thus integral over $A$.
\end{proof}

\oldpage[II]{122}
\begin{corollary}[6.4.4]
\label{II.6.4.4}
Under the hypotheses of \sref{II.6.4.3}, if we further suppose $A$ to be normal, then the $\sigma_i(u)$ (and, in particular, $\Tr u$ and $\det u$) belong to $A$.
\end{corollary}

\begin{proposition}[6.4.5]
\label{II.6.4.5}
Let $A$ be an integral ring, $E$ an $A$-module of finite type of rank~$n$, and $u$ an endomorphism of $E$ such that the $\sigma_i(u)$ belong to $A$.
For $u$ to be an automorphism of $E$, it is necessary that $\det u$ be invertible in $A$;
this condition is sufficient if $E$ is torsion free.
\end{proposition}

\begin{proof}
The condition is \emph{sufficient} when $E$ is torsion free, since the hypothesis and equations~\sref{II.6.4.1.4} and \sref{II.6.4.1.5} (which hold in $E$, and not just in $E\otimes_A K$, since $E$ is torsion free) prove that $(\det u)^{-1}Q(u)$ is the inverse of $u$.

The condition is \emph{necessary}, since, if $u$ is invertible, then it follows from \sref{II.6.4.3} that $\det(u^{-1})$ belongs to the integral closure $A'$ of $A$ in its field of fractions $K$, and is evidently the inverse of $\det u$ \emph{in $A'$}.
Our claim then follows from:
\end{proof}

\begin{lemma}[6.4.5.1]
\label{II.6.4.5.1}
Let $A$ be a subring of a ring $A'$ such that $A'$ is \emph{integral} over $A$.
If an element $x\in A$ is invertible in $A'$, then it is also invertible in $A$.
\end{lemma}

\begin{proof}
In the contrary case, $x$ would belong to a maximal ideal $\mathfrak{m}$ of $A$, and it follows from the first theorem of Cohen--Seidenberg \cite[t.~I, p.~257, th.~3]{I-13} that there would be a maximal ideal $\mathfrak{m}'$ of $A'$ such that $\mathfrak{m}=\mathfrak{m}'\cap A$;
we would then have that $x\in\mathfrak{m}'$, which is a contradiction.
\end{proof}

\begin{corollary}[6.4.6]
\label{II.6.4.6}
Let $A$ be an integral and integrally closed ring, $E$ a torsion-free $A$-module of finite type, and $u$ an endomorphism of $E$>
For $u$ to be an automorphism of $E$, it is necessary and sufficient that $\det u$ be invertible in $A$.
\end{corollary}

\begin{proof}
This follows from \sref{II.6.4.4} and \sref{II.6.4.5}.
\end{proof}

\begin{remark}[6.4.7]
\label{II.6.4.7}
We will later need a generalisation of the above results.
Consider a \emph{reduced} Noetherian ring $A$;
let $\mathfrak{p}_\alpha$ (for $1\leq\alpha\leq r$) be its minimal ideals, $K_\alpha$ the field of fractions of the integral ring $A/\mathfrak{p}_\alpha$, and $K$ the total ring of fractions of $A$, which is the \emph{direct sum} of the fields $K_\alpha$.
Let $E$ be an $A$-module of finite type, and \emph{suppose} that $E\otimes_A K$ is a \emph{free} $K$-module of rank~$n$ (which here is merely a consequence of the other hypotheses);
equivalently, we can ask that all the $K_\alpha$-vector spaces $E\otimes_A K_\alpha=E_\alpha$ have \emph{the same dimension~$n$};
if then $u$ is an endomorphism of $E$, we again set $P(u,T)=P(u\otimes1,T)$ and $\sigma_j(u)=\sigma_j(u\otimes1)$, and, in particular, $\det u=\det(u\otimes1)$;
the $\sigma_j(u)$ are thus elements of $K$.
It is immediate that $E\otimes_A K$ is the direct sum of the $E_\alpha$, and that the latter are stable under $u\otimes1$;
the restriction of $u\otimes1$ to $E_\alpha$ is exactly the extension $u_\alpha$ of $u$ to this $K_\alpha$-vector space;
we thus conclude that $\sigma_j(u)$ is the element of $K$ who components in the $K_\alpha$ are the $\sigma_j(u_\alpha)$.
Since the integral closure of $A$ in $K$ is the \emph{direct sum} of the integral closures of $A$ in the $K_\alpha$, the $\sigma_j(u)$ are \emph{integral} over $A$.

\begin{lemma}[6.4.7.1]
\label{II.6.4.7.1}
The sub-$A$-algebra of $K$ generated by all the elements $\sigma_j(u)$ (for $1\leq j\leq n$), where $u$ runs over $\Hom_A(E,E)$, is an $A$-module of finite type.
\end{lemma}

\begin{proof}
It suffices to prove that the sub-$A[T]$-algebra of $K[T]$ generated by the $P(u,T)$ is an $A[T]$-module of finite type, since if the $F_i(T)$ (for $1\leq i\leq m$) form a system of generators for this $A[T]$-module, then the coefficients of the $F_i(T)$ are integral over $A$, and thus generate an $A$-algebra that is an $A$-module of finite type \cite[t.~I, p.~255, th.~1]{I-13}.
\oldpage[II]{123}
We can thus replace $A$ by $A[T]$ (which is Noetherian), and $E$ by $E\otimes_A A[T]=E'$, which is such that $E'\otimes_{A[T]}K[T]=E\otimes_A K[T]$ is a free $K[T]$-module of rank~$n$.
Using the initial notation, we thus see that it suffices to prove that the $A$-module generated by the elements $\det u$, where $u$ runs over $\Hom_A(E,E)$, is of finite type;
\emph{a fortiori} (since every submodule of an $A$-module of finite type is of finite type) it suffices to prove that, as $v$ runs over the set of endomorphisms of $\wedge^n E$, the $A$-module generated by the $\det v$ is of finite type;
in other words, we can again reduce to the case where $n=1$.
But then the proposition follows from the fact that $\Hom_A(E,E)$ is an $A$-module of finite type, and that $v\mapsto\det v$ is a homomorphism from this $A$-module into $K$.
\end{proof}

Let $F$ be the kernel of the canonical homomorphism $E\to E\otimes_A K$, and let $E_0=E/F$;
we see, as above, that $E\otimes_A K$ can be identified with $E_0\otimes_A K$, that $u(F)\subset F$, and that, if $u_0$ is the endomorphism of $E_0$ induced from $u$ by passing to the quotient, then $u\otimes1$ can be identified with $u_0\otimes1$, and $\sigma_j(u)=\sigma_j(u_0)$ for all $j$.
\emph{If we have $F=0$}, then equations~\sref{II.6.4.1.3} and \sref{II.6.4.1.5} still make sense and hold true whenever $E$ is identified with a submodule of $E\otimes_A K$, and the $u^j$ with homomorphisms $E\to E\otimes_A K$;
then Proposition~\sref{II.6.4.5} extends also to this case, with the same proof.
\end{remark}

\begin{env}[6.4.8]
\label{II.6.4.8}
Let $(X,\sh{A})$ be a ringed space, $\sh{E}$ a \emph{locally free} $\sh{A}$-module (of finite rank), and $u$ an endomorphism of $\sh{E}$.
By hypothesis, there exists a base $\mathfrak{B}$ for the topology of $X$ such that, for all $V\in\mathfrak{B}$, $\sh{E}|V$ is isomorphic to some $\sh{A}^n|V$ (where $n$ may vary with $V$).
Let $u$ be an endomorphism of $\sh{E}$;
for all $V\in\mathfrak{B}$, $u_V$ is thus an endomorphism of the $\Gamma(V,\sh{A})$-module $\Gamma(V,\sh{E})$, which is free by hypothesis;
the determinant $\det u_V$ is thus defined and belongs to $\Gamma(V,\sh{A})$.
Furthermore, if $e_1,\ldots,e_n$ form a basis of $\Gamma(V,\sh{E})$, then their restrictions to every open $W\subset V$ form a basis of $\Gamma(W,\sh{E})$ over $\Gamma(W,\sh{A})$, and so $\det u_W$ is the restriction of $\det u_V$ to $W$.
There thus exists exactly one section of $\sh{A}$ over $X$, which we denote by $\det u$, and call the \emph{determinant} of $u$, such that the restriction of $\det u$ to each $V\in\mathfrak{B}$ is $\det u_V$.
It is clear that, for all $x\in X$, we have $(\det u)_x=\det u_x$;
for endomorphisms $u$ and $v$ of $\sh{E}$, we have
\[
\label{II.6.4.8.1}
  \det(u\circ v) = (\det u)(\det v)
\tag{6.4.8.1}
\]
as well as
\[
\label{II.6.4.8.1}
  \det(1_{\sh{E}}) = 1_{\sh{A}}
\tag{6.4.8.1}
\]
and, if the rank of $\sh{E}$ is constant (which will be the case \sref[0]{0.5.4.1} if $X$ is connected) and equal to $n$,
\[
\label{II.6.4.8.3}
  \det(s\cdot u) = s^n\det u
\tag{6.4.8.3}
\]
for all $s\in\Gamma(X,\sh{A})$ (note that $\det(0)=0_{\sh{A}}$ if $n\geq1$, but $\det(0)=1_{\sh{A}}$ for $n=0$).
Furthermore, for $u$ to be an \emph{automorphism} of $\sh{E}$, it is necessary and sufficient that $u$ be \emph{invertible} in $\Gamma(X,\sh{A})$.

If the rank of $\sh{E}$ is constant, then we can similarly define the elementary symmetric functions $\sigma_i(u)$, which are elements of $\Gamma(X,\sh{A})$;
we again have the relations in \sref{II.6.4.1.3} and \sref{II.6.4.1.5}.

\oldpage[II]{124}
We have thus defined a homomorphism $\det:\Hom_{\sh{A}}(\sh{E},\sh{E})\to\Gamma(X,\sh{A})$ of multiplicative monoids;
if we note that $\Hom_{\sh{A}}(\sh{E},\sh{E})=\Gamma(X,\shHom_{\sh{A}}(\sh{E},\sh{E}))$ by definition, then we see that we can replace $X$ in this definition by an arbitrary open $U$ of $X$, which immediately defines a homomorphism $\det:\shHom_{\sh{A}}(\sh{E},\sh{E})\to\sh{A}$ of sheaves of multiplicative monoids.
If $\sh{E}$ has constant rank, then we similarly define homomorphisms $\sigma_i:\shHom_{\sh{A}}(\sh{E},\sh{E})\to\sh{A}$ of sheaves of sets;
for $i=1$, the homomorphism $\sigma_1=\Tr$ is a homomorphism of $\sh{A}$-modules.

Let $(Y,\sh{B})$ be another ringed space, and let $f:(X,\sh{A})\to(Y,\sh{B})$ be a morphism of ringed spaces;
if $\sh{F}$ is a locally free $\sh{B}$-module, then $f^*(\sh{F})$ is a locally free $\sh{A}$-module (which is of the same rank as $\sh{F}$ if the latter is of constant rank) \sref[0]{0.5.4.5}.
For every endomorphism $v$ of $\sh{F}$, $f^*(v)$ is an endomorphism of $f^*(\sh{F})$, and it follows immediately from the definitions that $\det f^*(v)$ is the section of $\sh{A}=f^*(\sh{B})$ over $X$ that canonically corresponds to $\det v\in\Gamma(Y,\sh{B})$.
We can further say that the homomorphism $f^*(\det):f^*(\shHom_{\sh{B}}(\sh{F},\sh{F}))\to f^*(\sh{B})=\sh{A}$ is the composition
\[
  f^*(\shHom_{\sh{B}}(\sh{F},\sh{F}))
  \xrightarrow{\gamma^\sharp} \shHom_{\sh{A}}(f^*(\sh{F}),f^*(\sh{F}))
  \xrightarrow{\det} \sh{A}
\]
\sref[0]{0.4.4.6}.
We have analogous results for the $\sigma_i$.
\end{env}

\begin{env}[6.4.9]
Now suppose that $X$ is a \emph{locally integral prescheme}, so that the sheaf $\sh{R}(X)$ of rational functions on $X$ is a locally simple sheaf of fields \sref[I]{I.7.4.3}, and quasi-coherent as an $\sh{O}_X$-module.
If $\sh{E}$ is a quasi-coherent $\sh{O}_X$-module \emph{of finite type}, then $\sh{E}'=\sh{E}\otimes_{\sh{O}_X}\sh{R}(X)$ is a locally free $\sh{R}(X)$-module \sref[I]{I.7.3.6};
for every endomorphism $u$ of $\sh{E}$, $u\otimes1_{\sh{R}(X)}$ is then an endomorphism of $\sh{E}'$, and $\det(u\otimes1)$ is a section of $\sh{R}(X)$ over $X$, which we also call the \emph{determinant} of $u$, and also denote by $\det u$.
It follows from \sref{II.6.4.3} that $\det u$ is a section of the \emph{integral closure} of $\sh{O}_X$ in $\sh{R}(X)$ \sref{II.6.3.2};
if, further, $X$ is \emph{normal}, then $\det u$ is a \emph{section of $\sh{O}_X$} over $X$, and if we further suppose that $\sh{E}$ is \emph{torsion free}, then for $u$ to be an automorphism of $\sh{E}$ it is necessary and sufficient that $\det u$ be \emph{invertible}, by \sref{II.6.4.6}.
Equations~\sref{II.6.4.8.1} to \sref{II.6.4.8.3} still hold true;
from the homomorphism $u\mapsto\det u$, applied to the modules of sections of $\shHom_{\sh{O}_X}(\sh{E},\sh{E})$, we obtain a homomorphism of sheaves $\det:\shHom_{\sh{O}_X}(\sh{E},\sh{E})\to\sh{R}(X)$, which takes values in $\sh{O}_X$ if $X$ is normal.
We have analogous definitions and results for the other elementary symmetric functions $\sigma_j(u)$, if $\sh{E}'$ is of \emph{constant rank};
if, further, $X$ is \emph{normal}, then the $\sigma_j(u)$ are \emph{sections of $\sh{O}_X$} over $X$.

Finally, let $X$ and $Y$ be integral preschemes, and $f:X\to Y$ a \emph{dominant} morphism.
We know that there exists a canonical homomorphism $f^*(\sh{R}(Y))\to\sh{R}(X)$ \sref[I]{I.7.3.8}, whence we obtain, for every quasi-coherent $\sh{O}_Y$-module $\sh{F}$ of finite type, a canonical homomorphism $\theta:f^*(\sh{F}\otimes_{\sh{O}_Y}\sh{R}(Y))\to f^*(\sh{F})\otimes_{\sh{O}_X}\sh{R}(X)$.
\oldpage[II]{125}
If $v$ is an endomorphism of $\sh{F}$, then $f^*(v\otimes1_{\sh{R}(Y)})$ is an endomorphism of $f^*(\sh{F}\otimes_{\sh{O}_Y}\sh{R}(Y))$, and we have a commutative diagram
\[
  \xymatrix{
    f^*(\sh{F}\otimes_{\sh{O}_Y}\sh{R}(Y))
      \ar[r]^{f^*(v\otimes1)}
      \ar[d]_{\theta}
    & f^*(\sh{F}\otimes{\sh{O}_Y}\sh{R}(Y))
      \ar[d]^{\theta}
  \\f^*(\sh{F})\otimes_{\sh{O}_X}\sh{R}(X)
      \ar[r]_{f^*(v)\otimes1}
    & f^*(\sh{F})\otimes_{\sh{O}_X}\sh{R}(X)
  }
\]

We thus easily conclude that $\det f^*(v)$ is the canonical image, under the homomorphism $f^*(\sh{R}(Y))\to\sh{R}(X)$, of the section $\det v$ of $\sh{R}(Y)$;
indeed, we can immediately reduce to the case where $X=\Spec(A)$ and $Y=\Spec(B)$ are affine, with $A$ and $B$ integral rings with fields of fractions $K$ and $L$ (resp.), with the homomorphism $B\to A$ an injection and thus extending to a monomorphism $L\to K$;
if $\sh{F}=\widetilde{M}$, where $M$ is a $B$-module of finite type, then the rank of $M\otimes_B L$ over $L$ is \emph{equal} to that of $(M\otimes_B A)\otimes_A K$ over $K$, and $\det((u\otimes1)\otimes1)$ is the image in $K$ of $\det(u\otimes1)$ for any endomorphism $u$ of $M$, whence the conclusion.
\end{env}

\begin{env}[6.4.10]
\label{II.6.4.10}
Finally, suppose that $X$ is a \emph{locally Noetherian reduced prescheme}, so that the sheaf $\sh{R}(X)$ of rational functions on $X$ is again a quasi-coherent $\sh{O}_X$-module \sref[I]{I.7.3.4};
let $\sh{E}$ be a \emph{coherent} $\sh{O}_X$-module such that $\sh{E}'=\sh{E}\otimes_{\sh{O}_X}\sh{R}(X)$ is \emph{locally free} (of finite rank).
By \sref{II.6.4.7}, if $\sh{E}'$ is of constant rank, then we can, for any endomorphism $u$ of $\sh{E}$, define the $\sigma_j(u)$, which are sections of $\sh{R}(X)$ over $X$.
If we do not suppose $\sh{E}'$ to be of constant rank, we can still define the homomorphism $\det:\shHom_{\sh{O}_X}(\sh{E},\sh{E})\to\sh{R}(X)$.
\end{env}


\subsection{Norm of an invertible sheaf}
\label{subsection:II.6.5}

\begin{env}[6.5.1]
\label{II.6.5.1}
Let $(X,\sh{A})$ be a ringed space and $\sh{B}$ a (commutative) $\sh{A}$-algebra.
The $\sh{A}$-module $\sh{B}$ is canonically identified with a sub-$\sh{A}$-module of $\shHom_{\sh{A}}(\sh{B},\sh{B})$, and a section $f$ of $\sh{B}$ over an open $U$ of $X$ is identified with multiplication by this section.
If $(X,\sh{A})$ and $\sh{B}$ satisfy one of the conditions given in \sref{II.6.4.8}, \sref{II.6.4.9}, or \sref{II.6.4.10}, then we can define $\det(f)$ (and, in certain cases, the $\sigma_j(f)$) as sections of $\sh{A}$ or $\sh{R}(X)$ over $U$, that we call the \emph{norm} of $f$ (resp. the \emph{elementary symmetric functions} of $f$), and denote by $N_{\sh{B}/\sh{A}}(f)$.
We will suppose that \emph{one} of the following conditions is satisfied:
\begin{enumerate}
  \item[(I)] \emph{$\sh{B}$ is a locally free $\sh{A}$-module (of finite rank).}
  \item[(II)] \emph{$(X,\sh{A})$ is a locally Noetherian reduced prescheme, $\sh{B}$ is a coherent $\sh{A}$-module such that $\sh{B}\otimes_{\sh{A}}\sh{R}(X)$ is a locally free $\sh{R}(X)$-module, and, for every section $f\in\Gamma(U,\sh{B})$ over an open $U$ of $X$, the norm $N_{\sh{B}/\sh{A}}(f)$ is a section of $\sh{A}$ over $U$.}
\end{enumerate}

\oldpage[II]{126}
Note that the latter condition is automatically satisfied whenever the locally Noetherian prescheme $X$ is \emph{normal} \sref{II.6.4.9}.

The hypothesis that $\sh{B}\otimes_{\sh{A}}\sh{R}(X)$ is locally free can also be expressed in the following way: denote by $X_\alpha$ the closed reduced subpreschemes of $X$ whose underlying spaces are the irreducible components of $X$ \sref[I]{I.5.2.1}, which are thus locally Noetherian integral preschemes.
Every $x\in X$ belongs to a finite number of the subspaces $X_\alpha$;
on the other hand, $\sh{B}\otimes_{\sh{A}}\sh{R}(X_\alpha)$ is a locally free $\sh{R}(X_\alpha)$-module of constant rank~$k_\alpha$ \sref[I]{I.7.3.6};
to say that $\sh{B}\otimes_{\sh{A}}\sh{R}(X)$ is a locally free $\sh{R}(X)$-module implies that, for all $x\in X$, \emph{the ranks $k_\alpha$ that correspond to the indices such that $x\in X_\alpha$ are equal}.
This is a local statement, and we can reduce to the case $X=\Spec(C)$, where $C$ is a reduced Noetherian ring, and $\sh{B}=\widetilde{D}$, where $D$ is a $C$-algebra that is a $C$-module of finite type;
if $\mathfrak{p}_i$ (for $1\leq i\leq m$) are the minimal prime ideals of $C$, then the total ring of fractions $L$ of $C$ is the direct sum of the fields of fractions $K_i$ of the integral rings $C/\mathfrak{p}_i$, and $D\otimes_C L$ is the direct sum of the $D\otimes_C K_i$, whence the conclusion.

It is clear that, under hypothesis~(I) or (II), we have thus defined a \emph{homomorphism of sheaves of multiplicative monoids} $N_{\sh{B}/\sh{A}}:\sh{B}\to\sh{A}$, also denoted by $N$ if no confusion may arise, and we call this homomorphism the \emph{norm}.
For sections $f$ and $g$ of $\sh{B}$ over the same open $U$, we thus have
\[
\label{II.6.5.1.1}
  N_{\sh{B}/\sh{A}}(fg) = N{\sh{B}/\sh{A}}(f){\sh{B}/\sh{A}}(g)
\tag{6.5.1.1}
\]
for the corresponding sections of $\sh{A}$ over $U$;
\[
\label{II.6.5.1.2}
  N{\sh{B}/\sh{A}}(1_{\sh{B}}) = 1_{\sh{A}};
\tag{6.5.1.2}
\]
finally, for any section $s$ of $\sh{A}$ over $U$,
\[
\label{II.6.5.1.3}
  N{\sh{B}/\sh{A}}(s\cdot1_{\sh{B}}) = s^n
\tag{6.5.1.3}
\]
if the rank of $\sh{B}$ is constant and equal to $n$ (for $s=0_{\sh{A}}$, this formula gives $N(0_{\sh{B}})=0_{\sh{A}}$ if $n\geq1$, and $N(0_{\sh{B}})=N(1_{\sh{B}})=1_{\sh{A}}$ if $n=0$).

In hypothesis~(I), for $f\in\Gamma(U,\sh{B})$ to be invertible it is necessary and sufficient that $N(f)\in\Gamma(U,\sh{A})$ be invertible.
In hypothesis~(II), this condition is necessary;
it is sufficient (by \sref{II.6.4.7}) if we suppose that $\sh{B}\to\sh{B}\otimes_{\sh{A}}\sh{R}(X)$ is \emph{injective} and that the following more restrictive hypothesis is satisfied:
\begin{enumerate}
  \item[(II~\emph{bis})] \emph{$(X,\sh{A})$ is a locally Noetherian reduced prescheme, $\sh{B}$ is a coherent $\sh{A}$-module such that $\sh{B}\otimes_{\sh{A}}\sh{R}(X)$ is a locally free $\sh{R}(X)$-module, and, for any section $f\in\Gamma(U,\sh{B})$ over an open $U$ such that $\sh{B}\otimes_{\sh{A}}\sh{R}(X)|U$ is of constant rank~$n$ on $\sh{R}(X)|U$, the $\sigma_j(f)$ (for $1\leq j\leq n$) are sections of $\sh{A}$ over $U$.}
\end{enumerate}

(We again note that this condition is satisfied if $X$ is \emph{normal}.)
\end{env}

\begin{env}[6.5.2]
\label{II.6.5.2}
Suppose that one of the hypotheses (I) or (II) of \sref{II.6.5.1} are satisfied, and let $\sh{L}'$ be an \emph{invertible} $\sh{B}$-module.
We will canonically associate (up to unique isomorphism) an \emph{invertible} $\sh{A}$-module in the following way.
Denote by $\sh{A}^*$ (resp. $\sh{B}^*$) the subsheaf of $\sh{A}$ (resp. of $\sh{B}$) such that $\Gamma(U,\sh{A}^*)$ (resp. $\Gamma(U,\sh{B}^*)$) is the set of invertible elements of $\Gamma(U,\sh{A})$ (resp. of $\Gamma(U,\sh{B})$) for every open $U\subset X$;
\oldpage[II]{127}
these are sheaves of multiplicative groups, and $N_{\sh{B}/\sh{A}}$ restricted to $\sh{B}^*$ is a \emph{homomorphism} $\sh{B}^*\to\sh{A}^*$ of sheaves of groups \sref{II.6.5.1}.
Let $\mathfrak{L}$ be the set of pairs $(U_\lambda,\nu_\lambda)$ that have the following property: $U_\lambda$ is an open of $X$, and $\nu_\lambda$ is an isomorphism $\sh{L}'|U_\lambda\simto\sh{B}|U_\lambda$ of $(B|U_\lambda)$-modules.
By hypothesis, the $U_\lambda$ form a cover of $X$;
for arbitrary indices $\lambda$ and $\mu$, set $\omega_{\lambda\mu}=(\eta_\lambda|U_\lambda\cap U_\mu)\circ(\nu_\mu|U_\lambda\cap U_\mu)^{-1}$, which is an automorphism of $\sh{B}|U_\lambda\cap U_\mu$ which is canonically identified with a section of $\sh{B}^*$ over $U_\lambda\cap U_\mu$, and $(\omega_{\lambda\mu})$ is a $1$-cocycle for the cover $\mathfrak{U}=(U_\lambda)$ with values in $\sh{B}^*$ \sref[0]{0.5.4.7}.
The fact that $N_{\sh{B}/\sh{A}}:\sh{B}^*\to\sh{A}^*$ is a homomorphism implies that $(N_{\sh{B}/\sh{A}}\omega_{\lambda\mu})$ is a $1$-cocycle of $\mathfrak{U}$ with values in $\sh{A}^*$, and thus corresponds (up to unique isomorphism) to an invertible $\sh{A}$-module $\sh{L}$ \sref[0]{0.5.4.7} which we denote by $N_{\sh{B}/\sh{A}}(\sh{L}')$, and call the \emph{norm} of the invertible $\sh{B}$-module $\sh{L}'$.

Let $\mathfrak{M}$ be a subset of $\mathfrak{L}$ such that the corresponding $U_\lambda$ still form a cover of $X$, and let $\mathfrak{B}$ be the resulting cover;
the restriction of the cocycle $(\omega_{\lambda\mu})$ to $\mathfrak{B}$ defines a $1$-cocycle $(N_{\sh{B}/\sh{A}}\omega_{\lambda\mu})$ of $\mathfrak{B}$ with values in $\sh{A}^*$, the restriction of the $1$-cocycle $(N_{\sh{B}/\sh{A}}\omega_{\lambda\mu})$ of $\mathfrak{U}$;
it is clear that there is a canonical isomorphism from the invertible $\sh{A}$-module defined by this $1$-cocycle of $\mathfrak{B}$ to $N_{\sh{B}/\sh{A}}(\sh{L}')$, which allows us to define the invertible $\sh{A}$-module by an arbitrary sub-cover of $\mathfrak{U}$.
This possibility immediately shows that, if $\sh{L}'_1$ and $\sh{L}'_2$ are invertible $\sh{B}$-modules, then, by \sref{II.6.5.1.1} and \sref{II.6.5.1.2},
\[
\label{II.6.5.2.1}
  N(\sh{L}'_1\otimes_{\sh{B}}\sh{L}'_2) = N(\sh{L}'_1)\otimes_{\sh{A}}N(\sh{L}'_2)
\tag{6.5.2.1}
\]
and
\[
\label{II.6.5.2.2}
  N_{\sh{B}/\sh{A}}(\sh{B}) = \sh{A}
\tag{6.5.2.2}
\]
as well as
\[
\label{II.6.5.2.3}
  N({\sh{L}'}^{-1}) = (N(\sh{L}'))^{-1}
\tag{6.5.2.3}
\]
up to canonical isomorphisms.
Furthermore, it follows from \sref{II.6.5.1.3} that, if $\sh{L}$ is an invertible $\sh{A}$-module = and if $\sh{B}$ is of constant rank~$n$ on $\sh{A}$ in case~(I) (resp. $\sh{B}\otimes_{\sh{A}}\sh{R}(X)$ is of constant rank~$n$ on $\sh{R}(X)$ in case~(II)), we have, up to canonical isomorphism,
\[
\label{II.6.5.2.4}
  N_{\sh{B}/\sh{A}}(\sh{L}\otimes_{\sh{A}}\sh{B}) = \sh{L}^{\otimes n}.
\tag{6.5.2.4}
\]
\end{env}

\begin{env}[6.5.3]
\label{II.6.5.3}
We now show that $N_{\sh{B}/\sh{A}}$ is a covariant \emph{functor} in the category of invertible $\sh{B}$-modules.
Let $h':\sh{L}'_1\to\sh{L}'_2$ be a homomorphism of invertible $\sh{B}$-modules, and let $\mathfrak{B}=(U_\lambda)$ be an open cover of $X$ such that, for all $\lambda$, we have $(\sh{B}|U_\lambda)$-isomorphisms $\eta_\lambda^{(1)}:\sh{L}'_1|U_\lambda\simto\sh{B}|U_\lambda$ and $\eta_\lambda^{(2)}:\sh{L}'_2|U_\lambda\simto\sh{B}|U_\lambda$;
there is thus, for each $\lambda$, an endomorphism $h'_\lambda$ of $\sh{B}|U_\lambda$ such that $h'_\lambda\circ\eta_\lambda^{(1)}=\eta_\lambda^{(2)}\circ(h'|U_\lambda)$, and we can evidently identify $h'_\lambda$ with a section of $\sh{B}$ over $U_\lambda$ \sref[0]{0.5.1.1}.
So, for every pair $(\lambda,\mu)$ of indices, the restrictions to $U_\lambda\cap U_\mu$ of $(\eta_\lambda^{(2)})^{-1}\circ h'_\lambda\circ\eta_\lambda^{(1)}$ and $(\eta_\mu^{(2)})^{-1}\circ h'_\mu\circ\eta_\mu^{(1)}$ agree.
We thus obtain, for $1$-cocycles $(\omega_{\lambda\mu}^{(1)})$ and $(\omega_{\lambda\mu}^{(2)})$ with values in $\sh{B}^*$ corresponding to $\sh{L}'_1$ and $\sh{L}'_2$, the relation
\[
  \omega_{\lambda\mu}^{(2)}h'_\mu = h'_\lambda\omega_{\lambda\mu}^{(1)}.
\]

\oldpage[II]{128}
If we set $h_\lambda=N(h'_\lambda)$, we thus have the analogous relations
\[
  N(\omega_{\lambda\mu}^{(2)})h_\mu = h_\lambda N(\omega_{\lambda\mu}^{(1)})
\]
and so the $h_\lambda$ define a homomorphism $N(\sh{L}'_1)\to N(\sh{L}'_2)$ which we denote by $N_{\sh{B}/\sh{A}}(h')$, or $N(h')$.
In hypothesis~(I), for $h'$ to be an \emph{isomorphism}, it is necessary and sufficient (since it is a local question) for $N_{\sh{B}/\sh{A}}(h')$ to be an isomorphism.
In hypothesis~(II), this condition is again necessary;
it is sufficient if hypothesis~(II~\emph{bis}) is satisfied and $\sh{B}\to\sh{B}\otimes_{\sh{A}}\sh{R}(X)$ is injective.

Take, in particular, $\sh{L}'_1=\sh{B}$;
the homomorphisms $\sh{B}\to\sh{L}'$ can then be identified \sref[0]{0.5.1.1} with the sections of $\sh{L}'$ over $X$, whence a canonical map
\[
  N_{\sh{B}/\sh{A}} : \Gamma(X,\sh{L}')\to\Gamma(X,N_{\sh{B}/\sh{A}}(\sh{L}')).
\]

It again follows from \sref{II.6.5.1.1} that, if $f'_1\in\Gamma(X,\sh{L}'_1)$ and $f'_2\in\Gamma(X,\sh{L}'_2)$, then
\[
\label{II.6.5.3.1}
  N(f'_1\otimes f'_2) = N(f'_1)\otimes N(f'_2).
\tag{6.5.3.1}
\]

For every invertible $\sh{A}$-module $\sh{L}$ and every section $f\in\Gamma(X,\sh{L})$, we have, taking \sref{II.6.5.2.4} into account, that
\[
\label{II.6.5.3.2}
  N_{\sh{B}/\sh{A}}(f\otimes 1_{\sh{B}}) = f^{\otimes n}
\tag{6.5.3.2}
\]
whenever $\sh{B}$ is of constant rank~$n$ in hypothesis~(I) (resp. whenever $\sh{B}\otimes_{\sh{A}}\sh{R}(X)$ is of constant rank~$n$ in hypothesis~(II)).
Finally, for the homomorphism $\sh{B}\to\sh{L}'$ corresponding to a section $f'$ of $\sh{L}'$ over $X$ to be an isomorphism, it is necessary and sufficient for $f'_x$ to be a basis for $\sh{L}'_x$ for all $x\in X$;
in hypothesis~(I), this condition is thus equivalent to saying that $(N(f'))_x$ is a basis for $(N(\sh{L}'))_x$ for all $x$;
in hypothesis~(II), this condition is again necessary, and it is sufficient whenever $\sh{B}$ satisfies hypothesis~(II~\emph{bis}) and $\sh{B}\to\sh{B}\otimes_{\sh{A}}\sh{R}(X)$ is injective.
\end{env}

\begin{env}[6.5.4]
\label{II.6.5.4}
Let $(X,\sh{A})$ and $(X',\sh{A}')$ be ringed spaces, $f:X'\to X$ a morphism, $\sh{B}$ an $\sh{A}$-algebra, and $\sh{B}'=f^*(\sh{B})$ the inverse image $\sh{A}'$-algebra.
Suppose that one of the following hypotheses is satisfied:
\begin{enumerate}
  \item $\sh{B}$ satisfies hypothesis~(I) of \sref{II.6.5.1}.
  \item $(X,\sh{A})$ and $\sh{B}$ satisfy hypothesis~(II) of \sref{II.6.5.1}, $(X',\sh{A}')$ is a locally Noetherian reduced prescheme, and, if we denote by $X_\alpha$ and $X'_\beta$ the closed reduced subpreschemes of $X$ and $X'$ (respectively) that have the irreducible components of these spaces as their underlying spaces, then the restriction of $f$ to each $X'_\beta$ is a \emph{dominant} morphism from $X'_\beta$ to $X_\alpha$.
\end{enumerate}

Under these conditions, $\sh{B}'$ satisfies hypothesis~(I) or hypothesis~(II) (respectively) of \sref{II.6.5.1}.
The first claim is immediate;
to establish the second, it suffices to show that, for all $x'\in X'$, the ranks of the $\sh{B}'\otimes_{\sh{O}_{X'}}\sh{R}(X'_\beta)$ are \emph{the same} for the $\beta$ such that $x'\in X'_\beta$.
But if the restriction of $f$ to $X'_\beta$ is a dominant morphism to $X_\alpha$, then the rank of $\sh{B}'\otimes_{\sh{O}_{X'}}\sh{R}(X'_\beta)$ is equal to that of $\sh{B}\otimes_{\sh{O}_X}\sh{R}(X_\alpha)$ (as we immediately see by reducing to the affine case, as in \sref{II.6.4.9}), whence our claim, by hypothesis~(II) and \sref{II.6.5.1}.

\oldpage[II]{129}
With this in mind, it follows from \sref{II.6.4.8}, \sref{II.6.4.9}, and \sref{II.6.4.10} that, if $s$ is a section of $\sh{B}$ over an open $U\subset X$, and $s'$ the corresponding section of $\sh{B}'$ over $f^{-1}(U)$, then $N_{\sh{B}'/\sh{A}'}(s')$ is the section of $\sh{A}'$ over $f^{-1}(U)$ that corresponds to the section $N_{\sh{B}/\sh{A}}(s)$ of $\sh{A}$ over $U$.

If $\sh{M}$ is an invertible $\sh{B}$-module, then we deduce from the above that, if $\sh{M}'=f^*(\sh{M})$ (which is an invertible $\sh{B}'$-module), we have $N_{\sh{B}'/\sh{A}'}(\sh{M}')=f^*(N_{\sh{B}/\sh{A}}(\sh{M}))$ up to canonical isomorphism.
\end{env}

\begin{env}[6.5.5]
\label{II.6.5.5}
Suppose from now on that $(X,\sh{A})$ is a \emph{prescheme}.
The data of a quasi-coherent $\sh{A}$-algebra $\sh{B}$, which is an \emph{$\sh{A}$-module of finite type}, is then equivalent, as we know, to that of a \emph{finite} morphism $g:X'\to X$ such that $g_*(\sh{O}_{X'})=\sh{B}$, defined up to $X$-isomorphism (\sref{II.6.1.2} and \sref{II.1.3.1}).
Furthermore, the data of a quasi-coherent $\sh{O}_{X'}$-module $\sh{F}'$ is equivalent to that of a quasi-coherent $\sh{B}$-module $\sh{F}$ such that $g_*(\sh{F})=\sh{F}$ \sref{II.1.4.3}, and for $\sh{F}'$ to be invertible it is necessary and sufficient that $\sh{F}$ be invertible \sref{II.6.1.12}.
To translate the above results in terms of finite morphisms $g$, it will be necessary to suppose either that $g_*(\sh{O}_{X'})$ is a \emph{locally free} $\sh{O}_X$-module (of finite type) or that $(X,\sh{O}_X)$ and $g_*(\sh{O}_{X'})$ satisfy hypothesis~(II).
For every invertible $\sh{O}_{X'}$-module $\sh{L}'$, we thus set
\[
\label{II.6.5.5.1}
  N_{X'/X}(\sh{L}') = N_{g_*(\sh{O}_{X'})/\sh{O}_X}(g_*(\sh{L}'))
\tag{6.5.5.1}
\]
and we call this the \emph{norm} (with respect to $g$) of $\sh{L}'$.
Similarly, if $h':\sh{L}'_1\to\sh{L}'_2$ is a homomorphism of invertible $\sh{O}_{X'}$-modules, then we set
\[
\label{II.6.5.5.2}
  N_{X'/X}(h') = N_{g_*(\sh{O}_{X'})/\sh{O}_X}(g_*(h')) : N_{X'/X}(\sh{L}'_1)\to N_{X'/X}(\sh{L}'_2).
\tag{6.5.5.2}
\]

In particular, for $\sh{L}'_1=\sh{O}_{X'}$, we thus obtain a canonical map
\[
\label{II.6.5.5.3}
  N_{X'/X} : \Gamma(X',\sh{L}') \to \Gamma(X,N_{X'/X}(\sh{L}')).
\tag{6.5.5.3}
\]

We leave to the reader the majority of these translations, and we restrict ourselves to spelling out the details of the following:
\end{env}

\begin{proposition}[6.5.6]
\label{II.6.5.6}
Let $g:X'\to X$ be a finite morphism, and suppose that either $g_*(\sh{O}_{X'})$ is a locally free $\sh{O}_X$-module or that $(X,\sh{O}_X)$ and $g_*(\sh{O}_{X'})$ satisfy (II~\emph{bis}) (which will be the case, in particular, if $X$ is locally Noetherian and normal).
For a homomorphism $h':\sh{L}'_1\to\sh{L}'_2$ of invertible $\sh{O}_{X'}$-modules to be an isomorphism, it is necessary and sufficient, in the first hypothesis, that $N_{X'/X}(h')$ be an isomorphism;
in the second hypothesis, this condition is again necessary, and is sufficient if the homomorphism $g_*(\sh{O}_{X'})\to g_*(\sh{O}_{X'})\otimes_{\sh{O}_X}\sh{R}(X)$ is injective.
\end{proposition}

\begin{proof}
Note that we use here the fact that, for $g_*(h')$ to be an isomorphism, it is necessary and sufficient for $h'$ to be an isomorphism \sref{II.1.4.2}.
\end{proof}

\begin{corollary}[6.5.7]
\label{II.6.5.7}
Let $g:X'\to X$ be a finite morphism, and suppose that either $g_*(\sh{O}_{X'})$ is a locally free $\sh{O}_X$-module or that $(X,\sh{O}_X)$ and $g_*(\sh{O}_{X'})$ satisfy (II~\emph{bis}) and $g_*(\sh{O}_{X'})\to g_*(\sh{O}_{X'})\otimes_{\sh{O}_X}\sh{R}(X)$ is injective.
Let $\sh{L}'$ be an invertible $\sh{O}_{X'}$-module, $f'$ a section of $\sh{L}'$ over $X'$, and $f=N_{X'/X}(f')$ the corresponding section of $\sh{L}=N_{X'/X}(\sh{L}')$ over $X$ \sref{II.6.5.5.1}.
\oldpage[II]{130}
Then $g(X'\setminus X'_{f'})=X\setminus X_f$, and $X_f$ is the largest open subset $U$ of $X$ such that $g^{-1}(U)\subset X'_{f'}$.
\end{corollary}

\begin{proof}
Indeed, $g(X'\setminus X'_{f'})$ is closed in $X$ \sref{II.6.1.10};
it thus suffices to prove the last claim.
But the relation $U\subset X_f$ is equivalent to the fact that the homomorphism $\sh{O}_X|U\to\sh{L}|U$ defined by $f|U$ is an isomorphism.
By \sref{II.6.5.6}, this is equivalent to saying that the homomorphism $\sh{O}_{X'}|g^{-1}(U)\to\sh{L}'|g^{-1}(U)$ defined by $f'|g^{-1}(U)$ is an isomorphism, which is equivalent to the relation $g^{-1}(U)\subset X'_{f'}$.
\end{proof}

\begin{proposition}[6.5.8]
\label{II.6.5.8}
Let $g:X'\to X$ be a finite morphism, and $f:Y\to X$ a morphism;
let $Y'=X'_{(Y)}$, $g'=g_{(Y)}$, and $f'=f_{(X')}$, so that the diagram
\[
  \xymatrix{
    X' \ar[d]_{g}
    & Y' \ar[l]_{f'} \ar[d]^{g'}
  \\X
    & Y \ar[l]^{f}
  }
\]
commutes.

Suppose that either $g_*(\sh{O}_{X'})$ is a locally free $\sh{O}_X$-module or that $(X,\sh{O}_X)$ and $g_*(\sh{O}_{X'})$ satisfy (II).
Suppose further that $Y$ is a locally Noetherian reduced prescheme, and that the restriction of $f$ to any irreducible component of $Y$ is a dominant morphism to an irreducible component of $X$.
Then, for every invertible $\sh{O}_{X'}$-module $\sh{L}'$, we have
\[
  N_{Y'/Y}({f'}^*(\sh{L}')) = f^*(N_{X'/X}(\sh{L}'))
\]
up to canonical isomorphism.
\end{proposition}

\begin{proof}
Note that we have $f^*(g_*(\sh{L}'))=g'_*({f'}^*(\sh{L}'))$, by \sref{II.1.5.2}, and, in particular, $g'_*(\sh{O}_{Y'})=f^*(g_*(\sh{O}_{X'}))$;
if $g_*(\sh{O}_{X'})$ is locally free, then so too is $g'_*(\sh{O}_{Y'})$.
The conclusion then follows from the definitions and \sref{II.6.5.4}.
\end{proof}

\begin{remark}[6.5.9]
\label{II.6.5.9}
We later generalise the notion of norm developed above, by placing it relation to the notion of direct image of a divisor.
\end{remark}


\subsection{Application: criteria for ampleness}
\label{subsection:II.6.6}

\begin{proposition}[6.6.1]
\label{II.6.6.1}
Let $Y$ be a prescheme, $f:X\to Y$ a quasi-compact morphism, and $g:X'\to X$ a finite and surjective morphism.
Suppose that either $g_*(\sh{O}_{X'})$ is a locally free $\sh{O}_X$-module or that $(X,\sh{O}_X)$ and $g_*(\sh{O}_{X'})$ satisfy (II~\emph{bis}).
Then, for every invertible $\sh{O}_{X'}$-module $\sh{L}'$ that is ample for $f\circ g$, $\sh{L}=N_{X'/X}(\sh{L}')$ is ample for $f$.
\end{proposition}

\begin{proof}
We can suppose $Y$ to be affine \sref{II.4.6.4}, and then, by \sref{II.4.6.6}, the statement is equivalent to:
\end{proof}

\begin{corollary}[6.6.2]
\label{II.6.6.2}
Let $X$ be a quasi-compact prescheme, $g:X'\to X$ a finite surjective morphism, such that either $g_*(\sh{O}_{X'})$ is a locally free $\sh{O}_X$-module or that $(X,\sh{O}_X)$ and $g_*(\sh{O}_{X'})$ satisfy (II~\emph{bis}).
Then, for every ample $\sh{O}_{X'}$-module $\sh{L}'$, $\sh{L}=N_{X'/X}(\sh{L}')$ is ample.
\end{corollary}

\begin{proof}
In the second hypothesis, we can further suppose that the canonical homomorphism $g_*(\sh{O}_{X'})\to g_*(\sh{O}_{X'})\otimes_{\sh{O}_X}\sh{R}(X)$ is \emph{injective}.
\oldpage[II]{131}
Indeed, if not, let $\sh{T}$ be the kernel of this homomorphism, which is a coherent ideal of $\sh{B}=g_*(\sh{O}_{X'})$ \sref[I]{I.6.1.1}, and set $X''=\Spec(\sh{B}/\sh{T})$;
we thus have a commutative diagram
\[
  \xymatrix{
    X'' \ar[rr]^{h} \ar[dr]_{g'}
    && X' \ar[dl]_{g}
  \\&X
  }
\]
where $h$ is a closed immersion \sref{II.1.4.10}.
Furthermore, we know that the support of $\sh{T}$ is a closed set \sref[0]{0.5.2.2} that is rare in $X$ \sref[I]{I.7.4.6}, whence we conclude that, for the generic point $x$ of an irreducible component of $X$, there is an affine open neighbourhood $U$ of $x$ such that $\sh{B}|U=(\sh{B}/\sh{T})|U$.
Since $g$ is, by hypothesis, surjective, we thus conclude that $x\in g'(X'')$;
$g'$ is thus dominant, and, since it is a finite morphism, it is \emph{surjective} \sref{II.6.1.10};
by definition,
\[
  g'_*(\sh{O}_{X''})\otimes_{\sh{O}_X}\sh{R}(X)
  = (\sh{B}/\sh{T})\otimes_{\sh{O}_X}\sh{R}(X)
  = g_*(\sh{O}_{X'})\otimes_{\sh{O}_X}\sh{R}(X)
\]
thus $(X,\sh{O}_X)$ and $g'_*(\sh{O}_{X''})$ satisfy (II~\emph{bis}), and furthermore $g'_*(\sh{O}_{X''})\to g'_*(\sh{O}_{X''})\otimes_{\sh{O}_X}\sh{R}(X)$ is injective.
Finally, $h^*(\sh{L}')=\sh{L}''$ is an ample $\sh{O}_{X''}$-module \sref{II.4.6.13}[(i~\emph{bis})], and we have that $N_{X''/X}(\sh{L}'')=N_{X'/X}(\sh{L}')$.
Indeed, to define these two invertible $\sh{O}_X$-modules, we can use the same affine open cover $(U_\lambda)$ of $X$ such that the restrictions of $g_*(\sh{L}')$ and $g'_*(\sh{L}'')$ to $U_\lambda$ are isomorphic to $\sh{B}|U_\lambda$ and $(\sh{B}/\sh{T})|U_\lambda$ (respectively).
We immediately see that, for every isomorphism $\eta_\lambda:g_*(\sh{L}')|U_\lambda\to\sh{B}|U_\lambda$, there is a canonically corresponding isomorphism
\[
  \eta'_\lambda : g'_*(\sh{L}'')|U_\lambda \to (\sh{B}/\sh{T})|U_\lambda
\]
so that, if $(\omega_{\lambda\mu})$ and $(\omega'_{\lambda\mu})$ are the $1$-cocycles corresponding to the systems of isomorphisms $(\eta_\lambda)$ and $(\eta'_\lambda)$ \sref{II.6.5.2}, $\omega'_{\lambda\mu}$ is the canonical image in $\Gamma(U_\lambda\cap U_\mu,\sh{B}/\sh{T})$ of $\omega_{\lambda\mu}\in\Gamma(U_\lambda\cap U_\mu,\sh{B})$.
By the definition of $\sh{T}$, we thus conclude that
\[
  N_{\sh{B}/\sh{A}}(\omega_{\lambda\mu}) = N_{(\sh{B}/\sh{T})/\sh{A}}(\omega'_{\lambda\mu})
\]
(where $\sh{A}=\sh{O}_X$), whence the claimed equality.

So suppose that the homomorphism $g_*(\sh{O}_{X'})\to g_*(\sh{O}_{X'})\otimes_{\sh{O}_X}\sh{R}(X)$ is injective when we are in hypothesis~(II~\emph{bis}).
It suffices to prove that, as $f$ runs over the sections of $\sh{L}^{\otimes n}$ (for $n>0$) over $X$, the $X_f$ form a base for the topology of $X$ \sref{II.4.5.2}.
But let $x\in X$, and let $U$ be an arbitrary neighbourhood of $x$;
since $g^{-1}(x)$ is finite \sref{II.6.1.7} and $\sh{L}'$ is ample, there exists an integer $n>0$ and a section $f'$ of ${\sh{L}'}^{\otimes n}$ over $X'$ such that $X'_{f'}$ is a neighbourhood of $g^{-1}(x)$ contained inside $g^{-1}(U)$ \sref{II.4.5.4}.
Since
\[
  \sh{L}^{\otimes n} = N_{X'/X}({\sh{L}'}^{\otimes n})
\]
it suffices to take $f=N_{X'/X}(f')$;
indeed, then $X\setminus X_f=g(X'\setminus X'_{f'})$ \sref{II.6.5.7}, and so $x\in X_f\subset U$.
\end{proof}

\begin{corollary}[6.6.3]
\label{II.6.6.3}
Under the hypotheses of \sref{II.6.6.1}, for an invertible $\sh{O}_X$-module $\sh{L}$ to be ample for $f$, it is necessary and sufficient that $\sh{L}'=g^*(\sh{L})$ be ample for $f\circ g$.
\end{corollary}

\begin{proof}
The condition is necessary, since $g$ is affine \sref{II.5.1.12}.
To prove that the condition is sufficient, we can suppose that $Y$ is affine \sref{II.4.6.4}, and so $X$ and $X'$ are quasi-compact and $\sh{L}'$ is ample \sref{II.4.6.6}, and we need to show that $\sh{L}$ is ample.
\oldpage[II]{132}
But the set of points $x\in X$ that admit a neighbourhood where $g_*(\sh{O}_{X'})$ (resp. $g_*(\sh{O}_{X'})\otimes_{\sh{O}_X}\sh{R}(X)$) has a given rank~$n$ in the first (resp. second) hypothesis is simultaneously open and closed in $X$, and so $X$ is the prescheme given by the sum of a finite number of these opens, and we can thus suppose that it is equal to one of them \sref{II.4.6.17}.
But then $N_{X'/X}(\sh{L}')=\sh{L}^{\otimes n}$, and so $\sh{L}^{\otimes n}$ is ample by \sref{II.6.6.2}, and thus so too is $\sh{L}$ \sref{II.4.5.6}.
\end{proof}

\begin{corollary}[6.6.4]
\label{II.6.6.4}
Suppose that the hypotheses of \sref{II.6.6.1} are satisfied, and suppose further that $f:X\to Y$ is of finite type.
Then, for $f$ to be quasi-projective, it is necessary and sufficient that $f\circ g$ be quasi-projective.
If we further suppose that $Y$ is a quasi-compact scheme, or a prescheme whose underlying space is Noetherian, then for $f$ to be projective, it is necessary and sufficient that $f\circ g$ be projective.
\end{corollary}

\begin{proof}
The hypothesis implies that $f\circ g$ is of finite type.
Taking into account the definition of quasi-projective morphisms \sref{II.5.3.1}, the first claim follows from \sref{II.6.6.1} and \sref{II.6.6.3}.
Taking into account this result, along with \sref{II.5.5.3}[(ii)], it remains to show that, if $f$ is quasi-projective, then for $f$ to be proper, it is necessary and sufficient that $f\circ g$ be proper.
But $f$ is then separated \sref{II.5.3.1} and of finite type;
since $g$ is surjective, our claim follows from \sref{II.5.4.2}[(ii)] and \sref{II.5.4.3}[(ii)].
\end{proof}

In particular:

\begin{corollary}[6.6.5]
\label{II.6.6.5}
Let $X$ be a prescheme of finite type over a field $K$, and $K'$ a finite-degree extension of $K$.
For $X$ to be projective (resp. quasi-projective) over $K$, it is necessary and sufficient that $X'=X\otimes_K K'$ be projective (resp. quasi-projective) over $K'$.
\end{corollary}

\begin{proof}
The condition is necessary (\sref{II.5.3.4}[(iii)] and \sref{II.5.5.5}[(iii)]).
Conversely, suppose that it is satisfied, and let $g:X'\to X$ be the canonical projection.
It is clear that $g$ is a finite morphism \sref{II.6.1.5}[(iii)] and surjective \sref[I]{I.3.5.2}[(ii)].
Furthermore, $g_*(\sh{O}_{X'})$ is a locally free $\sh{O}_X$-module, since it is isomorphic to $\sh{O}_X\otimes_K K'$ \sref{II.1.5.2}.
It then follows from the hypothesis and from \sref{II.6.1.11} and \sref{II.5.5.5}[(ii)] that $X'$ is projective (resp. quasi-projective) \emph{over $K$};
we then deduce from \sref{II.6.6.4} that $X$ is projective (resp. quasi-projective) over $K$.
\end{proof}

In Chapter~V, we will show that the statement of \sref{II.6.6.5} remains true when $K'$ is an \emph{arbitrary} extension of $K$.

The rest of this section is dedicated to the proof of the criterion in \sref{II.6.6.11}, which is a rather technical refinement of \sref{II.6.6.1};
it can be omitted on a first reading.

\begin{lemma}[6.6.6]
\label{II.6.6.6}
Let $X$ be a reduced Noetherian prescheme, and $\sh{E}$ a coherent $\sh{O}_X$-module such that $\sh{E}\otimes_{\sh{O}_X}\sh{R}(X)$ is a locally free $\sh{R}(X)$-module of constant rank~$n$.
Then there exists a reduced Noetherian prescheme $Z$ and a birational finite morphism $h:Z\to X$ that has the following property: the morphisms of sheaves of sets $\sigma_i:\shHom_{\sh{O}_X}(\sh{E},\sh{E})\to\sh{R}(X)$ (for $1\leq i\leq n$) (cf. \sref{II.6.4.10}) send $\shHom_{\sh{O}_X}(\sh{E},\sh{E})$ to the coherent $\sh{O}_X$-algebra $h_*(\sh{O}_Z)$.
\end{lemma}

\begin{proof}
Consider an affine open $U$ of $X$, of ring $A(U)=A$;
let $E=\Gamma(U,\sh{E})$, and let $C_U$ be the subalgebra of $R(U)$ generated by the $\sigma_i(u)$ where $u$ runs over $\Hom_A(E,E)$;
we have seen \sref{II.6.4.7.1} that this $A$-algebra is of \emph{finite rank}.
Furthermore, it is clear that forming the algebras $C_U$ commutes with the restriction operations of an affine open $U$ to an affine open $U'\subset U$.
We have thus defined a \emph{finite} sub-$\sh{O}_X$-algebra $\sh{C}$ of $\sh{R}(X)$ such that $\Gamma(U,\sh{C})=C_U$ for every affine open $U$ of $X$.
We take $Z=\Spec(\sh{C})$, and take $h$ to be the structure morphism, which is thus finite \sref{II.6.1.2};
since $\sh{C}$ is reduced, $Z$ is a reduced Noetherian prescheme \sref{II.1.3.8}.
Finally, the total ring of fractions of $C_U$ is $R(U)$, by definition, and since $C_U$ is contained inside the integral closure of $A(U)$ in $R(U)$, there is a bijective correspondence between minimal prime ideals of $A(U)$ and minimal prime ideals of $C_U$ \cite[t.~I, p.~259]{I-13}, which proves that $h$ is birational and finishes the proof.
\end{proof}

\begin{corollary}[6.6.7]
\label{II.6.6.7}
Under the hypotheses of \sref{II.6.6.6}, let $W$ be an open of $X$ such that, for all $x\in W$, either $X$ is normal at the point $x$, or $\sh{E}_x$ is a free $\sh{O}_x$-module.
Then we can suppose $h$ to be defined such that the restriction of $h$ to $h^{-1}(W)$ is an isomorphism from $h^{-1}(W)$ to $W$.
\end{corollary}

\begin{proof}
Either hypothesis implies that, if $U\subset W$ is an affine open, then, with the notation of \sref{II.6.6.6}, $(\sigma_i(u))_x\in A_x$ for all $x\in U$ \sref{II.6.4.3}, and so $\sigma_i(u)\in A$, and the conclusion follows from the definition of $h$ given in \sref{II.6.6.6}.
\end{proof}

\begin{env}[6.6.8]
\label{II.6.6.8}
Let $X$ be a reduced Noetherian prescheme, and $g:X'\to X$ a finite surjective morphism, so that $\sh{B}=g_*(\sh{O}_{X'})$ is a \emph{coherent} $\sh{O}_X$-algebra;
suppose further that $\sh{B}\otimes_{\sh{O}_X}\sh{R}(X)$ is a \emph{locally free} $\sh{R}(X)$-module \emph{of constant rank~$n$}.
We can then apply Lemma~\sref{II.6.6.6}, taking $\sh{E}=\sh{B}$, whence, with the notation of \sref{II.6.6.6}, we obtain a homomorphism of sheaves of multiplicative monoids $\sigma_n:\shHom_{\sh{O}_X}(\sh{B},\sh{B})\to h_*(\sh{O}_Z)$, and by composing this homomorphism with the canonical homomorphism $\sh{B}\to\shHom_{\sh{O}_X}(\sh{B},\sh{B})$ \sref{II.6.5.1}, we thus obtain a homomorphism of sheaves of multiplicative monoids:
\[
\label{II.6.6.8.1}
  N' : \sh{B} = g_*(\sh{O}_{X'}) \to h_*(\sh{O}_Z) = \sh{C}.
\tag{6.6.8.1}
\]

With this in mind, for every invertible $\sh{O}_{X'}$-module $\sh{L}'$, $g_*(\sh{L}')$ is an invertible $\sh{B}$-module \sref{II.6.1.12}, and the method of \sref{II.6.5.2} allows us to functorially associate to $\sh{L}'$ an invertible $\sh{C}$-module, which we denote by $N'(g_*(\sh{L}'))$.
\end{env}

\begin{lemma}[6.6.9]
\label{II.6.6.9}
Let $X$ be a reduced Noetherian prescheme, and $g:X'\to X$ a finite surjective morphism such that $g_*(\sh{O}_{X'})\otimes_{\sh{O}_X}\sh{R}(X)$ is a locally free $\sh{R}(X)$-module of constant rank.
Then there exists a reduced Noetherian prescheme $Z$ and a finite birational morphism $h:Z\to X$ that has the following property: for every ample $\sh{O}_{X'}$-module $\sh{L}'$, the invertible $\sh{O}_Z$-module $\sh{M}$ such that $h_*(\sh{M})=N'(g_*(\sh{L}'))$ (using the notation of \sref{II.6.6.8}) is ample.
\end{lemma}

\begin{proof}
Suppose first of all that the homomorphism $\sh{B}\to\sh{B}\otimes_{\sh{O}_X}\sh{R}(X)$ is \emph{injective}.
Define $Z$ and $h$ as in \sref{II.6.6.6} (with $\sh{E}=g_*(\sh{O}_{X'})$).
Let $z\in Z$;
we have to show that there exists an integer $m>0$ and a section $t$ of $\sh{M}^{\otimes m}$ over $Z$ such that $Z_t$ is an affine open that contains $z$ \sref{II.4.5.2}.
Let $x=h(z)$, and let $U$ be an affine open of $X$ that contains $x$;
then $h^{-1}(U)$ is an affine open neighbourhood of $z$, and it suffices to find $t$ such that $z\in Z_t\subset h^{-1}(U)$, since $Z_t$ will then necessarily be affine \sref{II.5.5.8}.
There exists, by hypothesis, an integer $n>0$ and a section $s'$ of ${\sh{L}'}^{\otimes n}$ over $X'$ such that
\[
\label{II.6.6.9.1}
  g^{-1}(x) \subset X'_{s'} \subset g^{-1}(U)
\tag{6.6.9.1}
\]
by \sref{II.4.5.4}.
By definition, $s'$ is also a section of $g_*(\sh{L}')$ over $X$, and it corresponds, as in \sref{II.6.5.2}, to a section $s=N'(s')$ of $N'(g_*(\sh{L}'))$ over $X$.
\oldpage[II]{134}
We will show that, if $t$ is the section $s$ considered as a section of $\sh{M}$ over $Z$, then $t$ is the desired section.
Set
\[
\label{II.6.6.9.2}
  V = X\setminus g(X'\setminus X'_{s'})
\tag{6.6.9.2}
\]
which is an open of $X$ that contains $x$ and is contained in $U$, by \sref{II.6.6.9.1} and \sref{II.6.1.10}.
We will show that
\[
\label{II.6.6.9.3}
  h^{-1}(V) \subset Z_t \subset h^{-1}(U)
\tag{6.6.9.3}
\]
which will finish the proof.
It is equivalent to say that the set $T$ of $y\in X$ such that $s_y$ is invertible contains $V$ and is contained in $U$.
For this, consider first of all an affine open $W$ contained in $V$;
then $g^{-1}(W)$ is an affine open in $X'$, and by \sref{II.6.6.9.2} $s'_{y'}$ is invertible for all $y'\in g^{-1}(W)$;
by the hypotheses on $X$ and $\sh{B}$, we can apply the results of \sref{II.6.4.7}, and we see that, if $y=g(y')$, then $s_y$ is invertible;
in other words, $V\subset T$.
On the other hand, it also follows from \sref{II.6.4.7} that, conversely, if $s_y$ is invertible, then so too is $s'_{y'}$, which implies that $y'\in g^{-1}(U)$ by \sref{II.6.6.9.1}, and so $y\in U$, whence $T\subset U$ in this case.

We pass from this to the general case by the same argument as in \sref{II.6.6.2}, replacing $X'$ by $X''$ such that $g'_*(\sh{O}_{X''})\to g'_*(\sh{O}_{X''})\otimes_{\sh{O}_X}\sh{R}(X)$ is injective, and $\sh{L}'$ by an ample $\sh{O}_{X''}$-module $\sh{L}''$ such that $N'(g_*(\sh{L}'))=N'(g'_*(\sh{L}''))$.
Lemma~\sref{II.6.6.9} is then proven in all cases (with a suitable choice of $h$).
\end{proof}

\begin{corollary}[6.6.10]
\label{II.6.6.10}
Suppose that the hypotheses of \sref{II.6.6.9} are satisfied;
for every invertible $\sh{O}_X$-module $\sh{L}$ such that $g^*(\sh{L})$ is ample, $h^*(\sh{L})$ is ample.
\end{corollary}

\begin{proof}
If
\end{proof}


% \subsection{Chevalley's theorem}
% \label{subsection:II.6.7}

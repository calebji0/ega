\section{Integral morphisms and finite morphisms}
\label{section:II.6}


\subsection{Preschemes integral over another prescheme}
\label{subsection:II.6.1}

\begin{definition}[6.1.1]
\label{II.6.1.1}
Let $X$ be an $S$-prescheme, with structure morphism $f:X\to S$.
We say that $X$ is \emph{integral over $S$}, or that $f$ is an \emph{integral morphism}, if there exists a cover $(S_\alpha)$ of $S$ by affine opens such that, for all $\alpha$, the induced prescheme $f^{-1}(S_\alpha)$ is an affine scheme whose ring $B_\alpha$ is an integral algebra over the ring $A_\alpha$ of $S_\alpha$.
We say that $X$ is \emph{finite over $S$}, or that $f$ is a \emph{finite morphism} if $X$ is integral and of finite type over $S$.
\end{definition}

If $S$ is affine of ring $A$, then we also say ``integral (resp. finite) over $A$'' instead of ``integral (resp. finite) over $S$''.

\begin{env}[6.1.2]
\label{II.6.1.2}
It is clear that, if $X$ is integral over $S$, then it is \emph{affine} over $S$.
For an affine prescheme $X$ over $S$ to be integral (resp. finite) over $S$ it is necessary and sufficient that the associated quasi-coherent $\sh{O}_S$-algebra $\sh{A}(X)$ be such that there exist a cover $(S_\alpha)$ of $S$ by affine opens having the property that, for all $\alpha$, $\Gamma(S_\alpha,\sh{A}(X))$ is an integral (resp. integral and of finite type) algebra over $\Gamma(S_\alpha,\sh{O}_S)$.
A quasi-coherent $\sh{O}_S$-algebra with this property is said to be \emph{integral} (resp. \emph{finite}) over $\sh{O}_S$.
Giving an integral (resp. finite) prescheme over $S$ is thus \sref{II.1.3.1} the same as giving a quasi-coherent $\sh{O}_S$-algebra that is integral (resp. finite) over $\sh{O}_S$.
Note that a quasi-coherent $\sh{O}_S$-algebra $\sh{B}$ is finite if and only if it is an \emph{$\sh{O}_S$-module of finite type} \sref[I]{I.1.3.9};
it is equivalent to say that $\sh{B}$ is an \emph{integral} $\sh{O}_S$-algebra \emph{of finite type}, since an algebra that is integral and of finite type over a ring $A$ is an $A$-module of finite type.
\end{env}

\begin{proposition}[6.1.3]
\label{II.6.1.3}
Let $S$ be a locally Noetherian prescheme.
For an affine prescheme $X$ over $S$ to be finite over $S$, it is necessary and sufficient that the $\sh{O}_S$-algebra $\sh{A}(X)$ be coherent.
\end{proposition}

\begin{proof}
Taking the preceding remark into account, this reduces to noting that, if $S$ is locally Noetherian, then the quasi-coherent $\sh{O}_S$-modules of finite type are exactly the coherent $\sh{O}_S$-modules \sref[I]{I.1.5.1}.
\end{proof}

\begin{proposition}[6.1.4]
\label{II.6.1.4}
Let $X$ be an integral (resp. finite) prescheme over $S$, with structure morphism $f:X\to S$.
Then, for every affine open $U\subset S$ of ring $A$, $f^{-1}(U)$ is an affine scheme whose ring $B$ is an integral (resp. finite) algebra over $A$.
\end{proposition}

\oldpage[II]{111}
\begin{proof}
We first prove the following lemma:

  \begin{lemma}[6.1.4.1]
  \label{II.6.1.4.1}
  Let $A$ be a ring, $M$ an $A$-module, and $(g_i)_{1\leq i\leq m}$ a finite system of elements of $A$ such that the $D(g_i)$ (for $1\leq i\leq m$) cover $\Spec(A)$.
  If, for all $i$, $M_{g_i}$ is an $A_{g_i}$-module of finite type, then $M$ is an $A$-module of finite type.
  \end{lemma}

  \begin{proof}
  We can assume that $M_{g_i}$ admits a finite system of generators $(m_{ij}/g_i^n)$ with $m_{ij}\in M$, with $n$ the same for all indices $i$.
  We will show that the $m_{ij}$ for a system of generators of $M$.
  By hypothesis, for each $i$, there exist $a_{ij}\in A$ and some integer $p$ (independent of $i$) such that, in $M_{g_i}$, $m/1=(\sum_i a_{ij}m_{ij})/g_i^p$;
  this implies that there exists an integer $r\geq p$ such that, for all $i$, we have $g_i^rm\in M'$.
  But, since the $D(g_i^r)=D(g_i)$ cover $\Spec(A)$, the ideal of $A$ generated by the $g_i^r$ is equal to $A$, or, in other words, there exist elements $a_i\in A$ such that $\sum_i a_ig_i^r$;
  then $m=(\sum_i a_i g_i^r)m\in M'$, whence the lemma.
  \end{proof}

Now we already know \sref{II.1.3.2} that $f^{-1}(U)$ is affine.
If $\vphi$ is the homomorphism $A\to B$ corresponding to $f$, then there exists a finite cover of $U$ by opens $D(g_i)$ (where $g_i\in A$) such that, if $h_i=\vphi(g_i)$, $B_{h_i}$ is an integral (resp. integral and finite) algebra over $A_{g_i}$.
Indeed, there exists a cover of $U$ by affine opens $V_\alpha\subset U$ such that, if $A_\alpha=A(V_\alpha)$, $B_\alpha=A(f^{-1}(V_\alpha))$ is an integral (resp. finite) algebra over $A_\alpha$.
Every $x\in U$ belongs to some $V_\alpha$, so there exists $g\in A$ such that $x\in D(g)\subset V_\alpha$;
if $g_\alpha$ is the image of $g$ in $A_\alpha$, then $A(D(g))=A_g=(A_\alpha)_{g_\alpha}$;
let $h=\vphi(g)$, and let $h_\alpha$ be the image of $g_\alpha$ in $B_\alpha$;
we have
\[
  A(D(h)) = B_h = (B_\alpha)_{h_\alpha}
\]
and, since $B_\alpha$ is integral (resp. finite) over $A_\alpha$, $(B_\alpha)_{h_\alpha}$ is integral (resp. finite) over $(A_\alpha)_{g_\alpha}$.
It now suffices to use the fact that $U$ is quasi-compact to obtain the desired cover.

If we suppose first of all that the $B_{h_i}$ are integral and finite over the $A_{g_i}$, then since $B_{h_i}$ can also be written as $B_{g_i}$ as an $A_{g_i}$-module, Lemma~\sref{II.6.1.4.1} shows that, in this case, $B$ is an $A$-module of finite type.

Now suppose only that each $B_{h_i}$ is integral over $A_{g_i}$;
let $b\in B$, and let $C$ be the sub-$A$-algebra of $B$ generated by $b$.
For all $i$, $C_{h_i}$ is the algebra over $A_{g_i}$ generated by $b/1$ in $B_{h_i}$;
it follows from the hypothesis that each $C_{h_i}$ is an $A_{g_i}$-module of finite type, and so \sref{II.6.1.4.1} $C$ is an $A$-module of finite type, which proves that $B$ is integral over $A$.
\end{proof}

\begin{proposition}[6.1.5]
\label{II.6.1.5}
\medskip\noindent
\begin{enumerate}
  \item[(i)] A closed immersion is finite (and \emph{a fortiori} integral).
  \item[(ii)] The composition of two finite (resp. integral) morphisms is finite (resp. integral).
  \item[(iii)] If $f:X\to Y$ is a finite (resp. integral) $S$-morphism, then $f_{(S')}:X_{(S')}\to Y_{(S')}$ is finite (resp. integral) for any base extension $S'\to S$.
  \item[(iv)] If $f:X\to Y$ and $g:X'\to Y'$ are finite (resp. integral) $S$-morphisms, then $f\times_S g:X\times_S Y\to X'\times_S Y'$ is finite (resp. integral).
  \item[(v)] If $f:X\to Y$ and $g:Y\to Z$ are morphisms such that $g\circ f$ is finite (resp. integral), if $g$ is separated, then $f$ is finite (resp. integral).
  \item[(vi)] If $f:X\to Y$ is a finite (resp. integral) morphism, then $f_\red$ is finite (resp. integral).
\end{enumerate}
\end{proposition}

\oldpage[II]{112}
\begin{proof}
By \sref[I]{I.5.5.12}, it suffices to prove (i), (ii), and (iii).
To prove that a closed immersion $X\to S$ is finite, we can restrict to the case where $S=\Spec(A)$, and everything then follows from noting that a quotient ring $A/\mathfrak{J}$ is a monogeneous $A$-module.
To prove that the composition of two finite (resp. integral) morphism $X\to Y$, $Y\to Z$ is finite (resp. integral), we can again assume that $Z$ (and thus $X$ and $Y$ \sref{II.1.3.4}) is affine, and then the claim is equivalent to saying that, if $B$ is a finite (resp. integral) $A$-algebra and $C$ a finite (resp. integral) $B$-algebra, then $C$ is a finite (resp. integral) $A$-algebra, which is immediate.
Finally, to prove (iii), we can restrict to the case where $S=Y$, since $X_{(S')}$ can be identified with $X\times_Y Y_{(S')}$ \sref[I]{I.3.3.11};
we can further suppose that $S=\Spec(A)$ and $S'=\Spec(A')$;
then $X$ is affine of ring $B$ \sref{II.1.3.4}, and $X_{(S')}$ affine of ring $A'\otimes_A B$, and it suffices to note that, if $B$ is a finite (resp. integral) $A$-algebra, then $A'\otimes_A B$ is a finite (resp. integral) $A'$-algebra.
\end{proof}

We also note that, if $X$ and $Y$ are $S$-preschemes that are finite (resp. integral) over $S$, then their \emph{sum} $X\sqcup Y$ is a finite (resp. integral) prescheme over $S$, since this reduces to showing that, if $B$ and $C$ are finite (resp. integral) $A$-algebras over $A$, then so too is $B\times C$.

\begin{corollary}[6.1.6]
\label{II.6.1.6}
If $X$ is an integral (resp. finite) prescheme over $S$, then, for every open $U\subset S$, $f^{-1}(U)$ is integral (resp. finite) over $U$.
\end{corollary}

\begin{proof}
This is a particular case of \sref{II.6.1.5}[(iii)].
\end{proof}

\begin{corollary}[6.1.7]
\label{II.6.1.7}
Let $f:X\to Y$ be a finite morphism.
Then, for all $y\in Y$, the fibre $f^{-1}(y)$ is a finite algebraic scheme over $\kres(y)$, and \emph{a fortiori} its underlying space is discrete and finite.
\end{corollary}

\begin{proof}
Indeed, as a $\kres(y)$-prescheme, $f^{-1}(y)$ can be identified with $X\times_Y\Spec(\kres(y))$ \sref[I]{I.3.6.1}, which is finite over $\Spec(\kres(y))$ \sref{II.6.1.5}[(iii)];
it is thus an affine scheme whose ring is an algebra of finite rank over $\kres(y)$ \sref{II.6.1.4}.
The corollary then follows from \sref[I]{I.6.4.4}.
\end{proof}

\begin{corollary}[6.1.8]
\label{II.6.1.8}
Let $X$ and $S$ be integral preschemes, and $f:X\to S$ a \emph{dominant} morphism.
If $f$ is integral (resp. finite) then the field $R(X)$ of rational functions on $X$ is algebraic (resp. algebraic of finite degree) over the field $R(S)$ of rational functions on $S$.
\end{corollary}

\begin{proof}
Let $s$ be the generic point of $S$;
the $\kres(s)$-prescheme $f^{-1}(s)$ is integral (resp. finite) over $\Spec(\kres(s))$ \sref{II.6.1.5}[(iii)] and contains, by hypothesis, the generic point $x$ of $X$;
since the local ring of $x$ in $f^{-1}(s)$ is equal to $\kres(x)$ \sref[I]{I.3.6.5}, and is thus a local ring of an integral (resp. finite) algebra over $\kres(s)$ \sref{II.6.1.4}, whence the corollary.
\end{proof}

\begin{remark}[6.1.9]
\label{II.6.1.9}
The hypothesis that $g$ be \emph{separated} is essential for the validity of \sref{II.6.1.5}[(v)]: if $Y$ is not separated over $S$, then the identity $1_Y$ is the composite morphism $Y\xrightarrow{\Delta_Y}Y\times_Z Y\xrightarrow{p_1}Y$, but $\Delta_Y$ is not an integral morphism, as follows from \sref{II.6.1.10}:
\end{remark}

\begin{proposition}[6.1.10]
\label{II.6.1.10}
Every integral morphism is universally closed.
\end{proposition}

\begin{proof}
Let $f:X\to Y$ be an integral morphism;
by \sref{II.6.1.5}[(iii)], it suffices to show that $f$ is \emph{closed}.
Let $Z$ be a closed subset of $X$;
\oldpage[II]{113}
then there exists a subprescheme of $X$ whose underlying space is $Z$ \sref[I]{I.5.4.1}, and it thus follows from \sref{II.6.1.5}[(i) and (ii)] that it suffices to prove that $f(X)$ is \emph{closed} in $Y$.
By \sref{II.6.1.5}[(vi)], we can suppose that $X$ and $Y$ are \emph{reduced};
further, if $T$ is the closed reduced subprescheme of $Y$ whose underlying space is $\overline{f(X)}$ \sref[I]{I.5.2.1} then we know that $f$ factors as $X\to T\xrightarrow{j}Y$, where $j$ is the injection morphism \sref[I]{I.5.2.2}, and since $j$ is separated \sref[I]{I.5.5.1}[(i)], it follows from \sref{II.6.1.5}[(v)] that $g$ is an integral morphism.
We can thus suppose that $f(X)$ is \emph{dense} in $Y$.
Finally, since the question is local on $Y$, we can restrict to the case where $Y=\Spec(A)$.
Then $X=\Spec(B)$, where $B$ is an $A$-algebra that is integral over $A$ \sref{II.6.1.4};
furthermore, $A$ is reduced \sref[I]{I.5.1.4}, and the hypothesis that $f(X)$ be dense in $Y$ implies that the homomorphism $\vphi:A\to B$ corresponding to $f$ is \emph{injective} \sref[I]{I.1.2.7}.
Under these conditions, saying that $f(X)=Y$ implies that every prime ideal of $A$ is the intersection with $A$ of a prime ideal of $B$, which is exactly the first theorem of Cohen--Seidenberg (\cite[t.~I, p.~257, th.~3]{II-13}).
\end{proof}

\begin{corollary}[6.1.11]
\label{II.6.1.11}
Every finite morphism $f:X\to Y$ is projective.
\end{corollary}

\begin{proof}
Since $f$ is affine, $\sh{O}_X$ is a \emph{very ample} $\sh{O}_X$-module with respect to $f$ \sref{II.5.1.2};
furthermore, $f_*(\sh{O}_X)$ is a quasi-coherent $\sh{O}_Y$-module \emph{of finite type} \sref{II.6.1.2};
finally, $f$ is separated, of finite type, and universally closed \sref{II.6.1.10}, and thus satisfies the conditions of criterion~\sref{II.5.5.4}[(i)].
\end{proof}

\begin{proposition}[6.1.12]
\label{II.6.1.12}
Let $f:X'\to X$ be a finite morphism, and let $\sh{B}=f_*(\sh{O}_{X'})$ (which is a quasi-coherent $\sh{O}_X$-algebra, and a $\sh{O}_X$-module of finite type).
Let $\sh{F}'$ be a quasi-coherent $\sh{O}_{X'}$-module;
for $\sh{F}'$ to be locally free of rank~$r$, it is necessary and sufficient that $f_*(\sh{F}')$ be a locally free $\sh{B}$-module of rank~$r$.
\end{proposition}

\begin{proof}
It is clear that, if $f_*(\sh{F}')|U$ is isomorphic to $\sh{B}^r|U$ (where $U\subset X$ is open), then $\sh{F}'|f^{-1}(U)$ is isomorphic to $\sh{O}_{X'}^r|f^{-1}(U)$ \sref{II.1.4.2}.
Conversely, suppose that $\sh{F}'$ is locally free of rank~$r$; we will show that $f_*(\sh{F}')$ is locally isomorphic to $\sh{B}^r$ as a $\sh{B}$-module.
Let $x$ be a point of $X$;
as $U$ runs over a fundamental system of affine neighbourhoods of $x$, $f^{-1}(U)$ runs over a fundamental system of affine neighbourhoods \sref{II.1.2.5} of the finite set $f^{-1}(x)$, since $f$ is closed \sref{II.6.1.10}.
The proposition then follows from the following lemma:
\end{proof}

\begin{lemma}[6.1.12.1]
\label{II.6.1.12.1}
Let $Y$ be a prescheme, $\sh{E}$ a locally free $\sh{O}_Y$-module of rank~$r$, and $Z$ a finite subset of $Y$ contained inside some affine open $V$.
Then there exists a neighbourhood $U\subset V$ of $Z$ such that $\sh{E}|U$ is isomorphic to $\sh{O}_Y^r|U$.
\end{lemma}

\begin{proof}
We can evidently assume that $Y$ is affine;
for all $z_i\in Z$, there exists in the closure $\overline{\{z_i\}}$ at least one closed point $z'_i$ \sref[0]{0.2.1.3};
if $Z'$ is the set of the $z'_i$ then every neighbourhood of $Z'$ is a neighbourhood of $Z$, and we can thus assume that $Z$ is discrete and closed in $Y$.
Consider the closed reduced subprescheme of $Y$ that has $Z$ has its underlying space \sref[I]{I.5.2.1} and let $j:Z\to Y$ be the canonical injection;
$j^*(\sh{E})=\sh{E}\otimes_Y\sh{O}_Z$ is locally free of rank~$r$ on the discrete scheme $Z$, and is thus isomorphic to $\sh{O}_Z^r$;
in other words, there exist $r$ sections $s_i$ (for $1\leq i\leq r$) of $\sh{E}\otimes_Y\sh{O}_Z$ over $Z$ such that the homomorphism $\sh{O}_Z^r\to\sh{E}\otimes_Y\sh{O}_Z$ defined by these sections is bijective.
But $Y=\Spec(A)$ is affine, $Z$ is defined by an ideal $\mathfrak{J}$ of $A$, and we have $\sh{E}=\widetilde{M}$, where $M$ is an $A$-module;
the $s_i$ are elements of $M\otimes_A(A/\mathfrak{J})$ and are thus images of $r$ elements $t_i\in M=\Gamma(Y,\sh{E})$.
\oldpage[II]{114}
For all $z_j\in Z$ there is thus a neighbourhood $V_j$ of $z_j$ such that the restrictions of the $t_i$ to $V_j$ defined an isomorphism $\sh{O}_Y^r|V_j\to\sh{E}|V_j$ \sref[0]{0.5.5.4};
the neighbourhood $U$ given by the union of the $V_j$ is the desired neighbourhood.
\end{proof}

\begin{proposition}[6.1.13]
\label{II.6.1.13}
Let $g:X'\to X$ be an integral morphism of preschemes, $Y$ a normal locally integral prescheme, and $f$ a rational map from $Y$ to $X'$ such that $g\circ f$ is an everywhere defined rational map \sref[I]{I.7.2.1}.
Then $f$ is everywhere defined.
\end{proposition}

\begin{proof}
If $f_1$ and $f_2$ are morphisms (from dense open subsets of $Y$ to $X'$) in the class $f$, then it is clear that $g\circ f_1$ and $g\circ f_2$ are equivalent morphisms, which justifies the notation $g\circ f$ for their equivalence class.
We recall also that, if we further suppose $Y$ to be \emph{locally Noetherian}, then the hypothesis that $Y$ is normal already implies that $Y$ is locally integral \sref[I]{I.6.1.13}.

To prove the proposition, note first of all that the question is local on $Y$, and so we can suppose that there exists in the class $g\circ f$ a \emph{morphism} $h:Y\to X$.
Consider the inverse image $Y'=X'_{(h)}=X'_{(Y)}$, and note that the morphism $g'=g_{(Y)}:Y'\to Y$ is \emph{integral} \sref{II.6.1.5}[(iii)].
Given the correspondence between rational maps from $Y$ to $X'$ and rational $Y$-sections of $Y'$ \sref[I]{I.7.1.2}, we see that it suffices to prove the specific case of \sref{II.6.1.13} where $X=Y$; in other words, the following:
\end{proof}

\begin{corollary}[6.1.14]
\label{II.6.1.14}
Let $X$ be a normal locally integral prescheme, $g:X'\to X$ an \emph{integral} morphism, and $f$ a rational $X$-section of $X'$.
Then $f$ is everywhere defined.
\end{corollary}

\begin{proof}
Since the question is local on $X$, we can assume that $X$ is integral, and then $f$ is identified with a morphism from an open $U$ of $X$ to $X'$ \sref[I]{I.7.2.2} that is a $U$-section of $g^{-1}(U)$.
Since $g$ is separated, $f$ is a closed immersion from $U$ into $g^{-1}(U)$ \sref[I]{I.5.4.6};
let $Z$ be the closed subprescheme of $g^{-1}(U)$ associated to $f$ \sref[I]{I.4.2.1}, which is isomorphic to $U$, and thus integral;
let $X_1$ be the reduced subprescheme of $X'$ whose underlying space is the closure $\overline{Z}$ of $Z$ in $X'$ \sref[I]{I.5.2.1};
then $Z$ is an induced subprescheme on an open of $X_1$ \sref[I]{I.5.2.3}, and, since it is irreducible, so too is $X_1$, which is thus integral.
The morphism $f$ can then be considered as a rational $X$-section of $X_1$;
since the restriction of $g$ to $X_1$ is an integral morphism \sref{II.6.1.5}[(i) and (ii)], we can finally reduce to proving \sref{II.6.1.14} in the specific case where $X'=X_1$;
in other words, the following:
\end{proof}

\begin{corollary}[6.1.15]
\label{II.6.1.15}
Let $X$ be a normal integral prescheme, $X'$ an integral prescheme, and $g:X'\to X$ an \emph{integral} morphism.
If there exists a rational $X$-section $f$ of $X'$, then $g$ is an isomorphism.
\end{corollary}

\begin{proof}
Since the question is local on $X$, we can assume that $X$ is affine of integral ring $A$, and then $X'$ is affine of ring $A'$ with $A'$ integral over $A$ \sref{II.6.1.4} and integral;
furthermore, the argument of \sref{II.6.1.14} shows that there exists a dense open of $X$ that is isomorphic to a dense open of $X'$, and so $A$ and $A'$ have the same field of fractions.
Also, by \sref[I]{I.8.2.1.1}, and the hypothesis that the $\sh{O}_x$ are integrally closed, the ring $A$ is integrally closed, and so $A'=A$, which finishes the proof of \sref{II.6.1.13}
\end{proof}


\subsection{Quasi-finite morphisms}
\label{subsection:II.6.2}

\begin{proposition}[6.2.1]
\label{II.6.2.1}
Let $f:X\to Y$ be morphism locally of finite type, and $x$ a point of $X$.
Then the following conditions are equivalent:
\oldpage[II]{115}
\begin{enumerate}
  \item[a)] The point $x$ is isolated in its fibre $f^{-1}(f(x))$.
  \item[b)] The ring $\sh{O}_x$ is a quasi-finite $\sh{O}_{f(x)}$-module \sref[0]{0.7.4.1}.
\end{enumerate}
\end{proposition}

\begin{proof}
The
\end{proof}


% \subsection{Integral closure of a prescheme}
% \label{subsection:II.6.3}


% \subsection{Determinant of an endomorphism of $\mathcal{O}_X$-modules}
% \label{subsection:II.6.4}


% \subsection{Norm of an invertible sheaf}
% \label{subsection:II.6.5}


% \subsection{Application: criteria for ampleness}
% \label{subsection:II.6.6}


% \subsection{Chevalley's theorem}
% \label{subsection:II.6.7}

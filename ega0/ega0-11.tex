\section{Supplement on homological algebra}
\label{section:0.11}

\subsection{Review of spectral sequences}
\label{subsection:0.11.1}

\begin{env}[11.1.1]
\label{0.11.1.1}
In the following, we use a more general notion of a spectral sequence than that defined in (T, 2.4); keeping the notation of (T, 2.4), we call a \emph{spectral sequence} in an abelian category $\cat{C}$ a system $E$ consisting of the following parts:
\begin{enumerate}
  \item[(a)] A family $(E_r^{pq})$ of objects of $\cat{C}$ defined for $p,q\in\bb{Z}$ and $r\geq 2$.
  \item[(b)] A family of morphisms $d_r^{pq}:E_r^{pq}\to E_r^{p+r,q-r+1}$ such that $d_r^{p+r,q-r+1}d_r^{pq}=0$.
    We set $Z_{r+1}(E_r^{pq})=\Ker(d_r^{pq})$ and $B_{r+1}(E_r^{pq})=\Im(d_r^{p+r,q-r+1})$, so that
    \[
      B_{r+1}(E_r^{pq})\subset Z_{r+1}(E_r^{pq})\subset E_r^{pq}.
    \]
  \item[(c)] A family of isomorphisms $\alpha_r^{pq}:Z_{r+1}(E_r^{pq})/B_{r+1}(E_r^{pq})\isoto E_{r+1}^{pq}$.

  We then define for $k\geq r+1$, by induction on $k$, the subobjects $B_k(E_r^{pq})$ and $Z_k(E_r^{pq})$ as the inverse images, under the canonical morphism $E_r^{pq}\to E_r^{pq}/B_{r+1}(E_r^{pq})$ of the subobjects of this quotient identified via $\alpha_r^{pq}$ with the subobjects $B_k(E_{r+1}^{pq})$ and $Z_k(E_{r+1}^{pq})$ respectively.
It is clear that we then have, up to isomorphism,
  \[
  \label{0.11.1.1.1}
    Z_k(E_r^{pq})/B_k(E_r^{pq})=E_k^{pq}\text{ for }k\geq r+1,
    \tag{11.1.1.1}
  \]
  and, if we set $B_r(E_r^{pq})=0$ and $Z_r(E_r^{pq})=E_r^{pq}$, then we have the inclusion relations
  \[
  \label{0.11.1.1.2}
    0=B_r(E_r^{pq})\subset B_{r+1}(E_r^{pq})\subset B_{r+2}(E_r^{pq})\subset\cdots\subset Z_{r+2}(E_r^{pq})\subset Z_{r+1}(E_r^{pq})\subset Z_r(E_r^{pq})=E_r^{pq}.
    \tag{11.1.1.1.2}
  \]
  The other parts of the data of $E$ are then:
  \item[(d)] Two subobjects $B_\infty(E_2^{pq})$ and $Z_\infty(E_2^{pq})$ of $E_2^{pq}$ such that we have $B_\infty(E_2^{pq})\subset Z_\infty(E_2^{pq})$ and, for every $k\geq 2$,
    \[
      B_k(E_2^{pq})\subset B_\infty(E_2^{pq})\text{ and }Z_\infty(E_2^{pq})\subset Z_k(E_2^{pq}).
    \]
    We set
    \[
    \label{0.11.1.1.3}
      E_\infty^{pq}=Z_\infty(E_2^{pq})/B_\infty(E_2^{pq}).
      \tag{11.1.1.1.3}
    \]
  \item[(e)]
\oldpage[0\textsubscript{III}]{24}
    A family $(E^n)$ of objects of $\cat{C}$, each equipped with a \emph{decreasing filtration $(F^p(E^n))_{p\in\bb{Z}}$}.
    As usual, we denote by $\gr(E^n)$ the graded object associated to the filtered object $E^n$, the direct sum of the $\gr_p(E^n)=F^p(E^n)/F^{p+1}(E^n)$.
  \item[(f)] For every pair $(p,q)\in\bb{Z}\times\bb{Z}$, an isomorphism $\beta^{pq}:E_\infty^{pq}\isoto\gr_p(E^{p+q})$.
\end{enumerate}

The family $(E^n)$, without the filtrations, is called the \emph{abutment} (or \emph{limit}) of the spectral sequence $E$.

Suppose that the category $\cat{C}$ admits infinite direct sums, or that for every $r\geq 2$ and every $n\in\bb{Z}$, there are finitely many pairs $(p,q)$ such that $p+q=n$ and $E_r^{pq}\neq 0$ (it suffices for it to hold for $r=2$).
Then the $E_r^{(n)}=\sum_{p+q=n}E_r^{pq}$ are defined, and we if denote by $d_r^{(n)}$ the morphism $E_r^{(n)}\to E_r^{(n+1)}$ whose restriction to $E_r^{pq}$ is $d_r^{pq}$ for every pair $(p,q)$ such that $p+q=n$, then $d_r^{(n+1)}\circ d_r^{(n)}=0$, in other words, $(E_r^{(n)})_{n\in\bb{Z}}$ is a \emph{complex $E_r^{(\bullet)}$} in $\cat{C}$, with differentials of degree $+1$, and it follows from (c) that $\HH^n(E_n^{(\bullet)})$ is \emph{isomorphic to $E_{r+1}^{(n)}$} for every $r\geq 2$.
\end{env}

\begin{env}[11.1.2]
\label{0.11.1.2}
A \emph{morphism $u:E\to E'$} from a spectral sequence $E$ to a spectral sequence $E'=(E_r^{\prime pq},E^{\prime n})$ consists of systems of morphisms $u_r^{pq}:E_r^{pq}\to E_r^{\prime pq}$ and $u^n:E^n\to E^{\prime n}$, the $u^n$ compatible with the filtrations on $E^n$ and $E^{\prime n}$, and the diagrams
\[
  \xymatrix{
    E_r^{pq}\ar[r]^-{d_r^{pq}}\ar[d]_{u_r^{pq}} &
    E_r^{p+r,q-r+1}\ar[d]^{u_r^{p+r,q-r+1}}\\
    E_r^{\prime pq}\ar[r]^-{d_r^{\prime pq}} &
    E_r'^{\prime p+r,q-r+1}
  }
\]
being commutative; in addition, by passing to quotients, $u_r^{pq}$ gives a morphism $\overline{u}_r^{pq}:Z_{r+1}(Z_r^{pq})/B_{r+1}(E_r^{pq})\to Z_{r+1}(E_r^{\prime pq})/B_{r+1}(E_r^{\prime pq})$ and we must have $\alpha_r^{\prime pq}\circ\overline{u}_r^{pq}=u_{r+1}^{pq}\circ\alpha_r^{pq}$; finally, we must have $u_2^{pq}(B_\infty(E_2^{pq}))\subset B_\infty(E_2^{\prime pq})$ and $u_2^{pq}(Z_\infty(E_2^{pq}))\subset Z_\infty(E_2^{\prime pq})$; by passing to quotients, $u_2^{pq}$ then gives a morphism $u_\infty^{\prime pq}:E_\infty^{pq}\to E_\infty^{\prime pq}$, and the diagram
\[
  \xymatrix{
    E_\infty^{pq}\ar[r]^-{u_\infty^{\prime pq}}\ar[d]_{\beta^{pq}} &
    E_\infty^{\prime pq}\ar[d]^{\beta^{\prime pq}}\\
    \gr_p(E^{p+q})\ar[r]^-{\gr_p(u^{p+q})} &
    \gr_p(E^{\prime p+q})
  }
\]
must be commutative.

The above definitions show, by induction on $r$, that if the $u_2^{pq}$ are \emph{isomorphisms}, then so are the $u_r^{pq}$ for $r\geq 2$; \emph{if in addition we know that $u_2^{pq}(B_\infty(E_2^{pq}))=B_\infty(E_2^{\prime pq})$ and $u_2^{pq}(Z_\infty(E_2^{pq}))=Z_\infty(E_2^{\prime pq})$ and the $u^n$ are isomorphisms}, then we can conclude that $u$ is an \emph{isomorphism}.
\end{env}

\begin{env}[11.1.3]
\label{0.11.1.3}
Recall that if $(F^p(X))_{p\in\bb{Z}}$ is a (decreasing) \emph{filtration} of an object $X\in\cat{C}$, then we say that this filtration is \emph{separated} if $\inf(F^p(X))=0$, \emph{discrete} if there exists a $p$ such that $F^p(X)=0$, \emph{exhaustive} (or \emph{coseparated}) if $\sup(F^p(X))=X$, \emph{codiscrete} if there exists a $p$ such that $F^p(X)=X$.

We say that a spectral sequence $E=(E_r^{pq},E^n)$ is \emph{weakly convergent} if we have $B_\infty(E_2^{pq})=\sup_k(B_k(E_2^{pq}))$ and $Z_\infty(E_2^{pq})=\inf_k(Z_k(E_2^{pq}))$ (in other words, the objects of $B_\infty(E_2^{pq})$ and $Z_\infty(E_2^{pq})$ are determined from the data of (a) and (c) of the spectral sequence $E$).
We say that the spectral sequence $E$ is \emph{regular} if it is weakly convergent and if, in addition:
\begin{enumerate}
  \item[(1st)] For every pair $(p,q)$, the decreasing sequence $(Z_k(E_2^{pq}))_{k\geq 2}$ is \emph{stable}; the hypothesis that $E$ is weakly convergent then implies that $Z_\infty(E_2^{pq})=Z_k(E_2^{pq})$ for $k$ large enough (depending on $p$ and $q$).
  \item[(2nd)] For every $n$, the filtration $(F^p(E^n))_{p\in\bb{Z}}$ of $E^n$ is \emph{discrete} and \emph{exhaustive}.
\end{enumerate}

We say that the spectral sequence $E$ is \emph{coregular} if it is weakly convergent and if, in addition:
\begin{enumerate}
  \item[(3rd)] For every pair $(p,q)$, the increasing sequence $(B_k(E_2^{pq}))_{k\geq 2}$ is \emph{stable}, which implies that $B_\infty(E_2^{pq})=B_k(E_2^{pq})$, and as a result, $E_\infty^{pq}=\inf E_k^{pq}$.
  \item[(4th)] For every $n$, the filtration of $E^n$ is \emph{codiscrete}.
\end{enumerate}

Finally, we say that $E$ is \emph{biregular} if it is both regular and coregular, in other words if we have the following conditions:
\begin{enumerate}
  \item[(a)] For every pair $(p,q)$, the sequences $(B_k(E_2^{pq}))_{k\geq 2}$ and $(Z_k(E_2^{pq}))_{k\geq 2}$ are \emph{stable} and we have $B_\infty(E_2^{pq})=B_k(E_2^{pq})$ and $Z_\infty(E_2^{pq})=Z_k(E_2^{pq})$ for $k$ large enough (which implies that $E_\infty^{pq}=E_k^{pq}$).
  \item[(b)] For every $n$, the filatration $(F^p(E^n))_{p\in\bb{Z}}$ is \emph{discrete} and \emph{codiscrete} (which we also call \emph{finite}).
\end{enumerate}

The spectral sequences defined in (T, 2.4) are thus biregular spectral sequences.
\end{env}

\begin{env}[11.1.4]
\label{0.11.1.4}
Suppose that in the category $\cat{C}$, filtered inductive limits exist and the functor $\varinjlim$ is \emph{exact} (which is equivalent to saying that the axiom (AB~5) of (T, 1.5) is satisfied (cf.~T, 1.8)).
The condition that the filtration $(F^p(X))_{p\in\bb{Z}}$ of an object $X\in\cat{C}$ is exhaustive is then expressed as $\varinjlim_{p\to-\infty}F^p(X)=X$.
If a spectral sequence $E$ is weakly convergent, then we have $B_\infty(E_2^{pq})=\varinjlim_{k\to\infty}B_k(E_2^{pq})$; if in addition $u:E\to E'$ is a morphism from $E$ to a weakly convergent spectral sequence $E'$ in $\cat{C}$, then we have $u_2^{pq}(B_\infty(E_2^{pq}))=B_\infty(E_2^{\prime pq})$, by the exactness of $\varinjlim$.
In addition:
\end{env}

\begin{proposition}[11.1.5]
\label{0.11.1.5}
Let $\cat{C}$ be an abelian category in which filtered inductive limits are exact, $E$ and $E'$ two regular spectral sequences in $\cat{C}$, $u:E\to E'$ a morphism of spectral sequences.
If the $u_2^{pq}$ are isomorphisms, then so is $u$.
\end{proposition}

\begin{proof}
We already know \sref{0.11.1.2} that the $u_r^{pq}$ are isomorphisms and that
\[
  u_2^{pq}(B_\infty(E_2^{pq}))=B_\infty(E_2^{\prime pq});
\]
\oldpage[0\textsubscript{III}]{26}
the hypothesis that $E$ and $E'$ are regular also implies that $u_2^{pq}(Z_\infty(E_2^{pq}))=Z_\infty(E_2^{\prime pq})$, and as $u_2^{pq}$ is an isomorphism, so is $u_\infty^{\prime pq}$; we thus conclude that $\gr_p(u^{p+q})$ is also an isomorphism.
But as the filtrations of the $E^n$ and the $E^{\prime n}$ are discrete and exhaustive, this implies that the $u^n$ are also isomorphisms (Bourbaki, \emph{Alg. comm.}, chap.~III, \textsection2, n\textsuperscript{o}8, th.~1).
\end{proof}

\begin{env}[11.1.6]
\label{0.11.1.6}
It follows from (11.1.1.2) and the definition (11.1.1.3) that if, for a spectral sequence $E$, we have $E_r^{pq}=0$, then we have $E_k^{pq}=0$ for $k\geq r$ and $E_\infty^{pq}=0$.
We say that a spectral sequence \emph{degenerates} if there exists an integer $r\geq 2$ and, for every integer $n\in\bb{Z}$, an integer $q(n)$ such that \emph{$E_r^{n-q,q}=0$ for every $q\neq q(n)$}.
We first deduce from the previous remark that we also have $E_k^{n-q,q}=0$ for $k\geq r$ (including $k=\infty$) and $q\neq q(n)$.
In addition, the definition of $E_{r+1}^{pq}$ shows that we have $E_{r+1}^{n-q(n),q(n)}=E_r^{n-q(n),q(n)}$; if $E$ is \emph{weakly convergent}, then we also have $E_\infty^{n-q(n),q(n)}=E_r^{n-q(n),q(n)}$; in other words, for every $n\in\bb{Z}$, $\gr_p(E^n)=0$ for $p\neq q(n)$ and $\gr_{q(n)}(E^n)=E_r^{n-q(n),q(n)}$.
If in addition the filtration of $E^n$ is \emph{discrete} and \emph{exhaustive}, then the spectral sequence $E$ is \emph{regular}, and we have $E^n=E_r^{n-q(n),q(n)}$ up to unique isomorphism.
\end{env}

\begin{env}[11.1.7]
\label{0.11.1.7}
Suppose that filtered inductive limits exist and are exact in the category $\cat{C}$, and let $(E_\lambda,u_{\mu\lambda})$ be an inductive system (over a filtered set of indices) of spectral sequeneces in $\cat{C}$.
Then the \emph{inductive limit} of this inductive system exists in the additive category of spectral sequences of objects of $\cat{C}$: to see this, it suffices to define $E_r^{pq}$, $d_r^{pq}$, $\alpha_r^{pq}$, $B_\infty(E_2^{pq})$, $Z_\infty(E_2^{pq})$, $E^n$, $F^p(E^n)$, and $\beta^{pq}$ as the respective inductive limits of the $E_{r,\lambda}^{pq}$, $d_{r,\lambda}^{pq}$, $\alpha_{r,\lambda}^{pq}$, $B_\infty(E_{2,\lambda}^{pq})$, $Z_\infty(E_{2,\lambda}^{pq})$, $E_\lambda^n$, $F^p(E_\lambda^n)$, and $\beta_\lambda^{pq}$; the verification of the conditions of \sref{0.11.1.1} follows from the exactness of the functor $\varinjlim$ on $\cat{C}$.
\end{env}

\begin{remark}[11.1.8]
\label{0.11.1.8}
Suppose that the category $\cat{C}$ is the category of $A$-modules over a \emph{Noetherian} ring $A$ (resp. a ring $A$).
Then the definitions of \sref{0.11.1.1} show that if, for a given $r$, the $E_r^{pq}$ are $A$-modules \emph{of finite type} (resp. \emph{of finite length}), then so are each of the modules $E_s^{pq}$ for $s\geq r$, hence so is $E_\infty^{pq}$.
If in addition the filtration of the abutment/limit $(E^n)$ is \emph{discrete} or \emph{codiscrete} for all $n$, then we conclude that each of the $E^n$ is also an $A$-module \emph{of finite type} (resp. \emph{of finite length}).
\end{remark}

\begin{env}[11.1.9]
\label{0.11.1.9}
We will have to consider conditions which ensure that a spectral sequence $E$ is biregular is a ``uniform'' way in $p+q=n$.
We will then use the following lemma:
\end{env}

\begin{lemma}[11.1.10]
\label{0.11.1.10}
Let $(E_r^{pq})$ be a family of objects of $\cat{C}$ related by the data of \emph{(a)}, \emph{(b)}, and \emph{(c)} of \sref{0.11.1.1}.
For a fixed integer $n$, the following properties are equivalent:
\begin{enumerate}
  \item[{\rm(a)}] There exists an integer $r(n)$ such that for $r\geq r(n)$, $p+q=n$ or $p+q=n-1$, the morphisms $d_r^{pq}$ are zero.
  \item[{\rm(b)}] There exists an integer $r(n)$ such that for $p+q=n$ or $p+q=n+1$, we have $B_r(E_2^{pq})=B_s(E_2^{pq})$ for $s\geq r\geq r(n)$.
  \item[{\rm(c)}] There exists an integer $r(n)$ such that for $p+q=n$ or $p+q=n-1$, we have $Z_r(E_2^{pq})=Z_s(E_2^{pq})$ for $s\geq r\geq r(n)$.
  \item[{\rm(d)}] There exists an integer $r(n)$ such that for $p+q=n$, we have $B_r(E_2^{pq})=B_s(E_2^{pq})$ and $Z_r(E_2^{pq})=Z_s(E_2^{pq})$ for $s\geq r\geq r(n)$.
\end{enumerate}
\end{lemma}

\begin{proof}
According to the conditions (a), (b), and (c) of \sref{0.11.1.1}, we have that saying $Z_{r+1}(E_2^{pq})=Z_r(E_2^{pq})$ is equivalent to saying that $d_r^{pq}=0$ and that saying $B_r(E_2^{p+r,q-r+1})=B_{r+1}(E_2^{p+r,q-r+1})$ is equivalent to saying that $d_r^{pq}=0$; the lemma immediately follows from this remark.
\end{proof}

\subsection{The spectral sequence of a filtered complex}
\label{subsection:0.11.2}

\begin{env}[11.2.1]
\label{0.11.2.1}
Given an abelian category $\cat{C}$, we will agree to denote by notation such as $K^\bullet$ the \emph{complexes $(K^i)_{i\in\bb{Z}}$} of objects of $\cat{C}$ whose differential is of degree $+1$, and by the notation such as $K_\bullet$ the complexes $(K_i)_{i\in\bb{Z}}$ of objects of $\cat{C}$ whose differential is of degree $-1$.
To each complex $K^\bullet=(K^i)$ whose differential $d$ is of degree $+1$, we can associate a complex $K_\bullet'=(K_i')$ by setting $K_i'=K^{-i}$, the differential $K_i'\to K_{i-1}'$ being the operator $d:K^{-i}\to K^{-i-1}$; and \emph{vice versa}, which, depending on the circumstances, will allow one to consider either one of the types of complexes and translate  any result from one type into results for the other.
We similarly denote by notation such as $K^{\bullet\bullet}=(K^{ij})$ (resp. $K_{\bullet\bullet}=(K_{ij})$) the \emph{bicomplexes} (or \emph{double complexes}) of objects of $\cat{C}$ in which the \emph{two} differentials are of degree $+1$ (resp. $-1$); we can still pass from one type to the other by changing the signs of the indices, and we have similar notation and remarks for any multicomplexes.
The notation $K^\bullet$ and $K_\bullet$ will also be used for \emph{$\bb{Z}$-graded objects} of $\cat{C}$, which are not necessarily complexes (they can be considered as such for the \emph{zero} differentials); for example, we write $\HH^\bullet(K^\bullet)=(\HH^i(K^\bullet))_{i\in\bb{Z}}$ for the \emph{cohomology} of a complex $K^\bullet$ whose differential is of degree $+1$, and $H_\bullet(K_\bullet)=(\HH_i(K_\bullet))_{i\in\bb{Z}}$ for the \emph{homology} of a complex $K_\bullet$ whose differential is of degree $-1$; when we pass from $K^\bullet$ to $K_\bullet'$ by the method described above, we have $\HH_i(K_\bullet')=\HH^{-i}(K^\bullet)$.

Recall in this case that for a complex $K^\bullet$ (resp. $K_\bullet$), we will write in general $Z^i(K^\bullet)=\Ker(K^i\to K^{i+1})$ (``object of cocycles'') and $B^i(K^\bullet)=\Im(K^{i-1}\to K^i)$ (``object of coboundaries'') (resp. $Z_i(K_\bullet)=\Ker(K_i\to K_{i-1})$ (``object of cycles'') and $B_i(K_\bullet)=\Im(K_{i+1}\to K_i)$ (``object of boundaries'')) so that $\HH^i(K^\bullet)=Z^i(K^\bullet)/B^i(K^\bullet)$ (resp. $\HH_i(K_\bullet)=Z_i(K_\bullet)/B_i(K_\bullet)$).

If $K^\bullet=(K^i)$ (resp. $K_\bullet=(K_i)$) is a complex in $\cat{C}$ and $T:\cat{C}\to\cat{C}'$ a functor from $\cat{C}$ to an abelian category $\cat{C}'$, then we denote by $T(K^\bullet)$ (resp. $T(K_\bullet)$) the complex $(T(K^i))$ (resp. $(T(K_i))$) in $\cat{C}'$.

We will not review the definition of the \emph{$\partial$-functors} (T, 2.1), except to note that we \emph{also} say $\partial$-functor in place of $\partial^*$-functor when the morphism $\partial$ decreases the degree of a unit, the context clarifying this point if there is cause for confusion.

Finally, we say that a \emph{graded object $(A_i)_{i\in\bb{Z}}$} of $\cat{C}$ is \emph{bounded below} (resp. \emph{above}) if there exists an $i_0$ such that $A_i=0$ for $i<i_0$ (resp. $i>i_0$).
\end{env}

\begin{env}[11.2.2]
\label{0.11.2.2}
Let $K^\bullet$ be a complex in $\cat{C}$ whose differential $d$ is of degree $+1$, and suppose it is equipped with a \emph{filtration $F(K^\bullet)=(F^p(K^\bullet))_{p\in\bb{Z}}$} consisting of \emph{graded} subobjects
\oldpage[0\textsubscript{III}]{28}
of $K^\bullet$, in other words, $F^p(K^\bullet)=(K^i\cap F^p(K^\bullet))_{i\in\bb{Z}}$; in addition, we assume that $d(F^p(K^\bullet))\subset F^p(K^\bullet)$ for every $p\in\bb{Z}$.
Let us quickly recall how one \emph{functorially} defines a spectral sequence $E(K^\bullet)$ from $K^\bullet$ (M,~XV,~4 and G,~I,~4.3).
For $r\geq 2$, the canonical morphism $F^p(K^\bullet)/F^{p+r}(K^\bullet)\to F^p(K^\bullet)/F^{p+1}(K^\bullet)$ defines a morphism in cohomology
\[
  \HH^{p+q}(F^p(K^\bullet)/F^{p+r}(K^\bullet))\to\HH^{p+q}(F^p(K^\bullet)/F^{p+1}(K^\bullet)).
\]

We denote by $Z_r^{pq}(K^\bullet)$ the image of this morphism.
Similarly, from the exact sequence
\[
  0\to F^p(K^\bullet)/F^{p+1}(K^\bullet)\to F^{p-r+1}(K^\bullet)/F^{p+1}(K^\bullet)\to F^{p-r+1}(K^\bullet)/F^p(K^\bullet)\to 0,
\]
we deduce from the exact sequence in cohomology a morphism
\[
  \HH^{p+q-1}(F^{p-r+1}(K^\bullet)/F^p(K^\bullet))\to\HH^{p+q}(F^p(K^\bullet)/F^{p+1}(K^\bullet)),
\]
and we denote by $B_r^{pq}(K^\bullet)$ the image of this morphism; we show that $B_r^{pq}(K^\bullet)\subset Z_r^{pq}(K^\bullet)$ and we take $E_r^{pq}(K^\bullet)=Z_r^{pq}(K^\bullet)/B_r^{pq}(K^\bullet)$; we will not specify the definition of the $d_r^{pq}$ or the $\alpha_r^{pq}$.

We note here that all the $Z_r^{pq}(K^\bullet)$ and $B_r^{pq}(K^\bullet)$, for $p$ and $q$ fixed, are subobjects of the same object $\HH^{p+q}(F^p(K^\bullet)/F^{p+1}(K^\bullet))$, which we denote by $Z_1^{pq}(K^\bullet)$; we set $B_1^{pq}(K^\bullet)=0$, so that the above definitions of $Z_r^{pq}(K^\bullet)$ and $B_r^{pq}(K^\bullet)$ also work for $r=1$; we still set $E_1^{pq}(K^\bullet)=Z_1^{pq}(K^\bullet)$.
We define $d_1^{pq}$ and $\alpha_1^{pq}$ such that the conditions of \sref{0.11.1.1} are satisfied for $r=1$.
On the other hand, we define the subobjects $Z_\infty^{pq}(K^\bullet)$ as the image of the morphism
\[
  \HH^{p+q}(F^p(K^\bullet))\to\HH^{p+q}(F^p(K^\bullet)/F^{p+1}(K^\bullet))=E_1^{pq}(K^\bullet),
\]
and $B_\infty^{pq}(K^\bullet)$ as the image of the morphism
\[
  \HH^{p+q-1}(K^\bullet/F^p(K^\bullet))\to\HH^{p+q}(F^p(K^\bullet)/F^{p+1}(K^\bullet))=E_1^{pq}(K^\bullet),
\]
induced as above from the exact sequence in cohomology.
We set $Z_\infty(E_2^{pq}(K^\bullet))$ and $B_\infty(E_2^{pq}(K^\bullet))$ to be the canonical images of $E_2^{pq}(K^\bullet)$ in $Z_\infty^{pq}(K^\bullet)$ and $B_\infty^{pq}(K^\bullet)$.

Finally, we denote by $F^p(\HH^n(K^\bullet))$ the image in $\HH^n(K^\bullet)$ of the morphism $\HH^n(F^p(K^\bullet))\to\HH^n(K^\bullet)$ induced from the canonical injection $F^p(K^\bullet)\to K^\bullet$; by the exact sequence in cohomology, this is also the kernel of the morphism $\HH^n(K^\bullet)\to\HH^n(K^\bullet/F^p(K^\bullet))$.
This defines a filtration on $E^n(K^\bullet)=\HH^n(K^\bullet)$; we will not give here the definition of the isomorphisms $\beta^{pq}$.
\end{env}

\begin{env}[11.2.3]
\label{0.11.2.3}
The \emph{functorial} nature of $E(K^\bullet)$ is understood in the following way: given two \emph{filtered} complexes $K^\bullet$ and $K^{\prime\bullet}$ in $\cat{C}$ and a morphism of complexes $u:K^\bullet\to K^{\prime\bullet}$ that is \emph{compatible with the filtrations}, we induce in an evident way the morphisms $u_r^{pq}$ (for $r\geq 1$) and $u^n$, and we show that these morphisms are compatible with the $d_r^{pq}$, $\alpha_r^{pq}$, and $\beta^{pq}$ in the sense of \sref{0.11.1.2}, and thus given a well-defined morphism $E(u):E(K^\bullet)\to E(K^{\prime\bullet})$ of spectral sequences.
In addition, we show that if $u$ and $v$ are morphisms $K^\bullet\to K^{\prime\bullet}$ of the above type, \emph{homotopic in degree $\leq k$}, then $u_r^{pq}=v_r^{pq}$ for $r>k$ and $u^n=v^n$ for all $n$ (M,~XV,~3.1).
\end{env}

\begin{env}[11.2.4]
\label{0.11.2.4}
\oldpage[0\textsubscript{III}]{29}
Suppose that filtered inductive limits in $\cat{C}$ are exact.
Then if the filtration $(F^p(K^\bullet))$ of $K^\bullet$ is \emph{exhaustive}, then so is the filatration $(F^p(\HH^n(K^\bullet)))$ for all $n$, since by hypothesis we have $K^\bullet=\varinjlim_{p\to-\infty}F^p(K^\bullet)$ and since the hypothesis on $\cat{C}$ implies that cohomology commutes with inductive limits.
In addition, for the same reason, we have $B_\infty(E_2^{pq}(K^\bullet))=\sup_k B_k(E_2^{pq}(K^\bullet))$.
We say that the filtration $(F^p(K^\bullet))$ of $K^\bullet$ is \emph{regular} if for every $n$ there exists an integer $u(n)$ such that $\HH^n(F^p(K^\bullet))=0$ for $p>u(n)$.
This is particularly the case when the filtration of $K^\bullet$ is \emph{discrete}.
When the filtration of $K^\bullet$ is regular and exhaustive, and filtered inductive limits are exact in $\cat{C}$, we have (M,~XV,~4) that the spectral sequence $E(K^\bullet)$ is \emph{regular}.
\end{env}

\subsection{The spectral sequences of a bicomplex}
\label{subsection:0.11.3}

\begin{env}[11.3.1]
\label{0.11.3.1}
With regard the conventions for bicomplexes, we follow those of~(T,~2.4) rather than those of~(M), the two differentials $d'$ and $d''$ (of degree~$+1$) of such a bicomplex $K^{\bullet\bullet}=(K^{ij})$ thus being assumed to be \emph{permutable}.
Suppose that \emph{one} of the following two conditions is satisfied:
\begin{enumerate}
   \item \emph{Infinite direct sums} exist in $\cat{C}$;
   \item For all $n\in\bb{Z}$, there is only a \emph{finite} number of pairs $(p,q)$ such that $p+q=n$ and $K^{pq}\neq 0$.
 \end{enumerate}
Then the bicomplex $K^{\bullet\bullet}$ defines a (simple) \emph{complex} $(K^{\prime n})_{n\in\bb{Z}}$ with $K^{\prime n}=\sum_{i+j=n}K^{ij}$, the differential $d$ (of degree~$+1$) of this complex being given by $dx=d'x+(-1)^i d''x$ for $x\in K^{ij}$.
When we later speak of the spectral sequence of a (simple) \emph{complex} that is \emph{defined by a bicomplex $K^{\bullet\bullet}$}, it will always be understood that one of the above conditions is satisfied.
We adopt the analogous conventions for multicomplexes.

We denote by $K^{i,\bullet}$ (resp.~$K^{\bullet,j}$) the simple complex $(K^{ij})_{j\in\bb{Z}}$ (resp.~$(K^{ij})_{i\in\bb{Z}}$), by $Z_\mathrm{II}^p(K^{i,\bullet})$, $B_\mathrm{II}^p(K^{i,\bullet})$, $\HH_\mathrm{II}^p(K^{i,\bullet})$ (resp.~$Z_\mathrm{I}^p(K^{\bullet,j})$, $B_\mathrm{I}^p(K^{\bullet,j})$, $\HH_\mathrm{I}^p(K^{\bullet,j})$) its $p$ objects of cocycles, of coboundaries, and of cohomology, respectively;
the differential $d':K^{i,\bullet}\to K^{i+1,\bullet}$ is a morphism of complexes, which thus gives an operator on the cocycles, coboundaries, and cohomology,
\begin{align*}
  d'&:Z_\mathrm{II}^p(K^{i,\bullet})\to Z_\mathrm{II}^p(K^{i+1,\bullet})
\\d'&:B_\mathrm{II}^p(K^{i,\bullet})\to B_\mathrm{II}^p(K^{i+1,\bullet})
\\d'&:\HH_\mathrm{II}^p(K^{i,\bullet})\to\HH_\mathrm{II}^p(K^{i+1,\bullet})
\end{align*}
and it is clear that for these operators, $(Z_\mathrm{II}^p(K^{i,\bullet}))_{i\in\bb{Z}}$, $(B_\mathrm{II}^p(K^{i,\bullet}))_{i\in\bb{Z}}$, and $(\HH_\mathrm{II}^p(K^{i,\bullet}))_{i\in\bb{Z}}$ are complexes;
we denote the complex $(\HH_\mathrm{II}^p(K^{i,\bullet}))_{i\in\bb{Z}}$ by $\HH_\mathrm{II}^p(K^{\bullet\bullet})$, and its $q$ objects of cocycles, coboundaries, and cohomology by $Z_\mathrm{I}^q(\HH_\mathrm{II}^p(K^{\bullet\bullet}))$, $B_\mathrm{I}^q(\HH_\mathrm{II}^p(K^{\bullet\bullet}))$, and $\HH_\mathrm{I}^q(\HH_\mathrm{II}^p(K^{\bullet\bullet}))$.
We similarly define the complexes $\HH_\mathrm{I}^p(K^{\bullet\bullet})$ and their cohomology objects $\HH_\mathrm{II}^q(\HH_\mathrm{I}^p(K^{\bullet\bullet}))$.
Recall, however, that $\HH^n(K^{\bullet\bullet})$ denotes the $n$ object of the cohomology of the \emph{(simple)} complex defined by $K^{\bullet\bullet}$.
\end{env}

\begin{env}[11.3.2]
\label{0.11.3.2}
On the complex defined by a bicomplex $K^{\bullet\bullet}$, we can consider two canonical filtrations, $(F_\mathrm{I}^p(K^{\bullet\bullet}))$ and $(F_\mathrm{II}^p(K^{\bullet\bullet}))$, given by
\[
\label{0.11.3.2.1}
  F_\mathrm{I}^p(K^{\bullet\bullet})
  = \left(\sum_{i+j=n,i\geq p}K^{ij}\right)_{n\in\bb{Z}}
  \quad\text{and}\quad
  F_\mathrm{II}^p(K^{\bullet\bullet})
  = \left(\sum_{i+j=n,j\geq p}K^{ij}\right)_{n\in\bb{Z}}
  \tag{11.3.2.1}
\]
\oldpage[0\textsubscript{III}]{30}
which, by definition, are graded subobjects of the (simple) complex defined by $K^{\bullet\bullet}$, and thus make this complex a filtered complex;
moreover, is is clear that these filtrations are \emph{exhaustive} and \emph{separated}.

There corresponds to each of these filtrations a spectral sequence~\sref{0.11.2.2};
we denote by $'E(K^{\bullet\bullet})$ and $''E(K^{\bullet\bullet})$ the spectral sequences corresponding to $(F_\mathrm{I}^p(K^{\bullet\bullet}))$ and $(F_\mathrm{II}^p(K^{\bullet\bullet}))$ respectively, called the \emph{spectral sequence of the bicomplex $K^{\bullet\bullet}$}, and both having as their abutment the cohomology $(\HH^n(K^{\bullet\bullet}))$.
We can further show~(M,~XV,~6) that we have
\[
\label{0.11.3.2.2}
  'E_2^{pq}(K^{\bullet\bullet})
  = \HH_\mathrm{I}^p(\HH_\mathrm{II}^p(K^{\bullet\bullet}))
  \quad\text{and}\quad
  ''E_2^{pq}(K^{\bullet\bullet}))
  = \HH_\mathrm{II}^p(\HH_\mathrm{I}^p(K^{\bullet\bullet})).
  \tag{11.3.2.2}
\]

Every morphism $u:K^{\bullet\bullet}\to K^{\prime\bullet\bullet}$ of bicomplexes is \emph{ipso facto} compatible with the filtrations of the same type of $K^{\bullet\bullet}$ and $K^{\prime\bullet\bullet}$, thus defines a morphism for each of the two spectral sequences;
in addition, two \emph{homotopic} morphisms define a homotopy \emph{of order $\leq 1$} of the corresponding filtered (simple) complexes, thus the \emph{same} morphism for each of the two spectral sequences~(M,~XV,~6.1).
\end{env}

\begin{proposition}[11.3.3]
\label{0.11.3.3}
Let $K^{\bullet\bullet}=(K^{ij})$ be a bicomplex in an abelian category $\cat{C}$.
\begin{enumerate}
  \item[{\rm(i)}] If there exist $i_0$ and $j_0$ such that $K^{ij}=0$ for $i<i_0$ or $j<j_0$ (resp.~$i>i_0$ or $j>j_0$), then the two spectral sequences $'E(K^{\bullet\bullet})$ and $''E(K^{\bullet\bullet})$ are biregular.
  \item[{\rm(ii)}] If there exist $i_0$ and $i_1$ such that $K^{ij}=0$ for $i<i_0$ or $i>i_1$ (resp.~if there exist $j_0$ and $j_1$ such that $K^{ij}=0$ for $j<j_0$ or $j>j_1$), then the two spectral sequences $'E(K^{\bullet\bullet})$ and $''E(K^{\bullet\bullet})$ are biregular.
  \item[{\rm(iii)}] If there exists $i_0$ such that $K^{ij}=0$ for $i>i_0$ (resp.~if there exists $j_0$ such that $K^{ij}=0$ for $j<j_0$), then the spectral sequence $'E(K^{\bullet\bullet})$ is regular.
  \item[{\rm(iv)}] If there exists $i_0$ such that $K^{ij}=0$ for $i<i_0$ (resp.~if there exists $j_0$ such that $K^{ij}=0$ for $j>j_0$), then the spectral sequence $''E(K^{\bullet\bullet})$ is regular.
\end{enumerate}
\end{proposition}

\begin{proof}
The proposition follows immediately from the definitions~\sref{0.11.1.3} and from~\sref{0.11.2.4}, as well as from the following observations relating to the filtration $F_\mathrm{I}$ (and similar observations that we can deduce for $F_\mathrm{II}$ by exchange the roles of the two indices in $K^{\bullet\bullet}$):
\begin{enumerate}
  \item[{$1^{\circ}$}] If there exists $i_0$ such that $K^{ij}=0$ for $i>i_0$, then the filtration $F_\mathrm{I}(K^{\bullet\bullet})$ is \emph{discrete}.
  \item[{$2^{\circ}$}] If there exists $i_0$ such that $K^{ij}=0$ for $i<i_0$, then the filtration $F_\mathrm{I}(K^{\bullet\bullet})$ is \emph{co-discrete}.
    We can then immediate deduce that it is the same for the corresponding filtration $F_\mathrm{I}(\HH^n(K^{\bullet\bullet}))$ for all $n$. 
    Furthermore, the definition of the $B_{r}^{pq}$ corresponding to the filtration $F_\mathrm{I}(K^{\bullet\bullet})$ \sref{0.11.2.2} shows that for any pair $(p,q)$, the sequence $(B_{r}^{pq})_{r\leq{2}}$ is stable.
  \item[{$3^{\circ}$}] If there exists $j_0$ such that $K^{ij}=0$ for $j<j_0$, then we have 
    \[
      F_\mathrm{I}^{p+r}(K^{\bullet\bullet})
      \cap (\sum_{i+j=n}K^{ij})
      = 0
    \]
    as soon as $p+r+j_{0}>n$, so $Z_{r}^{pq} = Z_\infty(E_2^{pq})$ for $r>q-j_{0}+1$.
    On the other hand, $\HH^{n}(F_\mathrm{I}^{p}(K^{\bullet\bullet})) = 0$ for $p>n-j_{0}+1$.
  \item[{$4^{\circ}$}] If there exists $j_0$ such that $K^{ij}=0$ for $j>j_0$, then we have 
    \[
      F_\mathrm{I}^{p-r+1}(K^{\bullet\bullet})
      \cap (\sum_{i+j=n}K^{ij})
      = \sum_{i+j=n}K^{ij}
    \]
    as soon as $p-r+1+j_{0}<n$, so $B_{r}^{pq} = B_\infty(E_2^{pq})$ for $r<j_{0}-q+1$.
    On the other hand, $\HH^{n}(F_\mathrm{I}^{p}(K^{\bullet\bullet})) = \HH^{n}(K^{\bullet\bullet})$ for $p+j_{0}<n-1$.
\end{enumerate}
\end{proof}

\begin{env}[11.3.4]
\label{0.11.3.4}
\oldpage[0\textsubscript{III}]{31}
Suppose that the bicomplex $K^{\bullet\bullet}=(K^{ij})$ is such that $K^{ij}=0$ for $i<0$ or $j<0$.
We know that we can define, for all $p\in\bb{Z}$, a canonical ``edge-homomorphism''.
\[
\label{0.11.3.4.1}
  'E_2^{p0}(K^{\bullet\bullet})
  \rightarrow \HH^{p}(K^{\bullet\bullet})
\tag{11.3.4.1}
\]
(M,~XV,~6).
Recall that this is due to, on the one hand, the spectral sequence
$'E(K^{\bullet\bullet})$, since $Z_r^{p0}=Z_{\mathrm{I}}^{p}(Z_{\mathrm{II}}^{0}(K^{\bullet\bullet}))$ for $2\leq r \leq +\infty$, and on the other hand, that $\HH^{p}(F_\mathrm{I}^{p+1}(K^{\bullet\bullet})) = 0$, so that the isomorphism $\beta^{p0}:\ 'E_\infty^{p0} \simto \HH^{p}(F_1^p)/\HH^{p}(F_1^{p+1})$ gives a homomorphism $'E_\infty^{p0} \rightarrow \HH^{p}(F_1^p(K^{\bullet\bullet})) \rightarrow \HH^{p}(K^{\bullet\bullet})$.
The equality of all the $Z_r^{p0}$ allows us to define a canonical homomorphism $'E_r^{p0} \rightarrow 'E_s^{p0}$ for $r \leq s$, and, in particular, a homomorphism $'E_2^{p0} \rightarrow 'E_\infty^{p0}$, from the composition of the edge-homomorphism $'E_2^{p0}(K^{\bullet\bullet}) \rightarrow \HH^{p}(K^{\bullet\bullet})$.
Furthermore, we can immediately verify that, in the class $\mod B_2^{p0}$ of an element $z \in\bb{Z}_\mathrm{II}^0(K^{\bullet\bullet}) \subset K^{p0}$ such that $d'z=0$, the edge-homomorphism thus defined associates, to the class of $\mod B_\infty^{p0}$ in $'E_{\infty}^{p0}$, the cohomology class of $z$ in $H^p(K^{\bullet\bullet})$.
We therefore finally see that the \emph{edge-homomorphism} \sref{0.11.3.4.1} \emph{comes from, by passing to cohomology, the canonical injection} $Z_\mathrm{II}^0(K^{\bullet\bullet}) \rightarrow K^{\bullet\bullet}$  (where $K^{\bullet\bullet}$ is considered as simple complex).
We naturally interpret the edge-homomorphism in the same way 
\[
\label{0.11.3.4.2}
  ''E_2^{p0}(K^{\bullet\bullet})
  \rightarrow \HH^{p}(K^{\bullet\bullet})
\tag{11.3.4.2}
\]
as coming from the canonical injection $Z_\mathrm{I}^0(K^{\bullet\bullet}) \rightarrow K^{\bullet\bullet}$.
\end{env}

\begin{env}[11.3.5]
\label{0.11.3.5}
Now let $K_{\bullet\bullet} = (K_{ij})$ be a bicomplex in $\cat{C}$ whose two differential operators are of degrees~$-1$.
We then write $K_{i,\bullet}$ (resp. $K_{\bullet,j}$) to mean the simple complex $(K_{ij})_{j\in\bb{Z}}$ (resp. $(K_{ij})_{i\in\bb{Z}}$), $\HH_{p}^\mathrm{II}(K_{i, \bullet})$ (resp. $\HH_{p}^\mathrm{I}(K_{\bullet, j})$) to mean the $p$-th homology object, $\HH_{p}^\mathrm{II}(K_{\bullet\bullet})$ (resp. $\HH_{p}^\mathrm{I}(K_{\bullet\bullet})$) to mean the complex $(\HH_{p}^\mathrm{II}(K_{i, \bullet}))_{i \in\bb{Z}}$ (resp. $(\HH_{p}^\mathrm{I}(K_{\bullet, j}))_{j \in\bb{Z}}$), and $\HH_q^\mathrm{I}(\HH_{p}^\mathrm{II}(K_{\bullet\bullet}))$ (resp. $\HH_q^\mathrm{II}(\HH_{p}^\mathrm{I}(K_{\bullet\bullet}))$) to mean the $q$-th homology object;
we use analogous notation for the objects of cycles and objects of boundaries;
finally, we will denote by $\HH_n(K_{\bullet\bullet})$ the $n$-th homology object (when it exists) of the simple complex (with a differential operator of degree~$-1$) defined by $K_{\bullet\bullet}$.
Let $K'^{\bullet\bullet} = (K^{ij})$ with $K'^{\bullet\bullet} = (K_{-i,-j})$ be the bicomplex with differential operators of degrees~$+1$ associated to $K_{\bullet\bullet}$.
By definition, the \emph{spectral sequences} of $K_{\bullet\bullet}$ are those of $K'^{\bullet\bullet}$, that we write as $'E(K_{\bullet\bullet})$ and $''E(K_{\bullet\bullet})$, where, however, we change the notation, by putting
\[
  'E_{pq}^{r}(K_{\bullet\bullet})
  = 'E_{r}^{-p,-q}(K'^{\bullet\bullet})
  \quad\text{and}\quad
  ''E_{pq}^{r}(K_{\bullet\bullet})
  = ''E_{r}^{-p,-q}(K'^{\bullet\bullet}), 
\]
for $2\leq r \leq \infty$.
With this notation, we have
\[
  'E_{pq}^{2}(K_{\bullet\bullet})
  = \HH_{p}^\mathrm{I}(\HH_{q}^\mathrm{II}(K_{\bullet\bullet}))
  \quad\text{and}\quad
  'E_{pq}^{2}(K_{\bullet\bullet})
  = \HH_{q}^\mathrm{II}(\HH_{p}^\mathrm{I}(K_{\bullet\bullet})).
\]
To avoid sign errors, it will be preferable in general to return to the complex $K'^{\bullet\bullet}$ for the relations between these spectral sequences and their abutments.
We note, however, the criteria corresponding to \sref{0.11.3.3}:
\end{env}

\begin{env}[11.3.6]
\label{0.11.3.6}
\oldpage[0\textsubscript{III}]{32}
The spectral sequences $'E(K_{\bullet\bullet})$ and $''E(K_{\bullet\bullet})$ are \emph{biregular} in the following cases:
\begin{enumerate}
  \item[(a)] There exist $i_0$ and $j_0$ such that $K_{ij}=0$ for $i>i_0$ and $j>j_0$ (resp. for $i<i_0$ and $j<j_0$);
  \item[(b)] There exist $i_0$ and $i_1$ such that $K_{ij}=0$ for $i<i_0$ and $i>i_1$;
  \item[(c)] There exist $j_0$ and $j_1$ such that $K_{ij}=0$ for $j<j_0$ and $j>j_1$.
\end{enumerate}

The sequence $'E(K_{\bullet\bullet})$ is \emph{regular} if there exists $i_0$ such that $K_{ij}=0$ for $i<i_0$ and $K_{ij}=0$ for $j>j_0$.

The sequence $''E(K_{\bullet\bullet})$ is \emph{regular} if there exists $i_0$ such that $K_{ij}=0$ for $i>i_0$ and $K_{ij}=0$ for $j<j_0$.
\end{env}


\subsection{Hypercohomology of a functor with respect to a complex $K^{\bullet}$}
\label{subsection:0.11.4}

\begin{env}[11.4.1]
\label{0.11.4.1}
Let $\cat{C}$ be an abelian category.
Recall that we defined the \emph{right resolution} (or \emph{cohomological resolution}) of a object $A$ in $\cat{C}$ to be a complex of objects in $\cat{C}$, whose differential operator is of degree $+1$,
\[
  0 \rightarrow L^0 \rightarrow L^1 \rightarrow L^2 \ldots
\]
endowed with a morphism $\varepsilon: A \rightarrow L^0$, called the \emph{augmentation} of the resolution, which we can consider as a morphism of complexes
\[
  \xymatrix{
    0 \ar[r]
    & A \ar[r] \ar[d]_{\varepsilon}
    & 0 \ar[r] \ar[d]
    & 0 \ar[r] \ar[d]
    & \ldots
  \\0 \ar[r]
    & L^0 \ar[r]
    & L^1 \ar[r]
    & L^2 \ar[r]
    & \ldots
  }
\]
such that the sequence
\[
  0 \rightarrow A \xrightarrow{\varepsilon} L^0 \rightarrow L^1 \ldots
\]
is exact.
Similarly, a \emph{left resolution} (or \emph{homological resolution}) of $A$ is a complex $0 \leftarrow L_0 \leftarrow L_1 \leftarrow L_2 \ldots$ of objects in $\cat{C}$, whose differential operator is of degree~$-1$, endowed with an \emph{augmentation} $L_0 \xrightarrow{\varepsilon} A$ such that the sequence
\[
  0 \leftarrow A \xleftarrow{\varepsilon} L_0 \leftarrow L_1 \ldots
\]
is exact.

When the right resolution $(L_i)_i\geq 0$ of a object $A$ is such that $L_i=0$ for $i\geq n+1$, we say that this resolution is \emph{of length $\leq n$}.
We define a left resolution of length $\leq n$ similarly.
A resolution which is of of length $\leq n$ for some integer $n$ is said to be \emph{finite}.

A resolution of $A$ is called \emph{projective} (resp. \emph{injective}) if the objects of $\cat{C}$, apart from $A$, of which it is composed are \emph{projective} (resp. \emph{injective}). 
When $\cat{C}$ is the category of (left, say) modules over a ring, we say that a resolution of $A$ is \emph{flat} (resp. \emph{free}) when the modules, apart from $A$, of which it is composed are \emph{flat} (resp. \emph{free}).
\end{env}

\begin{env}[11.4.2]
\label{0.11.4.2}
Let $K^\bullet = (K^i)_{i\in\bb{Z}}$ be a complex of objects of $\cat{C}$, whose differential operator is of degree~$+1$.
We define the \emph{right Cartan--Eilenberg resolution} of $K^\bullet$ to be a pair consisting of a bicomplex $L^{\bullet\bullet} = (L^{ij})$ with differential operator of degree~$+1$ and such that $L^{ij}=0$ for $j\leq 0$, and a morphism of simple complexes $\varepsilon: K^\bullet \rightarrow L^{\bullet, 0}$, such that the following condition are fulfilled:
\oldpage[0\textsubscript{III}]{33}
\begin{enumerate}
  \item[(i)] For each index $i$, the sequences
    \[
      \begin{aligned}
        0 \rightarrow
        & K^i
        \xrightarrow{\varepsilon} L^{i0}
        \rightarrow L^{i1}
        \rightarrow \ldots
      \\0 \rightarrow
        & B^i(K^\bullet)
        \xrightarrow{\varepsilon} B_\mathrm{I}^i(L^{\bullet,0})
        \rightarrow B_\mathrm{I}^i(L^{\bullet,1})
        \rightarrow \ldots
      \\0 \rightarrow
        & Z^i(K^\bullet)
        \xrightarrow{\varepsilon} Z_\mathrm{I}^i(L^{\bullet,0})
        \rightarrow Z_\mathrm{I}^i(L^{\bullet,1})
        \rightarrow \ldots
      \\0 \rightarrow
        & \HH^i(K^\bullet)
        \xrightarrow{\varepsilon} \HH_\mathrm{I}^i(L^{\bullet,0})
        \rightarrow \HH_\mathrm{I}^i(L^{\bullet,1})
        \rightarrow \ldots
      \end{aligned}
    \]
    are exact, or, in other words, $(L^{i,\bullet}), (B_\mathrm{I}^i(L^{\bullet\bullet})), (Z_\mathrm{I}^i(L^{\bullet\bullet}))$ and $(\HH_\mathrm{I}^i(L^{\bullet\bullet}))$ are \emph{resolutions} of $K^i, B^i(K^\bullet), Z^i(K^\bullet)$ and $\HH^i(K^\bullet)$ (respectively).
  \item[(ii)] For each $j$, the simple complex $L^{\bullet, j}$ is \emph{split}, or, in other words, the exact sequences
    \[
      \label{0.11.4.2.1}
      0 \rightarrow B_\mathrm{I}^i(L^{\bullet,j}) \rightarrow Z_\mathrm{I}^i(L^{\bullet,j}) \rightarrow \HH_\mathrm{I}^i(L^{\bullet,j}) \rightarrow 0
      \tag{11.4.2.1}
    \]
    \[
      \label{0.11.4.2.2}
      0 \rightarrow Z_\mathrm{I}^i(L^{\bullet,j}) \rightarrow L^{ij} \rightarrow B_\mathrm{I}^{i+1}(L^{\bullet,j}) \rightarrow 0
      \tag{11.4.2.2}
    \]
    are \emph{split}.
\end{enumerate}

We can prove (M,~XVII,~1.2) that if every object of $\cat{C}$ is subobjects of an injective object, then every complex $K^\bullet$ of $\cat{C}$ admits an \emph{injective} Cartan--Eilenberg resolution, i.e. consisting of injective objects of $L^{ij}$ (the condition (ii) above then implying that the $(B_\mathrm{I}^i(L^{\bullet,j})), (Z_\mathrm{I}^i(L^{\bullet,j}))$ and $(\HH_\mathrm{I}^i(L^{\bullet,j}))$ also consist of injective objects).
Furthermore, for any morphsim $f: K^\bullet \rightarrow K'^\bullet$ of complexes of $\cat{C}$, 
any Cartan--Eilenberg resolution $L^{\bullet\bullet}$ of $K^\bullet$, and any \emph{injective} Cartan--Eilenberg resolution $L'^{\bullet\bullet}$ of $K'^\bullet$, there exists a morphism of bicomplexes $F: L^{\bullet\bullet} \rightarrow L'^{\bullet\bullet}$ that is compatible with $f$ and its augmentation, and if $f$ and $g$ are homotopic morphisms from $K^\bullet$ to $K'^\bullet$, then the corresponding morphisms from $L^{\bullet\bullet}$ to $L'^{\bullet\bullet}$ are also homotopic \emph{(loc. cit.)}.

When $K^{\bullet}$ is \emph{bounded below} (resp. \emph{bounded above}), we can take $L^{\bullet\bullet}$ to be such that $L^{ij}=0$ for $i<i_0$ (resp. $i>i_0$) if $K^i=0$ for $i<i_0$ (resp. $i>i_0$) (M,~XVII,~1.3).

Suppose, on the other hand, that there exists a integer $n$ such that every object of $\cat{C}$ admits a \emph{injective resolution of length $\leq n$};
then we can assume that we have $L^{ij}=0$ for $j>n$ (M,~XVII,~1.4).
\end{env}


\begin{env}[11.4.3]
\label{0.11.4.3}
Now let T be an \emph{additive covariant functor} from $\cat{C}$ to an abelian category $\cat{C'}$.
Given a complex $K^\bullet$ of $\cat{C}$ and an \emph{injective} Cartan--Eilenberg resolution of $L^{\bullet\bullet}$ of $K^{\bullet}$, suppose that the (simple) complex defined by the bicomplex $T(L^{\bullet\bullet})$ exists \sref{0.11.3.1};
then the two spectral sequences $'E(T(L^{\bullet\bullet}))$ and $''E(T(L^{\bullet\bullet}))$ of this bicomplex are called the \emph{hypercohomology spectral sequences} of $T$ with respect to $K^\bullet$;
by \sref{0.11.4.2} and \sref{0.11.4.3}, they effectively only depend on the complex $K^\bullet$ effectively, not on the choice of the injective Cartan--Eilenberg resolution $L^{\bullet\bullet}$;
furthermore, they depend \emph{functorially} on $K^\bullet$.
They have the same abutment $\HH^\bullet(T(L^{\bullet\bullet}))$, also called the \emph{hypercohomology} of $T$ with respect to $K^\bullet$, and denoted by $\RR^\bullet T(K^\bullet)$.
We can show that the $E_2$ terms of the two spectral sequences above are given by 
\[
\label{0.11.4.3.1}
  'E_2^{pq}
  = H^p(\RR^q T(K^\bullet))
\tag{11.4.3.1}
\]
\[
\label{0.11.4.3.2}
  ''E_2^{pq}
  = \RR^p T(H^q(K^\bullet))
\tag{11.4.3.2}
\]
\oldpage[0\textsubscript{III}]{34}
where $\RR^p T$ denotes, as usual, the $p$-th \emph{derived functor} of $T$ for $p\in\bb{Z}$;
$\RR^q T(K^\bullet)$ denotes the complexes $(\RR^q T(K^i))_{i\in\bb{Z}}$.
Unless explicitly otherwise mentioned, we will henceforth assume that every object of $\cat{C}$ is a subobject of an injective object of $\cat{C}$, so that injective Cartan--Eilenberg resolutions exist for any complex of $\cat{C}$.
Since $L^{ij}=0$ for $j<0$, the criteria of \sref{0.11.3.3} show that the \emph{two} hypercohomology spectral sequences of $T$ with respect to $K^\bullet$ exist and are \emph{biregular} in the each of the following two cases:
\begin{enumerate}
  \item $K^\bullet$ is bounded below;
  \item Every object of $\cat{C}$ admits an injective resolution of length at most $n$, for some integer $n$ independent of the object in question.
\end{enumerate}
Indeed, we can suppose, in the first case, that \sref{0.11.4.2} there exists $i_0$ such that $L^{ij}=0$ for $i<i_0$, and, in the second case, that there exists $j_1$ such that $L^{ij}=0$ for $j>j_1$;
in each of these two cases, it is furthermore clear that, for any given $n$, there exist only a finite number of pairs $(i,j)$ such that $L^{ij}\neq 0$ and $i+j=n$, which proves our claims.

If we assume that filtrant inductive limits exist in $\cat{C}'$ and are exact (which implies, in particular, the existence of infinite direct sums in $\cat{C}'$), then the complex defined by the bicomplex $T(L^{\bullet\bullet})$ exists, and criterion \sref{0.11.3.3} shows that the sequence $'E(T(L^{\bullet\bullet}))$ is always \emph{regular}.

If $K^\bullet$ is a complex such that all the $K^i$ are zero except for a single $K^{i_0}$, then
$\RR^n T(K^\bullet)$ is isomorphic to $\RR^{n-i_0} T(K^\bullet)$, as follows immediately from the definitions by taking a Cartan--Eilenberg resolution $L^{\bullet\bullet}$ such that $L^{ij}=0$ for $i\neq i_0$.

If $K^\bullet$ and $K'^\bullet$ are two complexes of $\cat{C}$, and $f$ and $g$ are homotopic morphism from $K^\bullet$ to $K'^\bullet$, then the morphisms $\RR^\bullet T(K^\bullet) \rightarrow \RR^\bullet T(K'^\bullet)$ induced by $f$ and $g$ are identical, and the same is true for the morphisms of the cohomology spectral sequences.
\end{env}

\begin{proposition}[11.4.5]
\label{0.11.4.5}
Suppose that filtrant inductive limits exist in $\cat{C}'$ and are exact.
If $\RR^n T(K^i) = 0$ for all $n>0$ and all $i\in\bb{Z}$, then we have functorial isomorphisms
\[
\label{0.11.4.5.1}
  \RR^i T(K^\bullet) \simto \HH^i(T(K^\bullet))
\tag{11.4.5.1}
\]
for all $i\in\bb{Z}$.
\end{proposition}

\begin{proof}
Indeed, the only non-zero $E_2$ terms of the first spectral sequence \sref{0.11.4.3.1} are then $'E_2^{p0} = \HH^p(T(K^\bullet))$;
in other words, this sequence is \emph{degenerate};
since it is regular \sref{0.11.4.4}, the conclusion follows from \sref{0.11.1.6}.
\end{proof}

\begin{env}[11.4.6]
\label{0.11.4.6}
Now consider, for example, a covariant bifunctor $(M, N) \rightarrow T(M, N)$ from $\cat{C} \times \cat{C}$ to $\cat{C}''$, where $\cat{C}$, $\cat{C}'$, and $\cat{C}''$ are three abelian categories;
we assume, for simplicity, that $T$ is additive in each of its arguments, and furthermore that every object of $\cat{C}$ and every object of $\cat{C}'$ are subobjects of an injective object, 
and that filtrant inductive limits exist in $\cat{C}$ and are exact.
We then define the \emph{hypercohomology} of $T$ with respect to the complexes $K^\bullet$ and $K'^\bullet$ of $\cat{C}$ and $\cat{C}'$ (respectively), with differential operators of degree~$+1$, by taking $K^\bullet$ (resp. $K'^\bullet$) to be an injective Cartan--Eilenberg resolution $L^{\bullet\bullet}$ (resp. $L'^{\bullet\bullet}$);
then $T(L^{\bullet\bullet}, L'^{\bullet\bullet})$ is a quadricomplex of $\cat{C}''$, which we consider as a \emph{bicomplex} of $\cat{C}'$, with the degree of $T(L^{ij}, L'^{hk})$ being the integers $i+h$ and $j+k$.
The \emph{hypercohomology} of $T$ with respect to $K^\bullet$ and $K'^\bullet$ is by definition the cohomology $\HH^\bullet(T(L^{\bullet\bullet}, L'^{\bullet\bullet}))$ of this bicomplex  (in other words, that of the associated simple complex) and is denoted by $\RR^\bullet T(K^\bullet, K'^\bullet)$;
\oldpage[0\textsubscript{III}]{35}
it is the abutment of two spectral sequences whose $E_2$ terms are given by
\[
\label{0.11.4.6.1}
  'E_2^{pq}
  = \HH^p(\RR^q T(K^\bullet, K'^\bullet))
\tag{11.4.6.1}
\]
\[
\label{0.11.4.6.2}
  ''E_2^{pq}
  = \sum_{q'+q''=q}\RR^p T(\HH^{q'}{K^\bullet},\HH^{q''}{K'^\bullet})
\tag{11.4.6.2}
\]
(cf.~M,~XVII,~2).

Here $\RR^q T(K^\bullet, K'^\bullet)$ is the bicomplex $(\RR^q T(K^i, K'^j))_{(i,j)\in \bb{Z}\times \bb{Z}}$, and the right-hand side of \sref{0.11.4.6.1} is its cohomology when we consider it as a simple complex.

Furthermore, the first spectral sequence is always regular, and the two spectral sequences are biregular when there exists $n$ such that every object of $\cat{C}$ and every object of $\cat{C}'$ admit an injective resolution of length $\leq n$, or when $K^\bullet$ and $K'^\bullet$ are bounded below;
in the latter case, we can furthermore omit the hypothesis that inductive limits exist in $\cat{C}$ and $\cat{C}'$.

If $K^\bullet$ and $K'^\bullet$ are two other complexes of $\cat{C}$ and $\cat{C}'$ (respectively), $f$ and $g$ \emph{homotopic} morphisms from $K^\bullet$ to $K_1^\bullet$, 
and $f'$ and $g'$ \emph{homotopic} morphisms from $K'^\bullet$ to ${K'_1}^\bullet$, then the morphisms $\RR^\bullet  T(K^\bullet, K'^\bullet) \rightarrow \RR^\bullet  T(K_1^\bullet, {K'_1}^\bullet)$ induced by $f$ and $f'$ on the one hand, and by $g$ and $g'$ on the other hand, are identical, and the same is true for the morphisms of the hypercohomology spectral sequences.

We can generalize easily to any additive covariant multifunctor.
\end{env}

\begin{proposition}[11.4.7]
\label{0.11.4.7}
Suppose that for any injective object $\cat{I}$ of $\cat{C}$ (resp. $\cat{I'}$ of $\cat{C}'$), $\cat{A'}\mapsto T(\cat{I}, \cat{A'})$ (resp. $\cat{A}\mapsto T(\cat{A}, \cat{I'})$) is an exact functor.
Then, with the notation of \sref{0.11.4.6}, we have a canonical isomorphism
\[
  \RR^\bullet T(K^\bullet, K'^\bullet) \simto \HH^\bullet(T(L^{\bullet\bullet}, K'^\bullet)) \simto \HH^\bullet(T(K^\bullet, L'^{\bullet\bullet})) 
\]
where the last two terms are the cohomology of simple complexes defined by the tricomplexes $T(L^{\bullet\bullet}, K'^\bullet)$ and $T(K^\bullet, L^{\bullet\bullet})$ (respectively).
\end{proposition}

\begin{proof}
Let us define, for example, the first of these isomorphisms.
The quadricomplex $T(L^{\bullet\bullet}, L'^{\bullet\bullet})$ can be considered as a bicomplex, where the degrees of $T(L^{ij}, L'^{hk})$ are the integers $i+j$ and $h+k$.
Since, for each $h$, ${L'}^{h,\bullet}$ is a \emph{resolution} of ${K'}^h$, we have, for this bicomplex, by virtue of the hypotheses on $T$, $\HH_\mathrm{II}^q(T(L^{\bullet\bullet}, L'^{\bullet\bullet}))=0$ for $q\neq 0$, and $\HH_\mathrm{II}^0(T(L^{\bullet\bullet}, L'^{\bullet\bullet})) = T(L^{\bullet\bullet}, K'^\bullet)$;
the first spectral sequence of this bicomplex is thus \emph{degenerate};
since $L'^{hk}=0$ for $k<0$, this sequence is also regular \sref{0.11.3.3}, and the conclusion then follows from \sref{0.11.1.6}.
\end{proof}

We have similar results for a covariant multifunctor of any number $n$ of arguments: in the calculation of the hypercohomology, it is not necessary to replace \emph{all} the complexes 
by a Cartan--Eilenberg resolution, but only $n-1$ of them, provided that, when we fix $n-1$ arbitrary arguments by taking them to be \emph{injective} objects, the covariant functor in the remaining argument is \emph{exact}.


\subsection{Passage to the inductive limit in the hypercohomology}
\label{subsection:0.11.5}
\oldpage[0\textsubscript{III}]{36}

\subsection{Hypercohomology of a functor with respect to a complex $K_{\bullet}$}
\label{subsection:0.11.6}

\subsection{Hypercohomology of a functor with respect to a bicomplex $K_{\bullet\bullet}$}
\label{subsection:0.11.7}

\subsection{Supplement on the cohomology of simplicial complexes}
\label{subsection:0.11.8}

\subsection{A lemma on the complexes of finite type}
\label{subsection:0.11.9}

\subsection{Euler-Poincare characteristic of a complex of finite length modules}
\label{subsection:0.11.10}